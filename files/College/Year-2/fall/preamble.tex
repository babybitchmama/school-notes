%%%%%%%%%%%%%%%%%%%%%%%%%%%%%%%%%%%%%%%%%%%%%%%%%%%%%%%%%%%%%%%%%%%%%%%%%%%%%%%
%                                Basic Packages                               %
%%%%%%%%%%%%%%%%%%%%%%%%%%%%%%%%%%%%%%%%%%%%%%%%%%%%%%%%%%%%%%%%%%%%%%%%%%%%%%%

% Gives us multiple colors.
\usepackage[usenames,dvipsnames,pdftex]{xcolor}
% Let's us style link colors.
\usepackage{hyperref}
% Let's us import images and graphics.
\usepackage{graphicx}
% Let's us use figures in floating environments.
\usepackage{float}
% Let's us create multiple columns.
\usepackage{multicol}
% Gives us better math syntax.
\usepackage{amsmath,amsfonts,mathtools,amsthm,amssymb}
% Let's us strikethrough text.
\usepackage{cancel}
% Let's us edit the caption of a figure.
\usepackage{caption}
% Let's us import pdf directly in our tex code.
\usepackage{pdfpages}
\def\class{article}


%%%%%%%%%%%%%%%%%%%%%%%%%%%%%%%%%%%%%%%%%%%%%%%%%%%%%%%%%%%%%%%%%%%%%%%%%%%%%%%
%                                Basic Settings                               %
%%%%%%%%%%%%%%%%%%%%%%%%%%%%%%%%%%%%%%%%%%%%%%%%%%%%%%%%%%%%%%%%%%%%%%%%%%%%%%%

%%%%%%%%%%%%%
%  Symbols  %
%%%%%%%%%%%%%

\let\implies\Rightarrow
\let\impliedby\Leftarrow
\let\iff\Leftrightarrow
\let\epsilon\varepsilon

%%%%%%%%%%%%
%  Tables  %
%%%%%%%%%%%%

\setlength{\tabcolsep}{5pt}
\renewcommand\arraystretch{1.5}

%%%%%%%%%%%%%%
%  SI Unitx  %
%%%%%%%%%%%%%%

\usepackage{siunitx}
\sisetup{locale = FR}

%%%%%%%%%%
%  TikZ  %
%%%%%%%%%%

\usepackage[framemethod=TikZ]{mdframed}
\usepackage{tikz}
\usepackage{tikz-cd}

\usetikzlibrary{intersections, angles, quotes, calc, positioning}
\usetikzlibrary{arrows.meta}

\tikzset{
  force/.style={thick, {Circle[length=2pt]}-stealth, shorten <=-1pt}
}

%%%%%%%%%%%%%%%
%  PGF Plots  %
%%%%%%%%%%%%%%%

\usepackage{pgfplots}
\pgfplotsset{compat=1.13}

%%%%%%%%%%%%%%%%%%%%%%%
%  Center Title Page  %
%%%%%%%%%%%%%%%%%%%%%%%

\usepackage{titling}
\renewcommand\maketitlehooka{\null\mbox{}\vfill}
\renewcommand\maketitlehookd{\vfill\null}

%%%%%%%%%%%%%%%%%%%
%  Todo Commands  %
%%%%%%%%%%%%%%%%%%%

\usepackage[colorinlistoftodos]{todonotes}
\makeatletter
\@ifclasswith\class{working}{
  \newcommand\unsure[2][1=]{\todo[linecolor=red,backgroundcolor=red!25,bordercolor=red,#1]{#2}}
  \newcommand\change[2][1=]{\todo[linecolor=blue,backgroundcolor=blue!25,bordercolor=blue,#1]{#2}}
  \newcommand\info[2][1=]{\todo[linecolor=OliveGreen,backgroundcolor=OliveGreen!25,bordercolor=OliveGreen,#1]{#2}}
  \newcommand\improvement[2][1=]{\todo[linecolor=Plum,backgroundcolor=Plum!25,bordercolor=Plum,#1]{#2}}

  \newcommand\listnotes{
    \newpage
    \listoftodos[Notes]
  }
}{
  \newcommand\unsure[2][1=]{}
  \newcommand\change[2][1=]{}
  \newcommand\info[2][1=]{}
  \newcommand\improvement[2][1=]{}

  \newcommand\listnotes{}
}
\makeatother

%%%%%%%%%%%%%%%%%%%%%%%%%%%%%%%%%%%%%%%%%%%%%%%%%%%%%%%
%  Create a grey background in the middle of the PDF  %
%%%%%%%%%%%%%%%%%%%%%%%%%%%%%%%%%%%%%%%%%%%%%%%%%%%%%%%

% \usepackage{eso-pic}
% \definecolor{reallylightgray}{HTML}{FAFAFA}
% \AddToShipoutPicture{
%   \ifthenelse{\isodd{\thepage}}{
%     \AtPageLowerLeft{
%       \put(\LenToUnit{\dimexpr\paperwidth-222pt},0){
%         \color{reallylightgray}\rule{222pt}{297mm}
%       }
%     }
%   }
%   {
%     \AtPageLowerLeft{
%       \color{reallylightgray}\rule{222pt}{297mm}
%     }
%   }
% }

%%%%%%%%%%%%%%%%%%%%%%%%
%  Modify Links Color  %
%%%%%%%%%%%%%%%%%%%%%%%%

\hypersetup{
  % Enable highlighting links.
  colorlinks,
  % Change the color of links to blue.
  linkcolor=blue,
  % Change the color of citations to black.
  citecolor={black},
  % Change the color of url's to blue with some black.
  urlcolor={blue!80!black}
}


%%%%%%%%%%%%%%%%%%%%%%%%%%%%%%%%%%%%%%%%%%%%%%%%%%%%%%%%%%%%%%%%%%%%%%%%%%%%%%%
%                           School Specific Commands                          %
%%%%%%%%%%%%%%%%%%%%%%%%%%%%%%%%%%%%%%%%%%%%%%%%%%%%%%%%%%%%%%%%%%%%%%%%%%%%%%%

%%%%%%%%%%%%%%%%%%%%%%%%%%%
%  Initiate New Counters  %
%%%%%%%%%%%%%%%%%%%%%%%%%%%

\newcounter{lecturecounter}
\newcounter{exercisecounter}
\newcounter{solutioncounter}

%%%%%%%%%%%%%%%%%%%%%
%  Lecture Command  %
%%%%%%%%%%%%%%%%%%%%%

\makeatletter

\usepackage{xifthen}

% EXAMPLE:
% 1. \lesson{Oct 17 2022 Mon (08:46:48)}{Lecture Title}
% 2. \lesson[4]{Oct 17 2022 Mon (08:46:48)}{Lecture Title}
% 3. \lesson{Oct 17 2022 Mon (08:46:48)}{}
% 4. \lesson[4]{Oct 17 2022 Mon (08:46:48)}{}
% Parameters:
% 1. (Optional) Lesson number.
% 2. Time and date of lecture.
% 3. Lecture Title.
\def\@lesson{}
\newcommand\lesson[3][\arabic{lecturecounter}]{
  % Add 1 to the lecture counter.
  \addtocounter{lecturecounter}{1}

  % Set the section number to the lecture counter.
  \setcounter{section}{#1}

  % Reset the exercise and solution counter back to 0.
  \setcounter{exercisecounter}{0}
  \setcounter{solutioncounter}{0}

  \renewcommand\thesubsection{#1.\arabic{subsection}}

  % Check if user passed the lecture title or not.
  \ifthenelse{\isempty{#3}}{
    \def\@lesson{Lecture \arabic{lecturecounter}}
  }{
    \def\@lesson{Lecture \arabic{lecturecounter}: #3}
  }

  % Display the information like the following:
  %                                                  Oct 17 2022 Mon (08:49:10)
  % ---------------------------------------------------------------------------
  % Lecture 1: Lecture Title
  \hfill\small{#2}
  \hrule
  \vspace*{-0.3cm}
  \section*{\@lesson}
  \addcontentsline{toc}{section}{\@lesson}
}

%%%%%%%%%%%%%%%%%%%%%%
%  Exercise Command  %
%%%%%%%%%%%%%%%%%%%%%%

% EXAMPLE:
% 1. \exercise
% 2. \exercise[3]
% Parameters:
% 1. (Optional) The exercise number. You don't need to add this, because by
%    default, a counter will be doing it for you. Here's an example: 5.7. The
%    first number is the section number, and the second number is the exercise
%    number.
\def\@exercise{}
\newcommand\exercise[1][\arabic{exercisecounter}]{
  % Add 1 to the exercise counter.
  \addtocounter{exercisecounter}{1}

  % Edit the \@exercise variable.
  \def\@exercise{Exercise \arabic{section}.#1}
  % Create a subsection without numbering.
  \subsection*{\@exercise}
  % Add the subsection to the table of contents.
  \addcontentsline{toc}{subsection}{\@exercise}
}

%%%%%%%%%%%%%%%%%%%%%%
%  Solution Command  %
%%%%%%%%%%%%%%%%%%%%%%

% EXAMPLE:
% 1. \solution
% 2. \solution[3]
% Parameters:
% 1. (Optional) The solution number. You don't need to add this, because by
%    default, a counter will be doing it for you. Here's an example: 5.7. The
%    first number is the section number, and the second number is the solution
%    number.
\def\@solution{}
\newcommand\solution[1][\arabic{solutioncounter}]{
  % Add 1 to the solution counter.
  \addtocounter{solutioncounter}{1}

  % Edit the \@solution variable.
  \def\@solution{Solution \arabic{section}.#1}
  % Create a subsection without numbering.
  \subsection*{\@solution}
  % Add the subsection to the table of contents.
  \addcontentsline{toc}{subsection}{\@solution}
}

%%%%%%%%%%%%%%%%%%%%
%  Import Figures  %
%%%%%%%%%%%%%%%%%%%%

\usepackage{import}
\pdfminorversion=7

% EXAMPLE:
% 1. \incfig{limit-graph}
% 2. \incfig[0.4]{limit-graph}
% Parameters:
% 1. The figure name. It should be located in figures/NAME.tex_pdf.
% 2. (Optional) The width of the figure. Example: 0.5, 0.35.
\newcommand\incfig[2][1]{%
  \def\svgwidth{#1\columnwidth}
  \import{./figures/}{#2.pdf_tex}
}

\begingroup\expandafter\expandafter\expandafter\endgroup
\expandafter\ifx\csname pdfsuppresswarningpagegroup\endcsname\relax
\else
  \pdfsuppresswarningpagegroup=1\relax
\fi

%%%%%%%%%%%%%
%  Correct  %
%%%%%%%%%%%%%

% EXAMPLE:
% 1. \correct{INCORRECT}{CORRECT}
% Parameters:
% 1. The incorrect statement.
% 2. The correct statement.
\definecolor{correct}{HTML}{009900}
\newcommand\correct[2]{{\color{red}{#1 }}\ensuremath{\to}{\color{correct}{ #2}}}

%%%%%%%%%%%%%%%%%
% Fancy Headers %
%%%%%%%%%%%%%%%%%

\usepackage{fancyhdr}
\newcommand\forcenewpage{\clearpage\mbox{~}\clearpage\newpage}
\newcommand\createintro{
  \pagenumbering{roman}

  \maketitle
  \thispagestyle{empty}
  \newpage
  Lecture notes from the course \MyTitle, given by professor Victor Ostrik at the \faculty~at \location~in the academic year \academicyear, during the \term term. This course covers symbolic logic, basic set theory, analyzing functions and their properties, modular arithmetic, counting and other problems in discrete mathematics, induction, and convergence of sequences and continuity of functions. Credit for the material in these notes is due to professor Victor, while the structure is loosely taken from the \href{https://www.amazon.com/Mathematical-Reasoning-Writing-Proof-2nd/dp/0131877186}{Mathematical Reasoning: Writing and Proof} textbook. The credit for the typesetting is my own.

\textit{Disclaimer:} This document will inevitably contain some mistakes--both simple typos and legitimate errors. Keep in mind that these are the notes of an undergraduate student in the process of learning the material himself, so take what you read with a grain of salt. If you find mistakes and feel like telling me, I will be grateful and happy to hear from you, even for the most trivial of errors. You can reach me by email, in English, Arabic, Hebrew, or Spanish at \href{mailto:singularisartt@gmail.com}{singularisartt@gmail.com}.


  \pagestyle{fancy}
  \renewcommand\headrulewidth{0pt}
  \fancyhead{}
  \fancyfoot[C]{
    \textit{For more notes like this, visit
    \href{singularisart.github.io/notes}{singularisart.github.io/notes}}. \\
    \vspace{0.1cm}
    \hrule
    \vspace{0.1cm}
    Hashem A. Damrah, \\
    \term: \academicyear, \\
    Last Update: \today, \\
    \faculty
  }

  \newpage

  \fancyfoot[C]{\thepage}
  \tableofcontents
  \forcenewpage

  \pagestyle{fancy}
  \pagenumbering{arabic}
  \setcounter{page}{1}

  \renewcommand\headrulewidth{0.4pt}
  \fancyhead[R]{\@lesson}
  \fancyhead[L]{Hashem A. Damrah}
  \fancyfoot[C]{\thepage}
}

\makeatother


%%%%%%%%%%%%%%%%%%%%%%%%%%%%%%%%%%%%%%%%%%%%%%%%%%%%%%%%%%%%%%%%%%%%%%%%%%%%%%%
%                                 Environments                                %
%%%%%%%%%%%%%%%%%%%%%%%%%%%%%%%%%%%%%%%%%%%%%%%%%%%%%%%%%%%%%%%%%%%%%%%%%%%%%%%

\mdfsetup{skipabove=1em,skipbelow=0em}
\theoremstyle{definition}

%%%%%%%%%%%%%%%%%%%%%%%%%%
%  Helpful New Commands  %
%%%%%%%%%%%%%%%%%%%%%%%%%%

% EXAMPLE:
% 1. \createnewtheorem{thmredbox}{}{}
% 2. \createnewtheorem{thmbluebox}{}{}
% 3. \createnewtheorem{thmblueline}{rightline=false, topline=false, bottomline=false}{}
% 4. \createnewtheorem{thmproofbox}{rightline=false, topline=false, bottomline=false}{qed=\qedsymbol}
% Parameters:
% 1. Theorem name.
% 2. Any extra parameters to pass to mdframed.
% 3. Any extra parameters to pass to declare theorem directly.
\newcommand\createnewtheorem[3]{
  \declaretheoremstyle[
  headfont=\bfseries\sffamily, bodyfont=\normalfont, #3,
  mdframed={
    #2
  },
  ]{#1}
}

% EXAMPLE:
% 1. \createcoloredtheorem{thmredbox}{RawSienna}{false}{true}{false}{false}{}{}
% 2. \createcoloredtheorem{thmbluebox}{NavyBlue}{false}{true}{false}{false}{}{}
% 3. \createcoloredtheorem{thmblueline}{NavyBlue}{false}{true}{false}{false}{backgroundcolor=white}{}
% 4. \createcoloredtheorem{thmproofbox}{RawSienna}{false}{true}{false}{false}{backgroundcolor=white}{qed=\qedsymbol}
% Parameters:
% 1. Theorem name.
% 2. Color of theorem.
% 3. Enable right line.
% 4. Enable left line.
% 5. Enable top line.
% 6. Enable bottom line.
% 7. Any extra parameters to pass to mdframed.
% 8. Any extra parameters to pass to declare theorem directly.
\newcommand\createcoloredtheorem[8]{
  \createnewtheorem{#1}{
    linewidth=2pt,
    rightline=#3, leftline=#4, topline=#5, bottomline=#6,
    linecolor=#2, backgroundcolor=#2!5, #7
  }{headfont=\bfseries\sffamily\color{#2}, #8}
}

%%%%%%%%%%%%%%%%%%%%%%%%%%%%%%%%%%%%%%%%%%%%%%%%%%%%%%%%%
%  Create Environments Styles Based on Given Parameter  %
%%%%%%%%%%%%%%%%%%%%%%%%%%%%%%%%%%%%%%%%%%%%%%%%%%%%%%%%%

\usepackage{thmtools}
\usepackage{tcolorbox}

\makeatletter
\@ifclasswith\class{nocolor}{
  % No color environments.
  \createnewtheorem{thmgreenbox}{}{}
  \createnewtheorem{thmredbox}{}{}
  \createnewtheorem{thmbluebox}{}{}
  \createnewtheorem{thmblueline}{rightline=false, topline=false, bottomline=false}{}
  \createnewtheorem{thmproofbox}{rightline=false, topline=false, bottomline=false}{qed=\qedsymbol}
}{
  % Color environments.
  \createcoloredtheorem{thmgreenbox}{ForestGreen}{false}{true}{false}{false}{}{}
  \createcoloredtheorem{thmredbox}{RawSienna}{false}{true}{false}{false}{}{}
  \createcoloredtheorem{thmbluebox}{NavyBlue}{false}{true}{false}{false}{}{}
  \createcoloredtheorem{thmblueline}{NavyBlue}{false}{true}{false}{false}{backgroundcolor=white}{}
  \createcoloredtheorem{thmproofbox}{RawSienna}{false}{true}{false}{false}{backgroundcolor=white}{qed=\qedsymbol}
}
\makeatother

%%%%%%%%%%%%%%%%%%%%%%%%%%%%%
%  Create the Environments  %
%%%%%%%%%%%%%%%%%%%%%%%%%%%%%

\declaretheorem[numberwithin=section, style=thmgreenbox, name=Definition]{definition}
\declaretheorem[sibling=definition, style=thmredbox, name=Corollary]{corollary}
\declaretheorem[sibling=definition, style=thmredbox, name=Proposition]{prop}
\declaretheorem[sibling=definition, style=thmredbox, name=Theorem]{theorem}
\declaretheorem[sibling=definition, style=thmredbox, name=Lemma]{lemma}
\declaretheorem[numbered=no, style=thmproofbox, name=Proof]{replacementproof}
\declaretheorem[numbered=no, style=thmblueline, name=Proof]{expl}
\declaretheorem[style=thmbluebox, numbered=no, name=Example]{example}
\declaretheorem[style=thmblueline, numbered=no, name=Remark]{remark}

%%%%%%%%%%%%%%%%%%%%%%%%%%%%
%  Edit Proof Environment  %
%%%%%%%%%%%%%%%%%%%%%%%%%%%%

\renewenvironment{proof}[1][\proofname]{\vspace{-10pt}\begin{replacementproof}}{\end{replacementproof}}
\newenvironment{explanation}[1][\proofname]{\vspace{-10pt}\begin{expl}}{\end{expl}}

%%%%%%%%%%%%%%%%%%%%%%%%%%%%%%%%
%  Create Simple Environments  %
%%%%%%%%%%%%%%%%%%%%%%%%%%%%%%%%

\newtheorem*{notation}{Notation}
\newtheorem*{previouslyseen}{As previously seen}
\newtheorem*{problem}{Problem}
\newtheorem*{observe}{Observe}
\newtheorem*{property}{Property}
\newtheorem*{intuition}{Intuition}
\newtheorem*{note}{Note}
