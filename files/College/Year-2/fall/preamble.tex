%%%%%%%%%%%%%%%%%%%%%%%%%%%%%%%%%%%%%%%%%%%%%%%%%%%%%%%%%%%%%%%%%%%%%%%
%                           Basic Packages                            %
%%%%%%%%%%%%%%%%%%%%%%%%%%%%%%%%%%%%%%%%%%%%%%%%%%%%%%%%%%%%%%%%%%%%%%%

\usepackage{hyperref}
\hypersetup{
  colorlinks,
  linkcolor=blue,
  citecolor={black},
  urlcolor={blue!80!black}
}
\usepackage{graphicx,wrapfig}
\usepackage{float}
\usepackage{multicol}
\usepackage[usenames,dvipsnames]{xcolor}
\usepackage{amsmath,amsfonts,mathtools,amsthm,amssymb}
\usepackage{cancel}
\usepackage{xifthen}
\usepackage{tcolorbox}
\usepackage{float}
\usepackage{extramarks}
\usepackage[export]{adjustbox}
\usepackage{caption}
\usepackage{acro}
\usepackage[symbols,nogroupskip,record]{glossaries-extra}

%%%%%%%%%%%%
%  Tables  %
%%%%%%%%%%%%

\setlength{\tabcolsep}{5pt}
\renewcommand{\arraystretch}{1.5}

%%%%%%%%%%%%%%%%%%%%%
%  Lecture Command  %
%%%%%%%%%%%%%%%%%%%%%

\def\testdateparts#1{\dateparts#1\relax}
\def\dateparts#1 #2 #3 #4 #5\relax{
  \hspace*{\fill}\small\textsf{\mbox{#1 #2 #3 #5}}
}

\makeatletter

\def\@lesson{}
\newcommand{\lesson}[3]{
  \setcounter{section}{#1}
  \renewcommand\thesubsection{#1.\arabic{subsection}}

  \ifthenelse{\isempty{#3}}{
    \def\@lesson{Lecture #1}
  }{
    \def\@lesson{Lecture #1: #3}
  }
  \testdateparts{#2}
  \hrule
  \vspace*{-0.3cm}
  \section*{\@lesson}
  \addcontentsline{toc}{section}{\@lesson}
}

%%%%%%%%%%%%%%%%%
% Fancy Headers %
%%%%%%%%%%%%%%%%%

\usepackage{fancyhdr}

\newcommand{\createintro}{
  \maketitle
  Lecture notes from the course \MyTitle, given by professor Victor Ostrik at the \faculty~at \location~in the academic year \academicyear, during the \term term. This course covers symbolic logic, basic set theory, analyzing functions and their properties, modular arithmetic, counting and other problems in discrete mathematics, induction, and convergence of sequences and continuity of functions. Credit for the material in these notes is due to professor Victor, while the structure is loosely taken from the \href{https://www.amazon.com/Mathematical-Reasoning-Writing-Proof-2nd/dp/0131877186}{Mathematical Reasoning: Writing and Proof} textbook. The credit for the typesetting is my own.

\textit{Disclaimer:} This document will inevitably contain some mistakes--both simple typos and legitimate errors. Keep in mind that these are the notes of an undergraduate student in the process of learning the material himself, so take what you read with a grain of salt. If you find mistakes and feel like telling me, I will be grateful and happy to hear from you, even for the most trivial of errors. You can reach me by email, in English, Arabic, Hebrew, or Spanish at \href{mailto:singularisartt@gmail.com}{singularisartt@gmail.com}.


  \pagestyle{fancy}
  \renewcommand{\headrulewidth}{0pt}
  \fancyhead{}
  \fancyfoot[C]{
    \textit{For more notes like this, visit
    \href{singularisart.github.io/notes}{singularisart.github.io/notes}}. \\
    \vspace{0.1cm}
    \hrule
    \vspace{0.1cm}
    Hashem A. Damrah, \\
    \term: \academicyear, \\
    Last Update: \today, \\
    \faculty
  }

  \newpage
  \pagestyle{empty}
  \tableofcontents
  \clearpage\mbox{~}\clearpage\newpage
  \pagestyle{fancy}

  \renewcommand{\headrulewidth}{0.4pt}
  \pagestyle{fancy}

  \fancyhead[RO,LE]{\@lesson}
  \fancyhead[RE,LO]{Hashem A. Damrah}
  \fancyfoot[C]{\thepage}
}

%%%%%%%%%%%
%  Hline  %
%%%%%%%%%%%

\def\thickline{%
  \noalign{\ifnum0=`}\fi\hrule \@height\thickarrayrulewidth\futurelet
\reserved@a\@xthickhline}
\def\@xthickhline{\ifx\reserved@a\thickhline
    \vskip\doublerulesep
    \vskip-\thickarrayrulewidth
  \fi
\ifnum0=`{\fi}}

\makeatother

%%%%%%%%%%%%%%%%%%
% Figure Support %
%%%%%%%%%%%%%%%%%%

\usepackage{import}
\pdfminorversion=7
\usepackage{pdfpages}
\usepackage{transparent}
\newcommand{\incfig}[2][1]{%
  \def\svgwidth{#1\columnwidth}
  \import{./figures/}{#2.pdf_tex}
}

\pdfsuppresswarningpagegroup=1


%%%%%%%%%%%%%%%%%%%%%%%%%%%%%%%%%%%%%%%%%%%%%%%%%%%%%%%%%%%%%%%%%%%%%%%%%%%%%%%
%                                 Environments                                %
%%%%%%%%%%%%%%%%%%%%%%%%%%%%%%%%%%%%%%%%%%%%%%%%%%%%%%%%%%%%%%%%%%%%%%%%%%%%%%%

\usepackage{thmtools}
\usepackage[framemethod=TikZ]{mdframed}
\mdfsetup{skipabove=1em,skipbelow=0em}

\renewcommand{\qed}{\hfill\color{RawSienna}{Q.E.D.}}

\theoremstyle{definition}

\newcommand{\createnewtheorem}[3]{
  \declaretheoremstyle[
  headfont=\bfseries\sffamily, bodyfont=\normalfont, #3,
  mdframed={
    #2
  },
  ]{#1}
}

\makeatletter
\@ifclasswith{article}{nocolor}{
  % No color environments.
  \createnewtheorem{thmgreenbox}{}{}
  \createnewtheorem{thmredbox}{}{}
  \createnewtheorem{thmbluebox}{}{}
  \createnewtheorem{thmredlines}{rightline=false, leftline=false}{}
  \createnewtheorem{thmblueline}{rightline=false, topline=false, bottomline=false}{}
  \createnewtheorem{thmproofbox}{rightline=false, topline=false, bottomline=false}{qed=\qedsymbol}
}{
  \newcommand{\createcoloredtheorem}[7]{
    \createnewtheorem{#1}{
      linewidth=2pt,
      rightline=#3, leftline=#4, topline=#5, bottomline=#6,
      linecolor=#2, backgroundcolor=#2!5
    }{headfont=\bfseries\sffamily\color{#2}, #7}
  }

  % Color environments.
  \createcoloredtheorem{thmgreenbox}{ForestGreen}{false}{true}{false}{false}{}
  \createcoloredtheorem{thmredbox}{RawSienna}{false}{true}{false}{false}{}
  \createcoloredtheorem{thmbluebox}{NavyBlue}{false}{true}{false}{false}{}
  \createcoloredtheorem{thmredlines}{RawSienna}{true}{false}{true}{true}{}
  \createcoloredtheorem{thmblueline}{NavyBlue}{true}{false}{true}{true}{}
  \createcoloredtheorem{thmproofbox}{RawSienna}{false}{true}{false}{false}{qed=\qedsymbol}
}
\makeatother

\declaretheorem[numberwithin=section, style=thmgreenbox, name=Definition]{definition}
\declaretheorem[sibling=definition, style=thmredbox, name=Corollary]{corollary}
\declaretheorem[sibling=definition, style=thmredbox, name=Proposition]{prop}
\declaretheorem[sibling=definition, style=thmredbox, name=Theorem]{theorem}
\declaretheorem[sibling=definition, style=thmredbox, name=Lemma]{lemma}
\declaretheorem[sibling=definition, style=thmredlines, name=Exercise]{exercise}
\declaretheorem[numbered=no, style=thmproofbox, name=Proof]{replacementproof}
\renewenvironment{proof}[1][\proofname]{\vspace{-10pt}\begin{replacementproof}}{\end{replacementproof}}
\declaretheorem[style=thmbluebox, numbered=no, name=Example]{example}
\declaretheorem[style=thmblueline, numbered=no, name=Remark]{remark}

\newtheorem*{notation}{Notation}
\newtheorem*{previouslyseen}{As previously seen}
\newtheorem*{problem}{Problem}
\newtheorem*{observe}{Observe}
\newtheorem*{property}{Property}
\newtheorem*{intuition}{Intuition}
\newtheorem*{note}{Note}
