%%%%%%%%%%%%%%%%%%%%%%%%%%%%%%%%%%%%%%%%%%%%%%%%%%%%%%%%%%%%%%%%%%%%%%%%%%%%%%%%
%                                                                              %
%                              Required Packages                               %
%                                                                              %
%%%%%%%%%%%%%%%%%%%%%%%%%%%%%%%%%%%%%%%%%%%%%%%%%%%%%%%%%%%%%%%%%%%%%%%%%%%%%%%%

% Basic packages.
\usepackage[utf8]{inputenc}
\usepackage[T1]{fontenc}
% Gives us multiple colors.
\usepackage[usenames,dvipsnames,pdftex]{xcolor}
% Lets us style link colors.
\usepackage{hyperref}
% Lets us import images and graphics.
\usepackage{graphicx}
% Let's us modify list stuff.
\usepackage{enumitem}
% Lets us use figures in floating environments.
\usepackage{float}
% Lets us create multiple columns.
\usepackage{multicol}
\usepackage{multirow}
% Gives us better math syntax.
\usepackage{amsmath,amsfonts,mathtools,amsthm,amssymb}
% Lets us strike through text.
\usepackage{cancel}
% Lets us import pdf directly in our tex code.
\usepackage{pdfpages}
% Derivative stuff.
\usepackage{derivative}
% Lets us add vectors graphically.
\usepackage{physics}
% Table stuff.
\usepackage{tablists}
\usepackage{tabularx}
\usepackage{wasysym}


%%%%%%%%%%%%%%%%%%%%%%%%%%%%%%%%%%%%%%%%%%%%%%%%%%%%%%%%%%%%%%%%%%%%%%%%%%%%%%%%
%                                                                              %
%                                Basic Settings                                %
%                                                                              %
%%%%%%%%%%%%%%%%%%%%%%%%%%%%%%%%%%%%%%%%%%%%%%%%%%%%%%%%%%%%%%%%%%%%%%%%%%%%%%%%

%%%%%%%%%
% Tasks %
%%%%%%%%%

\usepackage{tasks}
\usepackage{extramarks}

\settasks{label=\bfseries\arabic*.),label-width=2em}

%%%%%%%%%%%
%  Table  %
%%%%%%%%%%%

\newcolumntype{C}{>{\Centering\arraybackslash}X}
\setlength{\tabcolsep}{5pt}
\renewcommand\arraystretch{1.5}

% Caption setup.
\usepackage[font=bf]{caption}
\renewcommand\thetable{\Roman{table}}
\captionsetup[figure]{font=small}
\captionsetup{justification=centering}

%%%%%%%%%%%%%
%  Symbols  %
%%%%%%%%%%%%%

\let\implies\Rightarrow
\let\impliedby\Leftarrow
\let\iff\Leftrightarrow
\let\epsilon\varepsilon
\let\svlim\lim\def\lim{\svlim\limits}
\let\svsum\sum\def\sum{\svsum\limits}

%%%%%%%%%%%
%  Lists  %
%%%%%%%%%%%

\setlist[itemize,1]{label=--}
\setlist[itemize,2]{label=\textbullet}
\setlist[enumerate,1]{label=\protect\circled{\arabic*}}

%%%%%%%%%%%%%%
%  SI Unitx  %
%%%%%%%%%%%%%%

\usepackage{siunitx}
\sisetup{
  locale = US,
  per-mode = symbol,
  propagate-math-font = true,
  reset-math-version = false,
  exponent-mode = engineering,
  round-mode = figures,
  round-precision = 3,
  drop-zero-decimal,
}

% Distance
\DeclareSIUnit{\millimeter}{mm}
\DeclareSIUnit{\centimeter}{cm}
\DeclareSIUnit{\decimeter}{dm}
\DeclareSIUnit{\inch}{in}
\DeclareSIUnit{\foot}{ft}
\DeclareSIUnit{\yard}{yd}
\DeclareSIUnit{\meter}{m}
\DeclareSIUnit{\kilometer}{km}
\DeclareSIUnit{\mile}{mi}

% Time
\DeclareSIUnit{\millisecond}{ms}
\DeclareSIUnit{\second}{sec}
\DeclareSIUnit{\minute}{min}
\DeclareSIUnit{\hour}{hr}
\DeclareSIUnit{\day}{d}
\DeclareSIUnit{\week}{wk}
\DeclareSIUnit{\month}{mos}
\DeclareSIUnit{\year}{yr}

% Weight
\DeclareSIUnit{\milligram}{mg}
\DeclareSIUnit{\gram}{g}
\DeclareSIUnit{\ounce}{oz}
\DeclareSIUnit{\pound}{lb}
\DeclareSIUnit{\kilogram}{kg}
\DeclareSIUnit{\ton}{t}

% Liquid
\DeclareSIUnit{\gallon}{gal}
\DeclareSIUnit{\liter}{L}
\DeclareSIUnit{\milliliter}{mL}

%%%%%%%%%%
%  TikZ  %
%%%%%%%%%%

\usepackage[framemethod=TikZ]{mdframed}
\usepackage{tikz}
\usepackage{animate}
\usepackage{tikz-cd}
\usepackage{bm}

\usetikzlibrary{
  intersections,
  angles,
  quotes,
  calc,
  positioning,
  3d,
  arrows,
  arrows.meta,
  patterns,
}

\tikzset{>=stealth}

\tikzstyle{vector label} = [midway,fill=white,sloped]
\tikzstyle{vector}=[->,very thick]
\tikzstyle{construction} = [->,thin,dashed,draw=green!45!black]

%%%%%%%%%%%%%%%
%  PGF Plots  %
%%%%%%%%%%%%%%%

\usepackage{pgfplots}
\pgfplotsset{compat=1.18}

\usepgfplotslibrary{patchplots}

\pgfplotsset{pccplot/.style={color=red,mark=none,line width=1pt,<->,solid}}
\pgfplotsset{asymptote/.style={color=gray,mark=none,line width=1pt,<->,dashed}}
\pgfplotsset{soldot/.style={color=red,only marks,mark=*}}
\pgfplotsset{holdot/.style={color=red,fill=white,only marks,mark=*}}
\pgfplotsset{blankgraph/.style={xmin=-10,xmax=10,ymin=-10,ymax=10,axis line style= {-, draw opacity=0 },axis lines=box,major tick length=0mm,xtick={-10,-9,...,10},ytick={-10,-9,...,10},grid=major,yticklabels={,,},xticklabels={,,},minor xtick=,minor ytick=,xlabel={},ylabel={},width=0.75\textwidth,grid style={solid,gray!40}}}

\pgfplotscreateplotcyclelist{pccstylelist}{
  pccplot \\
  color=blue,mark=none,line width=1pt,<->,dashdotted \\
  color=gray,mark=none,line width=1pt,<->,dashdotdotted \\
}

\def\axisdefaultwidth{175pt}
\def\axisdefaultheight{\axisdefaultwidth}

\pgfplotsset{every axis/.append style={
    axis x line=middle,
    axis y line=middle,
    axis line style={<->},
    xlabel={$x$},
    ylabel={$y$},
    xmin = -7,xmax = 7,
    ymin = -7,ymax = 7,
    yticklabel style={inner sep=0.333ex},
    minor xtick = {-7,-6,...,7},
    minor ytick = {-7,-6,...,7},
    scale only axis,
    cycle list name=pccstylelist,
    tick label style={font=\footnotesize},
    legend cell align=left,
    grid = minor,
    grid style = {solid,gray!40},
    try min ticks=6,
  },
  framed/.style={axis background/.style = {draw=gray}}
}

% framing the graphs
\pgfplotsset{axis background/.style={draw=gray}}

%%%%%%%%%%%%%%%%%%%%%%%
%  Center Title Page  %
%%%%%%%%%%%%%%%%%%%%%%%

\usepackage{titling}
\renewcommand\maketitlehooka{\null\mbox{}\vfill}
\renewcommand\maketitlehookd{\vfill\null}

%%%%%%%%%%%%%%%%%%%%%%%%%%%%%%%%%%%%%%%%%%%%%%%%%%%%%%%
%  Create a grey background in the middle of the PDF  %
%%%%%%%%%%%%%%%%%%%%%%%%%%%%%%%%%%%%%%%%%%%%%%%%%%%%%%%

\usepackage{eso-pic}

\newcommand\definegraybackground{
  \definecolor{reallylightgray}{HTML}{FAFAFA}
  \AddToShipoutPicture{
    \ifthenelse{\isodd{\thepage}}{
      \AtPageLowerLeft{
        \put(\LenToUnit{\dimexpr\paperwidth-222pt},0){
          \color{reallylightgray}\rule{222pt}{297mm}
        }
      }
    }
    {
      \AtPageLowerLeft{
        \color{reallylightgray}\rule{222pt}{297mm}
      }
    }
  }
}

%%%%%%%%%%%%%%%%%%%
%  Footnote Line  %
%%%%%%%%%%%%%%%%%%%

\renewcommand\footnoterule{\hrule\vspace{0.1cm}}

%%%%%%%%%%%%%%%%%%%%%%%%
%  Modify Links Color  %
%%%%%%%%%%%%%%%%%%%%%%%%

\makeatletter
\@ifclasswith\class{nocolor}{
  \hypersetup{
    colorlinks,
    linkcolor=black,
    citecolor=black,
    urlcolor=black,
  }
}{
  \hypersetup{
    colorlinks,
    linkcolor=main,
    citecolor=black,
    urlcolor={blue!80!black}
  }
}
\makeatother

%%%%%%%%%%%%%%%%%%
% Fix WrapFigure %
%%%%%%%%%%%%%%%%%%

\newcommand{\wrapfill}{\par\ifnum\value{WF@wrappedlines}>0
    \parskip=0pt
    \addtocounter{WF@wrappedlines}{-1}%
    \null\vspace{\arabic{WF@wrappedlines}\baselineskip}%
    \WFclear
\fi}

%%%%%%%%%%%%%%%%%
% Multi Columns %
%%%%%%%%%%%%%%%%%

\let\multicolmulticols\multicols
\let\endmulticolmulticols\endmulticols

\RenewDocumentEnvironment{multicols}{mO{}}
{%
  \ifnum#1=1
    #2%
  \else
    \multicolmulticols{#1}[#2]
  \fi
}
{%
  \ifnum#1=1
\else
  \endmulticolmulticols
\fi
}

\newlength{\thickarrayrulewidth}
\setlength{\thickarrayrulewidth}{5\arrayrulewidth}


%%%%%%%%%%%%%%%%%%%%%%%%%%%%%%%%%%%%%%%%%%%%%%%%%%%%%%%%%%%%%%%%%%%%%%%%%%%%%%%%
%                                                                              %
%                           School Specific Commands                           %
%                                                                              %
%%%%%%%%%%%%%%%%%%%%%%%%%%%%%%%%%%%%%%%%%%%%%%%%%%%%%%%%%%%%%%%%%%%%%%%%%%%%%%%%

%%%%%%%%%%%%%%%%%%%%%%
%  Helpful Commands  %
%%%%%%%%%%%%%%%%%%%%%%

\makeatletter

\newcommand\resetcounters{
  \setcounter{subsection}{0}
  \setcounter{subsubsection}{0}
  \setcounter{paragraph}{0}
  \setcounter{subparagraph}{0}
}

\def\@lecnum{}
\newcommand\lec[1]{
  \ifnum #1<10
    \def\@lecnum{0#1}
  \else
    \def\@lecnum{#1}
  \fi

  % Set the section counter to the number passed.
  \setcounter{section}{#1}
  % Reset all counters.
  \resetcounters
  % Include the lecture file if it exists.
  \IfFileExists{lectures/lec-\@lecnum.tex}{\input{lectures/lec-\@lecnum.tex}}{}
}

%%%%%%%%%%%%%%%%%%%%%
%  Lecture Command  %
%%%%%%%%%%%%%%%%%%%%%

\usepackage{ifthen}
\usepackage{xifthen}

\def\@lecture{}
\newcommand\lecture[2]{%
  \ifthenelse{\isempty{#2}}{%
    \def\@lecture{Lecture \arabic{section}}
  }{%
    \def\@lecture{Lecture \arabic{section}: #2}
  }%

  \if@twocolumn
    \twocolumn[
    \hfill\footnotesize{#1}
    \hrule
    \vspace*{-0.3cm}
    \section*{\@lecture}
    \vspace{0.2cm}
    ]
  \else
    \hfill\sffamily\footnotesize{#1}
    \hrule
    \vspace*{-0.3cm}
    \section*{\@lecture}
  \fi

  \addcontentsline{toc}{section}{\@lecture}
}

%%%%%%%%%%%%%%%%%
% Fancy Headers %
%%%%%%%%%%%%%%%%%

\usepackage{fancyhdr}

\newcommand\forcenewpage{\clearpage\mbox{~}\clearpage\newpage}

\newcommand\createintro{
  \@ifclasswith\class{twocolumn}{\onecolumn}{}
  \pagenumbering{roman}

  % Create title page.
  \begin{center}
    {\LARGE\@title} \\
    {\Large\vspace{0.25cm}\@author} \\
    {\large\vspace{0.25cm}\@date}
  \end{center}

  % Check if the intro.tex file exists.
  % If it does, include it, otherwise, just ignore.
  \IfFileExists{./intro.tex}{Lecture notes from the course \MyTitle, given by professor Victor Ostrik at the \faculty~at \location~in the academic year \academicyear, during the \term term. This course covers symbolic logic, basic set theory, analyzing functions and their properties, modular arithmetic, counting and other problems in discrete mathematics, induction, and convergence of sequences and continuity of functions. Credit for the material in these notes is due to professor Victor, while the structure is loosely taken from the \href{https://www.amazon.com/Mathematical-Reasoning-Writing-Proof-2nd/dp/0131877186}{Mathematical Reasoning: Writing and Proof} textbook. The credit for the typesetting is my own.

\textit{Disclaimer:} This document will inevitably contain some mistakes--both simple typos and legitimate errors. Keep in mind that these are the notes of an undergraduate student in the process of learning the material himself, so take what you read with a grain of salt. If you find mistakes and feel like telling me, I will be grateful and happy to hear from you, even for the most trivial of errors. You can reach me by email, in English, Arabic, Hebrew, or Spanish at \href{mailto:singularisartt@gmail.com}{singularisartt@gmail.com}.
}{}

  % Set the pagestyle to fancy.
  \pagestyle{fancy}
  % Remove the header line.
  \renewcommand\headrulewidth{0pt}

  % Reset fancyhead styles.
  \fancyhead{}
  % Add a fancyfoot center style.
  \fancyfoot[C]{
    \textit{For more notes like this, visit \href{\linktootherpages}{\shortlinkname}}.
  }

  % Create a box with more information.
  \vspace{0.5cm}
  \begin{tcolorbox}[enhanced,colback=white,center upper,size=fbox, drop shadow southwest,sharp corners]
    \term: \academicyear, \\
    Last Update: \today, \\
    \faculty, \location.
  \end{tcolorbox}

  % Create a table of contents.
  \tableofcontents

  \@ifclasswith\class{twocolumn}{%
    % If user passed twocolumn as a parameter to \documentclass, then set the
    % rest of the page to two column.
    \twocolumn%
  }{%
    % Otherwise, just create a new page.
    \newpage%
  }

  % Change the numbering style back to arabic from roman.
  \pagenumbering{arabic}
  % Reset page numbers back to 1.
  \setcounter{page}{1}

  % Add back the header line.
  \renewcommand\headrulewidth{0.4pt}
  % Add the lecture name in the top right.
  \fancyhead[R]{\@lecture}
  % Add the author name in the top left.
  \fancyhead[L]{\@author}
  % Add the page number in the bottom center.
  \fancyfoot[C]{\thepage}
  \@ifclasswith\class{grayfg}{%
    % If user passed twocolumn as a parameter to \documentclass, then set the
    % rest of the page to two column.
    \definegraybackground%
  }{}
}

%%%%%%%%%%%%%%%%%%%%
%  Import Figures  %
%%%%%%%%%%%%%%%%%%%%

\usepackage{import}
\pdfminorversion=7
\newcommand\incfig[2][1]{
  \def\figlocation{./figures/lec-\@lecnum}
  \def\svgwidth{#1\columnwidth}
  \import{\figlocation}{#2.pdf_tex}
}

\makeatother

%%%%%%%%%%%%%%%%%%%%%%%%%%%%%%%%%%%%%%%%%%%%%%%%%%%%%%%%%%%%%%%%%
%  Add vectors using the parallelogram and head to tail method  %
%%%%%%%%%%%%%%%%%%%%%%%%%%%%%%%%%%%%%%%%%%%%%%%%%%%%%%%%%%%%%%%%%

\usepackage{pdftexcmds}
\newcommand\parallelogramRule[6]{
  \begin{tikzpicture}[inner sep=1pt,>=stealth]
    \draw[<->] (#1)--(#2) node[right]{$x$};
    \draw[<->] (#3)--(#4) node[above]{$y$};

    \coordinate (O) at (#1);
    \coordinate (A) at (#5);
    \coordinate (B) at (#6);
    \coordinate (A+B) at ($(A)+(B)$);

    \draw[vector,blue] (O)--(A) node[vector label] {$\vec{x}$};
    \draw[vector,red] (O)--(B) node[vector label] {$\vec{y}$};

    \draw[construction] (A)--(A+B);
    \draw[construction] (B)--(A+B);

    \draw[vector,purple] (O)--(A+B) node[vector label] {$\vec{x}+\vec{y}$};
  \end{tikzpicture}
}

\newcommand\headToTailRule[6]{
  \begin{tikzpicture}[inner sep=1pt,>=stealth]
    \draw[<->] (#1)--(#2) node[right]{$x$};
    \draw[<->] (#3)--(#4) node[above]{$y$};

    \coordinate (O) at (#1);
    \coordinate (A) at (#5);
    \coordinate (B) at (#6);
    \coordinate (A+B) at ($(A)+(B)$);

    \draw[vector,blue] (O)--(A) node[vector label] {$\vec{x}$};
    \draw[construction] (O)--(B);

    \draw[vector,red] (A)--(A+B) node[vector label] {$\vec{y}$};
    \draw[construction] (B)--(A+B);

    \draw[vector,purple] (O)--(A+B) node[vector label] {$\vec{x}+\vec{y}$};
  \end{tikzpicture}
}

\makeatletter

\newcommand\addVectors[7][parallelogram]{
  \ifnum\pdf@strcmp{\unexpanded{#1}}{parallelogram}=0 %
     \expandafter\@firstoftwo
  \else
    \expandafter\@secondoftwo
  \fi
  {\parallelogramRule{#2}{#3}{#4}{#5}{#6}{#7}}
  {\headToTailRule{#2}{#3}{#4}{#5}{#6}{#7}}
}

\makeatother

%%%%%%%%%%%%
%  Circle  %
%%%%%%%%%%%%

\newcommand*\circled[1]{\tikz[baseline=(char.base)]{
  \node[shape=circle,draw,inner sep=1pt] (char) {#1};}
}

%%%%%%%%%%%%%
%  Correct  %
%%%%%%%%%%%%%

\definecolor{correct}{HTML}{009900}
\newcommand\correct[2]{{\color{red}{#1 }}\ensuremath{\to}{\color{correct}{ #2}}}

%%%%%%%%%%%%%%%
%  Important  %
%%%%%%%%%%%%%%%

\newcommand\imp[1]{{\color{red}#1}}

%%%%%%%%%
%  QED  %
%%%%%%%%%

\usepackage{stmaryrd}
\newcommand\contra{\phantom\qedhere\hfill\scalebox{1.1}{$\lightning$}}

%%%%%%%%%%%%%%%%%%%
%  Todo Commands  %
%%%%%%%%%%%%%%%%%%%

\usepackage[colorinlistoftodos]{todonotes}

\makeatletter

\@ifclasswith\class{working}{
  \newcommand\improvement[2][]{\todo[linecolor=Plum,backgroundcolor=Plum!25,bordercolor=Plum,#1]{#2}}
  \newcommand\unsure[2][]{\todo[linecolor=red,backgroundcolor=red!25,bordercolor=red,#1]{#2}}
  \newcommand\change[2][]{\todo[linecolor=yellow,backgroundcolor=yellow!25,bordercolor=yellow,#1]{#2}}
  \newcommand\add[2][]{\todo[linecolor=blue,backgroundcolor=blue!25,bordercolor=blue,#1]{#2}}
  \newcommand\info[2][]{\todo[linecolor=OliveGreen,backgroundcolor=OliveGreen!25,bordercolor=OliveGreen,#1]{#2}}

  \newcommand\improvementinline[2][]{\todo[inline,linecolor=Plum,backgroundcolor=Plum!25,bordercolor=Plum,#1]{#2}}
  \newcommand\unsureinline[2][]{\todo[inline,linecolor=red,backgroundcolor=red!25,bordercolor=red,#1]{#2}}
  \newcommand\changeinline[2][]{\todo[inline,linecolor=yellow,backgroundcolor=yellow!25,bordercolor=yellow,#1]{#2}}
  \newcommand\addinline[2][]{\todo[inline,linecolor=blue,backgroundcolor=blue!25,bordercolor=blue,#1]{#2}}
  \newcommand\infoinline[2][]{\todo[inline,linecolor=OliveGreen,backgroundcolor=OliveGreen!25,bordercolor=OliveGreen,#1]{#2}}
  \newcommand\listnotes{
    \newpage
    \listoftodos[Notes]
  }
}{
  \newcommand\improvement[2][]{}
  \newcommand\unsure[2][]{}
  \newcommand\change[2][]{}
  \newcommand\add[2][]{}
  \newcommand\info[2][]{}

  \newcommand\improvementinline[2][]{}
  \newcommand\unsureinline[2][]{}
  \newcommand\changeinline[2][]{}
  \newcommand\addinline[2][]{}
  \newcommand\infoinline[2][]{}
  \newcommand\listnotes{}
}

\makeatother

%%%%%%%%%%%%%%%%%%%%%%%%%%%%%%%%%%%%%%%%%%%%%%
%  Edit Sections/Subsections/Subsubsections  %
%%%%%%%%%%%%%%%%%%%%%%%%%%%%%%%%%%%%%%%%%%%%%%

\makeatletter
\renewcommand\tagform@[1]{%
  \tikz \draw[color=main] (0,0) rectangle (0.5,0.5)%
  node[pos=0.5] {{\color{main}\textbf{#1}}};%
}
\def\@eqnnum{{\normalfont \normalcolor \theequation}}

\usepackage{titlesec}
\titleformat*{\section}{\sffamily\fontsize{14}{16}\bfseries}
\titleformat*{\subsection}{\sffamily\fontsize{12}{14}\bfseries}

%%%%%%%%%%%%%%%%%%%%%%%%%%%%%%%%%%
%  Edit equation number display  %
%%%%%%%%%%%%%%%%%%%%%%%%%%%%%%%%%%

\makeatletter
\titleformat{\subsubsection}{\color{subsubsection}\sffamily\fontsize{11}{14}\bfseries}{}{0em}{}
\def\@seccntformat#1{\llap{\csname the#1\endcsname\quad}}
\makeatother

%%%%%%%%%%%%%%%%%%
%  Equivalently  %
%%%%%%%%%%%%%%%%%%

\newcommand\equivalently[1]{[ \textbf{Equivalently:} #1 ]}


%%%%%%%%%%%%%%%%%%%%%%%%%%%%%%%%%%%%%%%%%%%%%%%%%%%%%%%%%%%%%%%%%%%%%%%%%%%%%%%%
%                                                                              %
%                                 Environments                                 %
%                                                                              %
%%%%%%%%%%%%%%%%%%%%%%%%%%%%%%%%%%%%%%%%%%%%%%%%%%%%%%%%%%%%%%%%%%%%%%%%%%%%%%%%

\usepackage{varwidth}
\usepackage{thmtools}
\usepackage[most,many,breakable]{tcolorbox}

\mdfsetup{skipabove=1em,skipbelow=0em}
\tcbuselibrary{theorems,skins,hooks}

%%%%%%%%%%%%%%%%%%%
%  Define Colors  %
%%%%%%%%%%%%%%%%%%%

\makeatletter
\@ifclasswith\class{nocolor}{
  \definecolor{main}{HTML}{000000}
  \definecolor{definition}{HTML}{000000}
  \definecolor{example}{HTML}{000000}
  \definecolor{examplebg}{HTML}{ffffff}
  \definecolor{solution}{HTML}{000000}
  \definecolor{qedsolution}{HTML}{000000}
  \definecolor{qedproof}{HTML}{000000}
  \definecolor{subsubsection}{HTML}{000000}

  \newcommand\colframecolor{black}
  \newcommand\colbackcolor{white}
  \newcommand\coluppercolor{black}
}{
  \definecolor{main}{HTML}{F035A3}
  \definecolor{definition}{HTML}{F035A3}
  \definecolor{example}{HTML}{00A6E4}
  \definecolor{examplebg}{HTML}{CAE4F0}
  \definecolor{solution}{HTML}{00AEEF}
  \definecolor{qedsolution}{HTML}{76B4CF}
  \definecolor{qedproof}{HTML}{F035A3}
  \definecolor{subsubsection}{HTML}{B00D15}

  \newcommand\colframecolor{main}
  \newcommand\colbackcolor{white}
  \newcommand\coluppercolor{main}
}
\makeatother

%%%%%%%%%%%%%%%%%%%%%%%%%%%%%%%%%%%%%%%%%%%%%%%%%%%%%%%%%
%  Create Environments Styles Based on Given Parameter  %
%%%%%%%%%%%%%%%%%%%%%%%%%%%%%%%%%%%%%%%%%%%%%%%%%%%%%%%%%

\newcommand\qedsolution{{\color{qedsolution}\rule{4mm}{1.5mm}}}
\newcommand\qedproof{{\color{qedproof}\rule{4mm}{1.5mm}}}
\newcommand\boxitem[1]{
  \tcbox[
    enhanced,
    box align=base,
    nobeforeafter,
    top=0pt,
    bottom=0pt,
    left=0pt,
    right=0pt,
    toprule=0.5pt,
    bottomrule=0.5pt,
    leftrule=0.5pt,
    rightrule=0.5pt,
    fontupper=\scriptsize,
    sharp corners,
    colframe=\colframecolor,
    colback=\colbackcolor,
    colupper=\coluppercolor,
  ]{#1}
}
\makeatother

%%%%%%%%%%%%%%%%%%%%%%%%%%%%%%%%%%%
%  Create the Environment Styles  %
%%%%%%%%%%%%%%%%%%%%%%%%%%%%%%%%%%%

\declaretheoremstyle[
  headfont=\sffamily\bfseries\color{definition},
  headformat=\boxitem{\NUMBER}~\NAME,
  bodyfont=\normalfont,
  notebraces={~ },
  notefont=\bfseries,
  headpunct=,
  mdframed={
    linewidth=0.5pt,
    linecolor=main,
  },
]{thmdefinitionbox}

\declaretheoremstyle[
  headfont=\sffamily\bfseries\color{example}\colorbox{examplebg}{EXAMPLE \arabic{example}},
  headformat=\NOTE,
  notefont=\bfseries,
  bodyfont=\normalfont,
  notefont={\color{black}\bfseries},
  notebraces={~ },
  headpunct=,
]{thmexamplebox}

\declaretheoremstyle[
  headfont=\sffamily\color{solution},
  bodyfont=\normalfont,
  notefont=\bfseries,
  headpunct=,
  qed=\qedsolution,
  spaceabove=\topsep,
  spacebelow=\topsep,
]{thmsolutionbox}

\declaretheoremstyle[
  headfont=\sffamily\bfseries\color{main},
  headformat=\boxitem{\NUMBER}\hspace{-0.1cm}\NOTE,
  bodyfont=\normalfont,
  notebraces={~ },
  notefont=\bfseries,
  headpunct=,
  mdframed={
    linewidth=0.5pt, linecolor=main,
  },
]{thmtheorembox}

\declaretheoremstyle[
  headfont=\sffamily\bfseries\color{main},
  bodyfont=\normalfont,
  notebraces={~ },
  notefont=\bfseries,
  headpunct=,
  mdframed={
    linewidth=0.5pt, linecolor=main,
    rightline=false, topline=false, bottomline=false,
  },
]{thmremarkbox}

\declaretheoremstyle[
  headfont=\sffamily\color{main},
  headindent=0mm,
  bodyfont=\normalfont,
  notefont=\bfseries,
  headpunct=,
  qed=\qedproof,
]{thmreplacementproofbox}

\declaretheoremstyle[
  headfont=\sffamily\color{main},
  headindent=0mm,
  bodyfont=\normalfont,
  notefont=\bfseries,
  notebraces={~ },
  headpunct=,
  mdframed={
    linewidth=0.5pt, linecolor=main,
    topline=true, bottomline=true, leftline=true, rightline=true,
  }
]{thmpurpleframebox}

%%%%%%%%%%%%%%%%%%%%%%%%%%%%%
%  Create the Environments  %
%%%%%%%%%%%%%%%%%%%%%%%%%%%%%

\declaretheorem[numberwithin=section, style=thmdefinitionbox, name=Definition]  {definition}
\declaretheorem[numberwithin=section, style=thmexamplebox,    name=]            {example}
\declaretheorem[numbered=no,          style=thmsolutionbox,   name=SOLUTION]    {solution}
\declaretheorem[numberwithin=section, style=thmtheorembox,    name=Theorem]     {theorem}
\declaretheorem[numbered=no,          style=thmremarkbox,     name=Remark]      {remark}
\declaretheorem[numbered=no,          style=thmreplacementproofbox, name=PROOF] {replacementproof}
\declaretheorem[numbered=no,          style=thmpurpleframebox, name=]           {purpleframe}

\renewenvironment{proof}[1][\proofname]{\begin{replacementproof}}{\end{replacementproof}}

\renewcommand{\thedefinition}{\arabic{definition}}
\renewcommand{\theexample}{\arabic{example}}
\renewcommand{\thetheorem}{\arabic{theorem}}

%%%%%%%%%%%%%%%%%%%%%%%%%%%%%%%
%  Create Plain Environments  %
%%%%%%%%%%%%%%%%%%%%%%%%%%%%%%%

\theoremstyle{definition}

\newtheorem*{note}{Note}
\newtheorem*{notation}{Notation}
\newtheorem*{previouslyseen}{As previously seen}
\newtheorem*{problem}{Problem}
\newtheorem*{observe}{Observe}
\newtheorem*{property}{Property}
\newtheorem*{intuition}{Intuition}
\newtheorem*{questionexpl}{Explanation}
