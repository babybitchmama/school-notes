%%%%%%%%%%%%%%%%%%%%%%%%%%%%%%%%%%%%%%%%%%%%%%%%%%%%%%%%%%%%%%%%%%%%%%%%%%%%%%%%
%                                                                              %
%                              Required Packages                               %
%                                                                              %
%%%%%%%%%%%%%%%%%%%%%%%%%%%%%%%%%%%%%%%%%%%%%%%%%%%%%%%%%%%%%%%%%%%%%%%%%%%%%%%%

% Required for creating documents
\usepackage[utf8]{inputenc}
\usepackage[T1]{fontenc}

% Required math packages
\usepackage{amsmath}
\usepackage{amsfonts}
\usepackage{mathtools}
\usepackage{amsthm}
\usepackage{amssymb}
\usepackage{mathrsfs}

\usepackage{multicol} % for multiple columns
\usepackage[usenames,dvipsnames,pdftex]{color} % Required for nicer colors
\usepackage{hyperref} % Required for hyperlinks
\usepackage{xparse} % Required for \NewDocumentCommand
\usepackage{graphicx} % Required for including images
\usepackage{enumitem} % Required for customizing lists
\usepackage{float} % Required for positioning figures and tables
\usepackage{array} % Required for customizing tables
\usepackage{systeme} % Required for \systeme
\usepackage{cancel} % Required for \cancel
\usepackage{derivative} % Required for \odv and \pdv
\usepackage{authoraftertitle} % Required for \MyTitle
\usepackage{geometry} % Required for adjusting page dimensions
\usepackage{minitoc} % Required for creating a mini table of contents
\usepackage[noabbrev]{cleveref} % Required for clever referencing
\usepackage{emptypage} % Required for removing page numbers on empty pages
\usepackage{nicematrix} % Required for better matrices
\usepackage{booktabs} % Required for better tables
\usepackage{cellspace} % Required for better spacing in tables
\usepackage{longtable} % Required for long tables
\usepackage{xfrac} % Required for extra fraction options
\usepackage{diagbox} % Required for creating diagonal boxes
\usepackage{polynom} % Required for polynomial long division

\usepackage{tasks} % Required for creating tasks
\usepackage[font=bf]{caption} % Required for customizing captions
\usepackage{subcaption} % Required for creating subfigures
\usepackage{siunitx} % Required for SI units
\usepackage{titletoc} % Required for customizing table of contents
\usepackage[framemethod=TikZ]{mdframed} % Required for creating boxes
\usepackage{braket} % Required for creating braket notation

% Required for drawing figures
\usepackage{tikz}
\usepackage{tkz-euclide}
\usepackage{tikz-cd}
\usepackage{tikz-3dplot}

% Required for creating plots
\usepackage{pgfplots}
\usepackage{pgfplotstable}

\usepackage{titling} % Required for customizing title page
\usepackage{ifthen} % Required for if-then-else statements
\usepackage{xifthen} % Required for if-then-else statements
\usepackage{fancyhdr} % Required for customizing headers and footers
\usepackage{import} % Required for importing pdf_tex files
\usepackage{titlesec} % Required for customizing sectioning commands
\usepackage{etex} % Required for more registers

% Required for creating boxes
\usepackage{varwidth}
\usepackage{thmtools}
\usepackage{etoolbox}
\usepackage[most,many,breakable]{tcolorbox}

\makeatletter


%%%%%%%%%%%%%%%%%%%%%%%%%%%%%%%%%%%%%%%%%%%%%%%%%%%%%%%%%%%%%%%%%%%%%%%%%%%%%%%%
%                                                                              %
%                                Basic Settings                                %
%                                                                              %
%%%%%%%%%%%%%%%%%%%%%%%%%%%%%%%%%%%%%%%%%%%%%%%%%%%%%%%%%%%%%%%%%%%%%%%%%%%%%%%%

\ifx\nauthor\undefined
  \def\nauthor{Hashem A. Damrah}
\else
\fi

\ifx\class\undefined
  \def\class{report}
\else
\fi

% Define Colors
\newcommand\newcolor[3]{
  \@ifclasswith\class{nocolor}{
    \definecolor{#1}{HTML}{#3}
  }{
    \definecolor{#1}{HTML}{#2}
  }
}

\newcolor{linecolor1}{3DC6F3}{000000}
\newcolor{linecolor2}{F034A3}{000000}
\newcolor{linecolor3}{F57215}{000000}
\newcolor{linecolor4}{733786}{000000}
\newcolor{linecolor5}{80CF5C}{000000}
\newcolor{linecolor6}{FFD700}{000000}
\newcolor{correct}{00FF00}{00FF00}
\newcolor{incorrect}{FF0000}{FF0000}
\newcolor{tbl}{F035A3}{000000}
\newcolor{main}{F035A3}{000000}
\newcolor{solution}{00AEEF}{000000}
\newcolor{qedsolution}{76B4CF}{000000}
\newcolor{proof}{F035A3}{000000}
\newcolor{subsubsection}{B00D15}{000000}
\newcolor{example}{00A6E4}{000000}
\newcolor{examplebg}{F2FBF8}{FFFFFF}

\colorlet{definition}{main}
\colorlet{theorem}{main}
\colorlet{lemma}{main}
\colorlet{remark}{main}
\colorlet{qedproof}{proof}
\colorlet{question}{example}
\colorlet{questionfg}{examplebg}

% Geometry
\geometry{a4paper, total={170mm,257mm}, left=15mm, top=20mm}

% Tables
\newcolumntype{C}{>{\Centering\arraybackslash}X}
\setlength{\tabcolsep}{5pt}
\renewcommand\arraystretch{1.5}
\renewcommand\thetable{\Roman{table}}
\captionsetup[figure]{font=small}
\captionsetup{justification=centering}
\setlength\cellspacetoplimit{6pt}
\setlength\cellspacebottomlimit{6pt}

% Symbols
\allowdisplaybreaks
\let\svlim\lim\def\lim{\svlim\limits}
\let\svsum\sum\def\sum{\svsum\limits}

% Lists
\setlist[itemize,1]{label=--}
\setlist[itemize,2]{label=\textbullet}
\setlist[enumerate,1]{label=\protect\circled{\arabic*}}

% SI Unitx
\sisetup{
  locale = US,
  per-mode = symbol,
  propagate-math-font = true,
  reset-math-version = false,
  exponent-mode = engineering,
  round-mode = figures,
  round-precision = 3,
  drop-zero-decimal,
}

\DeclareSIUnit{\millimeter}{mm}
\DeclareSIUnit{\centimeter}{cm}
\DeclareSIUnit{\decimeter}{dm}
\DeclareSIUnit{\inch}{in}
\DeclareSIUnit{\foot}{ft}
\DeclareSIUnit{\yard}{yd}
\DeclareSIUnit{\meter}{m}
\DeclareSIUnit{\kilometer}{km}
\DeclareSIUnit{\mile}{mi}
\DeclareSIUnit{\astronomicalunit}{au}
\DeclareSIUnit{\lightyear}{ly}
\DeclareSIUnit{\fahrenheit}{F}
\DeclareSIUnit{\celsius}{C}
\DeclareSIUnit{\millisecond}{ms}
\DeclareSIUnit{\second}{sec}
\DeclareSIUnit{\minute}{min}
\DeclareSIUnit{\hour}{hr}
\DeclareSIUnit{\day}{d}
\DeclareSIUnit{\week}{wk}
\DeclareSIUnit{\month}{mos}
\DeclareSIUnit{\year}{yr}
\DeclareSIUnit{\milligram}{mg}
\DeclareSIUnit{\gram}{g}
\DeclareSIUnit{\ounce}{oz}
\DeclareSIUnit{\pound}{lb}
\DeclareSIUnit{\kilogram}{kg}
\DeclareSIUnit{\ton}{t}
\DeclareSIUnit{\gallon}{gal}
\DeclareSIUnit{\liter}{L}
\DeclareSIUnit{\milliliter}{mL}

% Center Title Page
\renewcommand\maketitlehooka{\null\mbox{}\vfill}
\renewcommand\maketitlehookd{\vfill\null}

% Footnote Line
\renewcommand\footnoterule{\hrule\vspace{0.1cm}}

% Modify Links Color
\hypersetup{
  colorlinks,
  linkcolor=main!90,
  citecolor=black,
  urlcolor=black,
}

% TikZ
\usetikzlibrary{
  intersections,
  angles,
  quotes,
  calc,
  positioning,
  3d,
  arrows,
  arrows.meta,
  patterns,
}

\tikzset{>=stealth}
\tikzset{->-/.style={decoration={markings,mark=at position .5 with {\arrow{>}}},postaction={decorate}}}

% PGF Plots
\pgfplotsset{compat=1.18}

\usepgfplotslibrary{fillbetween}
\usepgfplotslibrary{patchplots}
\usepgfplotslibrary{external}
\usetikzlibrary{decorations.pathreplacing,calligraphy}
\tikzexternalenable

% Tikz styles
\tikzset{style1/.style={color=linecolor1,mark=none,line width=1pt,solid}}
\tikzset{style2/.style={color=linecolor2,mark=none,line width=1pt,solid}}
\tikzset{style3/.style={color=linecolor3,mark=none,line width=1pt,solid}}
\tikzset{style4/.style={color=linecolor4,mark=none,line width=1pt,solid}}
\tikzset{style5/.style={color=linecolor5,mark=none,line width=1pt,solid}}
\tikzset{asymptote/.style={color=gray,mark=none,line width=1pt,<->,dashed}}
\tikzset{soldot/.style={color=linecolor2,fill=linecolor2,only marks,mark=*}}
\tikzset{holdot/.style={color=linecolor2,fill=white,only marks,mark=*}}
\tikzset{integration/.style={name path=curve,color=linecolor1,mark=none,line width=0.5pt,solid}}

\pgfplotsset{style1/.style={color=linecolor1,mark=none,line width=1pt,solid}}
\pgfplotsset{style2/.style={color=linecolor2,mark=none,line width=1pt,solid}}
\pgfplotsset{style3/.style={color=linecolor3,mark=none,line width=1pt,solid}}
\pgfplotsset{style4/.style={color=linecolor4,mark=none,line width=1pt,solid}}
\pgfplotsset{style5/.style={color=linecolor5,mark=none,line width=1pt,solid}}
\pgfplotsset{asymptote/.style={color=gray,mark=none,line width=1pt,<->,dashed}}
\pgfplotsset{soldot/.style={color=linecolor2,only marks,mark=*}}
\pgfplotsset{holdot/.style={color=linecolor2,fill=white,only marks,mark=*}}
\pgfplotsset{integration/.style={name path=curve,color=linecolor1,mark=none,line width=0.5pt,solid}}
\pgfplotsset{horizontal marks/.style={}}
\pgfplotsset{vertical marks/.style={}}

\pgfplotscreateplotcyclelist{stylelist}{
  style1,
  style2,
  style3,
  style4,
  style5,
}

\def\axisdefaultwidth{175pt}
\def\axisdefaultheight{\axisdefaultwidth}
\pgfplotsset{
  every axis/.append style={
    axis x line=middle, x axis line style={name path=xaxis},
    axis y line=middle, y axis line style={name path=yaxis},
    ticks=none,
    axis line style={->},
    xlabel={$x$},
    ylabel={$y$},
    samples=1000,
    cycle list name=stylelist
  },
}

% Edit Section/Subsection/Subsubsection
\titleformat*{\section}{\sffamily\scshape\fontsize{14}{16}\bfseries}
\titleformat*{\subsection}{\color{subsubsection}\scshape\sffamily\fontsize{13}{15}\bfseries}
\def\@seccntformat#1{\llap{\csname the#1\endcsname\quad}}
\renewcommand\thesection{\arabic{section}}

\renewcommand\theequation{\arabic{equation}}
\renewcommand\tagform@[1]{%
  {\color{main}\boxed{\textbf{#1}}}
}
\def\@eqnnum{{\normalfont \normalcolor \theequation}}


%%%%%%%%%%%%%%%%%%%%%%%%%%%%%%%%%%%%%%%%%%%%%%%%%%%%%%%%%%%%%%%%%%%%%%%%%%%%%%%%
%                                                                              %
%                           School Specific Commands                           %
%                                                                              %
%%%%%%%%%%%%%%%%%%%%%%%%%%%%%%%%%%%%%%%%%%%%%%%%%%%%%%%%%%%%%%%%%%%%%%%%%%%%%%%%

%%%%%%%%%%%%%%%%%%%%%%
%  Helpful Commands  %
%%%%%%%%%%%%%%%%%%%%%%

\newcommand\resetcounters{
  \setcounter{section}{0}
  \setcounter{subsection}{0}
  \setcounter{subsubsection}{0}
  \setcounter{paragraph}{0}
  \setcounter{subparagraph}{0}
}

\newcommand*\cleartoleftpage{%
  \clearpage
  \thispagestyle{empty}
  \ifodd\value{page}\else\hbox{}\newpage\fi
}

%%%%%%%%%%%%%%%%%%%%%%%%%%%%%
%  Lecture/Chapter Command  %
%%%%%%%%%%%%%%%%%%%%%%%%%%%%%

\def\notenum{}
\newcommand\includenote[1]{
  \ifnum #1<10
    \def\notenum{0#1}
  \else
    \def\notenum{#1}
  \fi

  \setcounter{chapter}{#1}
  \resetcounters
  \IfFileExists{\noteloc-\notenum.tex}{\input{\noteloc-\notenum.tex}}{}
}

\newcommand\includenotes[2]{
  \foreach \n in {#1,...,#2}{
    \includenote{\n}
  }
}

\newcommand\localtoc{
  \startcontents
  \printcontents{}{1}{\noindent{\color{main}\rule{\textwidth}{0.4pt}\par}\vspace*{-0.3cm}\subsection*{{\color{main}Lecture Note Overview}}}
  \noindent{{\color{main}\rule{\textwidth}{0.4pt}\par}}
}

\newcommand\removetocentry[1]{%
  \addtocontents{toc}{\protect\setcounter{tocdepth}{-1}}
  #1
  \addtocontents{toc}{\protect\setcounter{tocdepth}{\arabic{tocdepth}}}
}

\newcounter{nte}[chapter]
\def\@note{}
\NewDocumentCommand\nte{O{} O{} m m}{%
  \cleartoleftpage
  \setcounter{nte}{\arabic{chapter}}
  \resetcounters
  \ifthenelse{\isempty{#3}}{%
    \def\@note{\lecorchap~\arabic{chapter}}
  }{%
    \def\@note{\lecorchap~\arabic{chapter}: #4}
  }%
  \phantomsection\addcontentsline{toc}{chapter}{\protect\numberline{\thechapter}#4}
  {\fontsize{10}{12}\selectfont\sffamily\ifstrequal{#1}{}{}{\noindent#1}\ifstrequal{#3}{}{}{\hfill#3}}
  \vspace*{0.03cm}
  \hrule
  \vspace*{0.3cm}
  \noindent{\bfseries\sffamily\fontsize{20}{30}\selectfont\@note}

  \ifstrequal{#2}{false}{\vspace{0.25cm}}{\localtoc}
  \setcounter{section}{0}
}

% Intro
\newcommand\createintro{
  \maketitle

  \pagenumbering{roman}
  \begin{center}
    \textbf{{\LARGE Introduction}}
  \end{center}

  \IfFileExists{./intro.tex}{Lecture notes from the course \MyTitle, given by professor Victor Ostrik at the \faculty~at \location~in the academic year \academicyear, during the \term term. This course covers symbolic logic, basic set theory, analyzing functions and their properties, modular arithmetic, counting and other problems in discrete mathematics, induction, and convergence of sequences and continuity of functions. Credit for the material in these notes is due to professor Victor, while the structure is loosely taken from the \href{https://www.amazon.com/Mathematical-Reasoning-Writing-Proof-2nd/dp/0131877186}{Mathematical Reasoning: Writing and Proof} textbook. The credit for the typesetting is my own.

\textit{Disclaimer:} This document will inevitably contain some mistakes--both simple typos and legitimate errors. Keep in mind that these are the notes of an undergraduate student in the process of learning the material himself, so take what you read with a grain of salt. If you find mistakes and feel like telling me, I will be grateful and happy to hear from you, even for the most trivial of errors. You can reach me by email, in English, Arabic, Hebrew, or Spanish at \href{mailto:singularisartt@gmail.com}{singularisartt@gmail.com}.
}{}

  \pagestyle{fancy}
  \renewcommand\headrulewidth{0pt}

  \fancyhead{}
  \fancyfoot[C]{%
    \textit{For more notes like this, visit \href{\linktootherpages}{\shortlinkname}}.%
  }%

  \begin{tcolorbox}[enhanced,colback=white,center upper,size=fbox, drop shadow southwest,sharp corners]
    \term: \academicyear, \\
    Last Update: \today, \\
    \faculty, \location.
  \end{tcolorbox}

  \newpage
  \tableofcontents

  \pagenumbering{arabic}
  \setcounter{page}{1}

  \renewcommand\headrulewidth{0.4pt}
  \fancyhead[R]{\@note}
  \fancyhead[L]{\nauthor}
  \fancyfoot[C]{\thepage}
}

%%%%%%%%%%%%%%%%%%%%%
%  Random Commands  %
%%%%%%%%%%%%%%%%%%%%%

% Import Figures
\newcommand\incimg[2][1]{%
  \includegraphics[width=#1\columnwidth]{\figloc-\notenum/#2}%
}

\newcommand\incfig[2][1]{
  \def\svgwidth{#1\columnwidth}
  \import{\figloc-\notenum}{#2.pdf_tex}
}

% Circle
\newcommand*\circled[1]{
  \tikz[baseline=(char.base)] {
    \node[shape=circle,draw,inner sep=1pt] (char) {#1};
  }
}
\newcommand\coloredcircle[1]{\tikz[baseline=(char.base)] {
    \node[shape=circle,draw,inner sep=1pt,color=#1,fill=#1!50] (char) {\phantom{$1$}};
  }
}

% Correct
\newcommand\correct[1]{\textcolor{correct}{#1}}
\newcommand\incorrect[1]{{\color{incorrect}#1}}
\newcommand\inctocor[2]{\incorrect{#1} \ensuremath{\to} \correct{#2}}

% Bracket
\renewcommand\bra[1]{\left\langle#1\right|}
\renewcommand\ket[1]{\left|#1\right\rangle}
\renewcommand\braket[2]{\left\langle#1\middle|#2\right\rangle}
\renewcommand\ang[1]{\left\langle#1\right\rangle}

% Important
\newcommand\imp[1]{{\color{main}#1}}

% For diagonal strikeout in red
\newcommand\rcancel[1]{\renewcommand\CancelColor{\color{red}}\cancel{#1}}

% For differentials
\newcommand\dd[1]{\textrm{d}#1}
\newcommand\dA{\textrm{d}A}
\newcommand\dm{\textrm{d}m}
\newcommand\dt{\textrm{d}t}
\newcommand\dT{\textrm{d}T}
\newcommand\du{\textrm{d}u}
\newcommand\dv{\textrm{d}v}
\newcommand\dx{\textrm{d}x}
\newcommand\dy{\textrm{d}y}
\newcommand\dz{\textrm{d}z}

% Helpful text in math
\newcommand\echelon{\underrightarrow{\textrm{ echelon form }}}
\newcommand\rref{\underrightarrow{\textrm{ rref }}}
\newcommand\pick[1]{\xrightarrow[#1]{\textrm{ pick }}}
\newcommand\generalsol{\xrightarrow[\textrm{solution}]{\textrm{ general }}}
\newcommand\ngeneralsol{\parbox{4em}{general \\ solution}\textrm{:}}
\renewcommand\and{\text{and}}
\newcommand\aand{\quad\text{and}\quad}
\newcommand\adj{\text{adj}}
\newcommand\qtq[1]{\quad\textrm{#1}\quad}
\newcommand\oor{\quad\text{or}\quad}
\newcommand\Col{\textrm{Col}}
\newcommand\Nul{\textrm{Null}}
\newcommand\Tr{\textrm{Tr}}
\newcommand\Row{\textrm{Row}}
\newcommand\ran{\textrm{rank}}
\newcommand\dist{\text{dist}}
\newcommand\sz{\stackrel{\textrm{set}}{=}}
\newcommand\ce{\overset{\checkmark}{=}}
\newcommand\proj{\text{proj}}
\newcommand\comp{\text{comp}}
\newcommand\geogebra{\textsf{GeoGebra}}
\newcommand\Sspan{\text{Span}}

% Used to switch tag position
\newcommand\leqnomode{\tagsleft@true}
\newcommand\reqnomode{\tagsleft@false}

% Laplace
\newcommand\laplace[1]{\mathscr{L}\left\{#1\right\}}
\newcommand\ilaplace[1]{\mathscr{L}^{-1}\left\{#1\right\}}

% Vectors
\renewcommand\a{\mathbf{a}}
\renewcommand\b{\mathbf{b}}
\renewcommand\c{\mathbf{c}}
\renewcommand\d{\mathbf{d}}
\newcommand\e{\mathbf{e}}
\newcommand\f{\mathbf{f}}
\newcommand\g{\mathbf{g}}
\newcommand\n{\mathbf{n}}
\newcommand\p{\mathbf{p}}

\newcommand\RR{\mathbf{R}}
\newcommand\FF{\mathbf{F}}

\renewcommand\r{\mathbf{r}}
\newcommand\rr{\mathbf{r}^{\prime}}
\newcommand\rrr{\mathbf{r}^{\prime\prime}}

\newcommand\Ta{\mathbf{T}}
\newcommand\Taa{\mathbf{T}^{}}
\newcommand\N{\mathbf{N}}

\renewcommand\u{\mathbf{u}}
\newcommand\uu{\mathbf{u}^{\prime}}
\newcommand\uuu{\mathbf{u}^{\prime\prime}}

\renewcommand\v{\mathbf{v}}
\newcommand\vv{\mathbf{v}^{\prime}}
\newcommand\vvv{\mathbf{v}^{\prime\prime}}

\newcommand\w{\mathbf{w}}
\newcommand\x{\mathbf{x}}
\newcommand\y{\mathbf{y}}
\newcommand\z{\mathbf{z}}

\newcommand\zero{\mathbf{0}}

% Hat vectors
\newcommand\ah{\hat{\mathbf{a}}}
\newcommand\bh{\hat{\mathbf{b}}}
\newcommand\ch{\hat{\mathbf{c}}}
\renewcommand\dh{\hat{\mathbf{d}}}
\newcommand\eh{\hat{\mathbf{e}}}
\newcommand\ph{\hat{\mathbf{p}}}
\newcommand\uh{\hat{\mathbf{u}}}
\newcommand\vh{\hat{\mathbf{v}}}
\newcommand\wh{\hat{\mathbf{w}}}
\newcommand\xh{\hat{\mathbf{x}}}
\newcommand\yh{\hat{\mathbf{y}}}
\newcommand\zh{\hat{\mathbf{z}}}

% Unit vectors
\newcommand\ui{\mathbf{i}}
\newcommand\uj{\mathbf{j}}
\newcommand\uk{\mathbf{k}}

% Subspaces
\newcommand\B{\mathcal{B}}
\newcommand\CC{\mathbb{C}}
\newcommand\C{\mathcal{C}}
\newcommand\D{\mathcal{D}}
\newcommand\E{\mathcal{E}}
\newcommand\F{\mathcal{F}}
\newcommand\R{\mathbb{R}}
\newcommand\Z{\mathbb{Z}}
\newcommand\U{\mathcal{U}}
\newcommand\X{\mathcal{X}}
\newcommand\Y{\mathcal{Y}}
\newcommand\W{\mathbf{w}}

% Create command to box equations
\newcommand*\colorboxed{}
\def\colorboxed#1#{%
  \colorboxedAux{#1}%
}
\newcommand*\colorboxedAux[3]{%
  \begingroup
    \colorlet{cb@saved}{.}%
    \color#1{#2}%
    \boxed{%
      \color{cb@saved}%
      #3%
    }%
  \endgroup
}
\newcommand\empheq[1]{%
  \colorboxed{black}{%
    \begin{aligned}[b]
      #1
    \end{aligned}%
  }%
}

% L'Hopital's Rule
\newcommand\lop{\stackrel{\textrm{H}}{=}}

% Check mark
\def\checkmark{\tikz\fill[scale=0.4](0,.35) -- (.25,0) -- (1,.7) -- (.25,.15) -- cycle;} 

% Vinculum
\newcommand\vinculum[1]{\frac{\hspace{#1cm}}{}}

% Phase Lines (vertical/horizontal)
\newcommand*\TickSize{2pt}%
\newcommand*\AxisMin{0}%
\newcommand*\AxisMax{0}%
\newcommand*\PhaseLine[4][]{%
  \gdef\AxisMin{0}%
  \gdef\AxisMax{0}%
  \edef\MyList{#2}%
  \foreach \X in \MyList {
    \draw (-\TickSize,\X) -- (\TickSize,\X) node [right] {$\X$};
    \ifnum\AxisMin>\X
      \xdef\AxisMin{\X}%
    \fi
    \ifnum\AxisMax<\X
      \xdef\AxisMax{\X}%
    \fi
  }

  \edef\MyList{#3}%
  \foreach \X in \MyList {% Up arrows
    \draw [->] (0,\X-0.1) -- (0,\X);
    \ifnum\AxisMin>\X
      \xdef\AxisMin{\X}%
    \fi
    \ifnum\AxisMax<\X
      \xdef\AxisMax{\X}%
    \fi
  }

  \edef\MyList{#4}%
  \foreach \X in \MyList {% Down arrows
    \draw [->] (0,\X+0.1) -- (0,\X);
    \ifnum\AxisMin>\X
      \xdef\AxisMin{\X}%
    \fi
    \ifnum\AxisMax<\X
      \xdef\AxisMax{\X}%
    \fi
  }

  \draw  (0,\AxisMin-1) -- (0,\AxisMax+1) node [above] {#1};
}%


%%%%%%%%%%%%%%%%%%%%%%%%%%%%%%%%%%%%%%%%%%%%%%%%%%%%%%%%%%%%%%%%%%%%%%%%%%%%%%%%
%                                                                              %
%                                 Environments                                 %
%                                                                              %
%%%%%%%%%%%%%%%%%%%%%%%%%%%%%%%%%%%%%%%%%%%%%%%%%%%%%%%%%%%%%%%%%%%%%%%%%%%%%%%%

\mdfsetup{skipabove=1em,skipbelow=0em}
\tcbuselibrary{theorems,skins,hooks}

\@ifclasswith{\class}{nocolor}{
  \renewcommand\qedsymbol{{\color{black}\rule{4mm}{1.5mm}}}

  \theoremstyle{definition}

  \newtheorem{definition}{Definition}
  \newtheorem{question}{Question}
  \newtheorem{sol}{Solution}

  % This is a very hacky way of creating an environment that has a qed symbol at
  % the end of it, as you cannot create it using \newtheorem for some odd reason
  % (unless I'm just an idiot).
  \newcounter{solutioncnt}
  \setcounter{solutioncnt}{1}
  \newcommand\solutionname{\normalfont\textbf{Solution \arabic{solutioncnt}.}}
  \newenvironment{solution}%
  {\renewcommand\proofname\solutionname%
  \begin{proof}}%
  {\stepcounter{solutioncnt}\end{proof}}

  \newtheorem*{remark}{Remark}
  \newtheorem*{note}{Note}
  \newtheorem*{example}{Example}

  \newtheorem*{notation}{Notation}
  \newtheorem*{recall}{Recall}
  \newtheorem*{claim}{Claim}
  \newtheorem{case}{Case}
  \newtheorem*{acknowledgment}{Acknowledgment}
  \newtheorem*{conclusion}{Conclusion}

  \newtheorem{theorem}{Theorem}
  \newtheorem{lemma}{Lemma}[theorem]
  \newtheorem{corollary}{Corollary}[theorem]
  \newtheorem{proposition}{Proposition}[theorem]
  \newtheorem{conjecture}{Conjecture}[theorem]
  \newtheorem{explanation}{Explanation}[theorem]
}{
  \newcommand\qedsolution{{\color{qedsolution}\rule{4mm}{1.5mm}}}
  \newcommand\qedproof{{\color{qedproof}\rule{4mm}{1.5mm}}}

  \declaretheoremstyle[
    headfont=\sffamily\bfseries\color{definition},
    headformat=\fbox{\arabic{definition}}~\NAME\NOTE,
    bodyfont=\normalfont,
    headpunct=,
    mdframed={
      linewidth=0.5pt,
      linecolor=definition,
    },
  ]{thmdefinitionbox}

  \declaretheoremstyle[
    headfont=\sffamily\bfseries\color{theorem},
    headformat=\fbox{\arabic{theorem}}~\NAME\NOTE,
    bodyfont=\normalfont,
    headpunct=,
    mdframed={
      linewidth=0.5pt,
      linecolor=theorem,
    },
  ]{thmtheorembox}

  \declaretheoremstyle[
  headfont=\sffamily\bfseries\color{lemma},
  headformat=\fbox{\arabic{lemma}}~\NAME\NOTE,
  bodyfont=\normalfont,
  headpunct=,
  mdframed={
    linewidth=0.5pt,
    linecolor=lemma,
  },
  ]{thmlemmabox}

  \declaretheoremstyle[
  headfont=\sffamily\bfseries\color{question}\colorbox{questionfg}{QUESTION \arabic{question}},
  headformat=\textbf{\NOTE},
  notefont=\bfseries,
  bodyfont=\normalfont,
  notefont={\color{black}\bfseries},
  notebraces={~ },
  headpunct=,
  ]{thmquestionbox}

  \declaretheoremstyle[
  headfont=\sffamily\bfseries\color{example},
  headformat=\NAME\NOTE,
  bodyfont=\normalfont,
  headpunct=,
  mdframed={
    linewidth=0.5pt,
    linecolor=example,
    backgroundcolor=examplebg,
  },
  ]{thmexamplebox}

  \declaretheoremstyle[
  headfont=\sffamily\color{solution},
  headformat=\NAME~\arabic{solution},
  notefont=\bfseries,
  headpunct=,
  qed=\qedsolution,
  spaceabove=\topsep,
  spacebelow=\topsep,
  ]{thmsolutionbox}

  \declaretheoremstyle[
  headfont=\sffamily\bfseries\color{remark},
  bodyfont=\itshape,
  notebraces={~ },
  notefont=\bfseries,
  headpunct=,
  mdframed={
    linewidth=0.5pt, linecolor=remark,
    rightline=false, topline=false, bottomline=false,
  },
  ]{thmremarkbox}

  \declaretheoremstyle[
  headfont=\sffamily\color{proof},
  headformat=\NAME,
  headindent=0mm,
  bodyfont=\normalfont,
  notefont=\bfseries,
  headpunct=,
  qed=\qedproof,
  ]{thmreplacementproofbox}

  \declaretheoremstyle[
  headformat=\NAME\NOTE,
  headfont=\sffamily\bfseries,
  headindent=24pt,
  bodyfont=\normalfont,
  notefont=\bfseries,
  notebraces={~},
  headpunct=:,
  ]{thmmaindefinitionbox}

  \declaretheoremstyle[
  headfont=\sffamily,
  headindent=24pt,
  bodyfont=\itshape,
  notefont=\bfseries,
  notebraces={~ },
  headpunct=:,
  ]{thmmainplainbox}

  \declaretheorem[style=thmdefinitionbox,       name=Definition]      {definition}
  \declaretheorem[style=thmtheorembox,          name=Theorem]         {theorem}
  \declaretheorem[style=thmlemmabox,            name=Lemma]           {lemma}
  \declaretheorem[style=thmexamplebox,          name=Example]         {example}
  \declaretheorem[style=thmquestionbox,         name=]                {question}
  \declaretheorem[style=thmsolutionbox,         name=SOLUTION]        {solution}
  \declaretheorem[style=thmremarkbox,           name=Remark]          {remark}
  \declaretheorem[style=thmreplacementproofbox, name=PROOF]           {replacementproof}

  \declaretheorem[style=thmmaindefinitionbox,   name=Note]            {note}
  \declaretheorem[style=thmmaindefinitionbox,   name=Problem]         {problem}

  \declaretheorem[numbered=no,          style=thmmainplainbox,        name=Corollary]       {corollary}
  \declaretheorem[numbered=no,          style=thmmainplainbox,        name=Proposition]     {proposition}
  \declaretheorem[numbered=no,          style=thmmainplainbox,        name=Conjecture]      {conjecture}
  \declaretheorem[numbered=no,          style=thmmainplainbox,        name=Explanation]     {explanation}
  \declaretheorem[numbered=no,          style=thmmainplainbox,        name=Notation]        {notation}
  \declaretheorem[numbered=no,          style=thmmainplainbox,        name=Recall]          {recall}
  \declaretheorem[numbered=no,          style=thmmainplainbox,        name=Claim]           {claim}
  \declaretheorem[numbered=no,          style=thmmainplainbox,        name=Case]            {case}
  \declaretheorem[numbered=no,          style=thmmainplainbox,        name=Acknowledgment]  {acknowledgment}
  \declaretheorem[numbered=no,          style=thmmainplainbox,        name=Conclusion]      {conclusion}

  \renewenvironment{proof}[1][\proofname]{\begin{replacementproof}}{\end{replacementproof}}
}

\declaretheoremstyle[
  headfont=\sffamily\color{main},
  headformat=\NOTE,
  headindent=0mm,
  bodyfont=\normalfont\centering,
  notefont=\bfseries,
  notebraces={~ },
  headpunct=,
  mdframed={
    linewidth=0.5pt, linecolor=main,
    topline=true, bottomline=true, leftline=true, rightline=true,
  }
]{thmpurpleframebox}

\declaretheorem[style=thmpurpleframebox, name=]{purpleframe}

%%%%%%%%%%%%%%%%%%
%  Cref Styling  %
%%%%%%%%%%%%%%%%%%

\crefname{definition}{definition}{definitions}
\Crefname{definition}{Definition}{Definitions}

\crefname{theorem}{theorem}{theorems}
\Crefname{theorem}{Theorem}{Theorems}

\crefname{example}{example}{examples}
\Crefname{example}{Example}{Examples}

\crefname{question}{question}{questions}
\Crefname{question}{Question}{Questions}

\crefname{solution}{solution}{solutions}
\Crefname{solution}{Solution}{Solutions}

\crefname{remark}{remark}{remarks}
\Crefname{remark}{Remark}{Remarks}

\crefname{note}{note}{notes}
\Crefname{note}{Note}{Notes}

\crefname{problem}{problem}{problem}
\Crefname{problem}{Problem}{Problems}

\crefname{lemma}{lemma}{lemmas}
\Crefname{lemma}{Lemma}{Lemmas}

\crefname{corollary}{corollary}{corollaries}
\Crefname{corollary}{Corollary}{Corollaries}

\crefname{proposition}{proposition}{propositions}
\Crefname{proposition}{Proposition}{Propositions}

\crefname{conjecture}{conjecture}{conjectures}
\Crefname{conjecture}{Conjecture}{Conjectures}

\crefname{explanation}{explanation}{explanations}
\Crefname{explanation}{Explanation}{Explanations}

\crefname{notation}{notation}{notations}
\Crefname{notation}{Notation}{Notations}

\crefname{recall}{recall}{recalls}
\Crefname{recall}{Recall}{Recalls}

\crefname{claim}{claim}{claims}
\Crefname{claim}{Claim}{Claims}

\crefname{case}{case}{cases}
\Crefname{case}{Case}{Cases}

\crefname{acknowledgment}{acknowledgment}{acknowledgments}
\Crefname{acknowledgment}{Acknowledgment}{Acknowledgments}

\crefname{conclusion}{conclusion}{conclusions}
\Crefname{conclusion}{Conclusion}{Conclusions}

\creflabelformat{equation}{#2\textup{#1}#3}
\creflabelformat{table}{#2\textup{#1}#3}
\creflabelformat{figure}{#2\textup{#1}#3}

%%%%%%%%%%%%%%%%%%%%%%%%%%%%%%%%%%%%%%%%%%%%%%%%%%%%%%%%%%%%%%%%%%%%%%%
%                                                                     %
%                          Table of Contents                          %
%                                                                     %
%%%%%%%%%%%%%%%%%%%%%%%%%%%%%%%%%%%%%%%%%%%%%%%%%%%%%%%%%%%%%%%%%%%%%%%

\contentsmargin{0cm}
\makeatletter

\titlecontents{chapter}[3.7pc]
  {%
    \addvspace{30pt}%
    \begin{tikzpicture}[remember picture, overlay]%
      \draw[fill=tbl,draw=tbl] (-7,-.1) rectangle (0.2,.5);%
      \pgftext[left,x=-3cm,y=0.2cm]{\color{white}\Large\sc\bfseries \lecorchap~\thecontentslabel\\*\hspace*{0.7em}\ \thecontentslabel};%
    \end{tikzpicture}%
    ~~\color{tbl}\large\sc\bfseries%
  }%
  {}
  {}
  {\;\titlerule\;\large\sc\bfseries Page \thecontentspage
    \begin{tikzpicture}[remember picture, overlay]
      \draw[fill=tbl,draw=tbl] (2pt,0) rectangle (4,0.1pt);
    \end{tikzpicture}%
  }

\titlecontents{section}[3.7pc]
  {\addvspace{2pt}}
  {\contentslabel[\thecontentslabel]{2pc}}
  {}
  {\hfill\small \thecontentspage}
  []%

\titlecontents{subsection}[5.75pc]
  {\addvspace{2pt}}
  {\contentslabel[\thecontentslabel]{2pc}}
  {}
  {\hfill\small \thecontentspage}
  []%

\renewcommand\tableofcontents{%
  \chapter*{%
    \vspace*{-20\p@}%
    \begin{tikzpicture}[remember picture, overlay]%
      \pgftext[right,x=15cm,y=0.2cm]{\color{tbl}\Huge\sc\bfseries \contentsname};%
      \draw[fill=tbl,draw=tbl] (13,-.75) rectangle (20,1);%
      \clip (13,-.75) rectangle (20,1);%
      \pgftext[right,x=15cm,y=0.2cm]{\color{white}\Huge\sc\bfseries \contentsname};%
    \end{tikzpicture}}%
  \@starttoc{toc}%
}%

\makeatother
