\begin{problem}[3.2.1]
  Show that at a hyperbolic point, the principle directions bisect the asymptotic directions.
\end{problem}

\begin{solution}
  Let $p$ be a hyperbolic point on a regular surface, with principal curvatures $\kappa_1 > 0$ and $\kappa_2 < 0$, and corresponding orthonormal principal directions $\e_1$ and $\e_2$. Let $\phi$ be the angle a direction makes with $\e_1$. By Euler's formula, the normal curvature in that direction is
  \[%
    \kappa_n(\phi) = \kappa_1 \cos^2(\phi) + \kappa_2 \sin^2(\phi)
  .\]%
  Asymptotic directions satisfy $\kappa_n(\phi) = 0$. Solving,
  \[%
    \kappa_1 + \kappa_2 \tan^2(\phi) = 0 \implies \tan^2(\phi) = -\frac{\kappa_1}{\kappa_2} = C^2 > 0
  ,\]%
  so $\tan(\phi) = \pm C$. These correspond to angles $\phi = \pm \phi_0$ symmetric about $\phi = 0$.

  The angle bisectors of two lines at angles $\pm\phi_0$ are $\phi = 0$ and $\phi = \sfrac{\pi}{2}$, which correspond to the principal directions $\e_1$ and $\e_2$. Thus, the principal directions bisect the asymptotic directions.
\end{solution}

\begin{problem}[3.2.3]
  Let $C \subset S$ be a regular curve on a regular surface $S$ with Gaussian curvature $K > 0$. Show that the curvature $\kappa$ of $C$ at $p$ satisfies
  \[%
    \abs{\kappa} \ge \min(\abs{\kappa_1}, \abs{\kappa_2})
  ,\]%
  where $\kappa_1$ and $\kappa_2$ are the principal curvatures of $S$ at $p$.
\end{problem}

\begin{solution}
  Let $C \subset S$ be a regular curve on a regular surface $S$, and let $p \in C$. Suppose the Gaussian curvature at $p$ satisfies $K(p) > 0$, so the principal curvatures $\kappa_1, \kappa_2$ are both nonzero and have the same sign. Let $\kappa$ be the curvature of $C$ at $p$.

  By Meusnier's Proposition, the curvature $\kappa$ of $C$ satisfies $\abs{\kappa} \geq \abs{\kappa_n}$, where $\kappa_n$ is the normal curvature of $S$ at $p$ in the direction of the tangent vector to $C$. Euler's formula expresses $\kappa_n$ as
  \[%
    \kappa_n = \kappa_1 \cos^2(\phi) + \kappa_2 \sin^2(\phi)
  ,\]%
  for some angle $\phi$. Since $\kappa_1, \kappa_2$ have the same sign and neither is zero, it follows that
  \[%
    \abs{\kappa_n} \geq \min(\abs{\kappa_1}, \abs{\kappa_2})
  .\]%
  Combining the two inequalities, we conclude
  \[%
    \abs{\kappa} \geq \abs{\kappa_n} \geq \min(\abs{\kappa_1}, \abs{\kappa_2})
  .\qedhere\]%
\end{solution}

\begin{problem}[3.2.4]
  Assume that a surface $S$ has the property that $\abs{\kappa_1} \le 1$, $\abs{\kappa_1} \le 1$ everywhere. Is it true that the curvature $\kappa$ of a curve on $S$ also satisfies $\abs{\kappa} < 1$?
\end{problem}

\begin{solution}
  We are given a surface $S$ such that the principal curvatures $\kappa_1$ and $\kappa_2$ at every point satisfy $\abs{\kappa_1} \le 1$ and $\abs{\kappa_2} \le 1$. We are asked whether this implies that the curvature $\kappa$ of any curve $C \subset S$ satisfies $\abs{\kappa} < 1$. The curvature $\kappa$ of a curve on a surface satisfies
  \[%
    \kappa^2 = \kappa_n^2 + \kappa_g^2
  ,\]%
  where $\kappa_n$ is the normal curvature in the direction of the curve and $\kappa_g$ is the geodesic curvature. As in the previous problem, the normal curvature $\kappa_n$ is bounded by the principal curvatures:
  \[%
    \abs{\kappa_n} \le \max(\abs{\kappa_1}, \abs{\kappa_2}) \le 1
  .\]%
  However, the geodesic curvature $\kappa_g$ is not bounded by the principal curvatures of the surface—it depends on how the curve bends within the surface, not how the surface bends in space.

  Therefore, the curve can have large geodesic curvature even if the surface is almost flat (i.e., small or zero principal curvatures). For example, consider a small circle drawn on a flat plane: the plane has $\kappa_1 = \kappa_2 = 0$, but the circle can have arbitrarily large curvature depending on its radius.

  Hence, a curve on such a surface can have $\abs{\kappa} > 1$, even if $\abs{\kappa_i} \le 1$. Therefore, the statement is false.
\end{solution}

\begin{problem}[3.2.5]
  Show that the mean curvature $H$ at $p \in S$ is given by
  \[%
    H = \frac{1}{\pi} \int_0^\pi \kappa_n(\theta) \dd{\theta}
  ,\]%
  where $\kappa_n(\theta)$ is the normal curvature at $p$ along a direction making an angle $\theta$ with a fixed direction.
\end{problem}

\begin{solution}
  Let $S$ be a regular surface and $p \in S$ a point. Denote the principal curvatures at $p$ by $\kappa_1$ and $\kappa_2$, and let $H = \sfrac{(\kappa_1 + \kappa_2)}{2}$ be the mean curvature at $p$. We are asked to show that
  \[%
    H = \frac{1}{\pi} \int_0^\pi \kappa_n(\theta) \dd{\theta}
  ,\]%
  where $\kappa_n(\theta)$ is the normal curvature in the direction forming an angle $\theta$ with a fixed principal direction.

  To proceed, we use Euler's formula for normal curvature. Let $\theta$ be the angle between a given direction in the tangent plane and the principal direction corresponding to $\kappa_1$. Then Euler's formula states:
  \[%
    \kappa_n(\theta) = \kappa_1 \cos^2(\theta) + \kappa_2 \sin^2(\theta)
  .\]%
  Integrating both sides from $\theta = 0$ to $\pi$, we compute
  \[%
    \int_0^\pi \kappa_n(\theta) \dd{\theta} = \int_0^\pi \kappa_1 \cos^2(\theta) + \kappa_2 \sin^2(\theta) \dd{\theta} = \kappa_1 \int_0^\pi \cos^2(\theta) \dd{\theta} + \kappa_2 \int_0^\pi \sin^2(\theta) \dd{\theta}
  .\]%

  Using the standard trigonometric identities
  \[%
    \cos^2(\theta) = \frac{1 + \cos(2\theta)}{2}
    \aand
    \sin^2(\theta) = \frac{1 - \cos(2\theta)}{2}
  ,\]%
  we find
  \[%
    \int_0^\pi \cos^2(\theta) \dd{\theta} = \int_0^\pi \frac{1 + \cos(2\theta)}{2} \dd{\theta} = \frac{\pi}{2}
  ,\]%
  \[%
    \int_0^\pi \sin^2(\theta) \dd{\theta} = \frac{\pi}{2}
  .\]%
  So the integral becomes
  \[%
    \int_0^\pi \kappa_n(\theta) \dd{\theta} = \kappa_1 \cdot \frac{\pi}{2} + \kappa_2 \cdot \frac{\pi}{2} = \frac{\pi}{2}(\kappa_1 + \kappa_2)
  .\]%
  Dividing both sides by $\pi$, we obtain
  \[%
    \frac{1}{\pi} \int_0^\pi \kappa_n(\theta) \dd{\theta} = \frac{1}{2}(\kappa_1 + \kappa_2) = H
  .\qedhere\]%
\end{solution}

\begin{problem}[3.2.8]
  Describe the region of the unit sphere covered by the image of the Gauss map of the following surfaces:
  \begin{enumerate}
    \item Paraboloid of revolution $z = x^2 + y^2$.

    \item Hyperboloid of revolution $x^2 + y^2 - z^2 = 1$.

    \item Catenoid $x^2 + y^2 = \cosh^2(z)$.
  \end{enumerate}
\end{problem}

\begin{solution}[(i)]
  Consider the paraboloid of revolution $z = x^2 + y^2$. This surface is strictly convex and opens upward. At any point $(x, y, z)$, the upward-pointing unit normal vector (obtained from the gradient of the implicit function $F(x, y, z) = z - x^2 - y^2$) is
  \[%
    \mathbf{N}(x, y, z) = \frac{(-2x, -2y, 1)}{\sqrt{4x^2 + 4y^2 + 1}}
  .\]%
  Since the denominator is always positive, the $z$-component of the normal vector is always positive. Therefore, the image of the Gauss map lies entirely in the upper hemisphere of the unit sphere.

  Furthermore, as $(x, y) \to \infty$, the normal vectors approach the horizontal plane (i.e., the $z$-component of $\Na$ approaches 0), so the Gauss map approaches the equator of the sphere. Meanwhile, near the origin, the normal vector is nearly vertical, so we also cover points near the north pole.

  Hence, the image of the Gauss map is the open upper hemisphere (excluding the equator).
\end{solution}

\begin{solution}[(ii)]
  The hyperboloid of revolution $x^2 + y^2 - z^2 = 1$ is a two-sheeted surface when restricted to $z^2 > 1$, but usually we refer to the one-sheeted hyperboloid, which is connected and given by this equation.

  At each point, the unit normal vector (from the gradient of $F(x, y, z) = x^2 + y^2 - z^2 - 1$) is
  \[%
    \Na(x, y, z) = \frac{(2x, 2y, -2z)}{\sqrt{4x^2 + 4y^2 + 4z^2}} = \frac{(x, y, -z)}{\sqrt{x^2 + y^2 + z^2}}
  .\]%
  Since $z$ ranges over all real numbers, the $z$-component of the unit normal vector, $-\sfrac{z}{\sqrt{x^2 + y^2 + z^2}}$, also ranges over all values in $(-1, 1)$.

  As a result, the image of the Gauss map covers the entire unit sphere except possibly the north and south poles, because the normal vectors never exactly point in the $\pm z$-directions due to the surface always being slanted. So the Gauss map image is the entire unit sphere minus the north and south poles.
\end{solution}

\begin{solution}[(iii)]
  For the catenoid $x^2 + y^2 = \cosh^2(z)$, we consider its parametrization
  \[%
    \x(u, v) = (\cosh u \cos v, \cosh u \sin v, u)
  ,\]%
  where $u \in \R$, $v \in [0, 2\pi)$. This surface is minimal, meaning its mean curvature vanishes everywhere, and it is symmetric about the $z$-axis.

  The unit normal vector has both upward and downward components depending on the sign of $u$, and the $z$-component of the normal vector can take on both positive and negative values.

  Since the catenoid extends infinitely in both the positive and negative $z$-directions, and at large $\abs{z}$, the surface becomes nearly cylindrical (i.e., the normals approach the horizontal directions), the Gauss map image approaches the equator. Near $z = 0$, the surface flares out, and the normals are more vertical (i.e., closer to the poles).

  Thus, the Gauss map image is symmetric and covers the entire unit sphere minus the north and south poles, just like the hyperboloid. Therefore, the Gauss map image of the catenoid is the entire unit sphere minus the poles.
\end{solution}
