\begin{problem}[2.2.1]
  Show that the cylinder $\{(x, y, z) \in \R^3 \mid x^2 + y^2 = 1\}$ is a
  regular surface, and find parametrizations whose coordinate neighborhoods
  cover it.
\end{problem}

\begin{solution}
  We first show that the cylinder is a regular surface. Let $S = \{(x, y, z) \in
  \R^3 \mid x^2 + y^2 = 1\}$. We parametrize $S$ by
  \[%
    f(x, y, z) = x^2 + y^2 - 1
  .\]%
  The gradient of $f$ is
  \[%
    \nabla f = \langle 2x, 2y, 0 \rangle
  .\]%
  The gradient is non-zero for all points on the cylinder. By proposition 1, the
  graph of a differentiable function is a regular surface. Thus, $S$ is a
  regular surface.

  Now, we find a parametrization whose coordinate neighborhoods cover $S$.
\end{solution}

\begin{problem}[2.2.3]
  Show that the two-sheeted cone, with its vertex at the origin, that is, the
  set $\{(x, y, z) \in \R^3 \mid x^2 + y^2 - z^2 = 0\}$, is not a regular
  surface.
\end{problem}

\begin{solution}
\end{solution}

\begin{problem}[2.2.8]
  Let $\x(u, v)$ be as in Def. 1. Verify that $\dd{\x_q} : \R^2 \to \R^3$ is
  one-to-one if and only if
  \[%
    \pdv{\x}{u} \wedge \pdv{\x}{v} \ne 0
  .\]%
\end{problem}

\begin{solution}
\end{solution}

\begin{problem}[2.2.11]
  Show that the set $S = \{(x, y, z) \in \R^3 \mid z = x^2 - y^2\}$ is a regular
  surface and check that parts (i) and (ii) are parametrizations for $S$:
  \begin{enumerate}
    \item $\x(u, v) = (u + v, u - v, 4uv), (u, v) \in \R^2$.

    \item $\x(u, v) = (u\cosh(v), u\sinh(v), u^2), (u, v) \in \R^2$, $u \ne 0$.
  \end{enumerate}
\end{problem}

\begin{solution}[(i)]
\end{solution}

\begin{solution}[(ii)]
\end{solution}

\begin{problem}[2.2.12]
  Show that $\x : U \subset \R^2 \to \R^2$ given by
  \[%
    \x(u, v) = (a\sin(u)\cos(v), b\sin(u)\sin(v), c\cos(u)), \quad a, b, c \ne 0
  ,\]%
  where $0 < u < \pi$, $0 < v < 2\pi$, is a parametrization for the ellipsoid
  \[%
    \frac{x^2}{a^2} + \frac{y^2}{b^2} + \frac{z^2}{c^2} = 1
  .\]%
  Describe geometrically the curves $u =$ const. on the ellipsoid.
\end{problem}

\begin{solution}
\end{solution}

\begin{problem}[2.2.16]
  One way to define a system of coordinates for the sphere $S^2$, given by $x^2
  + y^2 + (z - 1)^2 = 1$, is to consider the so-called \textit{stenographic
  projection} $\pi : S^2 \sim \{N\} \to \R^2$ which carries a point $p = (x, y,
  z)$ of the sphere $S^2$ minus the north pole $N = (0, 0, 2)$ onto the
  intersection of the $xy$-plane with the straight line which connects $N$ to
  $p$. Let $(u, v) = \pi(x, y, z)$, where $(x, y, z) \in S^2 \sim \{N\}$ and
  $U_9, v) \in xy$-plane.
  \begin{enumerate}
    \item Show that $\pi^{-1} : \R^2 \to S^2$ is given by
      \[%
        \pi^{-1} = \begin{cases}
          x &= \sfrac{4u}{u^2 + v^2 + 4} \\
          y &= \sfrac{4v}{u^2 + v^2 + 4} \\
          z &= \sfrac{2(u^2 + v^2)}{u^2 + v^2 + 4}
        \end{cases}
      .\]%

    \item Show that it is possible, using stereographic projection, to cover the
      sphere with two coordinate neighborhoods.
  \end{enumerate}
\end{problem}

\begin{solution}[(i)]
\end{solution}

\begin{solution}[(ii)]
\end{solution}
