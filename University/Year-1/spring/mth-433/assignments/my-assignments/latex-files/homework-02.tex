\begin{problem}[2.2.1]
  Show that the cylinder $\{(x, y, z) \in \R^3 \mid x^2 + y^2 = 1\}$ is a regular surface, and find parametrizations whose coordinate neighborhoods cover it.
\end{problem}

\begin{solution}
  Let $S = \{(x, y, z) \in \R^3 \mid x^2 + y^2 = 1\}$. We first show that this
  set is a regular surface.

  Define the function $f : \R^3 \to \R$ by $f(x, y, z) = x^2 + y^2 - 1$. Then $S
  = f^{-1}(0)$, and $f$ is differentiable. The gradient is $\nabla f(x, y, z) =
  \langle 2x, 2y, 0 \rangle$. For any point on $S$, we have $x^2 + y^2 = 1$,
  which implies that $x$ and $y$ cannot both be zero. Thus, $\nabla f \ne 0$ on
  all of $S$. Therefore, by proposition 1, $S$ is a regular surface.

  Now, we construct parametrizations for $S$ whose coordinate neighborhoods
  cover it. We split $S$ into two coordinate neighborhoods
  \begin{align*}
    U_+ &= \{(x, y, z) \in \R^3 \mid x^2 + y^2 = 1~\text{and}~z > 0\} \\
    U_- &= \{(x, y, z) \in \R^3 \mid x^2 + y^2 = 1~\text{and}~z < 0\} \\
    U_0 &= \{(x, y, z) \in \R^3 \mid x^2 + y^2 = 1~\text{and}~z = 0\}
  .\end{align*}
  Define the parametrizations
  \begin{align*}
    \Phi_1(\theta, z) &: (-\pi, \pi) \times (0, \infty) \to U_+ \subset \R^3 \\
    \Phi_2(\theta, z) &: (-\pi, \pi) \times (-\infty, 0) \to U_- \subset \R^3 \\
    \Phi_3(\theta, z) &: (-\pi, \pi) \times \{0\} \to U_0 \subset \R^3
  .\end{align*}
  Each $\Phi_i$ is smooth, has injective differential (since the vectors
  $\pd{\theta} \Phi_i = \langle -\sin \theta, \cos \theta, 0 \rangle$ and
  $\pd{z} \Phi_i = \langle 0, 0, 1 \rangle$ are linearly independent), and maps
  into $S$.

  The union of the images of $\Phi_1$, $\Phi_2$, and $\Phi_3$ covers the entire
  cylinder $S$. Therefore, we have constructed parametrizations for $S$ whose
  coordinate neighborhoods cover it.
\end{solution}

\begin{problem}[2.2.3]
  Show that the two-sheeted cone, with its vertex at the origin, that is, the
  set $\{(x, y, z) \in \R^3 \mid x^2 + y^2 - z^2 = 0\}$, is not a regular
  surface.
\end{problem}

\begin{solution}
  The two-sheeted cone is not a regular surface because it does not satisfy the
  regularity condition. The set $S = \{(x, y, z) \in \R^3 \mid x^2 + y^2 - z^2 =
  0\}$ is not a regular surface because the gradient of the function $f(x, y, z)
  = x^2 + y^2 - z^2$ is zero at the origin $(0, 0, 0)$, which is a point in $S$.
  Thus, the cone does not have a well-defined tangent plane at the vertex.
\end{solution}

\begin{problem}[2.2.8]
  Let $\x(u, v)$ be as in Def. 1. Verify that $\dd{\x_q} : \R^2 \to \R^3$ is
  one-to-one if and only if
  \[%
    \pdv{\x}{u} \wedge \pdv{\x}{v} \ne 0
  .\]%
\end{problem}

\begin{solution}
  Assume that $\dd{\x_q} : \R^2 \to \R^3$ is one-to-one, where $\dd{\x_q}$ is given by
  \[%
    \dd{\x_q} = \begin{pmatrix}
      \pdv{x}/{u} & \pdv{x}/{v} \\
      \pdv{y}/{u} & \pdv{y}/{v} \\
      \pdv{z}/{u} & \pdv{z}/{v} \\
    \end{pmatrix}
  .\]%
  This map is one-to-one if and only if the two column vectors $\pdv{\x}/{u}$
  and $\pdv{\x}/{v}$ are linearly independent in $\R^3$. This is equivalent to
  their cross product (or wedge product) being non-zero
  \[%
    \pdv{\x}{u} \wedge \pdv{\x}{v} \ne 0
  .\]%
  Therefore, $\dd{\x_q}$ is one-to-one if and only if $\pdv{\x}/{u} \wedge
  \pdv{\x}/{v} \ne 0$.
\end{solution}

\begin{problem}[2.2.11]
  Show that the set $S = \{(x, y, z) \in \R^3 \mid z = x^2 - y^2\}$ is a regular
  surface and check that parts (i) and (ii) are parametrizations for $S$:
  \begin{enumerate}
    \item $\x(u, v) = \langle u + v, u - v, 4uv \rangle$, $(u, v) \in \R^2$.

    \item $\x(u, v) = \langle u\cosh(v), u\sinh(v), u^2 \rangle$, $(u, v) \in
      \R^2$, $u \ne 0$.
  \end{enumerate}
\end{problem}

\begin{solution}[(i)]
  We first show that the set $S = \{(x, y, z) \in \R^3 \mid z = x^2 - y^2\}$ is
  a regular surface. The function $f(x, y, z) = z - x^2 + y^2$ is a
  differentiable function and $0$ is a regular value of $f$. The gradient of $f$
  is given by
  \[%
    \nabla f(x, y, z) = \langle -2x, 2y, 1 \rangle
  ,\]%
  which is non-zero for all points in $S$ except for the points $(0, 0, z)$,
  where $z \in \R$. Thus, the set $S$ is a regular surface.

  Now, we check that the parametrization $\x(u, v) = \langle u + v, u - v, 4uv
  \rangle$ is a parametrization for $S$. Clearly, $\x(u, v)$ is a smooth
  function from $\R^2 \to \R^3$. Now, we check that the image of $\x$ lies in
  $S$. Let $x = u + v$, $y = u - v$, and $z = 4uv$. Then, we have
  \[%
    x^2 - y^2 = (u + v)^2 - (u - v)^2 = 4uv = z \in S
  .\]%
  Therefore, the image of $\x$ lies in $S$. Next, we compute the partial
  derivatives
  \[%
    \pdv{\x}{u} = \langle 1, 1, 4v \rangle \aand \pdv{\x}{v} = \langle 1, -1, 4u \rangle
  .\]%
  To verify regularity, we check that the partial derivatives $\pdv{\x}/{u}$ and
  $\pdv{\x}/{v}$ are linearly independent. Since their cross product is given by
  \[%
    \pdv{\x}{u} \wedge \pdv{\x}{v} = \langle -8v, 8u, -2 \rangle
  ,\]%
  which is non-zero for all $(u, v) \in \R^2$ (except for $u = 0$ and $v = 0$,
  and since $u = 0$ and $v = 0$ are not in the domain of $\x$), we conclude that
  $\pdv{\x}/{u}$ and $\pdv{\x}/{v}$ are linearly independent. Thus, $\x$ is a
  regular parametrization whose image lies in $S$, so it parametrizes (part of)
  the surface $S$.
\end{solution}

\begin{solution}[(ii)]
  Again, clearly, $\x(u, v) = \langle u\cosh(v), u\sinh(v), u^2 \rangle$ is a
  smooth function from $\R^2 \to \R^3$. Now, we check that the image of $\x$
  lies in $S$. Let $x = u\cosh(v)$, $y = u\sinh(v)$, and $z = u^2$. Then, we
  have
  \[%
    x^2 - y^2 = u^2\cosh^2(v) - u^2\sinh^2(v) = u^2(\cosh^2(v) - \sinh^2(v)) = u^2 = z \in S
  .\]%
  Therefore, the image of $\x$ lies in $S$. Next, we compute the partial
  derivatives
  \[%
    \pdv{\x}{u} = \langle \cosh(v), \sinh(v), 2u \rangle \aand \pdv{\x}{v} = \langle u\sinh(v), u\cosh(v), 0 \rangle
  .\]%
  Again, we take the cross product
  \[%
    \pdv{\x}{u} \wedge \pdv{\x}{v} = \langle 2u\cosh(v), 2u\sinh(v), u^2 \rangle
  .\]%
  Since $u \ne 0$, the cross product is non-zero for all $(u, v) \in \R^2$,
  meaning that $\pdv{\x}/{u}$ and $\pdv{\x}/{v}$ are linearly independent. Thus,
  $\x$ is a regular parametrization whose image lies in $S$, so it parametrizes
  (part of) the surface $S$.
\end{solution}

\begin{problem}[2.2.12]
  Show that $\x : U \subset \R^2 \to \R^2$ given by
  \[%
    \x(u, v) = \langle a\sin(u)\cos(v), b\sin(u)\sin(v), c\cos(u) \rangle, \quad a, b, c \ne 0
  ,\]%
  where $0 < u < \pi$, $0 < v < 2\pi$, is a parametrization for the ellipsoid
  \[%
    \frac{x^2}{a^2} + \frac{y^2}{b^2} + \frac{z^2}{c^2} = 1
  .\]%
  Describe geometrically the curves $u = \text{const}$ on the ellipsoid.
\end{problem}

\begin{solution}
  Clearly, $\x(u, v) = \langle a\sin(u)\cos(v), b\sin(u)\sin(v), c\cos(u)
  \rangle$ is a smooth function from $U \subset \R^2$ to $\R^3$. Now, we check
  that the image of $\x$ lies in the ellipsoid. Let $x = a\sin(u)\cos(v)$, $y =
  b\sin(u)\sin(v)$, and $z = c\cos(u)$. Then, we have
  \begin{align*}
    \frac{x^2}{a^2} + \frac{y^2}{b^2} + \frac{z^2}{c^2} &= \frac{a^2\sin^2(u)\cos^2(v)}{a^2} + \frac{b^2\sin^2(u)\sin^2(v)}{b^2} + \frac{c^2\cos^2(u)}{c^2} \\
                                                        &= \sin^2(u)\cos^2(v) + \sin^2(u)\sin^2(v) + \cos^2(u) \\
                                                        &= \sin^2(u)(\cos^2(v) + \sin^2(v)) + \cos^2(u) \\
                                                        &= \sin^2(u) + \cos^2(u) \\
                                                        &= 1
  .\end{align*}
  Therefore, the image of $\x$ lies on the ellipsoid.

  Lastly, we check regularity. The partial derivatives are given by
  \[%
    \pdv{\x}{u} = \langle a\cos(u)\cos(v), b\cos(u)\sin(v), -c\sin(u) \rangle \aand \pdv{\x}{v} = \langle -a\sin(u)\sin(v), b\sin(u)\cos(v), 0 \rangle
  .\]%
  The cross product is given by
  \[%
    \pdv{\x}{u} \wedge \pdv{\x}{v} = \langle -bc\sin(u)\cos(v), ac\sin(u)\sin(v), ab\sin^2(u) \rangle
  ,\]%
  which is non-zero for all $(u, v) \in U$ (except for $u = 0$ and $u = \pi$,
  and since $0 < u < \pi$, we have $u \ne 0$ and $u \ne \pi$). Thus,
  $\pdv{\x}/{u}$ and $\pdv{\x}/{v}$ are linearly independent, so $\x$ is a
  regular parametrization whose image lies on the ellipsoid, so it parametrizes
  (part of) the surface of the ellipsoid.

  Geometrically, the curves $u = \text{const}$ on the ellipsoid are circles of
  latitude, which are the circles obtained by fixing the angle $u$ and varying
  the angle $v$. These circles lie in planes parallel to the $xy$-plane and are
  centered at the $z$-axis. The radius of these circles depends on the value of
  $u$, with the largest circle corresponding to $u = \sfrac{\pi}{2}$ (the
  equator) and the smallest circle corresponding to $u = 0$ or $u = \pi$ (the
  poles).
\end{solution}

\begin{problem}[2.2.16]
  One way to define a system of coordinates for the sphere $S^2$, given by $x^2
  + y^2 + (z - 1)^2 = 1$, is to consider the so-called \textit{stenographic
  projection} $\pi : S^2 \setminus \{N\} \to \R^2$ which carries a point $p =
  (x, y, z)$ of the sphere $S^2$ minus the north pole $N = (0, 0, 2)$ onto the
  intersection of the $xy$-plane with the straight line which connects $N$ to
  $p$. Let $(u, v) = \pi(x, y, z)$, where $(x, y, z) \in S^2 \setminus \{N\}$
  and $(u, v) \in xy$-plane.
  \begin{enumerate}
    \item Show that $\pi^{-1} : \R^2 \to S^2$ is given by
      \[%
        \pi^{-1} = \begin{cases}
          x = \dfrac{4u}{u^2 + v^2 + 4}\vspace{0.9em} \\
          y = \dfrac{4v}{u^2 + v^2 + 4}\vspace{0.9em} \\
          z = \dfrac{2(u^2 + v^2)}{u^2 + v^2 + 4}.
        \end{cases}
      \]%

    \item Show that it is possible, using stereographic projection, to cover the
      sphere with two coordinate neighborhoods.
  \end{enumerate}
\end{problem}

\begin{solution}[(i)]
  To derive the formula for $\pi^{-1} : \R^2 \to S^2$, consider the line
  from the north pole $N = (0, 0, 2)$ to a point $(u, v, 0)$ on the $xy$-plane.
  A point $(x, y, z)$ on this line can be written as
  \[%
    (x, y, z) = (tu, tv, 2 - 2t)
  ,\]%
  for some parameter $t \in (0, 1)$. We now find $t$ such that this point lies
  on the sphere $S^2$, defined by
  \[%
    x^2 + y^2 + (z - 1)^2 = 1
  .\]%
  Substituting the parameterized coordinates
  \begin{align*}
    (tu)^2 + (tv)^2 + (2 - 2t - 1)^2 &= 1 \\
    t^2(u^2 + v^2) + (1 - 2t)^2 &= 1 \\
    t^2(u^2 + v^2 + 4) - 4t + 1 &= 1 \\
    t^2(u^2 + v^2 + 4) - 4t &= 0 \\
    t(t(u^2 + v^2 + 4) - 4) &= 0
  .\end{align*}
  Discarding the solution $t = 0$ (which gives the north pole), we get
  \[%
    t = \frac{4}{u^2 + v^2 + 4}
  .\]%
  Now substitute back into the parameterized line
  \[%
    x = tu = \frac{4u}{u^2 + v^2 + 4}, \quad y = tv = \frac{4v}{u^2 + v^2 + 4}, \aand z = 2 - 2t = \frac{2(u^2 + v^2)}{u^2 + v^2 + 4}
  .\]%
  Therefore, the inverse stereographic projection is
  \[%
    \pi^{-1}(u, v) = \left( \frac{4u}{u^2 + v^2 + 4}, \frac{4v}{u^2 + v^2 + 4}, \frac{2(u^2 + v^2)}{u^2 + v^2 + 4} \right)
  .\qedhere\]%
\end{solution}

\begin{solution}[(ii)]
  The stereographic projection from the \textit{north pole} $N = (0, 0, 2)$
  defines a smooth bijection
  \[%
    \pi_N : S^2 \setminus \{N\} \to \R^2
  ,\]%
  by projecting each point on the sphere (except $N$) onto the $xy$-plane along
  the line connecting that point to $N$. This provides a coordinate chart
  covering all of $S^2$ except the north pole.

  Similarly, we can define a second stereographic projection from the
  \textit{south pole} $S = (0, 0, 0)$
  \[%
    \pi_S : S^2 \setminus \{S\} \to \R^2
  ,\]%
  which maps all of the sphere except the south pole onto the $xy$-plane. The
  union of the domains of these two maps is
  \[%
    (S^2 \setminus \{N\}) \cup (S^2 \setminus \{S\}) = S^2
  ,\]%
  so together, $\pi_N$ and $\pi_S$ form two coordinate neighborhoods whose union
  covers the entire sphere.

  Therefore, it is possible to cover $S^2$ with two coordinate neighborhoods
  using stereographic projection.
\end{solution}
