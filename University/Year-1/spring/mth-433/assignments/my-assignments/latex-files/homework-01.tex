\begin{problem}[1.2.2]
  Let $\alpha(t)$ be a parametrized curve which does not pass through the
  origin. If $\alpha(t_0)$ is a point of the trace of $\alpha$ closest to the
  origin and $\alpha'(t_0) \ne 0$, show that the position vector $\alpha(t_0)$
  is orthogonal to $\alpha'(t_0)$.
\end{problem}

\begin{proof}[Solution]
  Let $\alpha : I \to \R$ be a parametrized curve, for some interval $I$. Let
  $f(t) = \lVert \alpha(t) \rVert = \alpha(t) \cdot \alpha(t)$. The derivative
  is given by
  \[%
    f'(t) = \odv{}{t} [\alpha(t) \cdot \alpha(t)] = 2 \alpha(t) \cdot \alpha'(t)
  .\]%
  Since $t_0 \in I$ is a global minimum, we have
  \[%
    f'(t_0) = 0 \implies \alpha(t_0) \cdot \alpha'(t_0) = 0
  .\]%
  Since $\alpha(t_0) \ne 0 \ne \alpha'(t_0)$, we have that $\alpha(t_0)$ is
  orthogonal to $\alpha'(t_0)$.
\end{proof}

\begin{problem}[1.2.4]
  Let $\alpha : I \to \R^3$ be a parametrized curve and let $\v \in \R^3$ be a
  fixed vector. Assume that $\alpha'(t)$ is orthogonal to $\v$ for all $t \in I$
  and that $\alpha(0)$ is also orthogonal to $\v$. Prove that $\alpha(t)$ is
  orthogonal to $\v$ for all $t \in I$.
\end{problem}

\begin{proof}[Solution]
  Let $f(t) = \alpha(t) \cdot \v$. Then
  \[%
    f'(t) = \odv{}{t} [\alpha(t) \cdot \v] = \alpha'(t) \cdot \v
  .\]%
  Since $\alpha'(t)$ is orthogonal to $\v$, we have that $f'(t) = 0$ for all $t
  \in I$. Thus, $f(t)$ is constant. Since $f(0) = \alpha(0) \cdot v = 0$, we
  have that $f(t) = 0$ for all $t \in I$. Thus, $\alpha(t) \cdot \v = 0$ for all
  $t \in I$.
\end{proof}

\begin{problem}[1.3.1]
  Show that the tangent lines to the regular parametrized curve $\alpha(t) =
  \left\langle 3t, 3t^2, 2t^3 \right\rangle$ make a constant angle with the line
  $y = 0$, $z = x$.
\end{problem}

\begin{proof}[Solution]
  The tangent line to the curve $\alpha(t)$ is given by
  \[%
    \alpha'(t) = \left\langle 3, 6t, 6t^2 \right\rangle
  .\]%
  The line $y = 0$, $z = x$ is given by $\beta(s) = \langle s, 0, s \rangle$.
  The direction vector of $\beta(s)$ is given by $\langle 1, 0, 1 \rangle$. The
  angle between the two lines is given by
  \[%
    \cos(\theta) = \frac{\u \cdot \v}{\lVert \u \rVert \lVert \v \rVert} = \frac{\alpha'(t) \cdot \langle 1, 0, 1 \rangle}{\lVert \alpha'(t) \rVert \lVert \langle 1, 0, 1 \rangle \rVert} = \frac{3 + 6t^2}{\sqrt{9 + 36t^2 + 36t^4} \cdot \sqrt{2}} = \frac{1}{\sqrt{2}} = \frac{\sqrt{2}}{2}
  .\]%
  Thus, the angle between the two lines is constant.
\end{proof}

\begin{problem}[1.3.4]
  Let $\alpha : (0, \pi) \to \R^2$ be given by
  \[%
    \alpha(t) = \left\langle \sin(t), \cos(t) + \log\left(\tan\left(\frac{t}{2}\right)\right) \right\rangle
  ,\]%
  where $t$ is the angle that the $y$-axis makes with the vector $\alpha'(t)$.
  The trace of $\alpha$ is called the tractrix. Show that
  \begin{enumerate}
    \item $\alpha$ is a differentiable parametrized curve, regular except at $t
      = \sfrac{\pi}{2}$.

    \item The length of the segment of the tangent of the tractrix between the
      point of tangency and the $y$-axis is constantly equal to $1$.
  \end{enumerate}
\end{problem}

\begin{proof}[Solution to (i)]
  At $t = \sfrac{\pi}{2}$, we get
  \[%
    \alpha(\sfrac{\pi}{2}) = \langle 1, 1 \rangle
  .\]%
  Differentiating $\alpha(t)$, we obtain
  \[%
    \alpha'(t) = \left\langle \cos(t), -\sin(t) + \frac{1}{2}\csc\left(\frac{t}{2}\right)\sec^2\left(\frac{t}{2}\right) \right\rangle
  .\]%
  Evaluating at $t = \sfrac{\pi}{2}$, we find
  \[%
    \alpha'(\sfrac{\pi}{2}) = \langle 0, 0 \rangle
  .\]%
  Thus, $\alpha$ is not regular at $t = \sfrac{\pi}{2}$.

  Let $I = (0, \pi) \setminus \{\sfrac{\pi}{2}\}$. For $t \in I$, we note that
  \[%
    \lvert \cos(t) \rvert > 0
  ,\]%
  since $t \neq \sfrac{\pi}{2}$. Therefore, $\alpha'(t) \neq 0$ for all $t \in
  I$, meaning that $\alpha$ is regular on $I$.
\end{proof}

\begin{proof}[Solution to (ii)]
  From part (i), the slope of the tangent line is given by
  \[%
    m = \frac{y'(t)}{x'(t)} = \frac{-\sin(t) + \frac{1}{2} \csc\left(\frac{t}{2}\right) \sec^2\left(\frac{t}{2}\right)}{\cos(t)}
  .\]%
  The equation of the tangent line at $\alpha(t) = (x(t), y(t))$ is
  \[%
    y = m(x - x(t)) + y(t)
  .\]%
  To find the $y$-intercept, we set $x = 0$ and obtain
  \[%
    y = -m x(t) + y(t)
  .\]%

  The distance between the point of tangency $(x(t), y(t))$ and the $y$-axis at
  $(0, -m x(t) + y(t))$ is
  \begin{align*}
    d &= \sqrt{(x(t) - 0)^2 + (y(t) - (-m x(t) + y(t)))^2} \\
      &= \sqrt{x^2(t) + m^2 x^2(t)} \\
      &= x(t) \sqrt{1 + m^2}.
  \end{align*}
  Substituting $x(t) = \sin(t)$, we get
  \[%
    d = \sin(t) \sqrt{1 + \left( \frac{-\sin(t) + \frac{1}{2} \csc\left(\frac{t}{2}\right) \sec^2\left(\frac{t}{2}\right)}{\cos(t)} \right)^2}
  .\]%
  Simplifying further, we obtain
  \[%
    d = 1
  ,\]%
  proving that the segment of the tangent line between the point of tangency and
  the $y$-axis is always of unit length.
\end{proof}

\begin{problem}[1.3.6]
  Let $\alpha(t) = \left\langle ae^{bt}\cos(t), ae^{bt}\sin(t) \right\rangle$,
  $t \in \R$, $a$ and $b$ are constants, $a > 0$, $b < 0$, be parametrized
  curve.
  \begin{enumerate}
    \item Show that as $t \to +\infty$, $\alpha(t)$ approaches the origin $0$,
      spiraling around it (because of this, the trace of $\alpha$ is called the
      \textit{logarithmic spiral}).

    \item Show that $\alpha'(t) \to (0, 0)$ as $t \to +\infty$ and that
      \[%
        \lim_{t \to +\infty} \int_{t_0}^t \lVert \alpha'(x) \rVert \dx
      ,\]%
      is finite; that is, $\alpha$ has finite arc length in $[t_0, \infty)$.
  \end{enumerate}
\end{problem}

\begin{proof}[Solution to (i)]
  Since $b < 0$, we have $e^{bt} \to 0$ as $t \to +\infty$. Thus, we obtain
  \[%
    \alpha(t) = \left\langle ae^{bt}\cos(t), ae^{bt}\sin(t) \right\rangle \to (0,0) \qtq{as} t \to +\infty
  .\]%
  The presence of the $\cos(t)$ and $\sin(t)$ terms indicates that the
  trajectory of $\alpha(t)$ traces a spiral as it approaches the origin.
\end{proof}

\begin{proof}[Solution to (ii)]
  We compute
  \begin{align*}
    \alpha'(t) &= \left\langle \frac{d}{dt} \big( ae^{bt} \cos t \big), \frac{d}{dt} \big( ae^{bt} \sin t \big) \right\rangle \\
               &= \left\langle a\big( be^{bt} \cos t - e^{bt} \sin t \big), a\big( be^{bt} \sin t + e^{bt} \cos t \big) \right\rangle
  .\end{align*}
  Again, since $b < 0$, we see that $e^{bt} \to 0$ as $t \to +\infty$, and so
  \[%
    \alpha'(t) \to (0, 0) \qtq{as} t \to +\infty
  .\]%

  Now, we compute the arc length
  \begin{align*}
    \int_{t_0}^t \lVert \alpha'(x) \rVert \dx &= \int_{t_0}^t \sqrt{(abe^{bx} \cos x - ae^{bx} \sin x)^2 + (abe^{bx} \sin x + ae^{bx} \cos x)^2} \dx \\
                                              &= \int_{t_0}^t ae^{bx} \sqrt{b^2 + 1} \dx \\
                                              &= a\sqrt{b^2 + 1} \int_{t_0}^t e^{bx} \dx
  .\end{align*}
  Evaluating the integral, we obtain
  \[%
    a\sqrt{b^2 + 1} \frac{e^{bx}}{b} \bigg|_{t_0}^t = a\sqrt{b^2 + 1} \left(\frac{e^{bt}}{b} - \frac{e^{bt_0}}{b}\right).
  \]%
  Taking the limit as $t \to +\infty$, we find
  \[%
    \lim_{t \to +\infty} \int_{t_0}^t \lVert \alpha'(x) \rVert \dx = a\sqrt{b^2 + 1} \left( 0 - \frac{e^{bt_0}}{b} \right) = -\frac{a\sqrt{b^2 + 1}}{b}e^{bt_0} < \infty
  .\]%
  Thus, $\alpha$ has finite arc length in $[t_0, \infty)$.
\end{proof}

\begin{problem}[1.3.10]
  Let $\alpha : I \to \R^3$ be a parametrized curve. Let $[a, b] \subset I$ and
  set $\alpha(a) = \p$, $\alpha(b) = \q$.
  \begin{enumerate}
    \item Show that, for any constant vector $\v$, $\lVert \v \rVert = 1$,
      \[%
        (\q - \p) \cdot \v = \int_a^b \alpha'(t) \cdot \v \dt \le \int_a^b \lVert \alpha'(t) \rVert \dt
      .\]%

    \item Set
      \[%
        \v = \frac{\q - \p}{\lVert \q - \p \rVert}
      ,\]%
      and show that
      \[%
        \lVert \alpha(b) - \alpha(a) \rVert \le \int_a^b \lVert \alpha'(t) \rVert \dt
      ;\]%
      that is, the curve of shortest length from $\alpha(a)$ to $\alpha(b)$ is
      the straight line joining these points.
  \end{enumerate}
\end{problem}

\begin{proof}[Solution to (i)]
  We first show the equality on the left-hand side. Since $\v$ is a constant
  vector, we can factor it out of the integral. Thus, we have
  \[%
    \int_a^b \alpha'(t) \cdot \v \dt = \int_a^b \alpha'(t) \dt \cdot \v = (\alpha(b) - \alpha(a)) \cdot \v = (\q - \p) \cdot \v
  .\]%
  Thus, we have shown the equality.

  Now, we can show the inequality.Using the Cauchy-Schwartz inequality, we know
  that for any vectors $a$ and $b$, $a \cdot b \le \lvert a \rvert \lvert b
  \rvert$. Applying this to $\alpha'(t)$ and $v$, we have $\alpha'(t) \cdot v
  \le \lVert \alpha'(t) \rVert$. Therefore, we have
  \[%
    \int_a^b \alpha'(t) \cdot \v \dt \le \int_a^b \lVert \alpha'(t) \rVert \dt
  .\]%
  Thus, we have shown the inequality.
\end{proof}

\begin{proof}[Solution to (ii)]
  Computing the original integral with the new value of $\v$, we have
  \[%
    \int_a^b \alpha'(t) \cdot \v \dt = \int_a^b \alpha'(t) \dt \cdot \left(\frac{\q - \p}{\lVert \q - \p \rVert}\right) = (\q - \p) \cdot \frac{\q - \p}{\lVert \q - p \rVert}
  .\]%
  Since $(\q - \p)(\q - \p) = \lVert \q - \p \rVert^2$, we get
  \[%
    \int_a^b \alpha'(t) \dt \cdot \v = \lVert \q - \p \rVert
  .\]%
  Since we've already established the inequality in part (i), we have
  \[%
    \int_a^b \alpha'(t) \cdot \frac{\q - \p}{\lVert \q - \p \rVert} \dt = \lVert \q - \p \rVert = \lVert \alpha(b) - \alpha(a) \rVert \le \int_a^b \lVert \alpha'(t) \rVert \dt
  .\]%
  Thus, we have shown the inequality.
\end{proof}

\begin{problem}[1.4.2]
  A plane $P$ contained in $\R^3$ is given by the equation $ax + by + cz + d =
  0$. Show that the vector $v = \langle a, b, c \rangle$ is perpendicular to the
  plane and that $\sfrac{\lvert d \rvert}{\sqrt{a^2 + b^2 + c^2}}$ measures the
  distance from the plane to the origin $(0, 0, 0)$.
\end{problem}

\begin{proof}[Solution]
  Essentially, this question is asking to show that the normal vector to the
  plane is the vector $\v = \langle a, b, c \rangle$.

  Let $P_1$ and $P_2$ be points on the plane $P$. Then, we have $\p =
  \overline{P_1P_2} = \langle x_2 - x_1, y_2 - y_1, z_2 - z_1 \rangle$. Since
  they are both on the plane, we have
  \begin{align*}
    ax_1 + by_1 + cz_1 + d &= 0 \\
    ax_2 + by_2 + cz_2 + d &= 0
  .\end{align*}
  Subtracting the two equations, we have
  \[%
    a(x_2 - x_1) + b(y_2 - y_1) + c(z_2 - z_1) = 0
  .\]%
  Thus, we have
  \[%
    a(x_2 - x_1) + b(y_2 - y_1) + c(z_2 - z_1) = 0 \implies \v \cdot \p = 0
  .\]%
  Thus, we have shown that $\v$ is perpendicular to the plane.

  Now, we can show that the distance from the plane to the origin is given by
  \[%
    D = \frac{\lvert d \rvert}{\sqrt{a^2 + b^2 + c^2}}
  .\]%
\end{proof}

\begin{problem}[1.4.10]
  The natural orientation of $\R^2$ makes it possible to associate a sign to the
  area $A$ of a parallelogram generated by two linearly independent vectors $u,
  v \in \R^2$. To do this, let $\{\e_i\}$, $i = 1, 2$, be the natural ordered
  basis of $\R^2$, and write $u = u_1\e_1 + u_2\e_2$, $v = v_1\e_1 + v_2\e_2$.
  Observe the matrix relation
  \[%
    \begin{pmatrix}
      u \cdot u & u \cdot v \\
      v \cdot u & v \cdot v
    \end{pmatrix}
    = \begin{pmatrix}
      u_1 & u_2 \\
      v_1 & v_2
    \end{pmatrix}
    \begin{pmatrix}
      u_1 & v_1 \\
      u_2 & v_2
    \end{pmatrix}
  ,\]%
  and conclude that
  \[%
    A^2 = \begin{vmatrix}
      u_1 & u_2 \\
      v_1 & v_2
    \end{vmatrix}^2
  .\]%
  Since the last determinant has the same sign as the basis $\{u, v\}$, we can
  say that $A$ is positive or negative according to whether the orientation of
  $\{u, v\}$ is positive or negative. This is called the \textit{orientated
  area} in $\R^2$.
\end{problem}

\begin{proof}[Solution]
\end{proof}

\begin{problem}[1.4.11]
  \begin{enumerate}
    \item Show that the volume $V$ of a parallelepiped generated by three
      linearly independent vectors $\u, \v, \w \in \R^3$ is given by $V = \lvert
      (\u \wedge \v) \cdot \w \rvert$, and introduce an \textit{orientated
      volume} in $\R^3$.

    \item Prove that
      \[%
        V^2 = \begin{vmatrix}
          \u \cdot \u & \u \cdot \v & \u \cdot \w \\
          \v \cdot \u & \v \cdot \v & \v \cdot \w \\
          \w \cdot \u & \w \cdot \v & \w \cdot \w \\
        \end{vmatrix}
      .\]%
  \end{enumerate}
\end{problem}

\begin{proof}[Solution to (i)]
  The volume of a parallelepiped is given by $V = \textrm{Base} \times
  \textrm{Height}$. The base is given by the area of the parallelogram formed by
  the vectors $\u$ and $\v$, which is given by $\lvert \u \wedge \v \rvert$. The
  height is given by the component of $\w$ in the direction of the normal vector
  $\u \wedge \v$, which is given by
  \[%
    h = \frac{\lvert (\u \wedge \v) \cdot \w \rvert}{\lVert \u \wedge \v \rVert}
  .\]%
  Therefore, we have
  \[%
    V = \textrm{Base} \times \textrm{Height} = \lVert \u \wedge \v \rVert \cdot \frac{\lvert (\u \wedge \v) \cdot \w \rvert}{\lVert \u \wedge \v \rVert} = \lvert (\u \wedge \v) \cdot \w \rvert
  .\]%

  The oriented volume $V_{\text{oriented}}$ carries a sign that depends on
  whether $(\u, \v, \w)$ form a right-handed or left-handed basis. If $(\u, \v,
  \w)$ follows the right-hand rule, then $V_{\text{oriented}}$ is positive.
  Otherwise, it is negative.
\end{proof}

\begin{proof}[Solution to (ii)]
  Let $A = [\u~\v~\w]$. Then, we have
  \[%
    \det(A)\det(A) = \det(A^2) = \det(A^TA) = \det(A^T) \cdot \det(A)
    = \det\begin{pmatrix}
      \u \\
      \v \\
      \w \\
    \end{pmatrix}
    \cdot \det\begin{pmatrix}
      \u & \v & \w \\
    \end{pmatrix}
  .\]%
  This gives us
  \[%
    \det(A^TA) = \begin{vmatrix}
      \u \cdot \u & \u \cdot \v & \u \cdot \w \\
      \v \cdot \u & \v \cdot \v & \v \cdot \w \\
      \w \cdot \u & \w \cdot \v & \w \cdot \w \\
    \end{vmatrix}
  .\]%
  Notice that $\det(A^TA) = \lvert (\u \wedge \v) \cdot \w \rvert^2 = V^2$.
  Thus, we've shown the desired result.
\end{proof}

\begin{problem}[1.5.1]
  Given the parametrized curve (helix)
  \[%
    \alpha(s) = \left\langle a\cos\left(\frac{s}{c}\right), a\sin\left(\frac{s}{c}\right), b\frac{s}{c} \right\rangle, \quad s \in \R
  ,\]%
  where $c^2 = a^2 + b^2$.
  \begin{enumerate}
    \item Show that the parameter $s$ is the arc length.

    \item Determine the curvature and the torsion of $\alpha$.

    \item Determine the osculating plane of $\alpha$.

    \item Show that the lines containing $n(s)$ and passing through $\alpha(s)$
      meet the $z$-axis under a constant angle equal to $\sfrac{\pi}{2}$.

    \item Show that the tangent lines to $\alpha$ make a constant angle with the
      $z$-axis.
  \end{enumerate}
\end{problem}

\begin{proof}[Solution to (i)]
  If $\alpha(s)$ is parametrized by arc length, then the magnitude of the
  derivative of $\alpha(s)$ must be equal to $1$. We compute
  \[%
    \alpha'(s) = \left\langle -\frac{a}{c}\sin\left(\frac{s}{c}\right), \frac{a}{c}\cos\left(\frac{s}{c}\right), \frac{b}{c} \right\rangle
  .\]%
  The magnitude of the derivative is given by
  \begin{align*}
    \lvert \alpha'(s) \rvert &= \sqrt{\frac{a^2}{c^2}\sin^2\left(\frac{s}{c}\right) + \frac{a^2}{c^2}\cos^2\left(\frac{s}{c}\right) + \frac{b^2}{c^2}} \\
                             &= \sqrt{\frac{a^2}{c^2} + \frac{b^2}{c^2}} = \sqrt{1} = 1
  .\end{align*}
  Therefore, we have shown that $s$ is the arc length.
\end{proof}

\begin{proof}[Solution to (ii)]
  The curvature is given by $\kappa(s) = \lvert \alpha''(s) \rvert$.
\end{proof}

\begin{proof}[Solution to (iii)]
\end{proof}

\begin{proof}[Solution to (iv)]
\end{proof}

\begin{proof}[Solution to (v)]
\end{proof}

\begin{problem}[1.5.11]
  One often gives a plane curve in polar coordinates by $\rho =\rho(\theta)$,
  $a \le \theta \le b$.
  \begin{enumerate}
    \item Show that the arc length is
      \[%
        \int_a^b \sqrt{\rho^2 + (\rho')^2} \dd{\theta}
      ,\]%
      where the prime denotes the derivative relative to $\theta$.

    \item Show that the curvature is
      \[%
        \kappa(\theta) = \frac{2(\rho')^2 - \rho\rho'' + \rho^2}{[(\rho')^2 + \rho^2]^{\sfrac{3}{2}}}
      .\]%
  \end{enumerate}
\end{problem}

\begin{proof}[Solution to (i)]
\end{proof}

\begin{proof}[Solution to (ii)]
\end{proof}

\begin{problem}[1.5.12]
  Let $\alpha : I \to \R^3$ be a regular parametrized curve (not necessarily by
  arc length), and let $\beta : J \to \R^3$ be a reparameterization of
  $\alpha(I)$ by the arc length $s = s(t)$, measured from $t_0 \in I$ (see
  Remark 2). Let $t = t(s)$ be the inverse function of $s$ and set
  $\odv{\alpha}/{t} = \alpha'$, $\odv[order={2}]{\alpha}/{t} = \alpha''$, etc.
  Prove that
  \begin{enumerate}
    \item $\odv{t}/{s} = \sfrac{1}{\lVert \alpha' \rVert}$,
      $\odv[order={2}]{t}/{s} = -(\sfrac{\alpha' \cdot \alpha''}{\lVert \alpha'
      \rVert^4})$.

    \item The curvature of $\alpha$ at $t \in I$ is
      \[%
        \kappa(t) = \frac{\lVert \alpha' \cdot \alpha'' \rVert}{\lVert \alpha' \rVert^3}
      .\]%

    \item The torsion of $\alpha$ at $t \in I$ is
      \[%
        \tau(t) = -\frac{(\alpha' \cdot \alpha'') \cdot \alpha'''}{\lVert \alpha' \cdot \alpha'' \rVert^2}
      .\]%

    \item If $\alpha : I \to \R^2$ is a plane curve $\alpha(t) = \langle x(t),
      y(t) \rangle$, the signed curvature (see Remark 1) of $\alpha$ at $t$ is
      \[%
        \kappa(t) = \frac{x' y'' - x'' y'}{[(x')^2 + (y')^2]^{\sfrac{3}{2}}}
      .\]%
  \end{enumerate}
\end{problem}

\begin{proof}[Solution to (i)]
\end{proof}

\begin{proof}[Solution to (ii)]
\end{proof}

\begin{proof}[Solution to (iii)]
  The osculating plane of $\alpha(s)$ is defined to be the plane spanned by
  $\alpha'(s)$ and $\alpha''(s)$.
\end{proof}

\begin{proof}[Solution to (iv)]
\end{proof}

\begin{problem}[1.5.14]
  Let $\alpha : (a, b) \to \R^2$ be a regular parametrized plane curve. Assume
  that there exists $t_0$, $a < t_0 < b$, such that the distance $\lVert
  \alpha(t) \rVert$ from the origin to the trace of $\alpha$ will be a maximum
  at $t_0$. Prove that the curvature $\kappa$ of $\alpha$ at $t_0$ satisfies
  $\lvert \kappa(t_0) \rvert \ge \sfrac{1}{\lVert \alpha(t_0) \rVert}$.
\end{problem}

\begin{proof}[Solution]
\end{proof}
