\begin{problem}[1.2.2]
  Let $\alpha(t)$ be a parametrized curve which does not pass through the
  origin. If $\alpha(t_0)$ is a point of the trace of $\alpha$ closest to the
  origin and $\alpha'(t_0) \ne 0$, show that the position vector $\alpha(t_0)$
  is orthogonal to $\alpha'(t_0)$.
\end{problem}

\begin{solution}
  Let $\alpha : I \to \R$ be a parametrized curve, for some interval $I$. Let
  $f(t) = \lVert \alpha(t) \rVert = \alpha(t) \cdot \alpha(t)$. The derivative
  is given by
  \[%
    f'(t) = \odv{}{t} [\alpha(t) \cdot \alpha(t)] = 2 \alpha(t) \cdot \alpha'(t)
  .\]%
  Since $t_0 \in I$ is a global minimum, we have
  \[%
    f'(t_0) = 0 \implies \alpha(t_0) \cdot \alpha'(t_0) = 0
  .\]%
  Since $\alpha(t_0) \ne 0 \ne \alpha'(t_0)$, we have that $\alpha(t_0)$ is
  orthogonal to $\alpha'(t_0)$.
\end{solution}

\begin{problem}[1.2.4]
  Let $\alpha : I \to \R^3$ be a parametrized curve and let $\v \in \R^3$ be a
  fixed vector. Assume that $\alpha'(t)$ is orthogonal to $\v$ for all $t \in I$
  and that $\alpha(0)$ is also orthogonal to $\v$. Prove that $\alpha(t)$ is
  orthogonal to $\v$ for all $t \in I$.
\end{problem}

\begin{solution}
  Let $f(t) = \alpha(t) \cdot \v$. Then
  \[%
    f'(t) = \odv{}{t} [\alpha(t) \cdot \v] = \alpha'(t) \cdot \v
  .\]%
  Since $\alpha'(t)$ is orthogonal to $\v$, we have that $f'(t) = 0$ for all $t
  \in I$. Thus, $f(t)$ is constant. Since $f(0) = \alpha(0) \cdot v = 0$, we
  have that $f(t) = 0$ for all $t \in I$. Thus, $\alpha(t) \cdot \v = 0$ for all
  $t \in I$.
\end{solution}

\begin{problem}[1.3.1]
  Show that the tangent lines to the regular parametrized curve $\alpha(t) =
  \left\langle 3t, 3t^2, 2t^3 \right\rangle$ make a constant angle with the line
  $y = 0$, $z = x$.
\end{problem}

\begin{solution}
  The tangent line to the curve $\alpha(t)$ is given by
  \[%
    \alpha'(t) = \left\langle 3, 6t, 6t^2 \right\rangle
  .\]%
  The line $y = 0$, $z = x$ is given by $\beta(s) = \langle s, 0, s \rangle$.
  The direction vector of $\beta(s)$ is given by $\langle 1, 0, 1 \rangle$. The
  angle between the two lines is given by
  \[%
    \cos(\theta) = \frac{\u \cdot \v}{\lVert \u \rVert \lVert \v \rVert} = \frac{\alpha'(t) \cdot \langle 1, 0, 1 \rangle}{\lVert \alpha'(t) \rVert \lVert \langle 1, 0, 1 \rangle \rVert} = \frac{3 + 6t^2}{\sqrt{9 + 36t^2 + 36t^4} \cdot \sqrt{2}} = \frac{1}{\sqrt{2}} = \frac{\sqrt{2}}{2}
  .\]%
  Thus, the angle between the two lines is constant.
\end{solution}

\begin{problem}[1.3.4]
  Let $\alpha : (0, \pi) \to \R^2$ be given by
  \[%
    \alpha(t) = \left\langle \sin(t), \cos(t) + \log\left(\tan\left(\frac{t}{2}\right)\right) \right\rangle
  ,\]%
  where $t$ is the angle that the $y$-axis makes with the vector $\alpha'(t)$. The trace of $\alpha$ is called the tractrix. Show that
  \begin{enumerate}
    \item $\alpha$ is a differentiable parametrized curve, regular except at $t = \sfrac{\pi}{2}$.

    \item The length of the segment of the tangent of the tractrix between the point of tangency and the $y$-axis is constantly equal to $1$.
  \end{enumerate}
\end{problem}

\begin{solution}[(i)]
  At $t = \sfrac{\pi}{2}$, we get
  \[%
    \alpha(\sfrac{\pi}{2}) = \langle 1, 1 \rangle
  .\]%
  Differentiating $\alpha(t)$, we obtain
  \[%
    \alpha'(t) = \left\langle \cos(t), -\sin(t) + \frac{1}{2}\csc\left(\frac{t}{2}\right)\sec^2\left(\frac{t}{2}\right) \right\rangle
  .\]%
  Evaluating at $t = \sfrac{\pi}{2}$, we find
  \[%
    \alpha'(\sfrac{\pi}{2}) = \langle 0, 0 \rangle
  .\]%
  Thus, $\alpha$ is not regular at $t = \sfrac{\pi}{2}$.

  Let $I = (0, \pi) \setminus \{\sfrac{\pi}{2}\}$. For $t \in I$, we note that
  \[%
    \lvert \cos(t) \rvert > 0
  ,\]%
  since $t \neq \sfrac{\pi}{2}$. Therefore, $\alpha'(t) \neq 0$ for all $t \in
  I$, meaning that $\alpha$ is regular on $I$.
\end{solution}

\begin{solution}[(ii)]
  We are told that for each $t \in (0, \pi)$, the angle between the tangent
  vector $\alpha'(t)$ and the $y$-axis is exactly $t$. Let us interpret this
  geometrically. Since $\alpha$ is a regular, differentiable parametrized curve,
  its derivative $\alpha'(t)$ points in the direction of the tangent line to the
  curve at the point $\alpha(t)$. The information that $\alpha'(t)$ makes an
  angle $t$ with the $y$-axis tells us that the tangent line at $\alpha(t)$
  intersects the $y$-axis at an angle of $t$.

  Now consider the segment of the tangent line that connects the point of
  tangency $\alpha(t)$ to its intersection with the $y$-axis. This segment lies
  along the direction of the tangent vector and intersects the $y$-axis, forming
  a right triangle with the horizontal leg extending from the $y$-axis to the
  point $\alpha(t) = \langle x(t), y(t) \rangle$.

  In this right triangle, the hypotenuse is the segment of the tangent line from
  $\alpha(t)$ to the $y$-axis, and the horizontal leg has length $x(t) =
  \sin(t)$. Since the angle between the tangent line and the $y$-axis is $t$,
  the angle between the hypotenuse and the vertical leg (i.e., the $y$-axis) is
  also $t$. By basic trigonometry, we then have
  \[%
    \sin(t) = (\text{hypotenuse}) \cdot \sin(t)
  ,\]%
  which implies that the length of the hypotenuse — that is, the distance from
  the point of tangency to the $y$-axis along the tangent line — is exactly $1$
  \[%
    \text{Length} = \frac{\sin(t)}{\sin(t)} = 1
  .\]%
  Therefore, the length of the segment of the tangent line between the point of
  tangency and the $y$-axis is always equal to $1$, regardless of the value of
  $t$.
\end{solution}

\begin{problem}[1.3.6]
  Let $\alpha(t) = \left\langle ae^{bt}\cos(t), ae^{bt}\sin(t) \right\rangle$,
  $t \in \R$, $a$ and $b$ are constants, $a > 0$, $b < 0$, be parametrized
  curve.
  \begin{enumerate}
    \item Show that as $t \to +\infty$, $\alpha(t)$ approaches the origin $0$,
      spiraling around it (because of this, the trace of $\alpha$ is called the
      \textit{logarithmic spiral}).

    \item Show that $\alpha'(t) \to (0, 0)$ as $t \to +\infty$ and that
      \[%
        \lim_{t \to +\infty} \int_{t_0}^t \lVert \alpha'(x) \rVert \dx
      ,\]%
      is finite; that is, $\alpha$ has finite arc length in $[t_0, \infty)$.
  \end{enumerate}
\end{problem}

\begin{solution}[(i)]
  Since $b < 0$, we have $e^{bt} \to 0$ as $t \to +\infty$. Thus, we obtain
  \[%
    \alpha(t) = \left\langle ae^{bt}\cos(t), ae^{bt}\sin(t) \right\rangle \to (0,0) \qtq{as} t \to +\infty
  .\]%
  The presence of the $\cos(t)$ and $\sin(t)$ terms indicates that the
  trajectory of $\alpha(t)$ traces a spiral as it approaches the origin.
\end{solution}

\begin{solution}[(ii)]
  We compute
  \begin{align*}
    \alpha'(t) &= \left\langle \frac{d}{dt} \big( ae^{bt} \cos t \big), \frac{d}{dt} \big( ae^{bt} \sin t \big) \right\rangle \\
               &= \left\langle a\big( be^{bt} \cos t - e^{bt} \sin t \big), a\big( be^{bt} \sin t + e^{bt} \cos t \big) \right\rangle
  .\end{align*}
  Again, since $b < 0$, we see that $e^{bt} \to 0$ as $t \to +\infty$, and so
  \[%
    \alpha'(t) \to (0, 0) \qtq{as} t \to +\infty
  .\]%

  Now, we compute the arc length
  \begin{align*}
    \int_{t_0}^t \lVert \alpha'(x) \rVert \dx &= \int_{t_0}^t \sqrt{(abe^{bx} \cos x - ae^{bx} \sin x)^2 + (abe^{bx} \sin x + ae^{bx} \cos x)^2} \dx \\
                                              &= \int_{t_0}^t ae^{bx} \sqrt{b^2 + 1} \dx \\
                                              &= a\sqrt{b^2 + 1} \int_{t_0}^t e^{bx} \dx
  .\end{align*}
  Evaluating the integral, we obtain
  \[%
    a\sqrt{b^2 + 1} \frac{e^{bx}}{b} \bigg|_{t_0}^t = a\sqrt{b^2 + 1} \left(\frac{e^{bt}}{b} - \frac{e^{bt_0}}{b}\right).
  \]%
  Taking the limit as $t \to +\infty$, we find
  \[%
    \lim_{t \to +\infty} \int_{t_0}^t \lVert \alpha'(x) \rVert \dx = a\sqrt{b^2 + 1} \left( 0 - \frac{e^{bt_0}}{b} \right) = -\frac{a\sqrt{b^2 + 1}}{b}e^{bt_0} < \infty
  .\]%
  Thus, $\alpha$ has finite arc length in $[t_0, \infty)$.
\end{solution}

\begin{problem}[1.3.10]
  Let $\alpha : I \to \R^3$ be a parametrized curve. Let $[a, b] \subset I$ and
  set $\alpha(a) = \p$, $\alpha(b) = \q$.
  \begin{enumerate}
    \item Show that, for any constant vector $\v$, $\lVert \v \rVert = 1$,
      \[%
        (\q - \p) \cdot \v = \int_a^b \alpha'(t) \cdot \v \dt \le \int_a^b \lVert \alpha'(t) \rVert \dt
      .\]%

    \item Set
      \[%
        \v = \frac{\q - \p}{\lVert \q - \p \rVert}
      ,\]%
      and show that
      \[%
        \lVert \alpha(b) - \alpha(a) \rVert \le \int_a^b \lVert \alpha'(t) \rVert \dt
      ;\]%
      that is, the curve of shortest length from $\alpha(a)$ to $\alpha(b)$ is
      the straight line joining these points.
  \end{enumerate}
\end{problem}

\begin{solution}[(i)]
  We first show the equality on the left-hand side. Since $\v$ is a constant
  vector, we can factor it out of the integral. Thus, we have
  \[%
    \int_a^b \alpha'(t) \cdot \v \dt = \int_a^b \alpha'(t) \dt \cdot \v = (\alpha(b) - \alpha(a)) \cdot \v = (\q - \p) \cdot \v
  .\]%
  Thus, we have shown the equality.

  Now, we can show the inequality.Using the Cauchy-Schwartz inequality, we know
  that for any vectors $a$ and $b$, $a \cdot b \le \lvert a \rvert \lvert b
  \rvert$. Applying this to $\alpha'(t)$ and $v$, we have $\alpha'(t) \cdot v
  \le \lVert \alpha'(t) \rVert$. Therefore, we have
  \[%
    \int_a^b \alpha'(t) \cdot \v \dt \le \int_a^b \lVert \alpha'(t) \rVert \dt
  .\]%
  Thus, we have shown the inequality.
\end{solution}

\begin{solution}[(ii)]
  Computing the original integral with the new value of $\v$, we have
  \[%
    \int_a^b \alpha'(t) \cdot \v \dt = \int_a^b \alpha'(t) \dt \cdot \left(\frac{\q - \p}{\lVert \q - \p \rVert}\right) = (\q - \p) \cdot \frac{\q - \p}{\lVert \q - p \rVert}
  .\]%
  Since $(\q - \p)(\q - \p) = \lVert \q - \p \rVert^2$, we get
  \[%
    \int_a^b \alpha'(t) \dt \cdot \v = \lVert \q - \p \rVert
  .\]%
  Since we've already established the inequality in part (i), we have
  \[%
    \int_a^b \alpha'(t) \cdot \frac{\q - \p}{\lVert \q - \p \rVert} \dt = \lVert \q - \p \rVert = \lVert \alpha(b) - \alpha(a) \rVert \le \int_a^b \lVert \alpha'(t) \rVert \dt
  .\]%
  Thus, we have shown the inequality.
\end{solution}

\begin{problem}[1.4.2]
  A plane $P$ contained in $\R^3$ is given by the equation $ax + by + cz + d =
  0$. Show that the vector $v = \langle a, b, c \rangle$ is perpendicular to the
  plane and that $\sfrac{\lvert d \rvert}{\sqrt{a^2 + b^2 + c^2}}$ measures the
  distance from the plane to the origin $(0, 0, 0)$.
\end{problem}

\begin{proof}[Solution]
  Essentially, this question is asking to show that the normal vector to the
  plane is the vector $\v = \langle a, b, c \rangle$.

  Let $P_1$ and $P_2$ be points on the plane $P$. Then, we have $\p =
  \overline{P_1P_2} = \langle x_2 - x_1, y_2 - y_1, z_2 - z_1 \rangle$. Since
  they are both on the plane, we have
  \begin{align*}
    ax_1 + by_1 + cz_1 + d &= 0 \\
    ax_2 + by_2 + cz_2 + d &= 0
  .\end{align*}
  Subtracting the two equations, we have
  \[%
    a(x_2 - x_1) + b(y_2 - y_1) + c(z_2 - z_1) = 0
  .\]%
  Thus, we have
  \[%
    a(x_2 - x_1) + b(y_2 - y_1) + c(z_2 - z_1) = 0 \implies \v \cdot \p = 0
  .\]%
  Thus, we have shown that $\v$ is perpendicular to the plane.

  Now we compute the distance from the origin $(0, 0, 0)$ to the plane. The
  distance from a point $\vec{r}_0 = (x_0, y_0, z_0)$ to the plane defined by
  $ax + by + cz + d = 0$ is given by the formula
  \[%
    D = \frac{\lvert a x_0 + b y_0 + c z_0 + d \rvert}{\sqrt{a^2 + b^2 + c^2}}
  .\]%
  Plugging in the origin $(0, 0, 0)$, we get
  \[%
    D = \frac{\lvert a \cdot 0 + b \cdot 0 + c \cdot 0 + d \rvert}{\sqrt{a^2 + b^2 + c^2}} = \frac{\lvert d \rvert}{\sqrt{a^2 + b^2 + c^2}}
  .\]%
  This completes the proof.
\end{proof}

\begin{problem}[1.4.10]
  The natural orientation of $\R^2$ makes it possible to associate a sign to the
  area $A$ of a parallelogram generated by two linearly independent vectors $u,
  v \in \R^2$. To do this, let $\{\e_i\}$, $i = 1, 2$, be the natural ordered
  basis of $\R^2$, and write $u = u_1\e_1 + u_2\e_2$, $v = v_1\e_1 + v_2\e_2$.
  Observe the matrix relation
  \[%
    \begin{pmatrix}
      u \cdot u & u \cdot v \\
      v \cdot u & v \cdot v
    \end{pmatrix}
    = \begin{pmatrix}
      u_1 & u_2 \\
      v_1 & v_2
    \end{pmatrix}
    \begin{pmatrix}
      u_1 & v_1 \\
      u_2 & v_2
    \end{pmatrix}
  ,\]%
  and conclude that
  \[%
    A^2 = \begin{vmatrix}
      u_1 & u_2 \\
      v_1 & v_2
    \end{vmatrix}^2
  .\]%
  Since the last determinant has the same sign as the basis $\{u, v\}$, we can
  say that $A$ is positive or negative according to whether the orientation of
  $\{u, v\}$ is positive or negative. This is called the \textit{orientated
  area} in $\R^2$.
\end{problem}

\begin{solution}
\end{solution}

\begin{problem}[1.4.11]
  \begin{enumerate}
    \item Show that the volume $V$ of a parallelepiped generated by three
      linearly independent vectors $\u, \v, \w \in \R^3$ is given by $V = \lvert
      (\u \wedge \v) \cdot \w \rvert$, and introduce an \textit{orientated
      volume} in $\R^3$.

    \item Prove that
      \[%
        V^2 = \begin{vmatrix}
          \u \cdot \u & \u \cdot \v & \u \cdot \w \\
          \v \cdot \u & \v \cdot \v & \v \cdot \w \\
          \w \cdot \u & \w \cdot \v & \w \cdot \w \\
        \end{vmatrix}
      .\]%
  \end{enumerate}
\end{problem}

\begin{solution}[(i)]
  The volume of a parallelepiped is given by $V = \textrm{Base} \times
  \textrm{Height}$. The base is given by the area of the parallelogram formed by
  the vectors $\u$ and $\v$, which is given by $\lvert \u \wedge \v \rvert$. The
  height is given by the component of $\w$ in the direction of the normal vector
  $\u \wedge \v$, which is given by
  \[%
    h = \frac{\lvert (\u \wedge \v) \cdot \w \rvert}{\lVert \u \wedge \v \rVert}
  .\]%
  Therefore, we have
  \[%
    V = \textrm{Base} \times \textrm{Height} = \lVert \u \wedge \v \rVert \cdot \frac{\lvert (\u \wedge \v) \cdot \w \rvert}{\lVert \u \wedge \v \rVert} = \lvert (\u \wedge \v) \cdot \w \rvert
  .\]%

  The oriented volume $V_{\text{oriented}}$ carries a sign that depends on
  whether $(\u, \v, \w)$ form a right-handed or left-handed basis. If $(\u, \v,
  \w)$ follows the right-hand rule, then $V_{\text{oriented}}$ is positive.
  Otherwise, it is negative.
\end{solution}

\begin{solution}[(ii)]
  Let $A = [\u~\v~\w]$. Then, we have
  \[%
    \det(A)\det(A) = \det(A^2) = \det(A^TA) = \det(A^T) \cdot \det(A)
    = \det\begin{pmatrix}
      \u \\
      \v \\
      \w \\
    \end{pmatrix}
    \cdot \det\begin{pmatrix}
      \u & \v & \w \\
    \end{pmatrix}
  .\]%
  This gives us
  \[%
    \det(A^TA) = \begin{vmatrix}
      \u \cdot \u & \u \cdot \v & \u \cdot \w \\
      \v \cdot \u & \v \cdot \v & \v \cdot \w \\
      \w \cdot \u & \w \cdot \v & \w \cdot \w \\
    \end{vmatrix}
  .\]%
  Notice that $\det(A^TA) = \lvert (\u \wedge \v) \cdot \w \rvert^2 = V^2$.
  Thus, we've shown the desired result.
\end{solution}

\begin{problem}[1.5.1]
  Given the parametrized curve (helix)
  \[%
    \alpha(s) = \left\langle a\cos\left(\frac{s}{c}\right), a\sin\left(\frac{s}{c}\right), b\frac{s}{c} \right\rangle, \quad s \in \R
  ,\]%
  where $c^2 = a^2 + b^2$.
  \begin{enumerate}
    \item Show that the parameter $s$ is the arc length.

    \item Determine the curvature and the torsion of $\alpha$.

    \item Determine the osculating plane of $\alpha$.

    \item Show that the lines containing $\Na(s)$ and passing through
      $\alpha(s)$ meet the $z$-axis under a constant angle equal to
      $\sfrac{\pi}{2}$.

    \item Show that the tangent lines to $\alpha$ make a constant angle with the
      $z$-axis.
  \end{enumerate}
\end{problem}

\begin{solution}[(i)]
  If $\alpha(s)$ is parametrized by arc length, then the magnitude of the
  derivative of $\alpha(s)$ must be equal to $1$. We compute
  \[%
    \alpha'(s) = \left\langle -\frac{a}{c}\sin\left(\frac{s}{c}\right), \frac{a}{c}\cos\left(\frac{s}{c}\right), \frac{b}{c} \right\rangle
  .\]%
  The magnitude of the derivative is given by
  \begin{align*}
    \lVert \alpha'(s) \rVert &= \sqrt{\frac{a^2}{c^2}\sin^2\left(\frac{s}{c}\right) + \frac{a^2}{c^2}\cos^2\left(\frac{s}{c}\right) + \frac{b^2}{c^2}} \\
                             &= \sqrt{\frac{a^2}{c^2} + \frac{b^2}{c^2}} = \sqrt{1} = 1
  .\end{align*}
  Therefore, we have shown that $s$ is the arc length.
\end{solution}

\begin{solution}[(ii)]
  The curvature is given by $\kappa(s) = \lVert \alpha''(s) \rVert$. Computing
  the second derivative from part (i), we have
  \[%
    \alpha''(s) = \left\langle -\frac{a}{c^2}\cos\left(\frac{s}{c}\right), -\frac{a}{c^2}\sin\left(\frac{s}{c}\right), 0 \right\rangle
  .\]%
  The magnitude of the second derivative is given by
  \[%
    \kappa(s) = \lVert \Ta'(s) \rVert = \sqrt{\frac{a^2}{c^4}\cos^2\left(\frac{s}{c}\right) + \frac{a^2}{c^4}\sin^2\left(\frac{s}{c}\right)} = \frac{a}{c^2}
  .\]%
  Thus, the curvature is given by $\kappa(s) = \sfrac{a}{c^2}$.

  The torsion, $\tau(s)$, is given by $\Ba'(s) = \tau(s)\Na(s)$. Using the unit
  normal and the binormal vector from part (iii). Now, we compute the derivative
  of $\Ba(s)$,
  \[%
    \Ba'(s) = \left\langle \frac{b}{c^2}\cos\left(\frac{s}{c}\right), \frac{b}{c^2}\sin\left(\frac{s}{c}\right), 0 \right\rangle
  .\]%
  Therefore, we get
  \[%
    \left\langle \frac{b}{c^2}\cos\left(\frac{s}{c}\right), \frac{b}{c^2}\sin\left(\frac{s}{c}\right), 0 \right\rangle = \tau(s) \left\langle -\cos\left(\frac{s}{c}\right), -\sin\left(\frac{s}{c}\right), 0 \right\rangle
  .\]%
  Thus, the torsion is given by $\tau(s) = -\sfrac{b}{c^2}$.
\end{solution}

\begin{solution}[(iii)]
  Since $\alpha(s)$ is parametrized by arc length, then $\Ta(s) = \alpha'(s)$.
  Now, we need to find the unit normal vector, $\Na(s)$, which is given by
  \[%
    \Na(s) = \frac{\Ta(s)}{\lVert \Ta(s) \rVert} = \left\langle -\cos\left(\frac{s}{c}\right), -\sin\left(\frac{s}{c}\right), 0 \right\rangle
  .\]%
  The osculating plane at $s$ is the plane through $\alpha(s)$ orthogonal to
  $\Ba(s)$, i.e., $\Ba(s) = \Ta(s) \wedge \Na(s)$. Now, we need to find the
  binormal vector, $\Ba(s)$,
  \begin{align*}
    \Ba(s) = \Ta(s) \wedge \Na(s) &= \left\langle -\frac{a}{c}\sin\left(\frac{s}{c}\right), \frac{a}{c}\cos\left(\frac{s}{c}\right), \frac{b}{c} \right\rangle \wedge \left\langle -\cos\left(\frac{s}{c}\right), -\sin\left(\frac{s}{c}\right), 0 \right\rangle \\
           &= \begin{vmatrix}
             \ui & \uj & \uk \\
             -\frac{a}{c}\sin\left(\frac{s}{c}\right) & \frac{a}{c}\cos\left(\frac{s}{c}\right) & \frac{b}{c} \\
             -\cos\left(\frac{s}{c}\right) & -\sin\left(\frac{s}{c}\right) & 0 \\
           \end{vmatrix} \\
           &= \left\langle \frac{b}{c}\sin\left(\frac{s}{c}\right), -\frac{b}{c}\cos\left(\frac{s}{c}\right), \frac{a}{c}\sin^2\left(\frac{s}{c}\right) + \frac{a}{c}\cos^2\left(\frac{s}{c}\right) \right\rangle \\
           &= \left\langle \frac{b}{c}\sin\left(\frac{s}{c}\right), -\frac{b}{c}\cos\left(\frac{s}{c}\right), \frac{a}{c} \right\rangle
  .\end{align*}
  Now, we can find the osculating plane. The osculating plane is given by the
  equation
  \[%
    0 = \Ba(s) \cdot \left(x - x_0, y - y_0, z - z_0\right)
  ,\]%
  where $\alpha(s) = \langle x_0, y_0, z_0 \rangle$. Thus, the osculating plane
  is given by
  \[%
    \left\langle x - a\cos\left(\frac{s}{c}\right), y - a\sin\left(\frac{s}{c}\right), z - b \frac{s}{c} \right\rangle \cdot \left\langle \frac{b}{c}\sin\left(\frac{s}{c}\right), -\frac{b}{c}\cos\left(\frac{s}{c}\right), \frac{a}{c} \right\rangle = 0
  .\qedhere\]%
\end{solution}

\begin{solution}[(iv)]
  A line through $\alpha(s)$ in the direction of $\Na(s)$ is given by
  \[%
    \ell_s(t) = \alpha(s) + t \Na(s)
  ,\]%
  and the $z$-axis consists of all points of the form $(0,0,z)$. To find where
  the line intersects the $z$-axis, set the $x$ and $y$ components of
  $\ell_s(t)$ to zero:
  \begin{align*}
    x(t) &= a\cos\left(\frac{s}{c}\right) - t\cos\left(\frac{s}{c}\right) = 0 \\
    y(t) &= a\sin\left(\frac{s}{c}\right) - t\sin\left(\frac{s}{c}\right) = 0
  .\end{align*}
  Solving either equation (assuming $\cos(\frac{s}{c}) \neq 0$ or
  $\sin(\frac{s}{c}) \neq 0$), we get $t = a$. Plugging into the $z$-component
  \[%
    z = b\frac{s}{c} + 0 = b\frac{s}{c}
  .\]%
  Thus, the intersection point with the $z$-axis is
  \[%
    \left(0,0, b\frac{s}{c}\right)
  .\]%
  Now, consider the vector from $\alpha(s)$ to this point
  \begin{align*}
    v &= (0,0, b\frac{s}{c}) - \alpha(s)
      = \left\langle -a\cos\left(\frac{s}{c}\right), -a\sin\left(\frac{s}{c}\right), 0 \right\rangle
  .\end{align*}
  Since this vector is proportional to $\Na(s)$, and lies entirely in the
  $xy$-plane, it is orthogonal to the $z$-axis. Therefore, the angle between
  this line and the $z$-axis is $\sfrac{\pi}{2}$.

  Hence, we have shown that these lines intersect the $z$-axis at a constant
  angle of $\sfrac{\pi}{2}$.
\end{solution}

\begin{solution}[(v)]
  Since $\alpha(s)$ is parametrized by arc length, the unit tangent vector is
  \[%
    \Ta(s) = \alpha'(s) = \left\langle -\frac{a}{c}\sin\left(\frac{s}{c}\right), \frac{a}{c}\cos\left(\frac{s}{c}\right), \frac{b}{c} \right\rangle
  .\]%
  The direction vector of the $z$-axis is $\uk = \langle 0,0,1 \rangle$. Then
  the angle $\theta$ between $\Ta(s)$ and the $z$-axis satisfies
  \[%
    \cos(\theta) = \Ta(s) \cdot \uk = \frac{b}{c}
  .\]%
  Since this value is constant (independent of $s$), the angle between the
  tangent vector and the $z$-axis is constant, and is given by
  \[%
    \theta = \cos^{-1}\left(\frac{b}{c}\right)
  .\qedhere\]%
\end{solution}

\begin{problem}[1.5.11]
  One often gives a plane curve in polar coordinates by $\rho = \rho(\theta)$,
  $a \le \theta \le b$.
  \begin{enumerate}
    \item Show that the arc length is
      \[%
        \int_a^b \sqrt{\rho^2 + (\rho')^2} \dd{\theta}
      ,\]%
      where the prime denotes the derivative relative to $\theta$.

    \item Show that the curvature is
      \[%
        \kappa(\theta) = \frac{2(\rho')^2 - \rho\rho'' + \rho^2}{[(\rho')^2 + \rho^2]^{\sfrac{3}{2}}}
      .\]%
  \end{enumerate}
\end{problem}

\begin{solution}[(i)]
  The curve in polar coordinates is given by $\alpha(\theta) = \langle
  \rho\cos(\theta), \rho\sin(\theta) \rangle$. The general formula for the arc
  length is given by
  \[%
    s = \int_a^b \lVert \alpha'(\theta) \rVert \dd{\theta} = \int_a^b \sqrt{\left(\odv{x}{\theta}\right)^2 + \left(\odv{y}{\theta}\right)^2} \dd{\theta}
  .\]%
  We compute the derivatives
  \[%
    \odv{x}{\theta} = \rho'\cos(\theta) - \rho\sin(\theta)
    \aand
    \odv{y}{\theta} = \rho'\sin(\theta) + \rho\cos(\theta)
  .\]%
  Thus, we have
  \begin{align*}
    \left(\odv{x}{\theta}\right)^2 + \left(\odv{y}{\theta}\right)^2 &= \left(\rho'\cos(\theta) - \rho\sin(\theta)\right)^2 + \left(\rho'\sin(\theta) + \rho\cos(\theta)\right)^2 \\
                                                                    &= \cos^2(\theta)(\rho')^2 - 2\sin(\theta)\cos(\theta)\rho\rho' + \sin^2(\theta)\rho^2 \\
                                                                    &\qquad + \sin^2(\theta)(\rho')^2 + 2\sin(\theta)\cos(\theta)\rho\rho' + \cos^2(\theta)\rho^2 \\
                                                                    &= (\rho')^2 + \rho^2(\theta)
  .\end{align*}
  Therefore, we have
  \[%
    s = \int_a^b \sqrt{\left(\odv{x}{\theta}\right)^2 + \left(\odv{y}{\theta}\right)^2} \dd{\theta} = \int_a^b \sqrt{{\rho'}^2 + \rho^2} \dd{\theta}
  .\]%
  Thus, we have shown it's the arc length.
\end{solution}

\begin{solution}[(ii)]
  The curvature is given by
  \[%
    \kappa(\theta) = \frac{\lVert \alpha'(\theta) \wedge \alpha''(\theta) \rVert}{\lVert \alpha'(\theta) \rVert^3}
  .\]%
  We compute the first and second derivatives
  \begin{align*}
    \alpha'(\theta) &= \rho' \langle \cos(\theta), \sin(\theta) \rangle + \rho \langle -\sin(\theta), \cos(\theta) \rangle \\
      &= \langle \rho' \cos(\theta) - \rho \sin(\theta), \rho' \sin(\theta) + \rho \cos(\theta) \rangle \\
    \alpha''(\theta) &= \rho'' \langle \cos(\theta), \sin(\theta) \rangle + 2\rho' \langle -\sin(\theta), \cos(\theta) \rangle + \rho \langle -\cos(\theta), -\sin(\theta) \rangle \\
                     &= \langle \rho'' \cos(\theta) - 2\rho' \sin(\theta) - \rho \cos(\theta), \rho'' \sin(\theta) + 2\rho' \cos(\theta) - \rho \sin(\theta) \rangle
  .\end{align*}

  Now, we compute the cross product
  \begin{align*}
    \alpha'(\theta) \wedge \alpha''(\theta)
      &= (\rho' \cos\theta - \rho \sin\theta)(\rho'' \sin\theta + 2\rho' \cos\theta - \rho \sin\theta) \\
      &\quad - (\rho' \sin\theta + \rho \cos\theta)(\rho'' \cos\theta - 2\rho' \sin\theta - \rho \cos\theta) \\
      &= \rho^2 + 2(\rho')^2 - \rho \rho''.
  \end{align*}

  Next, compute the norm of the first derivative
  \[%
    \lVert \alpha'(\theta) \rVert^2 = (\rho' \cos\theta - \rho \sin\theta)^2 + (\rho' \sin\theta + \rho \cos\theta)^2 = (\rho')^2 + \rho^2
  .\]%
  So the curvature is
  \[%
    \kappa(\theta) = \frac{\rho^2 + 2(\rho')^2 - \rho\rho''}{\left[(\rho')^2 + \rho^2\right]^{\sfrac{3}{2}}} = \frac{2(\rho')^2 - \rho\rho'' + \rho^2}{\left[(\rho')^2 + \rho^2\right]^{\sfrac{3}{2}}}
  .\]%
  Thus, we have shown the curvature.
\end{solution}

\begin{problem}[1.5.12]
  Let $\alpha : I \to \R^3$ be a regular parametrized curve (not necessarily by
  arc length), and let $\beta : J \to \R^3$ be a reparameterization of
  $\alpha(I)$ by the arc length $s = s(t)$, measured from $t_0 \in I$ (see
  Remark 2). Let $t = t(s)$ be the inverse function of $s$ and set
  $\odv{\alpha}/{t} = \alpha'$, $\odv[order={2}]{\alpha}/{t} = \alpha''$, etc.
  Prove that
  \begin{enumerate}
    \item $\odv{t}/{s} = \sfrac{1}{\lVert \alpha' \rVert}$,
      $\odv[order={2}]{t}/{s} = -(\sfrac{\alpha' \cdot \alpha''}{\lVert \alpha'
      \rVert^4})$.

    \item The curvature of $\alpha$ at $t \in I$ is
      \[%
        \kappa(t) = \frac{\lVert \alpha' \wedge \alpha'' \rVert}{\lVert \alpha' \rVert^3}
      .\]%

    \item The torsion of $\alpha$ at $t \in I$ is
      \[%
        \tau(t) = -\frac{(\alpha' \wedge \alpha'') \cdot \alpha'''}{\lVert \alpha' \wedge \alpha'' \rVert^2}
      .\]%

    \item If $\alpha : I \to \R^2$ is a plane curve $\alpha(t) = \langle x(t),
      y(t) \rangle$, the signed curvature (see Remark 1) of $\alpha$ at $t$ is
      \[%
        \kappa(t) = \frac{x' y'' - x'' y'}{[(x')^2 + (y')^2]^{\sfrac{3}{2}}}
      .\]%
  \end{enumerate}
\end{problem}

\begin{solution}[(i)]
  Since $\beta(s) = \alpha(t(s))$, that means that $\beta$ is a
  reparameterization of $\alpha$ by arc length. Thus, we have $\lVert \beta'(s)
  \rVert = 1$. Since $s = s(t)$ is arc length from $t_0 \in I$ to $t$, we have
  \[%
    s = \int_{t_0}^t \lVert \alpha'(x) \rVert \dx
  .\]%
  Differentiating both sides with respect to $t$, we have
  \[%
    \odv{s}{t} = \lVert \alpha'(t) \rVert
  .\]%
  Since $s = s(t)$ and $t = t(s)$ is its inverse, using the Inverse Function
  Theorem, we have
  \[%
    \odv{t}{s} = \frac{1}{\odv{s}/{t}} = \frac{1}{\lVert \alpha'(t) \rVert}
  .\]%

  Computing the second derivative, we have
  \begin{align*}
    \odv[order={2}]{t}{s} &= \odv{}{s} \frac{1}{\lVert \alpha'(t(s)) \rVert} \\
                          &= -\frac{1}{2\lVert \alpha'(t(s)) \rVert^2} \cdot \odv{}{s} \lVert \alpha'(t(s)) \rVert \\
                          &= -\frac{1}{2\lVert \alpha'(t(s)) \rVert^2} \cdot \odv{}{s} \left[\alpha'(t(s)) \cdot \alpha'(t(s))\right] \\
                          &= -\frac{1}{2\lVert \alpha'(t(s)) \rVert^2} \cdot \frac{2 \alpha'(t(s)) \cdot \alpha''(t(s))}{\lVert \alpha'(t(s)) \rVert} \cdot \odv{t}{s} \\
                          &= -\frac{\alpha'(t(s)) \cdot \alpha''(t(s))}{\lVert \alpha'(t(s)) \rVert^4}
  .\end{align*}
  This completes the proof.
\end{solution}

\begin{solution}[(ii)]
  Recall that $\beta(s) = \alpha(t(s))$ is the arc length reparameterization of
  $\alpha$. Then
  \[%
    \odv{\beta}{s} = \odv{t}{s} \cdot \odv{\alpha}{t} = \frac{1}{\lVert \alpha'(t) \rVert} \cdot \alpha'(t) = \frac{\alpha'(t)}{\lVert \alpha'(t) \rVert}
  .\]%
  Taking the derivative again with respect to $s$, we apply the chain rule
  \[%
    \odv[order=2]{\beta}{s} = \odv{}{s} \left( \frac{\alpha'(t)}{\lVert \alpha'(t) \rVert} \right) = \odv{}{t} \left(\frac{\alpha'(t)}{\lVert \alpha'(t) \rVert}\right) \cdot \frac{1}{\lVert \alpha'(t) \rVert}
  .\]%
  Let us now compute the derivative inside
  \[%
    \odv{}{t} \left(\frac{\alpha'(t)}{\lVert \alpha'(t) \rVert}\right) = \frac{\lVert \alpha'(t) \rVert \odv{}{t} \alpha'(t) - \alpha'(t) \odv{}{t} \lVert \alpha'(t) \rVert}{\lVert \alpha'(t) \rVert^2}
  .\]%
  Computing each term, we have
  \[%
    \odv{}{t} \alpha'(t) = \alpha''(t) \aand \odv{}{t} \lVert \alpha'(t) \rVert = \frac{\alpha'(t) \cdot \alpha''(t)}{\lVert \alpha'(t) \rVert}
  .\]%
  Therefore, we have
  \[%
    \odv[order=2]{\beta}{s} = \frac{\lVert \alpha'(t) \rVert \alpha''(t) - \alpha'(t) \frac{\alpha'(t) \cdot \alpha''(t)}{\lVert \alpha'(t) \rVert}}{\lVert \alpha'(t) \rVert^3}
  .\]%

  We decompose $\alpha''(t)$ into two components: one parallel and one
  perpendicular to $\alpha'(t)$. The parallel component is given by the
  projection
  \[%
    \alpha_{\parallel}(t) = \frac{\alpha'(t) \cdot \alpha''(t)}{\lVert \alpha'(t) \rVert^2} \alpha'(t)
  ,\]%
  and the perpendicular component is
  \[%
    \alpha_{\perp}(t) = \alpha''(t) - \alpha_{\parallel}(t)
  .\]%
  Since $\alpha_{\perp}(t)$ is perpendicular to $\alpha'(t)$, we have
  $\alpha'(t) \cdot \alpha_{\perp}(t) = 0$.

  Substituting the decomposition of $\alpha''(t)$ into the second derivative, we
  get
  \[%
    \odv[order=2]{\beta}{s} = \frac{\lVert \alpha'(t) \rVert (\alpha_{\parallel}(t) + \alpha_{\perp}(t)) - \alpha'(t) \frac{\alpha'(t) \cdot \alpha''(t)}{\lVert \alpha'(t) \rVert}}{\lVert \alpha'(t) \rVert^3}
  .\]%
  The parallel components cancel out
  \[%
    \lVert \alpha'(t) \rVert \alpha_{\parallel}(t) = \alpha'(t) \frac{\alpha'(t) \cdot \alpha''(t)}{\lVert \alpha'(t) \rVert}
  ,\]%
  so we are left with
  \[%
    \odv[order=2]{\beta}{s} = \frac{\lVert \alpha'(t) \rVert \alpha_{\perp}(t)}{\lVert \alpha'(t) \rVert^3} = \frac{\alpha_{\perp}(t)}{\lVert \alpha'(t) \rVert^2}
  .\]%

  Since $\alpha_{\perp}(t)$ is perpendicular to $\alpha'(t)$, we can express the
  magnitude of $\alpha_{\perp}(t)$ in terms of the cross product. Specifically,
  \[%
    \lVert \alpha_{\perp}(t) \rVert = \frac{\lVert \alpha'(t) \wedge \alpha''(t) \rVert}{\lVert \alpha'(t) \rVert}
  .\]%
  Thus, the second derivative simplifies to
  \[%
    \kappa(t) = \odv[order=2]{\beta}{s} = \frac{\lVert \alpha'(t) \wedge \alpha''(t) \rVert}{\lVert \alpha'(t) \rVert^3}
  .\]%
  This completes the proof.
\end{solution}

\begin{solution}[(iii)]
  We know that $\Ba'(s) = \tau(s) \Na(s)$ and $\Ba(s(t)) = \Ta(s(t)) \wedge
  \Na(s(t))$. Differentiating with respect to $s$, we have
  \begin{align*}
    \odv{\Ba}{s} = \odv{}{s} (\Ta(s) \wedge \Na(s)) &= \odv{\Ta}{s} \wedge \Na(s) + \Ta(s) \wedge \odv{\Na}{s} \\
                                                    &= \kappa(s) \Na(s) \wedge \Na(s) + \Ta(s) \wedge (-\kappa(s)\Ta(s) + \tau(s)\Ba(s)) \\
                                                    &= \tau(s)(\Ta(s) \wedge \Ba(s)) \\
                                                    &= -\tau(s)\Na(s)
  .\end{align*}
  But, we want $\odv{\Ba}/{t}$, so, applying the chain rule, we have
  \[%
    \odv{\Ba}{t} = \odv{\Ba}{s} \cdot \odv{s}{t} = -\tau(s) \Na(s) \cdot \lVert \alpha'(t) \rVert
  .\]%
  Since torsion is a function of arc-length, we have $\tau(s) = \tau(t(s))$.
  Therefore, we can express the torsion as
  \[%
    \tau(t(s)) = -\odv{\Ba}{t} \cdot \Na(s) \frac{1}{\lVert \alpha'(t) \rVert}
  .\]%
  But, since $\Na(s)$ and $\Na(t)$ are the same vector, the expression can be
  rewritten as
  \[%
    \tau(t(s)) = -\odv{\Ba}{t} \cdot \frac{\Na(t)}{\lVert \alpha'(t) \rVert}
  .\]%
\end{solution}

\begin{solution}[(iv)]
  Computing $\alpha' \wedge \alpha''$, we have
  \[%
    \alpha' \wedge \alpha'' = \begin{vmatrix}
      \ui & \uj & \uk \\
      x' & y' & 0 \\
      x'' & y'' & 0 \\
    \end{vmatrix} = \left\langle 0, 0, x' y'' - x'' y' \right\rangle
  .\]%
  The magnitude of the cross product is given by
  \[%
    \lVert \alpha' \wedge \alpha'' \rVert = \sqrt{(x' y'' - x'' y')^2} = \lvert x' y'' - x'' y' \rvert
  .\]%
  The magnitude of the first derivative cubed is given by
  \[%
    \lVert \alpha' \rVert^2 = \left((x')^2 + (y')^2\right)^{\sfrac{3}{2}}
  .\]%
  Therefore, the curvature is given by
  \[%
    \kappa(t) = \frac{\lVert \alpha' \wedge \alpha'' \rVert}{\lVert \alpha' \rVert^3} = \frac{\lvert x' y'' - x'' y' \rvert}{\left[(x')^2 + (y')^2\right]^{\sfrac{3}{2}}}
  .\]%
  This completes the proof.
\end{solution}

\begin{problem}[1.5.14]
  Let $\alpha : (a, b) \to \R^2$ be a regular parametrized plane curve. Assume
  that there exists $t_0$, $a < t_0 < b$, such that the distance $\lVert
  \alpha(t) \rVert$ from the origin to the trace of $\alpha$ will be a maximum
  at $t_0$. Prove that the curvature $\kappa$ of $\alpha$ at $t_0$ satisfies
  $\lvert \kappa(t_0) \rvert \ge \sfrac{1}{\lVert \alpha(t_0) \rVert}$.
\end{problem}

\begin{solution}
  Let $f(t) = \|\alpha(t)\|^2$, the square of the distance from the origin to
  the curve. Since $f$ is maximized at $t_0$, we have
  \[%
    f'(t_0) = 0 \aand f''(t_0) \le 0
  .\]%
  Compute the first derivative
  \[%
    f'(t) = 2 \alpha(t) \cdot \alpha'(t)
  ,\]%
  so at $t_0$,
  \[%
    \alpha(t_0) \cdot \alpha'(t_0) = 0
  .\]%
  Thus, the position vector $\alpha(t_0)$ is perpendicular to the velocity
  vector $\alpha'(t_0)$, and lies in the direction of the unit normal vector
  $\Na(t_0)$. Hence we can write
  \[%
    \alpha(t_0) = \lVert \alpha(t_0) \rVert \Na(t_0)
  .\]%
  Now compute the second derivative
  \[%
    f''(t) = 2\left(\lVert \alpha'(t) \rVert^2 + \alpha(t) \cdot \alpha''(t)\right)
  .\]%
  At $t_0$, the condition $f''(t_0) \le 0$ implies
  \[%
    \lVert \alpha'(t_0) \rVert^2 + \alpha(t_0) \cdot \alpha''(t_0) \le 0
  .\]%
  Using $\alpha(t_0) = \lVert \alpha(t_0) \rVert \mathbf{N}(t_0)$, we compute
  \begin{align*}
    \alpha(t_0) \cdot \alpha''(t_0) &= \lVert \alpha(t_0) \rVert \Na(t_0) \cdot \alpha''(t_0) \\
                                    &= \lVert \alpha(t_0) \rVert \cdot \kappa(t_0) \lVert \alpha'(t_0) \rVert^2
  ,\end{align*}
  by the Frenet-Serret formula in the plane.

  Substituting into the inequality, we obtain
  \[%
    \lVert \alpha'(t_0) \rVert^2 + \lVert \alpha(t_0) \rVert \cdot \kappa(t_0) \lVert \alpha'(t_0) \rVert^2 \le 0
  .\]%
  Factoring out $\lVert \alpha'(t_0) \rVert^2 > 0$, we get
  \[%
    1 + \lVert \alpha(t_0) \rVert \cdot \kappa(t_0) \le 0 \implies \kappa(t_0) \le -\frac{1}{\lVert \alpha(t_0) \rVert}
  .\]%
  Alternatively, if $\kappa(t_0) \ge 0$, then this same argument applies with
  the curve reflected through the origin (i.e., apply the same proof to
  $-\alpha(t)$). In either case, we conclude
  \[%
    \lvert \kappa(t_0) \rvert \ge \frac{1}{\lVert \alpha(t_0) \rVert}
  .\]%
  This completes the proof.
\end{solution}
