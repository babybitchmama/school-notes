\begin{problem}[2.3.4]
  Construct a diffeomorphism between the ellipsoid
  \[%
    \frac{x^2}{a^2} + \frac{y^2}{b^2} + \frac{z^2}{c^2} = 1
   ,\]%
   and the sphere $x^2 + y^2 + z^2 = 1$.
\end{problem}

\begin{solution}
  Let $S_1$ be the ellipsoid and $S_2$ be the sphere. We can define the
  following parameterizations $\x_1 : (0, \pi) \times (0, 2\pi) \to \R^3$ and
  $\x_2 : (0, \pi) \times (0, 2\pi) \to \R^3$, defined by
  \[%
    \x_1 = \left\langle a\sin(\phi)\cos(\theta), b\sin(\phi)\sin(\theta), c\cos(\phi) \right\rangle
    \aand
    \x_2 = \left\langle \sin(\phi)\cos(\theta), \sin(\phi)\sin(\theta), \cos(\phi) \right\rangle
  ,\]%
  where $\x_1$ is the parameterization of the ellipsoid and $\x_2$ is the
  parameterization of the sphere. The diffeomorphism $\Phi : S_1 \to S_2$ can be
  defined as $\Phi(x, y, z) = \left(\x_2 \circ \x_1^{-1}\right)(x, y, z)$. We're
  trying to go from the ellipsoid to the sphere, and both are given in terms of
  the same angles, just scaled differently. All we are doing is just dividing by
  the semi-axes, giving us
  \[%
    \Phi(x, y, z) = \left\langle \frac{x}{a}, \frac{y}{b}, \frac{z}{c} \right\rangle
  .\]%

  Now, we show that $\Phi(x, y, z)$ is a diffeomorphism between $S_1$ and $S_2$.
  It is clear that $\Phi$ is a smooth map since it is composed of smooth
  functions. The inverse $\Phi^{-1} : S_2 \to S_1$ is given by
  \[%
    \Phi(x, y, z) = \left\langle ax, by, cz \right\rangle
  .\]%
  Again, $\Phi^{-1}$ is smooth since it is composed of smooth functions.

  Therefore, $\Phi$ is a diffeomorphism between the ellipsoid $S_1$ and the
  sphere $S_2$.
\end{solution}

\begin{problem}[2.3.7]
  Prove that the relation ``$S_1$ is diffeomorphism to $S_2$'' is an equivalence
  relation in the set of regular surfaces.
\end{problem}

\begin{solution}
  For the relation ``$S_1$ is diffeomorphic to $S_2$'' to be an equivalence
  relation, it must satisfy three properties: reflexivity, symmetry, and
  transitivity.

  We first prove reflexivity. For any regular surface $S$, we can define the
  identity map $I_S : S \to S$ given by $I_S(x) = x$ for all $x \in S$. The
  identity map is clearly a diffeomorphism since it is smooth and has a smooth
  inverse (itself). Thus, $S$ is diffeomorphic to itself, satisfying the
  reflexivity property.

  Next, we prove symmetry. If $\Phi$ is a diffeomorphic mapping from $S_1$ to
  $S_2$, then, there exists a smooth inverse $\Phi^{-1}$ from $S_2$ to $S_1$. By
  definition, $\Phi^{-1}$ is also a diffeomorphism. Therefore, if $S_1$ is
  diffeomorphic to $S_2$ via $\Phi$, then $S_2$ is diffeomorphic to $S_1$ via
  $\Phi^{-1}$, satisfying the symmetry property.

  Finally, we prove transitivity. If $S_1$ is diffeomorphic to $S_2$ via
  $\Phi_{12}$ and $S_2$ is diffeomorphic to $S_3$ via $\Phi_{23}$, then we can
  define a new map $\Phi_{13} = \Phi_{12} \circ \Phi_{23}$. The composition of
  two diffeomorphisms is also a diffeomorphism, as it is smooth and has a smooth
  inverse given by $(\Phi_{23})^{-1} \circ (\Phi_{12})^{-1}$. Thus, $S_1$ is
  diffeomorphic to $S_3$ via $\Phi_{13}$, satisfying the transitivity property.

  Since the relation satisfies reflexivity, symmetry, and transitivity, we can
  conclude that the relation ``$S_1$ is diffeomorphic to $S_2$'' is indeed an
  equivalence relation in the set of regular surfaces.
\end{solution}

\begin{problem}[2.3.14]
  Let $A \subset S$ be a subset of a regular surface $S$. Prove that $A$ is
  itself a regular surface if and only if $A$ is open in $S$; that is, $A = U
  \cap S$, where $U$ is an open set in $\R^3$.
\end{problem}

\begin{solution}
  Assume $A$ is a regular subsurface of $S$, with $\pd{A}$ being the boundary of
  $A$. Suppose $A$ is closed. Let $p \in \pd{A} \subset A$. Since $A$ is a
  regular surface, that means there exists a neighborhood $V_\delta(p)$, where
  $\delta > 0$, such that there is a differentiable, homeomorphic mapping from
  an open set $U \subset \R^2$ to $V_\delta(p) \cap A$. Since $p \in \pd{A}$,
  there doesn't exist such a neighborhood around $p$ that's contained in $A$,
  giving us a contradiction. Therefore, $A$ must be open in $S$.

  Conversely, assume $A$ is open in $S$. This means that for every point $p \in
  A$, there exists a neighborhood $V_\delta(p)$ such that $V_\delta(p) \cap S
  \subset A$. Since $S$ is a regular surface, we can find a differentiable,
  homeomorphic mapping from an open set $U \subset \R^2$ to $V_\delta(p) \cap
  S$. The restriction mapping $\x |_S : U \to V_\delta(p) \cap S$ is also a
  differentiable, homeomorphic mapping, which means that $A$ is a regular
  surface.

  Thus, $A$ is a regular surface if and only if it is open in $S$.
\end{solution}

\begin{problem}[2.4.2]
  Determine the tangent planes of $x^2 + y^2 - z^2 = 1$ at the points $(x, y,
  0)$ and show that they are all parallel to the $z$-axis.
\end{problem}

\begin{solution}
\end{solution}

\begin{problem}[2.4.8]
  Prove that if $L : \R^3 \to \R^3$ is a linear map and $S \subset \R^3$ is a
  regular surface invariant under $L$, i.e., $L(S) \subset S$, then the
  restriction $L \mid S$ is a differential map and
  \[%
    \dd{L_p}(w) = L(w), \quad p \in S, w \in T_p(S)
  .\]%
\end{problem}

\begin{solution}
\end{solution}

\begin{problem}[2.4.9]
  Show that the parametrized surface
  \[%
    \x(u, v) = \langle v\cos(u), v\sin(u), au \rangle, \quad a \ne 0
  ,\]%
  is regular. Compute its normal vector $\Na(u, v)$ and show that along the
  coordinate line $u = u_0$, the tangent plane of $\x$ rotates about this line
  in such a way that the tangent of its angle with the $z$-axis is proportional
  to the inverse of the distance $v = \sqrt{x^2 + y^2}$ of the point $\x(u_0,
  v)$ to the $z$-axis.
\end{problem}

\begin{solution}
\end{solution}

\begin{problem}[2.4.15]
  Show that if all normals to a connected surface pass through a fixed point,
  the surface is contained in a sphere.
\end{problem}

\begin{solution}
\end{solution}

\begin{problem}[2.4.24]
  Show that if $\Phi : S_1 \to S_2$ and $\Psi : S_2 \to S_3$ are differential
  maps and $p \in S_1$, then
  \[%
    \dd(\Psi \circ \Phi)_p = \dd{\Psi_{\Phi(p)}} \circ \dd{\Phi_p}
  .\]%
\end{problem}

\begin{solution}
  Define $\beta : (-\epsilon, \epsilon) \to S_1$ be a smooth curve such that
  $\beta(0) = p = \Phi^{-1}(q)$ and $\beta'(0) = \w$. Define $\x = \Phi \circ
  \beta : (-\epsilon, \epsilon) \to S_1 \to S_2$.
  % Then, this gives us that $\dd{\Phi_p} : T_p(S_1) \to T_{\Phi(p)}(S_2)$, where it's defined as
  % \[%
  %   \dd{\Phi_p}(\w) = \odv{}{t} \bigg|_{t=0} (\Phi \circ \beta)(t)
  % .\]%
\end{solution}

\begin{problem}[2.5.1]
  Compute the first fundamental forms of the following parametrized surfaces
  where they are regular:
  \begin{enumerate}
    \item $\x(u, v) = \langle a\sin(u)\cos(v), b\sin(u)\sin(v), c\cos(u)
      \rangle$; ellipsoid.

    \item $\x(u, v) = \langle au\cos(v),bu\sin(v), u^2 \rangle$; elliptic
      paraboloid.

    \item $\x(u, v) = \langle au\cosh(v), bu\sinh(v), u^2 \rangle$; hyperbolic
      paraboloid.

    \item $\x(u, v) = \langle a\sinh(u)\cos(v), b\sinh(u)\sin(v), c\cosh(u)
      \rangle$; hyperboloid of two sheets.
  \end{enumerate}
\end{problem}

\begin{solution}[(i)]
  Computing the partial derivatives, we have
  \[%
    \x_u = \langle a\cos(u)\cos(v), b\cos(u)\sin(v), -c\sin(u) \rangle \aand \x_v = \langle -a\sin(u)\sin(v), b\sin(u)\cos(v), 0 \rangle
  .\]%
  Computing $E$, $F$, and $G$, we have
  \begin{align*}
    E &= \langle \x_u, \x_u \rangle \\
      &= a^2\cos^2(u)\cos^2(v) + b^2\cos^2(u)\sin^2(v) + c^2\sin^2(u) \\
      &= \cos^2(u)(a^2\cos^2(v) + b^2\sin^2(v)) + c^2\sin^2(u) \\
    F &= \langle \x_u, \x_v \rangle \\
      &= -a^2\cos(u)\cos(v)\sin(u)\sin(v) + b^2\cos(u)\sin(v)\sin(u)\cos(v) \\
      &= \cos(u)\sin(u)(-a^2 + b^2)\cos(v)\sin(v) \\
    G &= \langle \x_v, \x_v \rangle \\
      &= a^2\sin^2(u)\sin^2(v) + b^2\sin^2(u)\cos^2(v) \\
      &= \sin^2(u)(a^2\sin^2(v) + b^2\cos^2(v))
  .\qedhere\end{align*}
\end{solution}

\begin{solution}[(ii)]
  Computing the partial derivatives, we have
  \[%
    \x_u = \langle a\cos(v), b\sin(v), 2u \rangle \aand \x_v = \langle -au\sin(v), bu\cos(v), 0 \rangle
  .\]%
  Computing $E$, $F$, and $G$, we have
  \begin{align*}
    E &= a^2\cos^2(v) + b^2\sin^2(v) + 4u^2 \\
    F &= -a^2u\cos(v)\sin(v) + b^2u\sin(v)\cos(v) \\
      &= u(b^2 - a^2)\sin(v)\cos(v) \\
    G &= a^2u^2\sin^2(v) + b^2u^2\cos^2(v) \\
      &= u^2(a^2\sin^2(v) + b^2\cos^2(v))
  .\qedhere\end{align*}
\end{solution}

\begin{solution}[(iii)]
  Computing the partial derivatives, we have
  \[%
    \x_u = \langle a\cosh(v), b\sinh(v), 2u \rangle \aand \x_v = \langle au\sinh(v), bu\cosh(v), 0 \rangle
  .\]%
  Computing $E$, $F$, and $G$, we have
  \begin{align*}
    E &= a^2\cosh^2(v) + b^2\sinh^2(v) + 4u^2 \\
    F &= a^2u\cosh(v)\sinh(v) + b^2u\sinh(v)\cosh(v) \\
      &= u(a^2 + b^2)\sinh(v)\cosh(v) \\
    G &= a^2u^2\sinh^2(v) + b^2u^2\cosh^2(v) \\
      &= u^2(a^2\sinh^2(v) + b^2\cosh^2(v))
  .\qedhere\end{align*}
\end{solution}

\begin{solution}[(iv)]
  Computing the partial derivatives, we have
  \[%
    \x_u = \langle a\cosh(u)\cos(v), b\cosh(u)\sin(v), c\sinh(u) \rangle \aand \x_v = \langle -a\sinh(u)\sin(v), b\sinh(u)\cos(v), 0 \rangle
  .\]%
  Computing $E$, $F$, and $G$, we have
  \begin{align*}
    E &= a^2\cosh^2(u)\cos^2(v) + b^2\cosh^2(u)\sin^2(v) + c^2\sinh^2(u) \\
      &= \cosh^2(u)(a^2\cos^2(v) + b^2\sin^2(v)) + c^2\sinh^2(u) \\
    F &= -a^2\cosh(u)\sinh(u)\cos(v)\sin(v) + b^2\cosh(u)\sinh(u)\sin(v)\cos(v) \\
      &= \cosh(u)\sinh(u)(-a^2 + b^2)\cos(v)\sin(v) \\
    G &= a^2\sinh^2(u)\sin^2(v) + b^2\sinh^2(u)\cos^2(v) \\
      &= \sinh^2(u)(a^2\sin^2(v) + b^2\cos^2(v))
  .\qedhere\end{align*}
\end{solution}

\begin{problem}[2.5.5]
  Show that the area $A$ of a bounded region $R$ of the surface $z = f(x, y)$ is
  \[%
    A = \iint_Q \sqrt{1 + f_x^2 + f_y^2} \dxy
  ,\]%
  where $Q$ is the normal projection of $R$ onto the $xy$-plane.
\end{problem}

\begin{solution}
  Let $\r(x, y) = \langle x, y, f(x, y) \rangle$ be a regular surface. Computing
  the cross product, we have
  \[%
    \r_x \times \r_y = \left\langle 1, 0, f_x \right\rangle \times \left\langle 0, 1, f_y \right\rangle = \langle -f_x, -f_y, 1 \rangle
  .\]%
  Therefore, we have
  \[%
    A = \iint_Q \lVert \r_x \wedge \r_y \rVert \dxy = \iint_Q \sqrt{1 + f_x^2 + f_y^2} \dxy
  .\qedhere\]%
\end{solution}

\begin{problem}[2.5.11]
  Let $S$ be a surface of revolution and $C$ its generating curve (cf. Example
  4, Sec. 2-3). Let $s$ be the arc length of $C$ and denote by $\rho = \rho(s)$
  the distance to the rotation axis of the point of $C$ corresponding to $s$.
  \begin{enumerate}
    \item \textit{(Pappus' Theorem.)} Show that the area of $S$ is
      \[%
        2\pi\int_0^l \rho(s) \ds
      ,\]%
      where $l$ is the length of $C$.

    \item Apply part (i) to compute the area of a torus of revolution.
  \end{enumerate}
\end{problem}

\begin{solution}[(i)]
  Every point $p \in C$ traces out a circle of radius $\rho(s)$ as it revolves
  around the axis of rotation. Let $C$ be the curve parametrized by $C = \langle
  x(s), y(s), z(s) \rangle$. Without loss of generality, we can assume that the
  axis of rotation will be the $z$-axis. Therefore, we get the surface $\r : [0,
  l] \times [0, 2\pi] \to S$, defined by $\r(s, \theta) = \langle
  \rho(s)\cos(\theta), \rho(s)\sin(\theta), z(s) \rangle$. The surface area of a
  given surface is given by
  \[%
    A = \iint_S \lVert \r_s \wedge \r_\theta \rVert \dd{s,\theta}
  .\]%
  Computing the partial derivatives, we have
  \[%
    \r_s = \langle \rho'(s)\cos(\theta), \rho'(s)\sin(\theta), z'(s) \rangle
    \aand
    \r_\theta = \langle -\rho(s)\sin(\theta), \rho(s)\cos(\theta), 0 \rangle
  .\]%
  Computing the cross product, we have
  \[%
    \lVert \r_s \times \r_\theta \rVert = \rho(s) \sqrt{(z'(s))^2 + (\rho'(s))^2} = \rho(s),
  ,\]%
  since $(z')^2 + (\rho')^2 = 1$, due to arc length parametrization. Therefore,
  we have
  \[%
    A = \iint_S \lVert \r_s \wedge \r_\theta \rVert \dd{s,\theta} = \int_0^{2\pi} \int_0^l \rho(s) \dd{s,\theta} = 2\pi \int_0^l \rho(s) \dd{s}
  .\qedhere\]%
\end{solution}

\begin{solution}[(ii)]
  The generating curve of a torus is a circle of radius $r$ centered at the
  origin, which is given by the parametrization, $C : [0, 2\pi] \to \R^2$,
  defined by
  \[%
    C(\theta) = \langle R + r\cos(\theta), R + r\sin(\theta) \rangle
  .\]%
  This curve is parametrized by angle, not arc length. Since it's a circle, we
  know its total length is $l = 2\pi r$. Without loss of generality, we can
  assume that the axis of rotation is the $z$-axis. Then, we get $\rho(\theta) =
  R + r\cos(\theta)$.

  But we want the arc length parametrization, so we must write everything in
  terms of $s$. Fortunately, for a circle of radius $r$, the natural arc length
  parametrization is $\theta = \sfrac{s}{r}$, where $s \in [0, 2\pi r]$. Now,
  applying Pappus' theorem, we have
  \[%
    A = 2\pi\int_0^{2\pi r} \rho(s) \dd{s} = 2\pi\int_0^{2\pi r} R + r\cos\left(\frac{s}{r}\right) \dd{s} = 4\pi^2 Rr
  .\qedhere\]%
\end{solution}
