\begin{problem}[4.4.8]
  Show that if all the geodesics of a connected sphere are plane curves, then the surface is contained in a plane or a sphere.
\end{problem}

\begin{solution}
\end{solution}

\begin{problem}[4.4.11]
  State precisely and prove: The algebraic value of the covariant derivative is invariant under orientation-preserving isometries.
\end{problem}

\begin{solution}
\end{solution}

\begin{problem}[4.5.1]
  Let $S \subset \R^3$ be a regular, compact, connected, orientable surface which is not homeomorphic to a sphere. Prove that there are points on $S$ where the Gaussian curvature is positive, negative, and zero.
\end{problem}

\begin{solution}
\end{solution}

\begin{problem}[4.5.2]
  Let $T$ be a torus of revolution. Describe the image of the Gauss map of $T$ and show, without using the Gauss-Bonnet theorem, that
  \[%
    \iint_T K \dd{\sigma} = 0
  .\]%
  Compute the Euler-Poincar\'e characteristic of $T$ and check the above results with the Gauss-Bonnet theorem.
\end{problem}

\begin{solution}
\end{solution}

\begin{problem}[4.5.3]
  Let $S \subset \R^3$ be a regular compact surface with $K > 0$. Let $\Gamma \subset S$ be a simple closed geodesic in $S$, and let $A$ and $B$ be the regions of $S$ which have $\Gamma$ as a common boundary. Let $N : S \to S^2$ be the Gauss map of $S$. Prove that $N(A)$ and $N(B)$ have the same area.
\end{problem}

\begin{solution}
\end{solution}

\begin{problem}[4.5.5]
  Let $C$ be a parallel of colatitude $\phi$ on an oriented unit sphere $S^2$, and let $w_0$ be a unit vector tangent to $C$ at a point $p \in C$ (cf. Example 1, Sec 4.4). Take the parallel transport of $w_0$ along $C$ and show that its position, after a complete turn, makes an angle $\Delta \phi = 2\pi(1 - \cos(\phi))$ with the initial position $w_0$. Check that
  \[%
    \lim_{R \to p} \frac{\Delta \phi}{A} = 1 =~\text{curvature of}~S^2
  ,\]%
  where $A$ is the area of the region $R$ of $S^2$ bounded by $C$.
\end{problem}

\begin{solution}
\end{solution}
