\begin{problem}[4.3.1]
  Show that if $\x$ is an orthogonal parametrization, that is, $F = 0$, then
  \[%
    K = -\frac{1}{2 \sqrt{EG}}\left[\left(\frac{E_v}{\sqrt{EG}}\right)_v + \left(\frac{G_u}{\sqrt{GE}}\right)_u\right]
  .\]%
\end{problem}

\begin{solution}
  Using the given equations in Do Carmo, we see that
  \[%
    \Gamma^1_{11} = \frac{E_u}{2E}, \quad \Gamma^2_{11} = -\frac{E_v}{2G}, \quad \Gamma^1_{12} = \frac{E_v}{2E}, \quad \Gamma^2_{12} = \frac{G_u}{2G}, \quad \Gamma^1_{22} = -\frac{G_u}{2E}, \aand \Gamma^2_{22} = \frac{G_v}{2G}
  .\]%
  We already know that
  \[%
    \left(\Gamma^2_{12}\right)_u - \left(\Gamma^2_{11}\right)_v + \Gamma^1_{12} \Gamma^2_{11} + \Gamma^2_{12} \Gamma^2_{12} - \Gamma^2_{11} \Gamma^2_{22} - \Gamma^1_{11} \Gamma^2_{12} = -EK
  .\]%
  Therefore, computing the partials of $\Gamma^2_{12}$ and $\Gamma^2_{11}$, we have
  \[%
    \left(\Gamma^2_{12}\right)_u = \left(\frac{G_u}{2G}\right)_u = \frac{GG_{uu} - G^2_u}{2G^2} \aand \left(\Gamma^2_{11}\right)_v = \left(\frac{E_v}{2G}\right)_v = \frac{GE_{vv} - G_vE_v}{2G^2}
  .\]%
\end{solution}

\begin{problem}[4.3.2]
  Show that if $\x$ is an isothermal parametrization, that is, $E = G = \lambda(u, v)$ and $F = 0$, then
  \[%
    K = -\frac{1}{2\lambda} \Delta (\log(\lambda))
  ,\]%
  where $\Delta \phi$ denotes the Laplacian $\left(\pdv[2]{\phi}/{u}\right) + \left(\pdv[2]{\phi}/{v}\right)$ of the function $\phi$. Conclude that when $E = G = (u^2 + v^2 + c)^{-2}$ and $F = 0$, then $K =$ const. $= 4c$.
\end{problem}

\begin{solution}
\end{solution}

\begin{problem}[4.3.4]
  Show that no neighborhood of a point in a sphere may be isometrically mapped into a plane.
\end{problem}

\begin{solution}
  Parameterizing a sphere by spherical coordinates, we have
  \[%
    \x(u, v) = (\rho\sin(u)\cos(v), \rho\sin(u)\sin(v), \rho\cos(u))
  ,\]%
  where $\rho$ is the radius of the sphere, $u \in [0, \pi]$ is the polar angle, and $v \in [0, 2\pi)$ is the azimuthal angle. Computing the partial derivatives, we have
  \[%
    \x_u = (\rho\cos(u)\cos(v), \rho\cos(u)\sin(v), -\rho\sin(u))
    \aand
    \x_v = (-\rho\sin(u)\sin(v), \rho\sin(u)\cos(v), 0)
  .\]%
  The first fundamental form coefficients are
  \[%
    E = \bra{\x_u, \x_u} = \rho^2,
    \quad
    F = \bra{\x_u, \x_v} = 0,
    \aand
    G = \bra{\x_v, \x_v} = \rho^2\sin^2(u)
  .\]%
  Therefore, using the Gauss formula from exercise 4.3.1, we have
  \begin{align*}
    K &= -\frac{1}{2\sqrt{EG}}\left[\left(\frac{E_v}{\sqrt{EG}}\right)_v + \left(\frac{G_u}{\sqrt{GE}}\right)_u\right] \\
      &= -\frac{1}{2\rho^2\sin(u)}\left[\left(\frac{0}{\rho\sin(u)}\right)_v + \left(\frac{\rho^2\cos(u)}{\rho^2}\right)_u\right] \\
      &= -\frac{-\sin(u)}{2\rho^2\sin(u)} \\
      &= \frac{1}{\rho^2}
  .\end{align*}

  Parametrizing a plane by Cartesian coordinates, we have
  \[%
    \bar{\x}(u, v) = (u, v, au + bv + c)
  .\]%
  The partial derivatives are
  \[%
    \bar{\x}_u = (1, 0, a)
    \aand
    \bar{\x}_v = (0, 1, b)
  .\]%
  The first fundamental form coefficients are
  \[%
    \bar{E} = \bra{\bar{\x}_u, \bar{\x}_u} = 1,
    \quad
    \bar{F} = \bra{\bar{\x}_u, \bar{\x}_v} = 0,
    \aand
    \bar{G} = \bra{\bar{\x}_v, \bar{\x}_v} = 1
  .\]%
  Therefore, using the Gauss formula from exercise 4.3.1, we have
  \[%
    \bar{K} = -\frac{1}{2\sqrt{\bar{E}\bar{G}}}\left[\left(\frac{\bar{E}_v}{\sqrt{\bar{E}\bar{G}}}\right)_v + \left(\frac{\bar{G}_u}{\sqrt{\bar{G}\bar{E}}}\right)_u\right] = -\frac{1}{2}\left[\left(\frac{0}{1}\right)_v + \left(\frac{0}{1}\right)_u\right] = 0
  .\]%

  Since the Gaussian is invariant under isometries and $\sfrac{1}{\rho^2} \ne 0$, $\forall \rho > 0$, we conclude that no neighborhood of a point in a sphere may be isometrically mapped into a plane.
\end{solution}

\begin{problem}[4.3.8]
  Compute the Christoffel symbols an open set of the plane
  \begin{enumerate}
    \item In Cartesian coordinates.
    \item In polar coordinates.
  \end{enumerate}
  Use the Gauss formula to compute $K$ in both cases.
\end{problem}

\begin{solution}[(i)]
  Parametrizing the plane by Cartesian coordinates, we have
  \[%
    \x(u, v) = (u, v, au + bv + c)
  .\]%
  From exercise 4.3.4, the first fundamental form coefficients are
  \begin{gather*}
    E = 1, \quad F = 0, \aand G = 1 \\
    E_u = 0 = E_v \aand G_u = 0 = G_v
  \end{gather*}
  The Christoffel symbols are given by
  \[%
    \begin{pmatrix}
      1 & 0 \\
      0 & 1 \\
    \end{pmatrix}
    \begin{pmatrix}
      \Gamma^1_{11} \\
      \Gamma^2_{11} \\
    \end{pmatrix}
    = \begin{pmatrix}
      0 \\
      0 \\
    \end{pmatrix},\quad
    \begin{pmatrix}
      1 & 0 \\
      0 & 1 \\
    \end{pmatrix}
    \begin{pmatrix}
      \Gamma^1_{12} \\
      \Gamma^2_{12} \\
    \end{pmatrix}
    = \begin{pmatrix}
      0 \\
      0 \\
    \end{pmatrix},\aand
    \begin{pmatrix}
      1 & 0 \\
      0 & 1 \\
    \end{pmatrix}
    \begin{pmatrix}
      \Gamma^1_{22} \\
      \Gamma^2_{22} \\
    \end{pmatrix}
    = \begin{pmatrix}
      0 \\
      0 \\
    \end{pmatrix}
  .\]%
  Clearly, all the Christoffel symbols are zero, which imply that $K = 0$.
\end{solution}

\begin{solution}[(ii)]
  Parametrizing the plane by polar coordinates, we have
  \[%
    \x(u, v) = (u\cos(v), u\sin(v), au\cos(v) + bu\sin(v) + c)
  .\]%
  The first fundamental form coefficients are
  \begin{gather*}
    E = u^2, \quad F = 0, \aand G = 1 \\
    E_u = 2u, \quad E_v = 0, \aand G_u = 0 = G_v
  .\end{gather*}
  The Christoffel symbols are given by
  \[%
    \begin{pmatrix}
      u^2 & 0 \\
      0 & 1 \\
    \end{pmatrix}
    \begin{pmatrix}
      \Gamma^1_{11} \\
      \Gamma^2_{11} \\
    \end{pmatrix}
    = \begin{pmatrix}
      u \\
      0 \\
    \end{pmatrix},\quad
    \begin{pmatrix}
      u^2 & 0 \\
      0 & 1 \\
    \end{pmatrix}
    \begin{pmatrix}
      \Gamma^1_{12} \\
      \Gamma^2_{12} \\
    \end{pmatrix}
    = \begin{pmatrix}
      0 \\
      0 \\
    \end{pmatrix},\aand
    \begin{pmatrix}
      u^2 & 0 \\
      0 & 1 \\
    \end{pmatrix}
    \begin{pmatrix}
      \Gamma^1_{22} \\
      \Gamma^2_{22} \\
    \end{pmatrix}
    = \begin{pmatrix}
      0 \\
      0 \\
    \end{pmatrix}
  .\]%
  Therefore, we get
  \[%
    \Gamma^1_{11} = \frac{1}{u},\quad \Gamma^2_{11} = 0,\quad \Gamma^1_{12} = 0,\quad \Gamma^2_{12} = 0,\quad \Gamma^1_{22} = 0,\quad \Gamma^2_{22} = 0
  .\]%
  Using the Gauss formula from exercise 4.3.1, we have
  \begin{align*}
    K &= -\frac{1}{2\sqrt{EG}}\left[\left(\frac{E_v}{\sqrt{EG}}\right)_v + \left(\frac{G_u}{\sqrt{GE}}\right)_u\right] = -\frac{1}{2u}\left[\left(\frac{0}{u}\right)_v + \left(\frac{0}{u}\right)_u\right] \\
      &= -\frac{1}{2u}\left[\left(\frac{0}{u}\right)_v + \left(\frac{0}{u}\right)_u\right] = 0
  .\qedhere\end{align*}
\end{solution}

\begin{problem}[4.4.1]\leavevmode
  \begin{enumerate}
    \item Show that if a curve $C \subset S$ is both a line of curvature and a geodesic, then $C$ is a plane curve.

    \item Show that if a (nonrectilinear) geodesic is a plane curve, then it is a line of curvature.

    \item Give an example of a line of curvature which is a plane curve and not a geodesic.
  \end{enumerate}
\end{problem}

\begin{solution}[(i)]
\end{solution}

\begin{solution}[(ii)]
\end{solution}

\begin{solution}{(iii)}
\end{solution}

\begin{problem}[4.4.2]
  Prove that a curve $C \subset S$ is both an asymptotic curve and a geodesic if and only if $C$ is a (segment of a) straight line.
\end{problem}

\begin{solution}
\end{solution}

\begin{problem}[4.4.3]
  Show, without using Prop. 5, that the straight lines are the only geodesics of a plane.
\end{problem}

\begin{solution}
\end{solution}

\begin{problem}[4.4.4]
  Let $u$ and $w$ be vector fields along a curve $\alpha : I \to S$. Prove that
  \[%
    \odv{}{t} \bra{v(t), w(t)} = \bra{\codv{v}{t}, w(t)} + \bra{v(t), \codv{w}{t}}
  .\]%
\end{problem}

\begin{solution}
\end{solution}
