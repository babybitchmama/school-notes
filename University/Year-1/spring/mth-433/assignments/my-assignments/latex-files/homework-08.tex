\begin{problem}[4.3.1]
  Show that if $\x$ is an orthogonal parametrization, that is, $F = 0$, then
  \[%
    K = -\frac{1}{2 \sqrt{EG}}\left[\left(\frac{E_v}{\sqrt{EG}}\right)_v + \left(\frac{G_u}{\sqrt{GE}}\right)_u\right]
  .\]%
\end{problem}

\begin{solution}
  Using the given equations in Do Carmo, we see that
  \[%
    \Gamma^1_{11} = \frac{E_u}{2E}, \quad \Gamma^2_{11} = -\frac{E_v}{2G}, \quad \Gamma^1_{12} = \frac{E_v}{2E}, \quad \Gamma^2_{12} = \frac{G_u}{2G}, \quad \Gamma^1_{22} = -\frac{G_u}{2E}, \aand \Gamma^2_{22} = \frac{G_v}{2G}
  .\]%
  We already know that
  \[%
    \left(\Gamma^2_{12}\right)_u - \left(\Gamma^2_{11}\right)_v + \Gamma^1_{12} \Gamma^2_{11} + \Gamma^2_{12} \Gamma^2_{12} - \Gamma^2_{11} \Gamma^2_{22} - \Gamma^1_{11} \Gamma^2_{12} = -EK
  .\]%
  Therefore, computing the partials of $\Gamma^2_{12}$ and $\Gamma^2_{11}$, we have
  \[%
    \left(\Gamma^2_{12}\right)_u = \left(\frac{G_u}{2G}\right)_u = \frac{GG_{uu} - G^2_u}{2G^2} \aand \left(\Gamma^2_{11}\right)_v = \left(\frac{E_v}{2G}\right)_v = \frac{GE_{vv} - G_vE_v}{2G^2}
  .\]%
  Now we combine like terms to get
  \begin{align*}
    K &= \frac{G^2_u - GG_{uu}}{2EG^2} + \frac{G_vE_v - GE_{vv}}{2EG^2} + \frac{E^2_v}{4E^2G} - \frac{G^2_u}{4EG^2} - \frac{E_vG_v}{4EG^2} + \frac{E_uG_u}{4E^2G} \\
      &= \frac{-GG_{uu} + G_vE_v - GE_{vv}}{2EG^2} + \frac{G^2_u}{2EG^2} - \frac{G^2_u}{4EG^2} - \frac{E_vG_v}{4EG^2} + \frac{E^2_v}{4E^2G} + \frac{E_uG_u}{4E^2G} \\
      &= \frac{-GG_{uu} - GE_{vv}}{2EG^2} + \frac{G_vE_v}{2EG^2} - \frac{E_vG_v}{4EG^2} + \frac{G^2_u}{4EG^2} + \frac{E^2_v}{4E^2G} + \frac{E_uG_u}{4E^2G}
  .\end{align*}
  Now notice that
  \[%
    \left(\frac{G_u}{\sqrt{EG}}\right)_u = \frac{G_{uu}}{\sqrt{EG}} - \frac{G_u}{2(EG)^{3/2}}(E_uG + EG_u)
  \quad \aand \quad
    \left(\frac{E_v}{\sqrt{EG}}\right)_v = \frac{E_{vv}}{\sqrt{EG}} - \frac{E_v}{2(EG)^{3/2}}(E_vG + EG_v)
  .\]%
  Then combining these results, we have
  \begin{align*}
    \left(\frac{E_v}{\sqrt{EG}}\right)_v + \left(\frac{G_u}{\sqrt{EG}}\right)_u
    &= \frac{E_{vv} + G_{uu}}{\sqrt{EG}} - \frac{1}{2(EG)^{3/2}}\left[E_v(E_vG + EG_v) + G_u(E_uG + EG_u)\right] \\
    &= \frac{E_{vv} + G_{uu}}{\sqrt{EG}} - \frac{1}{2(EG)^{3/2}}\left(E_v^2G + E_vEG_v + G_uE_uG + G_uEG_u\right)
  .\end{align*}
  Therefore,
  \[%
    K = -\frac{1}{2\sqrt{EG}}\left[\left(\frac{E_v}{\sqrt{EG}}\right)_v + \left(\frac{G_u}{\sqrt{EG}}\right)_u\right]
  .\qedhere\]%
\end{solution}

\begin{problem}[4.3.2]
  Show that if $\x$ is an isothermal parametrization, that is, $E = G = \lambda(u, v)$ and $F = 0$, then
  \[%
    K = -\frac{1}{2\lambda} \Delta (\log(\lambda))
  ,\]%
  where $\Delta \phi$ denotes the Laplacian $\left(\pdv[2]{\phi}/{u}\right) + \left(\pdv[2]{\phi}/{v}\right)$ of the function $\phi$. Conclude that when $E = G = (u^2 + v^2 + c)^{-2}$ and $F = 0$, then $K =$ const. $= 4c$.
\end{problem}

\begin{solution}
  Using the equation proven in exercise 4.3.1, we have
  \begin{align*}
    K &= -\frac{1}{2\sqrt{\lambda\lambda}}\left(\left(\frac{\lambda_v}{\sqrt{\lambda\lambda}}\right)_v + \left(\frac{\lambda_u}{\sqrt{\lambda\lambda}}\right)_u\right) \\
      &= -\frac{1}{2\lambda}\left(\left(\frac{\lambda_v}{\lambda}\right)_v + \left(\frac{\lambda_u}{\lambda}\right)_u\right) \\
      &= -\frac{1}{2\lambda}\left(\left(\ln_v(\lambda)\right)_v + \left(\ln_u(\lambda)\right)_u\right) \\
      &= -\frac{1}{2\lambda}\left(\pdv[2]{\ln(\lambda)}{u} + \pdv[2]{\ln(\lambda)}{v}\right) \\
      &= -\frac{1}{2\lambda} \Delta (\log(\lambda))
  .\qedhere\end{align*}
\end{solution}

\begin{problem}[4.3.4]
  Show that no neighborhood of a point in a sphere may be isometrically mapped into a plane.
\end{problem}

\begin{solution}
  Parameterizing a sphere by spherical coordinates, we have
  \[%
    \x(u, v) = (\rho\sin(u)\cos(v), \rho\sin(u)\sin(v), \rho\cos(u))
  ,\]%
  where $\rho$ is the radius of the sphere, $u \in [0, \pi]$ is the polar angle, and $v \in [0, 2\pi)$ is the azimuthal angle. Computing the partial derivatives, we have
  \[%
    \x_u = (\rho\cos(u)\cos(v), \rho\cos(u)\sin(v), -\rho\sin(u))
    \aand
    \x_v = (-\rho\sin(u)\sin(v), \rho\sin(u)\cos(v), 0)
  .\]%
  The first fundamental form coefficients are
  \[%
    E = \bra{\x_u, \x_u} = \rho^2,
    \quad
    F = \bra{\x_u, \x_v} = 0,
    \aand
    G = \bra{\x_v, \x_v} = \rho^2\sin^2(u)
  .\]%
  Therefore, using the Gauss formula from exercise 4.3.1, we have
  \begin{align*}
    K &= -\frac{1}{2\sqrt{EG}}\left[\left(\frac{E_v}{\sqrt{EG}}\right)_v + \left(\frac{G_u}{\sqrt{GE}}\right)_u\right] \\
      &= -\frac{1}{2\rho^2\sin(u)}\left[\left(\frac{0}{\rho\sin(u)}\right)_v + \left(\frac{\rho^2\cos(u)}{\rho^2}\right)_u\right] \\
      &= -\frac{-\sin(u)}{2\rho^2\sin(u)} \\
      &= \frac{1}{\rho^2}
  .\end{align*}

  Parametrizing a plane by Cartesian coordinates, we have
  \[%
    \bar{\x}(u, v) = (u, v, au + bv + c)
  .\]%
  The partial derivatives are
  \[%
    \bar{\x}_u = (1, 0, a)
    \aand
    \bar{\x}_v = (0, 1, b)
  .\]%
  The first fundamental form coefficients are
  \[%
    \bar{E} = \bra{\bar{\x}_u, \bar{\x}_u} = 1,
    \quad
    \bar{F} = \bra{\bar{\x}_u, \bar{\x}_v} = 0,
    \aand
    \bar{G} = \bra{\bar{\x}_v, \bar{\x}_v} = 1
  .\]%
  Therefore, using the Gauss formula from exercise 4.3.1, we have
  \[%
    \bar{K} = -\frac{1}{2\sqrt{\bar{E}\bar{G}}}\left[\left(\frac{\bar{E}_v}{\sqrt{\bar{E}\bar{G}}}\right)_v + \left(\frac{\bar{G}_u}{\sqrt{\bar{G}\bar{E}}}\right)_u\right] = -\frac{1}{2}\left[\left(\frac{0}{1}\right)_v + \left(\frac{0}{1}\right)_u\right] = 0
  .\]%

  Since the Gaussian is invariant under isometries and $\sfrac{1}{\rho^2} \ne 0$, $\forall \rho > 0$, we conclude that no neighborhood of a point in a sphere may be isometrically mapped into a plane.
\end{solution}

\begin{problem}[4.3.8]
  Compute the Christoffel symbols an open set of the plane
  \begin{enumerate}
    \item In Cartesian coordinates.
    \item In polar coordinates.
  \end{enumerate}
  Use the Gauss formula to compute $K$ in both cases.
\end{problem}

\begin{solution}[(i)]
  Parametrizing the plane by Cartesian coordinates, we have
  \[%
    \x(u, v) = (u, v, au + bv + c)
  .\]%
  From exercise 4.3.4, the first fundamental form coefficients are
  \begin{gather*}
    E = 1, \quad F = 0, \aand G = 1 \\
    E_u = 0 = E_v \aand G_u = 0 = G_v
  \end{gather*}
  The Christoffel symbols are given by
  \[%
    \begin{pmatrix}
      1 & 0 \\
      0 & 1 \\
    \end{pmatrix}
    \begin{pmatrix}
      \Gamma^1_{11} \\
      \Gamma^2_{11} \\
    \end{pmatrix}
    = \begin{pmatrix}
      0 \\
      0 \\
    \end{pmatrix},\quad
    \begin{pmatrix}
      1 & 0 \\
      0 & 1 \\
    \end{pmatrix}
    \begin{pmatrix}
      \Gamma^1_{12} \\
      \Gamma^2_{12} \\
    \end{pmatrix}
    = \begin{pmatrix}
      0 \\
      0 \\
    \end{pmatrix},\aand
    \begin{pmatrix}
      1 & 0 \\
      0 & 1 \\
    \end{pmatrix}
    \begin{pmatrix}
      \Gamma^1_{22} \\
      \Gamma^2_{22} \\
    \end{pmatrix}
    = \begin{pmatrix}
      0 \\
      0 \\
    \end{pmatrix}
  .\]%
  Clearly, all the Christoffel symbols are zero, which imply that $K = 0$.
\end{solution}

\begin{solution}[(ii)]
  Parametrizing the plane by polar coordinates, we have
  \[%
    \x(u, v) = (u\cos(v), u\sin(v), au\cos(v) + bu\sin(v) + c)
  .\]%
  The first fundamental form coefficients are
  \begin{gather*}
    E = u^2, \quad F = 0, \aand G = 1 \\
    E_u = 2u, \quad E_v = 0, \aand G_u = 0 = G_v
  .\end{gather*}
  The Christoffel symbols are given by
  \[%
    \begin{pmatrix}
      u^2 & 0 \\
      0 & 1 \\
    \end{pmatrix}
    \begin{pmatrix}
      \Gamma^1_{11} \\
      \Gamma^2_{11} \\
    \end{pmatrix}
    = \begin{pmatrix}
      u \\
      0 \\
    \end{pmatrix},\quad
    \begin{pmatrix}
      u^2 & 0 \\
      0 & 1 \\
    \end{pmatrix}
    \begin{pmatrix}
      \Gamma^1_{12} \\
      \Gamma^2_{12} \\
    \end{pmatrix}
    = \begin{pmatrix}
      0 \\
      0 \\
    \end{pmatrix},\aand
    \begin{pmatrix}
      u^2 & 0 \\
      0 & 1 \\
    \end{pmatrix}
    \begin{pmatrix}
      \Gamma^1_{22} \\
      \Gamma^2_{22} \\
    \end{pmatrix}
    = \begin{pmatrix}
      0 \\
      0 \\
    \end{pmatrix}
  .\]%
  Therefore, we get
  \[%
    \Gamma^1_{11} = \frac{1}{u},\quad \Gamma^2_{11} = 0,\quad \Gamma^1_{12} = 0,\quad \Gamma^2_{12} = 0,\quad \Gamma^1_{22} = 0,\quad \Gamma^2_{22} = 0
  .\]%
  Using the Gauss formula from exercise 4.3.1, we have
  \begin{align*}
    K &= -\frac{1}{2\sqrt{EG}}\left[\left(\frac{E_v}{\sqrt{EG}}\right)_v + \left(\frac{G_u}{\sqrt{GE}}\right)_u\right] = -\frac{1}{2u}\left[\left(\frac{0}{u}\right)_v + \left(\frac{0}{u}\right)_u\right] \\
      &= -\frac{1}{2u}\left[\left(\frac{0}{u}\right)_v + \left(\frac{0}{u}\right)_u\right] = 0
  .\qedhere\end{align*}
\end{solution}

\begin{problem}[4.4.1]
  \begin{enumerate}
    \item Show that if a curve $C \subset S$ is both a line of curvature and a geodesic, then $C$ is a plane curve.

    \item Show that if a (nonrectilinear) geodesic is a plane curve, then it is a line of curvature.

    \item Give an example of a line of curvature which is a plane curve and not a geodesic.
  \end{enumerate}
\end{problem}

\begin{solution}[(i)]
  Let $\alpha$ be the curve parametrized by arc length. Since $\alpha$ is geodesic, we have $k_g = 0$. Therefore,
  \[%
    k^2 = k_g^2 + k_n^2 \implies k^2 = k_n^2 \implies k = k_n = k \bra{n, \Na} \implies \bra{n, \Na} = 1 \implies n = \Na
  ,\]%
  where $n$ is the principal normal to the curve in $\R^3$ and $\Na$ is the unit normal to the surface.

  Since $\alpha$ is also a line of curvature, we have $\dd{\Na}(T) = \lambda T$ for some principal curvature $\lambda$. But we also have $n = \Na$, so
  \[%
    \odv{n}{s} = \lambda T
  .\]%
  This implies that the derivative of the principal normal vector lies in the direction of the tangent vector. Hence, the binormal vector $B = T \times n$ is constant. Since the osculating plane is spanned by $T$ and $n$, it remains fixed. Therefore, $\alpha$ lies entirely in a fixed plane, and is a plane curve.
\end{solution}

\begin{solution}[(ii)]
  Let $\alpha$ be a nonrectilinear geodesic parametrized by arc length. Since $\alpha$ is a geodesic, we have $k_g = 0$, and therefore the total curvature satisfies $n = \Na$, as we showed in the previous problem. Thus, the principal normal vector $n$ of the curve agrees with the surface normal vector $\Na$ along $\alpha$.

  Now suppose further that $\alpha$ lies in a plane $P \subset \R^3$. Since the binormal vector $B = T \times n$ is orthogonal to both $T$ and $n = \Na$, we have that $B$ is constant and perpendicular to the plane $P$.

  Because $\Na$ coincides with the principal normal $n$, which is perpendicular to the fixed binormal $B$, it follows that the surface normal vector $\Na$ stays in the same direction as the curve moves — i.e., the surface bends uniformly in the direction of the curve's tangent vector $T$. 

  Differentiating $\Na$ along $\alpha$,
  \[%
    \odv{\Na}{s} = \odv{n}{s} = -kT + \tau B
  .\]%
  But since $B$ is constant and the curve is planar, we must have $\tau = 0$. So
  \[%
    \odv{\Na}{s} = -k T
  .\]%
  This implies that the shape operator $S$ satisfies
  \[%
    \dd{\Na}(T) = -k T
  ,\]%
  i.e., $T$ is an eigenvector of the shape operator. Hence, the curve is a line of curvature.
\end{solution}

\begin{solution}{(iii)}
  Consider the parallel curves on a surface of revolution, such as a circle of latitude on a sphere (excluding the equator). Let $S$ be the unit sphere in $\R^3$, and let $\alpha(s)$ be the circle of latitude defined by
  \[%
    \alpha(s) = (\cos(s) \sin(\theta_0), \sin(s)\sin(\theta_0), \cos(\theta_0))
  ,\]%
  where $\theta_0 \in (0, \pi) \setminus \{\sfrac{\pi}{2}\}$ is fixed. Then $\alpha$ lies in the plane $z = \cos(\theta_0)$ and is clearly a plane curve.

  On a surface of revolution, parallels are always lines of curvature. However, $\alpha$ is not a geodesic unless $\theta_0 = \sfrac{\pi}{2}$, in which case the parallel is a great circle (i.e., a geodesic). Since $\theta_0 \ne \sfrac{\pi}{2}$, $\alpha$ is not a geodesic -- the geodesic curvature $k_g$ is nonzero.

  Therefore, $\alpha$ is a line of curvature, is planar, but not a geodesic.
\end{solution}

\begin{problem}[4.4.2]
  Prove that a curve $C \subset S$ is both an asymptotic curve and a geodesic if and only if $C$ is a (segment of a) straight line.
\end{problem}

\begin{solution}
  Let $\alpha(s)$ be a regular curve on $S$ parametrized by arc length. Suppose $\alpha$ is both a geodesic and an asymptotic curve. Since it is a geodesic, we have
  \[%
    k_g = 0 \implies k^2 = k_n^2
  ,\]%
  and since it is asymptotic, we have
  \[%
    k_n = \bra{n, \Na} = 0 \implies k = 0
  .\]%
  Thus, the total curvature $k = 0$, which implies that $\alpha''(s) = 0$, so $\alpha$ is a straight line in $\R^3$.

  Conversely, suppose that $\alpha(s)$ is a straight line in $\R^3$ lying on the surface $S$. Then $\alpha''(s) = 0 \implies k = 0$. In particular, both the normal curvature $k_n = k \bra{n, \Na} = 0$ and the geodesic curvature $k_g = 0$ vanish. Therefore, $\alpha$ is both an asymptotic curve and a geodesic, and hence a straight line.

  Thus, a curve $C \subset S$ is both an asymptotic curve and a geodesic if and only if $C$ is a (segment of a) straight line.
\end{solution}

\begin{problem}[4.4.3]
  Show, without using Prop. 5, that the straight lines are the only geodesics of a plane.
\end{problem}

\begin{solution}
  Since the surface is a plane, its normal vector $\Na$ is constant and perpendicular to the plane. For any curve $\alpha$ lying in the plane, the principal normal vector $n$ lies in the plane itself. Therefore,
  \[%
    \bra{n, \Na} = 0
  ,\]%
  for all points on $\alpha$. Consequently, the normal curvature of any curve on the plane satisfies
  \[%
    k_n = k \bra{n, \Na} = 0
  ,\]%
  so every curve in the plane is an asymptotic curve.

  Recall that the curvature $k$ of the curve satisfies
  \[%
    k^2 = k_g^2 + k_n^2 = k_g^2
  ,\]%
  since $k_n = 0$. Thus, the geodesic curvature $k_g$ equals the total curvature $k$ of the curve. Geodesics have zero geodesic curvature, so $k_g = 0$, which implies $k = 0$. Hence, the geodesics on the plane are precisely the curves with zero curvature -- that is, straight lines. Therefore, the straight lines are the only geodesics in a plane.
\end{solution}

\begin{problem}[4.4.4]
  Let $u$ and $w$ be vector fields along a curve $\alpha : I \to S$. Prove that
  \[%
    \odv{}{t} \bra{v(t), w(t)} = \bra{\codv{v}{t}, w(t)} + \bra{v(t), \codv{w}{t}}
  .\]%
\end{problem}

\begin{solution}
  The derivative of the inner product satisfies the product rule, i.e.,
  \[%
    \odv{}{t} \bra{v, w} = \bra{\odv{v}{t}, w} + \bra{v, \odv{w}{t}}
  .\]%
  However, the ordinary derivatives $\odv{v}/{t}$ and $\odv{w}/{t}$ may not lie in the tangent plane. We can decompose them as
  \[%
    \odv{v}{t} = \codv{v}{t} + v_n \Na \aand \odv{w}{t} = \codv{w}{t} + w_n \Na
  ,\]%
  where $v_n$ and $w_n$ are scalar functions and $\Na$ is the unit normal vector to the surface.

  Since $v$ and $w$ are tangent vector fields, we have $\bra{\Na, v} = \bra{\Na, w} = 0$. Using the bilinearity of the inner product,
  \begin{align*}
    \odv{}{t} \bra{v, w} &= \bra{\codv{v}{t} + v_n \Na, w} + \bra{v, \codv{w}{t} + w_n \Na} \\
                         &= \bra{\codv{v}{t}, w} + v_n \bra{\Na, w} + \bra{v, \codv{w}{t}} + w_n \bra{v, \Na} \\
                         &= \bra{\codv{v}{t}, w} + \bra{v, \codv{w}{t}}
  .\qedhere\end{align*}
\end{solution}
