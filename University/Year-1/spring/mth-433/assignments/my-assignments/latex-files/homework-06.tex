\begin{problem}[3.3.1]
  Show that at the origin $(0, 0, 0)$ of the hyperboloid $z = axy$ we have $K = -a^2$ and $H = 0$.
\end{problem}

\begin{solution}
  Let $h(x, y) = z = axy$. We compute the necessary partial derivatives to evaluate the Gaussian curvature $K$ and the mean curvature $H$ at the origin. Computing the first-order partial derivatives, we have
  \[%
    h_x = ay \aand h_y = ax
  .\]%
  Evaluated at the origin $(0, 0)$, we have
  \[%
    h_x(0, 0) = 0 \aand h_y(0, 0) = 0
  .\]%
  Next, we compute the second-order partial derivatives
  \[%
    h_{xx} = 0, \quad h_{yy} = 0, \aand h_{xy} = a
  .\]%
  Now, we can compute the Gaussian curvature $K$ at the origin, to get
  \[%
    K = \frac{h_{xx} h_{yy} - h_{xy}^2}{(1 + h_x^2 + h_y^2)^2} = \frac{0 \cdot 0 - a^2}{(1 + 0 + 0)^2} = \frac{-a^2}{1} = -a^2
  .\]%
  Lastly, we compute the mean curvature $H$ at the origin, to get
  \begin{align*}
    H &= \frac{\left(1 + h_y^2\right) h_{xx} - 2 h_x h_y h_{xy} + \left(1 + h_x^2\right) h_{yy}}{2\left(1 + h_x^2 + h_y^2\right)^{\sfrac{3}{2}}} \\
      &= \frac{(1 + 0) \cdot 0 - 2 \cdot 0 \cdot 0 \cdot a + (1 + 0) \cdot 0}{2 \cdot (1 + 0 + 0)^{\sfrac{3}{2}}} \\
      &= \frac{0}{2} = 0
  .\qedhere\end{align*}
\end{solution}

\begin{problem}[3.3.3]
  Determine the asymptotic curves of the catenoid
  \[%
    \x(u, v) = (\cosh(v)\cos(u), \cosh(v)\sin(u), v)
  .\]%
\end{problem}

\begin{solution}
\end{solution}

\begin{problem}[3.3.5]
  Consider the parametrized surface (Enneper's surface)
  \[%
    \x(u, v) = \left(u - \frac{u^3}{3} + uv^2, v - \frac{v^3}{3} + vu^2, u^2 - v^2\right)
  ,\]%
  and show that
  \begin{enumerate}
    \item The coefficients of the first fundamental form are
      \[%
        E = G = (1 + u^2 + v^2)^2, \quad F = 0
      .\]%

    \item The coefficients of the second fundamental form are
      \[%
        e = 2, \quad g = -2, \quad f = 0
      .\]%

    \item The principal curvatures are
      \[%
        \kappa_1 = \frac{2}{(1 + u^2 + v^2)^2}, \quad \kappa_2 = -\frac{2}{(1 + u^2 + v^2)^2}
      .\]%

    \item The lines of the curvature are the coordinate curves.

    \item The asymptotic curves are $u + v = \text{const.}$, $u - v = \text{const.}$
  \end{enumerate}
\end{problem}

\begin{solution}[(i)]
\end{solution}

\begin{solution}[(ii)]
\end{solution}

\begin{solution}[(iii)]
\end{solution}

\begin{solution}[(iv)]
\end{solution}

\begin{solution}[(v)]
\end{solution}

\begin{problem}[3.3.7]
  $(\phi(v)\cos(u), \phi(v)\sin(u), \psi(v))$, $\phi \ne 0$ is given as a surface of revolution with constant Gaussian curvature $K$. To determine the functions $\phi$ and $\psi$, choose the parameter $v$ in such a way that $(\phi')^2 + (\psi')^2 = 1$ (geometrically, this means that $v$ is the arc length of the generating curve $(\phi(v), \psi(v))$). Show that
  \begin{enumerate}
    \item $\phi$ satisfies $\phi'' + K\phi = 0$ and $\psi$ is given by $\psi = \int \sqrt{1 - (\phi')^2} \dv$; thus, $0 < u < 2\pi$, and the domain of $v$ is such that the last integral makes sense.

    \item All surfaces of revolution with constant curvature $K = 1$ which intersect perpendicularly the plane $xOy$ are given by
      \[%
        \phi(v) = C\cos(v), \quad \psi(v) = \int_0^v \sqrt{1 - C^2\sin^2(v)} \dv
      ,\]%
      where $C$ is a constant ($C = \phi(0)$). Determine the domain of $v$ and draw a rough sketch of the profile of the surface in the $xz$-plane for the cases $C = 1$, $C > 1$, $C < 1$. Observe that $C = 1$ gives a sphere.

    \item All surfaces of revolution with constant curvature $K = -1$ may be given by one of the following types:
      \begin{itemize}
        \item[1.] $\phi(v) = C\cosh(v)$,

          $\psi(v) = \int_0^v \sqrt{1 - C^2\sinh^2(v)} \dv$.

        \item[2.] $\phi(v) = C\sinh(v)$,

          $\psi(v) = \int_0^v \sqrt{1 - C^2\cosh^2(v)} \dv$.

        \item[3.] $\phi(v) = e^v$,

          $\psi(v) = \int_0^v \sqrt{1 - e^{2v}} \dv$.
      \end{itemize}
      Determine the domain of $v$ and draw a rough sketch of the profile of the surface in the $xz$-plane.

    \item The surface of type 3 in part (iii) is the pseudosphere of Exercise 6.

    \item The only surfaces of revolution with $K \equiv 0$ are the right circular cylinder, the right circular cone, and the plane.
  \end{enumerate}
\end{problem}

\begin{solution}[(i)]
\end{solution}

\begin{solution}[(ii)]
\end{solution}

\begin{solution}[(iii)]
\end{solution}

\begin{solution}[(iv)]
\end{solution}

\begin{solution}[(v)]
\end{solution}

\begin{problem}[3.3.22]
  Let $h : S \to \R$ be a differentiable function on a surface $S$, and let $p \in S$ be a critical point of $h$ (i.e., $\dd{h}_p = 0$). Let $w \in T_p(S)$ and let
  \[%
    \alpha : (-\epsilon, \epsilon) \to S
  ,\]%
  be a parametrized curve with $\alpha(0) = p$, $\alpha'(0) = \w$. Set
  \[%
    H_ph(\w) = \odv[order={2}]{(h \circ \alpha)}{t}\bigg\vert_{t = 0}
  .\]%
  \begin{enumerate}
    \item Let $\x : U \to S$ be a parametrization of $S$ at $p$, and show that (the fact that $p$ is a critical point of $h$ is essential here)
      \[%
        H_ph(u'\x_u + v'\x_v) = h_{uu}(p)(u')^2 + 2h_{uv}(p)u'v' + h_{vv}(p)(v')^2
      .\]%
      Conclude that $H_ph : T_p(S) \to \R$ is a well-defined (i.e., it does not depend on the choice of $\x$) quadratic form on $T_p(S)$. $H_ph$ is called the \emph{Hessian} of $h$ at $p$.

    \item Let $h : S \to \R$ be the height function of $S$ relative to $T_p(S)$; that is, $h(q) = \langle q - p, \Na(p) \rangle$, $q \in S$. Verify that $p$ is a critical point of $h$ and thus that the Hessian $H_ph$ is well defined. Show that if $\w \in T_p(S)$, $\abs{\w} = 1$, then
      \[%
        H_ph(\w) = \text{normal curvature at $p$ in the direction of $\w$}
      .\]%
      Conclude that the Hessian at $p$ of the height function relative to $T_p(S)$ is the second fundamental form of $S$ at $p$.
  \end{enumerate}
\end{problem}

\begin{solution}[(i)]
\end{solution}

\begin{solution}[(ii)]
\end{solution}
