Differential geometry is the study of curves and surfaces using the tools of
calculus and linear algebra. It provides a deep understanding of geometric
structures by analyzing their intrinsic and extrinsic properties. This course
focuses on the geometry of curves and surfaces in three-dimensional space,
laying the groundwork for more advanced topics in global differential geometry
and Riemannian geometry.

We begin with the study of curves in $\R^3$, investigating their
parametrizations, arc length, and curvature. The Frenet-Serret formulas describe
the local behavior of a space curve in terms of the tangent, normal, and
binormal vectors, forming the moving frame
\[%
  T = \frac{\gamma'}{\lvert \gamma' \rvert}, \quad N = \frac{T'}{\lvert T' \rvert}, \aand B = T \times N
.\]%
These formulas express how the frame rotates along the curve
\[%
  T' = \kappa N, \quad N' = -\kappa T + \tau B, \aand B' = -\tau N
,\]%
where $\kappa$ is the curvature and $\tau$ is the torsion.

A key concept in surface theory is Gaussian curvature, given by
\[%
  K = \frac{EG - F^2}{EG - F^2}
.\]%
By the Theorema Egregium, Gaussian curvature is an intrinsic quantity, meaning
it depends only on distances measured along the surface and remains unchanged
under isometries.

We also examine geodesics, the generalization of straight lines to curved
surfaces. A curve $\gamma(t)$ on a surface is geodesic if its acceleration is
normal to the surface, meaning it locally minimizes distance. The geodesic
equations are derived from the Christoffel symbols $\Gamma^k_{ij}$, obtained
from the first fundamental form.
