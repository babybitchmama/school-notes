\begin{problem}[1.2.11]
  Solve the equation $z^2 + z + 1 = 0$ for $z = (x, y)$ by writing
  \[%
    (x, y)(x, y) + (x, y) + (1, 0) = (0, 0)
  ,\]%
  and then solving a pair of simultaneous equations in $x$ and $y$.

  \textit{Suggestion:} Use the fact that no real number $x$ satisfies the given
  equation to show that $y \ne 0$.

  Ans. $\displaystyle z = \left(-\frac{1}{2}, \pm \frac{\sqrt{3}}{2}\right)$.
\end{problem}

\begin{proof}[Solution]
  Expanding the left-hand side and simplifying, we get
  \begin{alignat*}{3}
    z^2 + z + 1 = 0 &\implies (x^2 - y^2, 2xy) + (x, y) + (1, 0) &&= (0, 0) \\
                    &\implies (x^2 - y^2 + x + 1, 2xy + y) &&= (0, 0)
  .\end{alignat*}
  Therefore, we have the system of equations
  \begin{align*}
    x^2 - y^2 + x + 1 &= 0 \\
    2xy + y &= 0
  .\end{align*}
  From the second equation, we can factor out $y$ to get $y(2x + 1) = 0$. This
  gives us two cases to consider: either $y = 0$ or $2x + 1 = 0$.

  If $y = 0$, then substituting it back into the first equation gives us
  \[%
    x^2 + x + 1 = 0
  ,\]%
  which doesn't have any real solutions, as the discriminant is negative.
  Therefore, we must have $y \ne 0$.

  In the case where $2x + 1 = 0$, we can solve for $x$ to get $x =
  -\sfrac{1}{2}$. Substituting this value into the first equation gives us
  \begin{align*}
    0 &= \left(-\frac{1}{2}\right)^2 - y^2 - \frac{1}{2} + 1 \\
    0 &= \frac{1}{4} - y^2 - \frac{1}{2} + 1 \\
    0 &= -y^2 + \frac{3}{4} \\
    y^2 &= \frac{3}{4} \\
    y &= \pm\frac{\sqrt{3}}{2}
  .\end{align*}

  Thus, the solutions to the equation $z^2 + z + 1 = 0$ are
  \[%
    z = \left(-\frac{1}{2}, \pm \frac{\sqrt{3}}{2}\right) = -\frac{1}{2} \pm\frac{\sqrt{3}}{2}i
  .\qedhere\]%
\end{proof}

\begin{problem}[1.3.1]
  Reduce each of these quantities to a real number
  \[%
    \textrm{(i)}~\frac{1 + 2i}{3 - 4i} + \frac{2 - i}{5i}; \quad\quad \textrm{(ii)}~\frac{5i}{(1 - i)(2 - i)(3 - i)}; \quad\quad \textrm{(iii)}~(1 - i)^4
  .\]%
  Ans. (i) $-\sfrac{2}{5}$, (ii) $-\sfrac{1}{2}$, (iii) $-4$.
\end{problem}

\begin{proof}[Solution to (i)]
  Multiplying the numerator and denominator of the first term by the conjugate
  of the denominator, we have
  \[%
    \frac{1 + 2i}{3 - 4i} = \frac{(1 + 2i)(3 + 4i)}{(3 - 4i)(3 + 4i)} = \frac{3 + 4i + 6i - 8}{9 + 16} = \frac{-5 + 10i}{25} = -\frac{1}{5} + \frac{2}{5} i
  .\]%
  Now, we can simplify the second term
  \[%
    \frac{2 - i}{5i} = \frac{(2 - i)(-i)}{5i(-i)} = \frac{-2i - 1}{5} = -\frac{1}{5} - \frac{2}{5} i
  .\]%
  Adding these two results together, we have
  \[%
    \frac{1 + 2i}{3 - 4i} + \frac{2 - i}{5i} = \left(-\frac{1}{5} + \frac{2}{5} i\right) + \left(-\frac{1}{5} - \frac{2}{5} i\right) = -\frac{2}{5}
  .\qedhere\]%
\end{proof}

\begin{proof}[Solution to (ii)]
  Expanding the denominator, we have
  \[%
    \frac{5i}{(1 - i)(2 - i)(3 - i)} = \frac{5i}{(1 - 3i)(3 - i)} = -\frac{5i}{10i} = -\frac{1}{2}
  .\qedhere\]%
\end{proof}

\begin{proof}[Solution to (iii)]
  Expanding the polynomial $(1 - i)^2$, we have
  \[%
    (1 - i)^2 = (1 - i)(1 - i) = 1 - 2i + i^2 = 1 - 2i - 1 = -2i
  .\]%
  Now, we can expand $(1 - i)^4$ as follows
  \[%
    (1 - i)^4 = (1 - i)^2(1 - i)(1 - i) = -2i(1 - i)(1 - i) = (-2 - 2i)(1 - i) = (-2 + 2i - 2i - 2) = -4
  .\qedhere\]%
\end{proof}

\begin{problem}[1.4.1]
  Locate the numbers $z_1 + z_2$ and $z_1 - z_2$ vertically when
  \begin{alignat*}{6}
    &\textrm{(i)}~&&z_1 = 2i, \quad &&z_2 = \frac{2}{3} - i; \quad\quad&&\textrm{(ii)}~z_1 = (-\sqrt{3}, 1), \quad &&z_2 = (\sqrt{3}, 0); \\
    &\textrm{(iii)}~&&z_1 = (-3, 1), \quad &&z_2 = (1, 4); \quad\quad&&\textrm{(iv)}~z_1 = x_1 + y_1i, \quad &&z_2 = x_1 - y_1i
  .\end{alignat*}
\end{problem}

\begin{proof}[Solution to (i)]
\end{proof}

\begin{proof}[Solution to (ii)]
\end{proof}

\begin{proof}[Solution to (iii)]
\end{proof}

\begin{proof}[Solution to (iv)]
\end{proof}

\begin{problem}[1.4.4]
  Verify that $\sqrt{2}\lvert z \rvert \ge \lvert \Re(z) \rvert + \lvert \Im(z)
  \rvert$.

  \textit{Suggestion:} Reduce this inequality to $(\lvert x \rvert - \lvert y
  \rvert)^2 \ge 0$.
\end{problem}

\begin{proof}[Solution]
  We know that $\lvert \Re(z) \rvert = x$ and $\lvert \Im(z) \rvert = y$. We
  also know that $\lvert z \rvert = \sqrt{x^2 + y^2}$. Therefore, we can rewrite
  the inequality as
  \begin{align*}
    \phantom{\implies}\quad&2\lvert z \rvert \ge \lvert \Re(z) \rvert + \lvert \Im(z) \rvert \\
    \implies\quad&2\sqrt{x^2 + y^2} \ge \lvert x \rvert + \lvert y \rvert \\
    \implies\quad&2(x^2 + y^2) \ge (\lvert x \rvert + \lvert y \rvert)^2 \\
    \implies\quad&2(x^2 + y^2) \ge x^2 + 2\lvert x \rvert \lvert y \rvert + y^2 \\
    \implies\quad&2x^2 + 2y^2 - x^2 - 2\lvert x \rvert \lvert y \rvert - y^2 \ge 0 \\
    \implies\quad&(x^2 + y^2) - 2\lvert x \rvert \lvert y \rvert \ge 0 \\
    \implies\quad&(x - \lvert y \rvert)(x + \lvert y \rvert) \ge 0
  .\end{align*}
  Since $x^2 + y^2 = \lvert x \rvert^2 + \lvert y \rvert^2$, we can re-write the
  inequality as
  \begin{align*}
    \phantom{\implies}\quad&(x - \lvert y \rvert)(x + \lvert y \rvert) \ge 0 \\
    \implies\quad&(\lvert x \rvert - \lvert y \rvert)^2 \ge 0
  .\end{align*}
  Therefore, we have $(\lvert x \rvert - \lvert y \rvert)^2 \ge 0$, which is
  always true. This means that $\sqrt{2}\lvert z \rvert \ge \lvert \Re(z) \rvert
  + \lvert \Im(z) \rvert$ is true for all complex numbers $z$.
\end{proof}

\begin{problem}[1.4.6]
  Using the fact that $\lvert z_1 - z_2 \rvert$ is the distance between two
  points $z_1$ and $z_2$, give a geometric argument that
  \begin{enumerate}
    \item $\lvert z - 4i \rvert + \lvert z + 4i \rvert = 10$ represents an
      ellipse whose foci are $(0, \pm 4)$.

    \item $\lvert z - 1 \rvert = \lvert z + i \rvert$ represents the line
      through the origin whose slope is $-1$.
  \end{enumerate}
\end{problem}

\begin{proof}[Solution to (i)]
\end{proof}

\begin{proof}[Solution to (ii)]
\end{proof}

\begin{problem}[1.5.1(iv)]
  Use properties of conjugates and moduli established in Sec. $5$ to show that
  \begin{enumerate}
    \item[(iv)] $\lvert (2\zb + 5)(\sqrt{2} - i) \rvert = \sqrt{3} \lvert 2z + 5
      \rvert$.
  \end{enumerate}
\end{problem}

\begin{proof}[Solution to (iv)]
  Expanding the left-hand side, we have
  \[%
    \lvert (2\zb + 5)(\sqrt{2} - i) \rvert = \lvert 2\zb + 5 \rvert \cdot \lvert \sqrt{2} - i \rvert
  .\]%
  Taking the norm of the complex number $\sqrt{2} - i$, we have
  \[%
    \lvert \sqrt{2} - i \rvert = \sqrt{(\sqrt{2})^2 + (-1)^2} = \sqrt{2 + 1} = \sqrt{3}
  .\]%
  Therefore, we have
  \[%
    \lvert (2\zb + 5)(\sqrt{2} - i) \rvert = \sqrt{3} \cdot \lvert 2\zb + 5 \rvert
  .\]%
  Since $\lvert z \rvert = \lvert \zb \rvert$, we can replace $\lvert 2\zb + 5
  \rvert$ with $\lvert 2z + 5 \rvert$ to get
  \[%
    \lvert (2\zb + 5)(\sqrt{2} - i) \rvert = \sqrt{3} \lvert 2z + 5 \rvert
  .\qedhere\]%
\end{proof}

\begin{problem}[1.5.10]
  Prove that
  \begin{enumerate}
    \item $z$ is real if and only if $\zb = z$.

    \item $z$ is either real or pure imaginary if and only if $\zb^2 = z^2$.
  \end{enumerate}
\end{problem}

\begin{proof}[Solution to (i)]
  Assume $z$ is real. This means that $z = x + 0i$ for some real number $x$. The
  complex conjugate of $z$ is $\zb = x - 0i = x$. Therefore, we have $\zb = z$.

  Conversely, assume $\zb = z$. This means that $z = x + yi$ and $\zb = x - yi$.
  By assumption, we have $\zb = z$ giving us $x + yi = x - yi$. This implies
  that $yi = -yi$, giving us $y = -y$. This only holds true if $y = 0$.
  Therefore, we have $z = x + 0i$ for some real number $x$. This means that $z$
  is real.

  Thus, $z$ is real if and only if $\zb = z$.
\end{proof}

\begin{proof}[Solution to (ii)]
  Assume $z$ is either real or pure imaginary. This gives us two cases, when $z$
  is real, $z = x + 0i$ for some real number $x$, and when $z$ is pure
  imaginary, $z = 0 + yi$ for some real number $y$. Notice that the first case
  is already proven in part (i), i.e., $z = \zb$, giving us $z^2 = \zb^2$. In
  the second case, we have $z = 0 + yi$ and $\zb = 0 - yi = -yi$. Therefore, we
  have $\zb^2 = (-yi)^2 = -y^2$. On the other hand, $z^2 = (0 + yi)^2 = -y^2$.
  Therefore, we have $\zb^2 = z^2$.

  Assume $\zb^2 = z^2$. Let $z = x + yi$ and $\zb = x - yi$. By assumption, we
  have
  \[%
    z^2 = \zb^2 \implies (x + yi)^2 = (x - yi)^2 \implies x^2 - y^2 + 2xyi = x^2 - y^2 - 2xyi
  .\]%
  Comparing the real parts, we have $x^2 - y^2 = x^2 - y^2$, which is always
  true. Comparing the imaginary parts, we have $2xy = -2xy$. This implies that
  $4xy = 0$. This means that either $x = 0$ or $y = 0$. If $x = 0$, then $z$ is
  pure imaginary. If $y = 0$, then $z$ is real. Therefore, $z$ is either real or
  pure imaginary.

  Thus, $z$ is either real or pure imaginary if and only if $\zb^2 = z^2$.
\end{proof}

\begin{problem}[1.5.14]
  Using expressions ($6$), Sec. $5$, for $\Re(z)$ and $\Im(z)$, show that the
  hyperbola $x^2 - y^2 = 1$ can be written as
  \[%
    z^2 + \zb^2 = 2
  .\]%
\end{problem}

\begin{proof}[Solution]
  Substituting $\Re(z)$ and $\Im(z)$ for $x$ and $y$ into the equation for the
  hyperbola, we have
  \[%
    x^2 - y^2 = 1 \implies \Re(z)^2 - \Im(z)^2 = 1 \implies \left(\frac{z + \zb}{2}\right)^2 - \left(\frac{z - \zb}{2i}\right)^2 = 1
  .\]%
  Expanding the left-hand side, we have
  \[%
    \left(\frac{z + \zb}{2}\right)^2 - \left(\frac{z - \zb}{2i}\right)^2 = \frac{(z + \zb)^2}{4} - \frac{(z - \zb)^2}{-4} = \frac{(z + \zb)^2 + (z - \zb)^2}{4}
  .\]%
  Therefore, we have $(z + \zb)^2 + (z - \zb)^2 = 4$. Expanding the squares, we
  have
  \begin{alignat*}{3}
    \phantom{\implies}&\quad z^2 + 2z\zb + \zb^2 + z^2 -2z\zb + \zb^2 &&= 4 \\
    \implies&\quad 2z^2 + 2\zb^2 &&= 4 \\
    \implies&\quad z^2 + \zb^2 &&= 2
  .\qedhere\end{alignat*}
\end{proof}

\begin{problem}[(Extra)]
  Given $a + bi$, $a$, $b$ are real numbers, find $c$ and $d$ such that $(c +
  di)^2 = a + bi$.
\end{problem}

\begin{proof}[Solution]
\end{proof}
