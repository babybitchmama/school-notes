\begin{problem}[2.20.9]
  Let $f$ denote the function whose values are
  \[%
    f(z) = \begin{cases}
      \dfrac{\zb^2}{z} & \text{when}~z \ne 0\vspace{0.9em} \\
      0 & \text{when}~z = 0
    \end{cases}
  .\]%
  Show that if $z = 0$, then $\adv{w}/{z} = 1$ at each nonzero point on the real
  and imaginary axes in the $\Delta z$, or $\Delta x \Delta y$, plane. Then,
  show that $\adv{w}/{z} = -1$ at each nonzero point $(\Delta x, \Delta x)$ on
  the line $\Delta y = \Delta x$ in that plane. Conclude from these observations
  that $f'(0)$ does not exist. Note that to obtain this result, it is not
  sufficient to consider only horizontal and vertical approaches to the origin
  in the $\Delta z$ plane. (Compare with Example 2, Sec 19.)
\end{problem}

\begin{solution}
\end{solution}

\begin{problem}[2.23.4]
  Use the theorem in Sec. 23 to show that each of these functions is
  differentiable in the indicated domain of definition, and also to find
  $f'(z)$:
  \begin{enumerate}
    \item $f(z) = \sfrac{1}{z^4}$.

    \item $f(z) = \sqrt{r}e^{i\sfrac{\theta}{2}}$.

    \item $f(z) = e^{-\theta}\cos(\ln(r)) + ie^{-\theta}\sin(\ln(r))$.
  \end{enumerate}
\end{problem}

\begin{solution}[(i)]
\end{solution}

\begin{solution}[(ii)]
\end{solution}

\begin{solution}[(iii)]
\end{solution}

\begin{problem}[2.26.1]
  Show that $u(x, y)$ is harmonic in some domain and find a harmonic conjugate
  $v(x, y)$ when
  \begin{multicols}{2}
    \begin{enumerate}
      \item[(i)] $u(x, y) = 2x(1 - y)$.

      \item[(iii)] $u(x, y) = \sinh(x)\sin(y)$.

      \item[(ii)] $u(x, y) = 2x - x^3 + 3xy^2$.

      \item[(iv)] $u(x, y) = \sfrac{y}{x^2 + y^2}$.
    \end{enumerate}
  \end{multicols}
\end{problem}

\begin{solution}[(i)]
\end{solution}

\begin{solution}[(ii)]
\end{solution}

\begin{solution}[(iii)]
\end{solution}

\begin{solution}[(iv)]
\end{solution}

\begin{problem}[3.29.10]
  \begin{enumerate}
    \item Show that if $e^z$ is real, then $\Im(z) = n\pi$ ($n = 0, \pm 1, \pm
      2, \cdots$).

    \item If $e^z$ is pure imaginary, what restriction is placed on $z$?
  \end{enumerate}
\end{problem}

\begin{solution}[(i)]
\end{solution}

\begin{solution}[(ii)]
\end{solution}

\begin{problem}[3.33.1]
  Show that
  \begin{enumerate}
    \item $(1 + i)^i = \exp\left(-\frac{\pi}{4} +
      2n\pi\right)\exp\left(i\frac{\ln(2)}{2}\right)$ ($n = 0, \pm 1, \pm 2,
      \cdots$).

    \item $(-1)^{\sfrac{1}{\pi}} = e^{(2n+1)i}$ ($n = 0, \pm 1, \pm 2, \cdots$).
  \end{enumerate}
\end{problem}

\begin{solution}[(i)]
\end{solution}

\begin{solution}[(ii)]
\end{solution}

\begin{problem}[3.33.3]
  Use definition (1), Sec. 33, of $z^c$ to show that $(-1 +
  \sqrt{3}i)^{\sfrac{3}{2}} = \pm 2\sqrt{2}$.
\end{problem}

\begin{solution}
\end{solution}

\begin{problem}[(Extra)]
  Derive the Cauchy-Riemann equation in polar coordinates.
\end{problem}

\begin{solution}
\end{solution}
