\begin{problem}[2.23.6]
  Let $u$ and $v$ denote the real and imaginary components of the function $f$
  defined by means of the equations
  \[%
    f(z) = \begin{cases}
      \sfrac{\zb^2}{z} & \text{when}~z \ne 0 \\
      0 & \text{when}~z = 0
    \end{cases}
  .\]%
  Verify that the Cauchy-Riemann equations $u_x = v_y$ and $u_y = -v_x$ are
  satisfied at the origin $z = (0, 0)$.
\end{problem}

\begin{solution}
  First, write $z = x + iy$, so $\zb = x - iy$. Then compute $f(z)$ for $z \ne
  0$
  \[%
    f(z) = \frac{(x - iy)^2}{x + iy} = \frac{x^2 - 2ixy - y^2}{x + iy}
  .\]%
  Multiply numerator and denominator by the conjugate of the denominator
  \[
    f(z) = \frac{(x^2 - 2ixy - y^2)(x - iy)}{(x + iy)(x - iy)} = \frac{(x^2 - 2ixy - y^2)(x - iy)}{x^2 + y^2}
  .\]%
  Now expand the numerator
  \begin{align*}
    (x^2 - 2ixy - y^2)(x - iy)
    &= x(x^2 - 2ixy - y^2) - iy(x^2 - 2ixy - y^2) \\
    &= x^3 - 2ix^2y - x y^2 - i x^2 y + 2i^2 x y^2 + i y^3 \\
    &= x^3 - x y^2 - 3i x^2 y - 2 x y^2 + i y^3
  .\end{align*}
  (using $i^2 = -1$). So,
  \[%
    f(z) = \frac{x^3 - 3x y^2}{x^2 + y^2} + i \cdot \frac{y^3 - 3x^2 y}{x^2 + y^2}
  .\]%
  Define $u(x, y)$ and $v(x, y)$ by
  \begin{align*}
    &u(x, y) = \begin{cases}
      \sfrac{x^3 - 3xy^2}{x^2 + y^2} & (x, y) \ne (0, 0) \\
      0 & (x, y) = (0, 0)
    \end{cases} \\
    \aand&v(x, y) = \begin{cases}
      \sfrac{y^3 - 3x^2y}{x^2 + y^2} & (x, y) \ne (0, 0) \\
      0 & (x, y) = (0, 0)
    \end{cases}
  .\end{align*}
  Now compute the partial derivatives at the origin
  \begin{alignat*}{5}
    u_x(0, 0) &= \lim_{h \to 0} \frac{u(h, 0) - u(0, 0)}{h}
              &&= \lim_{h \to 0} \frac{h^3 / h^2}{h}
              &&= 1 \\
    u_y(0, 0) &= \lim_{h \to 0} \frac{u(0, h) - u(0, 0)}{h}
              &&= \lim_{h \to 0} \frac{-3 \cdot 0 \cdot h^2 / h^2}{h}
              &&= 0 \\
    v_x(0, 0) &= \lim_{h \to 0} \frac{v(h, 0) - v(0, 0)}{h}
              &&= \lim_{h \to 0} \frac{-3h^2 \cdot 0 / h^2}{h}
              &&= 0 \\
    v_y(0, 0) &= \lim_{h \to 0} \frac{v(0, h) - v(0, 0)}{h}
              &&= \lim_{h \to 0} \frac{h^3 / h^2}{h}
              &&= 1
  .\end{alignat*}
  Therefore, at the origin
  \[
    u_x = v_y = 1, \qquad u_y = -v_x = 0
  .\]%
  The Cauchy-Riemann equations are satisfied at $(0, 0)$.
\end{solution}

\begin{problem}[2.23.8]
  Let a function $f(z) = u + iv$ be differentiable at a non-zero point $z_0 = r_0\exp(i\theta_0)$. Use the expressions for $u_x$ and $v_x$ found in Exercise 7, together with the polar form
\end{problem}

\begin{solution}
  Let $f(z) = u + iv$ be differentiable at a nonzero point $z_0 = r_0 e^{i\theta_0}$. Since $f$ is differentiable at $z_0$, the Cauchy-Riemann equations hold at that point, and the partial derivatives of $u$ and $v$ exist and are continuous near $z_0$.

  Recall from Exercise 7 the relations between Cartesian and polar partials
  \[%
    u_x = \cos\theta \cdot u_r - \frac{\sin\theta}{r} \cdot u_\theta
    \aand
    v_x = \cos\theta \cdot v_r - \frac{\sin\theta}{r} \cdot v_\theta
  .\]%
  % Similarly, for $u_y$ and $v_y$
  % \[%
  %   u_y = \sin\theta \cdot u_r + \frac{\cos\theta}{r} \cdot u_\theta, \qquad
  %   v_y = \sin\theta \cdot v_r + \frac{\cos\theta}{r} \cdot v_\theta
  % .\]%
  % By the Cauchy-Riemann equations,
  % \[
  %   u_x = v_y, \qquad u_y = -v_x
  % .\]%
  % Substitute the expressions above into these equations:
  % \begin{align*}
  %   \cos\theta \cdot u_r - \frac{\sin\theta}{r} \cdot u_\theta &= \sin\theta \cdot v_r + \frac{\cos\theta}{r} \cdot v_\theta \\
  %   \sin\theta \cdot u_r + \frac{\cos\theta}{r} \cdot u_\theta &= -\left( \cos\theta \cdot v_r - \frac{\sin\theta}{r} \cdot v_\theta \right)
  % .\end{align*}
  % Now simplify the second equation:
  % \[
  %   \sin\theta \cdot u_r + \frac{\cos\theta}{r} \cdot u_\theta 
  %   = -\cos\theta \cdot v_r + \frac{\sin\theta}{r} \cdot v_\theta
  % .\]%
  % So the Cauchy-Riemann equations in polar coordinates are:
  % \[
  %   u_r = \frac{1}{r} v_\theta, \qquad v_r = -\frac{1}{r} u_\theta
  % .\]%
  % Therefore, using the expressions from Exercise 7 and the polar form, we have derived the polar version of the Cauchy-Riemann equations at the point $z_0 = r_0 e^{i\theta_0}$.
\end{solution}

\begin{problem}[2.25.7]
  Let a function $f$ be analytic everywhere in a domain $D$. Prove that if $f(z)$ is real-valued for all $z$ in $D$, then $f(z)$ must be constant throughout $D$.
\end{problem}

\begin{solution}
\end{solution}

\begin{problem}[3.31.4]
  Show that
  \[%
    \text{(i)}~\Log(1 +i )^2 = 2\Log(1 + i);\qquad\text{(ii)}~\Log(-1 + i)^2 \ne 2\Log(-1 + i)
  .\]%
\end{problem}

\begin{solution}[(i)]
\end{solution}

\begin{solution}[(ii)]
\end{solution}

\begin{problem}[3.38.2(iii)]
  Evaluate the following integral
  \[%
    \int_0^{\infty} e^{-zt} \dt \qquad (\Re(z) > 0)
  .\]%
\end{problem}

\begin{solution}
\end{solution}

\begin{problem}[3.38.3]
  Show that if $m$ and $n$ are integers,
  \[%
    \int_0^{2\pi} e^{im\theta}e^{-\in\theta} \dd{\theta} = \begin{cases}
      0 & \text{if}~m \ne n \\
      2\pi & \text{if}~m = n
    \end{cases}
  .\]%
\end{problem}

\begin{solution}
\end{solution}

\begin{problem}[3.38.5]
  Let $w(t) = u(t) + iv(t)$ denote a continuous complex-valued function defined on an interval $-a \le t \le a$.
  \begin{enumerate}
    \item Suppose that $w(t)$ is \emph{even}; that is, $w(-t) = w(t)$ for each point
      $t$ in the given interval. Show that
      \[%
        \int_{-a}^a w(t) \dt = 2\int_0^a w(t) \dt
      .\]%

    \item Show that if $w(t)$ is an \emph{odd} function, one where $w(-t) =
      -w(t)$ for each point $t$ in the given interval, then
      \[%
        \int_{-a}^a w(t) \dt = 0
      .\]%
  \end{enumerate}

  \emph{Suggestion:} In each part of this exercise, use the corresponding property of integrals of \emph{real-valued} functions of $t$, which is graphically evident.
\end{problem}

\begin{solution}[(i)]
\end{solution}

\begin{solution}[(ii)]
\end{solution}

\begin{problem}[3.39.6]
  Let $y(x)$ be a real-valued function defined on the interval $0 \le x \le 1$ by means of the equations
  \[%
    y(x) = \begin{cases}
      x^3\sin(\sfrac{\pi}{x}) & \text{when}~0 < x \le 1 \\
      0 & \text{when}~x = 0
    \end{cases}
  .\]%
  \begin{enumerate}
    \item Show that the equation
      \[%
        z = x + iy(x) \qquad (0 \le x \le 1)
      ,\]%
      represents an arc $C$ that intersects the real axis at the points $z = \sfrac{1}{n}$ ($n = 1, 2, \cdots$) and $z = 0$.

    \item Verify that the arc $C$ in part (i) is, in fact, a \emph{smooth} arc.

      \emph{Suggestion:} To establish the continuity of $y(x)$ at $x = 0$,
      observe that
      \[%
        0 \le \left\lvert x^3\sin\left(\frac{\pi}{x}\right) \right\rvert \le x^3
      ,\]%
      when $x > 0$. A similar remark applies in finding $y'(0)$ and showing that
      $y'(x)$ is continuous at $x = 0$.
  \end{enumerate}
\end{problem}

\begin{solution}
\end{solution}
