\begin{problem}[2.23.6]
  Let $u$ and $v$ denote the real and imaginary components of the function $f$
  defined by means of the equations
  \[%
    f(z) = \begin{cases}
      \sfrac{\zb^2}{z} & \text{when}~z \ne 0 \\
      0 & \text{when}~z = 0
    \end{cases}
  .\]%
  Verify that the Cauchy-Riemann equations $u_x = v_y$ and $u_y = -v_x$ are
  satisfied at the origin $z = (0, 0)$.
\end{problem}

\begin{solution}
  First, write $z = x + iy$, so $\zb = x - iy$. Then compute $f(z)$ for $z \ne
  0$
  \[%
    f(z) = \frac{(x - iy)^2}{x + iy} = \frac{x^2 - 2ixy - y^2}{x + iy}
  .\]%
  Multiply numerator and denominator by the conjugate of the denominator
  \[%
    f(z) = \frac{(x^2 - 2ixy - y^2)(x - iy)}{(x + iy)(x - iy)} = \frac{(x^2 - 2ixy - y^2)(x - iy)}{x^2 + y^2}
  .\]%
  Now expand the numerator
  \begin{align*}
    (x^2 - 2ixy - y^2)(x - iy)
    &= x(x^2 - 2ixy - y^2) - iy(x^2 - 2ixy - y^2) \\
    &= x^3 - 2ix^2y - x y^2 - i x^2 y + 2i^2 x y^2 + i y^3 \\
    &= x^3 - x y^2 - 3i x^2 y - 2 x y^2 + i y^3
  .\end{align*}
  (using $i^2 = -1$). So,
  \[%
    f(z) = \frac{x^3 - 3x y^2}{x^2 + y^2} + i \cdot \frac{y^3 - 3x^2 y}{x^2 + y^2}
  .\]%
  Define $u(x, y)$ and $v(x, y)$ by
  \begin{align*}
    &u(x, y) = \begin{cases}
      \sfrac{x^3 - 3xy^2}{x^2 + y^2} & (x, y) \ne (0, 0) \\
      0 & (x, y) = (0, 0)
    \end{cases} \\
    \aand&v(x, y) = \begin{cases}
      \sfrac{y^3 - 3x^2y}{x^2 + y^2} & (x, y) \ne (0, 0) \\
      0 & (x, y) = (0, 0)
    \end{cases}
  .\end{align*}
  Now compute the partial derivatives at the origin
  \begin{alignat*}{5}
    u_x(0, 0) &= \lim_{h \to 0} \frac{u(h, 0) - u(0, 0)}{h}
              &&= \lim_{h \to 0} \frac{h^3 / h^2}{h}
              &&= 1 \\
    u_y(0, 0) &= \lim_{h \to 0} \frac{u(0, h) - u(0, 0)}{h}
              &&= \lim_{h \to 0} \frac{-3 \cdot 0 \cdot h^2 / h^2}{h}
              &&= 0 \\
    v_x(0, 0) &= \lim_{h \to 0} \frac{v(h, 0) - v(0, 0)}{h}
              &&= \lim_{h \to 0} \frac{-3h^2 \cdot 0 / h^2}{h}
              &&= 0 \\
    v_y(0, 0) &= \lim_{h \to 0} \frac{v(0, h) - v(0, 0)}{h}
              &&= \lim_{h \to 0} \frac{h^3 / h^2}{h}
              &&= 1
  .\end{alignat*}
  Therefore, at the origin
  \[%
    u_x = v_y = 1, \qquad u_y = -v_x = 0
  .\]%
  The Cauchy-Riemann equations are satisfied at $(0, 0)$.
\end{solution}

\begin{problem}[2.23.8]
  Let a function $f(z) = u + iv$ be differentiable at a non-zero point $z_0 =
  r_0\exp(i\theta_0)$. Use the expressions for $u_x$ and $v_x$ found in Exercise
  7, together with the polar form
\end{problem}

\begin{solution}
  Let $f(z) = u + iv$ be differentiable at a nonzero point $z_0 = r_0
  e^{i\theta_0}$. Since $f$ is differentiable at $z_0$, the Cauchy-Riemann
  equations hold at that point, and the partial derivatives of $u$ and $v$ exist
  and are continuous near $z_0$.

  Recall from Exercise 7 the relations between Cartesian and polar partials
  \[%
    u_x = \cos(\theta) \cdot u_r - \frac{\sin(\theta)}{r} \cdot u_\theta
    \aand
    v_x = \cos(\theta) \cdot v_r - \frac{\sin(\theta)}{r} \cdot v_\theta
  .\]%
  Similarly, for $u_y$ and $v_y$
  \[%
    u_y = \sin(\theta) \cdot u_r + \frac{\cos(\theta)}{r} \cdot u_\theta
    \aand
    v_y = \sin(\theta) \cdot v_r + \frac{\cos(\theta)}{r} \cdot v_\theta
  .\]%
  By the Cauchy-Riemann equations,
  \[%
    u_x = v_y \aand u_y = -v_x
  .\]%
  Substitute the expressions above into these equations:
  \begin{align*}
    \cos(\theta) \cdot u_r - \frac{\sin(\theta)}{r} \cdot u_\theta &= \sin(\theta) \cdot v_r + \frac{\cos(\theta)}{r} \cdot v_\theta \\
    \sin(\theta) \cdot u_r + \frac{\cos(\theta)}{r} \cdot u_\theta &= -\left( \cos(\theta) \cdot v_r - \frac{\sin(\theta)}{r} \cdot v_\theta \right)
  .\end{align*}
  Now simplify the second equation:
  \[%
    \sin(\theta) \cdot u_r + \frac{\cos(\theta)}{r} \cdot u_\theta = -\cos(\theta) \cdot v_r + \frac{\sin(\theta)}{r} \cdot v_\theta
  .\]%
  So the Cauchy-Riemann equations in polar coordinates are:
  \[%
    u_r = \frac{1}{r} v_\theta \aand v_r = -\frac{1}{r} u_\theta
  .\]%
  Therefore, using the expressions from Exercise 7 and the polar form, we have
  derived the polar version of the Cauchy-Riemann equations at the point $z_0 =
  r_0 e^{i\theta_0}$.
\end{solution}

\begin{problem}[2.25.7]
  Let a function $f$ be analytic everywhere in a domain $D$. Prove that if
  $f(z)$ is real-valued for all $z$ in $D$, then $f(z)$ must be constant
  throughout $D$.
\end{problem}

\begin{solution}
  Since $f$ is analytic on $D$, it satisfies the Cauchy-Riemann equations in
  $D$. Write
  \[%
    f(z) = u(x, y) + i v(x, y)
  ,\]%
  where $z = x + iy$, and $u$ and $v$ are the real and imaginary parts of $f$,
  respectively. Given that $f(z) \in \mathbb{R}$ for all $z \in D$, we have
  $v(x, y) = 0$ for all $(x, y) \in D$.

  Because $f$ is analytic, the Cauchy-Riemann equations must hold:
  \[%
    u_x = v_y \aand u_y = -v_x
  .\]%
  Since $v(x, y) \equiv 0$, it follows that $v_x = 0 = v_y$ Substituting into
  the Cauchy-Riemann equations, we get $u_x = 0 = u_y$. Hence, all first partial
  derivatives of $u$ vanish on $D$, so $u$ is constant throughout $D$.
  Therefore,
  \[%
    f(z) = u(x, y) + i v(x, y) = \text{constant}
  .\qedhere\]%
\end{solution}

\begin{problem}[3.31.4]
  Show that
  \[%
    \text{(i)}~\Log(1 +i )^2 = 2\Log(1 + i);\qquad\text{(ii)}~\Log(-1 + i)^2 \ne 2\Log(-1 + i)
  .\]%
\end{problem}

\begin{solution}[(i)]
  We use the principal branch of the complex logarithm:
  \[%
    \Log(z) = \ln(\lvert z \rvert) + i \Arg(z) \qquad (-\pi < \Arg(z) \le \pi)
  .\]%
  Let $z = 1 + i$. Then
  \[%
    \lvert z \rvert = \sqrt{1^2 + 1^2} = \sqrt{2} \aand \Arg(z) = \frac{\pi}{4}
  .\]%
  So,
  \[%
    \Log(1 + i) = \ln(\sqrt{2}) + i\frac{\pi}{4}
  .\]%
  Then:
  \[%
    2\Log(1 + i) = 2\ln(\sqrt{2}) + i \cdot 2 \cdot\frac{\pi}{4} = \ln(2) + i\frac{\pi}{2}
  .\]%
  Now consider $\Log((1 + i)^2) = \Log(2i)$. Now, we can compute
  \[%
    \lvert 2i \rvert = 2 \aand \Arg(2i) = \frac{\pi}{2}
  .\]%
  So
  \[%
    \Log((1 + i)^2) = \ln(2) + i\frac{\pi}{2}
  .\]%
  Therefore, we get
  \[%
    \Log((1 + i)^2) = 2\Log(1 + i)
  .\qedhere\]%
\end{solution}

\begin{solution}[(ii)]
  Let $z = -1 + i$. Then:
  \[%
    \lvert z \rvert = \sqrt{(-1)^2 + 1^2} = \sqrt{2} \aand \Arg(z) = \frac{3\pi}{4}
  ,\]%
  since $z$ is in the second quadrant So,
  \[%
    \Log(-1 + i) = \ln(\sqrt{2}) + i\frac{3\pi}{4}
  .\]%
  Then:
  \[%
    2\Log(-1 + i) = \ln(2) + i \cdot \frac{3\pi}{2}
  .\]%
  Now consider:
  \[%
    (-1 + i)^2 = (-1)^2 + 2(-1)(i) + i^2 = 1 - 2i - 1 = -2i
  .\]%
  Then:
  \[%
    \lvert -2i \rvert = 2 \aand \Arg(-2i) = -\frac{\pi}{2}
  .\]%
  So:
  \[%
    \Log((-1 + i)^2) = \Log(-2i) = \ln(2) + i(-\frac{\pi}{2}) = \ln(2) - i\frac{\pi}{2}
  .\]%
  Comparing:
  \[%
    \Log((-1 + i)^2) = \ln(2) - i\frac{\pi}{2}
  ,\]%
  but
  \[%
    2\Log(-1 + i) = \ln(2) + i\frac{3\pi}{2}
  .\]%
  These are not equal. Therefore,
  \[%
    \Log((-1 + i)^2) \ne 2\Log(-1 + i)
  .\qedhere\]%
\end{solution}

\begin{problem}[4.38.2(iii)]
  Evaluate the following integral
  \[%
    \int_0^{\infty} e^{-zt} \dt \qquad (\Re(z) > 0)
  .\]%
\end{problem}

\begin{solution}
  Computing the anti-derivative of $e^{-zt}$, we have
  \[%
    \int e^{-zt} \dt = -\frac{1}{z} e^{-zt} + C
  .\]%
  As $t \to \infty$, since $\Re(z) > 0$, we have $e^{-zt} \to 0$ exponentially.
  Therefore, we get
  \[%
    \int_0^{\infty} e^{-zt} \dt = 0 - \left(-\frac{1}{z} \cdot 1\right) = \frac{1}{z}
  .\qedhere\]%
\end{solution}

\begin{problem}[4.38.3]
  Show that if $m$ and $n$ are integers,
  \[%
    \int_0^{2\pi} e^{im\theta}e^{-in\theta} \dd{\theta} = \begin{cases}
      0 & \text{if}~m \ne n \\
      2\pi & \text{if}~m = n
    \end{cases}
  .\]%
\end{problem}

\begin{solution}
  Simplifying the integrand, we have
  \[%
    \int_0^{2\pi} e^{im\theta}e^{-in\theta} \dd{\theta} = \int_0^{2\pi} e^{i(m - n)\theta} \dd{\theta}
  .\]%
  Let $k = m - n$. Then, we have
  \[%
    \int_0^{2\pi} e^{ik\theta} \dd{\theta} = \begin{cases}
      2\pi & \text{if}~k = 0 \\
      \sfrac{1}{ik} \cdot (e^{2\pi i k} - 1) = 0 & \text{if}~k \ne 0
    \end{cases}
  .\]%
  This is because $e^{2\pi ik} = 1$ for any integer $k$, so the numerator
  becomes $0$ when $k \ne 0$. Therefore,
  \[%
    \int_0^{2\pi} e^{im\theta}e^{-in\theta} \dd{\theta} = \begin{cases}
      0 & \text{if}~m \ne n \\
      2\pi & \text{if}~m = n
    \end{cases}
  .\qedhere\]%
\end{solution}

\begin{problem}[4.38.5]
  Let $w(t) = u(t) + iv(t)$ denote a continuous complex-valued function defined
  on an interval $-a \le t \le a$.
  \begin{enumerate}
    \item Suppose that $w(t)$ is \emph{even}; that is, $w(-t) = w(t)$ for each
      point $t$ in the given interval. Show that
      \[%
        \int_{-a}^a w(t) \dt = 2\int_0^a w(t) \dt
      .\]%

    \item Show that if $w(t)$ is an \emph{odd} function, one where $w(-t) =
      -w(t)$ for each point $t$ in the given interval, then
      \[%
        \int_{-a}^a w(t) \dt = 0
      .\]%
  \end{enumerate}

  \emph{Suggestion:} In each part of this exercise, use the corresponding
  property of integrals of \emph{real-valued} functions of $t$, which is
  graphically evident.
\end{problem}

\begin{solution}[(i)]
  Suppose $w(t) = u(t) + iv(t)$ is even, i.e., $w(-t) = w(t)$. Then both $u(t)$
  and $v(t)$ are even functions. Using the fact that the integral of an even
  real-valued function over a symmetric interval is twice the integral over $[0,
  a]$, we have
  \[%
    \int_{-a}^a w(t) \dt = \int_{-a}^a u(t) \dt + i \int_{-a}^a v(t) \dt = 2\int_0^a u(t) \dt + i \cdot 2\int_0^a v(t) \dt
  .\]%
  Thus,
  \[%
    \int_{-a}^a w(t) \dt = 2 \int_0^a w(t) \dt
  .\qedhere\]%
\end{solution}

\begin{solution}[(ii)]
  Now suppose $w(t) = u(t) + iv(t)$ is odd, i.e., $w(-t) = -w(t)$. Then both
  $u(t)$ and $v(t)$ are odd functions. Since the integral of an odd real-valued
  function over a symmetric interval is zero, we have
  \[%
    \int_{-a}^a w(t) \dt = \int_{-a}^a u(t) \dt + i \int_{-a}^a v(t) \dt = 0 + i \cdot 0 = 0
  .\]%
  Therefore,
  \[%
    \int_{-a}^a w(t) \dt = 0
  .\qedhere\]%
\end{solution}

\begin{problem}[4.39.6]
  Let $y(x)$ be a real-valued function defined on the interval $0 \le x \le 1$
  by means of the equations
  \[%
    y(x) = \begin{cases}
      x^3\sin(\sfrac{\pi}{x}) & \text{when}~0 < x \le 1 \\
      0 & \text{when}~x = 0
    \end{cases}
  .\]%
  \begin{enumerate}
    \item Show that the equation
      \[%
        z = x + iy(x) \qquad (0 \le x \le 1)
      ,\]%
      represents an arc $C$ that intersects the real axis at the points $z =
      \sfrac{1}{n}$ ($n = 1, 2, \cdots$) and $z = 0$.

    \item Verify that the arc $C$ in part (i) is, in fact, a \emph{smooth} arc.

      \emph{Suggestion:} To establish the continuity of $y(x)$ at $x = 0$,
      observe that
      \[%
        0 \le \left\lvert x^3\sin\left(\frac{\pi}{x}\right) \right\rvert \le x^3
      ,\]%
      when $x > 0$. A similar remark applies in finding $y'(0)$ and showing that
      $y'(x)$ is continuous at $x = 0$.
  \end{enumerate}
\end{problem}

\begin{solution}[(i)]
  To find where the arc $C$ intersects the real axis, we observe that this
  occurs when $\Im(z) = y(x) = 0$. Note that $y(x) =
  x^3\sin\left(\sfrac{\pi}{x}\right)$ for $x > 0$. Since
  $\sin\left(\sfrac{\pi}{x}\right) = 0$ when $\sfrac{\pi}{x} = n\pi$ (i.e., $x =
  \sfrac{1}{n}$ for $n = 1, 2, \dots$), it follows that $y(\sfrac{1}{n}) = 0$,
  and thus
  \[%
    z = \frac{1}{n} + i \cdot 0 = \frac{1}{n}
  ,\]%
  lies on the real axis. Also, since $y(0) = 0$, we have $z = 0 + i \cdot 0 = 0$
  also lies on the real axis. Therefore, the arc intersects the real axis at the
  points $z = \sfrac{1}{n}$ and $z = 0$.
\end{solution}

\begin{solution}[(ii)]
  First, we show that $y(x)$ is continuous on $[0, 1]$. Since $\lvert
  \sin(\pi/x) \rvert \le 1$, we have
  \[%
    |y(x)| = |x^3\sin(\pi/x)| \le x^3 \longrightarrow 0 \quad \text{as}~x \to 0
  .\]%
  Hence, $\lim_{x \to 0} y(x) = 0 = y(0)$, so $y(x)$ is continuous on $[0, 1]$.
  Computing the derivative, we have
  \[%
    y'(x) = \dv{x}\left(x^3\sin\left(\frac{\pi}{x}\right)\right) = 3x^2\sin\left(\frac{\pi}{x}\right) - \pi x \cos\left(\frac{\pi}{x}\right)
  .\]%
  To analyze continuity at $x = 0$, observe that
  \[%
    \left\lvert y'(x) \right\rvert \le 3x^2 + \pi x \to 0
  ,\]%
  as $x \to 0$. So $\lim_{x \to 0} y'(x) = 0$. Also, for $x = 0$, we define
  \[%
    y'(0) \coloneqq \lim_{x \to 0} \frac{y(x) - y(0)}{x - 0} = \lim_{x \to 0} \frac{x^3\sin\left(\sfrac{\pi}{x}\right)}{x} = \lim_{x \to 0} x^2\sin(\sfrac{\pi}{x}) = 0
  .\]%
  Thus, $y'(x)$ is continuous on $[0, 1]$, and $z(x) = x + i y(x)$ is
  continuously differentiable.

  Therefore, the arc $C$ is smooth.
\end{solution}
