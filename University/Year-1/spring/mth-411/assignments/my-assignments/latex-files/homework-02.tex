\begin{problem}[1.8.1]
  Find the principal argument $\Arg(z)$ when
  \[%
    \text{(i)}~z = \frac{i}{-2 - 2i}; \qquad\text{(ii)}~z = \left(\sqrt{3} - i\right)^6
  .\]%
\end{problem}

\begin{solution}[(i)]
  Simplifying the expression, we have
  \[%
    z = \frac{i}{-2 - 2i} \cdot \frac{-2 + 2i}{-2 + 2i} = \frac{i(-2 + 2i)}{8} = -\frac{1}{4} - \frac{1}{4}i
  .\]%
  The principal argument is
  \[%
    \Arg(z) = \tan^{-1}\left(\frac{-\sfrac{1}{4}}{-\sfrac{1}{4}}\right) = \tan^{-1}(1) = \frac{p}{4}
  .\]%
  Since $z$ is in the third quadrant, we have
  \[%
    \Arg(z) = -\pi + \frac{\pi}{4} = -\frac{3\pi}{4}
  .\qedhere\]%
\end{solution}

\begin{solution}[(ii)]
  Simplifying the expression using the Binomial Theorem, we have
  \[%
    z = \sum_{k=0}^{6} \binom{6}{k} (\sqrt{3})^{6-k} (-i)^k = \sum_{k=0}^{6} \binom{6}{k} (\sqrt{3})^{6-k} (-1)^k i^k
  .\]%
  Each term contributes either a real or imaginary value. At the end, all
  imaginary parts cancel out, and we are left with only the real part, giving us
  \[%
    z = \left(\sqrt{3} - i\right)^6 = -64
  .\]%
  Since the real number is negative, we have
  \[%
    \Arg(z) = \tan^{-1}\left(\frac{0}{-64}\right) = \pi
  .\qedhere\]%
\end{solution}

\begin{problem}[1.8.9]
  Establish the identity
  \[%
    1 + z + z^2 + \cdots + z^n = \frac{1 - z^{n+1}}{1 - z} \quad (z \ne 1)
  ,\]%
  and then use it to derive \textit{Lagrange's trigonometric identity:}
  \[%
    1 + \cos(\theta) + \cos(2\theta) + \cdots + \cos(n\theta) = \frac{1}{2} + \frac{\sin[\sfrac{(2n + 1)\theta}{2}]}{2\sin(\sfrac{\theta}{2})} \quad (0 < \theta < 2\pi)
  .\]%

  \textit{Suggestion:} As for the first identity, write $S = 1 + z + z^2 +
  \cdots + z^n$ and consider the difference $S - zS$. To derive the second
  identity, write $z = e^{i\theta}$ in the first one.
\end{problem}

\begin{solution}
  We first establish the first identity. Let $S = 1 + z + z^2 + \cdots + z^n$.
  We compute $S - zS$ to get
  \[%
    S - zS = (1 + z + z^2 + \cdots + z^n) - (z + z^2 + z^3 + \cdots + z^{n + 1}) = 1 - z^{n+1}
  .\]%
  Thus, provided $z \ne 1$, we have
  \[%
    S = \frac{1 - z^{n+1}}{1 - z}
  .\]%

  Now, we derive Lagrange's trigonometric identity. We write $z = e^{i\theta}$.
  So, $\lvert z \rvert = 1$ and $z^k = e^{ik\theta}$. By the geometric series
  formula,
  \[%
    \sum_{k=0}^n z^k = \frac{1 - z^{n+1}}{1 - z}
  ,\]%
  since $z^k = e^{ik\theta} = \cos(k\theta) + i\sin(k\theta)$, taking the real
  part of both sides, we have
  \[%
    \sum_{k=0}^n \cos(k\theta) = \Re\left(\sum_{k=0}^n z^k\right) = \Re\left(\frac{1 - z^{n+1}}{1 - z}\right)
  .\]%
  We know that $z^{n+1} = e^{i(n+1)\theta}$ and $z = e^{i\theta}$. Expanding $1
  - e^{i\theta}$, we have
  \begin{align*}
    1 - e^{i\theta} &= e^{\sfrac{i\theta}{2}} \left(e^{-\sfrac{i\theta}{2}} - e^{\sfrac{i\theta}{2}}\right) \\
                    &= e^{\sfrac{i\theta}{2}} \left(-2i\sin(\sfrac{\theta}{2})\right)
  .\end{align*}
  Expanding $1 - e^{i(n+1)\theta}$, we have
  \[%
    1 - e^{i(n+1)\theta} = -2i\sin\left(\frac{(n+1)\theta}{2}\right)e^{i(n+1)\theta/2}
  .\]%
  Thus, we have
  \begin{align*}
    \frac{1 - z^{n+1}}{1 - z} &= \frac{-2i\sin\left(\frac{(n + 1)\theta}{2}\right)e^{i(n+1)\sfrac{\theta}{2}}}{e^{\sfrac{i\theta}{2}} \left(-2i\sin(\sfrac{\theta}{2})\right)} \\
                              &= \frac{\sin((n + 1)\sfrac{\theta}{2})}{\sin(\sfrac{\theta}{2})} \cdot \frac{e^{i(n+1)\sfrac{\theta}{2}}}{e^{i\sfrac{\theta}{2}}} \\
                              &= \frac{\sin((n + 1)\sfrac{\theta}{2})}{\sin(\sfrac{\theta}{2})} \cdot e^{in\sfrac{\theta}{2}} \\
                              &= \frac{\sin((n + 1)\sfrac{\theta}{2})}{\sin(\sfrac{\theta}{2})} \cdot \left(\cos\left(\frac{n\theta}{2}\right) + i\sin\left(\frac{n\theta}{2}\right)\right)
  .\end{align*}
  Taking the real part, we have
  \[%
    \Re\left(\frac{1 - z^{n+1}}{1 - z}\right) = \frac{\sin((n + 1)\sfrac{\theta}{2})}{\sin(\sfrac{\theta}{2})} \cdot \cos\left(\frac{n\theta}{2}\right)
  .\]%
  We can now use the identity $2\sin(A)\cos(B) = \sin(A + B) + \sin(A - B)$, by
  letting
  \[%
    A = \frac{(n + 1)\theta}{2} \aand B = \frac{n\theta}{2}
  .\]%
  This gives us
  \[%
    2\sin\left(\frac{(n + 1)\theta}{2}\right)\cos\left(\frac{n\theta}{2}\right) = \sin\left(\frac{(2n + 1)\theta}{2}\right) + \sin\left(\frac{\theta}{2}\right)
  .\]%
  Therefore
  \[%
    \frac{\sin\left[\sfrac{(n + 1)\theta}{2}\right]}{\sin\left[\sfrac{\theta}{2}\right]} \cdot \cos\left(\frac{n\theta}{2}\right) = \frac{1}{2} + \frac{\sin\left[\sfrac{(2n + 1)\theta}{2}\right]}{2\sin\left(\sfrac{\theta}{2}\right)}
  .\]%

  Therefore, we have
  \[%
    1 + \cos(\theta) + \cos(2\theta) + \cdots + \cos(n\theta) = \frac{1}{2} + \frac{\sin\left[\sfrac{(2n + 1)\theta}{2}\right]}{2\sin\left(\sfrac{\theta}{2}\right)}
  .\qedhere\]%
\end{solution}

\begin{problem}[1.8.10]
  Use de Moivre's formula (Sec. 7) to derive the following trigonometric
  identities:
  \[%
    \text{(i)}~\cos(3\theta) = \cos^3(\theta) - 3\cos(\theta)\sin^2(\theta); \qquad \text{(ii)}~\sin(3\theta) = 3\cos^2(\theta)\sin(\theta) - \sin^3(\theta)
  .\]%
\end{problem}

\begin{solution}[(i)]
  De Moivre's formula states that
  \[%
    \cos(n\theta) + i\sin(n\theta) = \left(\cos(\theta) + i\sin(\theta)\right)^n
  ,\]%
  we can expand the right-hand side, when $n = 3$, to get
  \begin{equation}\label{eq:1.8.10}
    \left(\cos(\theta) + i\sin(\theta)\right)^3 = \cos^3(\theta) + 3i\cos^2(\theta)\sin(\theta) - 3\cos(\theta)\sin^2(\theta) - i\sin^3(\theta)
  \end{equation}
  Separating the real part from equation \ref{eq:1.8.10}, we have
  \[%
    \cos(3\theta) = \cos^3(\theta) - 3\cos(\theta)\sin^2(\theta)
  .\qedhere\]%
\end{solution}

\begin{solution}[(ii)]
  Separating the imaginary part from equation \ref{eq:1.8.10}, we have
  \[%
    \sin(3\theta) = 3\cos^2(\theta)\sin(\theta) - \sin^3(\theta)
  .\qedhere\]%
\end{solution}

\begin{problem}[1.10.3]
  In each case, find all the roots in rectangular coordinates, exhibit them as
  vertices of certain regular polygons, and identify the principle root
  \[%
    \text{(i)}~(-1)^{\sfrac{1}{3}}; \qquad \text{(ii)}~z^5 = 8^{\sfrac{1}{6}}
  .\]%
\end{problem}

\begin{solution}[(i)]
  We-writing $-1$ in polar form, we have
  \[%
    -1 = 1 \cdot \exp\left[i(-\pi + 2k\pi)\right]
  .\]%
  Taking the cube root, we have
  \[%
    (-1)^{\sfrac{1}{3}} = \exp\left[i\left(-\frac{\pi}{3} + \frac{2k\pi}{3}\right)\right]
  .\]%
  The principal root is when $k = 0$, giving us
  \[%
    (-1)^{\sfrac{1}{3}} = \exp\left[-\frac{\pi}{3}i\right] = \cos\left(-\frac{\pi}{3}\right) + i\sin\left(-\frac{\pi}{3}\right) = \frac{1}{2} - \frac{\sqrt{3}}{2}i
  .\]%
  The other roots are when $k = 1$ and $k = 2$, giving us
  \begin{alignat*}{5}
    (-1)^{\sfrac{1}{3}} &= \exp\left[i\left(-\frac{\pi}{3} + \frac{2\pi}{3}\right)\right] &&= \exp\left[i\left(\frac{\pi}{3}\right)\right] &&= \cos\left(\frac{\pi}{3}\right) + i\sin\left(\frac{\pi}{3}\right) &&= \frac{1}{2} + \frac{\sqrt{3}}{2}i \\
    (-1)^{\sfrac{1}{3}} &= \exp\left[i\left(-\frac{\pi}{3} + \frac{4\pi}{3}\right)\right] &&= \exp\left[i\left(\frac{5\pi}{3}\right)\right] &&= \cos\left(\frac{5\pi}{3}\right) + i\sin\left(\frac{5\pi}{3}\right) &&= -1
  .\end{alignat*}
  Thus, the three roots are
  \[%
    \frac{1}{2} - \frac{\sqrt{3}}{2}i, \quad \frac{1}{2} + \frac{\sqrt{3}}{2}i, \aand -1
  .\qedhere\]%
\end{solution}

\begin{solution}[(ii)]
  We first simplify $8^{1/6} = (2^3)^{1/6} = 2^{1/2} = \sqrt{2}$. So we are
  solving the equation $z^5 = \sqrt{2}$. We write $\sqrt{2}$ in polar form
  \[%
    \sqrt{2} = 2^{\sfrac{1}{2}} \cdot \exp(0)
  .\]%
  Thus, the five complex fifth roots of $\sqrt{2}$ are given by
  \[%
    z_k = 2^{\sfrac{1}{10}} \cdot \exp\left[\frac{2\pi ik}{5}\right]
  ,\]%
  where $k = 0, 1, 2, 3, 4$. The principal root is when $k = 0$, giving us
  \[%
    z_0 = 2^{\sfrac{1}{10}} \cdot \exp\left[\frac{2\pi i \cdot 0}{5}\right] = 2^{\sfrac{1}{10}} \cdot \exp(0) = 2^{\sfrac{1}{10}}
  .\]%
  The other roots are when $k = 1, 2, 3, 4$, giving us
  \begin{alignat*}{5}
    z_1 &= 2^{\sfrac{1}{10}} \cdot \exp\left[\frac{2\pi i}{5}\right] &&= 2^{\sfrac{1}{10}} \left(\cos\left(\frac{2\pi}{5}\right) + i\sin\left(\frac{2\pi}{5}\right)\right) &&= 2^{\sfrac{1}{10}} \left(\cos(72^\circ) + i\sin(72^\circ)\right) \\
    z_2 &= 2^{\sfrac{1}{10}} \cdot \exp\left[\frac{4\pi i}{5}\right] &&= 2^{\sfrac{1}{10}} \left(\cos\left(\frac{4\pi}{5}\right) + i\sin\left(\frac{4\pi}{5}\right)\right) &&= 2^{\sfrac{1}{10}} \left(\cos(144^\circ) + i\sin(144^\circ)\right) \\
    z_3 &= 2^{\sfrac{1}{10}} \cdot \exp\left[\frac{6\pi i}{5}\right] &&= 2^{\sfrac{1}{10}} \left(\cos\left(\frac{6\pi}{5}\right) + i\sin\left(\frac{6\pi}{5}\right)\right) &&= 2^{\sfrac{1}{10}} \left(\cos(216^\circ) + i\sin(216^\circ)\right) \\
    z_4 &= 2^{\sfrac{1}{10}} \cdot \exp\left[\frac{8\pi i}{5}\right] &&= 2^{\sfrac{1}{10}} \left(\cos\left(\frac{8\pi}{5}\right) + i\sin\left(\frac{8\pi}{5}\right)\right) &&= 2^{\sfrac{1}{10}} \left(\cos(288^\circ) + i\sin(288^\circ)\right)
  .\tag*{\qedhere}\end{alignat*}
\end{solution}

\begin{problem}[1.11.1]
  Sketch the following sets and determine which are domains:
  \begin{multicols}{2}
    \begin{enumerate}
      \item[(i)] $\lvert z - 2 + i \rvert \le 1$.

      \item[(iii)] $\Im(z) > 1$.

      \item[(v)] $0 \le \arg(z) \le \sfrac{\pi}{4}$ ($z \ne 0$).

      \item[(ii)] $\lvert 2z + 3 \rvert > 4$.

      \item[(iv)] $\Im(z) = 1$.

      \item[(vi)] $\lvert z - 4 \rvert \ge \lvert z \rvert$.
    \end{enumerate}
  \end{multicols}
\end{problem}

\begin{solution}[(i)]
\end{solution}

\begin{solution}[(ii)]
\end{solution}

\begin{solution}[(iii)]
\end{solution}

\begin{solution}[(iv)]
\end{solution}

\begin{solution}[(v)]
\end{solution}

\begin{solution}[(vi)]
\end{solution}

\begin{problem}[1.11.3]
  Which sets in Exercise 1 are bounded?
\end{problem}

\begin{solution}
\end{solution}

\begin{problem}[2.20.1]
  Use results in Sec. 20 to find $f'(z)$ when
  \begin{multicols}{2}
    \begin{enumerate}
      \item[(i)] $f(z) = 3z^2 - 2 + 4$.

      \item[(iii)] $\displaystyle f(z) = \frac{z - 1}{2z + 1}$ ($z \ne
        -\sfrac{1}{2}$).

      \item[(ii)] $f(z) = \left(1 - 4z^2\right)^3$.

      \item[(iv)] $\displaystyle f(z) = \frac{\left(1 + z^2\right)^4}{z^2}$ ($z
        \ne 0$).
    \end{enumerate}
  \end{multicols}
\end{problem}

\begin{solution}[(i)]
  We can use the power rule to find the derivative of $f(z) = 3z^2 - 2 + 4$.
  The derivative is given by
  \[%
    f'(z) = \odv{}{z}(3z^2) + \odv{}{z}(-2) + \odv{}{z}(4) = 6z + 0 + 0 = 6z
  .\qedhere\]%
\end{solution}

\begin{solution}[(ii)]
  Let $w = 1 - 4z^2$ and $W = w^3$. Then, we can use the chain rule to find the
  derivative of $f(z)$
  \[%
    \odv{W}{z} = \odv{W}{w} \cdot \odv{w}{z} = 3w^2 \cdot (-8z) = -24z(1 - 4z^2)^2
  .\qedhere\]%
\end{solution}

\begin{solution}[(iii)]
  We can use the quotient rule to find the derivative of $f(z) = \frac{z - 1}{2z
  + 1}$. The derivative is given by
  \[%
    f'(z) = \frac{(2z + 1)(1) - (z - 1)(2)}{(2z + 1)^2} = \frac{2z + 1 - 2z + 2}{(2z + 1)^2} = \frac{3}{(2z + 1)^2}
  .\qedhere\]%
\end{solution}

\begin{solution}[(iv)]
  Let $w = 1 + z^2$ and $W = w^4$. Then, we can use the quotient rule to find
  the derivative of $f(z)$
  \[%
    \odv{W}{z} = \odv{W}{w} \cdot \odv{w}{z} = 4w^3 \cdot (2z) = 8z(1 + z^2)^3
  .\]%
  Now, we can use the quotient rule to find the derivative of $f(z) =
  \frac{(1 + z^2)^4}{z^2}$, giving us
  \[%
    f'(z) = \frac{(z^2)(8z(1 + z^2)^3) - ((1 + z^2)^4)(2z)}{(z^2)^2} = \frac{8z^3(1 + z^2)^3 - 2z(1 + z^2)^4}{z^4} = \frac{2(3z^2 - 1)(1 + z^2)^3}{z^3}
  .\qedhere\]%
\end{solution}

\begin{problem}[2.20.2]
  Using results in Sec. 20, show that
  \begin{enumerate}
    \item a polynomial
      \[%
        P(z) = a_0 + a_1z + a_2z^2 + \cdots + a_nz^n \quad (a_n \ne 0)
      ,\]%
      of degree $n$ ($n \ge 1$) is differentiable everywhere, with derivative
      \[%
        P'(z) = a_1 + 2a_2z + \cdots + na_nz^{n-1}
      .\]%

    \item the coefficients in the polynomial $P(z)$ in part (i) can be written
      \[%
        a_0 = P(0), \quad a_1 = \frac{P'(0)}{1!}, \quad a_2 = \frac{P''(0)}{2!}, \quad \cdots, \quad a_n = \frac{P^{(n)}(0)}{n!}
      .\]%
  \end{enumerate}
\end{problem}

\begin{solution}[(i)]

\end{solution}

\begin{solution}[(ii)]
\end{solution}

\begin{problem}[2.23.1]
  Use the theorem in Sec. 21 to show that $f'(z)$ does not exist at any point if
  \begin{multicols}{2}
    \begin{enumerate}
      \item[(i)] $f(z) = \zb$.

      \item[(iii)] $f(z) = 2x + ixy^2$.

      \item[(ii)] $f(z) = z - \zb$.

      \item[(iv)] $f(z) = e^xe^{-iy}$.
    \end{enumerate}
  \end{multicols}
\end{problem}

\begin{solution}[(i)]
\end{solution}

\begin{solution}[(ii)]
\end{solution}

\begin{solution}[(iii)]
\end{solution}

\begin{solution}[(iv)]
\end{solution}

\begin{problem}[2.23.3]
  From results obtained in Secs. 21 and 22, determine where $f'(z)$ exists and
  find its value when
  \[%
    \text{(i)}~f(z) = \frac{1}{z}; \qquad \text{(ii)}~f(z) = x^2 + iy^2; \qquad \text{(iii)}~f(z) = z\Im(z)
  .\]%
\end{problem}

\begin{solution}[(i)]
\end{solution}

\begin{solution}[(ii)]
\end{solution}

\begin{solution}[(iii)]
\end{solution}
