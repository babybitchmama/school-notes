Complex analysis is a fundamental branch of mathematics that extends the
familiar concepts of calculus to functions of a complex variable. You might
think of complex analysis as a natural extension of real analysis. But, that is
not quite right. In fact, complex analysis is much easier than real analysis.
For example, integrations are easier in the complex plane.

We first start with the basics of complex numbers that many are familiar with
from basic high school, such as the algebra of complex numbers, the polar form,
and the exponential form. We then develop the concept of complex functions and
their differentiability, which leads to the Cauchy-Riemann equations
\[%
  \pdv{u}{x} = \pdv{v}{y} \aand \pdv{u}{y} = -\pdv{v}{x}
,\]%
which provide necessary conditions for a function $f(z) = u(x, y) + iv(x, y)$ to
be holomorphic.

A key focus of the course is the study of elementary functions such as the
complex exponential, trigonometric, and logarithmic functions. The complex
logarithm introduces the notion of branch cuts.

Integration in the complex plane differs significantly from its real counterpart
due to the profound results of Cauchy's theorem and Cauchy's integral formula.
Cauchy's theorem states that if $f$ is holomorphic in a simply connected domain
$D$, then for any closed contour $\gamma$ in $D$,
\[%
  \oint_\gamma f(z) \dz = 0
.\]%
Cauchy's integral formula further asserts that if $f$ is holomorphic inside and
on a simple closed contour $\gamma$, then for any point $z_0$ inside $\gamma$,
\[%
  f(z_0) = \frac{1}{2\pi i} \oint_\gamma \frac{f(z)}{z - z_0} \dz
.\]%

From these results, we develop power series representations of complex
functions, which lead to the classification of singularities and the study of
residues. Residue theory is one of the most powerful tools in complex analysis,
allowing the evaluation of intricate real integrals using contour integration.
The residue theorem states that if $f(z)$ meromorphic in a region containing a
simple closed contour $\gamma$, then
\[%
  \oint_\gamma f(z) \dz = 2\pi i \sum \text{Res}(f, z_k)
.\]%

Finally, we explore Laurent series, which generalize power series to include
terms with negative exponents. A function $f(z)$ analytic in an annulus $R_1 <
\lvert z - z_0 \rvert < R_2$ has a Laurent series expansion of the form
\[%
  f(z) = \sum_{n = -\infty}^{\infty} a_n (z - z_0)^n
.\]%
These expansions provide a systematic way to analyze singularities and
understand the local behavior of functions near points of interest.
