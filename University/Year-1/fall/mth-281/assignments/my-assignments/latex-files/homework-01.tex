\renewcommand\type{problem}

\begin{problem}
  Consider the following regions of $\R^3$ described by an inequality. Describe
  each region in words and sketch it to the best of your ability. Note:
  Inequalities define a solid region in space rather than a surface.
  \begin{enumerate}
    \item $4 < x^2 + y^2 + z^2$.

    \item $x^2 + z^2 \le 9$.
  \end{enumerate}
\end{problem}

\begin{probsolution}
  \begin{enumerate}
    \item The inequality represents the points $(x, y, z)$ whose distance from
      the origin is more than $2$. This creates a sphere of radius $2$ centered
      at the origin (see \cref{fig:1_a}). All the points are outside the sphere.

    \item Since $y = 0$ in this inequality, that means the inequality is being
      drawn on the $xz$-plane. This is just a circle of radius $3$ centered at
      the origin (see \cref{fig:1_b}). NOTE: The disk should be drawn on the
      $xz$-plane, but my artistic skills are very subpar.
  \end{enumerate}

  \begin{figure}[H]
    \centering

    \begin{subfigure}[b]{0.45\textwidth}
      \centering

      \includegraphics[width=\textwidth]{./image-files/homework-01/1.jpg}

      \caption{}
      \label{fig:1_a}
    \end{subfigure}
    \begin{subfigure}[b]{0.45\textwidth}
      \centering

      \includegraphics[width=\textwidth]{./image-files/homework-01/2.jpg}

      \caption{}
      \label{fig:1_b}
    \end{subfigure}

    \caption{}
    \label{fig:1}
  \end{figure}
\end{probsolution}

\newpage

\begin{problem}
  Consider the vectors $\a = \langle -2, 5, 4 \rangle$ and $\b = \langle 4, 8, 1
  \rangle$.
  \begin{enumerate}
    \item Find $2\a - 3\b$.

    \item Find the length of the vectors $\a$ and $\b$.

    \item Find the angle between the vectors $\a$ and $\b$.

    \item Find a unit vector in the direction of $\a$.

    \item Find $\comp_{\a}(\b)$ and $\proj_{\a}(\b)$.
  \end{enumerate}
\end{problem}

\begin{probsolution}
  \begin{enumerate}
    \item $\displaystyle2\langle -2, 5, 4 \rangle - 3\langle 4, 8, 1 \rangle =
      \langle -4, 10, 8 \rangle - \langle 12, 24, 3 \rangle = \langle -16, -14,
      5 \rangle$.

    \item $\displaystyle\lvert a \rvert = \sqrt{(-2)^2 + 5^2 + 4^2} =
      \sqrt{45}$.

      $\displaystyle\lvert b \rvert = \sqrt{4^2 + 8^2 + 1^2} = \sqrt{81} = 9$.

    \item $\displaystyle\frac{\a \cdot \b}{\lvert \a \rvert \cdot \lvert \b
      \rvert} = \frac{36}{\sqrt{45} \cdot 9} = \frac{4\sqrt{5}}{15}$. This gives
      us $\displaystyle\arccos\left(\frac{4 \sqrt{5}}{15}\right) \approx
      \SI{54}{\degree}$.

    \item The unit vector for $\a$ is $\displaystyle\frac{\a}{\lvert \a \rvert}
      = \frac{\langle -2, 5, 4 \rangle}{\sqrt{45}} = \left\langle
      -\frac{2}{\sqrt{45}}, \frac{5}{\sqrt{45}}, \frac{4}{\sqrt{45}}
      \right\rangle$.

    \item $\displaystyle\comp_{\a}(\b) = \frac{\a \cdot \b}{\lvert \a \rvert} =
      \frac{36}{\sqrt{45}}$.

      $\displaystyle\proj_{\a}(\b) = \frac{\a \cdot \b}{\lvert \a \rvert^2} \a =
      \frac{36}{45} \langle -2, 5, 4 \rangle = \left\langle -\frac{8}{5}, 4,
      \frac{16}{5} \right\rangle$.
  \end{enumerate}
\end{probsolution}

\newpage

\begin{problem}
  Find the values of $x$ such that the vectors $\langle 6, 3x, x \rangle$ and
  $\langle -3, 1, x \rangle$ are orthogonal.
\end{problem}

\begin{probsolution}
  Both vectors are orthogonal if and only if their dot product is zero. Thus,
  solving for $x$ gives us
  \begin{alignat*}{3}
    \langle 6, 3x, x \rangle \cdot \langle -3, 1, x \rangle = 0 &\implies = 6(-3) + 3x(1) + x(x) &&= 0 \\
                                                                &\implies -18 + 3x + x^2 &&= 0 \\
                                                                &\implies (x - 3)(x + 6) &&= 0
  .\end{alignat*}
  This gives us $x = 3$ and $x = -6$. Therefore, the following vector pairs are
  orthogonal to one another
  \begin{align*}
    \langle 6, 9, 3 \rangle &\aand \langle -3, 1, 3 \rangle \\
    \langle 6, -18, -6 \rangle &\aand \langle -3, 1, -6 \rangle
  .\end{align*}
\end{probsolution}

\newpage

\begin{problem}
  If $\r = \langle x, y, z \rangle$ is an arbitrary position vector and $\u =
  \langle u_1, u_2, u_3 \rangle$ and $\v = \langle v_1, v_2, v_3 \rangle$ are
  constant position vectors, show that $(\r - \u) \cdot (\r - \v) = 0$ defines a
  sphere. Find the center and radius of the sphere.
\end{problem}

\begin{probsolution}
  Expanding the expression gives us
  \begin{alignat*}{3}
    &\phantom{\implies}~(\r - \u) \cdot (\r - \v) &&= 0 \\
         &\implies \langle x - u_1, y -u_2, z - u_3 \rangle \cdot \langle x - v_1, y - v_2, z - v_3 \rangle &&= 0 \\
         &\implies (x - u_1)(x - v_1) + (y - u_2)(y - v_2) + (z - u_3)(z - v_3) &&= 0 \\
         &\implies x^2 - (u_1 + v_1)x + u_1v_1 + y^2 - (u_2 + v_2)y + u_2v_2 + z^2 - (u_3 + v_3)z + u_3v_3 &&= 0 \\
         &\implies x^2 - (u_1 + v_1)x + y^2 - (u_2 + v_2)y + z^2 - (u_3 + v_3)z &&= D_1
  ,\end{alignat*}
  where $D_n$ is the sum of all the constant terms. Lastly, we need to factor
  everything by completing the square, giving us
  \begin{alignat*}{3}
    &\phantom{\implies}~x^2 - (u_1 + v_1)x + y^2 - (u_2 + v_2)y + z^2 - (u_3 + v_3)z &&= D_1 \\
    &\implies x^2 - (u_1 + v_1)x + \left(\frac{u_1 + v_1}{2}\right)^2 + y^2 - (u_2 + v_2)y + \left(\frac{u_2 + v_2}{2}\right)^2 + z^2 - (u_3 + v_3)z + \left(\frac{u_3 + v_3}{2}\right)^2 &&= D_2 \\
    &\implies \left(x - \frac{u_1 + v_1}{2}\right)^2 + \left(y - \frac{u_2 + v_2}{2}\right)^2 + \left(z - \frac{u_3 + v_3}{2}\right)^2 &&= D_3
  ,\end{alignat*}
  where $\displaystyle \frac{u_1 + v_1}{2} = h$, $\displaystyle\frac{u_2 +
  v_2}{2} = k$, $\displaystyle\frac{u_3 + v_3}{2} = l$, and $\sqrt{D_3} = r$.
  This gives us the following center and radius for the sphere:
  \begin{align*}
    \text{Center} &= (h, k, l) = \left(\frac{u_1 + v_1}{2}, \frac{u_2 + v_2}{2}, \frac{u_3 + v_3}{2}\right) \\
    \text{Radius} &= r = \sqrt{D_3}
  .\end{align*}
\end{probsolution}

\newpage

\begin{problem}
  Show that if $\u + \v$ and $\u - \v$ are orthogonal, then $\u$ and $\v$ have
  the same length.
\end{problem}

\begin{probsolution}
  Suppose $\u + \v$ and $\u - \v$ are orthogonal. Then, by definition, we get
  \begin{align*}
    (\u + \v) \cdot (\u - \v) &= 0 \\
    \u^2 - \u \cdot \v + \v \cdot \u - \v^2 &= 0 \\
    \u^2 - \v^2 &= 0 \\
    \u^2 &= \v^2 \\
    \sqrt{\u \cdot \u} &= \sqrt{\v \cdot \v}
  .\end{align*}
  Expanding $\u \cdot \u$ and $\v \cdot \v$, we get
  \[%
    \underbrace{\sqrt{u_1^2 + u_2^2 + \cdots + u_n^2}}_{\lvert \u \rvert} = \underbrace{\sqrt{v_1^2 + v_2^2 + \cdots + v_n^2}}_{\lvert \v \rvert}
  .\]%
  Therefore, $\u$ and $\v$ have the same length if $\u + \v$ and $\u - \v$ are
  orthogonal.
\end{probsolution}
