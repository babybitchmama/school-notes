\begin{problem}
  Use traces to identify the surfaces. Sketch the region bounded by the surfaces
  and determine the curve of intersection. Find a vector function that
  parametrizes the curves of intersection.
  \begin{enumerate}
    \item $z = \sqrt{x^2 + y^2}$ and $x^2 + y^2 + z^2 = 6$ for $z \ge 0$.

    \item $z = x^2 + 3y^2$ and $z = 12 - 3x^2 - y^2$.
  \end{enumerate}
\end{problem}

\begin{probsolution}
  \begin{enumerate}
    \item The first equation is the equation of a cone and the second equation
      is the equation of a sphere.

      To find the curve where the cone and the sphere intersect, substitute $z =
      \sqrt{x^2 + y^2}$ into the equation of the sphere to get
      \[%
        x^2 + y^2 + x^2 + y^2 = 6 \implies 2x^2 + 2y^2 = 6 \implies x^2 + y^2 = 3
      .\]%
      So, their intersection creates a circle in the $xy$-plane with a radius of
      $\sqrt{3}$. Therefore, the curve of intersection is parametrized by the
      vector function
      \[%
        \r(t) = \langle \sqrt{3} \cos(t), \sqrt{3} \sin(t), \sqrt{3} \rangle, \quad 0 \le t \le 2\pi
      .\]%

    \item The first equation is the equation of an elliptic paraboloid and the
      second equation is the equation of an elliptic paraboloid.

      To find the curve where the two paraboloids intersect, set the two
      equations equal to each other
      \[%
        x^2 + 3y^2 = 12 - 3x^2 - y^2 \implies x^2 + y^2 = 3
      .\]%
      So, their intersection creates a circle in the $xy$-plane with a radius of
      $\sqrt{3}$. The circle of radius $\sqrt{3}$ in the $xy$-plane can be
      parameterized as
      \[%
        x = \sqrt{3}\cos(t) \aand y = \sqrt{3}\sin(t)
      .\]%
      Substitute these into either equation to find $z$
      \[%
        z = 3\cos^2(t) + 9\sin^2(t)
      .\]%
      Therefore, the curve of intersection is parametrized by the vector function
      \[%
        \r(t) = \langle \sqrt{3}\cos(t), \sqrt{3}\sin(t), 3\cos^2(t) + 9\sin^2(t) \rangle, \quad 0 \le t \le 2\pi
      .\]%
  \end{enumerate}
\end{probsolution}

\newpage

\begin{problem}
  Consider the vector function $\displaystyle\r(t) = \left\langle \sqrt{16 - t^2}, t^2 - 2t + 1, \frac{t + 3}{t^2 - 2t - 3} \right\rangle$.
  \begin{enumerate}
    \item Find the domain of $\r(t)$.

    \item Find $\r'(t)$.

    \item Find the vector equation for the tangent line to the curve at the
      point $(4, 1, -1)$.
  \end{enumerate}
\end{problem}

\begin{probsolution}
  \begin{enumerate}
    \item The domain of $\r(t)$ is the intersection of the domain of all the
      function components of $\r(t)$.

      The domain of $f(t)\ui$ is $t \ge 4$, or $[-4, 4]$.

      The domain of $g(t)\uj$ is $t \in \mathbb{R}$, or $(-\infty, \infty)$.

      The domain of $h(t)\uk$ is $t \neq -1, 3$, or $(-\infty, -1) \cup (-1,
      3) \cup (3, \infty)$.

      Therefore, the domain of $\r(t)$ is [\textbf{NOTE:} I left out the
      $(-\infty, \infty)$ interval, as it doesn't change anything]
      \begin{align*}
        [-4, 4] \cap [(-\infty, -1) \cup (-1, 3) \cup (3, \infty)] &= [4, \infty) \cap [(-\infty, -1) \cup (-1, 3) \cup (3, \infty)] \\
                                                                   &= [-4, -1) \cap (-1, 3) \cap (3, 4] \\
      .\end{align*}
      Therefore, the domain of $\r(t)$ is $[-4, 4]~\backslash~\{-1, 3\}$.

    \item The derivative of $\r(t)$ is just the derivatives of the components,
      giving us
      \begin{alignat*}{3}
        f'(t) \ui &= \odv{}{t} \left[\sqrt{16 - t^2}\right] &&= \frac{-t}{\sqrt{16 - t^2}} \\
        g'(t) \uj &= \odv{}{t} \left[t^2 - 2t + 1\right] &&= 2t - 2 \\
        h'(t) \uk &= \odv{}{t} \left[\frac{t + 3}{t^2 - 2t - 3}\right] &&= \frac{(t^2 - 2t - 3) - (t + 3)(2t - 2)}{(t^2 - 2t - 3)^2} \\
                    & &&= -\frac{t^2 + 6t - 3}{(t^2 - 2t - 3)^2}
      .\end{alignat*}
      Therefore, we get
      \[%
        \r'(t) = \left\langle -\frac{t}{\sqrt{16 - t^2}}, 2t - 2, -\frac{t^2 + 6t - 3}{(t^2 - 2t - 3)^2} \right\rangle
      .\]%

    \item Since $\r(0) = \langle 4, 1, -1 \rangle$ and $\r'(0) = \langle 0, -2,
      -\sfrac{1}{2} \rangle$, the unit tangent vector at the point $(4, 1, -1)$
      is
      \[%
        \Ta(0) = \frac{\r'(0)}{\lvert \r'(0) \rvert} = \frac{\langle 0, -2, -\sfrac{1}{2} \rangle}{\sqrt{0^2 + (-2)^2 + \left(-\frac{1}{2}\right)^2}} = \left\langle 0, -2 \cdot \frac{2}{\sqrt{17}}, -\frac{1}{2} \cdot \frac{2}{\sqrt{17}} \right\rangle = \left\langle 0, -\frac{4}{\sqrt{17}}, -\frac{1}{\sqrt{17}} \right\rangle
      .\]%
  \end{enumerate}
\end{probsolution}

\newpage

\begin{problem}
  In general, the magnitude of a vector function is a scalar function. We may
  want to know the rate at which the magnitude of the position vector, $\r(t)$
  changes along the space curve defined by the function. Suppose $\r(t) \ne 0$
  is a differentiable vector function. Show that
  \[%
    \odv{}{t} [\r(t)] = \frac{1}{\lvert \r(t) \rvert} \r(t) \cdot \r'(t)
  .\]%
  Hint: rewrite $\lvert \r(t) \rvert^2$ using the dot product and differentiate
  the result.
\end{problem}

\begin{probsolution}
  \begin{proof}
    The magnitude of a vector $\r(t)$ is given by $\lvert \r(t) \rvert$. The
    square of this magnitude can be written using the dot product
    \[%
      \lvert \r(t) \rvert^2 = \r(t) \cdot \r(t)
    .\]%
    Now, differentiate both sides of the equation $\lvert \r(t) \rvert^2 = \r(t)
    \cdot \r(t)$ with respect to $t$
    \[%
      \odv{}{t} \left[\r(t) \cdot \r(t)\right] = \r'(t) \cdot \r(t) + \r(t) \cdot \r'(t) = 2 \r(t) \cdot \r'(t)
    .\]%
    Now, observe that
    \[%
      \odv{}{t} \left[\lvert \r(t) \rvert^2 \right] = 2 \lvert \r(t) \rvert \odv{}{t} \left[\lvert \r(t) \rvert\right]
    .\]%
    Equating this with the result from Step 2 gives
    \[%
      2 \lvert \r(t) \rvert \odv{}{t} \left[\lvert \r(t) \rvert\right] = 2 \r(t) \cdot \r'(t)
    .\]%
    Finally, divide both sides by $2 \lvert \r(t) \rvert$ (assuming $\r(t) \neq
    0$) to isolate $\odv{}{t} \lvert \r(t) \rvert$
    \[%
      \odv{}{t} \left[\lvert \r(t) \rvert\right] = \frac{\r(t) \cdot \r'(t)}{\lvert \r(t) \rvert}
    .\]%

    Thus, we have shown that
    \[%
      \odv{}{t} [\lvert \r(t) \rvert] = \frac{1}{\lvert \r(t) \rvert} \r(t) \cdot \r'(t)
    .\qedhere\]%
  \end{proof}
\end{probsolution}
