\begin{problem}
  Find the tangent plane to the surface
  \[%
    x^3y\cos(z^2 + x - y) = 3
  ,\]%
  at the point $P = (-1, 3, 2)$.
\end{problem}

\begin{probsolution}
  Let $f(x, y, z) = x^3y\cos(z^2 + x - y) - 3$. The surface is defined
  implicitly by $f(x, y, z) = 0$.

  The gradient of $f$, $\displaystyle \nabla f(x, y, z) = \left\langle
  \pdv{f}{x}, \pdv{f}{y}, \pdv{f}{z} \right\rangle$, is normal to the tangent
  plane. Thus, we compute the partial derivatives of $f$
  \begin{align*}
    \pdv{f}{x} &= 3x^2y\cos(z^2 + x - y) - x^3y\sin(z^2 + x - y) \\
    \pdv{f}{y} &= x^3\cos(z^2 + x - y) + x^3y\sin(z^2 + x - y) \\
    \pdv{f}{z} &= -2zx^3y\sin(z^2 + x - y) \\
  .\end{align*}
  At $P$, substitute $x = -1$, $y = 3$, and $z = 2$ into $\nabla f(x, y, z)$.
  \[%
    \nabla f(-1, 3, 2) = \left\langle 3(1)(3)(1) - (-1)(3)(0), -1(1) + (-1)(3)(0), -2(2)(-1)(3)(0) \right\rangle = \langle 9, -1, 0 \rangle
  .\]%
  The equation of the tangent plane at $P = (-1, 3, 2)$ is given by
  \[%
    f_x(x - x_0) + f_y(y - y_0) + f_z(z - z_0) = 0
  .\]%
  Substituting $\nabla f(P) = (9, -1, 0)$ and $P = (-1, 3, 2)$ will give us the
  equation of the tangent plane. Thus, we get
  \[%
    9x - y + 12 = 0
  .\]%
\end{probsolution}

\newpage

\begin{problem}
  Find and classify (local maximum, local minimum, or saddle) the critical
  points of the functions.
  \begin{enumerate}
    \item $f(x, y) = 2x^3 + xy^2 + 5x^2 + y^2$.
    \item $f(x, y) = e^y(y^2 - x^2)$.
  \end{enumerate}
\end{problem}

\begin{probsolution}
  \begin{enumerate}
    \item Let $f(x, y) = 2x^3 + xy^2 + 5x^2 + y^2$. To find the critical points,
      we compute the first partial derivatives of $f$
      \[%
        \pdv{f}{x} = 6x^2 + y^2 + 10x \aand \pdv{f}{y} = 2xy + 2y
      .\]%
      Set $\pdv{f}{x} = 0$ and $\pdv{f}{y} = 0$ to solve for $x$ and $y$.
      \begin{alignat*}{4}
        \pdv{f}{y} = 0 &\implies 2y(x + 1) &&= 0 \implies y &&= 0 &&\oor x = -1 \\
        \pdv{f}{x} = 0 &\implies 6x^2 + 10x &&= 0 \implies x(6x + 10) &&= 0 \implies x = 0 &&\oor x = -\frac{5}{3}
      .\end{alignat*}
      Thus, the critical points are $(0, 0)$ and $(-\sfrac{5}{3}, 0)$.
      Substituting $x = -1$ into $\pdv{f}{x} = 0$ gives us
      \[%
        6(-1)^2 + y^2 + 10(-1) = 0 \implies 6 + y^2 - 10 = 0 \implies y^2 = 4 \implies y = \pm 2
      .\]%
      Thus, the additional critical points are $(-1, 2)$ and $(-1, -2)$.

      To classify the critical points, we compute the second partial derivatives
      \[%
        \pdv[2]{f}{x} = 12x + 10, \quad \pdv[2]{f}{y} = 2x + 2, \aand \pdv{f}{x}{y} = 2y
      .\]%
      The Hessian determinant is
      \[%
        H = \pdv[2]{f}{x} \pdv[2]{f}{y} - \left( \pdv{f}{x}{y} \right)^2 = (12x + 10)(2x + 2) - (2y)^2
      .\]%

      \begin{enumerate}
        \item At $(0, 0)$, we get the following values
          \[%
            \pdv[2]{f}{x} = 10, \quad \pdv[2]{f}{y} = 2, \aand \pdv{f}{x,y} = 0
          .\]%
          Plugging these values into the Hessian determinant gives us
          \[%
            H = (10)(2) - (0)^2 = 20 > 0, \quad \pdv[2]{f}{x} > 0
          .\]%
          Thus, $(0, 0)$ is a local minimum.

        \item At $(-\sfrac{5}{3}, 0)$, we get the following values
          \[%
            \pdv[2]{f}{x} = 12(-\frac{5}{3}) + 10 = -10, \quad \pdv[2]{f}{y} = 2(-\frac{5}{3}) + 2 = -\frac{4}{3}, \aand \pdv{f}{x}{y} = 0
          .\]%
          Plugging these values into the Hessian determinant gives us
          \[%
            H = (-10)(-\frac{4}{3}) - (0)^2 = \frac{40}{3} > 0, \quad \pdv[2]{f}{x} < 0
          .\]%
          Thus, $(-\sfrac{5}{3}, 0)$ is a local maximum.

        \item At $(-1, 2)$ and $(-1, -2)$, we get the following values
          \[%
            \pdv[2]{f}{x} = 12(-1) + 10 = -2, \quad \pdv[2]{f}{y} = 2(-1) + 2 = 0, \aand \pdv{f}{x}{y} = 4 \oor -4
          .\]%
          Plugging these values into the Hessian determinant gives us
          \[%
            H = (-2)(0) - (4)^2 = -16 < 0
          .\]%
          Thus, $(-1, 2)$ and $(-1, -2)$ are saddle points.
      \end{enumerate}

    \item Let $f(x, y) = e^y(y^2 - x^2)$. To find the critical points, compute
      the first partial derivatives
      \[%
        \pdv{f}{x} = -2xe^y \aand \pdv{f}{y} = e^y(2y + y^2 - x^2)
      .\]%
      Set $\pdv{f}{x} = 0$ and $\pdv{f}{y} = 0$ to get
      \begin{alignat*}{4}
        \pdv{f}{x} = 0 &\implies -2xe^y &&= 0 \implies x &&= 0 \\
        \pdv{f}{y} = 0 &\implies e^y(2y + y^2) &&= 0 \implies 2y + y^2 &&= 0 \implies y(y + 2) = 0 \implies y = 0 \oor y = -2
      .\end{alignat*}
      Thus, the critical points are $(0, 0)$ and $(0, -2)$.

      To classify the critical points, compute the second partial derivatives
      \[%
        \pdv[2]{f}{x} = -2e^y, \quad \pdv[2]{f}{y} = e^y(2 + 2y + y^2 - x^2), \aand \pdv{f}{x}{y} = -2xe^y
      .\]%
      The Hessian determinant is
      \[%
        H = \pdv[2]{f}{x} \pdv[2]{f}{y} - \left(\pdv{f}{x,y}\right)^2
      .\]%

      \begin{enumerate}
        \item At $(0, 0)$, we get the following values
          \[%
            \pdv[2]{f}{x} = -2, \quad \pdv[2]{f}{y} = 2, \aand \pdv{f}{x}{y} = 0
          .\]%
          Plugging these values into the Hessian determinant gives us
          \[%
            H = (-2)(2) - (0)^2 = -4 < 0
          .\]%
          Thus, $(0, 0)$ is a saddle point.

        \item At $(0, -2)$, we get the following values
          \[%
            \pdv[2]{f}{x} = -2e^{-2}, \quad \pdv[2]{f}{y} = e^{-2}(2 - 4 + 4) = 2e^{-2}, \aand \pdv{f}{x}{y} = 0
          .\]%
          Plugging these values into the Hessian determinant gives us
          \[%
            H = (-2e^{-2})(2e^{-2}) - (0)^2 = -4e^{-4} > 0, \quad \pdv[2]{f}{x} < 0
          .\]%
          Since $H > 0$ and $\pdv[2]{f}{x} < 0$, the critical point $(0, -2)$ is
          a local maximum.
      \end{enumerate}
  \end{enumerate}
\end{probsolution}
