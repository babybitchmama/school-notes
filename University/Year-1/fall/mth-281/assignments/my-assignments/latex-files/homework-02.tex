\begin{problem}
  Determine if the points $A(3, 1, 2)$, $B(1, -1, 1)$, $C(-2, 4, 3)$, and $D(-3,
  -5, 0)$ are coplanar. That is, determine if the points lie in the same plane.
\end{problem}

\begin{probsolution}
  We first find the vectors $\overrightarrow{AB}$, $\overrightarrow{AC}$, and
  $\overrightarrow{AD}$.
  \begin{alignat*}{3}
    \overrightarrow{AB} &= \langle 1 - 3, -1 - 1, 1 - 2 \rangle &&= \langle -2, -2, -1 \rangle \\
    \overrightarrow{AC} &= \langle -2 - 3, 4 - 1, 3 - 2 \rangle &&= \langle -5, 3, 1 \rangle \\
    \overrightarrow{AD} &= \langle -3 - 3, -5 - 1, 0 - 2 \rangle &&= \langle -6, -6, -2 \rangle
  .\end{alignat*}
  Then, we compute the scalar triple product $\overrightarrow{AB} \cdot
  (\overrightarrow{AC} \times \overrightarrow{AD})$.
  \[%
    \overrightarrow{AC} \times \overrightarrow{AD} = \begin{vmatrix}
      \ui & \uj & \uk \\
      -5 & 3 & 1 \\
      -6 & -6 & -2
    \end{vmatrix}
    = \ui\begin{vmatrix}
      3 & 1 \\
      -6 & -2
    \end{vmatrix}
    - \uj\begin{vmatrix}
      -5 & 1 \\
      -6 & -2
    \end{vmatrix}
    + \uk\begin{vmatrix}
      -5 & 3 \\
      -6 & -6
    \end{vmatrix}
  .\]%
  Therefore, we have $\overrightarrow{AC} \times \overrightarrow{AD} = \langle
  0, -16, 48 \rangle$. Then, we have
  \[%
    \overrightarrow{AB} \cdot (\overrightarrow{AC} \times \overrightarrow{AD}) = \langle -2, -2, -1 \rangle \cdot \langle 0, -16, 48 \rangle = (-2)(0) + (-2)(-16) + (-1)(48) = -16
  .\]%
  Since the scalar triple product is not zero, the points $A$, $B$, $C$, and $D$
  are not coplanar.
\end{probsolution}

\newpage

\begin{problem}
  Suppose $\a \ne 0$.
  \begin{enumerate}
    \item True or False : If $\a \cdot \b = \a \cdot \c$, then $\b = \c$.
      Justify your answer.

    \item True or False : If $\a \times \b = \a \times \c$, then $\b = \c$.
      Justify your answer.

    \item True or False : If $\a \cdot \b = \a \cdot \c$ and $\a \times \b = \a
      \times \c$, then $\b = \c$. Justify your answer.
  \end{enumerate}
\end{problem}

\begin{probsolution}
  \begin{enumerate}
    \item False: The dot product $\a \cdot \b = \a \cdot \c$ means that the
      projections of $\b$ and $\c$ in the direction of $\a$ are equal. However,
      the vectors $\b$ and $\c$ can be different.

    \item False: The cross product $\a \times \b = \a \times \c$ means that the
      vectors $\b$ and $\c$ have the same components perpendicular to $\a$.
      However, the vectors $\b$ and $\c$ can be different.

    \item True: If $\a \cdot \b = \a \cdot \c$, that means that $\b$ and $\c$
      have the same projection in the direction of $\a$. If $\a \times \b = \a
      \times \c$, that means that $\b$ and $\c$ have the same components
      perpendicular to $\a$. Therefore, $\b$ and $\c$ must be the same.
  \end{enumerate}
\end{probsolution}

\newpage

\begin{problem}
  \begin{enumerate}
    \item Find parametric equations and symmetric equations for the line through the point $P(5, 3, 1)$ and $Q(7, 4, -3)$.

    \item At what point does the line intersect the $xz$-plane?

    \item Where does the line intersect the plane $x - 3y - 2z = 8$?
  \end{enumerate}
\end{problem}

\begin{probsolution}
  \begin{enumerate}
    \item First, find $\overrightarrow{PQ}$
      \[%
        \overrightarrow{PQ} = \langle 7 - 5, 4 - 3, -3 - 1 \rangle = \langle 2, 1, -4 \rangle
      .\]%
      The parametric equations of a line with a point $P(x_0, y_0, z_0)$ and a
      direction vector $\langle a, b, c \rangle$ are
      \begin{align*}
        x &= 5 + 2t \\
        y &= 3 + t \\
        z &= 1 - 4t
      ,\end{align*}
      and the symmetric equations are
      \[%
        \frac{x - 5}{2} = \frac{y - 3}{1} = \frac{z - 1}{-4}
      .\]%

    \item To find the intersection point, set $y = 0$ in the parametric
      equations and solve for $t$, giving us $t = -3$. Substituting $t = -3$
      into the parametric equations gives us the intersection point $(-1, 0,
      13)$.

    \item To find the intersection point, substitute the parametric equations
      for $x$, $y$, and $z$ into the plane equation, giving us
      \[%
        (5 + 2t) - 3(3 + t) - 2(1 - 4t) = 8 \implies t = 2
      .\]%
      Substituting $t = 2$ into the parametric equations gives us the
      intersection point $(9, 5, -7)$.
  \end{enumerate}
\end{probsolution}

\newpage

\begin{problem}
  Consider the points $P(2, 6, -1)$, $Q(-1, 8, -2)$, and $R(3, 7, 0)$.
  \begin{enumerate}
    \item Find a vector orthogonal to the plane containing the points $P$, $Q$,
      and $R$.

    \item Find the area of the triangle with vertices $P$, $Q$, and $R$.

    \item Find the equation of the plane containing the points $P$, $Q$, and
      $R$.
  \end{enumerate}
\end{problem}

\begin{probsolution}
  \begin{enumerate}
    \item First, find $\overrightarrow{PQ}$ and $\overrightarrow{PR}$
      \begin{alignat*}{3}
        \overrightarrow{PQ} &= \langle -1 - 2, 8 - 6, -2 - (-1) \rangle &&= \langle -3, 2, -1 \rangle \\
        \overrightarrow{PR} &= \langle 3 - 2, 7 - 6, 0 - (-1) \rangle &&= \langle 1, 1, 1 \rangle
      .\end{alignat*}
      Now, find the cross product $\overrightarrow{PQ} \times
      \overrightarrow{PR}$.
      \[%
        \overrightarrow{PQ} \times \overrightarrow{PR} = \begin{vmatrix}
          \ui & \uj & \uk \\
          -3 & 2 & -1 \\
          1 & 1 & 1
        \end{vmatrix}
        = \ui\begin{vmatrix}
          2 & -1 \\
          1 & 1
        \end{vmatrix}
        - \uj\begin{vmatrix}
          -3 & -1 \\
          1 & 1
        \end{vmatrix}
        + \uk\begin{vmatrix}
          -3 & 2 \\
          1 & 1
        \end{vmatrix}
      .\]%
      Therefore, we have $\overrightarrow{PQ} \times \overrightarrow{PR} =
      \langle 3, 2, -5 \rangle$. Thus, the vector $\langle 3, 2, -5 \rangle$ is
      orthogonal to the plane containing the points $P$, $Q$, and $R$.

    \item The area of the triangle with vertices $P$, $Q$, and $R$ is given by
      \[%
        \frac{1}{2} \left\lvert \overrightarrow{PQ} \times \overrightarrow{PR} \right\rvert = \frac{1}{2} \sqrt{3^2 + 2^2 + (-5)^2} = \frac{1}{2} \sqrt{38}
      .\]%

    \item The general form of a plane equation is
      \[%
        a(x - x_0) + b(y - y_0) + c(z - z_0) = 0
      ,\]%
      where $\langle a, b, c \rangle$ is a normal vector to the plane, and
      $(x_0, y_0, z_0)$ is a point on the plane. We can use the point $P(2, 6,
      -1)$ and the normal vector $\langle 3, 2, -5 \rangle$ to find the equation
      of the plane $3(x - 2) + 2(y - 6) - 5(z + 1) = 0$. Simplifying gives us
      the final equation
      \[%
        3x + 2y - 5z - 23 = 0
      .\]%
  \end{enumerate}
\end{probsolution}

\newpage

\begin{problem}
  Find the equation of the plane that contains the points $A(2, 1, -1)$ and
  $B(0, -2, 4)$ and is perpendicular to the plane $2x - y + 3z = 10$.
\end{problem}

\begin{probsolution}
  We first need to find the direction vector
  \[%
    \overrightarrow{AB} = \langle 0 - 2, -2 - 1, 4 - (-1) \rangle = \langle -2, -3, 5 \rangle
  .\]%
  The normal vector to the plane $2x - y + 3z = 10$ is $\langle 2, -1, 3
  \rangle$. Since the plane we are looking for is perpendicular to the given
  plane, the normal vector to the plane we are looking for is the cross product
  of the direction vector $\overrightarrow{AB}$ and the normal vector $\langle
  2, -1, 3 \rangle$. Therefore, we have
  \[%
    \overrightarrow{AB} \times \langle 2, -1, 3 \rangle = \begin{vmatrix}
      \ui & \uj & \uk \\
      -2 & -3 & 5 \\
      2 & -1 & 3
    \end{vmatrix}
    = \ui\begin{vmatrix}
      -3 & 5 \\
      -1 & 3
    \end{vmatrix}
    - \uj\begin{vmatrix}
      -2 & 5 \\
      2 & 3
    \end{vmatrix}
    + \uk\begin{vmatrix}
      -2 & -3 \\
      2 & -1
    \end{vmatrix}
  .\]%
  Therefore, we have $\overrightarrow{AB} \times \langle 2, -1, 3 \rangle =
  \langle -4, 16, 8 \rangle$. We can use the point $A(2, 1, -1)$ and the normal
  vector $\langle -4, 16, 8 \rangle$ to find the equation of the plane $-4(x -
  2) + 16(y - 1) + 8(z + 1) = 0$. Simplifying gives us the final equation
  \[%
    x - 4y - 2z = 0
  .\]%
\end{probsolution}
