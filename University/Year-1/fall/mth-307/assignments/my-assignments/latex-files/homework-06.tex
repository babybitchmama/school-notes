\begin{problem}
  In each case, use mathematical notation to write the negation of the given statement, in such a way that no quantifier is immediately preceded by a negation sign. For parts (i) - (iv) decide which is true: the given statement or its negation.
  \begin{enumerate}
    \item $(\forall x \in \R)(\exists y \in \R)[x + y = 0]$.

    \item $(\exists x \in \R)(\forall y \in \R)[x + y = 0]$.

    \item $(\exists x, y \in \R)[x^2 + y^2 = -1]$.

    \item $(\forall x \in \R)[x > 0 \implies (\forall y, z \in \R)[(y > 0 \land z > 0 \land y^2 = x \land z^2 = x) \implies y = z]]$.

    \item $(\forall \epsilon \in \R)[\epsilon > 0 \implies (\exists \delta \in \R)[0 < \delta \land (\forall x \in \R)[1 - \delta < x < 1 + \delta \implies \lvert f(x) - 5 \rvert < \epsilon]]]$.

    \item $(\forall a, b \in \R)(a < b) \implies (\exists c \in \R)\left[a < c < b \land f'(c) = \frac{f(b) - f(a)}{b - a}\right]$.
  \end{enumerate}

  Part (vi) is a statement which is true for differentiable functions $f : \R
  \to \R$, and it is a well-known theorem taught in every calculus class. What
  is the common name of this theorem?
\end{problem}

\begin{probsolution}
  \begin{enumerate}
    \item The negated statement is $(\exists x \in \R)(\forall y \in \R)[x + y
      \ne 0]$. The original statement is true, because for every real number
      $x$, there is a real number $y = -x$ such that $x + y = 0$.

    \item The negated statement is $(\forall x \in \R)(\exists y \in \R)[x + y
      \ne 0]$. The negation is true because there is no real number $x$ such
      that for every real number $y$, $x + y = 0$.

    \item The negated statement is $(\forall x, y \in \R)[x^2 + y^2 \ne -1]$.
      The negation is true because the sum of two squares is always
      non-negative.

    \item The negated statement is $(\exists x \in \R)[x > 0 \land (\exists y, z
      \in \R)[(y > 0 \land z > 0 \land y^2 = x \land z^2 = x) \land y \neq z]]$.
      The original statement is true. For $x > 0$, if $y^2 = x$ and $z^2 = x$,
      then $y = z$ must hold when both $y$ and $z$ are positive.

    \item The negated statement is $(\exists \epsilon \in \R)[\epsilon > 0 \land
      (\forall \delta \in \R)[0 < \delta \implies (\exists x \in \R)[1 - \delta
      < x < 1 + \delta \land \lvert f(x) - 5 \rvert \geq \epsilon]]]$. Whether
      the original statement or its negation is true depends on the behavior of
      the function $f(x)$ near $x = 1$. If $f(x)$ is continuous at $x = 1$ and
      $f(1) = 5$, then the original statement is true.

    \item The negated statement is $(\exists a, b \in \R)\left[a < b \land
      (\forall c \in \R)\left[a < c < b \implies f'(c) \neq \frac{f(b) - f(a)}{b
    - a}\right]\right]$. The original statement is true for differentiable
    functions $f : \R \to \R$ and is known as the ``Mean Value Theorem''.
  \end{enumerate}
\end{probsolution}

\newpage

\begin{problem}
  In each part below I give the definition for a mathematical concept we have
  encountered, but using the shorthand notation in quantifiers. Fill in each box
  with the appropriate mathematical term or phrase that best completes the
  definition. In parts (ii)-(iv), $f : S \to T$ and $A \subseteq S$.
  \begin{enumerate}
    \item $\fbox{\parbox{2in}{\ \ \ \ \ \par \ \ }} \iff (\forall x \in A)[x \in
      B]$.

    \item $\fbox{\parbox{2in}{\ \ \ \ \ \par \ \ }} \iff (\forall a, b \in
      S)[f(a) = f(b) \implies a = b]$.

    \item $\fbox{\parbox{2in}{\ \ \ \ \ \par \ \ }} \iff (\forall t \in
      T)(\exists s \in S)[f(s) = t]$.

    \item $\fbox{\parbox{2in}{\ \ \ \ \ \par \ \ }} = \{z \mid (\exists v \in
      A)[z = f(v)]\}$.
  \end{enumerate}
\end{problem}

\begin{probsolution}
  \begin{enumerate}
    \item $A \subseteq B \iff (\forall x \in A)[x \in B]$.

    \item The function $f$ is one-to-one $\iff (\forall a, b \in S)[f(a) = f(b)
      \implies a = b]$.

    \item The function $f$ is onto $\iff (\forall t \in T)(\exists s \in S)[f(s)
      = t]$.

    \item Image of $A$ under $f$, $f(A) = \{z \mid (\exists v \in A)[z =
      f(v)]\}$.
  \end{enumerate}
\end{probsolution}

\newpage

\begin{problem}
  Suppose $f : S \to T$ is one-to-one, $A \subseteq S$, and $B \subseteq S$.
  Give a line proof showing that $f(A) \cap f(B) \subseteq f(A \cap B)$.
\end{problem}

\begin{probsolution}
  \begin{tabular}{ll}
    1. & Assume $z \in f(A) \cap f(B)$. \\
    2. & Then $z \in f(A)$ and $z \in f(B)$. \\
    3. & Then $f^{-1}(z) \in A$ and $f^{-1}(z) \in B$. \\
    4. & So $f^{-1}(z) \in A \cap B$. \\
    5. & Then $z = f(f^{-1}(z)) \in f(A \cap B)$. \\
    6. & Therefore, $f(A) \cap f(B) \subseteq f(A \cap B)$.
  \end{tabular}
\end{probsolution}

\newpage

\begin{problem}
  Give a line proof showing that if $A \cap B = \emptyset$ and $B \cup C = A
  \cup D$ then $B \subseteq D$.
\end{problem}

\begin{probsolution}
  \begin{tabular}{ll}
    1. & Assume $x \in B$. \\
    2. & Then, $x \notin A$ since $A$ and $B$ are disjoint. \\
    3. & Then, since $B \cup C = A \cup D$, $x \in D$. \\
    4. & Therefore, $B \subseteq D$. \\
  \end{tabular}
\end{probsolution}

\newpage

\begin{problem}
  Let $f : S \to T$ and suppose $f$ is onto. Let $A \subseteq S$. Give a line
  proof that $T - f(A) \subseteq f(S - A)$.
\end{problem}

\begin{probsolution}
  \begin{tabular}{ll}
    1. & Suppose $y \in T - f(A)$. \\
    2. & Since $f$ is onto, there exists some $x \in S$ such that $f(x) = y$. \\
    3. & Since $y \notin f(A)$, $x \notin A$. \\
    4. & Then, $x \in S - A$. \\
    5. & Therefore, $f(x) \in f(S - A)$, where $f(x) = y$. \\
    6. & Therefore, $T - f(A) \subseteq f(S - A)$. \\
  \end{tabular}
\end{probsolution}

\newpage

\begin{problem}
  Give a line proof showing $A \cap (X - B) = (A \cap X) - (A \cap B)$.
\end{problem}

\begin{probsolution}
  \begin{tabular}{ll}
    1. & Suppose $x \in A \cap (X - B)$. \\
    2. & Then, $x \in A$ and $x \in X - B$. \\
    3. & Since $x \in X - B$, then $x \in X$ and $x \notin B$. \\
    4. & Then, $x \in A \cap X$ and $x \notin A \cap B$. \\
    5. & Therefore, $x \in (A \cap X) - (A \cap B)$. \\
    6. & Therefore, $A \cap (X - B) \subseteq (A \cap X) - (A \cap B)$. \\
    7. & Suppose $x \in (A \cap X) - (A \cap B)$. \\
    8. & Then, $x \in A \cap X$ and $x \notin A \cap B$. \\
    9. & Since $x \in A \cap X$, then $x \in A$ and $x \in X$. \\
    10. & Since $x \notin A \cap B$, then $x \notin B$, since $x \in A$ on the previous line. \\
    11. & Then, $x \in A$ and $x \in X - B$. \\
    12. & Therefore, $x \in A \cap (X - B)$. \\
    13. & Therefore, $(A \cap X) - (A \cap B) \subseteq A \cap (X - B)$. \\
    14. & Therefore, $A \cap (X - B) = (A \cap X) - (A \cap B)$. \\
  \end{tabular}
\end{probsolution}

\newpage

\begin{problem}
  Let $f : \R \to \R$ be given by $f(x) = 2e^{-x} + 5$.
  \begin{enumerate}
    \item Prove that $f$ is one-to-one.

    \item Is $f$ onto? Justify your answer.

    \item Let $g : \R \to \R$ be given by $g(x) = 5x^3 + 41$. Prove that $g$ is
      onto.
  \end{enumerate}
\end{problem}

\begin{probsolution}
  \begin{enumerate}
    \item Let $a, b \in \R$ such that $f(a) = f(b)$. Then, $2e^{-a} + 5 =
      2e^{-b} + 5$. This implies $e^{-a} = e^{-b}$, which implies $a = b$.
      Therefore, $f$ is one-to-one.

    \item Let $y \in \R$. Suppose $y = f(x) = 2e^{-x} + 5$. Then $e^{-x} =
      \frac{y - 5}{2}$. The function $e^{-x}$ only takes positive values,
      specifically $e^{-x} > 0$ for all $x \in \R$. This implies that
      \[%
        \frac{y - 5}{2} > 0 \implies y > 5
      .\]%
      Therefore, $f$ is not onto.

    \item Let $y \in \R$. Suppose $y = g(x) = 5x^3 + 41$. Then,
      \[%
        x = \sqrt[3]{\frac{y - 41}{5}}
      .\]%
      Since the cube root function is defined for all real numbers, $g$ is onto.
  \end{enumerate}
\end{probsolution}

\newpage

\begin{problem}
  Suppose $f : A \to B$ and $g : B \to C$.
  \begin{enumerate}
    \item If $f$ and $g$ are one-to-one, prove that $g \circ f$ is onto.
    \item If $f$ and $g$ are both onto, prove that $g \circ f$ is onto.
  \end{enumerate}
\end{problem}

\begin{probsolution}
  \begin{enumerate}
    \item \begin{proof}
        Assume $f$ and $g$ are both one-to-one. Suppose $(g \circ f)(x_1) = (g
        \circ f)(x_2)$. Then, using the definition of composition, we have
        $g(f(x_1)) = g(f(x_2))$. Since $g$ is one-to-one, this implies $f(x_1) =
        f(x_2)$. Since $f$ is one-to-one, this implies $x_1 = x_2$. Therefore,
        $g \circ f$ is one-to-one.
      \end{proof}

    \item \begin{proof}
        Assume $f$ and $g$ are both onto. Let $z \in C$. Since $g$ is onto, then
        there exists some $y$ such that $g(y) = z$. Since $f$ is onto, then
        there exists some $x$ such that $f(x) = y$. Therefore, $(g \circ f)(x) =
        g(f(x)) = g(y) = z$. Therefore, $g \circ f$ is onto.
      \end{proof}
  \end{enumerate}
\end{probsolution}

\newpage

\begin{problem}
  Suppose $f : \R \to \R$ has the property that $(\forall x, y \in\ R)[x < y
  \implies f(x) < f(y)]$.
  \begin{enumerate}
    \item Prove that $f$ is one-to-one.

    \item Give an example of a function $f$ satisfying the given property but
      which is not onto.
  \end{enumerate}
\end{problem}

\begin{probsolution}
  \begin{enumerate}
    \item \begin{proof}
        Assume $f(x_1) = f(x_2)$, for some $x_1, x_2 \in \R$. By the trichotomy
        property of real numbers, we know that $x_1 < x_2$, $x_1 = x_2$, or $x_1
        > x_2$. If $x_1 < x_2$, then by the property of $f$, $f(x_1) < f(x_2)$.
        This contradicts our assumption that $f(x_1) = f(x_2)$. If $x_1 > x_2$,
        then by the property of $f$, $f(x_1) > f(x_2)$. This also contradicts
        our assumption that $f(x_1) = f(x_2)$. Therefore, $x_1 = x_2$.
        Therefore, $f$ is one-to-one.
      \end{proof}

    \item The function $\displaystyle f(x) = \frac{x}{x - 1}$, which holds the
      property that if $x_1 < x_2$, for some $x_1, x_2 \in \R$, then $f(x_1) <
      f(x_2)$. But $f(x)$ is not onto, since there doesn't exist any $x \in \R$
      such that $f(x) = 1$.
  \end{enumerate}
\end{probsolution}

\newpage

\begin{problem}
  \begin{enumerate}
    \item If $f : \Z \to \Z$ is given by $f(x) = x^2 - 1$ and $g : \Z \to \Z$ is
      given by $g(x) = 3x + 2$, determine $(g \circ f)(0)$ and $(g \circ f)(2)$.
      Determine an algebraic formula for $(g \circ f)(x)$ for any integer $x$.

    \item Suppose that $f : S \to T$ and $g : T \to U$. If $A \subseteq S$, give
      a line proof that $(g \circ f)(A) = g(f(A))$.
  \end{enumerate}
\end{problem}

\begin{probsolution}
  \begin{enumerate}
    \item The composition $(g \circ f)(x)$ is given by
      \[%
        (g \circ f)(x) = g(f(x)) = g(x^2 - 1) = 3(x^2 - 1) + 2 = 3x^2 - 1
      .\]%
      Therefore, $(g \circ f)(0) = -1$ and $(g \circ f)(2) = 11$.

    \item Here's the line proof for $(g \circ f)(A) = g(f(A))$.

      \hspace*{-0.5em}\begin{tabular}{ll}
        1. & By definition, $(g \circ f)(A) = \{(g \circ f) \mid a \in A\}$. \\
        2. & Then, $(\forall a \in A)[(g \circ f)(a) = g(f(a)]$. \\
        3. & Thus, $(g \circ f)(A) = \{g(f(a)) \mid a \in A\}$. \\
        4. & By definition, $g(f(A)) = \{g(f(a)) \mid a \in A\}$. \\
        5. & Therefore, $g(f(A)) = \{g(t) \mid t \in f(A)\} = \{g(f(a)) \mid a \in A\}$. \\
        6. & Therefore, $(g \circ f)(A) = g(f(A))$. \\
      \end{tabular}
  \end{enumerate}
\end{probsolution}

\newpage

\begin{problem}
  Consider the function $f : \Z_7 \to \Z_7$ given by $f(x) = x^3 + 1$. Answer
  the following questions
  \begin{enumerate}
    \item Is $f$ is one-to-one? Explain why or why not.

    \item Is $f$ onto? Explain why or why not.

    \item Determine $f(S)$, where $S = \{0, 2, 4, 6\}$.

    \item What is $f^{-1}(\{0\})$.

    \item If $A = \{1, 2, 3, 4\}$ and $B = \{0, 4, 5, 6\}$, determine
      $f^{-1}(A)$ and $f^{-1}(B)$. Also, determine $f^{-1}(A \cap B)$.
  \end{enumerate}
\end{problem}

\begin{probsolution}
  \begin{enumerate}
    \item To check if $f$ is one-to-one, we must create a table of values to see
      if any two distinct elements in the domain map to the same element in the
      codomain. The table is shown below.

      \begin{tabular}{c|ccccccc}
        $x$ & $0$ & $1$ & $2$ & $3$ & $4$ & $5$ & $6$ \\
        $f(x)$ & $1$ & $2$ & $2$ & $0$ & $2$ & $0$ & $1$ \\
      \end{tabular}

      Since $f(3) = f(5) = 0$, $f(0) = f(6) = 1$, and $f(1) = f(2) = f(4) = 2$,
      $f$ is not one-to-one.

    \item To check if $f$ is onto, we must check if every element in the
      codomain is mapped to by some element in the domain. The table of values
      shows that $f$ is not onto, since $f$ does not map $3$, $4$, $5$, or $6$
      to any element in the codomain.

    \item The set $f(S)$ is given by $\{f(0), f(2), f(4), f(6)\} = \{1, 2, 2,
      1\} = \{0, 1, 2\}$.

    \item The set $f^{-1}(\{0\})$ is given by $\{x \in \Z_7 \mid f(x) = 0\} =
      \{3, 5\}$.

    \item The set $f^{-1}(A) = \{f(1), f(2), f(3), f(4)\} = \{2, 2, 0, 4\} =
      \{0, 2, 4\}$ and $f^{-1}(B) = \{f(0), f(4), f(5), f(6)\} = \{1, 2, 0,
      1\}$. The set $f^{-1}(A \cap B) = \{f(4)\} = \{2\}$.
  \end{enumerate}
\end{probsolution}

\newpage

\begin{problem}
  Suppose $f : S \to T$, $A \subseteq S$, and $B \subseteq T$. Give a line proof
  for each of the following
  \begin{enumerate}
    \item $f(A) \subseteq B \implies A \subseteq f^{-1}(B)$.
    \item $f(A) \cap B = \emptyset \implies A \subseteq - f^{-1}(B)$.
  \end{enumerate}
\end{problem}

\begin{probsolution}
  \begin{enumerate}
    \item Here's the line proof for $f(A) \subseteq B \implies A \subseteq
      f^{-1}(B)$.

      \hspace*{-0.5em}\begin{tabular}{ll}
        1. & Suppose $x \in A$. \\
        2. & Then, $f(x) \in f(A)$. \\
        3. & Since $f(A) \subseteq B$, then $f(x) \in B$. \\
        4. & Then, $x \in f^{-1}(B)$. \\
        5. & Then, $A \subseteq f^{-1}(B)$. \\
        6. & Therefore, $f(A) \subseteq B \implies A \subseteq f^{-1}(B)$. \\
      \end{tabular}

    \item Here's the line proof for $f(A) \cap B = \emptyset \implies A \subseteq
      - f^{-1}(B)$.

      \hspace*{-0.5em}\begin{tabular}{ll}
        1. & Suppose $x \in A$. \\
        2. & Then, $f(x) \in f(A)$. \\
        3. & Then, $f(x) \notin B$. \\
        4. & Then, $x \notin f^{-1}(B)$. \\
        5. & Then, $A \subseteq - f^{-1}(B)$. \\
        6. & Therefore, $f(A) \cap B = \emptyset \implies A \subseteq - f^{-1}(B)$. \\
      \end{tabular}
  \end{enumerate}
\end{probsolution}

\newpage

\begin{problem}
  Suppose $f : S \to T$, $A \subseteq T$, $B \subseteq T$, and $C \subseteq S$.
  Give a line proof of each of the following
  \begin{enumerate}
    \item $f^{-1}(A \cap B) = f^{-1}(A) \cap f^{-1}(B)$.
    \item $f^{-1}(A \cup B) = f^{-1}(A) \cup f^{-1}(B)$.
    \item $f(f^{-1}(A)) \subseteq A$.
    \item $C \subseteq f^{-1}(f(c))$.
    \item If $f$ is onto, then $f(f^{-1}(A)) = A$.
    \item If $f$ is one-to-one, then $C = f^{-1}(f(C))$.
  \end{enumerate}
\end{problem}

\begin{probsolution}
  \begin{enumerate}
    \item Here's the line proof for $f^{-1}(A \cap B) = f^{-1}(A) \cap
      f^{-1}(B)$.

      \hspace*{-0.5em}\begin{tabular}{ll}
        1. & Suppose $x \in f^{-1}(A \cap B)$. \\
        2. & Then, $f(x) \in A \cap B$. \\
        3. & Then, $f(x) \in A$ and $f(x) \in B$. \\
        4. & Then, $x \in f^{-1}(A)$ and $x \in f^{-1}(B)$. \\
        5. & Therefore, $x \in f^{-1}(A) \cap f^{-1}(B)$. \\
        6. & Therefore, $f^{-1}(A \cap B) \subseteq f^{-1}(A) \cap f^{-1}(B)$. \\
        7. & Suppose $x \in f^{-1}(A) \cap f^{-1}(B)$. \\
        8. & Then, $x \in f^{-1}(A)$ and $x \in f^{-1}(B)$. \\
        9. & Then, $f(x) \in A$ and $f(x) \in B$. \\
        10. & Then, $f(x) \in A \cap B$. \\
        11. & Therefore, $x \in f^{-1}(A \cap B)$. \\
        12. & Therefore, $f^{-1}(A) \cap f^{-1}(B) \subseteq f^{-1}(A \cap B)$. \\
        13. & Therefore, $f^{-1}(A \cap B) = f^{-1}(A) \cap f^{-1}(B)$. \\
      \end{tabular}

    \item Here's the line proof for $f^{-1}(A \cup B) = f^{-1}(A) \cup
      f^{-1}(B)$.

      \hspace*{-0.5em}\begin{tabular}{ll}
        1. & Suppose $x \in f^{-1}(A \cup B)$. \\
        2. & Then, $f(x) \in A \cup B$. \\
        3. & Then, $f(x) \in A$ or $f(x) \in B$. \\
        4. & Then, $x \in f^{-1}(A)$ or $x \in f^{-1}(B)$. \\
        5. & Therefore, $x \in f^{-1}(A) \cup f^{-1}(B)$. \\
        6. & Therefore, $f^{-1}(A \cup B) \subseteq f^{-1}(A) \cup f^{-1}(B)$. \\
        7. & Suppose $x \in f^{-1}(A) \cup f^{-1}(B)$. \\
        8. & Then, $x \in f^{-1}(A)$ or $x \in f^{-1}(B)$. \\
        9. & Then, $f(x) \in A$ or $f(x) \in B$. \\
        10. & Then, $f(x) \in A \cup B$. \\
        11. & Therefore, $x \in f^{-1}(A \cup B)$. \\
        12. & Therefore, $f^{-1}(A) \cup f^{-1}(B) \subseteq f^{-1}(A \cup B)$. \\
        13. & Therefore, $f^{-1}(A \cup B) = f^{-1}(A) \cup f^{-1}(B)$. \\
      \end{tabular}

    \item Here's the line proof for $f(f^{-1}(A)) \subseteq A$.

      \hspace*{-0.5em}\begin{tabular}{ll}
        1. & Suppose $y \in f(f^{-1}(A))$. \\
        2. & Then, $y = f(x)$ for some $x \in f^{-1}(A)$. \\
        3. & Then, $x \in f^{-1}(A)$. \\
        4. & Then, $f(x) \in A$. \\
        5. & Therefore, $y \in A$. \\
        6. & Therefore, $f(f^{-1}(A)) \subseteq A$. \\
      \end{tabular}

    \item Here's the line proof for $C \subseteq f^{-1}(f(C))$.

      \hspace*{-0.5em}\begin{tabular}{ll}
        1. & Suppose $x \in C$. \\
        2. & Then, $f(x) \in f(C)$. \\
        3. & Then, $x \in f^{-1}(f(C))$. \\
        4. & Therefore, $C \subseteq f^{-1}(f(C))$. \\
      \end{tabular}

    \item Here's the line proof for if $f$ is onto, then $f(f^{-1}(A)) = A$.

      \hspace*{-0.5em}\begin{tabular}{ll}
        1. & From (iii), we know that $f(f^{-1}(A)) \subseteq A$. \\
        2. & If $f$ is onto, then for $y \in A$, there exists $x \in f^{-1}(A)$ such that $f(x) = y$. \\
        3. & Thus, $y \in f(f^{-1}(A))$. \\
        4. & So $A \subseteq f(f^{-1}(A))$. \\
        5. & Therefore, $f(f^{-1}(A)) = A$. \\
      \end{tabular}

    \item Here's the line proof for if $f$ is one-to-one, then $C = f^{-1}(f(C))$.

      \hspace*{-0.5em}\begin{tabular}{ll}
        1. & From (iv), we know that $C \subseteq f^{-1}(f(C))$. \\
        2. & If $f$ is one-to-one and $x \in f^{-1}(f(C))$, then $f(x) = f(c)$, for some $c \in C$. \\
        3. & This implies that $x = c$. \\
        4. & So, $f^{-1}(f(C)) \subseteq C$. \\
        5. & Therefore, $C = f^{-1}(f(C))$. \\
      \end{tabular}
  \end{enumerate}
\end{probsolution}

\newpage

\begin{problem}
  Construct an example of a function $f : \{0, 1, 2\} \to \{0, 1\}$ and a subset
  $A \subseteq \{0, 1\}$ where $f(f^{-1}(A)) \ne A$. Also construct an example
  of a function $g : \{0, 1, 2\} \to \{0, 1\}$ and a subset $C \subseteq \{0, 1,
  2\}$ where $C \ne g^{-1}(g(C))$.
\end{problem}

\begin{probsolution}
  We cannot construct such examples, because for any function $f : S \to T$ and
  any subset $A \subseteq T$, we have $f(f^{-1}(A)) = A$.

  We cannot construct such examples, because for any function $g : S \to T$ and
  any subset $C \subseteq S$, we have $C = g^{-1}(g(C))$.
\end{probsolution}

\newpage

\begin{problem}
  Suppose $f, g : \Z \to \Z$ are two functions such that $f(M_3) \subseteq M_6$,
  $g(M_2) \subseteq M_7$, and $g^{-1}(M_5) = M_3$. Prove that for all $x \in
  \Z$, if $3 \mid x$ then $35 \mid g(f(x))$.
\end{problem}

\begin{probsolution}
  \begin{tabular}{ll}
    1. & Assume $x \in \Z$ such that $3 \mid x$. \\
    2. & By the property of $f$, $f(M_3) \subseteq M_6$. \\
    3. & Thus, $f(x) \in M_6$. \\
    4. & Since $6 \mid f(x)$, we have $2 \mid f(x)$ and $3 \mid f(x)$. \\
    5. & Because $2 \mid f(x)$, $f(x) \in M_2$. \\
    6. & By condition 2, $g(M_2) \subseteq M_7$. \\
    7. & Thus, $g(f(x)) \in M_7$. \\
    8. & This means $7 \mid g(f(x))$. \\
    9. & By condition 3, $g^{-1}(M_5) = M_3$. \\
    10. & So, $f(x) \in M_3$, which means $g(f(x) \in M_5$. \\
    11. & This means $5 \mid g(f(x)$. \\
    12. & Since $g(f(x))$ is divisible by both $5$ and $7$, then $35 \mid g(f(x))$. \\
    13. & Therefore, for all $x \in \Z$, if $3 \mid x$ then $35 \mid g(f(x))$. \\
  \end{tabular}
\end{probsolution}

\newpage

\begin{problem}
  Given $[Q \land S] \implies R$ and $\neg S \implies T$, prove $[P \implies Q]
  \implies [\neg T \implies [\neg P \lor R]]$
\end{problem}

\begin{probsolution}
  \begin{tabular}{lll}
    1. & $[Q \land S] \implies R$ & Hypothesis \\
    2. & $\neg S \implies T$ & Hypothesis \\
    3. & Assume $P \implies Q$. & Dischargeable Hypothesis \\
    4. & Assume $\neg T$. & Dischargeable Hypothesis \\
    5. & Assume $P \land \neg R$. & Dischargeable Hypothesis \\
    6. & $\neg R$. & RCS, for 5 \\
    7. & $P$ & LCS, for 5 \\
    8. & $\neg[Q \land S]$. & MT, for 6, for 1 \\
    9. & $\neg[Q \land S] \iff \neg Q \lor \neg S$. & Tautology \\
    10. & $\neg Q \lor \neg S$. & MPB, for 8, for 9 \\
    11. & $Q$. & MP, for 7, for 3 \\
    12. & $\neg S$. & DI, for 10, for 11 \\
    13. & $T$. & MP, for 12, for 2 \\
    14. & $\neg T \land T$ & CI, for 4, for 13 \\
    15. & $\neg [P \land \neg R]$. & II, discharge for 5 [5 - 14 unusable] \\
    16. & $\neg [P \land \neg R] \iff \neg P \lor R$. & Tautology \\
    17. & $\neg P \lor R$. & MPB, for 15, for 16 \\
    15. & $\neg T \implies [\neg P \lor R]$. & DT, discharge for 4 [4 - 17 unusable] \\
    16. & $[P \implies Q] \implies [\neg T \implies [\neg P \lor R]]$. & DT, discharge for 3 [3 - 15 unusable] \\
  \end{tabular}
\end{probsolution}
