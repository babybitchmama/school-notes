\begin{problem}
  Identify each of the following statements as true or false. Where you can,
  prove the statement by giving an example or disprove it by giving a
  counterexample.
  \begin{enumerate}
    \item $(\exists x \in \Z_9)[x^2 \in \{5, 7\}]$.

    \item $(\forall a \in \Z)[a^2 \equiv_{11} 16 \implies a \equiv_{11} 4]$.

    \item $(\exists n \in \N)[\frac{1}{2}(n^2 + n) + 2~\textrm{is prime}]$.

    \item $(\forall n \in \N)[\frac{1}{2}(n^2 + n) + 2~\textrm{is prime}]$.

    \item $(\exists a, b \in \Z)[12a + 20b = 4]$.

    \item $\{x \mid x \in \Z_8 \land 4 \cdot x = 0 \pmod{8}\} \cap \{4 \cdot x
      \pmod{8} \mid x \in \Z_8\} = \emptyset$.

    \item $(\forall n \in \N)[(n \equiv_2 1 \land n > 3) \implies 3 \mid n^2 -
      1]$.

    \item $(\forall a, b \in \Z)[12 \mid ab \implies (12 \mid a \lor 12 \mid
      b)]$.
  \end{enumerate}
\end{problem}

\begin{probsolution}
  \begin{enumerate}
    \item True. Example: $4^2 \equiv_9 7$ and $5^2 \equiv_9 7$.

    \item False. Counterexample: $7^2 \equiv_{11} 49 \equiv_{11} 16$ since $7
      \not\equiv_{11} 4$.

    \item True. Example: $n = 1$ and $n = 2$.

    \item False. Counterexample: $n = 4$.

    \item True. Example: $a = 1$ and $b = -1$.

    \item False. $\{0, 2\} \cap \{0, 4\} = \{0\} \ne \emptyset$.

    \item True. For $n \equiv_2 1$ and $n > 3$, $n$ is an odd integer greater
      than $3$. For any odd $n$, $n^2 - 1$ can be factored as $(n - 1)(n + 1)$.
      Since $n$ is odd, both $n - 1$ and $n + 1$ are even. So one of them is
      divisible by $3$. Thus $3 \mid n^2 - 1$, for all add $n > 3$.

    \item False. Counterexample: $a = 6$ and $b = 4$.
  \end{enumerate}
\end{probsolution}

\newpage

\begin{problem}
  Given a line proof that $(\forall n \in \N)[n^2 + (n + 1)^2 + (n + 2)^2
  \equiv_3 2$.
\end{problem}

\begin{probsolution}
  \begin{table}[H]
    \begin{tabular}{lll}
      1. & Assume $n \in \N$. & Dischargeable Hypothesis \\
      2. & Then, $n^2 + (n + 1)^2 + (n + 2)^2 = 3(n^2 + 2n + 1) - 2$. \\
      3. & $(\exists r \in \Z)[n^2 + (n + 1)^2 + (n + 2)^2 - 2 = 3r]$. & EI \\
      4. & $3 \mid (n^2 + (n + 1)^2 + (n + 2)^2 - 2)$. \\
      5. & $n^2 + (n + 1)^2 + (n + 2)^2 \equiv_3 2$. \\
      6. & $n \in \N \implies n^2 + (n + 1)^2 + (n + 2)^2 \equiv_3 2$. & DT, discharge from 1 [(1) - (6) unusable] \\
      7. & $(\forall n \in \N)[n^2 + (n + 1)^2 + (n + 2)^2 \equiv_3 2]$ & IU \\
    \end{tabular}
  \end{table}
\end{probsolution}

\newpage

\begin{problem}
  Decide if the following statement is true or false. If it is true, give a line
  proof. If it is false, give a counterexample.
  \[%
    (\forall a, b, c \in \Z)[(a > 0 \land a \mid (b - 1) \land a \mid (c - 1)) \implies a \mid (bc - 1)]
  .\]%
\end{problem}

\begin{probsolution}
  \begin{table}[H]
    \begin{tabular}{lll}
      1. & Assume $a > 0$, $a \mid (b - 1)$, and $a \mid (c - 1)$. & Dischargeable Hypothesis \\
      2. & Then, $b - 1 = ak$ and $c - 1 = aq$, for some $k, q \in \Z$. \\
      3. & Then, $b = ak + 1$ and $c = aq + 1$. \\
      4. & $bc = (ak + 1)(aq + 1) = a(akq + k + q) + 1$. \\
      5. & $(\exists r \in \Z)[bc - 1 = ar]$. & EI \\
      6. & $a \mid (bc - 1)$. \\
      7. & $(a > 0 \land a \mid (b - 1) \land a \mid (c - 1)) \implies a \mid (bc - 1)$ & DT, discharge from 1 [(1) - (7) unusable] \\
      8. & $(\forall a, b, c \in \Z)[(a > 0 \land a \mid (b - 1) \land a \mid (c - 1)) \implies a \mid (bc - 1)]$ & IU \\
    \end{tabular}
  \end{table}
\end{probsolution}

\newpage

\begin{problem}
  Here is an important property of prime numbers. If $p$ is prime, then
  \[%
    \textrm{Property (P):}\quad(\forall x, y \in \Z)[p \mid xy \implies (p \mid x \lor p \mid y)]
  .\]%
  Using this, fill in the blanks below to give a proof of the following theorem:

  Theorem: $(\forall y \in \Z)[4y^2 \equiv_7 0 \implies y \equiv_7 0]$.
\end{problem}

\begin{probsolution}
  \begin{table}[H]
    \begin{tabular}{lll}
      1. & Assume $\dots$. & Dischargeable Hypothesis \\
    \end{tabular}
  \end{table}
\end{probsolution}

\newpage

\begin{problem}
  Give a line proof that $(\forall n \in \N)[(3 \mid n \land n \equiv_5 3)
  \implies n^2 + n \equiv_{15} 12]$.
\end{problem}

\begin{probsolution}
  \begin{table}[H]
    \begin{tabular}{lll}
      1. & Assume $n \in \N$, $3 \mid n$, and $n \equiv_5 3$. \\
      2. & $(\exists r \in \Z)[n^2 + n - 12 = 5r]$. & EI \\
      3. & $5 \mid (n^2 + n - 12)$. \\
      4. & $n = 5k + 3$, for some $k \in \Z$. \\
      5. & $(5k + 3^{2} + (5k + 3) = 5(5k^2 + 7k) + 12$. \\
      6. & $(\exists r \in \Z)[n^2 + n - 12 = 5r]$. & EI \\
      7. & $3 \mid 5k + 3$. \\
      8. & $(\exists l \in \Z)[5k + 3 = 3l]$. \\
      9. & $5k = 3(l - 1)$. \\
      10. & $(\exists m \in \Z)[5k = 3m]$. \\
      11. & By Property P, we get $3 \mid 5$ or $3 \mid k$. \\
      12. & Since $3 \nmid 5$, then $3 \mid k$. \\
      13. & Then $n^2 + n - 12 = 5k(5k + 7)$. \\
      14. & $(\exists n \in \Z)[k = 3n]$. \\
      15. & $n^2 + n -12 = 3 \cdot (5n \cdot (5 \cdot 3n + 7))$. \\
      16. & $3 \mid (n^2 + n - 12)$. \\
      17. & $15 \mid (n^2 + n - 12)$. \\
      18. & $n^2 + n \equiv_{15} 12$. \\
      19. & $(3 \mid n \land n \equiv_5 3) \implies n^2 + n \equiv_{15} 12$. & DT, discharge from 1 [(1) - (19) unusable] \\
      20. & $(\forall n \in \N)[(3 \mid n \land n \equiv_5 3) \implies n^2 + n \equiv_{15} 12]$ & IU \\
    \end{tabular}
  \end{table}
\end{probsolution}

\newpage

\begin{problem}
  A \textit{rational number} is a number of the form $\frac{a}{b}$ where $a, b
  \in \Z$ and $b \ne 0$. As you learned in elementary school, a rational number
  can always be written in the form where $\gcd(a, b) = 1$. Fill in the blanks
  below to give a proof of the following theorem; feel free to insert extra
  steps if you think it will help clarify the proof.

  Theorem: $\sqrt{2}$ is not a rational number.
\end{problem}

\begin{probsolution}
\end{probsolution}

\newpage

\begin{problem}
  Give a line proof showing that $\sqrt{10}$ is not a rational number.
\end{problem}

\begin{probsolution}
\end{probsolution}

\newpage

\begin{problem}
  A $3 \times 3$ grid has $14$ squares in it:

  There are $1 \times 1$ squares, $2 \times 2$ squares, and $3 \times 3$
  squares, and if you count all the squares that you see in the above grid you
  should get $14$.

  Figure out how many squares there are in a $10 \times 10$ grid, and explain
  your answer. Given a exact number, not just a formula for computing it.

  Hints for doing this: Get a sense of the problem by tackling smaller version.
  Try a $2 \times 2$ grid, you already did the $3 \times 3$ grid, maybe look at
  $4 \times 4$ and $5 \times 5$ grids. Analyze these smaller problems and try to
  find some underlying patterns.
\end{problem}

\begin{probsolution}
\end{probsolution}
