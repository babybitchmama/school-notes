\begin{problem}[1]
  Given $Q \implies R$, prove $[P \implies T] \implies [(Q \lor \neg T) \implies
  (\neg P \lor R)]$.
\end{problem}

\begin{proof}[Solution]
  $\textrm{%
    \begin{tabular}[t]{lll}
      1. & $Q \implies R$ & Hypothesis \\
      2. & Assume $P \implies T$ & Dischargeable hypothesis \\
      3. & Assume $Q \lor \neg T$ & Dischargeable hypothesis \\
      4. & Assume $P \land \neg R$ & Dischargeable hypothesis \\
      5. & $P$ & LCS, for 4 \\
      6. & $\neg R$ & RCS, for 4 \\
      7. & $T$ & MP, for 5, for 2 \\
      8. & $\neg T$ & DI, for 3, for 7 \\
      9. & $T \land \neg T$ & CI, for 7, for 8 \\
      10. & $\neg [P \land \neg R]$ & II, discharge for 4 [4 - 9 unusable] \\
      11. & $\neg [P \land \neg R] \iff \neg P \lor R$ & Tautology \\
      12. & $\neg P \lor R$ & MPB, for 10, for 11 \\
      13. & $[Q \lor \neg T] \implies (\neg P \lor R)$ & DT, discharge for 3 [3 - 12 unusable] \\
      14. & $[P \implies T] \implies [(Q \lor \neg T) \implies (\neg P \lor R)]$ & DT, discharge for 2 [2 - 13 unusable] \\
    \end{tabular}\qedhere
  }$%
\end{proof}

\medskip

\begin{problem}[2]
  \begin{enumerate}
    \item If $C \subseteq A$ and $D \subseteq B$, then prove $D - A \subseteq B
      - C$.

    \item Prove $A = X \cap A$ if and only if $A \subseteq X$.

    \item Prove $A = X \cup A$ if and only if $X \subseteq A$.
  \end{enumerate}
\end{problem}

\begin{proof}[Solution to (i)]
  $\textrm{
    \begin{tabular}[t]{ll}
      1. & Assume $C \subseteq A$ and $D \subseteq B$. \\
      2. & Assume $x \in D - A$. \\
      3. & Then, $x \in D$ and $x \notin A$. \\
      4. & Since $C \subseteq A$ and $D \subseteq B$, then $x \in B$ and $x \notin C$. \\
      5. & By the definition of set difference, $x \in B - C$. \\
      6. & Hence, $D - A \subseteq B - C$. \\
      7. & Therefore, $(C \subseteq A \land D \subseteq B) \implies (D - A \subseteq B - C)$. \\
    \end{tabular}\qedhere
  }$
\end{proof}

\begin{proof}[Solution to (ii)]
  $\textrm{
    \begin{tabular}[t]{ll}
      1. & Assume $A = X \cap A$. \\
      2. & Assume $x \in A$. \\
      3. & Then $x \in X$ and $x \in A$. \\
      4. & Hence, $x \in X$. \\
      5. & Therefore, $A \subseteq X$. \\
      6. & Assume $A \subseteq X$. \\
      7. & Then, by the definition of a subset, $x \in A \implies x \in X$. \\
      8. & Thus, $x \in A \implies x \in X \cap A$. \\
      9. & Similarly, if $x \in X \cap A$, then $x \in A$. \\
      10. & Therefore, $A = X \cap A$. \\
      11. & Therefore, $A = X \cap A \iff A \subseteq X$. \\
    \end{tabular}\qedhere
  }$
\end{proof}

\begin{proof}[Solution to (iii)]
  $\textrm{
    \begin{tabular}[t]{ll}
      1. & Assume $A = X \cup A$. \\
      2. & Assume $x \in X \cup A = A$. \\
      3. & If $x \in X$, then $x \in A$. \\
      4. & Hence, $X \subseteq A$. \\
      5. & Assume $X \subseteq A$. \\
      6. & Then, by the definition of a subset, $x \in X \implies x \in A$. \\
      7. & Then, $x \in X \cup A$. \\
      8. & Therefore, $A = X \cup A$. \\
    \end{tabular}\qedhere
  }$
\end{proof}

\medskip

\begin{problem}[3]
  Let $f : S \to T$ be a function. Prove that if $X \subseteq T$ and $Y
  \subseteq T$, then $f^{-1}(X) - f^{-1}(Y) = f^{-1}(X - Y)$.
\end{problem}

\begin{proof}[Solution]
  $\textrm{
    \begin{tabular}[t]{ll}
      1. & Assume $X \subseteq T$ and $Y \subseteq T$ \\
      2. & Assume $x \in f^{-1}(X) - f^{-1}(Y)$ \\
      3. & Then, $x \in f^{-1}(X)$ and $x \notin f^{-1}(Y)$ \\
      4. & Then, $f(x) \in X$ and $f(x) \notin Y$ \\
      5. & Then, $f(x) \in X - Y$ \\
      6. & Then, $x \in f^{-1}(X - Y)$ \\
      7. & Therefore, $f^{-1}(X) - f^{-1}(Y) \subseteq f^{-1}(X - Y)$ \\
      8. & Assume $x \in f^{-1}(X - Y)$ \\
      9. & Then, $f(x) \in X - Y$ \\
      10. & Then, $f(x) \in X$ and $f(x) \notin Y$ \\
      11. & Then, $x \in f^{-1}(X)$ and $x \notin f^{-1}(Y)$ \\
      12. & Then, $x \in f^{-1}(X) - f^{-1}(Y)$ \\
      13. & Therefore, $f^{-1}(X - Y) \subseteq f^{-1}(X) - f^{-1}(Y)$ \\
      14. & Therefore, $f^{-1}(X) - f^{-1}(Y) = f^{-1}(X - Y)$ \\
    \end{tabular}\qedhere
  }$
\end{proof}

\medskip

\begin{problem}[4]
  Let $f : S \to T$ be a function, let $A \subseteq S$ and $B \subseteq S$.
  \begin{enumerate}
    \item Prove $f(A) - f(B) \subseteq f(A - B)$.

    \item If $f$ is one-to-one, prove $f(A - B) \subseteq f(A) - f(B)$.

    \item Create an example of an $S$, $T$, $f$, $A$, and $B$ such that $f(A) -
      f(B) \ne f(A - B)$.
  \end{enumerate}
\end{problem}

\begin{proof}[Solution to (i)]
  $\textrm{
    \begin{tabular}[t]{ll}
      1. & Let $f : S \to T$ be a function. \\
      2. & Assume $A \subseteq S$ and $B \subseteq S$. \\
      3. & Assume $y \in f(A) - f(B)$. \\
      4. & Then, $y \in f(A)$ and $y \notin f(B)$. \\
      5. & Then, there exists $x \in A$ such that $f(x) = y$. \\
      6. & There does not exist $x \in B$ such that $f(x) = y$. \\
      7. & Then, $x \in A$ and $x \notin B$. \\
      8. & Then, $x \in A - B$. \\
      9. & Then, $y \in f(A - B)$. \\
      10. & Therefore, $f(A) - f(B) \subseteq f(A - B)$. \\
    \end{tabular}\qedhere
  }$
\end{proof}

\begin{proof}[Solution to (ii)]
  $\textrm{
    \begin{tabular}[t]{ll}
      1. & Let $f : S \to T$ be a one-to-one function. \\
      2. & Assume $A \subseteq S$ and $B \subseteq S$. \\
      3. & Assume $y \in f(A - B)$. \\
      4. & Then, there exists $x \in A - B$ such that $f(x) = y$. \\
      5. & Then, $x \in A$ and $x \notin B$. \\
      6. & Then, $f(x) \in f(A)$ and $f(x) \notin f(B)$. \\
      7. & Then, $y \in f(A) - f(B)$. \\
      8. & Therefore, $f(A - B) \subseteq f(A) - f(B)$. \\
    \end{tabular}\qedhere
  }$
\end{proof}

\begin{proof}[Solution to (iii)]
  Let $S = \{1, 2, 3, 4, 5\}$ and $T = \R$. Let
  \[%
    f(X) = \Card(X) + 1
  .\]%
  Let $A = \{1, 2, 3\}$ and $B = \{2, 3\}$. Then, $A - B = \{1\}$. Then, we get
  \[%
    f(A) = 4, \quad f(B) = 3, \quad f(A) - f(B) = 1, \aand f(A - B) = 2
  .\]%
  Therefore, $f(A) - f(B) \ne f(A - B)$.
\end{proof}

\medskip

\begin{problem}[5]
  Suppose $f : A \to B$, $X \subseteq A$, $W \subseteq B$, $f(X) \cap W = \emptyset$, and $f(X) \cup W = B$.
  \begin{enumerate}
    \item Prove that $X \cap f^{-1}(W) = \emptyset$.
    \item If $f$ is one-to-one, prove that $A = X \cup f^{-1}(W)$.
    \item If $f(A - X) = W$, prove that $f$ is onto.
  \end{enumerate}
\end{problem}

\begin{proof}[Solution to (i)]
  $\textrm{
    \begin{tabular}[t]{ll}
      1. & Assume $f : A \to B$, $X \subseteq A$, $W \subseteq B$, $f(X) \cap W = \emptyset$, and $f(X) \cup W = B$. \\
      2. & Assume $x \in X \cap f^{-1}(W)$. \\
      3. & Then, $x \in X$ and $x \in f^{-1}(W)$. \\
      4. & Then, $f(x) \in f(X)$ and $f(x) \in W$. \\
      5. & That implies that $f(x) \in f(X) \cap W$. \\
      6. & But that's a contradiction from the fact that $(X) \cap W = \emptyset$. \\
      7. & Therefore, $X \cup f^{-1}(W) = \emptyset$.
    \end{tabular}\qedhere
  }$
\end{proof}

\begin{proof}[Solution to (ii)]
  $\textrm{
    \begin{tabular}[t]{ll}
      1. & Assume $f: A \to B$, $X \subseteq A$, $W \subseteq B$, $f(X) \cap W = \emptyset$, and $f(X) \cup W = B$. \\
      2. & Take any $a \in A$. \\
      3. & Then $f(a) \in B$. By $f(X) \cup W = B$, we have $f(a) \in f(X)$ or $f(a) \in W$. \\
      4. & If $f(a) \in f(X)$, then $a \in X$ (since $f$ is one-to-one). \\
      5. & If $f(a) \in W$ then, $a \in f^{-1}(W)$. \\
      6. & Thus, $a \in X \cup f^{-1}(W)$. \\
      7. & Therefore, $A \subseteq X \cup f^{-1}(W)$. \\
      8. & Conversely, observe that $X \subseteq A$ and $f^{-1}(W) \subseteq A$. \\
      9. & Thus, $X \cup f^{-1}(W) \subseteq A$. \\
      10. & Combining both inclusions, $A = X \cup f^{-1}(W)$. \\
    \end{tabular}\qedhere
  }$
\end{proof}

\begin{proof}[Solution to (iii)]
  $\textrm{
    \begin{tabular}[t]{ll}
      1. & Assume $f: A \to B$, $X \subseteq A$, $W \subseteq B$, $f(X) \cap W = \emptyset$, and $f(X) \cup W = B$. \\
      2. & Assume $f(A - X) = W$. \\
      3. & Definition of onto is $(\forall b \in B)(\exists a \in A)[f(a) = b]$. \\
      4. & Take any $b \in B$. \\
      5. & Since $f(X) \cup W = B$, $b \in f(X)$ or $b \in W$. \\
      6. & Case 1: If $b \in f(X)$, then there exists $a \in X$ such that $f(a) = b$. \\
      7. & Case 2: If $b \in W$, then there exists $a \in A - X$ such that $f(a) = b$. \\
      8. & In both cases, there exists $a \in A$ such that $f(a) = b$. \\
      9. & Therefore, $f$ is onto. \\
    \end{tabular}\qedhere
  }$
\end{proof}

\medskip

\begin{problem}[6]
  Prove the statement $1 + 3 + 5 + 7 + \cdots + (2n + 1) = (n + 1)^2$, for all
  $n \ge 0$.
\end{problem}

\begin{proof}[Solution]
  Let $P(n) : 1 + 3 + 5 + 7 + \cdots + (2n + 1) = (n + 1)^2$.

  Base Case: $P(0) : 1 = (0 + 1)^2 = 1$.

  Induction Step: Assume $P(n)$ up to $n = k$. Then, adding the next term $2n +
  3$ to both sides gives us $1 + 3 + 5 + 7 + \cdots + (2n + 1) + (2n + 3) = (n +
  1)^2 + (2n + 3)$. Simplifying the right hand side gives us
  \[%
    (n + 1)^2 + (2n + 3) = n^2 + 2n + 1 + 2n + 3 = n^2 + 4n + 4 = (n + 2)^2 = ((n + 1) + 1)
  ,\]%
  which is equivalent to $P(n + 1)$. Hence, $(\forall n \in \N \cup \{0\})[P(n)
  \implies P(n + 1)]$.

  By PMI, we proved that $1 + 3 + 5 + 7 + \cdots + (2n + 1) = (n + 1)^2$ for all
  $n \ge 0$.
\end{proof}

\medskip

\begin{problem}[7]
  Prove the statement $\displaystyle 1^3 + 2^3 + \cdots + n^3 = \left[\frac{n(n
  + 1)}{2}\right]^2$, for all $n \ge 1$.
\end{problem}

\begin{proof}[Solution]
  Let $P(n) : 1^3 + 2^3 + \cdots + n^3 = \left[\frac{n(n + 1)}{2}\right]^2$.

  Base Case: $P(1) : 1^3 = \left[\frac{1(1 + 1)}{2}\right]^2 = 1$.

  Induction Step: Assume $P(n)$ up to $n = k$. Then, adding the next term $(n +
  1)^3$ to both sides gives us $\displaystyle 1^3 + \cdots + n^3 + (n + 1)^3 =
  \left[\frac{n(n + 1)}{2}\right]^2 + (n + 1)^3$. If the statement is true, then
  we get the following
  \[%
    \left[\frac{n(n + 1)}{2}\right]^2 + (n + 1)^3 = \left[\frac{(n + 1)(n + 2)}{2}\right]^2
  .\]%
  Performing some algebra gives us
  \begin{align*}
    \frac{n^2(n + 1)^2}{4} + (n + 1)^3 &= \frac{(n + 1)^2(n + 2)^2}{4} \\
    \frac{n^2(n + 1)^2 + 4(n + 1)^3}{4} &= \frac{(n + 1)^2(n + 2)^2}{4} \\
    n^2(n + 1)^2 + 4(n + 1)^3 &= (n + 1)^2(n + 2)^2 \\
    (n + 1)^2(n^2 + 4(n + 1)) &= (n + 1)^2(n + 2)^2 \\
    n^2 + 4(n + 1) &= (n + 2)^2 \\
    n^2 + 4n + 4 &= n^2 + 4n + 4
  .\end{align*}
  Therefore, we've shown that
  \[%
    \left[\frac{n(n + 1)}{2}\right]^2 + (n + 1)^3 = \left[\frac{(n + 1)(n + 2)}{2}\right]^2
  .\]%

  By EPMI, we proved that $\displaystyle 1^3 + 2^3 + \cdots + n^3 =
  \left[\frac{n(n + 1)}{2}\right]^2$ for all $n \ge 1$.
\end{proof}

\medskip

\begin{problem}[8]
  Prove the statement $\displaystyle\sum_{k=1}^{n} (-1)^k k^2 = (-1)^n \frac{n(n
  + 1)}{2}$, for all $n \ge 1$.
\end{problem}

\begin{proof}[Solution]
  Let $P(n) : \displaystyle\sum_{k=1}^{n} (-1)^k k^2 = (-1)^n \frac{n(n +
  1)}{2}$.

  Base Case: $P(1) : (-1)^1 1^2 = (-1)^1 \frac{1(1 + 1)}{2} = -1$.

  Induction Step: Assume $P(n)$ up to $n$. Then, adding the next term
  $(-1)^{n+1}(n + 1)^2$ to both sides gives us
  \[%
    \left(\sum_{k=1}^{n} (-1)^k k^2\right) + (-1)^{n+1} (n + 1)^2 = (-1)^n \frac{n(n + 1)}{2} + (-1)^{n+1} (n + 1)^2
  .\]%
  If the statement is true, then we get the following
  \[%
    (-1)^n \frac{n(n + 1)}{2} + (-1)^{n+1} (n + 1)^2 = (-1)^{n+1} \frac{(n + 1)(n + 2)}{2}
  .\]%
  Performing some algebra gives us
  \begin{align*}
    (-1)^n \frac{n(n + 1)}{2} + (-1)^{n+1} (n + 1)^2 &= (-1)^{n+1} \frac{(n + 1)(n + 2)}{2} \\
    (n + 1) \left((-1)^n \cdot \frac{n}{2} + (-1)^{n+1} \cdot (n + 1)\right) &= (n + 1) \cdot \left(\frac{(-1)^{n+1} \cdot (n + 2)}{2}\right) \\
    (-1)^n \cdot \frac{n}{2} + (-1)^{n+1} &= (-1)^{n+1} \cdot \frac{(n + 2)}{2} \\
    \frac{(-1)^n \cdot n +2 (-1)^{n+1} \cdot (n + 1)}{2} &= \frac{(-1)^{n+1} \cdot (n + 2)}{2} \\
    (-1)^n \cdot n + (-1)^{n+1} \cdot 2(n + 1) &= (-1)^{n+1} \cdot (n + 2) \\
    (-1)^n \left(n - 2(n + 1)\right) &= (-1)^n \left(-(n + 2)\right) \\
    n - 2n - 2 &= -n - 2 \\
    -n - 2 &= -n - 2
  .\end{align*}
  Therefore, we've shown that
  \[%
    (-1)^n \frac{n(n + 1)}{2} + (-1)^{n+1} (n + 1)^2 = (-1)^{n+1} \frac{(n + 1)(n + 2)}{2}
  .\]%

  By EPMI, we proved that $\displaystyle\sum_{k=1}^{n} (-1)^k k^2 = (-1)^n
  \frac{n(n + 1)}{2}$ for all $n \ge 1$.
\end{proof}

\medskip

\begin{problem}[9]
  Prove the statement $\frac{(2n)!}{n! \cdot 2^n}$ is an odd number, for every
  $n \in \N$.
\end{problem}

\begin{proof}[Solution]
  Let $P(n) : \frac{(2n)!}{n! \cdot 2^n}$ is an odd number.

  Base Case: $P(1)$. For $n = 1$,
  \[%
    \frac{(2 \cdot 1)!}{1! \cdot 2^1} = \frac{2!}{1 \cdot 2} = \frac{2}{2} = 1
  .\]%
  Since $1$ is an odd number, $P(1)$ holds.

  Induction Step: Assume $P(n)$ holds for some $n \in \N$, i.e.,
  \[%
    \frac{(2n)!}{n! \cdot 2^n} \text{ is odd.}
  \]%
  We need to show that $P(n + 1)$, i.e., $\frac{(2(n + 1))!}{(n + 1)! \cdot 2^{n
  + 1}}$, is also odd. Starting with
  \[%
    \frac{(2(n + 1))!}{(n + 1)! \cdot 2^{n + 1}} = \frac{(2n + 2)!}{(n + 1)! \cdot 2^{n + 1}}
  .\]%
  Using the factorial property $(2n + 2)! = (2n + 2)(2n + 1)(2n)!$, this becomes
  \[%
    \frac{(2n + 2)(2n + 1)(2n)!}{(n + 1)! \cdot 2^{n + 1}}
  .\]%
  Rearranging, we write it as
  \[%
    \frac{(2n + 2)(2n + 1)}{(n + 1) \cdot 2} \cdot \frac{(2n)!}{n! \cdot 2^n}
  .\]%
  By the induction hypothesis, $\frac{(2n)!}{n! \cdot 2^n}$ is odd. Now consider
  the term
  \[%
    \frac{(2n + 2)(2n + 1)}{(n + 1) \cdot 2}
  .\]%
  Factor $2$ from $(2n + 2)$, giving
  \[%
    \frac{2(n + 1)(2n + 1)}{(n + 1) \cdot 2} = 2n + 1
  .\]%
  Since $2n + 1$ is odd, the product
  \[%
    \frac{(2n + 2)(2n + 1)}{(n + 1) \cdot 2} \cdot \frac{(2n)!}{n! \cdot 2^n}
  \]%
  is odd because the product of odd numbers is odd.

  Therefore, $P(n + 1)$ holds, and $\frac{(2n)!}{n! \cdot 2^n}$ is odd for all
  $n \in \N$ by the principle of mathematical induction.
\end{proof}

\medskip

\begin{problem}[10]
  Prove the statement $2^n > n^2$, for all $n > 4$.
\end{problem}

\begin{proof}[Solution]
  Let $P(n) : 2^n > n^2$.

  Base Case: $P(5) : 2^5 = 32 > 25 = 5^2$.

  Induction Step: Assume $P(n)$ holds for some $n > 4$, i.e., $2^n > n^2$. We
  need to show that $P(n + 1)$, i.e., $2^{n+1} > (n+1)^2$, is true. Starting
  with the left-hand side
  \[%
    2^{n+1} = 2 \cdot 2^n
  .\]%
  By the induction hypothesis, $2^n > n^2$, so
  \[%
    2^{n+1} = 2 \cdot 2^n > 2 \cdot n^2
  .\]%
  Now compare $2 \cdot n^2$ to $(n + 1)^2$
  \[%
    2 \cdot n^2 > (n+1)^2 \quad \iff \quad 2 \cdot n^2 > n^2 + 2n + 1
  .\]%
  Simplifying gives u
  \[%
    2 \cdot n^2 - n^2 > 2n + 1 \quad \iff \quad n^2 > 2n + 1
  .\]%
  Since $n > 4$, this inequality holds because
  \[%
    n^2 - 2n - 1 = (n - 1)^2 - 2 > 0 \quad \text{for all}~n > 4
  .\]%
  Therefore, $2^{n+1} > (n + 1)^2$.

  By EPMI, we have proven that $2^n > n^2$ for all $n > 4$.
\end{proof}

\medskip

\begin{problem}[11]
  Consider the sequence given recursively by $a_0 = 0$ and $a_n = \sqrt{2 +
  a_{n-1}}$ for all $n \ge 1$. So $a_1 = \sqrt{2}$, $a_2 = \sqrt{2 + \sqrt{2}}$,
  $a_3 = \sqrt{2 + \sqrt{2 + \sqrt{2}}}$, and so forth. Then, prove that $a_n
  \le 2$ for all $n \ge 0$.
\end{problem}

\begin{proof}[Solution]
  Let $P(n) : a_n \le 2$.

  Base Case: $P(0) : a_0 = 0 \le 2$.

  Induction Step: Assume $P(n)$ holds for some $n \geq 0$, i.e., assume $a_n
  \leq 2$. We need to show that $a_{n+1} \leq 2$. By the recursive definition of
  the sequence,
  \[%
    a_{n+1} = \sqrt{2 + a_n}
  .\]%
  Starting with the inductive hypothesis $a_n \leq 2$, we add $2$ to both sides
  \[%
    2 + a_n \leq 2 + 2 = 4
  .\]%
  Taking the square root of both sides (noting that square roots preserve
  inequalities for non-negative numbers), we get
  \[%
    \sqrt{2 + a_n} \leq \sqrt{4}
  .\]%
  Since $\sqrt{4} = 2$, it follows that
  \[%
    a_{n+1} = \sqrt{2 + a_n} \leq 2
  .\]%
  Therefore, $P(n + 1)$ holds.

  By the principle of mathematical induction, $a_n \leq 2$ for all $n \geq 0$.
\end{proof}

\medskip

\begin{problem}[12]
  Prove the statement $\left(1 + \frac{1}{2}\right)^n > 1 + \frac{n}{2}$ for all
  $n \ge 2$.
\end{problem}

\begin{proof}[Solution]
  Let $P(n)$ denote the statement $\left(1 + \frac{1}{2}\right)^n > 1 +
  \frac{n}{2}$.

  Base Case: $P(2)$ states $\left(1 + \frac{1}{2}\right)^2 > 1 + \frac{2}{2}$.
  Compute the both sides gives us
  \[%
    \left(1 + \frac{1}{2}\right)^2 = \frac{3}{2} \cdot \frac{3}{2} = \frac{9}{4} \aand 1 + \frac{2}{2} = 2
  .\]%
  Clearly, $\frac{9}{4} = 2.25 > 2$, so $P(2)$ holds.

  Induction Step: Assume $P(n)$ is true for some $k \geq 2$, i.e., assume
  \[%
    \left(1 + \frac{1}{2}\right)^n > 1 + \frac{n}{2}
  .\]%
  We need to show that $P(n + 1)$ is true, i.e.,
  \[%
    \left(1 + \frac{1}{2}\right)^{n+1} > 1 + \frac{n + 1}{2}
  .\]%
  Starting with the left-hand side of $P(n + 1)$
  \[%
    \left(1 + \frac{1}{2}\right)^{n+1} = \left(1 + \frac{1}{2}\right)^n \cdot \left(1 + \frac{1}{2}\right)
  .\]%
  By the induction hypothesis, $\left(1 + \frac{1}{2}\right)^n > 1 +
  \frac{n}{2}$. Substituting this
  \[%
    \left(1 + \frac{1}{2}\right)^{n+1} > \left(1 + \frac{n}{2}\right) \cdot \left(1 + \frac{1}{2}\right)
  .\]%
  Expanding the product on the right-hand side
  \begin{align*}
    \left(1 + \frac{n}{2}\right) \cdot \left(1 + \frac{1}{2}\right) &= \left(1 + \frac{n}{2}\right) + \frac{1}{2} \left(1 + \frac{n}{2}\right) \\
                                                                    &= 1 + \frac{n}{2} + \frac{1}{2} + \frac{n}{4} = 1 + \frac{1}{2} + \frac{n}{2} + \frac{n}{4} \\
                                                                    &= 1 + \frac{n+2}{2} + \frac{n}{4}
  .\end{align*}
  Since $\frac{n}{4} > 0$ for $n \geq 2$, it follows that
  \[%
    \left(1 + \frac{n}{2}\right) \cdot \left(1 + \frac{1}{2}\right) > 1 + \frac{n+2}{2} = 1 + \frac{n+1}{2}
  .\]%
  Therefore, $\left(1 + \frac{1}{2}\right)^{n+1} > 1 + \frac{n + 1}{2}$.

  By the EPMI, $\left(1 + \frac{1}{2}\right)^n > 1 + \frac{n}{2}$ for all $n
  \geq 2$.
\end{proof}

\medskip

\begin{problem}[13]
  Prove the statement $\frac{1}{1^2} + \frac{1}{2^2} + \frac{1}{3^2} + \cdots
  \frac{1}{n^2} \le 2 - \frac{1}{n}$, for all $n \in \Z_{\ge 1}$.
\end{problem}

\begin{proof}[Solution]
  Let $P(n)$ denote the statement $\displaystyle \frac{1}{1^2} + \frac{1}{2^2} +
  \frac{1}{3^2} + \cdots + \frac{1}{n^2} \leq 2 - \frac{1}{n}$.

  Base Case: $P(1)$ states $\displaystyle \frac{1}{1^2} \leq 2 - \frac{1}{1}$.
  Compute both sides
  \[%
    \frac{1}{1^2} = 1 \aand 2 - \frac{1}{1} = 1
  .\]%
  Clearly, $1 \leq 1$, so $P(1)$ holds.

  Induction Step: Assume $P(n)$ holds for some $n \geq 1$, i.e.,
  \[%
    \frac{1}{1^2} + \frac{1}{2^2} + \cdots + \frac{1}{n^2} \leq 2 - \frac{1}{n}
  .\]%
  We need to show that $P(n + 1)$ is true, i.e.,
  \[%
    \frac{1}{1^2} + \frac{1}{2^2} + \cdots + \frac{1}{n^2} + \frac{1}{(n+1)^2} \leq 2 - \frac{1}{n+1}
  .\]%
  Starting with the left-hand side of $P(n+1)$
  \[%
    \frac{1}{1^2} + \frac{1}{2^2} + \cdots + \frac{1}{n^2} + \frac{1}{(n+1)^2}
  .\]%
  By the induction hypothesis, we know
  \[%
    \frac{1}{1^2} + \frac{1}{2^2} + \cdots + \frac{1}{n^2} \leq 2 - \frac{1}{n}
  .\]%
  Adding $\frac{1}{(n+1)^2}$ to both sides gives
  \[%
    \frac{1}{1^2} + \frac{1}{2^2} + \cdots + \frac{1}{n^2} + \frac{1}{(n+1)^2} \leq 2 - \frac{1}{n} + \frac{1}{(n+1)^2}
  .\]%
  Performing some algebra gives us
  \begin{align*}
    2 - \frac{1}{n} + \frac{1}{(n + 1)^2} &\leq 2 - \frac{1}{n + 1} \\
    -\frac{1}{n} + \frac{1}{(n + 1)^2} &\leq -\frac{1}{n + 1} \\
    \frac{1}{(n + 1)^2} &\leq \frac{1}{n} - \frac{1}{n + 1} \\
    \frac{1}{n} - \frac{1}{n + 1} &= \frac{n + 1 - n}{n(n + 1)} = \frac{1}{n(n + 1)} \\
    \frac{1}{(n + 1)^2} &\leq \frac{1}{n(n + 1)}
  .\end{align*}
  Since $n + 1 > n$, we have $(n + 1)^2 > n(n + 1)$, which implies
  \[%
    \frac{1}{(n + 1)^2} < \frac{1}{n(n + 1)}
  .\]%
  Hence, the inequality holds.

  By EPMI, we have proven that $\displaystyle \frac{1}{1^2} + \frac{1}{2^2} +
  \cdots + \frac{1}{n^2} \leq 2 - \frac{1}{n}$ for all $n \geq 1$.
\end{proof}

\medskip

\begin{problem}[14]
  Fill in each box below with a mathematical proposition that makes the
  biconditional true, and is not a tautology (for example, I don’t want you to
  write ``$A \subseteq B$'' in the first box, even though this makes the
  biconditional true). Copy the complete biconditional statements into your
  homework; do not actually write in the boxes on this worksheet.
  \begin{multicols}{2}
    \begin{enumerate}
      \item $A \subseteq B \iff $
      \item $A = B \iff $
      \item $x \in f(A) \iff $
      \item $y \in f^{-1}(B) \iff $
      \item $x \in A \cup B \iff $
      \item $x \in A \cap B \iff $
      \item $x \in A - B \iff $
      \item $f : S \to T$ is onto $\iff $
      \item $f : S \to T$ is one-to-one $\iff $
      \item $x \in A \cup (B - C) \iff $
      \item $X = \emptyset \iff $
    \end{enumerate}
  \end{multicols}
\end{problem}

\begin{proof}[Solution to (i)]
  $A \subseteq B \iff (\forall x \in A)[x \notin B^c]$.
\end{proof}

\begin{proof}[Solution to (ii)]
  $A = B \iff A \subseteq B \land B \subseteq A$.
\end{proof}

\begin{proof}[Solution to (iii)]
  $x \in f(A) \iff (\exists a \in A)[f(a) = x]$.
\end{proof}

\begin{proof}[Solution to (iv)]
  $y \in f^{-1}(B) \iff f(y) \in B$.
\end{proof}

\begin{proof}[Solution to (v)]
  $x \in A \cup B \iff x \in A \lor x \in B$.
\end{proof}

\begin{proof}[Solution to (vi)]
  $x \in A \cap B \iff x \in A \land x \in B$.
\end{proof}

\begin{proof}[Solution to (vii)]
  $x \in A - B \iff x \in A \land x \notin B$.
\end{proof}

\begin{proof}[Solution to (viii)]
  $f : S \to T$ is onto $\iff (\forall t \in T)(\exists s \in S)[f(s) = t]$.
\end{proof}

\begin{proof}[Solution to (ix)]
  $f : S \to T$ is one-to-one $\iff (\forall s_1, s_2 \in S)[(f(s_1) = f(s_2)) \implies (s_1 = s_2)]$.
\end{proof}

\begin{proof}[Solution to (x)]
  $x \in A \cup (B - C) \iff x \in A \lor (x \in B \land x \notin C$.
\end{proof}

\begin{proof}[Solution to (xi)]
  $X = \emptyset \iff (\forall x)[x \notin X]$.
\end{proof}

\medskip

\begin{problem}[15]
  Write definitions for the following sets, using set-builder notation. The
  first one is done for you.
  \begin{enumerate}
    \item $X - A$.
    \item $f^{-1}(B)$.
    \item $A \cup B$.
    \item $f(A)$.
    \item $C \cap f^{-1}(B)$.
  \end{enumerate}
\end{problem}

\begin{proof}[Solution to (i)]
  $X - A = \{x \in X \mid x \notin A\}$.
\end{proof}

\begin{proof}[Solution to (ii)]
  $f^{-1}(B) = \{x \in A \mid f(x) \in B\}$.
\end{proof}

\begin{proof}[Solution to (iii)]
  $A \cup B = \{x \mid x \in A \lor x \in B\}$.
\end{proof}

\begin{proof}[Solution to (iv)]
  $f(A) = \{y \in B \mid (\exists x \in A)[f(x) = y]\}$.
\end{proof}

\begin{proof}[Solution to (v)]
  $C \cap f^{-1}(B) = \{x \in C \mid f(x) \in B\}$.
\end{proof}
