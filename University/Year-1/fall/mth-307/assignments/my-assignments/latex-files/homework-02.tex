\begin{problem}
  In each of the following give a disjunction that is equivalent to the given proposition:
  \begin{enumerate}
    \item $P \implies Q$.
    \item $\neg P \implies Q$.
    \item $P \implies \neg Q$.
  \end{enumerate}
\end{problem}

\begin{probsolution}
  \begin{enumerate}
    \item $\neg P \lor Q$.
    \item $P \lor Q$.
    \item $\neg P \lor \neg Q$.
  \end{enumerate}
\end{probsolution}

\newpage

\begin{problem}
  Translate the following into a symbolic logic problem, then provide a proof:

  \noindent Given: If Smith wins the nomination, he will be happy, and if he is
  happy, he is not a good campaigner. But if he loses the nomination, he will
  lose the confidence of the party. He is not a good campaigner if he loses the
  confidence of the party. If he is not a good campaigner, then he should resign
  from the party. Either Smith wins the nomination or he loses it.

  \noindent Prove: Smith should resign from the party.
\end{problem}

\begin{probsolution}
  Let $W =$ ``Smith wins the nomination'', $H =$ ``Smith is happy'', $G =$
  ``Smith is a good campaigner'', $C =$ ``Smith has the confidence of the
  party'', and $R =$ ``Smith should resign from the party''.

  \noindent Translating the given information into symbolic logic:
  \begin{table}[H]
    \centering
    \begin{tabular}{lll}
      & Statement & Explanation \\
      1. & $W \implies H$ & Hypothesis \\
      2. & $H \implies \neg G$ & Hypothesis \\
      3. & $\neg W \implies \neg C$ & Hypothesis \\
      4. & $\neg G \implies R$ & Hypothesis \\
      5. & $W \lor \neg W$ & Hypothesis \\
      6. & $W$ & Dischargeable Hypothesis \\
      7. & $H$ & MP, for 6, for 1 \\
      8. & $\neg G$ & MP, for 7, for 2 \\
      9. & $R$ & MP, for 8, for 4 \\
      10 & $W \implies R$ & DT, discharge for 6 [(6) - (9) unusable] \\
    \end{tabular}
  \end{table}
\end{probsolution}

\newpage

\begin{problem}
  Fill in the blanks to give a proof of $R \lor [P \land Q]$, $\neg Q$ $\vdash$
  $R$.
  \begin{table}[H]
    \centering
    \begin{tabular}{lll}
      & Statement & Explanation \\
      1. & $R \lor [P \land Q]$ & ?? \\
      2. & $\neg R$ & ?? \\
      3. & ?? & Tautology \\
      4. & $\neg R \implies [P \land Q]$ & ?? \\
      5. & $P \land Q$ & ?? \\
      6. & ?? & RCS, ?? \\
      7. & ?? & Hypothesis \\
      8. & $Q \land \neg Q$ & ?? \\
      9. & $R$ & ?? \\
    \end{tabular}
  \end{table}
\end{problem}

\begin{probsolution}
  \begin{table}[H]
    \centering
    \begin{tabular}{lll}
      & Statement & Explanation \\
      1. & $R \lor [P \land Q]$ & Hypothesis \\
      2. & $\neg R$ & Dischargeable Hypothesis \\
      3. & $\neg R \lor R$ & Tautology \\
      4. & $\neg R \implies [P \land Q]$ & DI, for 2, for 1 \\
      5. & $P \land Q$ & MPD, for 2, for 4 \\
      6. & $Q$ & RCS, for 5 \\
      7. & $\neg Q$ & Hypothesis \\
      8. & $Q \land \neg Q$ & CI, for 6, for 7 \\
      9. & $R$ & II, discharge for 2 [(2) - (8) unusable] \\
    \end{tabular}
  \end{table}
\end{probsolution}

\newpage

\begin{problem}
  Fill in the blanks to give a proof of $(P \land \neg Q) \implies (R \implies
  Q)$ $\vdash$ $P \implies [Q \lor \neg R]$. [Note: This proof uses both DT and
  Indirect Inference].
  \begin{table}[H]
    \centering
    \begin{tabular}{lll}
      & Statement & Explanation \\
      1. & $(P \land \neg Q) \implies (R \implies Q)$ & ?? \\
      2. & $P$ & Dischargeable Hypothesis \\
      3. & $\neg[Q \lor \neg R]$ & Dischargeable Hypothesis \\
      4. & ?? & Tautology \\
      5. & $\neg Q \land R$ & ?? \\
      6. & $\neg Q$ & ?? \\
      7. & $P \land \neg Q$ & ?? \\
      8. & $R \implies Q$ & ?? \\
      9. & $\neg R$ & ?? \\
      10. & $R$ & ?? \\
      11. & $R \land \neg R$ & ?? \\
      12. & $Q \lor \neg R$ & ?? \\
      13. & $P \implies [Q \lor \neg R$ & ?? \\
    \end{tabular}
  \end{table}
\end{problem}

\begin{probsolution}
  \begin{table}[H]
    \centering
    \begin{tabular}{lll}
      & Statement & Explanation \\
      1. & $(P \land \neg Q) \implies (R \implies Q)$ & Hypothesis \\
      2. & $P$ & Dischargeable Hypothesis \\
      3. & $\neg[Q \lor \neg R]$ & Dischargeable Hypothesis \\
      4. & $\neg[Q \lor \neg R] \iff \neg Q \land R$ & Tautology \\
      5. & $\neg Q \land R$ & MPB, for 3, for 4 \\
      6. & $\neg Q$ & LCS, for 5 \\
      7. & $P \land \neg Q$ & CI, for 2, for 6 \\
      8. & $R \implies Q$ & MP, for 7, for 1 \\
      9. & $\neg R$ & MT, for 6, for 8 \\
      10. & $R$ & RCS, for 5 \\
      11. & $R \land \neg R$ & CI, for 10, for 9 \\
      12. & $Q \lor \neg R$ & II, discharge for 3 [(3) - (11) unusable] \\
      13. & $P \implies [Q \lor \neg R]$ & DT, discharge for 2 [(2) - (12) unusable] \\
    \end{tabular}
  \end{table}
\end{probsolution}

\newpage

\begin{problem}
  Show that $[P \land Q] \implies R$, $\neg R$, $P$ $\vdash$ $\neg Q$.
\end{problem}

\begin{probsolution}
  \begin{table}[H]
    \centering
    \begin{tabular}{lll}
      & Statement & Explanation \\
      1. & $[P \land Q] \implies R$ & Hypothesis \\
      2. & $\neg R$ & Hypothesis \\
      3. & $P$ & Hypothesis \\
      4. & $\neg [P \land Q]$ & MT, for 2, for 1 \\
      5. & $\neg[P \land Q] \iff [\neg P \lor \neg Q]$ & Tautology \\
      6. & $\neg P \lor \neg Q$ & MPB, for 4, for 5 \\
      7. & $\neg Q$ & DI, for 3, for 6 \\
    \end{tabular}
  \end{table}
\end{probsolution}

\newpage

\begin{problem}
  Show that $P \implies Q$, $R$, $R \implies [Q \implies P]$ $\vdash$ $P \iff
  Q$.
\end{problem}

\begin{probsolution}
  \begin{table}[H]
    \centering
    \begin{tabular}{lll}
      & Statement & Explanation \\
      1. & $P \implies Q$ & Hypothesis \\
      2. & $R$ & Hypothesis \\
      3. & $R \implies [Q \implies P]$ & Hypothesis \\
      4. & $Q \implies P$ & MP, for 2, for 3 \\
      5. & $P \iff Q$ & For 1, for 4 \\
    \end{tabular}
  \end{table}
\end{probsolution}

\newpage

\begin{problem}
  Show that $P \implies \neg Q$, $\neg R \implies Q$ $\vdash$ $P \implies R$.
\end{problem}

\begin{probsolution}
  \begin{table}[H]
    \centering
    \begin{tabular}{lll}
      & Statement & Explanation \\
      1. & $P \implies \neg Q$ & Hypothesis \\
      2. & $\neg R \implies Q$ & Hypothesis \\
      3. & $\neg Q \implies R$ & Contrapositive, for 2 \\
      4. & $P \implies R$ & SI, for 1, for 3 \\
    \end{tabular}
  \end{table}
\end{probsolution}

\newpage

\begin{problem}
  Show that $\neg P \implies Q$, $T \implies \neg P$, $\neg[Q \lor R]$ $\vdash$
  $\neg T$
\end{problem}

\begin{probsolution}
  \begin{table}[H]
    \centering
    \begin{tabular}{lll}
      & Statement & Explanation \\
      1. & $\neg P \implies Q$ & Hypothesis \\
      2. & $T \implies \neg P$ & Hypothesis \\
      3. & $\neg[Q \lor R]$ & Hypothesis \\
      4. & $\neg[Q \lor R] \iff [\neg Q \land \neg R]$ & Tautology \\
      5. & $\neg Q \land \neg R$ & MPB, for 3, for 4 \\
      6. & $\neg Q$ & RCS, for 5 \\
      7. & $P$ & MT, for 6, for 1 \\
      8. & $\neg T$ & MT, for 7, for 2 \\
    \end{tabular}
  \end{table}
\end{probsolution}

\newpage

\begin{problem}
  Show that $\neg P \implies Q$, $Q \implies [R \implies S]$, $\neg S$ $\vdash$
  $R \implies P$.
\end{problem}

\begin{probsolution}
  \begin{table}[H]
    \centering
    \begin{tabular}{lll}
      & Statement & Explanation \\
      1. & $\neg P \implies Q$ & Hypothesis \\
      2. & $Q \implies [R \implies S]$ & Hypothesis \\
      3. & $\neg S$ & Hypothesis \\
      4. & $\neg P$ & Dischargeable Hypothesis \\
      5. & $Q$ & MP, for 4, for 1 \\
      6. & $R \implies S$ & MP, for 5, for 2 \\
      7. & $\neg R$ & MT, for 3, for 6 \\
      8. & $\neg P \implies \neg R$ & DT, discharge for 4 [(4) - (7) unusable] \\
      9. & $R \implies P$ & Contrapositive \\
    \end{tabular}
  \end{table}
\end{probsolution}

\newpage

\begin{problem}
  Show that $P \implies T$, $Q \implies T$, $R \iff [P \lor Q]$, $R$ $\vdash$
  $T$.
\end{problem}

\begin{probsolution}
  \begin{table}[H]
    \centering
    \begin{tabular}{lll}
      & Statement & Explanation \\
      1. & $P \implies T$ & Hypothesis \\
      2. & $Q \implies T$ & Hypothesis \\
      3. & $R \iff [P \lor Q]$ & Hypothesis \\
      4. & $R$ & Hypothesis \\
      5. & $P \lor Q$ & MPB, for 4, for 3 \\
      6. & $[P \lor Q] \implies T$ & IC, for 1, for 2 \\
      7. & $T$ & MP, for 5, for 6 \\
    \end{tabular}
  \end{table}
\end{probsolution}

\newpage

\begin{problem}
  Show that $S \implies P$, $Q \implies R$, $S$ $\vdash$ $[P \implies Q] \implies R$.
\end{problem}

\begin{probsolution}
  \begin{table}[H]
    \centering
    \begin{tabular}{lll}
      & Statement & Explanation \\
      1. & $S \implies P$ & Hypothesis \\
      2. & $Q \implies R$ & Hypothesis \\
      3. & $S$ & Hypothesis \\
      4. & $P$ & MP, for 3, for 1 \\
      5. & $P \implies Q$ & Dischargeable Hypothesis \\
      6. & $Q$ & MP, for 4, for 5 \\
      7. & $R$ & MP, for 6, for 2 \\
      8. & $[P \implies Q] \implies R$ & DT, discharge for 5 [(5) - (7) unusable] \\
    \end{tabular}
  \end{table}
\end{probsolution}

\newpage

\begin{problem}
  Show that $R \implies T$, $\neg T \iff S$, $[R \land \neg S] \implies \neg Q$
  $\vdash$ $R \implies \neg Q$.
\end{problem}

\begin{probsolution}
  \begin{table}[H]
    \centering
    \begin{tabular}{lll}
      & Statement & Explanation \\
      1. & $R \implies T$ & Hypothesis \\
      2. & $\neg T \iff S$ & Hypothesis \\
      3. & $[R \land \neg S] \implies \neg Q$ & Hypothesis \\
      4. & $R$ & Dischargeable Hypothesis \\
      5. & $T$ & MP, for 4, for 1 \\
      6. & $\neg S$ & MT, for 5, for 2 \\
      7. & $R \land \neg S$ & CI, for 4, for 6 \\
      8. & $\neg Q$ & MP, for 7, for 3 \\
      9. & $R \implies \neg Q$ & DT, discharge for 4 [(4) - (8) unusable] \\
    \end{tabular}
  \end{table}
\end{probsolution}

\newpage

\begin{problem}
  Show that $\neg P \implies Q$, $[R \implies Q] \implies S$, $\neg S \lor T$,
  $R \implies \neg P$ $\vdash$ $T \lor V$.
\end{problem}

\begin{probsolution}
  \begin{table}[H]
    \centering
    \begin{tabular}{lll}
      & Statement & Explanation \\
      1. & $\neg P \implies Q$ & Hypothesis \\
      2. & $[R \implies Q] \implies S$ & Hypothesis \\
      3. & $\neg S \lor T$ & Hypothesis \\
      4. & $R \implies \neg P$ & Hypothesis \\
      5. & $R$ & Dischargeable Hypothesis \\
      6. & $\neg P$ & MP, for 5, for 4 \\
      7. & $Q$ & MP, for 6, for 1 \\
      8. & $R \implies Q$ & DT, discharge for 5 [(5) - (7) unusable] \\
      9. & $S$ & MP, for 8, for 2 \\
      10. & $T$ & DI, for 9, for 3 \\
      11. & $T \lor V$ & CI, for 10 \\
    \end{tabular}
  \end{table}
\end{probsolution}

\newpage

\begin{problem}
  Show that $[R \land \neg Q] \implies P$, $[T \implies S] \iff [R \implies Q]$,
  $R$ $\vdash$ $[\neg P \lor [T \implies S]] \implies Q$.
\end{problem}

\begin{probsolution}
  \begin{table}[H]
    \centering
    \begin{tabular}{lll}
      & Statement & Explanation \\
      1. & $[R \land \neg Q] \implies P$ & Hypothesis \\
      2. & $[T \implies S] \iff [R \implies Q]$ & Hypothesis \\
      3. & $R$ & Hypothesis \\
      4. & $\neg P \lor [T \implies S]$ & Dischargeable Hypothesis \\
      5. & $\neg P$ & Dischargeable Hypothesis \\
      6. & $R \land \neg Q$ & Dischargeable Hypothesis \\
      7. & $P$ & MP, for 1, for 6 \\
      8. & $P \land \neg P$ & CI, for 7, for 5 \\
      10. & $\neg [R \land \neg Q]$ & II, discharge for 6 [(6) - (9) unusable] \\
      11. & $\neg[R \land \neg Q] \iff [\neg R \lor Q]$ & Tautology \\
      12. & $\neg R \lor Q$ & MP, for 10, for 11 \\
      13. & $Q$ & DI, for 3, for 12 \\
      14. & $[\neg P \lor [T \implies S]] \implies Q$ & DT, discharge for 4 [(4) - (13) unusable] \\
    \end{tabular}
  \end{table}
\end{probsolution}

\newpage

\begin{problem}
  Use the Euclidean Algorithm to find integers $a$ and $b$ such that $37a + 100b = 1$. Use this information to solve $37x + 42 = 15$ in $\Z_{100}$.
\end{problem}

\begin{probsolution}
  \begin{enumerate}
    \item $100 = 2 \cdot 37 + 26 \implies 100 - 2 \cdot 37 = 26$,
    \item $37 = 1 \cdot 26 + 11 \implies 37 - 1 \cdot 26 = 11$,
    \item $26 = 2 \cdot 11 + 4 \implies 26 - 2 \cdot 11 = 4$,
    \item $11 = 2 \cdot 4 + 3 \implies 11 - 2 \cdot 4 = 3$,
    \item $4 = 1 \cdot 3 + 1 \implies 4 - 1 \cdot 3 = 1$.
  \end{enumerate}
  Finally, write the equation for the greatest common divisor $1 = 4 - 1 \cdot
  3$. Now, we back-substitute to express $1$ as a linear combination of $37$ and
  $100$.
  \begin{enumerate}
    \item Substitute $3 = 11 - 2 \cdot 4$: $1 = 4 - 1 \cdot (11 - 2 \cdot 4) = 3
      \cdot 4 - 1 \cdot 11$.

    \item Substitute $4 = 26 - 2 \cdot 11$: $1 = 3 \cdot (26 - 2 \cdot 11) - 1
      \cdot 11 = 3 \cdot 26 - 7 \cdot 11$.

    \item Substitute $11 = 37 - 1 \cdot 26$: $1 = 3 \cdot 26 - 7 \cdot (37 - 1
      \cdot 26) = 10 \cdot 26 - 7 \cdot 37$.

    \item Substitute $26 = 100 - 2 \cdot 37$: $1 = 10 \cdot (100 - 2 \cdot 37) -
      7 \cdot 37 = 10 \cdot 100 - 27 \cdot 37$.
  \end{enumerate}

  \noindent So, we find $a = -27, \quad b = 10$. Now we can solve the
  congruence.

  \noindent First, subtract 42 from both sides: $37x \equiv -27 \pmod{100}$.
  Since $-27 \equiv 73 \pmod{100}$, we can rewrite this as $37x \equiv 73
  \pmod{100}$. From Step 1, we found that $37 \cdot (-27) \equiv 1 \pmod{100}$,
  so the inverse of $37$ modulo $100$ is $-27 \equiv 73 \pmod{100}$. Now
  multiply both sides of the congruence by $73$: $x \equiv 73 \cdot 73
  \pmod{100}$ Calculate $73 \cdot 73 \pmod{100}$: $73 \cdot 73 = 5329.$ Find
  $5329 \pmod{100}$: $5329 \pmod{100} = 29$

  \noindent Thus, $x = 29$ is the solution to $37x + 42 \equiv 15 \pmod{100}$.
\end{probsolution}

\newpage

\begin{problem}
  For what primes $p$ is the element $p - 1$ a perfect square in $\Z_p$?
  Investigate this question by working out the cases $p = 2$, $p = 3$, $p = 5$,
  $p = 7$, $p = 11$, $p = 13$, $p = 17$, and $p = 19$. See if you notice any
  patterns an try to make a conjecture.
\end{problem}

\begin{probsolution}
  To see, we'll use the equation $x^2 = p - 1 \pmod{p}$. I'll use the Legendre
  symbol $\left(\frac{a}{p}\right)$ to determine if $a$ is a quadratic residue
  module $p$. For $a = p - 1$, we have $\left(\frac{p - 1}{p}\right)$ will
  determine if $p - 1$ is a quadratic residue of module $p$.

  \begin{enumerate}
    \item For $p = 2$: $p - 1 = 2 - 1 = 1$, which is a perfect square since $1^2
      \pmod{2} = 1 \pmod{2}$.

    \item For $p = 3$: $p - 1 = 3 - 1 = 2$. To see if $2$ is a perfect square in
      $\Z_2$, we need to check all perfect squares
      \[%
        0^2 = 0, \quad 1^2 = 1, \quad 2^2 = 0
      .\]%
      Therefore, $2$ isn't a perfect square.

    \item For $p = 5$: $p - 1 = 5 - 1 = 4$, which is clearly a perfect square
      since $2^2 \pmod{4} = 4 \pmod{2}$.

    \item For $p = 7$: $p - 1 = 7 - 1 = 6$
      \[%
        0^2 = 0, \quad 1^2 = 1, \quad 2^2 = 4, \quad 3^2 = 2, \quad 4^2 = 2, \quad 5^2 = 4, \quad 6^2 = 1
      .\]%
      Therefore, $6$ isn't a perfect square.

    \item For $p = 11$: $p - 1 = 11 - 1 = 10$
      \begin{align*}
        &0^1 = 0, \quad 1^2 = 1, \quad 2^2 = 4, \quad 3^2 = 9, \quad 4^2 = 5, \quad 5^2 = 3, \quad 6^2 = 3, \quad 7^2 = 5, \quad 8^2 = 9 \\
        &9^2 = 4, \quad 10^2 = 1
      .\end{align*}
      Therefore, $10$ isn't a perfect square.

    \item For $p = 13$: $p - 1 = 13 - 1 = 12$
      \[%
        0^1 = 0, \quad 1^2 = 1, \quad 2^2 = 4, \quad 3^2 = 9, \quad 4^2 = 1, \quad 5^2 = 12
      .\]%
      Therefore, $12$ is a perfect square in $\Z_{12}$.

    \item For $p = 17$: $p - 1 = 17 - 1 = 16$
      \[%
        0^2 = 0, \quad 1^2 = 1, \quad 2^2 = 4, \quad 3^2 = 9, \quad 4^2 = 16
      .\]%
      Therefore, $16$ is a perfect square in $\Z_{17}$.

    \item For $p = 19$: $p - 1 = 19 - 1 = 18$. I don't want to typeset the
      entire list, but $18$ is not a perfect square in $\Z_{19}$.
  \end{enumerate}

  Here's a summary table of everything
  \begin{table}[H]
    \centering
    \begin{tabular}{llll}
      $p$ & $p - 1$ & Quadratic Residue & $p \pmod{4}$ \\
      $2$ & $1$ & Yes & $2$ \\
      $3$ & $2$ & No & $3$ \\
      $5$ & $4$ & Yes & $1$ \\
      $7$ & $6$ & No & $3$ \\
      $11$ & $10$ & No & $3$ \\
      $13$ & $12$ & Yes & $1$ \\
      $17$ & $16$ & Yes & $1$ \\
      $19$ & $18$ & No & $3$ \\
    \end{tabular}
  \end{table}

  From this, I've come to the following conjecture:

  {\sffamily\textbf{Conjecture:}} \textit{Given a prime number $p$, $p - 1$ is a
  perfect square in $\Z_p$ if and only if $p \equiv 1 \pmod{4}$ (excluding the
  special case where $p = 2$). This is consistent with the property of the
  Legendre symbol}
  \[%
    \left(\frac{-1}{p}\right) = \begin{cases}
      \phantom{-}1 \quad \textrm{if}~p \equiv 1 \pmod{4}, \\
      -1 \quad \textrm{if}~p \equiv 1 \pmod{3}.
    \end{cases}
  \]%
\end{probsolution}

\newpage

\begin{problem}
  Find $2^{1000}$ in $\Z_7$. Then find $3^{1000}$ in $\Z_7$. Explain how you got your answers.
\end{problem}

\begin{probsolution}
  I'll write out all the squares of $2$ in $\Z_7$ until we start seeing a
  pattern or we get $2^n = 1$.
  \[%
    2^0 = 0, \quad 2^1 = 2, \quad 2^2 = 4, \quad 2^3 = 1
  \]%
  Breaking $2^{1000} = 2^{3 \cdot 333} \cdot 2^1 = 2^{999} \cdot 2 = 2$. I got
  my answer because I saw that $2^3 = 1$. I used this to my advantage by taking
  $\displaystyle\left\lfloor \frac{1000}{3} \right\rfloor = 999$, where $\lfloor
  x \rfloor$ is the biggest whole integer, $z$ such that $z \le x$. So, $2^{999}
  = 1$ and using rules of exponentiation, I got $2^{999 + 1} = 2^{999} \cdot 2^1
  = 2$.

  Now, the same process pretty much repeats for $3^{1000}$
  \[%
    3^0 = 0, \quad 3^1 = 3, \quad 3^2 = 2, \quad 3^3 = 6, \quad 3^4 = 4, \quad 3^5 = 5, \quad 3^6 = 1
  .\]%
  Breaking $3^{1000} = 3^{996} \cdot 3^4 = 3^4 = 4$. Again, just like with the
  previous answer, I found the biggest multiple of $6$, $x$, such that $6x \le
  1000$. That number was $166$, giving us $166 \cdot 6 = 996$. And we know that
  any number that's divisible by $6$, then $3^{6x} = 1$. Therefore, $3^{1000} =
  3^{996} \cdot 3^4 = 3^4 = 4$.
\end{probsolution}

\newpage

\begin{problem}
  Consider a sum of three consecutive squares (like $7^2 + 8^2 + 9^2$). What do you get when you reduce this $\mod$ $3$ (that is, when you compute the remainder when you divided by $3$)? Pick another sum of three consecutive squares and try it again. Try it one more time. State a conjecture and see if you can prove it.
\end{problem}

\begin{probsolution}
  The sum of $7^2 + 8^2 + 9^2  = 194$. Finding $194 \pmod{3}$, we get $194 -
  3 \cdot 64 = 2$. The sum of another three consecutive squares $10^2 + 11^2 +
  12^2 = 365$. Computing $365 \pmod{3}$, we get $365 - 3 \cdot 121 = 2$. Another
  three consecutive squares $4^2 + 5^2 + 6^2 = 77$, and $77 \pmod{3} \equiv 77 -
  3 \cdot 25 = 2$.

  \noindent {\sffamily\textbf{Conjecture:}} \textit{The sum of the squares of
  three consecutive integers is congruent to $2 \pmod{3}$.}

  \begin{plainproof}
    Let the three consecutive integers be $n - 1$, $n$, and $n + 1$. Then their
    squares are $(n - 1)^2$, $n^2$, and $(n + 1)^2$. We want to evaluate
    \[%
      (n - 1)^2 + n^2 + (n + 1)^2 \pmod{3}
    .\]%
    Expanding each term, we get
    \[%
      (n - 1)^2 = n^2 - 2n + 1, \quad n^2 = n^2, \quad (n + 1)^2 = n^2 + 2n + 1
    .\]%
    Adding them together we get
    \[%
      (n^2 - 2n + 1) + (n^2) + (n^2 + 2n + 1) = 3n^2 + 2
    .\]%
    Since we are multiplying $n^2$ by $3$, then $3n^2 \pmod{3} = 0$, leaving a
    remainder of $2$.
  \end{plainproof}
\end{probsolution}

\newpage

\begin{problem}
  The following proof has a mistake. Find what is wrong, and explain. $(R \lor
  \neg S) \implies \neg P$, $Q \implies R$, $S \implies T$ $\vdash$ $(P \implies
  \neg R) \land (Q \implies T)$.
  \begin{table}[H]
    \centering
    \begin{tabular}{lll}
      & Statement & Explanation \\
      1. & $(R \lor \neg S) \implies \neg P$ & Hypothesis \\
      2. & $Q \implies R$ & Hypothesis \\
      3. & $P$ & Dischargeable Hypothesis \\
      4. & $\neg(R \lor \neg S)$ & MT, for 1, for 3 \\
      5. & $\neg(R \lor \neg S) \iff (\neg R \land S)$ & Tautology \\
      6. & $\neg R \land S$ & MPB, for 5, for 4 \\
      7. & $S$ & RCS, for 6 \\
      8. & $\neg R$ & LCS for 6 \\
      9. & $P \implies \neg R$ & DT, discharge for 3 [(3) - (8) unusable] \\
      10. & $Q$ & Dischargeable Hypothesis \\
      11. & $S \implies T$ & Hypothesis \\
      12. & $T$ & MP, for 11, for 7 \\
      13. & $Q \implies T$ & DT, discharge for 10 [(10) - (12) unusable] \\
      14. & $(P \implies \neg R) \land (Q \implies T)$ & CI, for 9, for 13 \\
    \end{tabular}
  \end{table}
\end{problem}

\begin{probsolution}
  The mistake is on line $12$. The proof incorrectly uses the fact that $S$ is
  true from line $7$. This is not valid though, because the deduction ending on
  line $8$ was based on assuming that $P$ is true, which is not a given and was
  only a temporary assumption for a direct proof (DT). According to the rules of
  DT, once the assumption is discharged, all intermediate steps derived from
  that assumption (lines 3 to 8) become invalid outside the scope of the
  assumption.
\end{probsolution}

\newpage

\begin{problem}
  Show that $(R \lor \neg S) \implies \neg P$, $Q \implies R$, $S \implies T$
  $\vdash$ $(P \implies S) \land (Q \implies (\neg P \land R))$.
\end{problem}

\begin{probsolution}
  \begin{table}[H]
    \centering
    \begin{tabular}{lll}
      & Statement & Explanation \\
      1. & $(R \lor \neg S) \implies \neg P$ & Hypothesis \\
      2. & $Q \implies R$ & Hypothesis \\
      3. & $S \implies T$ & Hypothesis \\
      4. & $P$ & Dischargeable Hypothesis \\
      5. & $\neg(R \lor \neg S)$ & MT, for 4, for 1 \\
      6. & $\neg(R \lor \neg S) \iff (\neg R \land S)$ & Tautology \\
      7. & $\neg R \land S$ & MPB, for 5, for 6 \\
      8. & $S$ & RCS, for 7 \\
      9. & $P \implies S$ & DT, discharge for 4 [(4) - (8) unusable] \\
      10. & $Q$ & Dischargeable Hypothesis \\
      11. & $R$ & MP, for 10, for 2 \\
      12. & $\neg P$ & MP, for 11, for 1 \\
      13. & $\neg P \land R$ & CI, for 12, for 11 \\
      14. & $Q \implies (R \land \neg P)$ & DT, discharge for 10 [(10) - (13) unusable] \\
      15. & $(P = S) \land (Q \implies (\neg P \land R))$ & CI, for 9, for 14 \\
    \end{tabular}
  \end{table}
\end{probsolution}
