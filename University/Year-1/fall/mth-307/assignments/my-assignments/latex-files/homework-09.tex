\begin{problem}[1]
  Prove that $\odv[n]{}{x} (x^2 e^x) = (x^2 + 2nx + n(n - 1))e^x$, for all $n \in \N$.
\end{problem}

\begin{proof}[Solution]
  Let $P(n) : \odv[n]{}{x} (x^2 e^x) = (x^2 + 2nx + n(n - 1))e^x$, for all $n
  \in \N$.

  Base Case: $P(1) : \odv{}{x} (x^2 e^x) = e^x(x^2 + 2x) = e^x(x^2 + 2(1)x + 1(1
  - 1))$.

  Induction Step: Assume $P(n)$ is true for some $n \in \N$. Then,
  \[%
    P(n + 1) = \odv[n+1]{}{x} (x^2 e^x) = \odv{}{x} \left[(x^2 + 2nx + n(n - 1))e^x\right] = (2x + 2n + x^2 + 2nx + n(n - 1)) e^x
  .\]%
  If the original statement is true, then we get the following
  \begin{alignat*}{3}
    \phantom{\implies}&(2x + 2n + x^2 + 2nx + n(n - 1))e^x &&= (x^2 + 2(n + 1)x + (n + 1)(n + 1 - 1))e^{x} \\
    \implies\quad&2x + 2n + x^2 + 2nx + n(n - 1) &&= x^2 + 2(n + 1)x + n(n + 1) \\
    \implies\quad&2x + 2n + 2nx + n(n - 1) &&= 2x + 2nx + n(n + 1) \\
    \implies\quad&2n + n^2 - n &&= n^2 + n \\
    \implies\quad&n^2 &&= n^2
  .\end{alignat*}
  Thus, $P(n + 1)$ is true.

  Hence, by the PMI, we've show that $(\forall n \in \N)[P(n) \implies P(n +
  1)]$.
\end{proof}

\medskip

\begin{problem}[2]
  Here are two facts that hold for all real numbers $x$ and $y$:
  \begin{align*}
    \sin(x + y) = \sin(x) \cos(y) + \cos(x) \sin(y) \quad\quad& \textrm{(addition formula for sine)} \\
    \lvert x + y \rvert \le \lvert x \rvert + \lvert y \rvert \quad\quad& \textrm{(triangle inequality)}
  .\end{align*}
  Using these together with induction, prove that $(\forall x \in \R)(\forall n
  \in \N)[~\lvert \sin(nx) \rvert \le n \lvert \sin(x) \rvert~]$.
\end{problem}

\begin{proof}[Solution]
  Let $P(n): \lvert \sin(nx) \rvert \le n \lvert \sin(x) \rvert$ for all $x \in \R$.

  Base Case: $P(1) : \lvert \sin(1 \cdot x) \rvert = \lvert \sin(x) \rvert$.

  Induction Step: Assume $P(n)$ is true for some $n \in \N$. Using the addition
  formula for sine
  \[%
    \sin((n + 1)x) = \sin(nx + x) = \sin(nx)\cos(x) + \cos(nx)\sin(x)
  .\]%
  Taking the absolute value and applying the triangle inequality
  \[%
    \lvert \sin((n + 1)x) \rvert = \lvert \sin(nx)\cos(x) + \cos(nx)\sin(x) \rvert \le \lvert \sin(nx)\cos(x) \rvert + \lvert \cos(nx)\sin(x) \rvert
  .\]%
  Using the fact that $\lvert \cos(x) \rvert \le 1$, this becomes
  \[%
    \lvert \sin((n + 1)x) \rvert \le \lvert \sin(nx) \rvert + \lvert \sin(x) \rvert
  .\]%
  By the induction hypothesis, $\lvert \sin(nx) \rvert \le n \lvert \sin(x)
  \rvert$, so
  \[%
    \lvert \sin((n + 1)x) \rvert \le n \lvert \sin(x) \rvert + \lvert \sin(x) \rvert \implies \lvert \sin((n + 1)x) \rvert \le (n + 1) \lvert \sin(x) \rvert
  .\]%
  Thus, $P(n + 1)$ is true.

  Hence, by the PMI, we've shown that $(\forall n \in \N)[P(n) \implies P(n +
  1)]$.
\end{proof}

\medskip

\begin{problem}[3]
  Let $f_n$ denote the Fibonacci sequence: $f_0 = 0$, $f_1 = 1$, and $f_n =
  f_{n-1} + f_{n-2}$ for all $n \ge 2$. Prove that $f_n <
  \left(\frac{7}{4}\right)^n$ for all $n \in \N$.
\end{problem}

\begin{proof}[Solution]
  Let $P(n) : f_n < \left(\frac{7}{4}\right)^n$, for all $n \in \N$.

  Base Case: For $n = 0$, $f_0 = 0$ and $\left(\frac{7}{4}\right)^0 = 1$.
  Clearly, $f_0 < \left(\frac{7}{4}\right)^0$.  For $n = 1$, $f_1 = 1$ and
  $\left(\frac{7}{4}\right)^1 = \frac{7}{4}$. Since $1 < \frac{7}{4}$, $P(1)$
  holds.

  Induction Step: Assume $P(k)$ is true for some $k \in \N$ and $f_{k-1} <
  \left(\frac{7}{4}\right)^{k-1}$. We must show that $P(k + 1)$ is true.  By the
  recurrence relation for the Fibonacci sequence,
  \[%
    f_{k+1} = f_k + f_{k-1}
  .\]%
  By the induction hypothesis,
  \[%
    f_k < \left(\frac{7}{4}\right)^k \aand f_{k-1} < \left(\frac{7}{4}\right)^{k-1}
  .\]%
  Adding these inequalities gives
  \[%
    f_{k+1} = f_k + f_{k-1} < \left(\frac{7}{4}\right)^k + \left(\frac{7}{4}\right)^{k-1}
  .\]%
  Factorizing $\left(\frac{7}{4}\right)^k$
  \[%
    \left(\frac{7}{4}\right)^k + \left(\frac{7}{4}\right)^{k-1} = \left(\frac{7}{4}\right)^{k-1} \left(1 + \frac{7}{4}\right)
  .\]%
  Since $1 + \frac{7}{4} = \frac{11}{4}$, this becomes
  \[%
    f_{k+1} < \left(\frac{7}{4}\right)^{k-1} \cdot \frac{11}{4}
  .\]%
  Using $\frac{11}{4} < \left(\frac{7}{4}\right)^2$, we get
  \[%
    \left(\frac{7}{4}\right)^{k-1} \cdot \frac{11}{4} < \left(\frac{7}{4}\right)^{k+1}
  .\]%
  Hence,
  \[%
    f_{k+1} < \left(\frac{7}{4}\right)^{k+1}
  .\]%

  Thus, $P(k+1)$ is true.

  Hence, by the PMI, we have shown that $f_n < \left(\frac{7}{4}\right)^n$ for all $n \in \N$.
\end{proof}

\medskip

\begin{problem}[4]
  Consider the sequence given by $a_n = 2a_{n-1} + 4a_{n-2}$ and initial
  conditions $a_0 = 0$, $a_1 = 3$. Prove that $3 \mid a_n$ for all $n \ge 0$.
\end{problem}

\begin{proof}[Solution]
  Let $P(n) : 3 \mid a_n$ for all $n \ge 0$.

  Base Case: For $n = 0$, $a_0 = 0$. Clearly, $3 \mid a_0$. For $n = 1$, $a_1 =
  3$, and $3 \mid 3$. Thus, $P(0)$ and $P(1)$ are true.

  Induction Step: Assume $P(k)$ and $P(k - 1)$ are true for some $k \ge 1$. We
  must show that $P(k + 1)$ is true. By the recurrence relation,
  \[%
    a_{k+1} = 2a_k + 4a_{k-1}
  .\]%
  Since $3 \mid a_k$ and $3 \mid a_{k-1}$ by the induction hypothesis, there
  exist integers $m$ and $n$ such that $a_k = 3m$ and $a_{k-1} = 3n$.
  Substituting these into the recurrence relation gives
  \[%
    a_{k+1} = 2(3m) + 4(3n) = 6m + 12n = 3(2m + 4n)
  .\]%
  Since $2m + 4n$ is an integer, it follows that $3 \mid a_{k+1}$.

  Thus, $P(k + 1)$ is true.

  Hence, by the PMI, we've shown that $(\forall n \ge 0)[P(n) \implies P(n +
  1)]$.
\end{proof}

\medskip

\begin{problem}[5]
  Imagine that you have an infinite supply of 6-cent and 11-cent stamps. Prove
  that for all $n \ge 50$, you can make a combination of your stamps that
  exactly totals $n$ cents. Use 2nd principle of mathematical induction.
\end{problem}

\begin{proof}[Solution]
  Let $P(n): \text{For all } n \geq 50, \text{ there exists a combination of }
  6\text{-cent and } 11\text{-cent stamps that totals } n \text{ cents.}$

  Base Cases: We verify $P(50)$, $P(51)$, $P(52)$, $P(53)$, $P(54)$, and
  $P(55)$.

  \begin{itemize}
    \item For $n = 50$, use $50 = 4 \cdot 11 + 1 \cdot 6$ (4 stamps of 11 cents
      and 1 stamp of 6 cents).

    \item For $n = 51$, use $51 = 3 \cdot 11 + 3 \cdot 6$.

    \item For $n = 52$, use $52 = 2 \cdot 11 + 5 \cdot 6$.

    \item For $n = 53$, use $53 = 1 \cdot 11 + 7 \cdot 6$.

    \item For $n = 54$, use $54 = 9 \cdot 6$.

    \item For $n = 55$, use $55 = 5 \cdot 11$.
  \end{itemize}

  Thus, $P(50)$, $P(51)$, $P(52)$, $P(53)$, $P(54)$, and $P(55)$ are true.

  Induction Step: Assume $P(k)$ is true for all $k$ such that $50 \leq k \leq m$
  for some $m \geq 55$. We must show that $P(m + 1)$ is true. By the inductive
  hypothesis, there exists a combination of 6-cent and 11-cent stamps that
  totals $m - 5$ cents (since $m - 5 \geq 50$). Adding one more 6-cent stamp to
  this combination gives a total of $(m - 5) + 6 = m + 1$ cents.

  Hence, $P(m + 1)$ is true.

  By the 2nd PMI, we've shown that $(\forall n \geq 50)[P(n) \implies P(n +
  1)]$.
\end{proof}

\medskip

\begin{problem}[6]
  Define the ``Tribonacci'' sequence as follows: $T_1 = T_2 = T_3 = 1$ and $T_n
  = T_{n-1} + T_{n-2} + T_{n-3}$ for all $n \ge 4$. Prove that for all natural
  numbers $n \ge 1$ one has $T_n < 2^n$.
\end{problem}

\begin{proof}[Solution]
  Let $P(n): T_n < 2^n$, for all $n \geq 1$.

  Base Cases: For $n = 1, 2, 3$, we have $T_1 = T_2 = T_3 = 1$. Clearly, $1 <
  2^1$, $1 < 2^2$, and $1 < 2^3$. Thus, $P(1)$, $P(2)$, and $P(3)$ hold.

  Induction Step: Assume $P(n)$ is true for some $n \geq 3$ and for $n - 1$ and
  $n - 2$. That is, assume $T_n < 2^n$, $T_{n-1} < 2^{n-1}$, and $T_{n-2} <
  2^{n-2}$. We must show that $P(n + 1)$ is true. By the recurrence relation for
  the Tribonacci sequence,  $T_{n+1} = T_n + T_{n-1} + T_{n-2}$. Applying the
  induction hypothesis, we have $T_{n+1} < 2^n + 2^{n-1} + 2^{n-2}$. Factorizing
  powers of $2$, $T_{n+1} < 2^{n-2}(2^2 + 2^1 + 1) = 2^{n-2} \cdot 7$. Since $7
  < 2^3$, this simplifies to
  \[%
    T_{n+1} < 2^{n-2} \cdot 2^3 = 2^{n+1}
  .\]%
  Hence, $P(n + 1)$ is true.

  By the PMI, we've shown that $(\forall n \geq 1)[P(n) \implies P(n + 1)]$.
\end{proof}

\medskip

\begin{problem}[7]
  Recall the Finonacci sequence $f_0$, $f_1$, $f_2$, $\dots$. The following is a
  faulty proof (using the 2nd principle of mathematical induction) that all
  Fibonacci numbers are even:

  I. $f_0 = 0$ which is even.

  II. Suppose $n \in \N$ and $f_k$ is even for all $0 \le k \le n$.

  Then $f_{n+1} = f_n + f_{n-1}$. By the induction hypothesis, both $f_n$ and
  $f_{n-1}$ are even. So $f_{n+1}$ is the sum of two even numbers, hence even.

  III. By PSMI, $f_n$ is even for all $n \in \N$.

  Goal: Find the mistake in the above proof.
\end{problem}

\begin{proof}[Solution]
  The mistake in the proof lies in the induction hypothesis and its application.

  Specifically, the induction hypothesis assumes that all Fibonacci numbers
  $f_k$ are even for all $0 \leq k \leq n$. However, this is not true because
  Fibonacci numbers alternate (even, odd, odd, even, odd, odd, and so
  on). For instance
  \[%
    f_0 = 0~\textrm{(even)}, \quad f_1 = 1~\textrm{(odd)}, \quad f_2 = 1~\textrm{(odd)}, \quad f_3 = 2~\textrm{(even)}, \quad \cdots
  .\]%
  In particular
  \begin{itemize}
    \item The induction step incorrectly applies the hypothesis to $f_n$ and
      $f_{n-1}$, asserting that both are even. This assumption fails because
      Fibonacci numbers alternate, meaning if $f_n$ is even, then
      $f_{n-1}$ is odd (and vice versa).

    \item For $f_{n+1} = f_n + f_{n-1}$, if $f_n$ is even and $f_{n-1}$ is odd
      (or vice versa), $f_{n+1}$ will be odd, not even.
  \end{itemize}

  Thus, the proof incorrectly assumes that $f_n$ and $f_{n-1}$ are both even,
  leading to the false conclusion that all Fibonacci numbers are even.

  Key Counterexample: The Fibonacci sequence quickly disproves the claim
  \[%
    f_0 = 0 \text{ (even)}, \, f_1 = 1 \text{ (odd)}, \, f_2 = 1 \text{ (odd)}, \, f_3 = 2 \text{ (even)}, \, f_4 = 3 \text{ (odd)}, \, \dots
  .\]%
  Hence, the proof fails to account for the actual behavior of the Fibonacci
  sequence.
\end{proof}

\medskip

\begin{problem}[8]
  In this question we will prove that you can lift any cow. Let $P(n)$ be the
  statement ``you Can lif the cow on day $n$ of its life.'' We will use
  induction to prove $(\forall n \in \N)[n \ge 1 \implies P(n)]$.

  I. When the cow is bornw, it is very small, and of course you can lift it.
  This proves $P(1)$.

  II. Suppose that you can lift the cow on day $n$. It grows very little by the
  next day, so you will still be able to lift it on day $n + 1$.

  III. By induction, we have proven that you can lift the cow on every day of
  its life. Boy, you are strong.

  Explain what you think is wrong with this proof.
\end{problem}

\begin{proof}[Solution]
  The flaw in the proof is in the implicit assumption in the induction step. The
  proof claims that the growth of the cow between days $n$ and $n + 1$ is
  negligible. But this assumption does not hold indefinitely.

  \begin{itemize}
    \item The base case $P(1)$ is correct: when the cow is born, it is small,
      and it is reasonable to assume that you can lift it. Thus, $P(1)$ holds.

    \item The induction step assumes that if you can lift the cow on day $n$,
      then you can also lift it on day $n + 1$ because the cow grows ``very
      little'' between consecutive days. But this reasoning doesn't consider
      cumulative growth. While the cow's daily growth may be negligible at
      first, over time, this growth becomes significant, eventually making the
      cow too heavy to lift.

    \item The critical error is that the proof relies on the inductive step
      indefinitely without addressing the increasing weight of the cow over
      time. At some point, the assumption that ``you can lift the cow on day $n$
      implies you can lift it on day $n+1$'' breaks down because the cow's
      weight exceeds your lifting capacity.

    \item This reasoning leads to a contradiction with reality: cows grow to be
      large animals that are impossible to lift manually. Thus, the inductive
      argument fails because the assumption about negligible growth does not
      hold for all $n \ge 1$.
  \end{itemize}

  Therefore, the proof incorrectly extends the inductive step without accounting
  for the physical limits of lifting a growing cow over time.
\end{proof}

\medskip

\begin{problem}[9]
  Prove that $(A \cap B) \times C = (A \times C) \cap (B \times C)$.
\end{problem}

\begin{proof}[Solution]
  Assume $(x, y) \in (A \cap B) \times C$. Then, $x \in A \cap B$ and $y \in C$.
  Then, by definition, $x \in A$ and $x \in B$. Thus, $(x, y) \in A \times C$
  and $(x, y) \in B \times C$. Therefore, $(x, y) \in (A \times C) \cap (B
  \times C)$. Therefore, $(A \cap B) \times C \subseteq (A \times C) \cap (B
  \times C)$.

  Assume $(x, y) \in (A \times C) \cap (B \times C)$. Then, $(x, y) \in A \times
  C$ and $(x, y) \in B \times C$. Thus, $x \in A$ and $x \in B$. Therefore, $x
  \in A \cap B$. Thus, $(x, y) \in (A \cap B) \times C$. Therefore, $(A \times
  C) \cap (B \times C) \subseteq (A \cap B) \times C$.

  Hence, $(A \cap B) \times C = (A \times C) \cap (B \times C)$.
\end{proof}

\medskip

\begin{problem}[10]
  Suppose $A \subseteq X$ and $B \subseteq Y$. Prove that $(X \times Y) - [(A
  \times Y) \cup (X \times B)] = (X - A) \times (Y - B)$.
\end{problem}

\begin{proof}[Solution]
  Assume $(x, y) \in (X \times Y) - [(A \times Y) \cup (X \times B)]$. Then, $x
  \in X$ and $x \notin A \cup X$. Since $A \subseteq X$, then $x \notin A$.
  Therefore, $x \in X - A$. Similarly, $y \in Y - B$. Thus, $(x, y) \in (X - A)
  \times (Y - B)$. Therefore, $(X \times Y) - [(A \times Y) \cup (X \times B)]
  \subseteq (X - A) \times (Y - B)$.

  Assume $(x, y) \in (X - A) \times (Y - B)$. Then, $x \in X - A$ and $y \in Y -
  B$. Thus, $x \in X$ and $x \notin A$ and $y \in Y$ and $y \notin B$. Thus,
  $(x, y) \in X \times Y$ and $(x, y) \notin A \times Y$ and $(x, y) \notin X
  \times B$. Therefore, $(x, y) \in (X \times Y) - [(A \times Y) \cup (X \times
  B)]$. Therefore, $(X - A) \times (Y - B) \subseteq (X \times Y) - [(A \times
  Y) \cup (X \times B)]$.

  Hence, $(X \times Y) - [(A \times Y) \cup (X \times B)] = (X - A) \times (Y -
  B)$.
\end{proof}
