\begin{problem}
  Show that  $[U \land P] \implies [Q \land R]$, $P\iff [S \lor T]$, $R \land T$
  $\vdash$ $U \implies Q$.
\end{problem}

\begin{probsolution}
\end{probsolution}

\newpage

\begin{problem}
  Show that $\not P \implies Q$, $P \implies \not Q$, $P\iff R \vdash \not Q
  \iff R$.  (Hint: Recall that $X\iff Y$ is an abbreviation for something.)
\end{problem}

\begin{probsolution}
\end{probsolution}

\newpage

\begin{problem}
  Show that $R \lor S$, $\not P$, $Q \lor \not R$, $P \iff Q$ $\vdash$ $S$.
\end{problem}

\begin{probsolution}
\end{probsolution}

\newpage

\begin{problem}
  Translate each of the following into a normal English sentence. Also, identify
  the statement as True or False. For example, the proposition
  \[%
    (\forall x)[(x \in \Z \land x \mid 24) \implies x \mid 48]
  .\]%
  says that ``Every divisor of 24 is a divisor of 48''.  It is True.
  \begin{enumerate}
    \item $(\forall x)[(x \in \N \land x \mid 13) \implies x \in \{1,13\}]$.
    \item $(\exists a)[a \in \N \land (a > 51 \land a < 52)]$.
    \item $(\forall x)[(x \in \N \land x^2 = 2) \implies x = 5]$.
    \item $(\forall x)[x \in \N \implies (\exists y)[y \in \N \land y \mid x]]$.
    \item $(\forall n)[(n \in \N \land (4 \mid n \land 6 \mid n)) \implies 24
      \mid n]$.
  \end{enumerate}
\end{problem}

\begin{probsolution}
\end{probsolution}

\newpage

\begin{problem}
  The phrase ``the integer $x$ is a perfect square'' means $x \in \Z \land
  (\exists y \in \Z)[x = y^2]$. Keeping this in mind, write out mathematical
  statements---using only quantifiers and other math symbols, no English
  words---which say the same thing as the following sentences. Do not worry
  about whether the statements are true or false!
  \begin{enumerate}
    \item For every integer $a$, if $a$ is even then $a^2$ is a multiple of $5$.

    \item 5 is smaller than every integer.

    \item Every integer which is a perfect square is also a perfect cube.

    \item Every nonzero integer is either a multiple of six or a multiple of
      seven.

    \item There exists an integer which is larger than every other integer.

    \item There is an element of the set $A$ having the property that every
      element of the set $B$ divides it.

    \item Every element of the set $A$ is either even or a multiple of $13$.

    \item There exists an integer which is not divisible by any divisor of 240.

    \item Every nonzero element of $\Z_5$ has a multiplicative inverse.

    \item For all integers $x$, if $x$ is congruent to three mod eight then
      there exists an integer $y$ such that $x\cdot y$ is congruent to one mod
      eight.

    \item There exists an integer $p$ with the following properties: $p$ is even
      and for every pair of integers $a$ and $b$, if $p$ divides $ab$ then it
      must divide either $a$ or $b$.
  \end{enumerate}
\end{problem}

\begin{probsolution}
\end{probsolution}

\newpage

\begin{problem}
  In each part, identify the given set by listing all of its elements.  For
  example:
  \[%
    \textrm{Given the set $\{x\in \Z \mid x \equiv_4 3 \land 5 \leq x \land x \leq 20\}$ you would answer $\{7, 11, 15, 19\}$, as these two sets are equal.}
  .\]%
  \begin{enumerate}
    \item $\{x \in \Z_4 \mid (\exists y)[y \in \Z_4 \land y \neq 0 \land xy =
      0]\}$.

    \item $\{x^3 + 1 \mid x \in \N\} \cap \{y \mid y \in \N \land 1 \leq y \leq
      30\}$.

    \item $\{x \in \N \mid (\exists a)(\exists b)[a \in \N \land b \in \N \land
      x = a^2 + b^2]\} \cap \{x \mid x \in \N \land x \leq 20\}$.

    \item $\{z \in \Z_{30} \mid (\exists u)[u \in \Z_{30} \land zu = 1]\}$.

    \item $\{a \mid a\in \Z \land a \equiv_4 1\} \cap \{b \mid b\in \Z \land b
      \equiv_2 0\}$.

    \item $\{(a,b) \mid a \in \Z_2 \land b \in \Z_2 \land a + b = 1\}$.

    \item $\{x \in  \mid x \equiv_5 1 \land x \leq 40\} \cap \{x \in \N \mid x
      \equiv_6 4\}$.
  \end{enumerate}
\end{problem}

\begin{probsolution}
\end{probsolution}

\newpage

\begin{problem}
  In each part below, use mathematical notation to write the negation of the
  given statement in such a way that no quantifier is immediately preceded by a
  negation sign, every universal quantifer is applied to a conditional, and
  every existential quantifier is applied to a conjunction. In each part, decide
  which statement is true: the given statement or its negation.
  \begin{enumerate}
    \item $(\forall y)[y \in \Z \implies y > 0]$.

    \item $(\forall x)[(x \in \Z\land 2 \mid x) \implies 4 \mid x]$.

    \item $(\forall x)[x \in \Z_6 \implies (\exists y)[y \in \Z_6 \land x +_6 y
      = 0]]$.

    \item $(\exists x)[x \in \Z_6 \land (\forall y)[y \in \Z_6 \implies x +_6 y
      = 0]]$.

    \item $(\forall x)[x \in \Z \implies (\exists y)[y \in \Z \land x = y^2]]$.

    \item $(\exists a)[a \in \N \land (\forall b)[[b \in \N \land a \neq b]
      \implies b > a]]$.

    \item $(\exists m)(\exists n)[m, n\in \N \land m > n]$.

    \item $(\forall m)(\forall n)[m, n \in \N \implies m > n]$.

    \item $(\exists m)[m \in \N \land (\forall n)[n \in \N \implies m > n]$.

    \item $(\forall m)[m \in \N \implies (\exists n)[n \in \N \land m > n]$.

    \item $(\forall a)(\forall b)[(a, b \in \R \land a < b ) \implies (\exists
      c)[c \in \R \land [a < c\land c < b]]]$.
  \end{enumerate}
\end{problem}

\begin{probsolution}
\end{probsolution}

\newpage

\begin{problem}
  Fill in the blanks in the outline below to prove that $(\forall a, b, k)[[a,
  b, k \in \Z \land (k \geq 1 \land a \mid b)] \implies a^k \mid b^k]$.

  \noindent
  Proof:

  \begin{tabular}{lll}
    1. & Assume $a, b, k \in \Z$ and $k \geq 1$ and $a \mid b$.\\
    2. & $\fbox{\parbox{0.5in}{\ \ \ \ \ \par \ \ }} (y \in \Z \land b = y\cdot a)$. \\
    3. & $b = y \cdot a$ for some \fbox{\parbox{0.5in}{\ \ \ \ \ \par \ \ }}. \\
    4. & $b^k = \ $\fbox{\parbox{0.5in}{\ \ \ \ \ \par \ \ }}. \\
    5. & $(\exists u)(u \in \Z \land \ $\fbox{\parbox{0.5in}{\ \ \ \ \ \par \ \ }}$)$. \\
    6. & $a^k \mid b^k$. \\
    7. & \fbox{\parbox{0.5in}{\ \ \ \ \ \par \ \ }}  & DT, discharge For 1. \\
    8. & $(\forall a, b, k)([a, b, k\in \Z \land (k \geq 1 \land a \mid b)] \implies a^k \mid b^k)$. \\
  \end{tabular}

  \vspace{0.1in}

  Answer the following questions:
  \begin{enumerate}
    \item The above proof used one IE step, one EI step, and one IU step. Label
      them in column 2 of your proof (the same column with DT in it).

    \item The definition of $a \mid b$ used $k$ for the variable in the
      existential statement.  In step 2, why did we change and use $y$ instead?

    \item In step 5 we introduced the variable $u$ in the existential statement.
      Could we have used $k$ here instead?  What about $y$?
  \end{enumerate}
\end{problem}

\begin{probsolution}
\end{probsolution}

\newpage

\begin{problem}
  Fill in the blanks in the outline below to prove that $(\forall a, b, c)[ [a,
  b, c \in \Z \land (a \mid b \land b \mid c)] \implies a \mid c]$. [Hint: Notice
  that $p$ and $q$, from step 6, will have to appear in steps 1--5 somewhere.]

  \noindent
  Proof:

  \begin{tabular}{lll}
    1. & Assume \fbox{\parbox{0.5in}{\ \ \ \ \ \par \ \ }}. \\
    2. & $(\exists k)[k \in \Z \land b = k\cdot a]$. \\
    3. & \fbox{\parbox{0.5in}{\ \ \ \ \ \par \ \ }} for some \fbox{\parbox{0.5in}{\ \ \ \ \ \par \ \ }}. \\
    4. & $(\exists k)[k \in \Z \land c = k\cdot b]$. \\
    5. & \fbox{\parbox{0.5in}{\ \ \ \ \ \par \ \ }} for some \fbox{\parbox{0.5in}{\ \ \ \ \ \par \ \ }}. \\
    6. & $c = q\cdot b = p\cdot (qa) = pq \cdot a$. \\
    7. & $(\exists r)[r \in \Z \land \fbox{\parbox{0.5in}{\ \ \ \ \ \par \ \ }}\ ]$. \\
    8. & $a \mid c$. \\
    9. & \fbox{\parbox{0.5in}{\ \ \ \ \ \par \ \ }}  & DT, discharge For 1. \\
    10. & \fbox{\parbox{0.5in}{\ \ \ \ \ \par \ \ }}. \\
  \end{tabular}

  \vspace{0.3in}

  Answer the following questions:
  \begin{enumerate}
    \item The above proof used two IE steps, one EI step, and one IU step. Label
      them in column 2 of your proof (the same column with DT in it). You do not
      have to give reasons for any of the other steps.

    \item It is okay that we used $k$ in both step 2 and step 4.  Why? (This
      might be hard to explain in words, but at least try to come to some kind
      of understanding for yourself).
  \end{enumerate}
\end{problem}

\begin{probsolution}
\end{probsolution}

\newpage

\begin{problem}
  Fill in the blanks in the outline below to prove that $(\forall n)[(n \in \N
  \land 3 \not \mid n) \implies n^2 \equiv_3 1]$. Also:
  \begin{enumerate}
    \item There are three uses of the Deduction Theorem in this proof. Label the
      appropriate DT steps.

    \item There are two steps at the end of the proof with the phrase ``Logical
      rule?'' in bold. For these, label the appropriate rule from symbolic
      logic that is being used.
  \end{enumerate}

  \begin{table}[H]
  \begin{tabular}{lll}
    1. & Assume \fbox{\parbox{0.5in}{\ \ \ \ \ \par \ \ }}. \\
    2. & $n \not\equiv_3 0$. \\
    3. & So either $n \equiv_3 1$ or $n \equiv_3 2$. \\
    4. & Assume $n \equiv_3 1$. \\
    5. & Then $3 \mid \fbox{\parbox{0.5in}{\ \ \ \ \ \par \ \ }}$. \\
    6. & So $n - 1 = 3\cdot P$ for some \fbox{\parbox{0.5in}{\ \ \ \ \ \par \ \ }}. \\
    7. & $n = $\fbox{\parbox{0.5in}{\ \ \ \ \ \par \ \ }}. \\
    8. & $n^2 = $\fbox{\parbox{0.5in}{\ \ \ \ \ \par \ \ }}$ = 3\cdot \fbox{\parbox{0.5in}{\ \ \ \ \ \par \ \ }} + 1$. \\
    9. & $n^2 - 1 = \fbox{\parbox{0.5in}{\ \ \ \ \ \par \ \ }}$. \\
    10. & $(\exists y)[y \in \Z \land \fbox{\parbox{0.5in}{\ \ \ \ \ \par \ \ }}]$. \\
    11. & $3 \mid \fbox{\parbox{0.5in}{\ \ \ \ \ \par \ \ }}$. \\
    12. & $n^2 \equiv_3 1$. \\
    13. & So $n \equiv_3 1 \implies n^2 \equiv_3 1$. \\
    14. & Now assume $n \equiv_3 2$.\\
    16. & \fbox{\parbox{0.5in}{\ \ \ \ \ \par \ \ }}. \\
    17. & \fbox{\parbox{0.5in}{\ \ \ \ \ \par \ \ }}. \\
    18. & \fbox{\parbox{0.5in}{\ \ \ \ \ \par \ \ }}. \\
    19. & \fbox{\parbox{0.5in}{\ \ \ \ \ \par \ \ }}. \\
    20. & \fbox{\parbox{0.5in}{\ \ \ \ \ \par \ \ }}. \\
    21. & \fbox{\parbox{0.5in}{\ \ \ \ \ \par \ \ }}. \\
    22. & \fbox{\parbox{0.5in}{\ \ \ \ \ \par \ \ }}. \\
    23. & \fbox{\parbox{0.5in}{\ \ \ \ \ \par \ \ }}. \\
    24. & \fbox{\parbox{0.5in}{\ \ \ \ \ \par \ \ }}. \\
    25. & So $n \equiv_3 2 \implies n^2 \equiv_3 1$. \\
    26. & We therefore have $(n \equiv_3 1 \lor n \equiv_3 2) \implies n^2 \equiv_3 1$. & {\bf Logical rule?} \\
    27. & So $n^2 \equiv_3 1$. & {\bf Logical rule?} \\
    28. & $(n \in \N \land 3 \not\mid n) \implies n^2 \equiv_3 1$. \\
    29. & \fbox{\parbox{0.5in}{\ \ \ \ \ \par \ \ }}. \\
  \end{tabular}
  \end{table}
\end{problem}

\begin{probsolution}
\end{probsolution}

\newpage

\begin{problem}
  Prove $(\forall n)[(n \in \N \land n \equiv_6 3) \implies n^2 + 2n + 10
  \equiv_{12} 1]$.
\end{problem}

\begin{probsolution}
\end{probsolution}

\newpage

\begin{problem}
  Prove $(\forall n)[(n \in \N \land n \equiv_5 2) \implies (2 \not\mid n \lor
  n^2 \equiv_{20} 4)]$.

  [Hint: Remember that $(X \lor Y) \iff (\not X \implies Y)$. Use DT twice in
  this proof.]
\end{problem}

\begin{probsolution}
\end{probsolution}
