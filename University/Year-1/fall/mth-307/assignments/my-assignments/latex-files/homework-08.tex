\begin{problem}
  In this problem we have four functions, as indicated in the diagram below.

  You are given that $q \circ f = g \circ p$. Let $X \subseteq C$.
  \begin{enumerate}
    \item Prove that $f(p^{-1}(X)) \subseteq q^{-1}(g(X))$.

    \item If $p$ is onto and $q$ is one-to-one, prove that $f(p^{-1}(X)) =
      q^{-1}(g(X))$.
  \end{enumerate}
\end{problem}

\begin{probsolution}
  \begin{enumerate}
    \item Here's the line proof for $f(p^{-1}(X)) \subseteq q^{-1}(g(X))$.

      \hspace*{-0.5em}\begin{tabular}{ll}
        1. & Assume $x \in f(p^{-1}(X))$. \\
        2. & Then, $y = f(s)$, where $s \in p^{-1}(X)$. \\
        3. & So, $p(s) \in X$. \\
        4. & Then, $g(p(s)) \in g(X)$. \\
        5. & Using the property $q \circ f = g \circ p$, we have $q(f(s)) \in g(x)$. \\
        6. & Hence, $y = f(s) \in q^{-1}(g(X))$. \\
        7. & Therefore, $f(p^{-1}(X)) \subseteq q^{-1}(g(X))$. \\
      \end{tabular}

    \item here's the line proof for $f(p^{-1}(X)) = q^{-1}(g(X))$.

      \hspace*{-0.5em}\begin{tabular}{ll}
        1. & Assume $y \in f(p^{-1}(X))$. \\
        2. & Then $q(y) \in g(X)$. \\
        3. & So, $q(t) = g(s)$, for $s \in X$. \\
        4. & Hence, $q(y) = g(p(t))$, for some $t \in p^{-1}(X)$. \\
        5. & Then $q(y) = g(p(t)) = q(f(t))$. \\
        6. & Since $q$ is one-to-one, we have $y = f(t)$. \\
        7. & Therefore, $q^{-1}(g(X)) \subseteq f(p^{-1}(X))$. \\
        8. & Hence, $f(p^{-1}(X)) = q^{-1}(g(X))$. \\
      \end{tabular}
  \end{enumerate}
\end{probsolution}

\newpage

\begin{problem}
  For all $n \ge 2$, $\displaystyle\sum_{k=2}^{n} \frac{1}{k^2 - 1} = \frac{(n -
  1)(3n + 2)}{4n(n + 1)}$.
\end{problem}

\begin{probsolution}
  \begin{proof}
    Let $\displaystyle P(n) : \sum_{k=2}^{n} \frac{1}{k^2 - 1} = \frac{(n -
    1)(3n + 2)}{4n(n + 1)}$.

    Base Case: $P(2) : \frac{1}{3} \ce \frac{1}{3}$.

    Induction Step: Assume $P(k)$ up to $k = n$. Then, adding the next term
    $\frac{1}{(n + 1)^2 - 1}$ to both sides, we get
    \[%
      \left(\sum_{k=2}^{n} \frac{1}{k^2 - 1}\right) + \frac{1}{(n + 1)^2 - 1} = \frac{(n - 1)(3n + 2)}{4n(n + 1)} + \frac{1}{(n + 1)^2 - 1}
    .\]%
    If $P(n) \implies P(n + 1)$, then the following statement must be true
    \begin{align*}
      \frac{(n - 1)(3n + 2)}{4n(n + 1)} + \frac{1}{(n + 1)^2 - 1} &= \frac{n(3n + 5)}{4(n + 1)(n + 2)} \\
      \frac{[(n - 1)(3n + 2)][(n + 1)^2 - 1] + 4n(n + 1)}{[4n(n + 1)][(n + 1)^2 - 1]} &= \frac{n(3n + 5)}{4(n + 1)(n + 2)} \\
      \frac{3n^4 + 5n^3}{4n^4 + 12n^3 + 8n^2} &= \frac{3n^2 + 5n}{412n^2 + 12n + 8} \\
      \frac{3n^4 + 5n^3}{4n^4 + 12n^3 + 8n^2} &= \frac{3n^2 + 5n}{412n^2 + 12n + 8} \cdot \frac{n^2}{n^2} \\
      \frac{3n^4 + 5n^3}{4n^4 + 12n^3 + 8n^2} &\ce \frac{3n^4 + 5n^3}{4n^4 + 12n^3 + 8n^2}
    .\end{align*}
    Therefore, $(\forall n \in \N)[P(n) \implies P(n + 1)]$. Hence, by the EPMI,
    we have proven the statement.
  \end{proof}
\end{probsolution}

\newpage

\begin{problem}
  Let $a_n = 1 - 2 + 3 - 4 + \cdots + (-1)^{n+1} n$. Prove by induction that
  $a_{2n} = -n$ for all $n \ge 1$.
\end{problem}

\begin{probsolution}
  \begin{proof}
    Let $P(n) : a_{2n} = -n$ for all $n \ge 1$.

    Base Case: $P(1) : a_{2 \cdot 1} = a_2 = 1 - 2 = -1$.

    Induction Step: Assume $P(k)$ up to $k = n$. We need to prove that
    $a_{2(n+1)} = -(n+1)$. By definition,
    \[%
      a_{2(n+1)} = a_{2n} + (-1)^{2n+1}(2n + 1) + (-1)^{2n+2}(2n + 2)
    .\]%
    Substitute $(-1)^{2n+1} = -1$ and $(-1)^{2n+2} = 1$ to get
    \[%
      a_{2(n+1)} = a_{2n} - (2n + 1) + (2n + 2) \implies a_{2(n+1)} = a_{2n} + 1
    .\]%
    Using the inductive hypothesis that $a_{2n} = -n$, we get
    \[%
      a_{2(n+1)} = -n + 1 = -(n + 1)
    .\]%

    Therefore, $(\forall n \in \N)[P(n) \implies P(n + 1)]$. Hence, by the PMI,
    we have proven the statement.
  \end{proof}
\end{probsolution}

\newpage

\begin{problem}
  Suppose $f : \Z \to \Z$ is a function with the property that $(\forall x \in
  \Z)(\forall y \in \Z)[f(x + y) = f(x) + f(y)]$.
  \begin{enumerate}
    \item Prove by induction that $(\forall n \in \N)[n \ge 1 \implies (\forall
      x \in \Z)[f(nx) = n \cdot f(x)]]$.

    \item Prove that $(\forall k \in \N)[f(M_k) \subseteq M_k]$.
  \end{enumerate}
\end{problem}

\begin{probsolution}
  \begin{enumerate}
    \item \begin{proof}
        Let $P(n) : n \geq 1 \implies (\forall x \in \Z)[f(nx) = n \cdot f(x)].$

        Base Case: $P(1) : f(1 \cdot x) = 1 \cdot f(x) = f(x)$.

        Induction Step: Assume $P(k)$ is true for some $k \geq 1$, i.e.,
        \[%
          (\forall x \in \Z)[f(nx) = n \cdot f(x)]
        .\]%
        We must show $P(n + 1)$, i.e.,
        \[%
          (\forall x \in \Z)[f((n + 1)x) = (n + 1) \cdot f(x)]
        .\]%
        Let's consider $f((n + 1)x) = f(nx + x)$. By the given property of $f$,
        we get
        \[%
          f(nx + x) = f(nx) + f(x)
        .\]%
        Using the inductive hypothesis, $f(nx) = n \cdot f(x)$, we get
        \[%
          f((n + 1)x) = n \cdot f(x) + f(x) = (n + 1) \cdot f(x)
        .\]%
        Therefore, $(\forall n \in \N)[P(n) \implies P(n+1)]$. Hence, by the
        PMI, $(\forall n \in \N)[P(n) \implies P(n + 1)]$, and the result is
        proven.
      \end{proof}

    \item \begin{proof}
        Let $M_k = \{y \in \Z \mid y \equiv 0 \pmod{k}\}$. We need to prove
        \[%
          (\forall k \in \N)[f(M_k) \subseteq M_k]
        .\]%
        Fix $k \in \N$ and let $y \in M_k$. Then, by definition, $y = kx$ for
        some $x \in \Z$. From Part (1), $f(y) = f(kx) = k \cdot f(x)$. Since $k
        \cdot f(x)$ is a multiple of $k$, we have $f(y) \in M_k$. Therefore,
        $f(M_k) \subseteq M_k$ for all $k \in \N$.
      \end{proof}
  \end{enumerate}
\end{probsolution}

\newpage

\begin{problem}
  For all $n \ge 2$, $\sqrt{n} < \frac{1}{\sqrt{1}} + \frac{1}{\sqrt{2}} +
  \frac{1}{\sqrt{3}} + \cdots + \frac{1}{\sqrt{n}}$.
\end{problem}

\begin{probsolution}
  \begin{proof}
    Let $P(n) : n \geq 2 \implies \sqrt{n} < \sum_{k=1}^n \frac{1}{\sqrt{k}}.$

    Base Case: $P(2) : \sqrt{2} < \frac{1}{\sqrt{1}} + \frac{1}{\sqrt{2}} = 1 +
    \frac{1}{\sqrt{2}}$, which is true.

    Induction Step: Assume $P(n)$ is true for some $n \geq 2$. We must show
    $P(k+1)$, i.e.,
    \[%
      \sqrt{n + 1} < \sum_{k=1}^{n+1} \frac{1}{\sqrt{k}}
    .\]%
    Start with the right-hand side:
    \[
      \sum_{k=1}^{n+1} \frac{1}{\sqrt{k}} = \sum_{k=1}^n \frac{1}{\sqrt{k}} + \frac{1}{\sqrt{n + 1}}
    .\]%
    By the inductive hypothesis, $\sqrt{n} < \sum_{k=1}^n \frac{1}{\sqrt{k}}$,
    so
    \[%
      \sqrt{n + 1} < \sqrt{n} + \frac{1}{\sqrt{n + 1}}
    .\]%
    To prove this inequality, we show
    \[
      \sqrt{n + 1} - \sqrt{n} < \frac{1}{\sqrt{n + 1}}
    .\]%
    Rewrite $\sqrt{n + 1} - \sqrt{n}$ to get
    \[%
      \sqrt{n + 1} - \sqrt{n} = \frac{(n + 1) - n}{\sqrt{n + 1} + \sqrt{n}} = \frac{1}{\sqrt{n + 1} + \sqrt{n}}
    .\]%
    Since $\sqrt{n + 1} + \sqrt{n} > \sqrt{n + 1}$, it follows that
    \[%
      \frac{1}{\sqrt{n + 1} + \sqrt{n}} < \frac{1}{\sqrt{n + 1}}
    .\]%
    Hence,
    \[%
      \sqrt{n + 1} - \sqrt{n} < \frac{1}{\sqrt{n + 1}} \aand \sqrt{n + 1} < \sqrt{n} + \frac{1}{\sqrt{n + 1}}
    .\]%

    Therefore, $(\forall n \in \N)[P(n) \implies P(n+1)]$. By the PMI, $(\forall
    n \geq 2)[P(n) \implies P(n + 1)]$ is true, and the result is proven.
  \end{proof}
\end{probsolution}

\newpage

\begin{problem}
  For all $n \ge 2$, $\left(\frac{2^2 - 1}{2^2}\right) \cdot \left(\frac{3^2 -
  1}{3^2}\right) \cdots \left(\frac{n^2 - 1}{n}\right) = \frac{n + 1}{2n}$.
\end{problem}

\begin{probsolution}
\begin{proof}
  Let $\displaystyle P(n) : n \geq 2 \implies \prod_{k=2}^n \frac{k^2 - 1}{k^2} = \frac{n+1}{2n}$.

  Base Case: $\displaystyle P(2) : \prod_{k=2}^2 \frac{k^2 - 1}{k^2} = \frac{n+1}{2n} \implies \frac{3}{4} \ce \frac{3}{4}$.

  Induction Step: Assume $P(n)$ is true for some $n \geq 2$, i.e.,
  \[%
    \prod_{k=2}^n \frac{k^2 - 1}{k^2} = \frac{n + 1}{2n}
  .\]%
  If $P(n) \implies P(n + 1)$, then the following statement must be true:
  \begin{align*}
    \prod_{k=2}^{n+1} \frac{k^2 - 1}{k^2} &= \prod_{k=2}^n \frac{k^2 - 1}{k^2} \cdot \frac{(n+1)^2 - 1}{(n+1)^2} \\
    \prod_{k=2}^{n+1} \frac{k^2 - 1}{k^2} &= \frac{n + 1}{2n} \cdot \frac{(n+1)^2 - 1}{(n+1)^2} \\
    \prod_{k=2}^{n+1} \frac{k^2 - 1}{k^2} &= \frac{n + 1}{2n} \cdot \frac{n(n+2)}{(n+1)^2} \\
    \prod_{k=2}^{n+1} \frac{k^2 - 1}{k^2} &= \frac{(n+1)n(n+2)}{2n(n+1)^2} = \frac{n+2}{2(n+1)} \\
    \frac{(n+1)+1}{2(n+1)} &\ce \frac{n+2}{2(n+1)}
  .\end{align*}

  Therefore, $(\forall n \in \N)[P(n) \implies P(n+1)]$. By the PMI, $(\forall n
  \geq 2)[P(n) \implies P(n + 1)]$, and the result is proven.
\end{proof}
\end{probsolution}

\newpage

\begin{problem}
  For all $n \ge 2$, $\frac{1}{2} + \frac{2}{3} + \frac{3}{4} + \cdots + \frac{n}{n + 1} < \frac{n^2}{n + 1}$.
\end{problem}

\begin{probsolution}
  \begin{proof}
    Let $P(n) : n \geq 2 \implies \sum_{k=2}^n \frac{k}{k+1} < \frac{n^2}{n+1}.$

    Base Case: $P(2) : \frac{2}{3} < \frac{2^2}{2+1} \implies \frac{2}{3} < \frac{4}{3}$.

    Induction Step: Assume $P(n)$ is true for some $n \geq 2$. We must show
    $P(n) \implies P(n+1)$. Starting with $P(n + 1)$,
    \[%
      \sum_{k=2}^{n+1} \frac{k}{k + 1} = \sum_{k=2}^n \frac{k}{k + 1} + \frac{n + 1}{n + 2}
    .\]%
    By the inductive hypothesis, $\sum_{k=2}^n \frac{k}{k + 1} < \frac{n^2}{n + 1}$, so
    \[%
      \sum_{k=2}^{n+1} \frac{k}{k + 1} < \frac{n^2}{n + 1} + \frac{n + 1}{n + 2}
    .\]%
    We now need to prove
    \[%
      \frac{n^2}{n+1} + \frac{n+1}{n+2} < \frac{(n+1)^2}{n+2}
    .\]%
    Combine the terms on the left-hand side over a common denominator
    \[%
      \frac{n^2}{n + 1} + \frac{n + 1}{n + 2} = \frac{n^2(n + 2) + (n + 1)^2(n + 1)}{(n + 1)(n + 2)}
    .\]%
    Expanding the numerator gives us $n^2(n + 2) + (n + 1)^2(n + 1) = n^3 + 2n^2
    + n^3 + 2n^2 + n + 1 = 2n^3 + 4n^2 + n + 1$. Thus, the right hand side
    becomes
    \[%
      \frac{(n + 1)^2}{n + 2} = \frac{(n^2 + 2n + 1)(n + 2)}{(n + 1)(n + 2)} = \frac{n^3 + 2n^2 + n^2 + 2n + 2n + 4}{(n + 1)(n + 2)}
    .\]%
    Again, simplifying the numerator yields $n^3 + 3n^2 + 4n + 4$. Comparing the
    two numerators, $2n^3 + 4n^2 + n + 1 < n^3 + 3n^2 + 4n + 4$ holds for all $n
    \geq 2$ because the inequality simplifies to a valid comparison.

    Therefore, $(\forall n \in \N)[P(n) \implies P(n+1)]$. Hence, by the EPMI,
    $(\forall n \geq 2)[P(n) \implies P(n + 1)]$, and the result is proven.
  \end{proof}
\end{probsolution}

\newpage

\begin{problem}
  You have a huge collection of ``trionimo'' tiles. Prove by induction that for
  all $k \in \N$ such that $k \ge 1$, a $2^k \times 2^k$ checkerboard with the
  upper-right corner square removed can be tiled using trionimos. [Hint to get
  started: As scratchwork, do the cases $k = 1$, $k = 2$, and $k = 3$ by hand.
  Look for a link between the $2^{k+1} \times 2^{k+1}$ case and the $2^k \times
  2^k$ case].
\end{problem}

\begin{probsolution}
  \begin{proof}
    Let $P(k)$ denote the statement: ``A $2^k \times 2^k$ checkerboard with one
    square removed can be tiled using trionimos.''

    Base Case: For $P(1)$, the checkerboard is a $2 \times 2$ square with one
    square removed, leaving three squares. These three squares can be covered by
    a single trionimo.

    Induction Step: Assume $P(k)$ is true for some $k \geq 1$, i.e., any $2^k
    \times 2^k$ checkerboard with one square removed can be tiled using
    trionimos. We need to show $P(k + 1)$, i.e., a $2^{k+1} \times 2^{k+1}$
    checkerboard with one square removed can also be tiled using trionimos.

    Divide the $2^{k+1} \times 2^{k+1}$ checkerboard into four $2^k \times 2^k$
    subboards: top-left ($A$), top-right ($B$), bottom-left ($C$), and
    bottom-right ($D$). The removed square lies in one of these subboards. Place
    a single trionimo at the lower-left corner of $A$, the lower-right corner of
    $B$, and the upper-left corner of $D$, leaving one square removed from each
    of $A$, $C$, and $D$, while $B$ retains its removed square in the
    upper-right corner.

    By the inductive hypothesis, each of the four $2^k \times 2^k$ subboards
    with one square removed can be tiled using trionimos. Thus, the entire
    $2^{k+1} \times 2^{k+1}$ checkerboard can be tiled.

    Therefore, $(\forall k \in \N)[P(k) \implies P(k + 1)]$. By the PMI,
    $(\forall k \geq 1)[P(k)]$, and the result is proven.
  \end{proof}
\end{probsolution}

\newpage

\begin{problem}
  There is a famous proof that all horses are the same color. Let $P(n)$ be the
  statement ``for all sets of n horses, all the horses in the set have the same
  color''. We will prove this by induction. The base case $n = 1$ is clear,
  since in a set consisting of exactly $1$ horse all the horses have the same
  color. Now assume that $P(n)$ is true, and let $S$ be a set of $n + 1$ horses.
  Label the horses $1, 2, \cdots, n + 1$. Then the first $n$ horses constitute a
  set of $n$ horses, so by the induction hypothesis they all have the same
  color. Likewise, the last $n$ horses are a set of $n$ horses; so by induction
  they all have the same color. But if the first $n$ horses all have the same
  color, and the last $n$ horses all have the same color, then since these two
  sets overlap the two colors must be identical. So all the horses in $S$ have
  the same color, and we are done by induction.

  Find the mistake in the above proof.
\end{problem}

\begin{probsolution}
  The mistake in the proof lies in a subtle flaw in the inductive step. The
  argument does not hold for $n = 2$. For the base case, it's valid because a
  single horse trivially has the same color as itself. The argument fails for $n
  = 2$. When $S$ contains $3$ horses ($n + 1 = 3$), the two subsets considered
  are $\{1, 2\}$, where $P(2)$ claims they have the same color and $\{2, 3\}$,
  where $P(2)$ also claims that they have the same color. However, these two
  subsets overlap at only one horse. This does not guarantee that all three
  horses in $S$ have the same color because there is no information linking
  horse $1$ with horse $3$. The inductive step fails to establish that the color
  of horse $1$ is the same as the color of horse $3$.
\end{probsolution}
