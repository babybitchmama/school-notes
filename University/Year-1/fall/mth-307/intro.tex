{\small
  \noindent\textbf{Understand logic structure}\\
  Interpret and use logical structure in the context of proof. Examples of this
  are: If one wants to prove ``$P \implies Q$'', then one can use direct proof,
  proof by contrapositive, or proof by contradiction. Or if one wants to prove
  $P \lor Q$, then one can use a direct proof of $P$ or $Q$, or show that both
  $P$ and $Q$ cannot be false (i.e., show $\neg P \implies Q$ and $\neg Q
  \implies P$).\hspace*{\fill}

  \vspace{10pt}
  \noindent\textbf{Interpret and use Quantifiers}\\
  Understand the meaning of quantifiers ``$\forall$'' (for all) and
  ``$\exists$'' (there exists). Be able to interpret statements with quantifiers
  and rewrite them in logically equivalent forms. For example, the statement
  ``$(\forall x \in \R)(\exists y \in \R)[x < y]$'' can be interpreted as ``for
  every real number $x$, there exists a real number $y$ such that $x < y$''.
  Understand how to prove statements with quantifiers.\hspace*{\fill}

  \vspace{10pt}
  \noindent\textbf{Write basic proofs}\\
  Be able to write basic proofs in a variety of styles, including direct proof,
  proof by contrapositive, proof by contradiction, and proof by cases.

  \vspace{10pt}
  \noindent\textbf{Induction and strong Induction}\\
  Understand the principle of mathematical induction and be able to apply it to
  prove statements about integers or other well-ordered sets.
}
