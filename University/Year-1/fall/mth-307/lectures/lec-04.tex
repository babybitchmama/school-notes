\nte{Oct 07 2024 Mon (12:01:43)}{Solving equations in $\Z_n$}

\section{Solving Linear Equations}
\label{sec:solving_linear_equations}

$\Z_{13} = \left\{0, 1, 2, \dots, 12\right\}$. Solve the equation
\[%
  3 \cdot_{13} x +_{13} 6 = 3~\textrm{Solve for $x$}
.\]%
I want to move $6$ to the other side. In $\Z_{13}$, add the number $7$ to both
sides of the equation to get
\begin{align*}
  (3 \cdot_{13} x +_{13} 6) +_{13} 7 &= 3 +_{13} 7 \\
  3 \cdot_{13} x +_{13} (6 +_{13} 7) &= 3 +_{13} 7 \\
  3 \cdot_{13} x +_{13} 0 &= 3 +_{13} 7
,\end{align*}
since $6 +_{13} 7 = 0$ in $\Z_n$ and $3 +_{13} 7 = 10$. This gives us
\[%
  3 \cdot_{13} x = 10
.\]%

\begin{note}
  \label{nte:zero_property}

  $a +_m 0 = a$ and $a \cdot_m 1 = a$.
\end{note}

Multiplying both sides of the equation by $9$ to make $9 \cdot_{13} 3 = 1$ to
get
\begin{align*}
  9 \cdot_{13} (3 \cdot_{13} x) &= 9 \cdot_{13} 10 \\
  (9 \cdot_{13} 3) \cdot_{13} x &= 9 \cdot_{13} 10 \\
  1 \cdot_{13} x &= 9 \cdot_{13} 10 \\
  1 \cdot_{13} x &= 12 \\
  x &= 12
.\end{align*}

Checking to see if we got it right gives us
\[%
  3 \cdot_{13} 12 +_{13} 6 = 10 +_{13} 6 = 3
,\]%
which is correct.

\subsection{Inverses}
\label{sub_sec:inverses}

\begin{proposition}
  \label{prp:additive_inverse}

  From any $a$ from $\Z_n$, there exists an additive inverse $b$ such that $a
  +_n b = 0$.
\end{proposition}

\begin{note}
  \label{nte:additive_inverse}

  Just use the following equation $b = m - a$ (except for $a = 0$).
\end{note}

Multiplicative inverse of $a$ from $\Z_n$ is $b$ from $\Z_n$ such that $a
\cdot_m b = 1$.

\begin{note}
  \label{nte:multiplicative_inverse}

  Multiplicative inverse might fail to exist. For example, $a = 0$. Or, $a = 2$
  and $m = 4, 6, \dots$.
\end{note}

All elements in $\Z_n$ (except for $0$) have a multiplicative inverse (in
example from the beginning of lecture notes).

% subsection inverses (end)

% section solving_linear_equations (end)

\section{Solving Quadratic Equations}
\label{sec:solving_quadratic_equations}

In $\Z_{13}$, solve
\[%
  2 \cdot_{13} x^2 +_{13} 5 \cdot_{13} x = 11
.\]%
\begin{note}
  \label{nte:quadratic_equation}

  When I write $x^2$, I mean $x \cdot_{13} x$.
\end{note}

You can re-write all the annoying $+_n$ and $\cdot_n$ as
\[%
  2x^2 + 5x = 11 (\mod(13))
.\]%

Let's get rid of $2$ by multiplying everything by its multiplicative inverse,
which is $7$, giving us
\begin{align*}
  7 \cdot 2x^2 + 7 \cdot 5x &= 7 \cdot 11 (\mod(13)) \\
  x^2 + 9x &= 12 (\mod(13))
.\end{align*}

Rewriting $9$ as $2 \cdot 11 = 22 \equiv 9$. This gives us
\[%
  x^2 + (2 \cdot 11)x = 12 (\mod(13))
.\]%

Adding the $a^2$ term to both sides of the equation gives us
\begin{align*}
  x^2 + 2 \cdot 11 \cdot x + 11^2 &= 12 + 11^2 (\mod(13)) \\
  (x + 11)^2 &= 3 (\mod(13))
.\end{align*}

\subsection{Squares in $\Z_{13}$}
\label{sub_sec:squares_in_13}

\begin{tabular}{|c|c|c|c|c|c|c|c|c|c|c|c|c|c|}
  $m$ & $0$ & $1$ & $2$ & $3$ & $4$ & $5$ & $6$ & $7$ & $8$ & $9$ & $10$ & $11$ & $12$ \\
  \hline
  $m^2$ & $0$ & $1$ & $4$ & $9$ & $3$ & $12$ & $10$ & $10$ & $12$ & $3$ & $9$ & $4$ & $1$ \\
\end{tabular}

Taking the square root in $\Z_{13}$ is completely different from taking the
square root in $\R$. In $\R$, $x^2 = 9$ gives us $x = \pm 3$. In $\Z_{13}$,
$x^2 = 9$ gives us $x = 3, 10$.

Using the table to solve the equation $(x + 11)^2 = 3$ gives us
\[%
  x + 11 = 4 \oor x + 11 = 9
.\]%
Adding $11$'s additive inverse to both sides of the equation gives us
\[%
  x = 6 \oor 11
.\]%

Checking the solutions gives us
\begin{align*}
  2 \cdot 6^2 + 5 \cdot 6 &\ce 11 (\mod(13)) \\
  2 \cdot 11^2 + 5 \cdot 11 &\ce 11 (\mod(13))
.\end{align*}

% subsection squares_in_13 (end)

% section solving_quadratic_equations (end)

\newpage
