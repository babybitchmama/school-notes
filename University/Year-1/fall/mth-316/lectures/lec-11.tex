\lecture{11}{Oct 23 2024 Wed (13:03:26)}{Properties of Infinite Series}

% Just as we had an algebraic limit theorem for sequences, we have a similar theorem for series.
% \begin{theorem}[Algebraic Limit Theorem for Series]
%   If $\sum_{k=1}^\infty a_k = A$ and $\sum_{k=1}^\infty b_k = B$, then
%   \begin{enumerate}
%     \item $\sum_{k=1}^\infty ca_k = cA$ for all $c \in \R$.

%     \item $\sum_{k=1}^\infty (a_k + b_k) = A + B$.
%   \end{enumerate}
% \end{theorem}

% \begin{proof}\leavevmode
%   \begin{enumerate}
%     \item Let $S_n = \sum_{k=1}^n a_k$. Then, $\lim_{m \to \infty} S_m = A$. Thus, by the Algebraic Limit Theorem for Sequences, $c\lim_{m \to \infty} S_m = \lim_{m \to \infty} (cS_m) = cA$. Thus, $\sum_{k=1}^\infty ca_k = cA$.

%     \item Let $S_n = \sum_{k=1}^n a_k$ and $T_n = \sum_{k=1}^n b_k$. Then, $\lim_{m \to \infty} S_m = A$ and $\lim_{m \to \infty} T_m = B$. Thus, by the Algebraic Limit Theorem for Sequences, $\lim_{m \to \infty} (S_m + T_m) = A + B$. Thus, $\sum_{k=1}^\infty (a_k + b_k) = A + B$. \qedhere
%   \end{enumerate}
% \end{proof}

% Just as we had the Cauchy criterion for sequences, we have a similar criterion for series.
% \begin{theorem}[Cauchy Criterion for Series]
%   The series $\sum_{k=1}^\infty a_k$ converges if and only if, given $\epsilon > 0$, there exists an $N \in \N$ such that whenever $n > m > N$, it follows that
%   \[%
%     \lvert a_{m+1} + a_{m+2} + \cdots + a_n \rvert < \epsilon
%   .\]%
% \end{theorem}

% \begin{proof}
%   Notice that
%   \[%
%     \left\lvert \sum_{k=1}^n a_k - \sum_{k=1}^m a_k \right\rvert = \lvert S_n - S_m \rvert = \lvert a_{m+1} + a_{m+2} + \cdots + a_n \rvert
%   .\]%
%   Thus, the series converges if and only if the sequence of partial sums is a Cauchy sequence.
% \end{proof}

% This is a very powerful criterion. We use this theorem to prove all of the basic tests in the future.

% \begin{definition}[Absolute Convergence]
%   A series $\sum_{k=1}^\infty a_k$ is said to \emph{converge absolutely} if the series $\sum_{k=1}^\infty \lvert a_k \rvert$ converges.
% \end{definition}

% \begin{definition}[Conditional Convergence]
%   A series $\sum_{k=1}^\infty a_k$ is said to \emph{converge conditionally} if the series $\sum_{k=1}^\infty a_k$ converges, but the series $\sum_{k=1}^\infty \lvert a_k \rvert$ diverges.
% \end{definition}
