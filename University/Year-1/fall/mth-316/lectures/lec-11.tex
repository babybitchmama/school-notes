\lecture{11}{Oct 23 2024 Wed (13:03:26)}{Introduction to Series}

\begin{definition}[Infinite Series and Partial Sums]
  Let $b_n$ be a sequence. An \emph{infinite series} is a formal expression of the form
  \[%
    \sum_{n=1}^\infty b_n = b_1 + b_2 + b_3 + \cdots
  .\]%
  The $n$-th \emph{partial sum} of the series is
  \[%
    S_N = \sum_{n=1}^{N} b_n = b_1 + b_2 + \cdots + b_N
  .\]%
\end{definition}

If $(b_n)$ converges to $L$, then we say $\sum_{n=1}^\infty b_n$ converges. The following are equivalent
\[%
  \lim_{n \to \infty} S_n = L = \lim_{n \to \infty} \sum_{k=1}^n b_k
.\]%

\begin{theorem}
  If $(\forall n \in \N)[a_n \ge 0]$, then $\sum_{n=1}^\infty a_n$ converges if and only if the sequence of partial sums $(S_n)$ is bounded.
\end{theorem}

\begin{proof}
  If $a_n \ge 0$, then $S_{n+1} = S_n + a_n \ge S_n$. By the Monotone Convergence Theorem, $(S_n)$ converges if and only if it is bounded.
\end{proof}

\begin{example}
  Given the sequence $a_n = \frac{1}{n^2}$, we can re-write this as
  \[%
    \sum_{n=1}^N \frac{1}{n^2} = 1 + \sum_{n=2}^N \frac{1}{n^2}
  .\]%
  This gives us
  \begin{align*}
    n^2 &\ge n(n - 1) \\
        &< 1 + \sum_{n=2}^N \frac{1}{n(n - 1)} = 1 + \sum_{n=2}^N \left(\frac{1}{n - 1} - \frac{1}{n}\right) \\
        &= 1 + \left(1 - \frac{1}{N}\right) = 2 - \frac{1}{N} < 2
  .\end{align*}
  So, $n^2 \ge 2$ for all $n \in \N$. Thus, $a_n = \frac{1}{n^2}$ is bounded below by $0$ and above by $2$. Therefore, $\sum_{n=1}^\infty \frac{1}{n^2}$ converges.
\end{example}

\subsection{Types of Series}

\begin{example}[Geometric Series]
  Let $\lvert \rho \rvert < 1$. Then
  \[%
    \sum_{n=0}^\infty \rho^n = \frac{1}{1 - \rho}
  .\]%
  Consider the partial sum
  \[%
    S_N = \sum_{n=0}^N \rho^N = \frac{1 - \rho^{N+1}}{1 - \rho}
  .\]%
  Therefore, $\sum_{n=0}^\infty \rho^n$ converges to $\frac{1}{1 - \rho}$.
\end{example}

\begin{example}[Harmonic Series]
  Let $a_n = \frac{1}{n}$, then $a_n \to 0$ but $\sum_{n=1}^\infty
  \frac{1}{n} \to \infty$. This can be shown by
  \[%
    S_{2^n} - S_{2^{n-1}} = \frac{1}{S^{n-1} + 1} + \cdots + \frac{1}{S^n} \ge \frac{2^{n-1}}{2^n} = \frac{1}{2}
  .\]%
  Therefore, $S_{2^n} \ge S_1 + \frac{n}{2}$, meaning that the harmonic series diverges.
\end{example}

\begin{example}[Alternating Series]
  Let $a_n = \frac{(-1)^n}{n}$, then $a_n \to 0$. But, the series $\sum_{n=1}^\infty \frac{(-1)^n}{n} = -\ln(2)$ by the Alternating Series Test (see section \ref{sub_sec:conversion_tests}).
\end{example}

\begin{example}[P-Series]
  Let $a_n = \frac{1}{n^\rho}$, then $a_n \to 0$. The series $\sum_{n=1}^\infty \frac{1}{n^\rho}$ converges if $\rho > 1$ and diverges if $\rho \le 1$.
\end{example}

\subsection{Infinite Products}

\begin{note}
  This isn't part of the course, but it's interesting to know.
\end{note}

Infinite products are an extension of finite products, where an infinite sequence of terms is multiplied together. They are often used in advanced mathematics to study sequences, series, and special functions. An infinite product takes the form
\[%
  \prod_{n=1}^\infty (1 + a_n) = (1 + a_1)(1 + a_2)(1 + a_3) \cdots
.\]%
Similar to infinite series, the convergence of an infinite product depends on the behavior of the terms $a_n$.

\subsubsection{Convergence of Infinite Products}

An infinite product $\prod_{n=1}^\infty (1 + a_n)$ converges if the partial product
\[%
  P_N = \prod_{n=1}^{N} (1 + a_n)
.\]%
converges to a finite, nonzero limit as $N \to \infty$. For convergence, a necessary condition is that $a_n \to 0$ as $n \to \infty$, since large terms would make the product diverge.

\begin{remark}
  If $\prod_{n=1}^\infty (1 + a_n)$ converges, then $\sum_{n=1}^\infty \ln(1 + a_n)$ must also converge, provided $\ln(1 + a_n)$ is well-defined. Thus, infinite products can often be analyzed by studying the associated series $\sum_{n=0}^\infty \ln(1 + a_n)$.
\end{remark}

\subsubsection{Relation to Infinite Series}

Infinite products are closely related to infinite series via the logarithm function. For small $a_n$, the approximation $\ln(1 + a_n) \approx a_n$ can be used to study the convergence of the series
\[%
  \ln\left(\prod_{n=1}^\infty (1 + a_n) \right) = \sum_{n=1}^\infty \ln(1 + a_n)
.\]%
This connection allows us to convert between infinite products and infinite series in many cases.

\subsubsection{Examples}

\begin{example}[Geometric Infinite Product]
  Consider $a_n = \rho^n$ with $\lvert \rho \rvert < 1$. Then the infinite product is
  \[%
    \prod_{n=1}^\infty (1 + \rho^n)
  .\]%
  By taking the logarithm, this can be related to the sum $\sum_{n=0}^\infty \ln(1 + \rho^n)$, and convergence is determined by the geometric series properties of $\rho^n$.
\end{example}

\begin{example}[Euler's Product Formula for Primes]
  A famous infinite product representation of the Riemann zeta function is
  \[%
    \zeta(s) = \prod_{p~\textrm{prime}} \frac{1}{1 - p^{-s}}
  ,\]%
  for $s > 1$. This connects infinite products to the distribution of prime numbers and is a cornerstone in analytic number theory.
\end{example}

Infinite products often appear in advanced topics such as special functions, infinite series transformations, and number theory. While not covered in depth here, they are worth exploring further in future courses.
