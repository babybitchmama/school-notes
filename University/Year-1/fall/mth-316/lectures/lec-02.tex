\lecture{2}{Oct 2 2024 Wed (13:02:51)}{Intro to Basic Proofs}

\subsection{Basic Logic for Proofs}

Mathematics relies on precise logical reasoning. A \emph{proposition} is a statement that is either \emph{true} (T) or \emph{false} (F), but not both.

\begin{definition}[Logical Connectives]
  Let $P$ and $Q$ be propositions. We define:
  \begin{enumerate}
    \item $\neg P$ (\emph{not} $P$): True if $P$ is false, false if $P$ is true.
    \item $P \land Q$ (\emph{$P$ and $Q$}): True if both $P$ and $Q$ are true.
    \item $P \lor Q$ (\emph{$P$ or $Q$}, inclusive): True if at least one of $P$, $Q$ is true.
    \item $P \implies Q$ (\emph{$P$ implies $Q$}): False only when $P$ is true and $Q$ is false.
    \item $P \iff Q$ (\emph{$P$ if and only if $Q$}): True if $P$ and $Q$ have the same truth value.
  \end{enumerate}
\end{definition}

\begin{note}
  The implication $P \implies Q$ is equivalent to $\neg P \lor Q$.
\end{note}

\begin{definition}[Contrapositive and Converse]
  The \emph{contrapositive} of $P \implies Q$ is $\neg Q \implies \neg P$ (logically equivalent to the original).
  The \emph{converse} of $P \implies Q$ is $Q \implies P$ (not generally equivalent).
\end{definition}

\begin{example}
  Let $P$: ``$n$ is divisible by $4$'' and $Q$: ``$n$ is even.''
  Then $P \implies Q$ is true for all integers $n$, and the contrapositive is: ``If $n$ is not even, then $n$ is not divisible by $4$.''
  The converse, ``If $n$ is even, then $n$ is divisible by $4$,'' is false (counterexample: $n=2$).
\end{example}

\subsubsection*{Common Logical Equivalences}

For propositions $P$, $Q$, $R$:
\begin{alignat*}{3}
    \text{(Double negation)} &\quad \neg(\neg P) &&\equiv P, \\
    \text{(De Morgan's laws)} &\quad \neg(P \land Q) &&\equiv (\neg P) \lor (\neg Q), \\
                              &\quad \neg(P \lor Q) &&\equiv (\neg P) \land (\neg Q), \\
    \text{(Implication rewrite)} &\quad P \implies Q &&\equiv \neg P \lor Q, \\
    \text{(Contrapositive)} &\quad P \implies Q &&\equiv (\neg Q) \implies (\neg P), \\
    \text{(Distributive)} &\quad P \land (Q \lor R) &&\equiv (P \land Q) \lor (P \land R)
.\end{alignat*}

\subsubsection*{Negating Quantified Statements}

For predicates $P(x)$:
\begin{align*}
  \neg (\forall x\ P(x)) &\equiv \exists x \ (\neg P(x)), \\
  \neg (\exists x\ P(x)) &\equiv \forall x \ (\neg P(x)).
\end{align*}

\begin{example}
  Negate: ``For every $x \in \R$, $x^2 \ge 0$.''
  Negation: ``There exists $x \in \R$ such that $x^2 < 0$.''
  The negated statement is false, so the original statement is true.
\end{example}

\subsubsection*{Proof Strategy: Given $\to$ Goal}
When proving a statement:
\begin{enumerate}
  \item Identify the \emph{givens} — hypotheses you can use freely.
  \item Identify the \emph{goal} — what must be shown.
  \item Work \emph{forward} from the givens using definitions, algebra, and known theorems.
  \item Work \emph{backward} from the goal: ask ``What would guarantee this is true?''
  If the two chains meet, you have your proof.
\end{enumerate}

\begin{note}
  This ``two-way'' thinking is especially useful for direct proofs and for deciding when to switch to a contrapositive or contradiction argument.
\end{note}

\subsection{Fundamental Proof Techniques}

In mathematical reasoning, several standard proof strategies are used repeatedly. Mastery of these approaches will greatly enhance problem-solving ability.

\subsubsection{Direct Proof}

A \emph{direct proof} establishes the truth of an implication $P \implies Q$ by assuming $P$ and logically deducing $Q$.

Generally, you assume $P$ is true, then use definitions, axioms, and previously established results to show that $Q$ must also be true.

\begin{question}
  Prove if $n$ is even, then $n^2$ is even.
\end{question}

\begin{proof}
  The assumption is that $n$ is even, meaning there exists some integer $k$ such that $n = 2k$. Then, $n^2 = (2k)^2 = 4k^2 = 2(2k^2) = 2m$, where $m = 2k^2$, making $n^2$ even.
\end{proof}

\begin{question}
  Prove that if $a$ and $b$ are rational numbers, then $a + b$ is also a rational number.
\end{question}

\begin{solution}
  Write $a = \frac{p}{q}$ and $b = \frac{r}{s}$ with integers $p,q,r,s$, $q,s \neq 0$. Then
  \[%
    a + b = \frac{ps + rq}{qs}
  ,\]%
  is a ratio of integers, hence rational.
\end{solution}

\subsubsection{Proof by Contrapositive}

To prove $P \implies Q$, prove instead $\neg Q \implies \neg P$.

\begin{question}
  Prove if $n^2$ is odd, then $n$ is odd.
\end{question}

\begin{proof}
  If $n$ is even, then $n^2$ is even. This is immediate from the direct proof above.
\end{proof}

\begin{question}
  Prove if $\sfrac{n}{m}$ is irrational, then $n$ is irrational or $m$ is irrational.
\end{question}

\begin{proof}
  If $n$ and $m$ are rational, then $\sfrac{n}{m}$ is rational. Clear from closure of rationals under division.
\end{proof}

\subsubsection{Proof by Contradiction}

Assume $P$ is true and $Q$ is false, then derive a contradiction.

\paragraph{How to do it:}
\begin{enumerate}
  \item Assume the negation of the conclusion.
  \item Derive an impossibility (e.g., $0 = 1$, $x$ both rational and irrational).
\end{enumerate}

\paragraph{Tips:}
\begin{itemize}
  \item Useful when proving existence or impossibility.
  \item Often used with number theory and irrationality proofs.
\end{itemize}

\begin{question}
  Prove $\sqrt{2}$ is irrational.
\end{question}

\begin{proof}
  Assume $\sqrt{2} = \sfrac{p}{q}$ in lowest terms. Then $2q^2 = p^2$ implies $p$ even. Let $p=2k$, then $q$ is also even.
\end{proof}

\begin{question}
  Prove there are infinitely many primes.
\end{question}

\begin{proof}
  Assume finitely many primes $p_1, \cdots, p_n$. Consider $P = p_1 \cdots p_n + 1$. Then $P$ has a prime factor not in the list.
\end{proof}

\subsubsection{Proof by Mathematical Induction}

A method for statements involving $n \in \N$.

\begin{definition}[Mathematical induction]
  Let $P(n)$ be a statement involving $n \in \N$. If:
  \begin{enumerate}
    \item $P(1)$ is true (base case), and
    \item $P(n) \implies P(n+1)$ holds for all $n$ (inductive step),
  \end{enumerate}
  then $P(n)$ holds for all $n \in \N$.
\end{definition}

\begin{question}
  Prove $\sum_{i=1}^n i = \frac{n(n+1)}{2}$.
\end{question}

\begin{proof}
  It clearly holds for $n = 1$. Assume true for $n = k$. Then,
  \[%
    \sum_{i=1}^{k+1} i = \frac{k(k + 1)}{2} + (k + 1) = \frac{(k + 1)(k + 2)}{2}
  .\qedhere\]%
\end{proof}

\begin{question}
  Prove $2^n > n$ for $n \geq 1$.
\end{question}

\begin{proof}
  It holds for $n = 1$, since $2 > 1$. Assume that $2^k > k$, then $2^{k+1} = 2 \cdot 2^k > 2k \ge k + 1$, for $k \ge 1$.
\end{proof}

\begin{question}
  Let $x_1 = 1$ and $x_{n+1} = \frac{1}{2}x_n + 1$. Show that $(x_n)$ is increasing.
\end{question}

\begin{solution}
  Let $P(n)$ be the statement $x_n \le x_{n+1}$.

  \emph{Base case:} $x_1 = 1$ and $x_2 = \frac{1}{2}(1) + 1 = \frac{3}{2} > x_1$.

  \emph{Inductive step:} Assume $x_{n+1} \ge x_n$. Then:
  \[%
    x_{n+2} - x_{n+1} = \frac12(x_{n+1} + 1) - \frac12(x_n + 1) = \frac12(x_{n+1} - x_n) > 0
  .\]%
  Therefore $x_{n+2} > x_{n+1}$, completing the step. By induction, $(x_n)$ is increasing.
\end{solution}

\begin{question}
  Show that $(\forall n \in \N)[3 \mid (n^3 + 2n)]$.
\end{question}

\begin{solution}
  Let $P(n)$ be: $3 \mid (n^3 + 2n)$.

  \emph{Base case:} For $n=1$, $1^3 + 2(1) = 3$ is divisible by $3$.

  \emph{Inductive step:} Assume $n^3 + 2n$ is divisible by $3$. Then:
  \begin{align*}
    (n+1)^3 + 2(n+1)
      &= n^3 + 3n^2 + 3n + 1 + 2n + 2 \\
      &= (n^3 + 2n) + 3(n^2 + n + 1)
  .\end{align*}
  By hypothesis, $n^3 + 2n$ is divisible by $3$, and $3(n^2 + n + 1)$ is clearly divisible by $3$. Thus the sum is divisible by $3$.

  By induction, $3 \mid (n^3 + 2n)$ for all $n \in \N$.
\end{solution}
