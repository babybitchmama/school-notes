\lecture{2}{Oct 2 2024 Wed (13:02:51)}{Triangle Inequality}

\section{Functions}

\begin{definition}[Function]
  Let $A$ and $B$ be two sets. A \emph{function} $f : A \to B$ assigns an element from the \emph{domain} $A$ to the \emph{range} $B$. \end{definition}

\begin{example}
  Let $A$ be the set of chairs in a classroom and let $B$ be the set of all students in the room. Then, $f : A \to B$ assigns each chair to a student, like a seating chart.
\end{example}

\begin{example}
  Let $f(x) = x^2 + 1$, where $f : \R \to \R$. Then, the domain of $f$ is $\R$ and the range is the interval $[1, \infty)$.
\end{example}

\section{Triangle Inequality}

The absolute value function is so important, that it has it's own notation, $\abs{x}$ in place of the usual $f(x)$. It's defined as
\begin{equation}
  \abs{x} = \begin{cases}
    x & \text{if}~x \geq 0, \\
    -x & \text{if}~x < 0
  \end{cases}
\end{equation}

The absolute value function satisfies the following properties
\begin{enumerate}
  \item $\abs{ab} = \abs{a} \cdot \abs{b}$.

  \item $\abs{a + b} \le \abs{a} + \abs{b}$.
\end{enumerate}

The second property is known as the \emph{Triangle Inequality}, and it's mostly used with three numbers, $a$, $b$, and $c$. Rearranging the inequality, we get
\[%
  \abs{a - b} = \abs{(a - c) + (c - b)} \implies \abs{(a - c) + (c - b)} \le \abs{a - c} + \abs{c - b}
,\]%
giving us
\begin{equation}
  \abs{a - b} \le \abs{a - c} + \abs{c - b}
\end{equation}

\begin{note}
  The expression $\abs{a - b}$ is the distance between $a$ and $b$ on the real number line, which is also equivalent to $\abs{b - a}$.
\end{note}

Restating the triangle inequality in terms of distance, we get that the distance between $a$ and $b$ is less than or equal to the sum of the distances between $a$ and $c$ and $c$ and $b$.
