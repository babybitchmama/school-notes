\lecture{10}{Oct 23 2024 Wed (13:00:51)}{Intro to Infinite Series}

\section{Infinite Series}
\label{sec:infinite_series}

\begin{definition}[Infinite Series and Partial Sums]
  Let $b_n$ be a sequence. An \textit{infinite series} is a formal expression of
  the form
  \[%
    \sum_{n=1}^{\infty} b_n = b_1 + b_2 + b_3 + \cdots
  .\]%
  The $n$-th \textit{partial sum} of the series is
  \[%
    S_N = \sum_{n=1}^{N} b_n = b_1 + b_2 + \cdots + b_N
  .\]%
\end{definition}

If $(b_n)$ converges to $L$, then we say $\sum_{n=1}^{\infty} b_n$ converges.
The following are equivalent
\[%
  \lim_{n \to \infty} S_n = L = \lim_{n \to \infty} \sum_{k=1}^n b_k
.\]%

\begin{theorem}
  If $(\forall n \in \N)[a_n \ge 0]$, then $\sum_{n=1}^{\infty} a_n$ converges
  if and only if the sequence of partial sums $(S_n)$ is bounded.
\end{theorem}

\begin{proof}
  If $a_n \ge 0$, then $S_{n+1} = S_n + a_n \ge S_n$. By the Monotone
  Convergence Theorem, $(S_n)$ converges if and only if it is bounded.
\end{proof}

\begin{example}
  Given the sequence $a_n = \frac{1}{n^2}$, we can re-write this as
  \[%
    \sum_{n=1}^N \frac{1}{n^2} = 1 + \sum_{n=2}^N \frac{1}{n^2}
  .\]%
  This gives us
  \begin{align*}
    n^2 &\ge n(n - 1) \\
        &< 1 + \sum_{n=2}^N \frac{1}{n(n - 1)} = 1 + \sum_{n=2}^N \left(\frac{1}{n - 1} - \frac{1}{n}\right) \\
        &= 1 + \left(1 - \frac{1}{N}\right) = 2 - \frac{1}{N} < 2
  .\end{align*}
  So, $n^2 \ge 2$ for all $n \in \N$. Thus, $a_n = \frac{1}{n^2}$ is bounded
  below by $0$ and above by $2$. Therefore, $\sum_{n=1}^{\infty} \frac{1}{n^2}$
  converges.
\end{example}

\begin{example}[Harmonic Series]
  Let $a_n = \frac{1}{n}$, then $a_n \to 0$ but $\sum_{n=1}^{\infty}
  \frac{1}{n} \to \infty$. This can be shown by
  \[%
    S_{2^n} - S_{2^{n-1}} = \frac{1}{S^{n-1} + 1} + \cdots + \frac{1}{S^n} \ge \frac{2^{n-1}}{2^n} = \frac{1}{2}
  .\]%
  Therefore, $S_{2^n} \ge S_1 + \frac{n}{2}$, meaning that the harmonic series
  diverges.
\end{example}

\begin{example}[Geometric Series]
  Let $\lvert \rho \rvert < 1$. Then
  \[%
    \sum_{n=0}^{\infty} \rho^n = \frac{1}{1 - \rho}
  .\]%
  Consider the partial sum
  \[%
    S_N = \sum_{n=0}^N \rho^N = \frac{1 - \rho^{N+1}}{1 - \rho}
  .\]%
  Therefore, $\sum_{n=0}^{\infty} \rho^n$ converges to $\frac{1}{1 - \rho}$.
\end{example}

\subsection{Cauchy Condensation Test}
\label{sub_sec:cauchy_condensation_test}

\begin{theorem}[Cauchy Condensation Test]
  Suppose $(b_n)$ is decreasing and satisfies $b_n \ge 0$ for all $n \in \N$.
  Then, the series $\sum_{n=1}^{\infty} b_n$ converges if and only if the series
  \[%
    \sum_{n=0}^{\infty} 2^nb_{2^n} = b_1 + 2b_2 + 4b_4 + 8b_8 + 16b_{16} + \cdots
  ,\]%
  converges.
\end{theorem}

\begin{proof}
  First, assume that $\sum_{n=0}^{\infty} 2^nb_{2^n}$ converges. Then, the
  partial sum
  \[%
    t_k = b_1 + 2b_2 + 4b_4 + \cdots + 2^kb_{2^k}
  ,\]%
  is bounded, by some number $M$. Because $b_n \ge 0$, we know that the partial
  sums are increasing. Let $S_m = b_1 + b_2 + b_3 + \cdots + b_m$. Fix $m$ and
  let $k$ be large enough to ensure $m \le 2_{k+1} - 1$. Then, $S_m \le S_{2^k +
  1} - 1$ and
  \begin{align*}
    S_{2^{k+1}} - 1 &= b_1 + (b_2 + b_3) + (b_4 + \cdots + b_7) + \cdots + (b_{2^k} + \cdots + b_{2^{k+1} - 1}) \\
                    &\le b_1 + 2b_2 + 4b_4 + \cdots + 2^kb_{2^k} \\
                    &= t_k
  .\end{align*}
  Therefore, $S_m \le t_k \le M$ for all $m \in \N$. This implies that $(S_m)$
  is bounded and therefore converges.
\end{proof}

That was essentially a proof that the sum $\sum_{n=1}^{\infty} \frac{1}{n}$
diverges and that $\sum_{n=1}^{\infty} \frac{1}{n^{\rho}}$ converges for $\rho >
1$.

\begin{corollary}
  The series
  \[%
    \sum_{n=1}^{\infty} \frac{1}{n^{\rho}}
  .\]%
  converges if and only if $\rho > 1$.
\end{corollary}

\begin{proof}
  Let $b_n = \frac{1}{n^{\rho}}$. By the Cauchy Condensation Test, the series
  $\sum_{n=1}^{\infty} b_n$ converges if and only if the series
  \[%
    \sum_{n=0}^{\infty} 2^n \cdot \frac{1}{(2^n)^{\rho}} = \sum_{n=0}^{\infty} \frac{1}{2^{n(\rho-1)}}
  .\]%
  converges. The latter is a geometric series with common ratio $r =
  \frac{1}{2^{\rho-1}}$. A geometric series converges if and only if $\lvert r
  \rvert < 1$. Thus, $\lvert r \rvert < 1$ implies
  \[%
    \frac{1}{2^{\rho-1}} < 1,
  \]%
  which holds if and only if $\rho - 1 > 0$, or equivalently, $\rho > 1$.
  Therefore, $\sum_{n=1}^{\infty} \frac{1}{n^{\rho}}$ converges if and only if
  $\rho > 1$.
\end{proof}

% subsection cauchy_condensation_test (end)

% section infinite_series (end)

\section{Infinite Products}
\label{sec:infinite_products}

\begin{note}
  This isn't part of the course, but it's interesting to know.
\end{note}

Infinite products are an extension of finite products, where an infinite
sequence of terms is multiplied together. They are often used in advanced
mathematics to study sequences, series, and special functions. An infinite
product takes the form
\[%
  \prod_{n=1}^{\infty} (1 + a_n) = (1 + a_1)(1 + a_2)(1 + a_3) \cdots
.\]%
Similar to infinite series, the convergence of an infinite product depends on
the behavior of the terms $a_n$.

\subsection{Convergence of Infinite Products}
\label{sub_sec:convergence_of_infinite_products}

An infinite product $\prod_{n=1}^{\infty} (1 + a_n)$ converges if the partial
product
\[%
  P_N = \prod_{n=1}^{N} (1 + a_n)
.\]%
converges to a finite, nonzero limit as $N \to \infty$. For convergence, a
necessary condition is that $a_n \to 0$ as $n \to \infty$, since large terms
would make the product diverge.

\begin{remark}
  If $\prod_{n=1}^{\infty} (1 + a_n)$ converges, then $\sum_{n=1}^{\infty} \ln(1
  + a_n)$ must also converge, provided $\ln(1 + a_n)$ is well-defined. Thus,
  infinite products can often be analyzed by studying the associated series
  $\sum_{n=0}^{\infty} \ln(1 + a_n)$.
\end{remark}

% subsection convergence_of_infinite_products (end)

\subsection{Relation to Infinite Series}
\label{sub_sec:relation_to_infinite_series}

Infinite products are closely related to infinite series via the logarithm
function. For small $a_n$, the approximation $\ln(1 + a_n) \approx a_n$ can be
used to study the convergence of the series
\[%
  \ln\left(\prod_{n=1}^{\infty} (1 + a_n) \right) = \sum_{n=1}^{\infty} \ln(1 + a_n)
.\]%
This connection allows us to convert between infinite products and infinite
series in many cases.

% subsection relation_to_infinite_series (end)

\subsection{Examples}
\label{sub_sec:examples}

\begin{example}[Geometric Infinite Product]
  Consider $a_n = \rho^n$ with $\lvert \rho \rvert < 1$. Then the infinite
  product is
  \[%
    \prod_{n=1}^{\infty} (1 + \rho^n)
  .\]%
  By taking the logarithm, this can be related to the sum $\sum_{n=0}^{\infty}
  \ln(1 + \rho^n)$, and convergence is determined by the geometric series
  properties of $\rho^n$.
\end{example}

\begin{example}[Euler's Product Formula for Primes]
  A famous infinite product representation of the Riemann zeta function is
  \[%
    \zeta(s) = \prod_{p~\textrm{prime}} \frac{1}{1 - p^{-s}}
  ,\]%
  for $s > 1$. This connects infinite products to the distribution of prime
  numbers and is a cornerstone in analytic number theory.
\end{example}

Infinite products often appear in advanced topics such as special functions,
infinite series transformations, and number theory. While not covered in depth
here, they are worth exploring further in future courses.

% subsection examples (end)

% section infinite_products (end)

\newpage
