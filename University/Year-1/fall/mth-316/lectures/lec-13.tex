\lecture{13}{Oct 30 2024 Wed (13:02:06)}{Bolzano-Weierstrass Theorem}

A very powerful theorem is the Bolzano-Weierstrass Theorem.

\begin{theorem}[Bolzano-Weierstrass Theorem]
  Every bounded sequence has a convergent subsequence.
\end{theorem}

\begin{proof}
  Let $(a_n)$ be a bounded sequence so that there exists $M > 0$ satisfying that $(\forall n \in \N)[\lvert a_n \rvert < M]$. Bisect the closed interval $[-M, M]$ into two closed intervals $[-M, 0]$ and $[0, M]$. Select the interval that contains infinitely many terms of the sequence $(a_n)$ and call it $I_1$. Then, let $a_{n_1}$ be some term in the sequence $(a_n)$ satisfying $a_{n_1} \in I_1$. Bisect $I_1$ into two closed intervals and select the one that contains infinitely many terms of the sequence $(a_n)$ and call it $I_2$. Then, let $a_{n_2}$ be some term in the sequence $(a_n)$ satisfying $n_2 > n_1$ and $a_{n_2} \in I_2$. Continue this process indefinitely to construct a subsequence $(a_{n_k})$. Then, we get
  \[%
    I_1 \supseteq I_2 \supseteq I_3 \supseteq \cdots
  .\]%
  By the Nested Interval Property, there exists a point $a$ that is in all the intervals $I_k$.

  Let $\epsilon > 0$. The length of $I_k$ is $M\left(\frac{1}{2}\right)^{k-1}$ which converges to zero. Choose $N$ so that $k \ge N$ implies that the length of $I_k$ is less than $\epsilon$. Because $a, a_{n_k} \in I_k$, it follows that $\lvert a_{n_k} - a \rvert < \epsilon$.
\end{proof}
