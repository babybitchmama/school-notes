\lecture{13}{Nov 4 2024 Mon (13:02:06)}{Properties of Infinite Series}

\section{Properties of Infinite Series}

Just as we had an algebraic limit theorem for sequences, we have a similar theorem for series.
\begin{theorem}[Algebraic Limit Theorem for Series]
  If $\sum_{k=1}^\infty a_k = A$ and $\sum_{k=1}^\infty b_k = B$, then
  \begin{enumerate}
    \item $\sum_{k=1}^\infty ca_k = cA$ for all $c \in \R$.

    \item $\sum_{k=1}^\infty (a_k + b_k) = A + B$.
  \end{enumerate}
\end{theorem}

\begin{proof}\leavevmode
  \begin{enumerate}
    \item Let $S_n = \sum_{k=1}^n a_k$. Then, $\lim_{m \to \infty} S_m = A$. Thus, by the Algebraic Limit Theorem for Sequences, $c\lim_{m \to \infty} S_m = \lim_{m \to \infty} (cS_m) = cA$. Thus, $\sum_{k=1}^\infty ca_k = cA$.

    \item Let $S_n = \sum_{k=1}^n a_k$ and $T_n = \sum_{k=1}^n b_k$. Then, $\lim_{m \to \infty} S_m = A$ and $\lim_{m \to \infty} T_m = B$. Thus, by the Algebraic Limit Theorem for Sequences, $\lim_{m \to \infty} (S_m + T_m) = A + B$. Thus, $\sum_{k=1}^\infty (a_k + b_k) = A + B$. \qedhere
  \end{enumerate}
\end{proof}

Just as we had the Cauchy criterion for sequences, we have a similar criterion for series.
\begin{theorem}[Cauchy Criterion for Series]
  The series $\sum_{k=1}^\infty a_k$ converges if and only if, given $\epsilon > 0$, there exists an $N \in \N$ such that whenever $n > m > N$, it follows that
  \[%
    \lvert a_{m+1} + a_{m+2} + \cdots + a_n \rvert < \epsilon
  .\]%
\end{theorem}

\begin{proof}
  Notice that
  \[%
    \left\lvert \sum_{k=1}^n a_k - \sum_{k=1}^m a_k \right\rvert = \lvert S_n - S_m \rvert = \lvert a_{m+1} + a_{m+2} + \cdots + a_n \rvert
  .\]%
  Thus, the series converges if and only if the sequence of partial sums is a Cauchy sequence.
\end{proof}

This is a very powerful criterion. We use this theorem to prove all of the basic tests for series in section \ref{sub_sec:conversion_tests}.

\begin{definition}[Absolute Convergence]
  A series $\sum_{k=1}^\infty a_k$ is said to \emph{converge absolutely} if the series $\sum_{k=1}^\infty \lvert a_k \rvert$ converges.
\end{definition}

\begin{definition}[Conditional Convergence]
  A series $\sum_{k=1}^\infty a_k$ is said to \emph{converge conditionally} if the series $\sum_{k=1}^\infty a_k$ converges, but the series $\sum_{k=1}^\infty \lvert a_k \rvert$ diverges.
\end{definition}

\subsection{Conversion Tests}

Now, we can start proving tests that have some substance.

\begin{lemma}
  If the series $\sum_{k=1}^\infty a_k$ converges, then $a_k \to 0$.
\end{lemma}

\begin{proof}
  Let $S_n = \sum_{k=1}^n a_k$. Then, $a_n = S_n - S_{n-1}$. Since the series converges, the sequence of partial sums is Cauchy. Thus, given $\epsilon > 0$, there exists an $N \in \N$ such that whenever $n > m > N$, it follows that
  \[%
    \lvert S_n - S_m \rvert = \lvert a_{m+1} + a_{m+2} + \cdots + a_n \rvert < \epsilon
  .\]%
  In particular, if we take $m = n - 1$, we have
  \[%
    \lvert a_n \rvert = \lvert S_n - S_{n-1} \rvert < \epsilon
  .\]%
  Thus, $a_n \to 0$.
\end{proof}

\begin{note}
  The converse isn't true. Take, for example, the harmonic series $\sum_{k=1}^\infty \frac{1}{k}$. This series diverges, but $\frac{1}{k} \to 0$.
\end{note}

\begin{lemma}[Absolute Convergence Test]
  If $\sum_{k=1}^\infty \lvert a_k \rvert$ converges, then $\sum_{k=1}^\infty a_k$ converges.
\end{lemma}

\begin{proof}
  Assume $\sum_{k=1}^\infty \lvert a_k \rvert$ converges. Let $\epsilon > 0$. Then, there exists an $N \in \N$ such that for every $n > m > N$, we have
  \[%
    \lvert a_{m+1} \rvert + \lvert a_{m+2} \rvert + \cdots + \lvert a_n \rvert < \epsilon
  .\]%
  Notice that
  \[%
    \lvert a_{m+1} + a_{m+2} + \cdots + a_n \rvert \leq \lvert a_{m+1} \rvert + \lvert a_{m+2} \rvert + \cdots + \lvert a_n \rvert < \epsilon
  .\]%
  Thus, by the Cauchy Criterion for Series, $\sum_{k=1}^\infty a_k$ converges.
\end{proof}

\begin{lemma}[Comparison Test]
  Assume $(a_k)$ and $(b_k)$ are sequences satisfying $(\forall k \in \N)[0 \le a_k \le b_k]$
  \begin{enumerate}
    \item If $\sum_{k=1}^\infty b_k$ converges, then $\sum_{k=1}^\infty a_k$ converges.

    \item If $\sum_{k=1}^\infty a_k$ diverges, then $\sum_{k=1}^\infty b_k$ diverges.
  \end{enumerate}
\end{lemma}

\begin{proof}\leavevmode
  Both statements follow directly from the Cauchy Criterion for Series.
  \[%
    \lvert a_{m+1} + a_{m+2} + \cdots + a_n \rvert \le \lvert b_{m+1} + b_{m+2} + \cdots + b_n \rvert
  .\]%
  If the right-hand side is less than $\epsilon$, then the left-hand side is less than $\epsilon$. Thus, the series converges. The second statement follows from the contrapositive of the first statement.
\end{proof}

\begin{lemma}[Limit Comparison Test]
  Assume $(a_k)$ and $(b_k)$ are sequences satisfying $(\forall k \in \N)[0 < a_k \le b_k]$. If
  \[%
    \lim_{k \to \infty} \frac{a_k}{b_k} = c
  ,\]%
  where $c$ is a positive real number, then $\sum_{k=1}^\infty a_k$ converges if and only if $\sum_{k=1}^\infty b_k$ converges.
\end{lemma}

\begin{proof}
  Let $c > 0$ such that $\lim_{k \to \infty} \frac{a_k}{b_k} = c$. This means for any $\epsilon > 0$, there exists an $N \in \N$ such that for all $k \geq N$, we have
  \[%
    c - \epsilon < \frac{a_k}{b_k} < c + \epsilon
  .\]%
  Multiplying through by $b_k$ (which is positive), we get
  \[%
    (c - \epsilon)b_k < a_k < (c + \epsilon)b_k
  .\]%
  Choose $\epsilon$ such that $0 < \epsilon < c$, so $c - \epsilon > 0$. For $k \geq N$, we now have
  \[%
    0 < (c - \epsilon) b_k \leq a_k \leq (c + \epsilon) b_k
  .\]%

  \begin{enumerate}
    \item Convergence of $\sum_{k=1}^\infty b_k$:

      If $\sum_{k=1}^\infty b_k$ converges, then $(c - \epsilon)\sum_{k=1}^\infty b_k$ converges because $c - \epsilon > 0$ is a constant multiple. By the Comparison Test applied to $(c - \epsilon)b_k \leq a_k$, it follows that $\sum_{k=1}^\infty a_k$ also converges.

    \item Divergence of $\sum_{k=1}^\infty b_k$:

      If $\sum_{k=1}^\infty b_k$ diverges, then $(c + \epsilon)\sum_{k=1}^\infty b_k$ also diverges because $c + \epsilon > 0$ is a constant multiple. By the Comparison Test applied to $a_k \geq (c - \epsilon)b_k$, it follows that $\sum_{k=1}^\infty a_k$ also diverges.
  \end{enumerate}

  Hence, $\sum_{k=1}^\infty a_k$ converges if and only if $\sum_{k=1}^\infty b_k$ converges.
\end{proof}

\begin{lemma}[Ratio Test]
  Let $(a_n)$ be a sequence. If
  \[%
    \lim_{n \to \infty} \left\lvert \frac{a_{n+1}}{a_n} \right\rvert = L
  ,\]%
  then
  \begin{enumerate}
    \item If $L < 1$, then $\sum_{n=1}^\infty a_n$ converges absolutely.

    \item If $L > 1$, then $\sum_{n=1}^\infty a_n$ diverges.

    \item If $L = 1$, then the test is inconclusive.
  \end{enumerate}
\end{lemma}

\begin{proof}\leavevmode
  Let $L = \lim_{n \to \infty} \left\lvert \frac{a_{n+1}}{a_n} \right\rvert$.
  \begin{enumerate}
    \item $L < 1$:

      Choose $\epsilon > 0$ such that $L + \epsilon < 1$. By the definition of the limit,
      \[%
        (\exists N \in \N)(\forall n \ge N)\left[\left\lvert \frac{a_{n+1}}{a_n} \right\rvert < L + \epsilon\right]
      .\]%
      This implies that for $n \geq N$
      \[%
        \lvert a_{n+1} \rvert < (L + \epsilon) \lvert a_n \rvert
      .\]%
      Iterating this inequality, we get
      \[%
        \lvert a_{n+1} \rvert < (L + \epsilon)^{n-N+1} \lvert a_N \rvert
      .\]%
      Since $L + \epsilon < 1$, the geometric series $\sum_{n=N}^\infty (L + \epsilon)^{n-N+1}$ converges. Therefore, by the Comparison Test, $\sum_{n=N}^\infty \lvert a_n \rvert$ converges, which implies that $\sum_{n=1}^\infty \lvert a_n \rvert$ converges. Thus, $\sum_{n=1}^\infty a_n$ converges absolutely.

    \item $L > 1$:

      Choose $\epsilon > 0$ such that $L - \epsilon > 1$. By the definition of the limit,
      \[%
        (\exists N \in \N)(\forall n \ge N)\left[\left\lvert \frac{a_{n+1}}{a_n} \right\rvert > L - \epsilon\right]
      .\]%
      This implies that for $n \geq N$
      \[%
        \lvert a_{n+1} \rvert > (L - \epsilon) \lvert a_n \rvert
      .\]%
      Iterating this inequality, we get
      \[%
        \lvert a_{n+1} \rvert > (L - \epsilon)^{n-N+1} \lvert a_N \rvert
      .\]%
      Since $L - \epsilon > 1$, the terms $\lvert a_n \rvert$ do not approach zero, and hence $\sum_{n=1}^\infty \lvert a_n \rvert$ diverges. Thus, $\sum_{n=1}^\infty a_n$ diverges.

    \item $L = 1$:

      The test provides no information about the convergence or divergence of the series. For example, the series $\sum_{n=1}^\infty \frac{1}{n}$ diverges, but $\sum_{n=1}^\infty \frac{1}{n^2}$ converges, and both satisfy $\lim_{n \to \infty} \left\lvert \frac{a_{n+1}}{a_n} \right\rvert = 1$. Therefore, the test is inconclusive in this case. \qedhere
  \end{enumerate}
\end{proof}

\begin{lemma}[Root Test]
  Let $(a_n)$ be a sequence. If
  \[%
    \limsup_{n \to \infty} \sqrt[n]{\lvert a_n \rvert} = L
  ,\]%
  then
  \begin{enumerate}
    \item If $L < 1$, then $\sum_{n=1}^\infty a_n$ converges absolutely.

    \item If $L > 1$, then $\sum_{n=1}^\infty a_n$ diverges.

    \item If $L = 1$, then the test is inconclusive.
  \end{enumerate}
\end{lemma}

\begin{proof}
  Let $L = \limsup_{n \to \infty} \sqrt[n]{\lvert a_n \rvert}$.

  \begin{enumerate}
    \item $L < 1$:

      Choose $\epsilon > 0$ such that $L + \epsilon < 1$. By the definition of the $\limsup$, there exists $N \in \mathbb{N}$ such that for all $n \geq N$
      \[%
        \sqrt[n]{\lvert a_n \rvert} < L + \epsilon
      .\]%
      Raising both sides to the $n$th power, we have
      \[%
        (\forall n \geq N)\left[\lvert a_n \rvert < (L + \epsilon)^n\right]
      .\]%
      Since $L + \epsilon < 1$, the series $\sum_{n=N}^\infty (L + \epsilon)^n$ converges because it is a geometric series with ratio less than $1$. By the Comparison Test, $\sum_{n=N}^\infty \lvert a_n \rvert$ converges, and therefore $\sum_{n=1}^\infty \lvert a_n \rvert$ converges. Thus, $\sum_{n=1}^\infty a_n$ converges absolutely.

    \item $L > 1$:

      By the definition of $\limsup$, there exists a subsequence $(a_{n_k})$ such that
      \[%
        \sqrt[n_k]{\lvert a_{n_k} \rvert} \to L~\textrm{as}~k \to \infty
      .\]%
      Since $L > 1$, for sufficiently large $k$, we have
      \[%
        \sqrt[n_k]{\lvert a_{n_k} \rvert} > 1
      ,\]%
      which implies
      \[%
        \lvert a_{n_k} \rvert > 1^n = 1
      .\]%
      Thus, $\lvert a_{n_k} \rvert$ does not approach $0$, so the series $\sum_{n=1}^\infty a_n$ diverges.

    \item $L = 1$:

      The test is inconclusive because series with $L = 1$ may either converge or diverge. For instance
      \[%
        \sum_{n=1}^\infty \frac{1}{n^2}~\textrm{converges, while}~\sum_{n=1}^\infty \frac{1}{n}~\textrm{diverges}
      .\]%
      Both of these series satisfy $\limsup_{n \to \infty} \sqrt[n]{\lvert a_n \rvert} = 1$.\qedhere
  \end{enumerate}
\end{proof}

\begin{lemma}[Alternating Series Test]
  Let $(a_n)$ be a sequence satisfying the following requirements
  \begin{enumerate}
    \item $a_1 \ge a_2 \ge a_3 \ge \cdots \ge a_n \ge a_{n+1} \ge \cdots$,

    \item $a_n \to 0$,
  \end{enumerate}
  Then, the alternating series $\sum_{n=1}^\infty (-1)^{n+1} a_n$ converges.
\end{lemma}

\begin{proof}
  Let $S_n = \sum_{k=1}^n (-1)^{k+1} a_k$. Then, $S_{2n} = a_1 - a_2 + a_3 - a_4 + \cdots + a_{2n-1} - a_{2n}$. Since $a_n \to 0$, we have
  \[%
    \lvert S_{2n} - S_{2n-2} \rvert = \lvert a_{2n-1} - a_{2n} \rvert \to 0
  .\]%
  Thus, $(S_{2n})$ is a Cauchy sequence. Similarly, $(S_{2n+1})$ is also a Cauchy sequence. Thus, both sequences converge. Since the series is the sum of these two sequences, the series converges.
\end{proof}

\begin{lemma}[Condensation Test]
  Suppose $(b_n)$ is decreasing and satisfies $b_n \ge 0$ for all $n \in \N$. Then, the series $\sum_{n=1}^\infty b_n$ converges if and only if the series
  \[%
    \sum_{n=0}^\infty 2^nb_{2^n} = b_1 + 2b_2 + 4b_4 + 8b_8 + 16b_{16} + \cdots
  ,\]%
  converges.
\end{lemma}

\begin{proof}
  First, assume that $\sum_{n=0}^\infty 2^nb_{2^n}$ converges. Then, the partial sum
  \[%
    T_k = b_1 + 2b_2 + 4b_4 + \cdots + 2^kb_{2^k}
  ,\]%
  is bounded, by some number $M$. Because $b_n \ge 0$, we know that the partial sums are increasing. Let $S_m = b_1 + b_2 + b_3 + \cdots + b_m$. Fix $m$ and let $k$ be large enough to ensure $m \le 2_{k+1} - 1$. Then, $S_m \le S_{2^k + 1} - 1$ and
  \begin{align*}
    S_{2^{k+1}} - 1 &= b_1 + (b_2 + b_3) + (b_4 + \cdots + b_7) + \cdots + (b_{2^k} + \cdots + b_{2^{k+1} - 1}) \\
                    &\le b_1 + 2b_2 + 4b_4 + \cdots + 2^kb_{2^k} \\
                    &= T_k
  .\end{align*}
  Therefore, $S_m \le T_k \le M$ for all $m \in \N$. This implies that $(S_m)$ is bounded and therefore converges.
\end{proof}

This is a more general statement of the statement that the sum $\sum_{n=1}^\infty \frac{1}{n}$ diverges and that $\sum_{n=1}^\infty
\frac{1}{n^\rho}$ converges for $\rho > 1$.

\begin{corollary}[P-Series Test]
  The series
  \[%
    \sum_{n=1}^\infty \frac{1}{n^\rho}
  .\]%
  converges if and only if $\rho > 1$.
\end{corollary}

\begin{proof}
  Let $b_n = \frac{1}{n^\rho}$. By the Cauchy Condensation Test, the series $\sum_{n=1}^\infty b_n$ converges if and only if the series
  \[%
    \sum_{n=0}^\infty 2^n \cdot \frac{1}{(2^n)^\rho} = \sum_{n=0}^\infty \frac{1}{2^{n(\rho-1)}}
  .\]%
  converges. The latter is a geometric series with common ratio $r = \frac{1}{2^{\rho-1}}$. A geometric series converges if and only if $\lvert r \rvert < 1$. Thus, $\lvert r \rvert < 1$ implies
  \[%
    \frac{1}{2^{\rho-1}} < 1
  ,\]%
  which holds if and only if $\rho - 1 > 0$, or equivalently, $\rho > 1$. Therefore, $\sum_{n=1}^\infty \frac{1}{n^\rho}$ converges if and only if $\rho > 1$.
\end{proof}

\begin{lemma}[Abel's Test]
  Let $a_1 \ge a_2 \ge a_3 \ge \cdots \ge 0$, and suppose that $a_n \to 0$. Let $z \in \C$ such that $\lvert z \rvert = 1$ and $z \ne 1$. Then, the series $\sum_{n=1}^\infty a_nz^n$ converges.
\end{lemma}

\begin{proof}
  Notice that
  \begin{align*}
    \sum_{n=M}^N a_n z^n &= \sum_{n=M}^N a_n \frac{z^{n+1} - z^n}{z - 1} \\
                         &= \frac{1}{z - 1} \sum_{n=M}^N a_n\left(z^{n+1} - z^n\right) \\
                         &= \frac{1}{z - 1}\left(\sum_{n=M}^N a_n z^{n+1} - \sum_{n=M}^N a_n z^n\right) \\
                         &= \frac{1}{z - 1}\left(\sum_{n=M}^N a_n z^{n+1} - \sum_{n=M-1}^{N-1} a_{n+1} z^{n+1}\right) \\
                         &= \frac{1}{z - 1}\left(a_N z^{N+1} - a_M z^M + \sum_{n=M}^{N-1}\left(a_n - a_{n+1}\right) z^{n+1}\right)
  .\end{align*}
  Taking the absolute value gives us
  \begin{align*}
    \left\lvert \sum_{n=M}^N a_n z^n \right\rvert &\leq \frac{1}{\lvert z - 1 \rvert}\left(a_N + a_M + \sum_{n=M}^{N-1}\left(a_n - a_{n+1}\right)\right) \\
                                      &= \frac{1}{\lvert z - 1 \rvert}\left(a_N + a_M + \left(a_M - a_{M+1}\right) + \cdots + \left(a_{N-1} - a_N\right)\right) \\
                                      &= \frac{2 a_M}{\lvert z - 1 \rvert} \to 0
  .\end{align*}
  Thus, the series is Cauchy and converges.
\end{proof}

\begin{lemma}[Dirichlet's Test]
  Let $(a_n)$ and $(b_n)$ be sequences. If
  \begin{enumerate}
    \item $(a_n)$ is monotonic and converges to $0$, and

    \item the partial sums of $(b_n)$ are bounded,
  \end{enumerate}
  then the series $\sum_{n=1}^\infty a_nb_n$ converges.
\end{lemma}

This proof requires the usage of Summation by Parts.

\begin{proof}
  TODO
\end{proof}

\subsection{Rearrangements of Infinite Series}

\begin{definition}[Rearrangement of a Series]
  Let $\sum_{n=1}^\infty a_n$ be an infinite series. A \emph{rearrangement} of this series is a series obtained by permuting its terms, that is
  \[%
    \sum_{n=1}^\infty a_{\sigma(n)}
  ,\]%
  where $\sigma : \N \to \N$ is a bijection (a one-to-one and onto mapping).
\end{definition}

\begin{theorem}[Riemann's Rearrangement Theorem]
  If $\sum_{n=1}^\infty a_n$ is a conditionally convergent series, then for any real number $L \in \R$ (or even $\pm\infty$), there exists a rearrangement $\sigma$ such that,
  \[%
    \sum_{n=1}^\infty a_{\sigma(n)} = L
  .\]%
\end{theorem}

\begin{proof}
  Since $\sum_{n=1}^\infty a_n$ is conditionally convergent, the positive terms $\{a_n^+\}$ and negative terms $\{a_n^-\}$ both diverge to $+\infty$ in absolute value. To construct a rearrangement summing to $L$, proceed as follows.
  \begin{itemize}
    \item Select positive terms $\{a_n^+\}$ to sum until the partial sum exceeds $L$.

    \item Then add negative terms $\{a_n^-\}$ to decrease the partial sum until it falls below $L$.

    \item Alternate between adding positive and negative terms in this manner, ensuring the partial sums oscillate closer and closer to $L$.
  \end{itemize}
  This process guarantees convergence to $L$, as the series' terms tend to zero and the oscillations diminish.
\end{proof}

\begin{corollary}
  If $\sum_{n=1}^\infty a_n$ converges absolutely, then any rearrangement $\sum_{n=1}^\infty a_{\sigma(n)}$ also converges to the same sum,
  \[%
    \sum_{n=1}^\infty a_{\sigma(n)} = \sum_{n=1}^\infty a_n
  .\]%
\end{corollary}

\begin{proof}
  Absolute convergence implies that $\sum_{n=1}^\infty a_n \le \sum_{n=1}^\infty \lvert a_n \rvert \in \R$. Rearranging the terms does not affect the total sum since the series converges uniformly regardless of order.
\end{proof}

\begin{example}[Harmonic Series Rearrangements]
  The harmonic series $\sum_{n=1}^\infty \frac{(-1)^{n+1}}{n}$ converges conditionally to $\ln(2)$. By Riemann’s Rearrangement Theorem, this series can be rearranged to sum to any real number $L$ or diverge to $\pm\infty$.
\end{example}

\begin{lemma}[Group Rearrangements]
  Grouping terms of an absolutely convergent series into blocks does not change its sum. For example, if $a_n = \frac{1}{n^2}$, grouping the terms as $(a_1)$, $(a_2 + a_3)$, $(a_4 + a_5 + a_6)$, etc., preserves the sum,
  \[%
    \sum_{n=1}^\infty a_n = \sum_{k=1}^\infty \left( \sum_{j=1}^k a_j \right)
  .\]%
\end{lemma}

\begin{proof}
  Let $S_n$ denote the partial sums of $\sum_{n=1}^\infty a_n$. Grouping terms corresponds to defining a new sequence of partial sums $T_m$, where $T_m$ involves summing blocks of $S_n$. Since $\sum_{n=1}^\infty a_n$ converges absolutely, $T_m$ converges to the same limit as $S_n$.
\end{proof}

\begin{example}[Rearrangement and Divergence]
  Consider the series $\sum_{n=1}^\infty \frac{1}{n}$, which diverges. Any rearrangement of this series also diverges, as the partial sums tend to infinity regardless of the order of terms.
\end{example}

\begin{theorem}[Unordered Series Convergence]
  If $\sum_{n=1}^\infty a_n$ converges conditionally, the set of all sums obtained by rearranging the series is unbounded. That is, the sums can fill the entire real number line $\R$.
\end{theorem}

\begin{proof}
  By Riemann’s Rearrangement Theorem, for any $L \in \R$, there exists a rearrangement $\sigma$ such that $\sum_{n=1}^\infty a_{\sigma(n)} = L$. This implies that the set of rearranged sums is dense in $\R$ and unbounded.
\end{proof}

\subsection{Double Summations of Infinite Series}

\begin{note}
  These next two sections aren't part of the course, but they are interesting to know. In future courses, such as Complex Analysis, professors often assume familiarity with these concepts.
\end{note}

\begin{definition}[Iterated Summation]
  Given a doubly indexed array of real numbers $\{a_{ij} \mid i, j \in \N\}$, the \emph{iterated sum} of the array is defined as
  \[%
    S_{mn} = \sum_{i=1}^m \sum_{j=1}^n a_{ij}
  .\]%
  The \emph{double summation} is defined as
  \[%
    \sum_{i,j=1}^\infty a_{ij} = \lim_{m, n \to \infty} S_{mn}
  .\]%
\end{definition}

\begin{theorem}[Fubini's Theorem for Double Series]
  Let $\{a_{ij} \mid i, j \in \N\}$ be a doubly indexed array of real numbers.
  \begin{enumerate}
    \item If $\sum_{i,j=1}^\infty \lvert a_{ij} \rvert < \infty$ (absolute convergence), then the double series converges, and the order of summation does not matter
      \[%
        \sum_{i=1}^\infty \sum_{j=1}^\infty a_{ij} = \sum_{j=1}^\infty \sum_{i=1}^\infty a_{ij} = \sum_{i,j=1}^\infty a_{ij}
      .\]%

    \item If $\sum_{i=1}^\infty \sum_{j=1}^\infty a_{ij}$ converges conditionally (not absolutely), then rearranging the terms or changing the order of summation may alter the sum or even lead to divergence.
  \end{enumerate}
\end{theorem}

\begin{proof}
  Suppose $\sum_{i,j=1}^\infty \lvert a_{ij} \rvert < \infty$. Let $S_{mn} = \sum_{i=1}^m \sum_{j=1}^n a_{ij}$. For any fixed $m, n \in \N$, we can write
  \[%
    \lvert S_{mn} \rvert \leq \sum_{i=1}^m \sum_{j=1}^n \lvert a_{ij} \rvert \leq \sum_{i,j=1}^\infty \lvert a_{ij} \rvert
  .\]%
  By the monotone convergence theorem, the limits $\lim_{m \to \infty} \sum_{i=1}^m \sum_{j=1}^n a_{ij}$ and $\lim_{n \to \infty} \sum_{j=1}^n \sum_{i=1}^m a_{ij}$ exist and are equal. Thus,
  \[%
    \sum_{i=1}^\infty \sum_{j=1}^\infty a_{ij} = \sum_{j=1}^\infty \sum_{i=1}^\infty a_{ij}
  .\]%
\end{proof}

\begin{corollary}[Order of Summation for Nonnegative Terms]
  If $a_{ij} \geq 0$ for all $i, j \in \N$, then
  \[%
    \sum_{i=1}^\infty \sum_{j=1}^\infty a_{ij} = \sum_{j=1}^\infty \sum_{i=1}^\infty a_{ij}
  ,\]%
  even if the series does not converge absolutely.
\end{corollary}

\begin{proof}
  For nonnegative terms, $\sum_{i,j=1}^\infty a_{ij}$ converges by monotonicity if and only if the partial sums $\sum_{i=1}^m \sum_{j=1}^n a_{ij}$ converge. Thus, the iterated sums also converge and are equal.
\end{proof}

\begin{example}[Harmonic Series Grid]
  Consider $a_{ij} = \frac{1}{i+j}$. We investigate the sums
  \[%
    \sum_{i=1}^\infty \sum_{j=1}^\infty a_{ij} \aand \sum_{j=1}^\infty \sum_{i=1}^\infty a_{ij}
  .\]%
  Since $a_{ij}$ is not absolutely convergent (due to divergence of $\sum_{i,j=1}^\infty \frac{1}{i+j}$), the series cannot be rearranged freely. Direct computation of either summation requires careful handling of the partial sums.
\end{example}

\begin{lemma}[Symmetry of Convergent Double Series]
  If $\{a_{ij}\}$ is absolutely convergent and symmetric, i.e., $a_{ij} = a_{ji}$ for all $i, j$, then
  \[%
    \sum_{i,j=1}^\infty a_{ij} = 2 \sum_{i=1}^\infty \sum_{j=i+1}^\infty a_{ij}
  .\]%
\end{lemma}

\begin{proof}
  Partition the terms into two symmetric halves: $a_{ij}$ for $i < j$ and $a_{ij}$ for $j < i$. Add the diagonal terms $a_{ii}$ separately. Absolute convergence ensures this rearrangement is valid.
\end{proof}

\subsection{Products of Infinite Series}

\begin{definition}[Cauchy Product of Infinite Series]
  Let $\sum_{n=0}^\infty a_n$ and $\sum_{n=0}^\infty b_n$ be two infinite series. The \emph{Cauchy product} of these series is defined as
  \[%
    \left(\sum_{n=0}^\infty a_n\right) \cdot \left(\sum_{n=0}^\infty b_n\right) = \sum_{n=0}^\infty \underbrace{\sum_{k=0}^n a_k b_{n-k}}_{c_n}
  .\]%
\end{definition}

\begin{theorem}[Cauchy Product Convergence]
  If $\sum_{n=0}^\infty a_n$ and $\sum_{n=0}^\infty b_n$ converge absolutely, then their Cauchy product also converges, and
  \[%
    \sum_{n=0}^\infty c_n = \left(\sum_{n=0}^\infty a_n\right) \cdot \left(\sum_{n=0}^\infty b_n\right)
  .\]%
\end{theorem}

\begin{proof}
  Since $\sum_{n=0}^\infty \lvert a_n \rvert < \infty$ and $\sum_{n=0}^\infty \lvert b_n \rvert < \infty$, the absolute convergence allows us to rearrange and group terms freely. Using the definition of the Cauchy product and the triangle inequality, we can write
  \[%
    \left\lvert\sum_{n=0}^\infty \sum_{k=0}^n a_k b_{n-k}\right\rvert \leq \sum_{n=0}^\infty \sum_{k=0}^n \lvert a_k \rvert \lvert b_{n-k} \rvert
  .\]%
  By Fubini's theorem for double series, the double sum can be rearranged into a product of two convergent series
  \[%
    \sum_{n=0}^\infty \sum_{k=0}^n \lvert a_k \rvert \lvert b_{n-k} \rvert = \left(\sum_{n=0}^\infty \lvert a_n \rvert\right) \cdot \left(\sum_{n=0}^\infty \lvert b_n \rvert\right)
  .\]%
  Thus, the Cauchy product converges absolutely, and the value is given by the product of the original sums.
\end{proof}

\begin{corollary}[Conditional Convergence of the Cauchy Product]
  If $\sum_{n=0}^\infty a_n$ and $\sum_{n=0}^\infty b_n$ converge conditionally (not absolutely), the Cauchy product may fail to converge or yield a different sum than the product of the original series.
\end{corollary}

\begin{example}[Harmonic Series and Cauchy Product]
  Let $a_n = \frac{1}{n+1}$ and $b_n = \frac{1}{n+1}$, corresponding to the harmonic series. Since the harmonic series diverges, their Cauchy product also diverges. However, truncating the series to $n$ terms and analyzing the partial sums provides useful approximations.
\end{example}

\begin{lemma}[Cauchy Product and Power Series]
  Let $\sum_{n=0}^\infty a_n x^n$ and $\sum_{n=0}^\infty b_n x^n$ be power series with radii of convergence $R_1$ and $R_2$, respectively. Then the Cauchy product of these series,
  \[%
    \sum_{n=0}^\infty c_n x^n, \quad c_n = \sum_{k=0}^n a_k b_{n-k}
  ,\]%
  converges for $\lvert x \rvert < \min(\{R_1, R_2\})$.
\end{lemma}

\begin{proof}
  Within the radius of convergence, the partial sums of the power series converge absolutely. Applying the Cauchy product formula term by term preserves convergence, as long as $\lvert x \rvert$ is strictly less than the minimum of the radii.
\end{proof}

\begin{example}[Product of Geometric Series]
  Consider the geometric series $\sum_{n=0}^\infty x^n = \frac{1}{1-x}$ for $\lvert x \rvert < 1$. The product of two such series is given by
  \[%
    \left(\sum_{n=0}^\infty x^n\right) \cdot \left(\sum_{n=0}^\infty x^n\right) = \sum_{n=0}^\infty c_n x^n
  ,\]%
  where
  \[%
    c_n = \sum_{k=0}^n 1 = n + 1
  .\]%
  Thus, the product is
  \[%
    \sum_{n=0}^\infty (n + 1)x^n = \frac{1}{(1 - x)^2}
  \qedhere.\]%
\end{example}

\begin{theorem}[Multiplication of Exponential Series]
  Let $e^x = \sum_{n=0}^\infty \frac{x^n}{n!}$. The product of two exponential series satisfies
  \[%
    \left(\sum_{n=0}^\infty \frac{x^n}{n!}\right) \cdot \left(\sum_{n=0}^\infty \frac{y^n}{n!}\right) = \sum_{n=0}^\infty \frac{(x+y)^n}{n!}
  .\]%
\end{theorem}

\begin{proof}
  Using the Cauchy product, the coefficient of $z^n$ in the product is given by
  \[%
    \sum_{k=0}^n \frac{x^k}{k!} \cdot \frac{y^{n-k}}{(n-k)!} = \frac{1}{n!} \sum_{k=0}^n \binom{n}{k} x^k y^{n-k}
  .\]%
  By the binomial theorem, this simplifies to
  \[%
    \frac{1}{n!} (x+y)^n
  .\]%
  Thus, the product of the exponential series is the exponential series of $x + y$.
\end{proof}
