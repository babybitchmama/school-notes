\lecture{13}{Nov 4 2024 Mon (13:02:06)}{Properties of Infinite Series}

\section{Properties of Infinite Series}
\label{sec:properties_of_infinite_series}

Just as we had an algebraic limit theorem for sequences, we have a similar
theorem for series.
\begin{theorem}[Algebraic Limit Theorem for Series] If $\sum_{k=1}^{\infty} a_k
  = A$ and $\sum_{k=1}^{\infty} b_k = B$, then
  \begin{enumerate}
    \item $\sum_{k=1}^{\infty} ca_k = cA$ for all $c \in \R$.
    \item $\sum_{k=1}^{\infty} (a_k + b_k) = A + B$.
  \end{enumerate}
\end{theorem}

\begin{proof} $ $
  \begin{enumerate}
    \item 

    \item \qedhere
  \end{enumerate}
\end{proof}

\subsection{Cauchy Series}
\label{sub_sec:cauchy_series}

Just as we had the Cauchy criterion for sequences, we have a similar criterion
for series.
\begin{theorem}[Cauchy Criterion for Series]
  The series $\sum_{k=1}^{\infty} a_k$ converges if and only if, given $\epsilon
  > 0$, there exists an $N \in \N$ such that whenever $n > m > N$, it follows
  that
  \[%
    \lvert a_{m+1} + a_{m+2} + \cdots + a_n \rvert < \epsilon
  .\]%
\end{theorem}

\begin{proof}
  Notice that
  \[%
    \left\lvert \sum_{k=1}^n a_k - \sum_{k=1}^m a_k \right\rvert = \lvert S_n - S_m \rvert = \lvert a_{m+1} + a_{m+2} + \cdots + a_n \rvert
  .\]%
  Thus, the series converges if and only if the sequence of partial sums is a
  Cauchy sequence.
\end{proof}

We can use this very basic fact to prove multiple basic facts about series.

\begin{theorem}
  If the series $\sum_{k=1}^{\infty} a_k$ converges, then $a_k \to 0$.
\end{theorem}

\begin{proof}
  Let $S_n = \sum_{k=1}^n a_k$. Then, $a_n = S_n - S_{n-1}$. Since the series
  converges, the sequence of partial sums is Cauchy. Thus, given $\epsilon > 0$,
  there exists an $N \in \N$ such that whenever $n > m > N$, it follows that
  \[%
    \lvert S_n - S_m \rvert = \lvert a_{m+1} + a_{m+2} + \cdots + a_n \rvert < \epsilon
  .\]%
  In particular, if we take $m = n - 1$, we have
  \[%
    \lvert a_n \rvert = \lvert S_n - S_{n-1} \rvert < \epsilon
  .\]%
  Thus, $a_n \to 0$.
\end{proof}

\begin{note}
  The converse isn't true. Take, for example, the harmonic series
  $\sum_{k=1}^{\infty} \frac{1}{k}$. This series diverges, but $\frac{1}{k} \to
  0$.
\end{note}

% subsection cauchy_series (end)

\subsection{Absolute and Conditional Convergence}
\label{sub_sec:absolute_and_conditional_convergence}



% subsection absolute_and_conditional_convergence (end)

\subsection{Rearrangements}
\label{sub_sec:rearrangements}



% subsection rearrangements (end)

% section properties_of_infinite_series (end)
