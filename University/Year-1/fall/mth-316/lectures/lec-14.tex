\lecture{14}{Nov 6 2024 Wed (13:02:06)}{Limits points, Open, and Closed Sets}

\section{Open and Closed Sets}
\label{sec:open_and_closed_sets}

Recall the definition of the $\epsilon$-neighborhood around a point
\[%
  V_\epsilon(x) = \{y \in \R^n \mid \lvert x - y \rvert < \epsilon\}
.\]%
\begin{definition}[Open Set]
  A set $U \subseteq \R^n$ is \textit{open} if the following holds: $(\forall
  \epsilon > 0)(\forall x \in U)[V_\epsilon(x) \subseteq U]$.
\end{definition}

\begin{example}\leavevmode
  \begin{enumerate}
    \item Perhaps one of the simplest examples is the open interval $(a, b)$.
      Let $x \in (a, b)$. Then, we can take $\epsilon = \min\{x - a, b - x\}$.
      Then, $V_\epsilon(x) \subseteq (a, b)$.

    \item The set $\R^n$ is open.

    \item The empty set $\emptyset$ is open.

    \item The set of all open balls is open.
  \end{enumerate}
\end{example}

But what happens if we take the union of open sets?
\begin{theorem}\leavevmode
  \begin{enumerate}
    \item The union of any collection of open sets is open.
    \item The intersection of a finite number of open sets is open.
  \end{enumerate}
\end{theorem}

\begin{proof}\leavevmode
  \begin{enumerate}
    \item Let $\{O_\lambda \mid \lambda \in \Lambda\}$ be a collection of open
      sets, indexed by $\Lambda$. Let $O = \bigcup_{\lambda \in \Lambda}
      O_\lambda$. Let $\alpha$ be an arbitrary element of $O$. Then, there
      exists $\lambda_0 \in \Lambda$ such that $\alpha \in O_{\lambda_0}$. Since
      $O_{\lambda_0}$ is open, there exists $\epsilon > 0$ such that
      $V_\epsilon(\alpha) \subseteq O_{\lambda_0} \subseteq O$. We can then use
      the fact that $O_{\lambda_0} \subseteq O$, which allows us to conclude
      that $V_\epsilon(\alpha) \subseteq O$, which is what we wanted to show.

    \item 
      \qedhere
  \end{enumerate}
\end{proof}

% section open_and_closed_sets (end)

\section{Complements}
\label{sec:complements}



% section complements (end)
