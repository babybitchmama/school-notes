\lecture{14}{Nov 1 2024 Fri (13:02:06)}{Double Summations of Infinite Series}

\begin{definition}[Iterated Summation]
  Given a doubly indexed array of real numbers $\{a_{ij} \mid i, j \in \N\}$, the \emph{iterated sum} of the array is defined as
  \[%
    S_{mn} = \sum_{i=1}^m \sum_{j=1}^n a_{ij}
  .\]%
  The \emph{double summation} is defined as
  \[%
    \sum_{i,j=1}^\infty a_{ij} = \lim_{m, n \to \infty} S_{mn}
  .\]%
\end{definition}

\begin{theorem}[Fubini's Theorem for Double Series]
  Let $\{a_{ij} \mid i, j \in \N\}$ be a doubly indexed array of real numbers.
  \begin{enumerate}
    \item If $\sum_{i,j=1}^\infty \lvert a_{ij} \rvert < \infty$ (absolute convergence), then the double series converges, and the order of summation does not matter
      \[%
        \sum_{i=1}^\infty \sum_{j=1}^\infty a_{ij} = \sum_{j=1}^\infty \sum_{i=1}^\infty a_{ij} = \sum_{i,j=1}^\infty a_{ij}
      .\]%

    \item If $\sum_{i=1}^\infty \sum_{j=1}^\infty a_{ij}$ converges conditionally (not absolutely), then rearranging the terms or changing the order of summation may alter the sum or even lead to divergence.
  \end{enumerate}
\end{theorem}

\begin{proof}
  Suppose $\sum_{i,j=1}^\infty \lvert a_{ij} \rvert < \infty$. Let $S_{mn} = \sum_{i=1}^m \sum_{j=1}^n a_{ij}$. For any fixed $m, n \in \N$, we can write
  \[%
    \lvert S_{mn} \rvert \leq \sum_{i=1}^m \sum_{j=1}^n \lvert a_{ij} \rvert \leq \sum_{i,j=1}^\infty \lvert a_{ij} \rvert
  .\]%
  By the monotone convergence theorem, the limits $\lim_{m \to \infty} \sum_{i=1}^m \sum_{j=1}^n a_{ij}$ and $\lim_{n \to \infty} \sum_{j=1}^n \sum_{i=1}^m a_{ij}$ exist and are equal. Thus,
  \[%
    \sum_{i=1}^\infty \sum_{j=1}^\infty a_{ij} = \sum_{j=1}^\infty \sum_{i=1}^\infty a_{ij}
  .\qedhere\]%
\end{proof}

\begin{corollary}[Order of Summation for Nonnegative Terms]
  If $a_{ij} \geq 0$ for all $i, j \in \N$, then
  \[%
    \sum_{i=1}^\infty \sum_{j=1}^\infty a_{ij} = \sum_{j=1}^\infty \sum_{i=1}^\infty a_{ij}
  ,\]%
  even if the series does not converge absolutely.
\end{corollary}

\begin{proof}
  For nonnegative terms, $\sum_{i,j=1}^\infty a_{ij}$ converges by monotonicity if and only if the partial sums $\sum_{i=1}^m \sum_{j=1}^n a_{ij}$ converge. Thus, the iterated sums also converge and are equal.
\end{proof}

\begin{example}[Harmonic Series Grid]
  Consider $a_{ij} = \frac{1}{i+j}$. We investigate the sums
  \[%
    \sum_{i=1}^\infty \sum_{j=1}^\infty a_{ij} \aand \sum_{j=1}^\infty \sum_{i=1}^\infty a_{ij}
  .\]%
  Since $a_{ij}$ is not absolutely convergent (due to divergence of $\sum_{i,j=1}^\infty \frac{1}{i+j}$), the series cannot be rearranged freely. Direct computation of either summation requires careful handling of the partial sums.
\end{example}

\begin{lemma}[Symmetry of Convergent Double Series]
  If $\{a_{ij}\}$ is absolutely convergent and symmetric, i.e., $a_{ij} = a_{ji}$ for all $i, j$, then
  \[%
    \sum_{i,j=1}^\infty a_{ij} = 2 \sum_{i=1}^\infty \sum_{j=i+1}^\infty a_{ij}
  .\]%
\end{lemma}

\begin{proof}
  Partition the terms into two symmetric halves: $a_{ij}$ for $i < j$ and $a_{ij}$ for $j < i$. Add the diagonal terms $a_{ii}$ separately. Absolute convergence ensures this rearrangement is valid.
\end{proof}

\subsection{Products of Infinite Series}

\begin{definition}[Cauchy Product of Infinite Series]
  Let $\sum_{n=0}^\infty a_n$ and $\sum_{n=0}^\infty b_n$ be two infinite series. The \emph{Cauchy product} of these series is defined as
  \[%
    \left(\sum_{n=0}^\infty a_n\right) \cdot \left(\sum_{n=0}^\infty b_n\right) = \sum_{n=0}^\infty \underbrace{\sum_{k=0}^n a_k b_{n-k}}_{c_n}
  .\]%
\end{definition}

\begin{theorem}[Cauchy Product Convergence]
  If $\sum_{n=0}^\infty a_n$ and $\sum_{n=0}^\infty b_n$ converge absolutely, then their Cauchy product also converges, and
  \[%
    \sum_{n=0}^\infty c_n = \left(\sum_{n=0}^\infty a_n\right) \cdot \left(\sum_{n=0}^\infty b_n\right)
  .\]%
\end{theorem}

\begin{proof}
  Since $\sum_{n=0}^\infty \lvert a_n \rvert < \infty$ and $\sum_{n=0}^\infty \lvert b_n \rvert < \infty$, the absolute convergence allows us to rearrange and group terms freely. Using the definition of the Cauchy product and the triangle inequality, we can write
  \[%
    \left\lvert\sum_{n=0}^\infty \sum_{k=0}^n a_k b_{n-k}\right\rvert \leq \sum_{n=0}^\infty \sum_{k=0}^n \lvert a_k \rvert \lvert b_{n-k} \rvert
  .\]%
  By Fubini's theorem for double series, the double sum can be rearranged into a product of two convergent series
  \[%
    \sum_{n=0}^\infty \sum_{k=0}^n \lvert a_k \rvert \lvert b_{n-k} \rvert = \left(\sum_{n=0}^\infty \lvert a_n \rvert\right) \cdot \left(\sum_{n=0}^\infty \lvert b_n \rvert\right)
  .\]%
  Thus, the Cauchy product converges absolutely, and the value is given by the product of the original sums.
\end{proof}

\begin{corollary}[Conditional Convergence of the Cauchy Product]
  If $\sum_{n=0}^\infty a_n$ and $\sum_{n=0}^\infty b_n$ converge conditionally (not absolutely), the Cauchy product may fail to converge or yield a different sum than the product of the original series.
\end{corollary}

\begin{example}[Harmonic Series and Cauchy Product]
  Let $a_n = \frac{1}{n+1}$ and $b_n = \frac{1}{n+1}$, corresponding to the harmonic series. Since the harmonic series diverges, their Cauchy product also diverges. However, truncating the series to $n$ terms and analyzing the partial sums provides useful approximations.
\end{example}

\begin{lemma}[Cauchy Product and Power Series]
  Let $\sum_{n=0}^\infty a_n x^n$ and $\sum_{n=0}^\infty b_n x^n$ be power series with radii of convergence $R_1$ and $R_2$, respectively. Then the Cauchy product of these series,
  \[%
    \sum_{n=0}^\infty c_n x^n, \quad c_n = \sum_{k=0}^n a_k b_{n-k}
  ,\]%
  converges for $\lvert x \rvert < \min(\{R_1, R_2\})$.
\end{lemma}

\begin{proof}
  Within the radius of convergence, the partial sums of the power series converge absolutely. Applying the Cauchy product formula term by term preserves convergence, as long as $\lvert x \rvert$ is strictly less than the minimum of the radii.
\end{proof}

\begin{example}[Product of Geometric Series]
  Consider the geometric series $\sum_{n=0}^\infty x^n = \frac{1}{1-x}$ for $\lvert x \rvert < 1$. The product of two such series is given by
  \[%
    \left(\sum_{n=0}^\infty x^n\right) \cdot \left(\sum_{n=0}^\infty x^n\right) = \sum_{n=0}^\infty c_n x^n
  ,\]%
  where
  \[%
    c_n = \sum_{k=0}^n 1 = n + 1
  .\]%
  Thus, the product is
  \[%
    \sum_{n=0}^\infty (n + 1)x^n = \frac{1}{(1 - x)^2}
  .\qedhere\]%
\end{example}

\begin{theorem}[Multiplication of Exponential Series]
  Let $e^x = \sum_{n=0}^\infty \frac{x^n}{n!}$. The product of two exponential series satisfies
  \[%
    \left(\sum_{n=0}^\infty \frac{x^n}{n!}\right) \cdot \left(\sum_{n=0}^\infty \frac{y^n}{n!}\right) = \sum_{n=0}^\infty \frac{(x+y)^n}{n!}
  .\]%
\end{theorem}

\begin{proof}
  Using the Cauchy product, the coefficient of $z^n$ in the product is given by
  \[%
    \sum_{k=0}^n \frac{x^k}{k!} \cdot \frac{y^{n-k}}{(n-k)!} = \frac{1}{n!} \sum_{k=0}^n \binom{n}{k} x^k y^{n-k}
  .\]%
  By the binomial theorem, this simplifies to
  \[%
    \frac{1}{n!} (x+y)^n
  .\]%
  Thus, the product of the exponential series is the exponential series of $x + y$.
\end{proof}
