\lecture{5}{Oct 09 2024 Wed (13:02:05)}{Cardinality}

The set $\I$ is the set of all irrational numbers. So far, we know that $\Q$ and $\I$ are densely packed in $\R$. This means that, given an interval $(a, b)$, there exists rational and irrational numbers alike. For a while, we've thought that they're in equal proportions. But, we'll see there are way more irrational numbers than rational numbers.

The term \emph{cardinality} is used to refer to the size of a set. The cardinalities of finite sets can be compared by attaching a natural number to them and using the property that $\N$ is an ordered and you can compare numbers, i.e., $5 > 3$, $10 < 17$, and so on. But how do we do the same thing when we can't use any natural numbers, say, for a set with an infinite number of elements?

\begin{note}
  Bijective functions are also called \emph{one-to-one correspondences}.
\end{note}

Here's Cantor's idea: Create a one-to-one correspondence with each other. This method's incredible since it works with sizes of sets that have infinitely many elements. First, let's define a few more definitions before going into one-to-one correspondences.

\begin{definition}[Cardinality]
  The \emph{cardinality} of a set $A$ is the number of elements in $A$. If $A$ is finite, then the cardinality of $A$ is a natural number. If $A$ is infinite, then the cardinality of $A$ is $\aleph_0$.
\end{definition}

\begin{definition}
  The set $A$ \emph{has the same cardinality as} $B$ if there exists $f : A \to B$ that is a one-to-one correspondence. In this case, we write $A \sim B$.
\end{definition}

\begin{definition}
  A set $A$ is \emph{countable} if $A \sim \N$. An infinite set that is not countable is called \emph{uncountable}.
\end{definition}

\begin{question}
  Show that $\N \sim E$, where $E$ is the set of all even natural numbers. Also, show that $\N \sim \Z$.
\end{question}

\begin{proof}[Solution]
  Let $f : \N \to E$ be defined by $f(n) = 2n$.
  \[%
    \begin{matrix}
      \N: & 1 & 2 & 3 & 4 & \cdots & n & \cdots \\
          & \updownarrow & \updownarrow & \updownarrow & \updownarrow & \cdots & \updownarrow & \\
      E: & 2 & 4 & 6 & 8 & \cdots & 2n & \cdots \\
    \end{matrix}
  \]%

  Let $f : \N \to \Z$ be defined by
  \[%
    f(n) = \begin{cases}
      \frac{n - 1}{2} & \text{if $n$ is even} \\
      -\frac{n}{2} & \text{if $n$ is odd}
    \end{cases}
  .\]%
  Again, you can see that $f$ is a one-to-one correspondence.
  \[%
    \begin{matrix}
      \N: & 1 & 2 & 3 & 4 & 5 & 6 & 7 & \cdots \\
          & \updownarrow & \updownarrow & \updownarrow & \updownarrow & \updownarrow & \updownarrow & \updownarrow & \\
      \Z: & 0 & 1 & -1 & 2 & -2 & 3 & -3 & \cdots \\
    \end{matrix}
  \qedhere\]%
\end{proof}

Since $E \subset \N$, it makes perfect sense that Card$(E) <$ Card$(\N)$. However, we have shown that $E \sim \N$, and therefore, $E$ is just as large as $\N$. We've even shown that $\Z \sim \N$, even though $\N \subset \Z$. As the cardinality of the sets reach infinity, all common sense goes out the window, due to our overexposure to finite sets.

\begin{example}
  The interval $(-1, 1)$ and $\R$ have the same cardinality. Let $f : (-1, 1) \to \R$ be defined by $f(x) = \tan\left(\frac{\pi}{2}x\right)$. This function is a one-to-one correspondence between $(-1, 1)$.
\end{example}

\begin{theorem}\leavevmode
  \begin{enumerate}
    \item The set $\Q$ is is countable.

    \item The set $\R$ is uncountable.
  \end{enumerate}
\end{theorem}

\begin{proof}\leavevmode
  \begin{enumerate}
    \item Set $A_1 = \{0\}$ and for each $n \ge 2$, let $A_n$ be the set given by
      \[%
        A_n = \left\{\pm \frac{p}{q} \mid \textrm{where $p, q \in \N$ are in lowest terms with $p + q = n$}\right\}
      .\]%
      The first few lists look like
      \begin{gather*}
        A_1 = \{0\}, \quad A_2 = \left\{\frac{1}{1}, -\frac{1}{1}\right\}, \quad A_3 = \left\{\frac{1}{2}, -\frac{1}{2}, \frac{2}{1}, -\frac{2}{1}\right\} \\
        A_4 = \left\{\frac{1}{3}, -\frac{1}{3}, \frac{3}{1}, -\frac{3}{1}\right\}, \aand A_5 = \left\{\frac{1}{4}, -\frac{1}{4}, \frac{2}{3}, -\frac{2}{3}, \frac{3}{2}, -\frac{3}{2}, \frac{4}{1}, -\frac{4}{1}\right\}
      .\end{gather*}
      Each $A_n$ is finite and every rational number appears in exactly one of these sets. Therefore, we let the one-to-one correspondence from $\Q$ to $\N$ as
      \setcounter{MaxMatrixCols}{20}
      \[%
        \begin{NiceMatrix}
          \N: & 1 & 2 & 3 & 4 & 5 & 6 & 7 & 8 & 9 & 10 & 11 & 12 & \cdots \\
              & \updownarrow & \updownarrow & \updownarrow & \updownarrow & \updownarrow & \updownarrow & \updownarrow & \updownarrow & \updownarrow & \updownarrow & \updownarrow & \updownarrow  & \\
          \Q: & 0 & \frac{1}{1} & -\frac{1}{1} & \frac{1}{2} & -\frac{1}{2} & \frac{2}{1} & -\frac{2}{1} & \frac{1}{3} & -\frac{1}{3} & \frac{3}{1} & -\frac{3}{1} & \frac{1}{4} & \cdots \\
          \CodeAfter \UnderBrace[shorten,yshift=3pt] {3-2}{3-2}{A_1}
          \CodeAfter \UnderBrace[shorten,yshift=3pt] {3-3}{3-4}{A_2}
          \CodeAfter \UnderBrace[shorten,yshift=3pt] {3-5}{3-8}{A_3}
          \CodeAfter \UnderBrace[shorten,yshift=3pt] {3-9}{3-12}{A_4}
        \end{NiceMatrix}
      \]%

      \vspace{0.4cm}

      Writing an exploit formula would be just a waste of time. What matters is that we see every rational number appearing only once in the correspondence. Therefore, $\Q$ is countable.

    \item Assume there exists a one-to-one correspondence $f : \N \to \R$. If we let $x_1 = f(1)$ and $x_2 = f(2)$, and so on, then our assumption that $f$ is onto means that we can write
      \begin{equation}\label{eq:real_numbers}
        \R = \{x_1, x_2, x_3, x_4, \dots\}
      ,\end{equation}
      and be confident that every real number appears somewhere on the list. Using the Nested Interval Property, construct $I_n$ such that
      \begin{enumerate}
        \item $I_{n+1} \subseteq I_n$ and

        \item $x_{n+1} \notin I_{n+1}$.
      \end{enumerate}

      If $x_{n_0}$ is some real number from the list in \ref{eq:real_numbers}, then we have $x_{n_0} \notin I_{n_0}$, and it follows that
      \[%
        x_{n_0} \notin \bigcap_{n=1}^\infty I_n
      .\]%
      Assume that the list in created in \ref{eq:real_numbers} is complete, meaning it contains every real number. Then this to the conclusion that
      \[%
        \bigcap_{n=1}^\infty I_n = \emptyset
      .\]%
      However, the Nested Interval Property asserts that the intersection of nested intervals is nonempty. By NIP, there exists at least one $x \in \bigcap_{n=1}^\infty I_n$ that cannot be in the list in \ref{eq:real_numbers}. This is a contradiction. Therefore, $\R$ is uncountable. \qedhere
  \end{enumerate}
\end{proof}

Why is this a big deal? This shows us that if a set can be arranged into a single list, then deleting some elements from this list results in another, possibly shorter list. This shows that countable sets are the smallest type of infinite set. Anything smaller is either still countable or finite.

This theorem showed us that the cardinality of $\R$ is a larger infinity. The natural numbers are so outnumbered by the reals that even if you map all the natural numbers onto the reals, there'll always be some reals left over.

\begin{theorem}
  If $A \subseteq B$ and $B$ is countable, then $A$ is either countable or finite.
\end{theorem}

\begin{proof}
  Since $B$ is countable, there exists a one-to-one correspondence $f : B \to \N$. We define a function $g : A \to \N$ by restricting $f$ to $A$. That is, $g(x) = f(x)$, for all $x \in A$. Since $f$ one-to-one, $g$ is also one-to-one. Now, we consider two cases.

  \textbf{Case 1 ($A$'s finite):} If $A$ is finite, then, trivially, $A$ is countable.

  \textbf{Case 2 ($A$'s infinite):} In this case, $g : A \to \N$ is an one-to-one function from an infinite subset $A$ of the countable set $B$ to $\N$. This implies that $A$ itself is countable, as it can be put in a one-to-one correspondence with a subset of $\N$.
\end{proof}

\begin{theorem}\leavevmode
  \begin{enumerate}
    \item If $A_1$, $A_2$, $A_3$, $\dots$, $A_m$ are countable sets, then the union $A_1 \cup A_2 \cup A_3 \cup \cdots \cup A_m$ is countable.

    \item If $A_n$ is a countable set for each $n \in \N$, then $\bigcup_{n=1}^\infty A_n$ is countable.
  \end{enumerate}
\end{theorem}

\begin{proof}\leavevmode
  \begin{enumerate}
    \item Assume $A_1$, $A_2$, $A_3$, $\dots$, $A_m$ are countable sets. Since each $A_i$ is countable, there exists a one-to-one correspondence $f_i : A_i \to \N$ or a one-to-one correspondence from $A_i$ to a finite subset of $\N$.

      We construct a one-to-one mapping from $A_1 \cup A_2 \cup A_3 \cup \cdots A_m$ to $\N$ as follows:

      Define a function $g : A_1 \cup A_2 \cup A_3 \cup \cdots \cup A_m \to \N$ by mapping each element $x \in A_i$ to $2^{i-1} \cdot f_i(x)$. Since each $f_i(x) \in \N$, this function maps elements of $A_i$ to distinct values in $\N$ for different $i$.

      Therefore, $g$ is a one-to-one correspondence from $A_1 \cup A_2 \cup A_3 \cup \cdots \cup A_m$ to $\N$, and hence $A_1 \cup A_2 \cup A_3 \cup \cdots \cup A_m$ is countable.

    \item Let $A_n$ be a countable set for each $n \in \N$. Since each $A_n$ is countable, there exists a one-to-one correspondence $f_n : A_n \to \N$. Define a function
      \[%
        g : \bigcup_{n=1}^\infty \to \N \times \N
      ,\]%
      by $g(x) = (n, f(n))$, where $x \in A_n$. This function $g$ pairs each element $x \in A_n$ with its ``index'' $n$ and its mapped value $f_n(x)$, thereby creating a unique ordered pair for each element in $\bigcup_{n=1}^\infty A_n$.

      Now, since $\N \times \N$ is countable, there exists a one-to-one correspondence $h : \N \times \N \to \N$. Therefore, the composition
      \[%
        h \circ g : \bigcup_{n=1}^\infty A_n \to \N
      ,\]%
      is a one-to-one correspondence from $\bigcup_{n=1}^\infty A_n$ into $\N$, proving that $\bigcup_{n=1}^\infty A_n$ is countable. \qedhere
  \end{enumerate}
\end{proof}

Now, we will prove a surprising fact, that the set of irrational numbers $\I$ is uncountable. We will use the fact that $\R$ is uncountable and $\Q$ is countable. Before that, we need a few lemmas.

\begin{lemma}\leavevmode
  \begin{enumerate}
    \item The set $\Q$ is countable.
    \item The interval $(0, 1)$ is uncountable.
    \item If $A$ is uncountable and $B$ is countable, then $A \setminus B$ is uncountable.
  \end{enumerate}
\end{lemma}

\begin{proof}\leavevmode
  \begin{enumerate}
    \item We briefly indicate a standard enumeration. Every nonzero rational can be written uniquely in lowest terms as $\sfrac{p}{q}$ with $p \in \Z$, $q \in \N$. Arrange pairs $(p, q)$ in a two-dimensional grid and list them by increasing $|p| + q$, skipping those not in lowest terms; include $0 = \sfrac{0}{1}$. This produces a sequence containing every rational exactly once. Thus $\Q$ is countable, and so is $\Q \cap (0,1)$.

    \item The function $g(t) = \tan(t)$ is a bijection from $(-\sfrac{\pi}{2}, \sfrac{\pi}{2})$ onto $\R$, strictly increasing with limits $\pm\infty$ at the endpoints. The affine map $h(x) = \pi(x - \sfrac{1}{2})$ is a bijection from $(0,1)$ onto $(-\sfrac{\pi}{2}, \sfrac{\pi}{2})$. Then $f = g\circ h$ is a bijection $(0, 1) \to \R$.

    \item Suppose $A\setminus B$ were countable. Then
      \[%
        A = (A\setminus B)\,\cup\,(A\cap B)
      ,\]%
      a union of two countable sets (since $A\cap B\subseteq B$). Hence $A$ would be countable, contradicting the hypothesis. \qedhere
  \end{enumerate}
\end{proof}

\begin{theorem}
  The set $\I$ is uncountable.
\end{theorem}

\begin{proof}
  Applying the lemma with $A = (0,1)$ (i) and $B = \Q \cap (0, 1)$ (ii), we conclude that $(0, 1) \setminus \Q$ is uncountable. Since $(0, 1) \setminus \Q \subseteq \R \setminus \Q$, the set of all irrationals $\R \setminus \Q$ is uncountable as well.
\end{proof}

\begin{remark}
  An alternative route is to first prove directly that $\R$ is uncountable by the diagonal argument (e.g., on $(0,1)$), then combine this with the countability of $\Q$ and the lemma to conclude $\R \setminus \Q$ is uncountable. Both approaches are equivalent in substance.
\end{remark}
