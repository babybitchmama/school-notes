\nte{Oct 09 2024 Wed (13:02:05)}{Cardinality}

\section{Types of Functions}
\label{sec:types_of_functions}

\begin{definition}[Cardinality]
  The \textit{cardinality} of a set $A$ is the number of elements in $A$. If $A$
  is finite, then the cardinality of $A$ is a natural number. If $A$ is
  infinite, then the cardinality of $A$ is $\aleph_0$.
\end{definition}

\begin{notation}
  If $A$ and $B$ have the same cardinality, we write $A \sim B$.
\end{notation}

\begin{definition}
  Let $f$ be a function $f : A \to B$. We say $f$ is \textit{onto} if $\forall y
  \in B$, $\exists x \in A$, $f(x) = y$. We say $f$ is \textit{one-to-one} if
  $\forall x_1, x_2 \in A$, $f(x_1) = f(x_2) \implies x_1 = x_2$.
\end{definition}

\begin{example}
  Let $A$ be the set of all the students in a room and $B$ be the set of all the
  chairs in the room. Then, $f$ is onto if every chair has a student, and $f$ is
  one-to-one if no two students are sitting in the same chair.
\end{example}

\begin{example}
  The function $f : \N \to \{-1, 1\}$ defined by $f(n) = (-1)^n$ is onto, as all
  natural numbers are being mapped to either $-1$ or $1$. However, $f$ is not
  one-to-one, as $f(2) = f(4) = 1$.
\end{example}

\begin{definition}
  Let $f$ be a function $f : A \to B$. If $f$ is both onto and one-to-one, then
  we call $f$ a \textit{one-to-one correspondence} of $A$ and $B$.
\end{definition}

\begin{note}
  Sometimes, onto is referred to as \textit{surjective}, and one-to-one is
  referred to as \textit{injective}, and a one-to-one correspondence is referred
  to as a \textit{bijection}.
\end{note}

\begin{claim}
  If $f$ is a one-to-one correspondence of $A$ and $B$, then $A \sim B$.
\end{claim}

% section types_of_functions (end)

\section{Countable Sets}
\label{sec:countable_sets}

\begin{definition}
  A set $A$ is \textit{countable} if $A \sim \N$. An infinite set that is not
  countable is called \textit{uncountable}.
\end{definition}

\begin{question}
  Show that $\N \sim E$, where $E$ is the set of all even natural numbers. Also,
  show that $\N \sim \Z$.
\end{question}

\begin{proof}
  Let $f : \N \to E$ be defined by $f(n) = 2n$.
  \[%
    \begin{matrix}
      \N: & 1 & 2 & 3 & 4 & \cdots & n & \cdots \\
          & \updownarrow & \updownarrow & \updownarrow & \updownarrow & \cdots & \updownarrow & \\
      E: & 2 & 4 & 6 & 8 & \cdots & 2n & \cdots \\
    \end{matrix}
  \]%

  Let $f : \N \to \Z$ be defined by
  \[%
    f(n) = \begin{cases}
      \frac{n - 1}{2} & \text{if $n$ is even} \\
      -\frac{n}{2} & \text{if $n$ is odd}
    \end{cases}
  .\]%
  Again, you can see that $f$ is a one-to-one correspondence.
  \[%
    \begin{matrix}
      \N: & 1 & 2 & 3 & 4 & 5 & 6 & 7 & \cdots \\
          & \updownarrow & \updownarrow & \updownarrow & \updownarrow & \updownarrow & \updownarrow & \updownarrow & \\
      \Z: & 0 & 1 & -1 & 2 & -2 & 3 & -3 & \cdots \\
    \end{matrix}
  \qedhere\]%
\end{proof}

Since $E \subset \N$, it makes perfect sense that $\Card(E) < \Card(\N)$.
However, we have shown that $E \sim \N$, and therefore, $E$ is just as large as
$\N$. We've even shown that $\Z \sim \N$, even though $\N \subset \Z$. As the
cardinality of the sets reach infinity, all common sense goes out the window,
due to our overexposure to finite sets.

\begin{example}
  The interval $(-1, 1)$ and $\R$ have the same cardinality. Let $f : (-1, 1)
  \to \R$ be defined by $f(x) = \tan\left(\frac{\pi}{2}x\right)$. This function
  is a one-to-one correspondence between $(-1, 1)$.
\end{example}

% section countable_sets (end)

\begin{note}
  There will be no notes for the next lecture, as it's quiz day.
\end{note}

\newpage
