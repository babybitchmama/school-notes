\lecture{4}{Oct 7 2024 Mon (13:01:52)}{Consequences of Completeness}

% \subsection{Nested Interval Property}

% \begin{theorem}[Nested Interval Property]
%   For each $n \in \N$, assume we are given a closed interval $I_n = [a_n, b_n] = \left\{x \in \R \mid a_n \le x \le b_n\right\}$. Assume also that each $I_n$ contains the next interval, i.e., $I_n \supseteq I_{n+1}$. Then, the resulting sequence of closed intervals
%   \[%
%     I_1 \supseteq I_2 \supseteq I_3 \supseteq \cdots
%   ,\]%
%   has a nonempty intersection. That is,
%   \[%
%     \bigcap_{n=1}^\infty I_n \neq \emptyset
%   .\]%
% \end{theorem}

% \begin{proof}
%   Let $A = \{a_n : n \in \N\}$. Since $I_n \supseteq I_{n+1}$, we have $a_1 \le a_2 \le a_3 \le \cdots$ (the left endpoints are nondecreasing) and $b_1 \ge b_2 \ge b_3 \ge \cdots$ (the right endpoints are nonincreasing). Moreover, for each $n$ we have $a_n \le b_n \le b_1$, so $A$ is bounded above by $b_1$.

%   By the Axiom of Completeness, $A$ has a least upper bound $s = \sup(A)$. Combining these inequalities, for every $n$ we have
%   \[%
%     a_n \le s \le b_n
%   ,\]%
%   which means $s \in I_n$ for all $n \in \N$. Hence
%   \[%
%     s \in \bigcap_{n=1}^\infty I_n
%   ,\]%
%   proving that the intersection is nonempty.
% \end{proof}

% The Nested Interval Property can be viewed graphically, as shown below.
% \begin{figure}[H]
%   \centering

%   \begin{tikzpicture}
%     \draw[thick] (-7, 0) -- (7, 0);

%     \foreach \i/\pos in {1/-6, 2/-5, 3/-4, n/-2} {
%       \node at (\pos, 0) {$[$};
%       \node[below] at (\pos, -0.3) {$a_{\i}$};
%     }
%     \node[below] at (-3, -0.3) {$\cdots$};

%     \foreach \i/\pos in {n/2, 3/4, 2/5, 1/6} {
%       \node at (\pos, 0) {$]$};
%       \node[below] at (\pos, -0.3) {$b_{\i}$};
%     }
%     \node[below] at (3, -0.3) {$\cdots$};

%     \node[below] at (1, -0.3) {$\cdots$};
%     \node[below] at (-1, -0.3) {$\cdots$};

%     \draw[decorate,decoration={brace,amplitude=5pt},thick] (-6.2, 0.4) -- (-1.8, 0.4) node[midway, above=5pt] {{\small$A = \{a_n \mid n \in \N\}$}};
%   \end{tikzpicture}
% \end{figure}

% \begin{example}\leavevmode
%   \begin{enumerate}
%     \item The interval $I_n = \left[0, \frac{1}{n}\right]$, which is clearly $I_n \supset I_{n+1}$, since $\bigcap_{n=1}^\infty = \{0\}$.

%     \item The counter example is $I_n = \left(0, \frac{1}{n}\right)$, $\bigcap_{n=0}^\infty I_n = \emptyset$, since $0 \notin I_n$. \qedhere
%   \end{enumerate}
% \end{example}


% \begin{proposition}
%   The set $\N$ is unbounded.
% \end{proposition}

% \begin{proof}
%   Assume, for contradiction, that $\N$ is bounded above. Then, it follows that we can set $\alpha = \sup(\N)$. If we consider $\alpha - 1$, then we no longer have an upper bound for $\N$, and therefore, there exists an $n \in \N$ satisfying $\alpha - 1 < n$, which is equivalent to $\alpha < n + 1$. Since we have $n + 1 \in \N$, this contradicts the assumption that $\alpha$ is the least upper bound of $\N$.
% \end{proof}

% \begin{theorem}[Archimedian Property]\leavevmode
%   \begin{enumerate}
%     \item $(\forall x \in \R)(\exists n \in \N)[n > x]$.

%     \item $(\forall y > 0)(\exists n \in \N)[\sfrac{1}{n} < 1]$.
%   \end{enumerate}
% \end{theorem}

% \begin{proof}\leavevmode
%   \begin{enumerate}
%     \item Property (i) follows given that $\N$ is unbounded.

%     \item Apply property (i) with $x = \sfrac{1}{y} > 0$. \qedhere
%   \end{enumerate}
% \end{proof}

% % \section{Density of $\Q$ in $\R$}

% % \begin{theorem}[Density of $\Q$ in $\R$]
% %   Between any two real numbers $a$ and $b$ with $a < b$, there exists a rational number $q$ such that $a < q < b$.
% % \end{theorem}

% % \begin{proof}
% %   Let $a > 0$, giving us $0 < a < b$. Then, $b - a > 0$. By the archimedean property, $(\exists n \in \N)[\sfrac{1}{n} < b - a \iff a < b - \sfrac{1}{n}]$.

% %   There is an $m \in \N$ such that $m - 1 \le na \le m \iff \frac{m - 1}{n} \le a \le \frac{m}{n}$.

% %   On the right hand side, we have $a \le \sfrac{m}{n}$, and on the left hand side, we have
% %   \begin{align*}
% %     \phantom{\implies}\quad&m \le an + 1 \\
% %     \implies\quad&\frac{m}{n} \le b \\
% %     \iff\quad&a \le \frac{m}{n} \le b
% %   .\qedhere\end{align*}
% % \end{proof}

% % Essentially, there are an infinite number of rational numbers between any two real numbers, giving us
% % \[%
% %   a < r_1 < r_2 < r_3 < \dots < b
% % .\]%

% % \begin{corollary}
% %   Between any two real numbers $a < b$, then there are infinitely many irrational numbers.
% % \end{corollary}

% % \begin{proof}
% %   Assume $a > 0$. Given $a < b$, then $a - \sqrt{2} < b - \sqrt{2}$. By the density of $\Q$, then there exists $r \in \Q$ such that $a - \sqrt{2} < r_1 < b - \sqrt{2} \implies a < r + \sqrt{2} < b$. Since $r$ is rational, then $r + \sqrt{2}$ is irrational.
% % \end{proof}

% % \section{Square Roots}

% % \begin{proposition}
% %   There exists a real number $\alpha$ satisfying $\alpha^2 = 2$.
% % \end{proposition}

% % \begin{proof}
% %   Let $S = \left\{x \in \R \mid x^2 < 2\right\}$. Then $S$ is bounded, nonempty, and has a least upper bound. Let $\alpha = \sup(S)$. We want to show $\alpha^2 = 2$ by showing that $\alpha^2 \nless 2$ and $\alpha^2 \ngtr 2$.

% %   \textbf{Case 1} ($\alpha^2 \nless 2$): Suppose $\alpha^2 < 2$. We consider
% %   \begin{align*}
% %     \left(\alpha + \frac{1}{n}\right)^2 &= \alpha^2 + \frac{2\alpha}{n} + \frac{1}{n^2} \\
% %                                         &< \alpha^2 + \frac{2\alpha + 1}{n}
% %   .\end{align*}
% %   Since $\alpha^2 < 2$, we can choose $n$ large enough such that
% %   \[%
% %     \alpha^2 + \frac{2\alpha + 1}{n} < 2 \iff 0 < \frac{2\alpha + 1}{2 - \alpha^2} < n
% %   .\]%
% %   By the archimedian property, there is an $n \in \N$ such that
% %   \[%
% %     \left(\alpha + \frac{1}{n}\right)^2 < 2 \implies \alpha + \frac{1}{n} \in S
% %   ,\]%
% %   but $\alpha$ is the least upper bound, which is a contradiction. Therefore, $\alpha \nless 2$.

% %   \textbf{Case 2} ($\alpha^2 \ngtr 2$): Suppose $\alpha^2 > 2$. We consider
% %   \begin{align*}
% %     \left(\alpha - \frac{1}{n}\right)^2 &= \alpha^2 - \frac{2\alpha}{n} + \frac{1}{n^2} \\
% %                                         &> \alpha^2 - \frac{2\alpha}{n}
% %   .\end{align*}
% %   Since $\alpha^2 > 2$, we can choose $n$ large enough such that
% %   \[%
% %     \alpha^2 - \frac{2\alpha}{n} > 2 \iff 0 < \frac{2\alpha}{\alpha^2 - 2} < n
% %   .\]%
% %   By the archimedian property, there is an $n \in \N$ such that
% %   \[%
% %     \left(\alpha - \frac{1}{n}\right)^2 > 2 \implies \alpha - \frac{1}{n} \notin S
% %   ,\]%
% %   but $\alpha$ is the least upper bound, which is a contradiction. Therefore, $\alpha \ngtr 2$.
% % \end{proof}
