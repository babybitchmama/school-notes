\nte{Oct 07 2024 Mon (13:01:52)}{}

If $s$ is an upper bound of a nonempty $S \subseteq \R$. Then, $-s$ is the lower
bound of
\[%
  -S = \left\{-x \mid x \in S\right\}
.\]%

\begin{theorem}
  \label{thm:}

  Let $a_n < b_n$, $\forall n \in \N,$ where $n \ge 1$. Define $I_n = [a_n, b_n]
  = \left\{x \mid a_n \le x \le b_n\right\}$. If $[a_n, b_n] \supseteq [a_{n+1},
  b_{n+1}]$, $\forall n \in \N$, where $n \ge 1$, then $\exists x \in \R$ such
  that $x \in \bigcap_{n = 1}^{\infty} [a_n, b_n]$.
\end{theorem}

\begin{example}
  \label{exm:}

  The interval $\displaystyle I_n = \left[0, \frac{1}{n}\right]$, which is
  clearly $I_n \supset I_{n+1}$, since $\displaystyle\bigcap_{n=1}^{\infty} =
  \{0\}$.

  The counter example is $\displaystyle I_n = \left(0, \frac{1}{n}\right)$
  because $\displaystyle\bigcap_{n=0}^{\infty} I_n = \emptyset$.
\end{example}

\begin{proof}
  \label{prf:}

  Let $A = \left\{a_n \mid n \in \N\right\}$. From this, we know that $a_n \le
  b_n$ and $I_n \supseteq I_{n+1}$. From this, we know that $a_n \le b_1$,
  $\forall n \in \N$. This means that $A$ is bounded.

  By the Axiom of Completeness, $A$ has a least upper bound, $s = \sup(A)$.
  Then, $a_n \le s$, $\forall a_n \in A$. Since $s$ is the least upper bound, we
  get $s \le b_n$, giving us
  \[%
    a_n \le s \le b_n, \forall n \in \N \iff s \in \bigcup_{n=0}^{\infty} I_n
  .\qedhere\]%
\end{proof}

%%%%%%%%%%%%%%%%%%%%%%%%%%%%%%%%%%%%%%%%%%%%%%%%%%

Prove the statement $\N$ is unbounded.

Assume not, meaning assume $\N$ is bounded. Then, there is a least upper bound
$s = \sup(\N)$. Then, $s - 1$ is not an upper bound. Then, there is an $n \in
\N$ such that $s - 1 < n \implies s < n + 1$. But, $n + 1 \in \N$.

That contradicts to $s$ being an upper bound of $\N$.

%%%%%%%%%%%%%%%%%%%%%%%%%%%%%%%%%%%%%%%%%%%%%%%%%%

\begin{theorem}[Archimedian Property]
  \label{thm:archimedian_property} $ $

  \begin{enumerate}
    \label{enum:archimedian_property}

    \item Given any number $x \in \R$, $\exists n \in \N$ satisfying $n > x$.

    \item Given any real number $y > 0$, $\exists n \in \N$ such that
      $\sfrac{1}{n} < 1$.
  \end{enumerate}
\end{theorem}

\begin{proof}
  \label{prf:archimedian_property} $ $

  \begin{enumerate}
    \label{enum:archimedian_property_prf}

    \item Property $\circled{1}$ follows since $\N$ is unbounded.

    \item Apply property $\circled{1}$ with $x = \sfrac{1}{y} > 0$.
  \end{enumerate}
\end{proof}

\newpage
