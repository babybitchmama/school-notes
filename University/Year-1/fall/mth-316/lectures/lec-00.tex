In earlier calculus courses, we learned how to differentiate, integrate, and sum infinite series — but much of that work was built on intuition, geometric pictures, and a bit of “hand-waving.” For instance, we used the limit notation in derivatives and integrals without ever stating precisely what a limit actually is. We treated infinite series as “adding up infinitely many numbers,” but never defined what “adding infinitely many” really means. These shortcuts are fine for computation, but they leave gaps in the logical foundation.

This course, Fundamentals of Analysis I, is about filling in those gaps. Our primary goal is to put calculus on a rigorous footing. We will start by formalizing the real numbers themselves, introducing axioms such as the Completeness Axiom that distinguish $\mathbb{R}$ from $\mathbb{Q}$. With these in place, we will define limits of sequences using the precise $\varepsilon$–$N$ definition, and then prove many of the results we once took for granted: the algebra of limits, convergence tests for series, and foundational theorems like Bolzano–Weierstrass and the Monotone Convergence Theorem.

We will also rigorously study infinite series. In earlier courses, convergence tests such as the Comparison Test, Ratio Test, Root Test, and Integral Test were introduced as computational tools, often accompanied by intuitive explanations or diagrams. In this course, we will go further: each of these tests will be stated and proved from first principles, relying only on the definition of series convergence and the properties of limits. We will examine absolute and conditional convergence, explore rearrangements of series, and prove deeper results like Dirichlet’s and Abel’s tests. By the end, series will no longer be a collection of tricks, but a logical structure whose rules are understood and justified.

Beyond sequences, we will extend the concept of limits to functions. In calculus, we often wrote $\lim_{x \to a} f(x) = L$ and manipulated it according to familiar rules, but we rarely justified why those rules work. Here, we will replace intuition with precision, using the $\varepsilon$–$\delta$ definition to capture exactly what it means for $f(x)$ to approach $L$ as $x$ approaches $a$. We will prove the fundamental algebraic and order properties of functional limits, study one-sided limits, and explore how the limit process interacts with other operations such as composition and inversion. This deeper understanding will prepare us for the formal definition of continuity and the powerful theorems that follow from it.

Along the way, we will develop a careful understanding of sets and functions, logic and proof techniques, and the topology of $\mathbb{R}$. Concepts such as open and closed sets, compactness, and connectedness will be defined and proved rigorously — not just drawn in diagrams. These tools are essential not only for real analysis, but also for later courses in measure theory, topology, and functional analysis.

Another key topic in this course is continuity. In earlier classes, we said a function is continuous if its graph can be drawn “without lifting the pencil.” Now, we will define continuity precisely using $\varepsilon$–$\delta$ arguments, prove important results about continuous functions, and understand why compactness makes continuous functions behave “nicely.”

The course will also train us in mathematical writing. Proofs are at the heart of analysis, and by the end, you will be expected to construct careful, complete arguments — not just perform computations.

This is the first half of a sequence. In the second half, Fundamentals of Analysis II, we will extend these ideas to functional limits, differentiation, and integration in full rigor, study uniform convergence, and explore deeper properties of functions and series. But before we can climb that hill, we must first build the strongest possible foundation — and that’s what Analysis I is for.
