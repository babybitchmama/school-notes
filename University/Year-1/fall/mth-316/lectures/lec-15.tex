\lecture{15}{Nov 4 2024 Mon (13:02:06)}{Properties of Series, Rearrangements}

Just as we had an algebraic limit theorem for sequences, we have a similar theorem for series.
\begin{theorem}[Algebraic Limit Theorem for Series]
  If $\sum_{k=1}^\infty a_k = A$ and $\sum_{k=1}^\infty b_k = B$, then
  \begin{enumerate}
    \item $\sum_{k=1}^\infty ca_k = cA$ for all $c \in \R$.

    \item $\sum_{k=1}^\infty (a_k + b_k) = A + B$.
  \end{enumerate}
\end{theorem}

\begin{proof}\leavevmode
  \begin{enumerate}
    \item Let $S_n = \sum_{k=1}^n a_k$. Then, $\lim_{m \to \infty} S_m = A$. Thus, by the Algebraic Limit Theorem for Sequences, $c\lim_{m \to \infty} S_m = \lim_{m \to \infty} (cS_m) = cA$. Thus, $\sum_{k=1}^\infty ca_k = cA$.

    \item Let $S_n = \sum_{k=1}^n a_k$ and $T_n = \sum_{k=1}^n b_k$. Then, $\lim_{m \to \infty} S_m = A$ and $\lim_{m \to \infty} T_m = B$. Thus, by the Algebraic Limit Theorem for Sequences, $\lim_{m \to \infty} (S_m + T_m) = A + B$. Thus, $\sum_{k=1}^\infty (a_k + b_k) = A + B$. \qedhere
  \end{enumerate}
\end{proof}

Just as we had the Cauchy criterion for sequences, we have a similar criterion for series.
\begin{theorem}[Cauchy Criterion for Series]
  The series $\sum_{k=1}^\infty a_k$ converges if and only if, given $\epsilon > 0$, there exists an $N \in \N$ such that whenever $n > m > N$, it follows that
  \[%
    \lvert a_{m+1} + a_{m+2} + \cdots + a_n \rvert < \epsilon
  .\]%
\end{theorem}

\begin{proof}
  Notice that
  \[%
    \left\lvert \sum_{k=1}^n a_k - \sum_{k=1}^m a_k \right\rvert = \lvert S_n - S_m \rvert = \lvert a_{m+1} + a_{m+2} + \cdots + a_n \rvert
  .\]%
  Thus, the series converges if and only if the sequence of partial sums is a Cauchy sequence.
\end{proof}

This is a very powerful criterion. We use this theorem to prove all of the basic tests in the future.

\begin{definition}[Absolute Convergence]
  A series $\sum_{k=1}^\infty a_k$ is said to \emph{converge absolutely} if the series $\sum_{k=1}^\infty \lvert a_k \rvert$ converges.
\end{definition}

\begin{definition}[Conditional Convergence]
  A series $\sum_{k=1}^\infty a_k$ is said to \emph{converge conditionally} if the series $\sum_{k=1}^\infty a_k$ converges, but the series $\sum_{k=1}^\infty \lvert a_k \rvert$ diverges.
\end{definition}

\subsection{Rearrangements of Infinite Series}

\begin{definition}[Rearrangement of a Series]
  Let $\sum_{n=1}^\infty a_n$ be an infinite series. A \emph{rearrangement} of this series is a series obtained by permuting its terms, that is
  \[%
    \sum_{n=1}^\infty a_{\sigma(n)}
  ,\]%
  where $\sigma : \N \to \N$ is a bijection (a one-to-one and onto mapping).
\end{definition}

\begin{theorem}[Riemann's Rearrangement Theorem]
  If $\sum_{n=1}^\infty a_n$ is a conditionally convergent series, then for any real number $L \in \R$ (or even $\pm\infty$), there exists a rearrangement $\sigma$ such that,
  \[%
    \sum_{n=1}^\infty a_{\sigma(n)} = L
  .\]%
\end{theorem}

\begin{proof}
  Since $\sum_{n=1}^\infty a_n$ is conditionally convergent, the positive terms $\{a_n^+\}$ and negative terms $\{a_n^-\}$ both diverge to $+\infty$ in absolute value. To construct a rearrangement summing to $L$, proceed as follows.
  \begin{itemize}
    \item Select positive terms $\{a_n^+\}$ to sum until the partial sum exceeds $L$.

    \item Then add negative terms $\{a_n^-\}$ to decrease the partial sum until it falls below $L$.

    \item Alternate between adding positive and negative terms in this manner, ensuring the partial sums oscillate closer and closer to $L$.
  \end{itemize}
  This process guarantees convergence to $L$, as the series' terms tend to zero and the oscillations diminish.
\end{proof}

\begin{corollary}
  If $\sum_{n=1}^\infty a_n$ converges absolutely, then any rearrangement $\sum_{n=1}^\infty a_{\sigma(n)}$ also converges to the same sum,
  \[%
    \sum_{n=1}^\infty a_{\sigma(n)} = \sum_{n=1}^\infty a_n
  .\]%
\end{corollary}

\begin{proof}
  Absolute convergence implies that $\sum_{n=1}^\infty a_n \le \sum_{n=1}^\infty \lvert a_n \rvert \in \R$. Rearranging the terms does not affect the total sum since the series converges uniformly regardless of order.
\end{proof}

\begin{example}[Harmonic Series Rearrangements]
  The harmonic series $\sum_{n=1}^\infty \frac{(-1)^{n+1}}{n}$ converges conditionally to $\ln(2)$. By Riemann’s Rearrangement Theorem, this series can be rearranged to sum to any real number $L$ or diverge to $\pm\infty$.
\end{example}

\begin{lemma}[Group Rearrangements]
  Grouping terms of an absolutely convergent series into blocks does not change its sum. For example, if $a_n = \frac{1}{n^2}$, grouping the terms as $(a_1)$, $(a_2 + a_3)$, $(a_4 + a_5 + a_6)$, etc., preserves the sum,
  \[%
    \sum_{n=1}^\infty a_n = \sum_{k=1}^\infty \left( \sum_{j=1}^k a_j \right)
  .\]%
\end{lemma}

\begin{proof}
  Let $S_n$ denote the partial sums of $\sum_{n=1}^\infty a_n$. Grouping terms corresponds to defining a new sequence of partial sums $T_m$, where $T_m$ involves summing blocks of $S_n$. Since $\sum_{n=1}^\infty a_n$ converges absolutely, $T_m$ converges to the same limit as $S_n$.
\end{proof}

\begin{example}[Rearrangement and Divergence]
  Consider the series $\sum_{n=1}^\infty \frac{1}{n}$, which diverges. Any rearrangement of this series also diverges, as the partial sums tend to infinity regardless of the order of terms.
\end{example}

\begin{theorem}[Unordered Series Convergence]
  If $\sum_{n=1}^\infty a_n$ converges conditionally, the set of all sums obtained by rearranging the series is unbounded. That is, the sums can fill the entire real number line $\R$.
\end{theorem}

\begin{proof}
  By Riemann’s Rearrangement Theorem, for any $L \in \R$, there exists a rearrangement $\sigma$ such that $\sum_{n=1}^\infty a_{\sigma(n)} = L$. This implies that the set of rearranged sums is dense in $\R$ and unbounded.
\end{proof}
