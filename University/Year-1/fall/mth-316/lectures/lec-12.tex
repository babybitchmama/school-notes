\lecture{12}{Oct 30 2024 Wed (13:00:12)}{Cauchy Sequences}

\section{Cauchy Sequences}
\label{sec:cauchy_sequences}

Another alternative way of defining convergent sequences is by using Cauchy
sequences.

\begin{definition}[Cauchy Sequence]
  A sequence $(a_n)$ is called a \textit{Cauchy Sequence} if,
  \[%
    (\forall \epsilon > 0)(\exists N \in \N)(\forall m, n > N)[\lvert a_n - a_m \rvert < \epsilon]
  .\]%
\end{definition}

The apparent use of Cauchy sequences is that they don't require a limit to exist
in order to be defined. They only require that the terms of the sequence get
arbitrarily close to each other. But these two definitions are equivalent, which
we will prove later.

The simplest property of Cauchy sequences is that they are bounded.

\begin{theorem}
  Every Cauchy sequence is bounded.
\end{theorem}

\begin{proof}
  Given $\epsilon = 1$, there exists an $N \in \N$ such that $\lvert x_m - x_n \rvert < 1$ for all $m, n \in \N$. Fix $m = N + 1$. Then, for all $n > N$,
  \begin{align*}
    \lvert x_n - x_{N + 1} \rvert < 1 &\implies \lvert x_n \rvert < \lvert x_{N+1} \rvert + 1 \\
                                      &\implies 1 - \lvert x_{N+1} \rvert < x_n < 1 + \lvert x_{N+1} \rvert
  .\end{align*}
  It follows that
  \begin{align*}
    A &= \max(\{\lvert x_1 \rvert, \lvert x_2 \rvert, \lvert x_3 \rvert, \cdots, \lvert x_{N-1} \rvert, \lvert x_{N + 1} \rvert + 1\}) \\
    B &= \min(\{\lvert x_1 \rvert, \lvert x_2 \rvert, \lvert x_3 \rvert, \cdots, \lvert x_{N-1} \rvert, \lvert x_{N + 1} \rvert - 1\})
  .\end{align*}
  Then, $A \le x_n \le B$.
\end{proof}

Now, let's prove that the two definitions of convergence are equivalent.

\begin{theorem}[Cauchy Criterion]
  A sequence converges if and only if it is a Cauchy sequence.
\end{theorem}

\begin{proof} $ $
  \begin{enumerate}
    \item[$\implies$)] Assume $x_n \to x$. Given $\epsilon > 0$, there exists an
      $N_1 \in \N$ such that $\lvert x_n - x \rvert < \frac{\epsilon}{2}$ for
      all $n > N$ and there exists an $N_2 \in \N$ such that $\lvert x_m - x
      \rvert < \frac{\epsilon}{2}$ for all $m > N$. Let $N = \max(\{N_1,
      N_2\})$. Then, for all $m, n > N$,
      \begin{align*}
        \lvert x_n - x_m \rvert &= \lvert x_n - x + x - x_m \rvert \\
                                &\le \lvert x_n - x \rvert + \lvert x - x_m \rvert \\
                                &< \frac{\epsilon}{2} + \frac{\epsilon}{2} = \epsilon
      .\end{align*}

    \item[$\impliedby$)] Assume $(x_n)$ is a Cauchy sequence. Then, it is
      bounded. By the Bolzano-Weierstrass Theorem, there exists a convergent
      subsequence $(x_{n_k})$. Let $x = \lim x_{n_k}$. Let $\epsilon > 0$. Since
      $(x_n)$ is Cauchy, there exists $N \in \N$ such that $\lvert x_n - x_m
      \rvert < \frac{\epsilon}{2}$ for all $m, n > N$. Since $x_{n_k} \to x$,
      choose a term in the subsequence $x_{n_K}$, with $n_K > N$ and $\lvert
      x_{n_K} - x \rvert < \frac{\epsilon}{2}$. Notice that
      \begin{align*}
        \lvert x_n - x \rvert &= \lvert x_n - x_{n_K} + x_{n_K} - x \rvert \\
                              &\le \lvert x_n - x_{n_K} \rvert + \lvert x_{n_K} - x \rvert \\
                              &< \frac{\epsilon}{2} + \frac{\epsilon}{2} = \epsilon
      .\qedhere\end{align*}
  \end{enumerate}
\end{proof}

% section cauchy_sequences (end)

\section{Limit Supremum and Infimum}
\label{sec:limit_supremum_and_infimum}

\begin{note}
  This isn't part of the course, but in the next few courses, professors assume
  that you magically know this. So, better know it now.
\end{note}

\begin{definition}[Limit Supremum and Infimum]
  Let $(a_n)$ be a sequence of real numbers. We define the \textit{limit
  supremum} and \textit{limit infimum} as
  \begin{align*}
    \limsup_{n \to \infty} a_n &= \lim_{n \to \infty} \left(\sup_{k \geq n} a_k\right) \\
    \liminf_{n \to \infty} a_n &= \lim_{n \to \infty} \left(\inf_{k \geq n} a_k\right)
  .\end{align*}
  where $\sup_{k \geq n} a_k$ and $\inf_{k \geq n} a_k$ denote the supremum and
  infimum, respectively, of the tail of the sequence starting from the $n$-th
  term.
\end{definition}

\begin{remark}
  The \textit{limit supremum} (\textit{limsup}) is the largest value that the
  sequence approaches infinitely often, while the \textit{limit infimum}
  (\textit{liminf}) is the smallest value that the sequence approaches
  infinitely often.
\end{remark}

\subsection{Understanding the Definitions}
\label{subsec:understanding_the_definitions}

The \textit{limsup} and \textit{liminf} of a sequence provide insights into its
long-term behavior.
\begin{itemize}
  \item \textbf{Supremum} ($\sup$) of a set is the least upper bound: the
    smallest number that is greater than or equal to every element of the set.

  \item \textbf{Infimum} ($\inf$) of a set is the greatest lower bound: the
    largest number that is less than or equal to every element of the set.
\end{itemize}

By taking the $\sup$ or $\inf$ of the tail of the sequence ($k \geq n$) and
letting $n \to \infty$, we focus on how the sequence behaves ``eventually.''

% subsection understanding_the_definitions (end)

\subsection{Properties of Limsup and Liminf}
\label{subsec:properties_of_limsup_and_liminf}

\begin{theorem} $ $
  \begin{enumerate}
    \item $\liminf_{n \to \infty} a_n \leq \limsup_{n \to \infty} a_n$.

    \item If $\liminf_{n \to \infty} a_n = \limsup_{n \to \infty} a_n = L$, then the sequence converges to $L$.

    \item $\limsup_{n \to \infty} a_n$ is finite if and only if the sequence is bounded above, and $\liminf_{n \to \infty} a_n$ is finite if and only if the sequence is bounded below.
  \end{enumerate}
\end{theorem}

\begin{proof} $ $
  \begin{enumerate}
    \item To prove $\liminf_{n \to \infty} a_n \leq \limsup_{n \to \infty} a_n$,
      note that by definition
    \[%
      \liminf_{n \to \infty} a_n = \sup_{n \in \N} \inf_{k \geq n} a_k \aand \limsup_{n \to \infty} a_n = \inf_{n \in \N} \sup_{k \geq n} a_k
    .\]%
    For every $n$, $\inf_{k \geq n} a_k \leq \sup_{k \geq n} a_k$. Taking the
    supremum over all $n$ for the infimum and the infimum over all $n$ for the
    supremum preserves this inequality. Hence
    \[%
      \liminf_{n \to \infty} a_n \leq \limsup_{n \to \infty} a_n
    .\]%

    \item Assume $\liminf_{n \to \infty} a_n = \limsup_{n \to \infty} a_n = L$.
      By definition of $\limsup$ and $\liminf$, for every $\epsilon > 0$, there
      exists $N \in \N$ such that for all $n > N$,
    \[%
      L - \epsilon < \inf_{k \geq n} a_k \leq a_n \leq \sup_{k \geq n} a_k < L + \epsilon
    .\]%
    This implies $|a_n - L| < \epsilon$ for all $n > N$, so $(a_n) \to L$.

    \item To show $\limsup_{n \to \infty} a_n$ is finite if and only if $(a_n)$
      is bounded above.
      \begin{itemize}
        \item If $\limsup_{n \to \infty} a_n$ is finite, then for every
          $\epsilon > 0$, there exists $N \in \N$ such that for all $n > N$,
          $\sup_{k \geq n} a_k < \limsup_{n \to \infty} a_n + \epsilon$. Thus,
          $a_k$ is bounded above.

        \item Conversely, if $(a_n)$ is bounded above, then $\sup_{k \geq n}
          a_k$ is finite for every $n$, and $\limsup_{n \to \infty} a_n =
          \inf_{n} \sup_{k \geq n} a_k$ is also finite.
      \end{itemize}

      Similarly, $\liminf_{n \to \infty} a_n$ is finite if and only if $(a_n)$
      is bounded below. The argument follows the same structure as above by
      considering $\inf_{k \geq n} a_k$. \qedhere
  \end{enumerate}
\end{proof}

\begin{example}[Convergent Sequence]
  Let $a_n = \frac{1}{n}$. Since the sequence converges to $0$, we have
  \begin{align*}
    \limsup_{n \to \infty} a_n &= 0 \\
    \liminf_{n \to \infty} a_n &= 0
  .\end{align*}
\end{example}

\begin{example}[Oscillating Sequence]
  Let $a_n = (-1)^n$. The sequence oscillates between $1$ and $-1$, so
  \begin{alignat*}{3}
    \sup_{k \geq n} a_k &= 1\quad&&\textrm{(since $a_k = 1$ infinitely often)} \\
    \inf_{k \geq n} a_k &= -1\quad&&\textrm{(since $a_k = -1$ infinitely often)}
  .\end{alignat*}
  Thus
  \begin{align*}
    \limsup_{n \to \infty} a_n &= 1 \\
    \liminf_{n \to \infty} a_n &= -1
  .\end{align*}
\end{example}

If a sequence is Cauchy, then
\begin{itemize}
  \item $\limsup_{n \to \infty} a_n = \liminf_{n \to \infty} a_n$,
  \item And the sequence converges to their common value.
\end{itemize}

The \textit{limsup} captures the ``upper boundary'' of eventual values of the
sequence, while the \textit{liminf} captures the ``lower boundary.''

\begin{lemma}
  Let $(a_n)$ be a sequence of real numbers. Then the following are equivalent
  \begin{itemize}
    \item $a_n \to a$,
    \item $\limsup_{n \to \infty} a_n = \liminf_{n \to \infty} a_n = a$.
  \end{itemize}
\end{lemma}

\begin{proof}
  We will prove the equivalence in both directions.

  \begin{enumerate}
    \item[$\implies)$] Suppose $a_n \to a$. By definition of convergence, for
      any $\epsilon > 0$, there exists $N \in \N$ such that for all $n > N$,
      \[%
        |a_n - a| < \epsilon \implies a - \epsilon < a_n < a + \epsilon
      .\]%
      Hence, the terms of the sequence are eventually arbitrarily close to $a$. For any $n$, the supremum of the tail $\sup_{k \geq n} a_k$ and the infimum of the tail $\inf_{k \geq n} a_k$ also converge to $a$ as $n \to \infty$. Thus, 
      \[%
        \limsup_{n \to \infty} a_n = \liminf_{n \to \infty} a_n = a
      .\]%

    \item[$\impliedby)$] Suppose $\limsup_{n \to \infty} a_n = \liminf_{n \to
      \infty} a_n = a$. By definition of $\limsup$ and $\liminf$, for every
      $\epsilon > 0$, there exists $N \in \N$ such that for all $n > N$,
      \[%
        a - \epsilon < \inf_{k \geq n} a_k \leq a_n \leq \sup_{k \geq n} a_k < a + \epsilon
      .\]%
      This implies that the terms of the sequence are eventually confined to the
      interval $(a - \epsilon, a + \epsilon)$. Therefore, $a_n \to a$.

      Since both implications hold, the two statements are equivalent. \qedhere
  \end{enumerate}
\end{proof}

% section limit_supremum_and_infimum (end)
