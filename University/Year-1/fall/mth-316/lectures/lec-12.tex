\lecture{12}{Oct 28 2024 Mon (13:00:12)}{Conversion Tests for Series}

% Now, we can start proving tests that have some substance.

% \begin{lemma}
%   If the series $\sum_{k=1}^\infty a_k$ converges, then $a_k \to 0$.
% \end{lemma}

% \begin{proof}
%   Let $S_n = \sum_{k=1}^n a_k$. Then, $a_n = S_n - S_{n-1}$. Since the series converges, the sequence of partial sums is Cauchy. Thus, given $\epsilon > 0$, there exists an $N \in \N$ such that whenever $n > m > N$, it follows that
%   \[%
%     \lvert S_n - S_m \rvert = \lvert a_{m+1} + a_{m+2} + \cdots + a_n \rvert < \epsilon
%   .\]%
%   In particular, if we take $m = n - 1$, we have
%   \[%
%     \lvert a_n \rvert = \lvert S_n - S_{n-1} \rvert < \epsilon
%   .\]%
%   Thus, $a_n \to 0$.
% \end{proof}

% \begin{note}
%   The converse isn't true. Take, for example, the harmonic series $\sum_{k=1}^\infty \frac{1}{k}$. This series diverges, but $\frac{1}{k} \to 0$.
% \end{note}

% \subsection{Absolute Conversion Test}

% \begin{lemma}[Absolute Convergence Test]
%   If $\sum_{k=1}^\infty \lvert a_k \rvert$ converges, then $\sum_{k=1}^\infty a_k$ converges.
% \end{lemma}

% \begin{proof}
%   Assume $\sum_{k=1}^\infty \lvert a_k \rvert$ converges. Let $\epsilon > 0$. Then, there exists an $N \in \N$ such that for every $n > m > N$, we have
%   \[%
%     \lvert a_{m+1} \rvert + \lvert a_{m+2} \rvert + \cdots + \lvert a_n \rvert < \epsilon
%   .\]%
%   Notice that
%   \[%
%     \lvert a_{m+1} + a_{m+2} + \cdots + a_n \rvert \leq \lvert a_{m+1} \rvert + \lvert a_{m+2} \rvert + \cdots + \lvert a_n \rvert < \epsilon
%   .\]%
%   Thus, by the Cauchy Criterion for Series, $\sum_{k=1}^\infty a_k$ converges.
% \end{proof}

% \subsection{Comparison Tests}

% \begin{lemma}[Comparison Test]
%   Assume $(a_k)$ and $(b_k)$ are sequences satisfying $(\forall k \in \N)[0 \le a_k \le b_k]$
%   \begin{enumerate}
%     \item If $\sum_{k=1}^\infty b_k$ converges, then $\sum_{k=1}^\infty a_k$ converges.

%     \item If $\sum_{k=1}^\infty a_k$ diverges, then $\sum_{k=1}^\infty b_k$ diverges.
%   \end{enumerate}
% \end{lemma}

% \begin{proof}\leavevmode
%   Both statements follow directly from the Cauchy Criterion for Series.
%   \[%
%     \lvert a_{m+1} + a_{m+2} + \cdots + a_n \rvert \le \lvert b_{m+1} + b_{m+2} + \cdots + b_n \rvert
%   .\]%
%   If the right-hand side is less than $\epsilon$, then the left-hand side is less than $\epsilon$. Thus, the series converges. The second statement follows from the contrapositive of the first statement.
% \end{proof}

% \subsection{Limit Comparison Test}

% \begin{lemma}[Limit Comparison Test]
%   Assume $(a_k)$ and $(b_k)$ are sequences satisfying $(\forall k \in \N)[0 < a_k \le b_k]$. If
%   \[%
%     \lim_{k \to \infty} \frac{a_k}{b_k} = c
%   ,\]%
%   where $c$ is a positive real number, then $\sum_{k=1}^\infty a_k$ converges if and only if $\sum_{k=1}^\infty b_k$ converges.
% \end{lemma}

% \begin{proof}
%   Let $c > 0$ such that $\lim_{k \to \infty} \frac{a_k}{b_k} = c$. This means for any $\epsilon > 0$, there exists an $N \in \N$ such that for all $k \geq N$, we have
%   \[%
%     c - \epsilon < \frac{a_k}{b_k} < c + \epsilon
%   .\]%
%   Multiplying through by $b_k$ (which is positive), we get
%   \[%
%     (c - \epsilon)b_k < a_k < (c + \epsilon)b_k
%   .\]%
%   Choose $\epsilon$ such that $0 < \epsilon < c$, so $c - \epsilon > 0$. For $k \geq N$, we now have
%   \[%
%     0 < (c - \epsilon) b_k \leq a_k \leq (c + \epsilon) b_k
%   .\]%

%   \begin{enumerate}
%     \item Convergence of $\sum_{k=1}^\infty b_k$:

%       If $\sum_{k=1}^\infty b_k$ converges, then $(c - \epsilon)\sum_{k=1}^\infty b_k$ converges because $c - \epsilon > 0$ is a constant multiple. By the Comparison Test applied to $(c - \epsilon)b_k \leq a_k$, it follows that $\sum_{k=1}^\infty a_k$ also converges.

%     \item Divergence of $\sum_{k=1}^\infty b_k$:

%       If $\sum_{k=1}^\infty b_k$ diverges, then $(c + \epsilon)\sum_{k=1}^\infty b_k$ also diverges because $c + \epsilon > 0$ is a constant multiple. By the Comparison Test applied to $a_k \geq (c - \epsilon)b_k$, it follows that $\sum_{k=1}^\infty a_k$ also diverges.
%   \end{enumerate}

%   Hence, $\sum_{k=1}^\infty a_k$ converges if and only if $\sum_{k=1}^\infty b_k$ converges.
% \end{proof}

% \subsection{Ratio Test}

% \begin{lemma}[Ratio Test]
%   Let $(a_n)$ be a sequence. If
%   \[%
%     \lim_{n \to \infty} \left\lvert \frac{a_{n+1}}{a_n} \right\rvert = L
%   ,\]%
%   then
%   \begin{enumerate}
%     \item If $L < 1$, then $\sum_{n=1}^\infty a_n$ converges absolutely.

%     \item If $L > 1$, then $\sum_{n=1}^\infty a_n$ diverges.

%     \item If $L = 1$, then the test is inconclusive.
%   \end{enumerate}
% \end{lemma}

% \begin{proof}\leavevmode
%   Let $L = \lim_{n \to \infty} \left\lvert \frac{a_{n+1}}{a_n} \right\rvert$.
%   \begin{enumerate}
%     \item $L < 1$:

%       Choose $\epsilon > 0$ such that $L + \epsilon < 1$. By the definition of the limit,
%       \[%
%         (\exists N \in \N)(\forall n \ge N)\left[\left\lvert \frac{a_{n+1}}{a_n} \right\rvert < L + \epsilon\right]
%       .\]%
%       This implies that for $n \geq N$
%       \[%
%         \lvert a_{n+1} \rvert < (L + \epsilon) \lvert a_n \rvert
%       .\]%
%       Iterating this inequality, we get
%       \[%
%         \lvert a_{n+1} \rvert < (L + \epsilon)^{n-N+1} \lvert a_N \rvert
%       .\]%
%       Since $L + \epsilon < 1$, the geometric series $\sum_{n=N}^\infty (L + \epsilon)^{n-N+1}$ converges. Therefore, by the Comparison Test, $\sum_{n=N}^\infty \lvert a_n \rvert$ converges, which implies that $\sum_{n=1}^\infty \lvert a_n \rvert$ converges. Thus, $\sum_{n=1}^\infty a_n$ converges absolutely.

%     \item $L > 1$:

%       Choose $\epsilon > 0$ such that $L - \epsilon > 1$. By the definition of the limit,
%       \[%
%         (\exists N \in \N)(\forall n \ge N)\left[\left\lvert \frac{a_{n+1}}{a_n} \right\rvert > L - \epsilon\right]
%       .\]%
%       This implies that for $n \geq N$
%       \[%
%         \lvert a_{n+1} \rvert > (L - \epsilon) \lvert a_n \rvert
%       .\]%
%       Iterating this inequality, we get
%       \[%
%         \lvert a_{n+1} \rvert > (L - \epsilon)^{n-N+1} \lvert a_N \rvert
%       .\]%
%       Since $L - \epsilon > 1$, the terms $\lvert a_n \rvert$ do not approach zero, and hence $\sum_{n=1}^\infty \lvert a_n \rvert$ diverges. Thus, $\sum_{n=1}^\infty a_n$ diverges.

%     \item $L = 1$:

%       The test provides no information about the convergence or divergence of the series. For example, the series $\sum_{n=1}^\infty \frac{1}{n}$ diverges, but $\sum_{n=1}^\infty \frac{1}{n^2}$ converges, and both satisfy $\lim_{n \to \infty} \left\lvert \frac{a_{n+1}}{a_n} \right\rvert = 1$. Therefore, the test is inconclusive in this case. \qedhere
%   \end{enumerate}
% \end{proof}

% \subsection{Root Test}

% \begin{lemma}[Root Test]
%   Let $(a_n)$ be a sequence. If
%   \[%
%     \limsup_{n \to \infty} \sqrt[n]{\lvert a_n \rvert} = L
%   ,\]%
%   then
%   \begin{enumerate}
%     \item If $L < 1$, then $\sum_{n=1}^\infty a_n$ converges absolutely.

%     \item If $L > 1$, then $\sum_{n=1}^\infty a_n$ diverges.

%     \item If $L = 1$, then the test is inconclusive.
%   \end{enumerate}
% \end{lemma}

% \begin{proof}
%   Let $L = \limsup_{n \to \infty} \sqrt[n]{\lvert a_n \rvert}$.

%   \begin{enumerate}
%     \item $L < 1$:

%       Choose $\epsilon > 0$ such that $L + \epsilon < 1$. By the definition of the $\limsup$, there exists $N \in \mathbb{N}$ such that for all $n \geq N$
%       \[%
%         \sqrt[n]{\lvert a_n \rvert} < L + \epsilon
%       .\]%
%       Raising both sides to the $n$th power, we have
%       \[%
%         (\forall n \geq N)\left[\lvert a_n \rvert < (L + \epsilon)^n\right]
%       .\]%
%       Since $L + \epsilon < 1$, the series $\sum_{n=N}^\infty (L + \epsilon)^n$ converges because it is a geometric series with ratio less than $1$. By the Comparison Test, $\sum_{n=N}^\infty \lvert a_n \rvert$ converges, and therefore $\sum_{n=1}^\infty \lvert a_n \rvert$ converges. Thus, $\sum_{n=1}^\infty a_n$ converges absolutely.

%     \item $L > 1$:

%       By the definition of $\limsup$, there exists a subsequence $(a_{n_k})$ such that
%       \[%
%         \sqrt[n_k]{\lvert a_{n_k} \rvert} \to L~\textrm{as}~k \to \infty
%       .\]%
%       Since $L > 1$, for sufficiently large $k$, we have
%       \[%
%         \sqrt[n_k]{\lvert a_{n_k} \rvert} > 1
%       ,\]%
%       which implies
%       \[%
%         \lvert a_{n_k} \rvert > 1^n = 1
%       .\]%
%       Thus, $\lvert a_{n_k} \rvert$ does not approach $0$, so the series $\sum_{n=1}^\infty a_n$ diverges.

%     \item $L = 1$:

%       The test is inconclusive because series with $L = 1$ may either converge or diverge. For instance
%       \[%
%         \sum_{n=1}^\infty \frac{1}{n^2}~\textrm{converges, while}~\sum_{n=1}^\infty \frac{1}{n}~\textrm{diverges}
%       .\]%
%       Both of these series satisfy $\limsup_{n \to \infty} \sqrt[n]{\lvert a_n \rvert} = 1$.\qedhere
%   \end{enumerate}
% \end{proof}

% \subsection{Alternating Series Test}

% \begin{lemma}[Alternating Series Test]
%   Let $(a_n)$ be a sequence satisfying the following requirements
%   \begin{enumerate}
%     \item $a_1 \ge a_2 \ge a_3 \ge \cdots \ge a_n \ge a_{n+1} \ge \cdots$,

%     \item $a_n \to 0$,
%   \end{enumerate}
%   Then, the alternating series $\sum_{n=1}^\infty (-1)^{n+1} a_n$ converges.
% \end{lemma}

% \begin{proof}
%   Let $S_n = \sum_{k=1}^n (-1)^{k+1} a_k$. Then, $S_{2n} = a_1 - a_2 + a_3 - a_4 + \cdots + a_{2n-1} - a_{2n}$. Since $a_n \to 0$, we have
%   \[%
%     \lvert S_{2n} - S_{2n-2} \rvert = \lvert a_{2n-1} - a_{2n} \rvert \to 0
%   .\]%
%   Thus, $(S_{2n})$ is a Cauchy sequence. Similarly, $(S_{2n+1})$ is also a Cauchy sequence. Thus, both sequences converge. Since the series is the sum of these two sequences, the series converges.
% \end{proof}

% \subsection{Condensation Test}

% \begin{lemma}[Condensation Test]
%   Suppose $(b_n)$ is decreasing and satisfies $b_n \ge 0$ for all $n \in \N$. Then, the series $\sum_{n=1}^\infty b_n$ converges if and only if the series
%   \[%
%     \sum_{n=0}^\infty 2^nb_{2^n} = b_1 + 2b_2 + 4b_4 + 8b_8 + 16b_{16} + \cdots
%   ,\]%
%   converges.
% \end{lemma}

% \begin{proof}
%   First, assume that $\sum_{n=0}^\infty 2^nb_{2^n}$ converges. Then, the partial sum
%   \[%
%     T_k = b_1 + 2b_2 + 4b_4 + \cdots + 2^kb_{2^k}
%   ,\]%
%   is bounded, by some number $M$. Because $b_n \ge 0$, we know that the partial sums are increasing. Let $S_m = b_1 + b_2 + b_3 + \cdots + b_m$. Fix $m$ and let $k$ be large enough to ensure $m \le 2_{k+1} - 1$. Then, $S_m \le S_{2^k + 1} - 1$ and
%   \begin{align*}
%     S_{2^{k+1}} - 1 &= b_1 + (b_2 + b_3) + (b_4 + \cdots + b_7) + \cdots + (b_{2^k} + \cdots + b_{2^{k+1} - 1}) \\
%                     &\le b_1 + 2b_2 + 4b_4 + \cdots + 2^kb_{2^k} \\
%                     &= T_k
%   .\end{align*}
%   Therefore, $S_m \le T_k \le M$ for all $m \in \N$. This implies that $(S_m)$ is bounded and therefore converges.
% \end{proof}

% This is a more general statement of the statement that the sum $\sum_{n=1}^\infty \frac{1}{n}$ diverges and that $\sum_{n=1}^\infty
% \frac{1}{n^\rho}$ converges for $\rho > 1$.

% \subsection{P-Series Test}

% \begin{corollary}[P-Series Test]
%   The series
%   \[%
%     \sum_{n=1}^\infty \frac{1}{n^\rho}
%   .\]%
%   converges if and only if $\rho > 1$.
% \end{corollary}

% \begin{proof}
%   Let $b_n = \frac{1}{n^\rho}$. By the Cauchy Condensation Test, the series $\sum_{n=1}^\infty b_n$ converges if and only if the series
%   \[%
%     \sum_{n=0}^\infty 2^n \cdot \frac{1}{(2^n)^\rho} = \sum_{n=0}^\infty \frac{1}{2^{n(\rho-1)}}
%   .\]%
%   converges. The latter is a geometric series with common ratio $r = \frac{1}{2^{\rho-1}}$. A geometric series converges if and only if $\lvert r \rvert < 1$. Thus, $\lvert r \rvert < 1$ implies
%   \[%
%     \frac{1}{2^{\rho-1}} < 1
%   ,\]%
%   which holds if and only if $\rho - 1 > 0$, or equivalently, $\rho > 1$. Therefore, $\sum_{n=1}^\infty \frac{1}{n^\rho}$ converges if and only if $\rho > 1$.
% \end{proof}

% The discrete analogue of integration by parts is often called \emph{summation by parts} or \emph{Abel's transformation}. It is the fundamental identity used to prove Dirichlet's test.

% \subsection{Summation by Parts}

% \begin{lemma}[Summation by parts]
%   Let $(a_n)_{n \ge 1}$ and $(b_n)_{n \ge 1}$ be real sequences and set
%   \[%
%     B_n \coloneqq \sum_{k=1}^n b_k \quad (B_0 \coloneqq 0)
%   .\]%
%   For every integer $N\ge 1$ we have
%   \[%
%     \sum_{k=1}^N a_k b_k = a_N B_N + \sum_{k=1}^{N-1} (a_k - a_{k+1}) B_k
%   .\]%
% \end{lemma}

% \begin{proof}
%   Write each $b_k$ as a difference of partial sums, $b_k = B_k - B_{k-1}$. Then
%   \[%
%     \sum_{k=1}^N a_k b_k = \sum_{k=1}^N a_k (B_k - B_{k-1}) = \sum_{k=1}^N a_k B_k - \sum_{k=1}^N a_k B_{k-1}
%   .\]%
%   In the second sum shift the index $k \mapsto k + 1$ (noting $B_0 = 0$)
%   \[%
%     \sum_{k=1}^N a_k B_{k-1} = \sum_{k=0}^{N-1} a_{k+1} B_k = \sum_{k=1}^{N-1} a_{k+1} B_k
%   .\]%
%   Therefore
%   \[%
%     \sum_{k=1}^N a_k b_k = a_N B_N + \sum_{k=1}^{N-1} (a_k - a_{k+1}) B_k
%   ,\]%
%   as claimed.
% \end{proof}

% \subsection{Dirichlet's Test}

% \begin{lemma}[Dirichlet's Test]
%   Let $(a_n)$ and $(b_n)$ be sequences. Suppose
%   \begin{enumerate}
%     \item $(a_n)$ is monotone and $\lim_{n\to\infty} a_n = 0$,

%     \item the partial sums $B_n=\sum_{k=1}^n b_k$ are bounded: there exists $M>0$ with $\abs{B_n} \le M$ for all $n$.
%   \end{enumerate}
%   Then the series $\sum_{n=1}^\infty a_n b_n$ converges.
% \end{lemma}

% \begin{proof}
%   We give the standard proof using summation by parts. For clarity assume $(a_n)$ is nonincreasing (the other monotone cases are handled similarly by sign changes) and that $a_n\ge 0$ for all $n$; the general monotone-to-$0$ case is reduced to this by multiplying by $-1$ if necessary.

%   Let $S_N \coloneqq \sum_{k=1}^N a_k b_k$ denote the $N$-th partial sum. By summation by parts,
%   \[%
%     S_N = a_N B_N + \sum_{k=1}^{N-1} (a_k - a_{k+1}) B_k
%   .\]%
%   By hypothesis there is $M>0$ with $\abs{B_k} \le M$ for every $k$. Because $(a_k)$ is nonincreasing and nonnegative, each difference $a_k-a_{k+1}\ge 0$, and the finite sum in the identity involves nonnegative weights. Estimate the absolute value of $S_N$ as follows:
%   \[%
%     \abs{S_N} \le \abs{a_N B_N} + \sum_{k=1}^{N-1} (a_k-a_{k+1}) \abs{B_k} \le M a_N + M\sum_{k=1}^{N-1} (a_k-a_{k+1})
%   .\]%
%   The telescoping sum equals
%   \[%
%     \sum_{k=1}^{N-1} (a_k-a_{k+1}) = a_1 - a_N
%   .\]%
%   Hence
%   \[%
%     \abs{S_N} \le M a_N + M(a_1 - a_N) = M a_1
%   .\]%
%   In particular the sequence $(S_N)$ of partial sums is bounded. To prove convergence we show it is Cauchy: for $N>n$ the same identity applied to $S_N-S_n$ gives
%   \[%
%     S_N - S_n = a_N B_N - a_n B_n + \sum_{k=n}^{N-1} (a_k-a_{k+1}) B_k
%   ,\]%
%   and therefore, using $\abs{B_k} \le M$,
%   \[%
%     \abs{S_N - S_n} \le M\abs{a_N} + M\abs{a_n} + M\sum_{k=n}^{N-1} (a_k-a_{k+1}) = M(\abs{a_N} + \abs{a_n} + a_n - a_N)
%   .\]%
%   Because $a_n \ge 0$ and $a_N \ge 0$ this simplifies to
%   \[%
%     \abs{S_N - S_n} \le 2M a_n
%   .\]%
%   Since $a_n \to 0$ as $n \to \infty$, for every $\epsilon > 0$ we can choose $n$ large enough that $2M a_n<\epsilon$, and then for all $N>n$ we have $\abs{S_N-S_n} < \epsilon$. Thus $(S_N)$ is Cauchy and hence convergent. This proves that $\sum_{n=1}^\infty a_n b_n$ converges.

%   If $(a_n)$ is monotone but not nonnegative, write $a_n = a_n^+ - a_n^-$ with the monotone parts and apply the same argument to each part (or reduce to the nonnegative monotone case by multiplying by $-1$ when necessary). This completes the proof.
% \end{proof}
