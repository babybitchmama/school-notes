\lecture{3}{Oct 4 2024 Fri (13:00:10)}{Axiom of Completeness}

\section{Logic}

The easiest proof to see as a beginner is the proof that $\sqrt{2}$ is irrational.

\begin{theorem}
  There is no rational number whose square is $2$.
\end{theorem}

\begin{proof}
  Assume that $\sqrt{2}$ is rational, meaning that $(\exists p \in \Z)(\exists q \in \Z)[2 = \sfrac{p}{q}]$. Assume that $\sfrac{p}{q}$ is in simplest form, meaning $p$ and $q$ have no common factor. Then $\sqrt{2} = \sfrac{p}{q} \iff 2 = \sfrac{p^2}{q^2} \implies 2q^2 = p^2$. This means $p^2$ is an even integer. Then $(\exists r \in \N)[p = 2r]$, giving us $4r^2 = 2q^2$. This means $q^2$ is also an even integer, which is a contradiction, since $p$ and $q$ were assumed to be in simplest form. Therefore, $\sqrt{2}$ is irrational.
\end{proof}

\begin{theorem}
  Let $a, b \in \R$. Then $a = b$ if and only if $(\forall \epsilon > 0)[\abs{a - b} < \epsilon]$.
\end{theorem}

\begin{proof}
  Let $\epsilon > 0$. Assume that $a = b$. Then $\abs{a - b} = 0 < \epsilon$.

  Assume $a \ne b$. Then $\abs{a - b} \ne 0$. Let $\epsilon = \abs{a - b} > 0$. Then $\abs{a - b} = \epsilon < \epsilon$, which is a contradiction, implying that $a = b$.
\end{proof}

\subsection{Induction}

Let $S(n)$ be a statement. If $S(1)$ is true and $S(n) \implies S(n + 1)$ is true, $(\forall n \in \N)[S(n)]$.

\begin{question}
  Let $x_1 = 1$ and $x_{n+1} = \frac{1}{2} x_n + 1$. Show that $x_n$ is increasing, meaning, prove $(\forall n \in \N)[x_n \le x_{n+1}]$.
\end{question}

\begin{proof}[Solution]
  Let $P(n) : x_n \le x_{n+1}$.

  Base case: $x_1 = 1$ and $x_2 = \sfrac{3}{2} > x_1$.

  Induction step: Let $n$ be an arbitrary natural number and suppose that $x_{n+1} \ge x_n$. Then
  \begin{align*}
    x_{n+2} - x_{n+1} &= \frac{1}{2}(x_{n+1} + 1) - \frac{1}{2}(x_n + 1) \\
                      &= \frac{1}{2}(x_{n+1} - x_n) > 0
  .\end{align*}

  Therefore, by the principle of mathematical induction, $(\forall n \in \N)[x_n \le x_{n+1}]$.
\end{proof}

\begin{question}
  Show that $(\forall n \in \N)[3 \mid (n^3 + 2n)]$.
\end{question}

\begin{proof}[Solution]
  Let $P(n) : 3 \mid (n^3 + 2n)$.

  Base case: $1^3 + 2 \cdot 1 = 3$ is divisible by $3$.

  Induction step: Let $n$ be an arbitrary natural number and suppose that $n^3 + 2n$ is divisible by $3$. Then
  \begin{align*}
    (n + 1)^3 + 2(n + 1) &= n^3 + 3n^2 + 3n + 1 + 2(n + 1) \\
                         &= n^3 + 2n + 3(n^2 + n + 1)
  ,\end{align*}
  which is divisible by $3$, since both $n^3 + 2n$ and $3(n^2 + n + 1)$ are divisible by $3$.

  Therefore, by the principle of mathematical induction, $(\forall n \in \N)[3 \mid (n^3 + 2n)]$.
\end{proof}

\section{Axiom of Completeness}

The defining difference between $\R$ and $\Q$ is that $\R$ does not contain the gaps that permeate $\Q$. We will be precise about how we phrase this assumption. This is referred to as the \emph{Axiom of Completeness}.
\begin{frame}[Axiom of Completeness]
  \begin{center}
    Every nonempty set of real numbers that is bounded above has a least upper bound.
  \end{center}
\end{frame}

\begin{note}
  We will never be able to prove the Axiom of Completeness, as axioms are meant to be taken as true. However, this will be the only statement that we naively accept without any rigorous proof. Everything else will be proven from this axiom.
\end{note}

Now, let's break down the definition of the least upper bound.

\subsection{Least Upper Bound and Greatest Lower Bound}

\begin{definition}[Upper Bound]
  A set $A \subseteq \R$ is bounded above if $(\exists b \in \R)(\forall a \in A)[a \le b]$. The number $b$ is called the \emph{upper bound} for $A$.
\end{definition}

\begin{definition}[Least Upper Bound]
  A real number $s$ is the \emph{least upper bound} for a set $A \subseteq \R$ if it meets the following two criteria
  \begin{enumerate}
    \item $s$ is an upper bound for $A$.

    \item If $b$ is any upper bound for $A$, then $s \le b$.
  \end{enumerate}
\end{definition}

The least upper bound is also called the \emph{supremum} of the set $A$. The \emph{greatest lower bound}/\emph{infimum} for $A$ is also defined similarly (see the figure below for a graphical representation).
\begin{figure}[H]
  \centering

  \begin{tikzpicture}[scale=0.9]
    \draw[thick] (-7, 0) -- (7, 0);

    % lower bounds from -6.8 to -2.4
    \draw[decorate,decoration={mirror,brace,amplitude=5pt},thick] (-6.8, -0.4) -- (-2.4, -0.4) node[midway, below=5pt] {lower bounds};

    % The set A from -2.2 to 2.2
    \draw[thick,fill=black] (-2.2, 0) -- (-2.2, 0.15) -- (-1.2, 0.15) -- (-1.2, 0) -- cycle;
    \node[above=15pt] (Inf) at (-2.2,0.15) {$\inf(A)$};
    \node[above=1pt] (Box1) at (-2.2,0) {};
    \draw[arrows={->}] (Inf) edge (Box1);

    \draw[thick,fill=black] (-0.8, 0) -- (-0.8, 0.15) -- (0.8, 0.15) -- (0.8, 0) -- cycle;

    \draw[soldot] (2.2,0.07) circle (0.08);
    \draw[soldot] (2.1,0.07) circle (0.08);
    \draw[soldot] (1.95,0.07) circle (0.08);
    \draw[soldot] (1.75,0.07) circle (0.08);
    \draw[soldot] (1.5,0.07) circle (0.08);
    \draw[soldot] (1.2,0.07) circle (0.08);

    \node[above=15pt] (Sup) at (2.2,0.15) {$\sup(A)$};
    \node[above=1pt] (Box2) at (2.2,0) {};
    \draw [arrows={->}] (Sup) edge (Box2);

    \node[below=5pt] (A) at (0,-0.4) {$A$};
    \node[below=1pt] (O) at (0,0.2) {};
    \draw[arrows={->}] (A) edge (O);

    \node[below=1pt] (AE1) at (-1.7,0.2) {};
    \draw[->] (A.west) -- (AE1);

    \node[below=1pt] (AE2) at (1.7,0.2) {};
    \draw[->] (A.east) -- (AE2);

    % upper bounds from 2.2 to 6.8
    \draw[decorate,decoration={mirror,brace,amplitude=5pt},thick] (2.2, -0.4) -- (6.8, -0.4) node[midway, below=5pt] {upper bounds};
  \end{tikzpicture}
\end{figure}

\begin{notation}
  The supremum is denoted as $s = \sup(A)$ and the infimum is denoted as $s = \inf(A)$.
\end{notation}

\begin{claim}
  A set that's bounded can only have one least upper bound.
\end{claim}

\begin{proof}
  By property (ii) of the definition of the least upper bound, if $s$ and $t$ are both least upper bounds for $A$, then $s \le t$ and $t \le s$, implying that $s = t$.
\end{proof}

\begin{example}
  The closed interval $[a, b]$ has a supremum of $b$ and an infimum of $a$. So does the open interval $(a, b)$. But what's the difference between the two? We'll investigate that later on.
\end{example}

\begin{example}
  Let $A = \left.\left\{\frac{1}{n}~\right\rvert n \in \N\right\} = \left\{1, \frac{1}{2}, \frac{1}{3}, \dots\right\}$. It's clear that $\sup(A) = 1$ and $\inf(A) = 0$.
\end{example}

\subsection{Maximum and Minimum}

\begin{definition}[Maximum and Minimum]
  A real number $a_0$ is a \emph{maximum} of the set $A$ if $a_0$ is an element of $A$ and $(\forall a \in A)[a_0 \ge a]$. Similarly, $a_0$ is a \emph{minimum} of the set $A$ if $a_0$ is an element of $A$ and $(\forall a \in A)[a_0 \le a]$.
\end{definition}

The closed interval $[a, b]$ has a maximum of $b$ and a minimum of $a$, since they're both elements of the set. However, the open interval $(a, b)$ has neither a maximum nor a minimum, since $a$ and $b$ aren't in the set.

\begin{example}
  Let $A = \left.\left\{1 + \frac{1}{n}~\right\rvert n \in \N\right\}$, giving us $\sup(A) = 2 = \max(A)$, but $\inf(A) = 1 \ne \min(A)$, as it does not exist.
\end{example}

\begin{question}
  Let $A \subseteq \R$ be nonempty and bounded above, let $c \in \R$. Define the set $c + A = \left\{c + a \mid a \in A\right\}$. Then, $\sup(c + A) = c + \sup(A)$.
\end{question}

\begin{proof}[Solution]
  Let $s = \sup(A)$. Let $a \in A$. Since $s$ is an upper bound of $A$, then $a \le s \implies a + c \le s + c$ implies that $c + s$ is an upper bound of $c + A$. Then, $a \le s \implies a + c \le s + c \implies c + s$ is an upper bound of $c + A$. If $b$ is an upper bound of $A$, then $s \le b$. Now, let $d$ be an upper bound of $c + A$. Then, $c + a \le d \implies a \le d - c \iff d - c$ is an upper bound of $A$. Then,
  \begin{align*}
    \phantom{\implies}\quad&s = \sup(A) \le d - c \\
    \implies\quad&s + c \le d \\
    \implies\quad&s + c = \sup(c + A)
  .\qedhere\end{align*}
\end{proof}

\subsection{Proving the upper bound}

How do we actually show that $a$ is an upper bound for a set $A \subseteq \R$? We use the following
\begin{lemma}
  Assume $s \in \R$ is an upper bound for a set $A \subseteq \R$. Then, $s = \sup(A)$ if and only if, $(\forall \epsilon > 0)(\exists a \in A)[s - \epsilon < a]$.
\end{lemma}

\begin{proof}\leavevmode
  Assume $s = \sup(A)$. Let $\epsilon > 0$. Then $s - \epsilon < s$, implying that $s - \epsilon$ is not an upper bound for $A$. Then, there must exist some $a \in A$ such that $s - \epsilon < a$, because otherwise, $s - \epsilon$ would be an upper bound.

  Assume $b$ is an upper bound for $A$. Assume $b < s \iff s - b > 0$. Let $\epsilon = s - b$. Then, $(\exists a \in A)[(s - \epsilon < a \iff b < a)]$, which is a contradiction, since $b$ is an upper bound. Therefore, $s = \sup(A)$.
\end{proof}

\begin{question}
  Define $A = \left.\left\{1 - \frac{1}{n}~\right\rvert n \in \N\right\}$. Using the lemma, show that $\sup(A) = 1$.
\end{question}

\begin{proof}[Solution]
  Assume $s = \sup(A) = 1$. Then, $(\forall \epsilon > 0)(\exists a = 1 - \sfrac{1}{n} \in A)[s - \epsilon < a]$. Then, $1 - \epsilon < 1 - \sfrac{1}{n}$. This implies that $\epsilon > \sfrac{1}{n}$, which is true for all $n \in \N$. Therefore, $s = 1$.
\end{proof}
