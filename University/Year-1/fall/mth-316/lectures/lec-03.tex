\nte{Oct 4 2024 Fri (13:00:10)}{Axiom of Completeness}

\section{Logic}
\label{sec:logic}

The easiest proof to see as a beginner is the proof that $\sqrt{2}$ is
irrational.

\begin{theorem}
  \label{thm:sqrt2_is_irrational}

  There is no rational number whose square is $2$.
\end{theorem}

\begin{proof}
  \label{prf:sqrt2_is_irrational}

  Assume that $\sqrt{2}$ is rational, meaning that $\exists p, q \in \Z$. Assume
  that $\sfrac{p}{q}$ is in simpliest form, meaning $p$ and $q$ have no common
  factor. Then $\sqrt{2} = \sfrac{p}{q} \iff 2 = \sfrac{p^2}{q^2} \implies 2q^2
  = p^2$. This means $p^2$ is an even integer. Then $\exists r \in \N$ such that
  $p = 2r$, giving us $4r^2 = 2q^2$. This means $q^2$ is also an even integer,
  which is a contradiction, since $p$ and $q$ were assumed to be in simplest
  form. Therefore, $\sqrt{2}$ is irrational.
\end{proof}

\begin{theorem}
  \label{thm:a_eq_b_iff_a_minus_b_lt_epsilon}

  Let $a, b \in \R$. Then $a = b$ if and only if $\forall \epsilon > 0$, $\lvert
  a - b \rvert < \epsilon$.
\end{theorem}

\begin{proof}
  \label{prf:a_eq_b_iff_a_minus_b_lt_epsilon} $ $

  \begin{enumerate}
    \label{enum:a_eq_b_iff_a_minus_b_lt_epsilon_prf}

    \item[$\implies$:] Assume that $a = b$. Then $\forall \epsilon > 0$, $\lvert
      a - b \rvert = 0 < \epsilon$.

    \item[$\impliedby$:] Assume $a \ne b$. Then $\lvert a - b \rvert \ne 0$. Let
      $\epsilon = \lvert a - b \rvert > 0$. Then $\lvert a - b \rvert = \epsilon
      < \epsilon$, which is a contradiction, implying that $a = b$. \qedhere
  \end{enumerate}
\end{proof}

\subsection{Induction}
\label{sub_sec:induction}

Let $S(n)$ be a statement. If $S(1)$ is true and $S(n) \implies S(n + 1)$ is
true, then $S(n)$ is true, $\forall n \in \N$.

\begin{question}
  \label{qst:induction_1}

  Let $x_1 = 1$ and $x_{n+1} = \frac{1}{2} x_n + 1$. Show that $x_n$ is
  increasing ($x_n \le x_{n+1}$, $\forall n \in \N$).
\end{question}

\begin{solproof}
  \label{solprf:induction_1}

  Base case: $x_1 = 1$ and $x_2 = \sfrac{3}{2} > x_1$.

  Induction step: Let $n$ be an arbitrary natural number and suppose that
  $x_{n+1} \ge x_n$. Then
  \begin{align*}
    x_{n+2} - x_{n+1} &= \frac{1}{2}(x_{n+1} + 1) - \frac{1}{2}(x_n + 1) \\
                      &= \frac{1}{2}(x_{n+1} - x_n) > 0
  .\end{align*}
\end{solproof}

\begin{question}
  \label{qst:induction_2}

  Show that $n^3 + 2n$ is divisible by $3$, $\forall n \in \N$.
\end{question}

\begin{solproof}
  \label{solprf:induction_2}

  Base case: $1^3 + 2 \cdot 1 = 3$ is divisible by $3$.

  Induction step: Let $n$ be an arbitrary natural number and suppose that $n^3 +
  2n$ is divisible by $3$. Then
  \begin{align*}
    (n + 1)^3 + 2(n + 1) &= n^3 + 3n^2 + 3n + 1 + 2(n + 1) \\
                         &= n^3 + 2n + 3(n^2 + n + 1)
  ,\end{align*}
  which is divisible by $3$, since both $n^3 + 2n$ and $3(n^2 + n + 1)$ are
  divisible by $3$.
\end{solproof}

% subsection induction (end)

% section logic (end)

\section{Axiom of Completeness}
\label{sec:axiom_of_completeness}

The defining difference between $\R$ and $\Q$ is that $\R$ does not contain the
gaps that permeate $\Q$. We will be precise about how we phrase this assumption.
This is referred to as the \textbf{Axiom of Completeness}.

\begin{purpleframe}[Axiom of Completeness]
  \label{prpl:axiom_of_completeness}

  Every nonempty set of real numbers that is bounded above has a least upper
  bound.
\end{purpleframe}

\begin{note}
  \label{nte:proof_axiom_of_completeness}

  We will never be able to prove the Axiom of Completeness, as axioms are meant
  to be taken as true. However, this will be the only statement that we naively
  accept without any rigorous proof. Everything else will be proven from this
  axiom.
\end{note}

Now, let's break down the definition of the least upper bound.

\subsection{Least Upper Bound and Greatest Lower Bound}
\label{sub_sec:least_upper_bound_and_greatest_lower_bound}

\begin{definition}[Upper Bound]
  \label{def:upper_bound}

  A set $A \subseteq \R$ is bounded above if $\exists b \in \R$ such that $a \le
  b$, $\forall a \in A$. The number $b$ is called the \textbf{upper bound} for
  $A$.
\end{definition}

\begin{definition}[Least Upper Bound]
  \label{def:least_upper_bound}

  A real number $s$ is the \textbf{least upper bound} for a set $A \subseteq \R$
  if it meets the following two criteria
  \begin{enumerate}
    \label{enum:least_upper_bound_def}

    \item $s$ is an upper bound for $A$.

    \item If $b$ is any upper bound for $A$, then $s \le b$.
  \end{enumerate}
\end{definition}

The least upper bound is also called the \textbf{supremum} of the set $A$. The
\textbf{greatest lower bound}/\textbf{infimum} for $A$ is also defined similarly
(see \cref{fig:supremum_infimum_graphically} for a graphical representation).

\begin{notation}
  \label{nta:least_upper_bound}

  The supremum is denoted as $s = \sup(A)$ and the infimum is denoted as $s =
  \inf(A)$.
\end{notation}

\begin{claim}
  \label{clm:supremum_and_infimum_are_unique}

  A set that's bounded can only have one least upper bound.
\end{claim}

\begin{proof}
  \label{prf:supremum_and_infimum_are_unique}

  By property $\circled{2}$ from \cref{def:least_upper_bound}, if $s$ and $t$
  are both least upper bounds for $A$, then $s \le t$ and $t \le s$, implying
  that $s = t$.
\end{proof}

\begin{example}
  \label{exm:supremum_and_infimum_1}

  The closed interval $[a, b]$ has a supremum of $b$ and an infimum of $a$. So
  does the open interval $(a, b)$. But what's the difference between the two?
  We'll investigate that later on.
\end{example}

\begin{example}
  \label{exm:supremum_and_infimum_2}

  Let $\displaystyle A = \left\{\frac{1}{n} \mid n \in \N\right\} = \left\{1,
  \frac{1}{2}, \frac{1}{3}, \dots\right\}$. It's clear that $\sup(A) = 1$ and
  $\inf(A) = 0$.
\end{example}

% subsection least_upper_bound_and_greatest_lower_bound (end)

\subsection{Maximum and Minimum}
\label{sub_sec:maximum_and_minimum}

\begin{definition}[Maximum and Minimum]
  \label{def:maximum_and_minimum}

  A real number $a_0$ is a \textbf{maximum} of the set $A$ if $a_0$ is an
  element of $A$ and $a_0 \ge a$, $\forall a \in A$. Similarly, $a_0$ is a
  \textbf{minimum} of the set $A$ if $a_0$ is an element of $A$ and $a_0 \le a$,
  $\forall a \in A$.
\end{definition}

Take \cref{exm:supremum_and_infimum_1}. The closed interval $[a, b]$ has a
maximum of $b$ and a minimum of $a$, since they're both elements of the set.
However, the open interval $(a, b)$ has neither a maximum nor a minimum, since
$a$ and $b$ aren't in the set.

\begin{example}
  \label{exm:maximum_and_minimum}

  Let $\displaystyle A = \left\{1 + \frac{1}{n} \mid n \in \N\right\}$, giving
  us $\sup(A) = \max(A) = 2$, but $\inf(A) = 1 \ne \min(A)$, as it does not
  exist.
\end{example}

\begin{question}
  \label{qst:sup_c_plus_a_equals_c_plus_sup_a}

  Let $A \subseteq \R$ be nonempty and bounded above, let $c \in \R$. Define the
  set $\displaystyle c + A = \left\{c + a \mid a \in A\right\}$. Then, $\sup(c +
  A) = c + \sup(A)$.
\end{question}

\begin{solproof}
  \label{solprf:sup_c_plus_a_equals_c_plus_sup_a}

  Let $s = \sup(A)$. Since $s$ is an upper bound of $A$,
  \begin{align*}
    \forall a \in A, \quad a \le s &\implies a + c \le s + c \\
                                   &\implies c + s~\textrm{is an upper bound of $c + A$}
  .\end{align*}
  If $b$ is an upper bound of $A$, then $s \le b$. Now, let $d$ be an upper
  bound of $c + A$
  \begin{align*}
    \forall a \in A &\implies c + a \le d \\
                    &\implies a \le d - c \\
                    &\iff d - c~\textrm{is an upper bound of $A$}
  .\end{align*}
  Then,
  \begin{align*}
    \phantom{\implies}&s = \sup(A) \le d - c \\
    \implies&s + c \le d \\
    \implies&s + c = \sup(c + A)
  .\end{align*}
\end{solproof}

% subsection maximum_and_minimum (end)

\subsection{Proving the upper bound}
\label{sub_sec:proving_the_upper_bound}

How do we actually show that $a$ is an upper bound for a set $A \subseteq \R$?
We use the folloing
\begin{lemma}
  \label{lma:proving_upper_bound}

  Assume $s \in \R$ is an upper bound for a set $A \subseteq \R$. Then, $s =
  \sup(A)$ if and only if, $\forall \epsilon > 0$, $\exists a \in A$ satisfying
  $s - \epsilon < a$.
\end{lemma}

\begin{proof}
  \label{prf:proving_upper_bound}

  \begin{enumerate}
    \label{enum:proving_upper_bound_prf} $ $

    \item[$\implies$:] Assume $s = \sup(A)$. Let $\epsilon > 0$. Then $s -
      \epsilon < s$, implying that $s - \epsilon$ is not an upper bound for $A$.
      Then, there must exist some $a \in A$ such that $s - \epsilon < a$,
      because otherwise, $s - \epsilon$ would be an upper bound.

    \item[$\impliedby$:] Assume $b$ is an upper bound for $A$. Assume $b < s
      \iff s - b > 0$. Let $\epsilon = s - b$. Then, $\exists a \in A$ such that
      $s - \epsilon < a \iff b < a$, which is a contradiction, since $b$ is an
      upper bound. Therefore, $s = \sup(A)$. \qedhere
  \end{enumerate}
\end{proof}

\begin{question}
  \label{qst:proving_upper_bound}

  Define $A = \left\{1 - \frac{1}{n} \mid n \in \N\right\}$. Using
  \cref{lma:proving_upper_bound}, show that $\sup(A) = 1$.
\end{question}

\begin{solproof}
  \label{solprf:proving_upper_bound}

  Assume $s = \sup(A) = 1$. Then $\forall \epsilon > 0$, $\exists a = 1 -
  \sfrac{1}{n} \in A$ such that $s - \epsilon < a \iff 1 - \epsilon < 1 -
  \sfrac{1}{n} \iff 1 - (1 - \sfrac{1}{n}) < \epsilon \iff \sfrac{1}{n} <
  \epsilon$.
\end{solproof}

% subsection proving_the_upper_bound (end)

% section axiom_of_completeness (end)

\newpage
