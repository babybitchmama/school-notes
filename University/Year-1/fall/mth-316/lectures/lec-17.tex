\lecture{17}{Nov 8 2024 Fri (13:03:05)}{Limit Points, Open and Closed Sets}

\begin{definition}[$\epsilon$-Neighborhood]
  Let $x \in \R^n$ and let $\epsilon > 0$. The \textit{$\epsilon$-neighborhood} of $x$ is
  \[%
    V_\epsilon(x) = \{ y \in \R^n \mid \lvert y - x \rvert < \epsilon \}
  .\]%
  This is also referred to as the \textit{open ball} of radius $\epsilon$ centered at $x$.
\end{definition}

\begin{note}
  The $\epsilon$-neighborhood $V_\epsilon(x)$ consists of all points whose Euclidean distance from $x$ is \emph{strictly less} than $\epsilon$. Points exactly at distance $\epsilon$ are not included.
\end{note}

\subsection{Open Sets}

\begin{definition}[Open Set]
  A set $U \subseteq \R^n$ is \textit{open} if for every $x \in U$, there exists an $\epsilon > 0$ such that $V_\epsilon(x) \subseteq U$.
\end{definition}

In words: a set is open if every point of the set lies entirely in the interior of the set, in the sense that we can ``wiggle'' the point a little in any direction without leaving the set.

\begin{examples}\leavevmode
  \begin{enumerate}
    \item Consider the open interval $(a,b) \subset \R$. Let $x \in (a,b)$. The distances from $x$ to each endpoint are $x-a > 0$ and $b-x > 0$. Set $\epsilon = \min\{x-a, b-x\}$. Then $V_\epsilon(x) \subseteq (a,b)$.

    \item The entire space $\R^n$ is open: given $x \in \R^n$ and any $\epsilon > 0$, $V_\epsilon(x) \subseteq \R^n$ holds trivially.

    \item The empty set $\emptyset$ is open: the definition is vacuously satisfied because there is no $x$ to check.

    \item For any $x_0 \in \R^n$ and $r > 0$, the open ball $B_r(x_0) = \{x \in \R^n : \lvert x - x_0 \rvert < r\}$ is open. If $x \in B_r(x_0)$, then $d = r - \lvert x - x_0 \rvert > 0$. Taking $\epsilon = d$ ensures $V_\epsilon(x) \subseteq B_r(x_0)$. \qedhere
  \end{enumerate}
\end{examples}

\begin{theorem}\leavevmode
  \begin{enumerate}
    \item The union of any collection of open sets is open.

    \item The intersection of finitely many open sets is open.
  \end{enumerate}
\end{theorem}

\begin{proof}\leavevmode
  \begin{enumerate}
    \item Let $\{O_\lambda \mid \lambda \in \Lambda\}$ be a collection of open sets indexed by some set $\Lambda$. Let
    \[%
      O = \bigcup_{\lambda \in \Lambda} O_\lambda
    .\]%
    Suppose $x \in O$. Then, by the definition of union, $x$ belongs to at least one $O_{\lambda_0}$ for some $\lambda_0 \in \Lambda$. Because $O_{\lambda_0}$ is open, there exists $\epsilon > 0$ such that $V_\epsilon(x) \subseteq O_{\lambda_0}$. But $O_{\lambda_0} \subseteq O$, so we also have $V_\epsilon(x) \subseteq O$. Since $x$ was arbitrary, $O$ is open.

    \item Let $O_1, O_2, \dots, O_m$ be open sets and set $O = \bigcap_{i=1}^m O_i$. Take any $x \in O$. Then $x \in O_i$ for every $i$. Because each $O_i$ is open, there exists $\epsilon_i > 0$ such that $V_{\epsilon_i}(x) \subseteq O_i$. Let
    \[%
      \epsilon = \min\{\epsilon_1, \epsilon_2, \dots, \epsilon_m\}
    .\]%
    Since $\epsilon \leq \epsilon_i$ for each $i$, it follows that $V_\epsilon(x) \subseteq V_{\epsilon_i}(x) \subseteq O_i$ for all $i$. Hence $V_\epsilon(x) \subseteq O$, so $O$ is open. \qedhere
  \end{enumerate}
\end{proof}

\subsection{Closed Sets}

Closed sets are defined in terms of their complements.

\begin{definition}[Closed Set]
  A set $F \subseteq \R^n$ is \textit{closed} if its complement $\R^n \setminus F$ is open.
\end{definition}

This definition is purely topological: a set is closed precisely when everything \emph{outside} it forms an open set.

\begin{examples}\leavevmode
  \begin{enumerate}
    \item The closed interval $[a,b]$ is closed because its complement is $(-\infty,a) \cup (b,\infty)$, which is the union of two open sets.

    \item $\R^n$ is closed because its complement $\emptyset$ is open.

    \item $\emptyset$ is closed because its complement $\R^n$ is open.

    \item Any finite subset of $\R$ is closed. For example, $\{p\}$ is closed since $\R \setminus \{p\} = (-\infty, p) \cup (p, \infty)$ is open. By finite unions of closed sets being closed (a fact proved below), any finite set is closed. \qedhere
  \end{enumerate}
\end{examples}

\begin{theorem}\leavevmode
  \begin{enumerate}
    \item The intersection of any collection of closed sets is closed.

    \item The union of finitely many closed sets is closed.
  \end{enumerate}
\end{theorem}

\begin{proof}
  These are dual statements to the openness properties proved earlier. Let $\{F_\alpha \mid \alpha \in A\}$ be a collection of closed sets.
  \begin{enumerate}
    \item Consider $\bigcap_{\alpha \in A} F_\alpha$. Its complement is
    \[%
      \left( \bigcap_{\alpha \in A} F_\alpha \right)^c = \bigcup_{\alpha \in A} F_\alpha^c
    .\]%
    Each $F_\alpha^c$ is open (since $F_\alpha$ is closed), and the union of any collection of open sets is open. Hence the complement of $\bigcap_{\alpha \in A} F_\alpha$ is open, so the intersection itself is closed.

    \item Let $F_1, \dots, F_m$ be closed sets. Then
    \[%
      \left( \bigcup_{i=1}^m F_i \right)^c = \bigcap_{i=1}^m F_i^c
    .\]%
    Each $F_i^c$ is open, and the intersection of finitely many open sets is open. Therefore the complement of $\bigcup_{i=1}^m F_i$ is open, so the union is closed. \qedhere
  \end{enumerate}
\end{proof}

\subsection{Limit Points}

The idea of a \emph{limit point} (or \emph{accumulation point}) makes precise what it means for a set to have points ``arbitrarily close'' to a given point.

\begin{definition}[Limit Point]
  Let $E \subseteq \R^n$. A point $p \in \R^n$ is called a \textit{limit point} of $E$ if every $\epsilon$-neighborhood of $p$ contains at least one point $q \in E$ with $q \neq p$. Equivalently, $p$ is a limit point of $E$ if
  \[%
    (\forall \epsilon > 0)\big[(V_\epsilon(p) \setminus \{p\}) \cap E \neq \emptyset\big]
  .\]%
\end{definition}

In other words, $p$ is a limit point of $E$ if we can find points of $E$ \emph{distinct from $p$} that are as close to $p$ as we like.

\begin{examples}\leavevmode
  \begin{enumerate}
    \item Every point of the open interval $(0,1)$ is a limit point of $(0,1)$. This is because for any $x \in (0,1)$ and any $\epsilon > 0$, the neighborhood $V_\epsilon(x)$ contains other points of $(0,1)$ besides $x$.

    \item The point $0$ is a limit point of $(0,1)$ even though $0 \notin (0,1)$, since any neighborhood around $0$ contains points of $(0,1)$.

    \item Every rational number in $\R$ is a limit point of the set of rationals $\Q$, and every irrational number is also a limit point of $\Q$. This follows from the density of $\Q$ in $\R$.

    \item An isolated point, such as $5$ in the set $\{5\}$, is \emph{not} a limit point, because we can choose $\epsilon$ small enough so that $V_\epsilon(5)$ contains no points of the set other than $5$ itself. \qedhere
  \end{enumerate}
\end{examples}

\begin{theorem}
  A set $F \subseteq \R^n$ is closed if and only if it contains all its limit points.
\end{theorem}

\begin{proof}
  Suppose first that $F$ is closed. Let $p$ be a limit point of $F$. We must show $p \in F$. Assume for contradiction that $p \notin F$. Then $p \in F^c$, which is open because $F$ is closed. Since $F^c$ is open, there exists $\epsilon > 0$ such that $V_\epsilon(p) \subseteq F^c$. But then $V_\epsilon(p)$ contains no points of $F$, contradicting the fact that $p$ is a limit point of $F$. Thus $p \in F$.

  Conversely, suppose $F$ contains all its limit points. We will show $F^c$ is open. Let $p \in F^c$. If $p$ were a limit point of $F$, it would belong to $F$ by assumption, contradicting $p \in F^c$. Hence $p$ is not a limit point of $F$. By the definition of limit point, there exists $\epsilon > 0$ such that $V_\epsilon(p) \cap F = \emptyset$, which means $V_\epsilon(p) \subseteq F^c$. This shows $F^c$ is open, so $F$ is closed.
\end{proof}

\begin{definition}[Isolated Point]
  A point $x \in E$ is called an \textit{isolated point} of $E$ if it is not a limit point of $E$. Equivalently, there exists $\epsilon > 0$ such that $V_\epsilon(x) \cap E = \{x\}$.
\end{definition}

Thus, points of a set are either limit points or isolated points. A set with no isolated points is called \emph{perfect}, a concept we will return to in a future date.
