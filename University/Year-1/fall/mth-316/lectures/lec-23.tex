\lecture{23}{Nov 27 2024 Wed (13:03:15)}{Introduction to Continuous Functions}

Having developed the $\epsilon$--$\delta$ definition of functional limits, we can now define one of the most important concepts in analysis: \emph{continuity}. Intuitively, a function is continuous at a point if its value there matches the value approached by the function nearby -- that is, there are no jumps, holes, or sudden changes.

\begin{definition}[Continuity at a point]
  Let $A \subseteq \R$, $f : A \to \R$, and let $c \in A$.
  The function $f$ is \emph{continuous at $c$} if
  \[%
    \lim_{x \to c} f(x) = f(c)
  .\]%
  Equivalently, for every $\epsilon > 0$, there exists $\delta > 0$ such that for all $x \in A$,
  \[%
    |x - c| < \delta \quad \Rightarrow \quad |f(x) - f(c)| < \epsilon
  .\qedhere\]%
\end{definition}

\begin{remark}
  This definition uses the \emph{two-sided} limit at $c$. If $c$ is an endpoint of an interval, we use the appropriate one-sided limit.
\end{remark}

\begin{definition}[Continuity on a set]
  A function $f : A \to \R$ is \emph{continuous on $A$} if it is continuous at every $c \in A$.
\end{definition}

\subsubsection{Sequential Criterion for Continuity}

\begin{theorem}
  A function $f : A \to \R$ is continuous at $c \in A$ if and only if for every sequence $(x_n)$ in $A$ with $x_n \to c$, we have $f(x_n) \to f(c)$.
\end{theorem}

\begin{proof}
  This follows immediately from the sequential criterion for functional limits, applied to the condition $\lim_{x \to c} f(x) = f(c)$.
\end{proof}

\begin{example}
  Let $f(x) = 3x + 2$ on $\R$. For any $c \in \R$, $\lim_{x \to c} (3x + 2) = 3c + 2 = f(c)$, hence $f$ is continuous everywhere.
\end{example}

\begin{example}[A discontinuity]
  Let
  \[%
    f(x) =
    \begin{cases}
      x^2, & x \neq 1, \\
      3, & x = 1.
    \end{cases}
  .\]%
  Since $\lim_{x \to 1} x^2 = 1 \neq f(1)$, $f$ is not continuous at $x = 1$. The limit exists, so the discontinuity is \emph{removable}.
\end{example}

\begin{example}[Jump discontinuity]
  Let
  \[%
    f(x) =
    \begin{cases}
      1, & x > 0, \\
      0, & x \le 0.
    \end{cases}
  .\]%
  Then $\lim_{x \to 0^-} f(x) = 0$ and $\lim_{x \to 0^+} f(x) = 1$, which are not equal; $f$ is discontinuous at $0$.
\end{example}

\subsection{Algebra of Continuous Functions}

\begin{theorem}
  If $f$ and $g$ are continuous at $c$, then so are:
  \begin{enumerate}
    \item $f + g$ and $f - g$,

    \item $f g$,

    \item $\alpha f$ for any $\alpha \in \R$,

    \item $\frac{f}{g}$, provided $g(c) \neq 0$.
  \end{enumerate}
\end{theorem}

\begin{proof}
  Follows from the Algebraic Limit Theorem for functional limits and the definition of continuity.
\end{proof}

\subsection{Composition of Continuous Functions}

\begin{theorem}
  If $g$ is continuous at $c$ and $f$ is continuous at $g(c)$, then $f \circ g$ is continuous at $c$.
\end{theorem}

\begin{proof}
  Given $\epsilon > 0$, use the continuity of $f$ at $g(c)$ to find $\eta > 0$ such that $|y - g(c)| < \eta$ implies $|f(y) - f(g(c))| < \epsilon$. By continuity of $g$ at $c$, find $\delta > 0$ such that $|x - c| < \delta$ implies $|g(x) - g(c)| < \eta$. Combining these gives the result.
\end{proof}

\subsection{Polynomials and Rational Functions}

\begin{theorem}
  Every polynomial is continuous on $\R$. Every rational function is continuous on its domain.
\end{theorem}

\begin{proof}
  Polynomials are obtained from constant functions and the identity function by finitely many additions and multiplications, which preserve continuity. Rational functions are quotients of polynomials with nonzero denominator, and quotients preserve continuity when the denominator is nonzero.
\end{proof}
