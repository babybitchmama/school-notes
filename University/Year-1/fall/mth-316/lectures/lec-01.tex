\nte{Sep 30 2024 Mon (13:01:47)}{Set theory}

\section{Single Sets}
\label{sec:single_sets}

\begin{definition}
  A \textit{set} is a collection of objects, called \textit{elements} of the
  set.
\end{definition}

Let's define the following sets, as we'll be using them throughout the notes
\begin{align*}
  \N &= \{1, 2, 3, \ldots\} \\
  \N_0 &= \{0, 1, 2, 3, \ldots\} \\
  \Z &= \{\ldots, -2, -1, 0, 1, 2, \ldots\} \\
  \Q &= \left\{\frac{m}{n} \mid m \in \Z, n \in \N\right\} \\
  \R &= \textrm{real numbers}
.\end{align*}

\begin{note}
  We won't properly define $\R$, as we don't have the necessary tools to do so.
\end{note}

Let $S$ be a set. We have the following notation regarding sets
\begin{enumerate}
  \item $x \in S$: $x$ is an element of the set $S$.

  \item $A \cap B = \left\{x \mid x \in A \land x \in B\right\}$: the
    intersection of sets $A$ and $B$.

  \item $A \cup B = \left\{x \mid x \in A \lor x \in B\right\}$: the union of
    sets $A$ and $B$.

  \item $A \subseteq S$: $A$ is a subset of $S$.
\end{enumerate}

\begin{note}
  The notation $\land$ is the logical \textit{and} operator, and $\lor$ is the
  logical \textit{or} operator. Or in math is an inclusive or, meaning that the
  statement $A \lor B$ is true if at least one of $A$ or $B$ or both are true.
\end{note}

I'll be using the following notation throughout the notes
\begin{enumerate}
  \item $\forall x \in S$: for all $x$ in the set $S$.
  \item $\exists x \in S$: there exists an $x$ in the set $S$.
\end{enumerate}

So, for we can re-write all of the definitions as follows
\[%
  A \subseteq S \implies (\forall x) x \in A \implies x \in S
.\]%

A set $A$ is equal to another set $B$ if and only if $A \subseteq B$ and $B
\subseteq A$.

\begin{note}
  The symbol $\iff$ means \textit{if and only if}. The statement $A \iff B$ is
  shorthand notation for $(A \implies B) \land (B \implies A)$.
\end{note}

\begin{question}
  Using that statement, prove the following statement: $A \cap (B \cup C) = (A
  \cap B) \cup C$.
\end{question}

\begin{proof} $ $
  \begin{enumerate}
    \item[$\subseteq$:] Suppose $x \in A \cap (B \cup C)$. That means $(x \in A)
      \land (x \in B \cup C)$, implying $(x \in A) \land (x \in B \lor x \in
      C)$. This is equivalent to $x \in (A \cap B) \cup C$, since $x$ must be in
      $A$ and must be in either $B$ or $C$ (or both). If $x \in B$, then $x \in
      A \cup B \implies x \in (A \cap B) \cup C$ and if $x \in C$, then $x \in
      (A \cap B) \cup C$. Thus, $A \cap (B \cup C) \subseteq (A \cap B) \cup C$.

    \item[$\supseteq$:] Suppose $x \in (A \cap B) \cup C$. That means $(x \in A
      \land x \in B) \lor (x \in C)$. Notice that it isn't the same thing as the
      previous statement, as $x$ doesn't have to be in $A \cap B$ and can be in
      $C$. Example: $A = \{1, 2\}$, $B = \{1, 2\}$, and $C = \{1, 3\}$.
      Therefore, we get
      \[%
        A \cap (B \cup C) = \{1, 2\} \cap (\{1, 2\} \cup \{1, 2, 3\}) = \{1, 2\}
      ,\]%
      but
      \[%
        (A \cap B) \cup C = (\{1, 2\} \cap \{1, 2\}) \cup \{1, 3\} = \{1, 2, 3\}
      ,\]%
      meaning $(A \cap B) \cup C \not\subseteq A \cap (B \cup C)$.
  \end{enumerate}

  Therefore, $A \cap (B \cup C) \ne (A \cap B) \cup C$.
\end{proof}

\begin{definition}[Complement]
  Given $A \subseteq \R$, the \textit{complement} of $A$, written $A^c$, refers
  to the set of all elements of $\R$ not in $A$, written as
  \[%
    A^c = \left\{x \in \R \mid x \notin A\right\}
  .\]%
\end{definition}

This gives us our first
\begin{theorem}[DeMorgan's Law]
  Let $A$ and $B$ be sets. Then
  \begin{enumerate}
    \item $(A \cup B)^c = A^c \cap B^c$.

    \item $(A \cap B)^c = A^c \cup B^c$.
  \end{enumerate}
\end{theorem}

\begin{proof} $ $
  \begin{enumerate}
    \item Suppose $x \in (A \cap B)^c$, then $x \notin A \cap B$ so $x \notin A
      \lor x \notin B$, implying that $x \in A^c \lor x \in B^c$, which is
      equivalent to $x \in A^c \cup B^c$.

      Suppose $x \in A^c \cup B^c$, then $x \in A^c \lor x \in B^c$ so $x \notin
      A \lor x \notin B$, implying that $x \notin A \cap B$, which is equivalent
      to $x \in (A \cap B)^c$.

    \item Suppose $x \in (A \cup B)^c$, then $x \notin A \cup B$ so $x \notin A
      \land x \notin B$, implying that $x \in A^c \land x \in B^c$, which is
      equivalent to $x \in A^c \cap B^c$.

      Suppose $x \in A^c \cap B^c$, then $x \in A^c \land x \in B^c$ so $x
      \notin A \land x \notin B$, implying that $x \notin A \cup B$, which is
      equivalent to $x \in (A \cup B)^c$. \qedhere
  \end{enumerate}
\end{proof}

% section single_sets (end)

\section{Families of Sets}
\label{sec:families_of_sets}

We will, in the upcoming notes, be dealing with families of sets. A family of
sets is a set of sets.

\begin{example}
  Let $A_n = \left\{n, n + 1, n + 2, \ldots\right\}$, where $n \in \N$. The
  result is a nest chain of sets $A_1 \supseteq A_2 \supseteq A_3 \supseteq
  \ldots$, where each successive set is a subset of the previous set. The set
  $\{ A_n \}$ is a family of sets.
\end{example}

Notationally, we denote the union of a family of sets as
\[%
  \bigcup_{n=1}^{\infty} A_n, \quad \bigcup_{n \in \N} A_n, \oor A_1 \cup A_2 \cup A_3 \cup \ldots
.\]%
Similarly, we denote the intersection of a family of sets as
\[%
  \bigcap_{n=1}^{\infty} A_n, \quad \bigcap_{n \in \N} A_n, \oor A_1 \cap A_2 \cap A_3 \cap \ldots
.\]%

What exactly does this mean? Let's re-write the unions and intersections using
quantifiers
\begin{align}
  \begin{aligned}\label{eq:union_intersection_quantifiers}
    \empheq{
      x \in \bigcup_{n=1}^{\infty} A_n &\iff (\exists n \in \N)[x \in A_n] \\
      x \in \bigcap_{n=1}^{\infty} A_n &\iff (\forall n \in \N)[\exists x \in A_n]
    }
  \end{aligned}
\end{align}

Equation \ref{eq:union_intersection_quantifiers} will be very useful to prove
statements regarding unions and intersections.

% section families_of_sets (end)

\newpage
