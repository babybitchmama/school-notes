\lecture{7}{Oct 16 2024 Wed (13:01:48)}{Important Theorems on Limits}

\subsection{Algebraic Limit Theorem}

\begin{definition}[Bounded]
  A sequence $(x_n)$ is \emph{bounded} if there exists a number $M > 0$ such that $\lvert x_n \rvert \le M$ for all $n \in \N$.
\end{definition}

Geometrically, this means that we can find an interval $[-M, M]$ that contains every term in the sequence $(x_n)$.

\begin{theorem}
  Every convergent sequence is bounded.
\end{theorem}

\begin{worksheet}
  Given $(a_n)$, we need to show that if $\lim_{n \to \infty} a_n = a$, then $(a_n)$ is bounded. Using the definition of a limit, $(\forall \epsilon > 0)(\exists N \in \N)(\forall n > N)[\lvert a_n - a \rvert < \epsilon]$. Expanding the inequality using \ref{eq:triangle_inequality}, we get
  \[%
    \lvert a_n \rvert = \lvert a_n - a + a \rvert \le \lvert a_n + a \rvert - \lvert a \rvert < \epsilon + \lvert a \rvert
  .\]%
  We then can let $M$ be the biggest number in the set $\{\lvert a_1 \rvert, \lvert a_2 \rvert, \cdots, \lvert a_n \rvert, \lvert a \rvert + \epsilon\}$. Then, that means that every term in the sequence $(a_n)$ is less than or equal to $M$, so $(a_n)$ is bounded.
\end{worksheet}

\begin{proof}
  For $\epsilon > 1$, there exists an $N \in \N$ such that for all $n > N$, $\lvert a_n - a \rvert < \epsilon$. Then, by the triangle inequality, we have
  \[%
    \lvert a_n \rvert = \lvert a_n - a + a \rvert \le \lvert a_n - a \rvert + \lvert a \rvert < \epsilon + \lvert a \rvert
  .\]%
  We let $M = \max(\{\lvert a_1 \rvert, \lvert a_2 \rvert, \cdots, \lvert a_n \rvert, \lvert a \rvert + \epsilon \})$. Then, $(\forall n \in \N)[\lvert a_n \rvert \le M]$, implying that $(a_n)$ is bounded.
\end{proof}

\begin{theorem}[Algebraic Limit Theorem]
  Let $\lim_{n \to \infty} a_n = a$, and $\lim_{n \to \infty} b_n = b$. Then,
  \begin{enumerate}
    \item $\lim_{n \to \infty} (ca_n) = ca$, for all $c \in \R$.

    \item $\lim_{n \to \infty} (a_n + b_n) = a + b$.

    \item $\lim_{n \to \infty} (a_nb_n) = ab$.

    \item $\lim_{n \to \infty} \frac{a_n}{b_n} = \frac{a}{b}$, provided that $b
      \ne 0$.
  \end{enumerate}
\end{theorem}

\begin{proof}\leavevmode
  \begin{enumerate}
    \item Let $\epsilon > 0$. Then, there exists an $N \in \N$ such that for all $n > N$, $\lvert a_n - a \rvert < \frac{\epsilon}{\lvert c \rvert}$. Then, we have
      \[%
        \lvert ca_n - ca \rvert = \lvert c \rvert \cdot \lvert a_n - a \rvert < \lvert c \rvert \cdot \frac{\epsilon}{\lvert c \rvert} = \epsilon
      .\]%

    \item Let $\epsilon > 0$. Then, there exists an $N \in \N$ such that for all $n > N$, $\lvert a_n - a \rvert < \frac{\epsilon}{2}$ and $\lvert b_n - b \rvert < \frac{\epsilon}{2}$. Then, we have
      \[%
        \lvert (a_n + b_n) - (a + b) \rvert = \lvert (a_n - a) + (b_n - b) \rvert \le \lvert a_n - a \rvert + \lvert b_n - b \rvert < \frac{\epsilon}{2} + \frac{\epsilon}{2} = \epsilon
      .\]%

    \item Let $\epsilon > 0$. Observe the fact that
      \begin{align*}
        \lvert a_nb_n - ab \rvert &= \lvert a_nb_n - ab_n + ab_n - ab \rvert \\
                                  &\le \lvert a_nb_n - ab_n \rvert + \lvert ab_n - ab \rvert \\
                                  &= \lvert b_n \rvert \cdot \lvert a_n - a \rvert + \lvert a \rvert \cdot \lvert b_n - b \rvert
      .\end{align*}
      Since $(b_n)$ converges, that means it's bounded by some $M > 0$. Then, we have
      \[%
        \lvert b_n \rvert \cdot \lvert a_n - a \rvert + \lvert a \rvert \cdot \lvert b_n - b \rvert \le M \cdot \lvert a_n - a \rvert + \lvert a \rvert \cdot \lvert b_n - b \rvert
      .\]%
      Choose an $N_1$ and $N_2$ such that
      \[%
        (\forall n \ge N_1)\left[\lvert b_n - b \rvert < \frac{1}{\lvert a \rvert} \cdot \frac{\epsilon}{2}\right] \aand (\forall n \ge N_2)\left[\lvert a_n - a \rvert < \frac{1}{M} \cdot \frac{\epsilon}{2}\right]
      .\]%
      Pick $N = \max(\{N_1, N_2\})$. If $n \ge N$, then we get
      \begin{align*}
        \lvert a_nb_n - ab \rvert &< M \cdot \lvert a_n - a \rvert + \lvert a \rvert \cdot \lvert b_n - b \rvert \\
                                  &< M \left(\frac{\epsilon}{2M}\right) + \lvert a \rvert \cdot \left(\frac{\epsilon}{2 \lvert a \rvert}\right) = \epsilon
      .\end{align*}

    \item Using part (iii), we only need to show that $\lim_{n \to \infty} \frac{1}{b_n} = \frac{1}{b}$. Let $\epsilon > 0$. Observe the fact that
      \[%
        \left\lvert \frac{1}{b_n} - \frac{1}{b} \right\rvert = \frac{\lvert b - b_n \rvert}{\lvert b \rvert \cdot \lvert b_n \rvert}
      .\]%
      Choose $N_1$ and $N_2$ such that
      \[%
        (\forall n \ge N_1)\left[\lvert a_n - b \rvert < \frac{\lvert b \rvert}{2}\right] \aand (\forall n \ge N_2)\left[\lvert b_n - b \rvert < \frac{\epsilon \lvert b \rvert^2}{2}\right]
      .\]%
      Pick $N = \max(\{N_1, N_2\})$, then we get
      \[%
        \left\lvert \frac{1}{b_n} - \frac{1}{b} \right\rvert = \lvert b - b_n \rvert < \frac{\epsilon \lvert b \rvert^2}{2} \cdot \frac{1}{\lvert b \rvert \cdot \frac{\lvert b \rvert}{2}} = \epsilon
      .\qedhere\]%
  \end{enumerate}
\end{proof}

\subsection{Order Limit Theorem}

\begin{theorem}[Order Limit Theorem]
  Assume $\lim_{n \to \infty} a_n = a$ and $\lim_{n \to \infty} b_n = b$.
  \begin{enumerate}
    \item If $a_n \ge 0$ for all $n \in \N$, then $a \ge 0$.

    \item If $a_n \le b_n$ for all $n \in \N$, then $a \le b$.

    \item If there exists $c \in \R$ for which $c \le b_n$ for all $n \in \N$, then $c \le b$. Similarly, if $a_n \le c$ for all $n \in \N$, then $a \le c$.
  \end{enumerate}
\end{theorem}

\begin{proof}\leavevmode
  \begin{enumerate}
    \item Assume $a < 0$. Consider the particular value $\epsilon = \lvert a \rvert$. Then, we can find an $N$ such that $\lvert a_n - a \rvert < \lvert a \rvert$, for all $n \ge N$. This would mean that $\lvert a_N - a \rvert < \lvert a \rvert$, which implies that $a_N < 0$, which contradicts our hypothesis. Therefore, $a \ge 0$

    \item By the Algebraic Limit Theorem, the sequence $(b_n - a_n)$ converges to $b - a$. Because $b_n - a_n > 0$, we can apply part (i) to get $b - a \ge 0$, which implies that $a \le b$.

    \item Take $a_n = c$ or $b_n = c$ in part (ii) to get the desired result. \qedhere
  \end{enumerate}
\end{proof}

Limits and their properties don't depend on the first few terms of the sequence. Changing the value of the first ten or even ten thousand terms doesn't change the limit of the sequence.

\subsection{Monotone Convergence Theorem}

\begin{definition}[Monotone]
  A sequence $(a_n)$ is \emph{increasing} if $(\forall n \in \N)[a_n \le a_{n+1}]$ and \emph{decreasing} if $(\forall n \in \N)[a_n \ge a_{n+1}]$. A sequence is \emph{monotone} if it is either increasing or decreasing.
\end{definition}

\begin{theorem}[Monotone Convergence Theorem]
  Every bounded monotone sequence converges.
\end{theorem}

\begin{proof}
  Let $\left(a_n\right)$ be monotone and bounded. To prove $\left(a_n\right)$ converges using the definition of convergence, we are going to need a candidate for the limit. Let's assume the sequence is increasing (the decreasing case is handled similarly), and consider the set of points $\{a_n \mid n \in \N\}$. By assumption, this set is bounded, so we can let
  \[%
    s = \sup(\{a_n \mid n \in \N\})
  .\]%
  It seems reasonable to claim that $\lim a_n = s$.

  To prove this, let $\epsilon > 0$. Because $s$ is the least upper bound for $\{a_n \mid n \in \N\}$, $s - \epsilon$ is not an upper bound, so there exists a point in the sequence $a_N$ such that $s - \epsilon < a_N$. Now, the fact that $(a_n)$ is increasing implies that if $n \geq N$, then $a_N \leq a_n$. Hence,
  \[%
    s - \epsilon < a_N \leq a_n \leq s < s + \epsilon
  ,\]%
  which implies $\abs{a_n - s} <\epsilon$, as desired.
\end{proof}

We now have another tool to prove the Nested Interval Property.
\begin{theorem}[Nested Interval Property]
  For each $n \in \N$, assume we are given a closed interval $I_n = [a_n, b_n] = \left\{x \in \R \mid a_n \le x \le b_n\right\}$. Assume also that each $I_n$ contains the next interval, i.e., $I_n \supseteq I_{n+1}$. Then, the resulting sequence of closed intervals
  \[%
    I_1 \supseteq I_2 \supseteq I_3 \supseteq \cdots
  ,\]%
  has a nonempty intersection. That is,
  \[%
    \bigcap_{n=1}^\infty I_n \neq \emptyset
  .\]%
\end{theorem}

\begin{proof}
  Assume we have the following interval $I_n = [a_n, b_n]$ for each $n \in \N$. Since all $I_n \supseteq I_{n+1}$, then we have the following inequality
  \[%
    I_n = [a_n, b_n] \supseteq I_{n+1} = [a_{n+1}, b_{n+1}] \implies \begin{cases*}
      a_n \le a_{n+1} \\
      b_n \ge b_{n+1}
    \end{cases*}
  .\]%
  This means that $(a_n)$ is bounded by $b_1$ from above and $(b_n)$ is bounded by $a_1$ from below. Then, by the MCT, $(a_n)$ converges to $a$ and $(b_n)$ converges to $b$. Hence,
  \[%
    [a, b] \in \bigcap_{n=1}^\infty I_n
  .\qedhere\]%
\end{proof}

\begin{question}
  Let $S_1 = 1$ and $S_{n+1} = \sqrt{S_n + 1}$, for $n \ge 1$. Show $S_n$ converges and find the limit.
\end{question}

\begin{proof}[Solution]
  Let's find the first few terms of the sequence
  \[%
    S_1 = 1, \quad S_2 = \sqrt{2}, \quad S_3 = \sqrt{\sqrt{2} + 1}, \quad S_4 = \sqrt{\sqrt{\sqrt{2} + 1} + 1}
  .\]%
  Then, we get the following inequality
  \[%
    S_1 < S_2 < S_3 < S_4 < \cdots
  .\]%

  We can then use induction to show that $S_n < S_{n+1}$ for all $n \in \N$ and then show that the sequence is bounded. Then, we can conclude by the MTC that the sequence must converge.

  Using induction, assume $S_{n-1} \le S_n$. Now, we show that $S_n \le S_{n+1}$. We have
  \begin{align*}
    S_{n+1} - S_n &= \sqrt{S_n + 1} - \sqrt{S_{n-1} + 1} \\
                  &= \frac{(S_n + 1) - (S_{n-1} + 1)}{\sqrt{S_n + 1} + \sqrt{S_{n-1} + 1}} \\
                  &= \frac{S_n - S_{n-1}}{\sqrt{S_n + 1} + \sqrt{S_{n-1} + 1}} \ge 0
  .\end{align*}
  Therefore, $S_n \le S_{n+1}$ for all $n \in \N$, meaning that $S_n$ is increasing.

  We now show that $S_n$ is bounded. Assume $S_n \le 2$. Then $S_{n+1} = \sqrt{S_n + 1} \le \sqrt{2 + 1} = \sqrt{3} \le \sqrt{2}$.

  Hence, by induction, $S_n \le 2$ for all $n$. So, $(S_n)$ is an increasing bounded sequence. By MCT, $(S_n)$ converges to $L$.

  Now, all that's left is to find the limit of the sequence.
  \[%
    \lim_{n \to \infty} S_n = \lim_{n \to \infty} S_{n+1} = S \implies S = \sqrt{S + 1} \implies S^2 = S + 1
  .\]%
  Solving the equation using the quadratic equation, we get
  \[%
    S = \frac{1 \pm \sqrt{5}}{2}
  .\]%
  But, as we've seen before, the limit value of a sequence is always unique. So, how do we choose which one the limit is? We'll, it must be the positive term, since $S_n \ge 1$ for all $n \in \N$. Therefore, the limit of the sequence is
  \[%
    \lim_{n \to \infty} S_n = \frac{1 + \sqrt{5}}{2}
  .\qedhere\]%
\end{proof}
