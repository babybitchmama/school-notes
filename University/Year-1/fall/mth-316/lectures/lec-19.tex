\lecture{19}{Nov 15 2024 Fri (13:02:06)}{Perfect and Connected Sets}

\subsection{Perfect Sets}

We recall that a \emph{limit point} of a set $E \subseteq \R^n$ is a point $p$ such that every $\epsilon$-neighborhood of $p$ contains a point of $E$ different from $p$ itself (Lecture~14).

\begin{definition}[Perfect Set]
  A set $P \subseteq \R^n$ is \emph{perfect} if it is closed and every point of $P$ is a limit point of $P$.
\end{definition}

Equivalently, $P$ is perfect if $P = P'$, where $P'$ denotes the set of all limit points of $P$.

\begin{remark}
  A few basic observations, we notice that every perfect set is closed, by definition, perfect sets have no isolated points, and the closure of a perfect set is the set itself.
\end{remark}

\begin{example}\leavevmode
  \begin{enumerate}
    \item \textbf{Closed intervals:} $[0,1]$ is perfect. It is closed, and every point is a limit point: interior points because of nearby points on both sides, and endpoints because of points approaching from within the interval.

    \item \textbf{The Cantor set:} The standard middle-third Cantor set $C$ is closed, totally disconnected, and has no isolated points -- hence perfect.

    \item \textbf{Rationals in an interval:} $\Q \cap [0,1]$ is not perfect because it is not closed; its closure is $[0,1]$, which \emph{is} perfect.

    \item \textbf{Union with isolated points:} $[0,1] \cup \{2\}$ is not perfect because $2$ is isolated.

    \item The set
      \[%
        E = \{x \in [0,1] \mid x!\text{is irrational}\}
      ,\]%
      is perfect. It is closed (complement is $\Q \cap [0,1]$, which is open in the subspace topology) and every irrational in $[0,1]$ is a limit point because rationals and irrationals are both dense in $\R$.

    \item The set of all real numbers whose decimal expansions contain only the digits 4 and 7 is perfect. It is closed, and every point can be approached by other such numbers differing only far out in the decimal expansion. \qedhere
  \end{enumerate}
\end{example}

\begin{theorem}
  Every nonempty perfect set in $\R^n$ is uncountable.
\end{theorem}

\begin{proof}
  Suppose $P$ is perfect and countable, $P = \{p_1, p_2, p_3, \dots\}$. We inductively construct a nested sequence of closed balls
  \[%
    \overline{B}_1 \supseteq \overline{B}_2 \supseteq \cdots
  .\]%
  such that $\overline{B}_n$ contains points of $P$ but not $p_n$, and with diameters tending to zero. The Nested Interval Property implies the intersection contains exactly one point, but that point would have been excluded at some stage -- contradiction. Thus $P$ must be uncountable.
\end{proof}

\subsection{Connected Sets}

Intuitively, a set is connected if it is ``all in one piece'' and cannot be split into two nonempty parts that are apart from each other.

\begin{definition}[Connected Set]
  A set $E \subseteq \R^n$ is \emph{connected} if it cannot be expressed as the union $E = U \cup V$ of two disjoint nonempty sets $U$ and $V$ that are open in the relative topology on $E$.
\end{definition}

Equivalently: the only subsets of $E$ that are both open and closed (in the relative topology) are $\emptyset$ and $E$ itself.

\begin{example}\leavevmode
  \begin{enumerate}
    \item \textbf{Intervals:} Every interval in $\R$ (open, closed, half-open, bounded, or unbounded) is connected.

    \item \textbf{Disconnected union:} $(0,1) \cup (2,3)$ is disconnected: the pieces $(0,1)$ and $(2,3)$ are disjoint, nonempty, and relatively open.

    \item The set
      \[%
        E = \{(x, \sin(1/x)) \mid 0 < x \leq 1\}
      ,\]%
      is connected. The oscillations do not break the set into separated pieces; it is even path-connected.

    \item Adding the vertical segment $\{0\} \times [-1,1]$ to the set in the previous example produces a closed connected set -- it is the closure of the original connected set.

    \item The union of two connected sets that intersect is connected. For instance, $[0,2] \cup [2,3] = [0,3]$. \qedhere
  \end{enumerate}
\end{example}

\begin{theorem}[Characterization in $\R$]
  A subset $E \subseteq \R$ is connected if and only if it is an interval.
\end{theorem}

\begin{proof}
  Suppose $E$ is connected but not an interval. Then there exist $a < c < b$ with $a, b \in E$ but $c \notin E$. Let
  \[%
    U = E \cap (-\infty, c), \quad V = E \cap (c, \infty)
  .\]%
  These are nonempty, relatively open in $E$, disjoint, and their union is $E$, contradicting connectedness.

  If $E$ is an interval, any attempt to split it into two disjoint nonempty relatively open sets will fail because the completeness of $\R$ forces the two sets to touch, producing a common boundary point in $E$.
\end{proof}

\begin{theorem}\leavevmode
  \begin{enumerate}
    \item The continuous image of a connected set is connected.

    \item The closure of a connected set is connected.

    \item The union of connected sets with nonempty intersection is connected.
  \end{enumerate}
\end{theorem}

\begin{proof}\leavevmode
  \begin{enumerate}
    \item If $f: E \to \R^m$ is continuous and $f(E)$ were disconnected, the preimages of a separating pair of open sets would disconnect $E$ -- contradiction.

    \item Let $E$ be connected. If $\overline{E}$ were disconnected, the separation would pull back to a separation of $E$ by intersecting with open sets.

    \item If $E_1$ and $E_2$ are connected and intersect, their union cannot be split without breaking one of them. \qedhere
  \end{enumerate}
\end{proof}

%%%%%%%%%%%%%%%%%%%%%%%%%%%%%%%%%%%
% Definition of Functional Limits %
%%%%%%%%%%%%%%%%%%%%%%%%%%%%%%%%%%%

\subsection{Functional Limits}

In Chapter 2, we studied limits of sequences. We now extend this concept to functions of a real variable. Intuitively, we say that $f(x)$ approaches $L$ as $x$ approaches $c$ if the values $f(x)$ can be made arbitrarily close to $L$ by taking $x$ sufficiently close to (but not equal to) $c$. This definition concerns the \emph{local behavior} of $f$ near $c$.

\begin{definition}[Limit of a function]
  Let $A \subseteq \R$ and let $f : A \to \R$. A point $c \in \R$ is a \emph{limit point} of $A$ if every neighborhood of $c$ contains a point of $A$ different from $c$ itself. We say that
  \[%
    \lim_{x \to c} f(x) = L
  ,\]%
  if for every $\epsilon > 0$ there exists $\delta > 0$ such that whenever $x \in A$ satisfies $0 < |x - c| < \delta$, we have $|f(x) - L| < \epsilon$.
\end{definition}

\begin{remark}
  The condition $0 < |x - c|$ ensures that the value $f(c)$ (if defined) is irrelevant to the limit; limits depend only on nearby values. The notation $\lim_{x \to c} f(x) = L$ asserts that the limit exists and equals $L$.
\end{remark}

\begin{example}
  Let $f(x) = 3x + 2$ on $\R$. We claim that $\lim_{x \to 1} f(x) = 5$. Given $\epsilon > 0$, choose $\delta = \epsilon/3$. If $0 < |x - 1| < \delta$, then
  \[%
    |f(x) - 5| = |3x + 2 - 5| = 3|x - 1| < 3 \cdot \frac{\epsilon}{3} = \epsilon
  .\]%
  Thus, the definition is satisfied and the limit is $5$.
\end{example}

\begin{example}
  Let
  \[%
    f(x) = \begin{cases}
      \frac{x^2 - 1}{x - 1} & x \neq 1 \\
      7 & x = 1
    \end{cases}
  .\]%
  For $x \neq 1$, $f(x) = x + 1$, so $\lim_{x \to 1} f(x) = 2$. The value $f(1) = 7$ is irrelevant to the limit.
\end{example}

\begin{example}
  Let $f(x) = \frac{\sin x}{x}$ for $x \neq 0$. Using the inequality $|\sin x| \le |x|$, we see that
  \[%
    \left| \frac{\sin x}{x} - 1 \right| = \left| \frac{\sin x - x}{x} \right| \to 0
  ,\]%
  as $x \to 0$, hence $\lim_{x \to 0} \frac{\sin x}{x} = 1$.
\end{example}

\subsection{Basic Limit Laws}

\begin{theorem}[Limit laws]
  Suppose $\lim_{x \to c} f(x) = L$ and $\lim_{x \to c} g(x) = M$. Then:
  \begin{enumerate}
    \item $\lim_{x \to c} [f(x) + g(x)] = L + M$,

    \item $\lim_{x \to c} [f(x) - g(x)] = L - M$,

    \item $\lim_{x \to c} [f(x) g(x)] = LM$,

    \item If $M \neq 0$, then $\lim_{x \to c} \frac{f(x)}{g(x)} = \frac{L}{M}$.
  \end{enumerate}
\end{theorem}

\begin{proof}
  We prove (i); the others are similar. Let $\epsilon > 0$ be given. Choose $\delta_1 > 0$ such that $0 < |x - c| < \delta_1$ implies $|f(x) - L| < \epsilon/2$, and choose $\delta_2 > 0$ such that $0 < |x - c| < \delta_2$ implies $|g(x) - M| < \epsilon/2$. Let $\delta = \min\{\delta_1, \delta_2\}$. Then for $0 < |x - c| < \delta$,
  \[%
    |[f(x) + g(x)] - (L + M)| \le |f(x) - L| + |g(x) - M| < \frac{\epsilon}{2} + \frac{\epsilon}{2} = \epsilon
  .\qedhere\]%
\end{proof}

\subsection{One-Sided Limits}

\begin{definition}[One-sided limits]
  Let $c$ be a limit point of $A \subseteq \R$ from the left or right.
  \begin{itemize}
    \item The \emph{left-hand limit} $\lim_{x \to c^-} f(x) = L$ means that the $\epsilon$--$\delta$ condition holds for all $x < c$ sufficiently close to $c$.

    \item The \emph{right-hand limit} $\lim_{x \to c^+} f(x) = L$ means it holds for all $x > c$ sufficiently close to $c$.
  \end{itemize}
  The two-sided limit $\lim_{x \to c} f(x)$ exists if and only if both one-sided limits exist and are equal.
\end{definition}

\begin{example}
  Let $f(x) = \frac{|x|}{x}$ for $x \neq 0$. Then $\lim_{x \to 0^-} f(x) = -1$ and $\lim_{x \to 0^+} f(x) = 1$. Since these are unequal, $\lim_{x \to 0} f(x)$ does not exist.
\end{example}
