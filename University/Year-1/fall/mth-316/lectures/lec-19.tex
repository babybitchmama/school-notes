\lecture{19}{Nov 15 2024 Fri (13:02:06)}{Perfect and Connected Sets}

\subsection{Perfect Sets}

We recall that a \emph{limit point} of a set $E \subseteq \R^n$ is a point $p$ such that every $\epsilon$-neighborhood of $p$ contains a point of $E$ different from $p$ itself (Lecture~14).

\begin{definition}[Perfect Set]
  A set $P \subseteq \R^n$ is \emph{perfect} if it is closed and every point of $P$ is a limit point of $P$.
\end{definition}

Equivalently, $P$ is perfect if $P = P'$, where $P'$ denotes the set of all limit points of $P$.

\begin{remark}
  A few basic observations, we notice that every perfect set is closed, by definition, perfect sets have no isolated points, and the closure of a perfect set is the set itself.
\end{remark}

\begin{examples}\leavevmode
  \begin{enumerate}
    \item $[0,1]$ is perfect. It is closed, and every point is a limit point: interior points because of nearby points on both sides, and endpoints because of points approaching from within the interval.

    \item The standard middle-third Cantor set $C$ is closed, totally disconnected, and has no isolated points -- hence perfect.

    \item $\Q \cap [0,1]$ is not perfect because it is not closed; its closure is $[0,1]$, which \emph{is} perfect.

    \item $[0,1] \cup \{2\}$ is not perfect because $2$ is isolated.

    \item The set
      \[%
        E = \{x \in [0,1] \mid x!\text{is irrational}\}
      ,\]%
      is perfect. It is closed (complement is $\Q \cap [0,1]$, which is open in the subspace topology) and every irrational in $[0,1]$ is a limit point because rationals and irrationals are both dense in $\R$.

    \item The set of all real numbers whose decimal expansions contain only the digits 4 and 7 is perfect. It is closed, and every point can be approached by other such numbers differing only far out in the decimal expansion. \qedhere
  \end{enumerate}
\end{examples}

\begin{theorem}
  A nonempty perfect set is uncountable.
\end{theorem}

\begin{proof}
  If $P$ is perfect and nonempty, then it must be infinite because otherwise it would consist only of isolated points. Let's assume, for contradiction, that $P$ is countable. Thus, we can write
  \[%
    P=\left\{x_1, x_2, x_3, \cdots\right\}
  ,\]%
  where every element of $P$ appears on this list. The idea is to construct a sequence of nested compact sets $K_n$, all contained in $P$, with the property that
  $x_1 \notin K_2, x_2 \notin K_3, x_3 \notin K_4, \cdots$. We can use the Nested
  Compact Set Property to produce an
  \[%
    x \in \bigcap_{n=1}^{\infty} K_n \subseteq P
  ,\]%
  that cannot be on the list $\left\{x_1, x_2, x_3, \cdots\right\}$.

  Let $I_1$ be a closed interval that contains $x_1$ in its interior (i.e., $x_1$ is not an endpoint of $I_1$ ). Now, $x_1$ is not isolated, so there exists some other point $y_2 \in P$ that is also in the interior of $I_1$. Construct a closed interval $I_2$, centered on $y_2$, so that $I_2 \subseteq I_1$ but $x_1 \notin I_2$. More explicitly, if $I_1=[a, b]$, let
  \[%
    \epsilon = \min(\{y_2 - a, b - y_2, |x_1 - y_2|\})
  .\]%

  Then, the interval $I_2 = [y_2 - \epsilon/2, y_2 + \epsilon/2]$ has the desired properties.

  This process can be continued. Because $y_2 \in P$ is not isolated, there must exist another point $y_3 \in P$ in the interior of $I_2$, and we may insist that $y_3 \neq x_2$. Now, construct $I_3$ centered on $y_3$ and small enough so that $x_2 \notin I_3$ and $I_3 \subseteq I_2$. Observe that $I_3 \cap P \neq \emptyset$ because this intersection contains at least $y_3$.

  If we carry out this construction inductively, the result is a sequence of closed intervals $I_n$ satisfying
  \begin{enumerate}
    \item $I_{n+1} \subseteq I_n$,
    \item $x_n \notin I_{n+1}$, and
    \item $I_n \cap P \neq \emptyset$.
  \end{enumerate}

  To finish the proof, we let $K_n = I_n \cap P$. For each $n \in \N$, we have that $K_n$ is closed because it is the intersection of closed sets, and bounded because it is contained in the bounded set $I_n$. Hence, $K_n$ is compact. By construction, $K_n$ is not empty and $K_{n+1} \subseteq K_n$. Thus, we can employ the Nested Compact Set Property to conclude that the intersection
  \[%
    \bigcap_{n=1}^{\infty} K_n \neq \emptyset
  .\]%

  But each $K_n$ is a subset of $P$, and the fact that $x_n \notin I_{n+1}$ leads to the conclusion that $\bigcap_{n=1}^{\infty} K_n = \emptyset$, which is the sought-after contradiction.
\end{proof}

\subsection{Connected Sets}

Intuitively, a set is connected if it is ``all in one piece'' and cannot be split into two nonempty parts that are apart from each other.

\begin{definition}[Connected Set]
  A set $E \subseteq \R^n$ is \emph{connected} if it cannot be expressed as the union $E = U \cup V$ of two disjoint nonempty sets $U$ and $V$ that are open in the relative topology on $E$.
\end{definition}

Equivalently: the only subsets of $E$ that are both open and closed (in the relative topology) are $\emptyset$ and $E$ itself.

\begin{examples}\leavevmode
  \begin{enumerate}
    \item If we let $A=(1,2)$ and $B=(2,5)$, then it is not difficult to verify that $E=(1,2) \cup(2,5)$ is disconnected. Notice that the sets $C=(1,2]$ and $D=(2,5)$ are not separated because $C \cap \bar{D}=\{2\}$ is not empty. This should be comforting. The union $C \cup D$ is equal to the interval $(1,5)$, which better not qualify as a disconnected set. We will prove in a moment that every interval is a connected subset of $\R$ and vice versa.

    \item Let's show that the set of rational numbers is disconnected. If we let
      \[%
        A = \Q \cap(-\infty, \sqrt{2}) \aand B = \Q \cap(\sqrt{2}, \infty)
      ,\]%
      then we certainly have $\Q = A \cup B$. The fact that $A \subseteq(-\infty, \sqrt{2})$ implies (by the Order Limit Theorem) that any limit point of $A$ will necessarily fall in $(-\infty, \sqrt{2}]$. Because this is disjoint from $B$, we get $\bar{A} \cap B = \emptyset$. We can similarly show that $A \cap \bar{B} = \emptyset$, which implies that $A$ and $B$ are separated.
  \end{enumerate}
\end{examples}

\begin{theorem}
  A set $E \subseteq \R$ is connected if and only if, for all nonempty disjoint sets $A$ and $B$ satisfying $E = A \cup B$, there always exists a convergent sequence $(x_n) \to x$ with $(x_n)$ contained in one of $A$ or $B$, and $x$ an element of the other.
\end{theorem}

\begin{proof}
  We use the usual topological definition: $E$ is \emph{connected} iff there do not exist nonempty disjoint sets $U,V$ that are open in the subspace topology of $E$ and satisfy $E = U\cup V$.

  \begin{enumerate}
    \item[$\implies$] If $E$ is connected then the sequence property holds.

      Let $E$ be connected and suppose $E=A\cup B$ where $A$ and $B$ are nonempty and disjoint. We prove that either there is a sequence in $A$ converging to a point of $B$, or there is a sequence in $B$ converging to a point of $A$.

      Assume, toward a contradiction, that neither happens. That is,
      \[%
        \overline{A} \cap B = \emptyset \aand \overline{B} \cap A = \emptyset
      ,\]%
      where closures are taken in $\R$. (If there were a point $x \in \overline{A} \cap B$ then, by definition of closure, every neighborhood of $x$ meets $A$, so we could choose $x_n \in A \cap (x - 1/n, x + 1/n)$ and get $x_n \to x$ with $x \in B$, giving the desired sequence; the symmetric statement holds for $\overline{B} \cap A$.)

      From $\overline{A} \cap B = \emptyset$ we get $A \subseteq \R \setminus \overline{B}$, and since $\R \setminus \overline{B}$ is open in $\R$, the set
      \[%
        A = E \cap (\R \setminus \overline{B})
      ,\]%
      is open in the subspace topology of $E$. Similarly,
      \[%
        B = E \cap (\R \setminus \overline{A})
      ,\]%
      is open in $E$. Thus $A$ and $B$ are nonempty, disjoint, relatively open subsets of $E$ whose union is $E$. This contradicts the assumption that $E$ is connected. Therefore our assumption was false: at least one of $\overline{A} \cap B$ or $\overline{B} \cap A$ must be nonempty, which yields the required convergent sequence in one part converging to a point of the other.

    \item[$\impliedby$] If the sequence property holds then $E$ is connected.

      Assume the sequence property: every partition $E = A\cup B$ into nonempty disjoint sets forces a sequence contained in one part to converge to a point in the other part. Suppose, for contradiction, that $E$ is disconnected. Then there exist nonempty disjoint sets $U$ and $V$ which are open in the subspace topology of $E$ and satisfy $E = U \cup V$.

      We claim that $\overline{U} \cap V = \emptyset$ and $\overline{V} \cap U = \emptyset$. Indeed, take any $x \in V$. Since $V$ is open in $E$, there is an $\varepsilon > 0$ such that $(x - \varepsilon, x + \varepsilon) \cap E \subseteq V$. Thus $(x - \varepsilon, x + \varepsilon) \cap U = \emptyset$, so $x \notin \overline{U}$. As $x \in V$ was arbitrary we obtain $\overline{U} \cap V = \emptyset$. The same reasoning (swapping $U$ and $V$) gives $\overline{V} \cap U = \emptyset$.

      But $\overline{U} \cap V = \emptyset$ means there is no sequence from $U$ converging to any point of $V$, and $\overline{V} \cap U = \emptyset$ means there is no sequence from $V$ converging to any point of $U$. This contradicts the assumed sequence property applied to the partition $E = U \cup V$. Hence $E$ cannot be disconnected; that is, $E$ is connected.

      Combining the two directions proves the equivalence.\qedhere
  \end{enumerate}
\end{proof}

The concept of connectedness is more relevant when working with subsets of the plane and other higher-dimensional spaces. This is because, in $\R$, the connected sets coincide precisely with the collection of intervals (with the understanding that unbounded intervals such as $(-\infty, 3)$ and $[0, \infty)$ are included).

\begin{theorem}
  $A$ set $E \subseteq \R$ is connected if and only if whenever $a < c < b$ with $a, b \in E$, it follows that $c \in E$ as well.
\end{theorem}

\begin{proof}
  Assume $E$ is connected, and let $a, b \in E$ and $a<c<b$. Set
  \[%
    A = (-\infty, c) \cap E \aand B = (c, \infty) \cap E
  .\]%
  Because $a \in A$ and $b \in B$, neither set is empty and, neither set contains a limit point of the other. If $E = A \cup B$, then we would have that $E$ is disconnected, which it is not. It must then be that $A \cup B$ is missing some element of $E$, and $c$ is the only possibility. Thus, $c \in E$.

  Conversely, assume that $E$ is an interval in the sense that whenever $a, b \in E$ satisfy $a < c < b$ for some $c$, then $c \in E$. Our intent is to use the characterization of connected sets, so let $E = A \cup B$, where $A$ and $B$ are nonempty and disjoint. We need to show that one of these sets contains a limit point of the other. Pick $a_0 \in A$ and $b_0 \in B$, and, for the sake of the argument, assume $a_0 < b_0$. Because $E$ is itself an interval, the interval $I_0 = [a_0, b_0]$ is contained in $E$. Now, bisect $I_0$ into two equal halves. The midpoint of $I_0$ must either be in $A$ or $B$, and so choose $I_1 = [a_1, b_1]$ to be the half that allows us to have $a_1 \in A$ and $b_1 \in B$. Continuing this process yields a sequence of nested intervals $I_n = [a_n, b_n]$, where $a_n \in A, b_n \in B$, and the length $(b_n - a_n) \to 0$. The remainder of this argument should feel familiar. By the Nested Interval Property, there exists an
  \[%
    x \in \bigcap_{n=0}^{\infty} I_n
  ,\]%
  and it is straightforward to show that the sequences of endpoints each satisfy $\lim a_n = x$ and $\lim b_n = x$. But now $x \in E$ must belong to either $A$ or $B$, thus making it a limit point of the other. This completes the argument.
\end{proof}
