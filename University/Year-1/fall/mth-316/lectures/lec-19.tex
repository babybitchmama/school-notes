\lecture{19}{Nov 15 2024 Fri (13:02:06)}{Definition of Functional Limits}

% \subsection{Functional Limits}

% In Chapter 2, we studied limits of sequences. We now extend this concept to functions of a real variable. Intuitively, we say that $f(x)$ approaches $L$ as $x$ approaches $c$ if the values $f(x)$ can be made arbitrarily close to $L$ by taking $x$ sufficiently close to (but not equal to) $c$. This definition concerns the \emph{local behavior} of $f$ near $c$.

% \begin{definition}[Limit of a function]
%   Let $A \subseteq \R$ and let $f : A \to \R$. A point $c \in \R$ is a \emph{limit point} of $A$ if every neighborhood of $c$ contains a point of $A$ different from $c$ itself. We say that
%   \[%
%     \lim_{x \to c} f(x) = L
%   ,\]%
%   if for every $\epsilon > 0$ there exists $\delta > 0$ such that whenever $x \in A$ satisfies $0 < |x - c| < \delta$, we have $|f(x) - L| < \epsilon$.
% \end{definition}

% \begin{remark}
%   The condition $0 < |x - c|$ ensures that the value $f(c)$ (if defined) is irrelevant to the limit; limits depend only on nearby values. The notation $\lim_{x \to c} f(x) = L$ asserts that the limit exists and equals $L$.
% \end{remark}

% \begin{example}
%   Let $f(x) = 3x + 2$ on $\R$. We claim that $\lim_{x \to 1} f(x) = 5$. Given $\epsilon > 0$, choose $\delta = \epsilon/3$. If $0 < |x - 1| < \delta$, then
%   \[%
%     |f(x) - 5| = |3x + 2 - 5| = 3|x - 1| < 3 \cdot \frac{\epsilon}{3} = \epsilon
%   .\]%
%   Thus, the definition is satisfied and the limit is $5$.
% \end{example}

% \begin{example}
%   Let
%   \[%
%     f(x) = \begin{cases}
%       \frac{x^2 - 1}{x - 1} & x \neq 1 \\
%       7 & x = 1
%     \end{cases}
%   .\]%
%   For $x \neq 1$, $f(x) = x + 1$, so $\lim_{x \to 1} f(x) = 2$. The value $f(1) = 7$ is irrelevant to the limit.
% \end{example}

% \begin{example}
%   Let $f(x) = \frac{\sin x}{x}$ for $x \neq 0$. Using the inequality $|\sin x| \le |x|$, we see that
%   \[%
%     \left| \frac{\sin x}{x} - 1 \right| = \left| \frac{\sin x - x}{x} \right| \to 0
%   ,\]%
%   as $x \to 0$, hence $\lim_{x \to 0} \frac{\sin x}{x} = 1$.
% \end{example}

% \subsection{Basic Limit Laws}

% \begin{theorem}[Limit laws]
%   Suppose $\lim_{x \to c} f(x) = L$ and $\lim_{x \to c} g(x) = M$. Then:
%   \begin{enumerate}
%     \item $\lim_{x \to c} [f(x) + g(x)] = L + M$,

%     \item $\lim_{x \to c} [f(x) - g(x)] = L - M$,

%     \item $\lim_{x \to c} [f(x) g(x)] = LM$,

%     \item If $M \neq 0$, then $\lim_{x \to c} \frac{f(x)}{g(x)} = \frac{L}{M}$.
%   \end{enumerate}
% \end{theorem}

% \begin{proof}
%   We prove (i); the others are similar. Let $\epsilon > 0$ be given. Choose $\delta_1 > 0$ such that $0 < |x - c| < \delta_1$ implies $|f(x) - L| < \epsilon/2$, and choose $\delta_2 > 0$ such that $0 < |x - c| < \delta_2$ implies $|g(x) - M| < \epsilon/2$. Let $\delta = \min\{\delta_1, \delta_2\}$. Then for $0 < |x - c| < \delta$,
%   \[%
%     |[f(x) + g(x)] - (L + M)| \le |f(x) - L| + |g(x) - M| < \frac{\epsilon}{2} + \frac{\epsilon}{2} = \epsilon
%   .\qedhere\]%
% \end{proof}

% \subsection{One-Sided Limits}

% \begin{definition}[One-sided limits]
%   Let $c$ be a limit point of $A \subseteq \R$ from the left or right.
%   \begin{itemize}
%     \item The \emph{left-hand limit} $\lim_{x \to c^-} f(x) = L$ means that the $\epsilon$--$\delta$ condition holds for all $x < c$ sufficiently close to $c$.

%     \item The \emph{right-hand limit} $\lim_{x \to c^+} f(x) = L$ means it holds for all $x > c$ sufficiently close to $c$.
%   \end{itemize}
%   The two-sided limit $\lim_{x \to c} f(x)$ exists if and only if both one-sided limits exist and are equal.
% \end{definition}

% \begin{example}
%   Let $f(x) = \frac{|x|}{x}$ for $x \neq 0$. Then $\lim_{x \to 0^-} f(x) = -1$ and $\lim_{x \to 0^+} f(x) = 1$. Since these are unequal, $\lim_{x \to 0} f(x)$ does not exist.
% \end{example}
