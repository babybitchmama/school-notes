\lecture{8}{Oct 18 2024 Fri (13:02:05)}{Order Limit Theorem}

\begin{theorem}[Order Limith Theorem]
  Assume $\lim_{n \to \infty} a_n = a$ and $\lim_{n \to \infty} b_n = b$.
  \begin{enumerate}
    \item If $a_n \ge 0$ for all $n \in \N$, then $a \ge 0$.

    \item If $a_n \le b_n$ for all $n \in \N$, then $a \le b$.

    \item If there exists $c \in \R$ for which $c \le b_n$ for all $n \in \N$,
      then $c \le b$. Similarly, if $a_n \le c$ for all $n \in \N$, then $a \le
      c$.
  \end{enumerate}
\end{theorem}

\begin{proof} $ $
  \begin{enumerate}
    \item Assume $a < 0$. Consider the particular value $\epsilon = \lvert a
      \rvert$. Then, we can find an $N$ such that $\lvert a_n - a \rvert <
      \lvert a \rvert$, for all $n \ge N$. This would mean that $\lvert a_N - a
      \rvert < \lvert a \rvert$, which implies that $a_N < 0$, which contradicts
      our hypothesis. Therefore, $a \ge 0$

    \item By the Algebraic Limit Theorem, the sequence $(b_n - a_n)$ converges
      to $b - a$. Because $b_n - a_n > 0$, we can apply part (i) to get $b - a
      \ge 0$, which implies that $a \le b$.

    \item Take $a_n = c$ or $b_n = c$ in part (ii) to get the desired result.
      \qedhere
  \end{enumerate}
\end{proof}

Limits and their properties don't depend on the first few terms of the sequence.
Changing the value of the first ten or even ten thousand terms doesn't change
the limit of the sequence.
