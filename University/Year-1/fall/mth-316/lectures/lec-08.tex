\nte[Section 2.4]{Oct 22 2024 Mon (13:02:06)}{MCT and Infinite Series}

\section{Monotone Convergence Theorem}
\label{sec:monotone_convergence_theorem}

\begin{definition}[Monotone]
  A sequence $(a_n)$ is \textit{increasing} if $(\forall n \in \N)[a_n \le
  a_{n+1}]$ and \textit{decreasing} if $(\forall n \in \N)[a_n \ge a_{n+1}]$. A
  sequence is \textit{monotone} if it is either increasing or decreasing.
\end{definition}

\begin{theorem}[Monotone Convergence Theorem]
  Every bounded monotone sequence converges.
\end{theorem}

\begin{proof}

\end{proof}

We now have another tool to prove the Nested Interval Property.
\begin{theorem}[Nested Interval Property]
  For each $n \in \N$, assume we are given a closed interval $I_n = [a_n, b_n] =
  \left\{x \in \R \mid a_n \le x \le b_n\right\}$. Assume also that each $I_n$
  contains the next interval, i.e., $I_n \supseteq I_{n+1}$. Then, the resulting
  sequence of closed intervals
  \[%
    I_1 \supseteq I_2 \supseteq I_3 \supseteq \cdots
  ,\]%
  has a nonempty intersection. That is,
  \[%
    \bigcap_{n=1}^{\infty} I_n \neq \emptyset
  .\]%
\end{theorem}

\begin{proof}
  Assume we have the following interval $I_n = [a_n, b_n]$ for each $n \in \N$.
  Since all $I_n \supseteq I_{n+1}$, then we have the following inequality
  \[%
    I_n = [a_n, b_n] \supseteq I_{n+1} = [a_{n+1}, b_{n+1}] \implies \begin{cases*}
      a_n \le a_{n+1} \\
      b_n \ge b_{n+1}
    \end{cases*}
  .\]%
  This means that $(a_n)$ is bounded by $b_1$ from above and $(b_n)$ is bounded
  by $a_1$ from below. Then, by the MCT, $(a_n)$ converges to $a$ and $(b_n)$
  converges to $b$. Hence,
  \[%
    [a, b] \in \bigcap_{n=1}^{\infty} I_n
  .\qedhere\]%
\end{proof}

\begin{question}
  Let $S_1 = 1$ and $S_{n+1} = \sqrt{S_n + 1}$, for $n \ge 1$. Show $S_n$
  converges and find the limit.
\end{question}

\begin{proof}
  Let's find the first few terms of the sequence
  \[%
    S_1 = 1, \quad S_2 = \sqrt{2}, \quad S_3 = \sqrt{\sqrt{2} + 1}, \quad S_4 = \sqrt{\sqrt{\sqrt{2} + 1} + 1}
  .\]%
  Then, we get the following inequality
  \[%
    S_1 < S_2 < S_3 < S_4 < \cdots
  .\]%

  We can then use induction to show that $S_n < S_{n+1}$ for all $n \in \N$ and
  then show that the sequence is bounded. Then, we can conclude by the MTC that
  the sequence must converge.

  Using induction, assume $S_{n-1} \le S_n$. Now, we show that $S_n \le
  S_{n+1}$. We have
  \begin{align*}
    S_{n+1} - S_n &= \sqrt{S_n + 1} - \sqrt{S_{n-1} + 1} \\
                  &= \frac{(S_n + 1) - (S_{n-1} + 1)}{\sqrt{S_n + 1} + \sqrt{S_{n-1} + 1}} \\
                  &= \frac{S_n - S_{n-1}}{\sqrt{S_n + 1} + \sqrt{S_{n-1} + 1}} \ge 0
  .\end{align*}
  Therefore, $S_n \le S_{n+1}$ for all $n \in \N$, meaning that $S_n$ is
  increasing.

  We now show that $S_n$ is bounded. Assume $S_n \le 2$. Then $S_{n+1} =
  \sqrt{S_n + 1} \le \sqrt{2 + 1} = \sqrt{3} \le \sqrt{2}$.

  Hence, by induction, $S_n \le 2$ for all $n$. So, $(S_n)$ is an increasing
  bounded sequence. By MCT, $(S_n)$ converges to $L$.

  Now, all that's left is to find the limit of the sequence.
  \[%
    \lim_{n \to \infty} S_n = \lim_{n \to \infty} S_{n+1} = S \implies S = \sqrt{S + 1} \implies S^2 = S + 1
  .\]%
  Solving the equation using the quadratic equation, we get
  \[%
    S = \frac{1 \pm \sqrt{5}}{2}
  .\]%
  But, as we've seen before, the limit value of a sequence is always unique.
  So, how do we choose which one the limit is? We'll, it must be the positive
  term, since $S_n \ge 1$ for all $n \in \N$. Therefore, the limit of the
  sequence is
  \[%
    \lim_{n \to \infty} S_n = \frac{1 + \sqrt{5}}{2}
  .\qedhere\]%
\end{proof}

% section monotone_convergence_theorem (end)

\section{Infinite Series}
\label{sec:infinite_series}



% section infinite_series (end)

\newpage
