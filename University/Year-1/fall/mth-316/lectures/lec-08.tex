\lecture{8}{Oct 16 2024 Wed (13:02:05)}{Algebraic and Order Limit Theorems}

\subsection{Algebraic Limit Theorem}

\begin{definition}[Bounded]
  A sequence $(x_n)$ is \emph{bounded} if there exists a number $M > 0$ such that $|x_n| \le M$ for all $n \in \N$.
\end{definition}

Geometrically, this means that we can find an interval $[-M, M]$ that contains every term in the sequence $(x_n)$.

\begin{theorem}
  Every convergent sequence is bounded.
\end{theorem}

\begin{worksheet}
  Given $(a_n)$, we need to show that if $\lim_{n \to \infty} a_n = a$, then $(a_n)$ is bounded. Using the definition of a limit, $(\forall \epsilon > 0)(\exists N \in \N)(\forall n > N)[|a_n - a| < \epsilon]$. Expanding the inequality using \ref{eq:triangle_inequality}, we get
  \[%
    |a_n| = |a_n - a + a| \le |a_n + a| - |a| < \epsilon + |a|
  .\]%
  We then can let $M$ be the biggest number in the set $\{|a_1|, |a_2|, \cdots, |a_n|, |a| + \epsilon\}$. Then, that means that every term in the sequence $(a_n)$ is less than or equal to $M$, so $(a_n)$ is bounded.
\end{worksheet}

\begin{proof}
  For $\epsilon > 1$, there exists an $N \in \N$ such that for all $n > N$, $|a_n - a| < \epsilon$. Then, by the triangle inequality, we have
  \[%
    |a_n| = |a_n - a + a| \le |a_n - a| + |a| < \epsilon + |a|
  .\]%
  We let $M = \max(\{|a_1|, |a_2|, \cdots, |a_n|, |a| + \epsilon \})$. Then, $(\forall n \in \N)[|a_n| \le M]$, implying that $(a_n)$ is bounded.
\end{proof}

\begin{theorem}[Algebraic Limit Theorem]
  Let $\lim_{n \to \infty} a_n = a$, and $\lim_{n \to \infty} b_n = b$. Then,
  \begin{enumerate}
    \item $\lim_{n \to \infty} (ca_n) = ca$, for all $c \in \R$.

    \item $\lim_{n \to \infty} (a_n + b_n) = a + b$.

    \item $\lim_{n \to \infty} (a_nb_n) = ab$.

    \item $\lim_{n \to \infty} \frac{a_n}{b_n} = \frac{a}{b}$, provided that $b
      \ne 0$.
  \end{enumerate}
\end{theorem}

\begin{proof}\leavevmode
  \begin{enumerate}
    \item Let $\epsilon > 0$. Then, there exists an $N \in \N$ such that for all $n > N$, $|a_n - a| < \frac{\epsilon}{|c|}$. Then, we have
      \[%
        |ca_n - ca| = |c| \cdot |a_n - a| < |c| \cdot \frac{\epsilon}{|c|} = \epsilon
      .\]%

    \item Let $\epsilon > 0$. Then, there exists an $N \in \N$ such that for all $n > N$, $|a_n - a| < \frac{\epsilon}{2}$ and $|b_n - b| < \frac{\epsilon}{2}$. Then, we have
      \[%
        |(a_n + b_n) - (a + b)| = |(a_n - a) + (b_n - b)| \le |a_n - a| + |b_n - b| < \frac{\epsilon}{2} + \frac{\epsilon}{2} = \epsilon
      .\]%

    \item Let $\epsilon > 0$. Observe the fact that
      \begin{align*}
        |a_nb_n - ab| &= |a_nb_n - ab_n + ab_n - ab| \\
                                  &\le |a_nb_n - ab_n| + |ab_n - ab| \\
                                  &= |b_n| \cdot |a_n - a| + |a| \cdot |b_n - b|
      .\end{align*}
      Since $(b_n)$ converges, that means it's bounded by some $M > 0$. Then, we have
      \[%
        |b_n| \cdot |a_n - a| + |a| \cdot |b_n - b| \le M \cdot |a_n - a| + |a| \cdot |b_n - b|
      .\]%
      Choose an $N_1$ and $N_2$ such that
      \[%
        (\forall n \ge N_1)\left[|b_n - b| < \frac{1}{|a|} \cdot \frac{\epsilon}{2}\right] \aand (\forall n \ge N_2)\left[|a_n - a| < \frac{1}{M} \cdot \frac{\epsilon}{2}\right]
      .\]%
      Pick $N = \max(\{N_1, N_2\})$. If $n \ge N$, then we get
      \begin{align*}
        |a_nb_n - ab| &< M \cdot |a_n - a| + |a| \cdot |b_n - b| \\
                                  &< M \left(\frac{\epsilon}{2M}\right) + |a| \cdot \left(\frac{\epsilon}{2 |a|}\right) = \epsilon
      .\end{align*}

    \item Using part (iii), we only need to show that $\lim_{n \to \infty} \frac{1}{b_n} = \frac{1}{b}$. Let $\epsilon > 0$. Observe the fact that
      \[%
        \left|\frac{1}{b_n} - \frac{1}{b} \right\rvert = \frac{|b - b_n|}{|b| \cdot |b_n|}
      .\]%
      Choose $N_1$ and $N_2$ such that
      \[%
        (\forall n \ge N_1)\left[|a_n - b| < \frac{|b|}{2}\right] \aand (\forall n \ge N_2)\left[|b_n - b| < \frac{\epsilon |b|^2}{2}\right]
      .\]%
      Pick $N = \max(\{N_1, N_2\})$, then we get
      \[%
        \left|\frac{1}{b_n} - \frac{1}{b} \right\rvert = |b - b_n| < \frac{\epsilon |b|^2}{2} \cdot \frac{1}{|b| \cdot \frac{|b|}{2}} = \epsilon
      .\qedhere\]%
  \end{enumerate}
\end{proof}

\subsection{Order Limit Theorem}

\begin{theorem}[Order Limit Theorem]
  Assume $\lim_{n \to \infty} a_n = a$ and $\lim_{n \to \infty} b_n = b$.
  \begin{enumerate}
    \item If $a_n \ge 0$ for all $n \in \N$, then $a \ge 0$.

    \item If $a_n \le b_n$ for all $n \in \N$, then $a \le b$.

    \item If there exists $c \in \R$ for which $c \le b_n$ for all $n \in \N$, then $c \le b$. Similarly, if $a_n \le c$ for all $n \in \N$, then $a \le c$.
  \end{enumerate}
\end{theorem}

\begin{proof}\leavevmode
  \begin{enumerate}
    \item Assume $a < 0$. Consider the particular value $\epsilon = |a|$. Then, we can find an $N$ such that $|a_n - a| < |a|$, for all $n \ge N$. This would mean that $|a_N - a| < |a|$, which implies that $a_N < 0$, which contradicts our hypothesis. Therefore, $a \ge 0$

    \item By the Algebraic Limit Theorem, the sequence $(b_n - a_n)$ converges to $b - a$. Because $b_n - a_n > 0$, we can apply part (i) to get $b - a \ge 0$, which implies that $a \le b$.

    \item Take $a_n = c$ or $b_n = c$ in part (ii) to get the desired result. \qedhere
  \end{enumerate}
\end{proof}

Limits and their properties don't depend on the first few terms of the sequence. Changing the value of the first ten or even ten thousand terms doesn't change the limit of the sequence.
