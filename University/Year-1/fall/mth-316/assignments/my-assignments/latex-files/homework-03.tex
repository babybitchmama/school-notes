\begin{center}
  SECTION 2.2
\end{center}

\setcounter{chapter}{2}
\setcounter{section}{2}

\begin{exercise}[1]
  What happens if we reverse the order of the quantifiers in Definition 2.2.3?

  \textit{Definition:} A sequence $(x_n)$ \textit{verconges} to $x$ if
  \textit{there exists} an $\epsilon > 0$ such that  $N \in \N$ it is true that
  $n \ge N$ implies $\lvert x_n - x \rvert < \epsilon$.

  Give an example of a vercongent sequence. Is there an example of a vercongent
  sequence that is divergent? Can a sequence verconge to two different values?
  What exactly is being described in this strange definition?
\end{exercise}

\begin{proof}[Solution]
  A series $(x_n)$ \textit{verconges} to $x$ if $\lvert x_n - x \rvert$ is
  bounded. Meaning, the sequence is vercongent \textit{if and only if} it is
  bounded.
\end{proof}

\medskip

\begin{exercise}[2]
  Verify, using the definition of convergence of a sequence, that the following
  sequences converge to the proposed limit.
  \begin{enumerate}
    \item $\lim_{n \to \infty} \frac{2n + 1}{5n + 4} = \frac{2}{5}$.
    \item $\lim_{n \to \infty} \frac{2n^2}{n^3 + 3} = 0$.
    \item $\lim_{n \to \infty} \frac{\sin(n^2)}{\sqrt[3]{n}} = 0$.
  \end{enumerate}
\end{exercise}

The definition of convergence of a sequence is as follows
\[%
  \empheq{\lim_{n \to \infty} a_n = a \iff (\forall \epsilon > 0)(\exists N \in \N)(\forall n > N)[\lvert a_n - a \rvert < \epsilon]}
\]%

\begin{proof}[Solution to (i)]
  Given $\epsilon > 0$, choose
  \[%
    N = \frac{3}{25\epsilon} - \frac{4}{5}
  .\]%
  Suppose $n > N > 0$. Then,
  \[%
    \lvert a_n - a \rvert = \left\lvert \frac{2n + 1}{5n + 4} - \frac{2}{5} \right\rvert = \frac{3}{5(5n + 4)} < \frac{3}{5(5N + 4)} = \frac{3}{5\left(5\left(\frac{3}{25\epsilon} - \frac{4}{5}\right) + 4\right)} = \frac{3}{5\left(\frac{3}{5\epsilon}\right)} = \epsilon
  .\qedhere\]%
\end{proof}

\begin{proof}[Solution to (ii)]
  Given $\epsilon > 0$, choose
  \[%
    N = \frac{1}{\epsilon}
  .\]%
  Suppose $n > N > 0$. Then,
  \[%
    \lvert a_n - a \rvert = \left\lvert \frac{2n^2}{n^3 + 3} - 0 \right\rvert = \frac{2n^2}{n^3 + 3} < \frac{2n^2}{n^3} = \frac{1}{n} < \frac{1}{N} < \epsilon
  .\qedhere\]%
\end{proof}

\begin{proof}[Solution to (iii)]
  Given $\epsilon > 0$, choose
  \[%
    N = \frac{1}{\epsilon^3}
  .\]%
  Suppose $n > N > 0$. Then,
  \[%
    \lvert a_n - a \rvert = \left\lvert \frac{\sin(n^2)}{\sqrt[3]{n}} \right\rvert \le \frac{1}{\sqrt[3]{n}} < \frac{1}{\sqrt[3]{N}} < \epsilon
  .\qedhere\]%
\end{proof}

\medskip

\begin{exercise}[3]
  Describe what we would have to demonstrate in order to disprove each of the
  following statements.
  \begin{enumerate}
    \item At every college in the United States, there is a student who is at
      least seven feet tall.

    \item For all colleges in the United States, there exists a professor who
      gives every student a grade of either A or B.

    \item There exists a college in the United States where every student is at
      least six feet tall.
  \end{enumerate}
\end{exercise}

\begin{proof}[Solution to (i)]
  Find a college in the United States where all students are less than seven
  feet tall.
\end{proof}

\begin{proof}[Solution to (ii)]
  Find a college in the United States where every professor does not only give
  A's or B's.
\end{proof}

\begin{proof}[Solution to (iii)]
  Show that all colleges in the United States have at least one student that's
  shorter than six feet tall.
\end{proof}

\medskip

\begin{exercise}[4]
  Give an example of each or state that the request is impossible. For any that
  are impossible, give a compelling argument for why that is the case.
  \begin{enumerate}
    \item A sequence with an infinite number of ones that does not converge to
      one.

    \item A sequence with an infinite number of ones that converges to a limit
      not equal to one.

    \item A divergent sequence such that for every $n \in \N$ it is possible to
      find n consecutive ones somewhere in the sequence.
  \end{enumerate}
\end{exercise}

\begin{proof}[Solution to (i)]
  $a_n = (-1)^n$.
\end{proof}

\begin{proof}[Solution to (ii)]
  Impossible, if $\lim_{n \to \infty} a_n = a \ne 1$ then for any $n \ge N$ we
  can find a $n$ with $a_n = 1$, meaning $\lvert 1 - a \rvert < \epsilon$ is
  impossible.
\end{proof}

\begin{proof}[Solution to (iii)]
  $a_n = (1, 2, 1, 1, 3, 1, 1, 1, \dots)$.
\end{proof}

\medskip

\begin{exercise}[6]
  \textbf{Theorem 2.2.7 (Uniqueness of Limits).} \textit{The limit of a
  sequence, when it exists, must be unique.}

  Prove Theorem 2.27. To get started, assume $(a_n) \to a$ and also that $(a_n)
  \to b$. Now argue $a = b$.
\end{exercise}

\begin{proof}[Solution]
  Let $\epsilon > 0$. Since $(a_n) \to a$, then, by the definition of the limit,
  we get $\exists N_1~\st~\forall n > N \implies \lvert a_n - a \rvert <
  \sfrac{\epsilon}{2}$. We get a similar thing for $(a_n) \to b$, $\exists
  N_2~\st~\forall n > N \implies \lvert a_n - b \rvert < \sfrac{\epsilon}{2}$

  Let $N = \max(\{N_1, N_2\})$. Then, for all $n > N$, we get
  \[%
    \lvert a - b \rvert \le \lvert (a_n - a) + (a_n - b) \rvert \le \frac{\epsilon}{2} + \frac{\epsilon}{2} = \epsilon
  .\]%
  Therefore, $a = b$.
\end{proof}

\medskip

\begin{center}
  SECTION 2.3
\end{center}

\setcounter{section}{3}

\begin{exercise}[1]
  \begin{enumerate}
    \item If $(x_n) \to 0$, show that $(\sqrt{x_n}) \to 0$.
    \item If $(x_n) \to x$, show that $(\sqrt{x_n}) \to \sqrt{x}$.
  \end{enumerate}
\end{exercise}

\begin{proof}[Solution to (i)]
  Let $\epsilon > 0$. Since $(x_n) \to 0$, by the definition of convergence,
  there exists $N$ such that for all $n > N$, we get $\lvert x_n \rvert <
  \epsilon^2$. This means $x_n < \epsilon^2$ for a large enough $n$. Taking the
  square root of both sides we get $\sqrt{x_n} < \epsilon$ (this is allowed, as
  $x_n \ge 0 $).

  Thus, for all $n > N$, we get $\lvert \sqrt{x_n} - 0 \rvert < \epsilon$,
  showing that if $(x_n) \to 0$, then $(\sqrt{x_n}) \to 0$.
\end{proof}

\begin{proof}[Solution to (ii)]
  Let $\epsilon > 0$. Since $(x_n) \to x$, by the definition of convergence,
  there exists $N$ such that for all $n > N$, we get $\lvert x_n - x \rvert <
  \epsilon^2$. Multiplying by $(\sqrt{x_n} + \sqrt{x})$ gives $\lvert x_n - x
  \rvert < (\sqrt{x_n} + \sqrt{x})\epsilon$. Since $x_n$ is convergent, it is
  bounded $\lvert x_n \rvert \le M$ implying $\sqrt{\lvert x_n \rvert} \le
  \sqrt{M}$. This implies that
  \[%
    \lvert \sqrt{x_n} - \sqrt{x} \rvert \le \frac{\lvert x_n - x \rvert}{\sqrt{M} + \sqrt{x}} < \epsilon
  .\]%

  Thus, for all $n > N$, we get $\lvert \sqrt{x_n} - x \rvert < \epsilon$,
  showing that if $(x_n) \to x$, then $(\sqrt{x_n}) \to x$.
\end{proof}

\medskip

\begin{exercise}[3]
  \textbf{Theorem (Squeeze Theorem).} Show that if $x_n \le y_n \le z_n$ for all
  $n \in \N$, and if $\lim_{n \to \infty} x_n = z_n = l$, then $\lim_{n \to
  \infty} y_n = l$ as well.
\end{exercise}

\begin{proof}[Solution]
  Let $\epsilon > 0$. Since $\lim_{n \to \infty} x_n = z_n = l$, then there
  exists $N_1$, $N_2$ such that for all $n > N_1$, $\lvert x_n - l \rvert <
  \sfrac{\epsilon}{2}$ and for all $n > N_2$ $\lvert z_n - l \rvert <
  \sfrac{\epsilon}{2}$. Define $N = \max(\{N_1, N_2\})$. Then, for all $n > N$,
  we get $\lvert x_n - l \rvert < \sfrac{\epsilon}{2}$ and $\lvert z_n - l
  \rvert < \sfrac{\epsilon}{2}$.

  Using the triangle inequality, we get
  \[%
    \lvert x_n - z_n \rvert \le \lvert x_n - l \rvert + \lvert z_n - l \rvert < \frac{\epsilon}{2} + \frac{\epsilon}{2} = \epsilon
  .\]%

  Since $x_n \le y_n \le z_n$, it follows that
  \[%
    \lvert y_n - l \rvert \le \lvert y_n - x_n \rvert + \lvert x_n - l \rvert < \lvert z_n - x_n \rvert + \lvert x_n - l \rvert < \epsilon
  .\]%

  Thus, for all $n > N$, we get $\lvert y_n - l \rvert < \epsilon$, showing that
  if $\lim_{n \to \infty} x_n = z_n = l$ and $x_n \le y_n \le z_n$, then
  $\lim_{n \to \infty} y_n = l$.
\end{proof}
