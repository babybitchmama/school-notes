\setcounter{chapter}{2}
\setcounter{section}{7}

\begin{exercise}[4]
  Give an example of each or explain why the request is impossible referencing
  the proper theorem(s).
  \begin{enumerate}
    \item Two series $\sum x_n$ and $\sum y_n$ that both diverge but where $\sum
      x_ny_n$ converges.

    \item A convergent series $\sum x_n$ and a bounded sequence $(y_n)$ such
      that $\sum x_ny_n$ diverges.

    \item Two sequences $(x_n)$ and $(y_n)$ where $\sum x_n$ and $\sum (x_n +
      y_n)$ both converge, but $\sum y_n$ diverges.

    \item A sequence $(x_n)$ satisfying $0 \le x_n \le \sfrac{1}{n}$ where $\sum
      (-1)^nx_n$ diverges.
  \end{enumerate}
\end{exercise}

\begin{exersolution}[4]
  \begin{enumerate}
    \item Let $x_n = \sfrac{1}{n}$ and $y_n = \sfrac{1}{n}$. Then, $\sum x_n =
      \sum \sfrac{1}{n}$ and $\sum y_n = \sum \sfrac{1}{n}$ diverges, but $\sum
      x_ny_n = \sum \sfrac{1}{n^2}$ converges to $\frac{\pi^2}{6}$.

    \item Let $x_n = \sfrac{(-1)^n}{n}$ and $y_n = (-1)^n$. Then, $\sum x_n =
      \sum \frac{(-1)^n}{n}$ converges, and $(y_n)$ is bounded, but $\sum x_ny_n
      = \sum \sfrac{1}{n}$ diverges.

    \item Impossible, since the algebraic limit theorem for series tells us that
      $\sum (x_n + y_n) - \sum x_n = \sum y_n$ converges.

    \item The sequence
      \[%
        x_n = \begin{cases*}
          \sfrac{1}{n} & \textrm{if $n$ is even} \\
          0 & \textrm{if $n$ is odd}
        \end{cases*}
      \]%
      satisfies $0 \le x_n \le \sfrac{1}{n}$, but $\sum (-1)^nx_n$, since it's
      always bouncing between $0$ and $\sfrac{1}{n}$.
  \end{enumerate}
\end{exersolution}

\newpage

\begin{exercise}[7]
  \begin{enumerate}
    \item Show that if $a_n > 0$ and $\lim_{n \to \infty} (na_n) = l$ with $l
      \ne 0$, then the series $\sum a_n$ diverges.

    \item Assume $a_n > 0$ and $\lim_{n \to \infty} (n^2a_n)$ exists. Show that
      $\sum a_n$ converges.
  \end{enumerate}
\end{exercise}

\begin{exersolution}[7]
  \begin{enumerate}
    \item \begin{proof}
        Setting $\epsilon = \frac{|l|}{2}$, there exists $N \in \mathbb{N}$ such
        that for all $n \ge N$, we have
        \[%
          |n a_n - l| < \frac{|l|}{2} \implies \frac{|l|}{2} < n a_n < \frac{3|l|}{2}
        .\]%
        Therefore, for all $n \ge N$,
        \[%
          a_n > \frac{|l|}{2n}
        .\]%
        Now, since $\sum \frac{1}{n}$ (the harmonic series) diverges, and $a_n$
        is greater than a constant multiple of $\frac{1}{n}$ for large $n$, we
        conclude by the Comparison Test that $\sum a_n$ must also diverge.
      \end{proof}

    \item \begin{proof}
        Setting $\epsilon = \frac{\lvert l \rvert}{2}$, there exists $N \in
        \N$ such that for all $n \ge N$, we have
        \[%
          \lvert n^2a_n - l \rvert < \epsilon \implies \frac{\lvert l \rvert}{2n^2} < a_n < \frac{3\lvert l \rvert}{2n^2}
        .\]%
        If that's the case, then for all $n \ge N$, we have
        \[%
          a_n > \frac{\lvert l \rvert}{2n^2}
        .\]%
        Since $\sum \frac{1}{n^2}$ converges, and $a_n \ge 0$, for all $n$,
        then, we can use the limit comparison test to conclude that $\sum a_n$
        converges
        \[%
          \lim_{n \to \infty} \frac{\frac{\lvert l \rvert}{2n^2}}{\frac{1}{n^2}}
        .\]%
        Using the properties from the algebraic limit theorem, we have
        \[%
          \lim_{n \to \infty} \frac{\frac{\lvert l \rvert}{2n^2}}{\frac{1}{n^2}} = \lim_{n \to \infty} \frac{\lvert l \rvert}{2n^2} \cdot \frac{n^2}{1} = \lim_{n \to \infty} \frac{\lvert l \rvert}{2} = \frac{\lvert l \rvert}{2}
        .\]%
        Since $\frac{\lvert l \rvert}{2}$ is a positive constant, then $\sum
        a_n$ converges.
      \end{proof}
  \end{enumerate}
\end{exersolution}

\newpage

\begin{exercise}[8]
  Consider each of the following proportions. Provide short proofs for those
  that are true and counterexamples for any that are not.
  \begin{enumerate}
    \item If $\sum a_n$ converges absolutely, then $\sum a_n^2$ also converges
      absolutely.

    \item If $\sum a_n$ converges and $(b_n)$ converges, then $\sum a_nb_n$
      converges.

    \item If $\sum a_n$ converges conditionally, then $\sum a_n^2$ diverges.
  \end{enumerate}
\end{exercise}

\begin{exersolution}[8]
  \begin{enumerate}
    \item The statement is true.
      \begin{proof}
        If $\sum a_n$ converges absolutely, then $\sum |a_n|$ converges. Since
        $|a_n| \le |a_n|^2$ for all $n$, then by the comparison test, $\sum
        a_n^2$ converges absolutely.
      \end{proof}

    \item The statement is false.

      Counterexample: Let $\displaystyle a_n = \frac{(-1)^n}{\sqrt{n}}$, which
      converges conditionally by the alternating series test, and let $b_n =
      a_n$, which converges to $0$. Then
      \[%
        a_nb_n = \frac{(-1)^n}{\sqrt{n}} \cdot \frac{(-1)^n}{\sqrt{n}} = \frac{1}{n}
      .\]%
      Now, $\sum \frac{1}{n}$ diverges by the $p$-series test, as it is a
      $p$-series with $p = 1 \nless 1$. Therefore, $\sum a_nb_n$ diverges.

    \item The statement is true.
      \begin{proof}
        If $\sum a_n$ converges conditionally, then $\sum \lvert a_n \rvert$
        diverges. Since $a_n^2 \le \lvert a_n \rvert$ but $a_n^2$ does not tend
        to zero fast enough to allow $\sum a_n^2$ to converge when $\sum \lvert
        a_n \rvert$ diverges, $\sum a_n^2$ must diverge.
      \end{proof}
  \end{enumerate}
\end{exersolution}

\newpage

\setcounter{chapter}{3}
\setcounter{section}{2}

\begin{exercise}[1]
  \begin{enumerate}
    \item Where in the proof of Theorem 3.2.3 part (ii) does the assumption that
      the collection of open sets be finite get used?

    \item Give an example of a countable collection of open sets $\{O_1, O_2,
      O_3, \dots\}$ whose intersection $\displaystyle \bigcap_{n=1}^{\infty}
      O_n$ is closed, not empty, and not all of $\R$.
  \end{enumerate}
\end{exercise}

\begin{exersolution}[1]
  \begin{enumerate}
    \item Taking the $\min(\{\epsilon_1, \epsilon_2, \cdots, \epsilon_N\})$ is
      only possible for finite sets.

    \item Let $O_n = \left(-\sfrac{1}{n}, 1 + \sfrac{1}{n}\right)$. Then,
      $\displaystyle\bigcap_{n=1}^{\infty} O_n = [0, 1]$.
  \end{enumerate}
\end{exersolution}

\newpage

\begin{exercise}[3]
  Decide whether the following sets are open, closed, or neither. If a set is
  not open, find a point in the set for which there is no
  $\epsilon$-neighborhood contained in the set. If a set is not closed, find a
  limit point that is not contained in the set.
  \begin{enumerate}
    \item $\Q$.

    \item $\N$.

    \item $\{x \in \R \mid x \ne 0\}$.
  \end{enumerate}
\end{exercise}

\begin{exersolution}[3]
  \begin{enumerate}
    \item $\Q$ is neither open nor closed, as $(a, b) \subseteq \Q$ is
      impossible since $\Q$ contains no irrationals but $(a, b)$ does. It's not
      closed since every irrationals are limit points of $\Q$, as they can be
      reached as a limit of rational numbers. Take $\sqrt{2}$ for instance.

    \item $\N$ is not open since it doesn't have any boundaries. It is closed,
      since it doesn't have any limit points.

    \item $\{x \in \R \mid x \ne 0\}$ is open, since it's the complement of
      $\{0\}$, which is closed. It's not closed, since $0$ is a limit point of
      the set.
  \end{enumerate}
\end{exersolution}

\newpage

\begin{exercise}[6]
  Decide whether the following statements are true or false. Provide
  counterexamples for those that are false, and supply proofs for those that are
  true.
  \begin{enumerate}
    \item An open set that contains every rational number must necessarily be
      all of $\R$.

    \item The Nested Interval Property remains true if the term ``closed
      interval'' is replaced by ``closed set''.

    \item Every nonempty open set contains a rational number.

    \item Every bounded infinite closed set contains a rational number.
  \end{enumerate}
\end{exercise}

\begin{exersolution}[6]
  \begin{enumerate}
    \item The statement is false.

      Counterexample: Let $A = (-\infty, \sqrt{5}) \cup (\sqrt{5}, \infty)$.
      Then, $A$ is open and contains every rational number except for $\sqrt{5}$
      and it's not all of $\R$.

    \item The statement is false.

      Counterexample: Let $C_n = [n, \infty)$ is closed. It has $C_{n+1}
      \supseteq C_n$ and $C_n \ne \emptyset$ but $\bigcap_{n=1}^{\infty} C_n =
      \emptyset$.

    \item The statement is true.

      \begin{proof}
        Let $x \in A$. Since $A$ is open, we have $(a, b) \subseteq A$ with $x
        \in (a, b)$, and by the density of the rationals, there exists $q \in
        \Q$ such that $q \in (a, b) \subseteq A$.
      \end{proof}

    \item The statement is false.

      Counterexample: Let $A = \{\sfrac{1}{n} + \sqrt{2} \mid n \in \N\} \cup
      \{\sqrt{2}\}$. Then, $A$ is closed, bounded, and infinite, but contains no
      rational numbers.
  \end{enumerate}
\end{exersolution}
