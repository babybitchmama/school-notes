\setcounter{chapter}{3}
\setcounter{section}{3}

\begin{exercise}[5]
  \begin{enumerate}
    \item The arbitrary intersection of compact sets is compact.

    \item The arbitrary union of compact sets is compact.

    \item Let $A$ be arbitrary, and let $K$ be compact. Then, the intersection
      $A \cap K$ is compact.

    \item If $f_1 \supseteq F_2 \supseteq F_3 \supseteq \cdots$ is a nested
      sequence of nonempty closed sets, then the intersection
      $\displaystyle\bigcap_{n=1}^{\infty} F_n \ne \emptyset$.
  \end{enumerate}
\end{exercise}

\begin{exersolution}[5]
  \begin{enumerate}
    \item True, since the intersection will be closed and bounded.

    \item False, since $\displaystyle \bigcup_{n=1}^{\infty} [0, n]$ is
      unbounded.

    \item False, since $(0, 1] \cap [0, 1] = (0, 1]$ is not closed, as $0$ is a
      limit point of $(0, 1]$, since $\displaystyle\left.\left\{\frac{1}{n} \right\rvert n \in
      \N\right\}$ is a sequence that converges to $0$.

    \item False, since $\displaystyle \bigcap_{n=1}^{\infty} [n, \infty) =
      \emptyset$.
  \end{enumerate}
\end{exersolution}

\newpage

\begin{exercise}[9]
  Follow these steps to prove that being compact implies every open cover has a
  finite subcover.

  Assume $K$ is compact, and let $\{O_{\lambda} \mid \lambda \in \Lambda\}$. For
  contradiction, let's assume that no finite subcover exists. Let $I_0$ be a
  closed interval containing $K$.
  \begin{enumerate}
    \item Show that there exists a nested sequence of closed intervals $I_0
      \supseteq I_1 \supseteq I_2 \supseteq \cdots$ with the property that, for
      each $n$, $I_n \cap K$ cannot be finitely covered and $\lim_{n \to \infty}
      \lvert I_n \rvert = 0$.

    \item Argue that there exists an $x \in K$ such that $x \in I_n$ for all
      $n$.

    \item Because $x \in K$, there must exist an open set $O_{\lambda_0}$ from
      the original collection that contains $x$ as an element. Explain how this
      leads to the desired contradiction.
  \end{enumerate}
\end{exercise}

\begin{exersolution}[9]
  \begin{enumerate}
    \item Bisect $I_0$ into two intervals. Let $I_1$ be the interval where $I_1
      \cap K$ cannot be finitely covered. Repeating this process gets us
      $\displaystyle\lim_{n \to \infty} \lvert I_n \rvert = \lim_{n \to \infty}
      \lvert I_0 \rvert \left(\frac{1}{2}\right)^n = 0$.

    \item The nested compact set property $K_n = I_n \cap K$ gives
      $\displaystyle x \in \bigcap_{n=1}^{\infty} K_n$, meaning $x \in K$ and $x
      \in I_n$ for all $n$.

    \item Since $x \in O_{\lambda_0}$ and $\lvert I_n \rvert \to 0$ with $x \in
      I_n$ for all $n$, there exists an $N$ where $n > N$ implies $I_n \subseteq
      O_{\lambda_0}$ contradicting the assumption that $I_n \cap K$ cannot be
      finitely covered since $\{O_{\lambda_0}\}$ is a finite subcover for $I_n
      \cap K$.
  \end{enumerate}
\end{exersolution}

\newpage

\setcounter{chapter}{4}
\setcounter{section}{2}

\begin{exercise}[2]
  For each stated limit, find the largest possible $\delta$-neighborhood that is
  a proper response to the given $\epsilon$ challenge.
  \begin{enumerate}
    \item $\lim_{x \to 3} (5x - 6) = 9$, where $\epsilon = 1$.
    \item $\lim_{x \to 4} \sqrt{x} = 2$, where $\epsilon = 1$.
    \item $\lim_{x \to \pi} [[x]] = 3$, where $\epsilon = 1$.
    \item $\lim_{x \to \pi} [[x]] = 3$, where $\epsilon = 0.01$.
  \end{enumerate}
\end{exercise}

\begin{exersolution}[2]
  \begin{enumerate}
    \item The largest possible $\delta$-neighborhood is
      \[%
        \lvert (5x - 6) - 9 \rvert = \lvert 5x - 15 \rvert = 5 \cdot \lvert x - 3 \rvert < 5\delta \implies \delta = \frac{1}{5}
      .\]%

    \item Expanding $\lvert \sqrt{x} - 2 \rvert < 1$ gives us
      \[%
        1 < \sqrt{x} < 3 \implies \delta = 3
      .\]%

    \item To satisfy $\lvert [[x]] - 1 \rvert < 1$, we require
      \[%
        [[x]] = 3
      ,\]%
      which happens when $3 \le x < 4$. Thus,
      \[%
        \lvert x - \pi \rvert < \min(\{\pi - 3, 4 - \pi\}) \implies \delta = \min(\{\pi - 3, 4 - \pi\}) = \pi - 3 \approx 0.1416
      .\]%

    \item For $\epsilon = 0.01$, we still need $\lvert [[x]] - 3 \rvert < 0.01$.
      Since $[[x]]$ is piecewise constant, this requires $[[x] = 3$. This occurs
      only when $3 \le x < 4$. The analysis of $\lvert x - \pi \rvert < \delta$
      is the same as before
      \[%
        \delta = \min(\{\pi - 3, 4 - \pi\}) = \pi - 3 \approx 0.1416
      .\]%
  \end{enumerate}
\end{exersolution}

\newpage

\begin{exercise}[5]
  Use Definition 4.2.1 to supply a proper proof for the following limit
  statements
  \begin{enumerate}
    \item $\lim_{x \to 2} (3x + 4) = 10$.
    \item $\lim_{x \to 0} x^3 = 0$.
    \item $\lim_{x \to 2} (x^2 + x - 1) = 5$.
    \item $\lim_{x \to 3} \frac{1}{x} = \frac{1}{3}$.
  \end{enumerate}
\end{exercise}

\begin{exersolution}[5]
  \begin{enumerate}
    \item \begin{proof}
        Let $\epsilon > 0$. Define $\delta = 3\epsilon$. Then, for all $x \in
        A$, we get Let $\epsilon$ be an arbitrary positive number. Let $\delta =
        3\epsilon$. Let $x$ be arbitrary. Suppose $0 < \lvert x - 2 \rvert <
        \delta$ Multiplying both sides by $3$ gives us
        \[%
          3 \cdot \lvert x - 2 \rvert < 3\delta \implies \lvert 3x - 6 \rvert < 3 \cdot \frac{\epsilon}{3} \implies \lvert (3x + 4) - 10 \rvert < \epsilon
        .\]%
        Therefore, $0 < \lvert x - 2 \rvert < \delta \implies \lvert (3x + 4) -
        10 \rvert < \epsilon$.

        It follows that $(\forall \epsilon > 0)(\exists \delta > 0)(\forall x
        \in A)[0 < \lvert x - 2 \rvert < \delta \implies \lvert (3x + 4) - 10
        \rvert < \epsilon]$.
      \end{proof}

    \item \begin{proof}
        Let $\epsilon$ be an arbitrary positive number. Let $\delta =
        \epsilon^{\sfrac{1}{3}}$. Let $x$ be arbitrary. Suppose $0 < \lvert x
        \rvert < \delta$. Cubing both sides gives
        us
        \[%
          0 < (\lvert x \rvert)^3 < \delta^3 \implies \lvert x^3 - 0 \rvert < \epsilon
        .\]%
        Therefore, $0 < \lvert x - 0 \rvert < \delta \implies \lvert x^3 - 0
        \rvert < \epsilon$.

        It follows that $(\forall \epsilon >
        0)(\exists \delta > 0)(\forall x \in A)[0 < \lvert x - 0 \rvert < \delta
          \implies \lvert x^3 - 0 \rvert < \epsilon]$.
      \end{proof}

    \item \begin{proof}
        Let $\epsilon$ be an arbitrary positive number. Let $\delta = \min(\{1,
        \frac{\epsilon}{6}\})$. Let $x$ be arbitrary. Suppose $0 < \lvert x - 2
        \rvert < \delta$. Multiplying both sides by $6$ gives us $6 \cdot \lvert
        x - 2 \rvert < 6\delta$. Playing with the inequality on the left hand
        side gives us
        \[%
          6\delta > 6 \cdot \lvert x - 2 \rvert = \lvert 6x - 12 \rvert = \lvert (3 + 3)(x - 2) \rvert \ge \lvert (x + 3)(x - 2) \rvert
        .\]%
        Therefore, we get
        \[%
          \lvert (x + 3)(x - 2) \rvert < 6\delta \implies \lvert x^2 + x - 6 \rvert < 6 \cdot \frac{\epsilon}{6} \implies \lvert (x^2 + x - 1) - 5 \rvert < \epsilon
        .\]%
        Therefore $0 < \lvert x - 2 \rvert < 6 \implies \lvert (x^2 + x - 1) - 5
        \rvert < \epsilon$.

        It follows that $(\forall \epsilon > 0)(\exists \delta > 0)(\forall x
        \in A)[0 < \lvert x - 2 \rvert < \delta \implies \lvert (x^2 + x - 1) -
        5 \rvert < \epsilon]$.
      \end{proof}

    \item \begin{proof}
        Let $\epsilon$ be an arbitrary positive number. Let $\delta = \min(\{1,
        6\epsilon\})$. Let $x$ be arbitrary. Suppose $0 < \lvert x - 3 \rvert <
        \delta$. Dividing the middle inequality by $6$ gives us
        \[%
          \frac{\lvert x - 3 \rvert}{6} = \frac{\lvert x - 3 \rvert}{3 \cdot 2} \le \frac{\lvert x - 3 \rvert}{3 \cdot \lvert x \rvert} = \left\lvert \frac{1}{x} - \frac{1}{3} \right\rvert < \epsilon
        .\]%
        Therefore $0 < \lvert x - 3 \rvert < 6 \implies \left\lvert \frac{1}{x}
        - \frac{1}{3} \right\rvert < \epsilon$.

        It follows that $(\forall \epsilon > 0)(\exists \delta > 0)(\forall x
        \in A)[0 < \lvert x - 3 \rvert < \delta \implies \left\lvert \frac{1}{x}
        - \frac{1}{3} \right\rvert < \epsilon]$.
      \end{proof}
  \end{enumerate}
\end{exersolution}

\newpage

\begin{exercise}[6]
  \begin{enumerate}
    \item If a particular $\delta$ has been constructed as a suitable response
      to a particular $\epsilon$ challenge, then any smaller positive $\delta$
      will also suffice.

    \item $\lim_{x \to a} f(x) = L$ and $a$ happens to be in the domain of $f$,
      then $L = f(a)$.

    \item If $\lim_{x \to a} f(x) = L$, then $\lim_{x \to a} 3[f(x) - 2]^2 = 3(L
      - 2)^2$.

    \item If $\lim_{x \to a} f(x) = 0$, then $\lim_{x \to a} f(x)g(x) = 0$ for
      any function $g$ (with the domain equal to the domain of $f$).
  \end{enumerate}
\end{exercise}

\begin{exersolution}[6]
  \begin{enumerate}
    \item True, since $\lvert x - a \rvert < \delta_2 < \delta$.

    \item False. Counterexample: If $x = 0$, then $f(x) = 1$. Otherwise, $f(x) =
      0$. The definition of the limit of a function states that $\lvert x - a
      \rvert < \delta$ implies that $\lvert f(x) - L \rvert < \epsilon$ for all
      $x$ that's not equal to $a$.

    \item True, as you can use the Algebraic Limit Theorem.

    \item False. Counterexample: Given two functions $f(x) = x$ and $g(x) =
      \sfrac{1}{x}$, then the limit $\lim_{x \to 0} f(x) = 0$, but $\lim_{x \to
      0} g(x)$ does not exist, as $\sfrac{1}{0}$ isn't defined as it's not
      continuous.
  \end{enumerate}
\end{exersolution}
