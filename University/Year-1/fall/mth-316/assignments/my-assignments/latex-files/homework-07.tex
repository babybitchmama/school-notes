\setcounter{chapter}{3}
\setcounter{section}{2}

\begin{exercise}[4]
  Let $A$ be a nonempty and bounded above so that $s = \sup(A)$ exists.
  \begin{enumerate}
    \item Show that $s \in \bar{A}$.
    \item Can an open set contain its supremum.
  \end{enumerate}
\end{exercise}

\begin{exersolution}[4]
\end{exersolution}

\newpage

\begin{exercise}[5]
  Prove Theorem 3.2.8.
\end{exercise}

\begin{exersolution}[5]
\end{exersolution}

\newpage

\begin{exercise}[7]
  Given $A \subseteq \R$, let $L$ be the set of all limit points of $A$.
  \begin{enumerate}
    \item Show that the set $L$ is closed.

    \item Argue that if $x$ is a limit point of $A \cup L$, then $x$ is a limit
      point of $A$. Use this observation to furnish a proof for Theorem 3.2.12.
  \end{enumerate}
\end{exercise}

\begin{exersolution}[7]
\end{exersolution}

\newpage

\begin{exercise}[11]
  \begin{enumerate}
    \item Prove that $\bar{A \cup B} = \bar{A} \cup \bar{B}$.
    \item Does this result about closures extend to infinite unions of sets?
  \end{enumerate}
\end{exercise}

\begin{exersolution}[11]
\end{exersolution}

\newpage

\setcounter{section}{3}

\begin{exercise}[1]
  Show that if $K$ is compact and nonempty, then $\sup(K)$ and $\inf(K)$ both
  exist and are elements of $K$.
\end{exercise}

\begin{exersolution}[1]
\end{exersolution}

\newpage

\begin{exercise}[2]
  Decide which of the following sets are compact. For those that are not
  compact, show how Definition 3.3.1 breaks down. In other words, give an
  example of a sequence contained in the given set that does not possess a
  subsequence converging to a limit in the set.
  \begin{enumerate}
    \item $\N$.

    \item $Q \cap [0, 1]$.

    \item $\{1, \sfrac{1}{2}, \sfrac{2}{3}, \sfrac{3}{4}, \sfrac{4}{5},
      \dots\}$.
  \end{enumerate}
\end{exercise}

\begin{exersolution}[2]
\end{exersolution}

\newpage

\begin{exercise}[3]
  Prove the converse of Theorem 3.3.4 by showing that if a set $K \subseteq \R$
  is closed and bounded, then it is compact.
\end{exercise}

\begin{exersolution}[3]
\end{exersolution}
