\setcounter{chapter}{3}
\setcounter{section}{2}

\begin{exercise}[4]
  Let $A$ be a nonempty and bounded above so that $s = \sup(A)$ exists.
  \begin{enumerate}
    \item Show that $s \in \bar{A}$.
    \item Can an open set contain its supremum.
  \end{enumerate}
\end{exercise}

\begin{exersolution}[4]
  \begin{enumerate}
    \item \begin{proof}
        Since every $s - \epsilon$ has an $a \in A$ with $a > s - \epsilon$, we
        can find $a \in V_{\epsilon}(s)$, for any $\epsilon > 0$. This implies
        that $s$ is a limit point of $A$. Therefore, $s \in \bar{A}$.
      \end{proof}

    \item No, there doesn't exist any $\epsilon$-neighborhood of $s$ such that
      $V_{\epsilon}(s) \not\subseteq (s, s + \epsilon)$.
  \end{enumerate}
\end{exersolution}

\newpage

\begin{exercise}[5]
  Prove Theorem 3.2.8.

  \begin{theorem}
    A set $F \subseteq \R$ is closed if and only if every Cauchy sequence
    contained in $F$ has a limit that is also an element of $F$.
  \end{theorem}
\end{exercise}

\begin{exersolution}[5]
  \begin{proof}
    Let $F \subseteq \R$ be closed. Given an arbitrary Cauchy sequence $(a_n)$
    contained in $F$, we know that $(a_n)$ converges to some $a \in \R$. Since
    $F$ is closed and $(a_n)$ is contained in $F$, $a \in F$.

    Suppose that every Cauchy sequence contained in $F$ converges to a point,
    $l$, in $F$. Since $l$ is a limit point of $F$, there exists a sequence
    $(a_n)$ contained in $F$ with $\lim_{n \to \infty} (a_n) = l$. Since $(a_n)$
    converges, it must be a Cauchy sequence (by the Cauchy Criterion). Since
    every Cauchy sequence converges to a limit inside $F$, we have $l \in F$,
    meaning that $F$ is closed.
  \end{proof}
\end{exersolution}

\newpage

\begin{exercise}[7]
  Given $A \subseteq \R$, let $L$ be the set of all limit points of $A$.
  \begin{enumerate}
    \item Show that the set $L$ is closed.
  \end{enumerate}
\end{exercise}

\begin{exersolution}[7]
  \begin{enumerate}
    \item \begin{proof}
        Assume $l$ is a limit point of $L$. Then, by definition, for every
        $\epsilon > 0$, we have
        \[%
          L_1 = V_{\epsilon}(l) \cap (L - \{l\}) \ne \emptyset
        .\]%
        But, since $L_1 \subseteq L$, for every $y \in L_1$ and $\delta > 0$, we have
        \[%
          L_2 = V_{\delta}(y) \cap (A - \{y\}) \ne \emptyset
        .\]%
        Now, since $l$ is a limit point of $L$, for any $\epsilon > 0$, there
        exists $y \in L - \{l\}$ such that $y \in V_{\epsilon}(l)$. By the
        definition of $y \in L$, there exists $z \in A$, distinct from $y$, such
        that $z \in V_{\delta}(y)$.

        Choose $\delta > 0$ small enough so that $V_{\delta}(y) \subseteq
        V_{\epsilon}(l)$. Then, $z \in V_{\delta}(y) \subseteq V_{\epsilon}(l)$,
        and since $z \neq y$ and $y \neq l$, it follows that $z \neq l$.
        Therefore, $z \in V_{\epsilon}(l) - \{l\}$.

        Since this holds for every $\epsilon > 0$, $l$ is a limit point of $A$.
        By definition of $L$, we have $l \in L$.
      \end{proof}
  \end{enumerate}
\end{exersolution}

\newpage

\begin{exercise}[11]
  \begin{enumerate}
    \item Prove that $\overline{A \cup B} = \bar{A} \cup \bar{B}$.
    \item Does this result about closures extend to infinite unions of sets?
  \end{enumerate}
\end{exercise}

\begin{exersolution}[11]
  \begin{enumerate}
    \item \begin{proof}
        Let $L$ be the set of limit points of $A \cup B$, $L_a$ and $L_b$ be the
        sets of limit points of $A$ and $B$, respectively.

        Let $x \in L$. Thus, there exists $a_n \in A \cup B$ with $x = \lim_{n
        \to \infty} x_n$ such that $x_n$ is bounded in $A \cup B$.
      \end{proof}
  \end{enumerate}
\end{exersolution}

\newpage

\setcounter{section}{3}

\begin{exercise}[1]
  Show that if $K$ is compact and nonempty, then $\sup(K)$ and $\inf(K)$ both
  exist and are elements of $K$.
\end{exercise}

\begin{exersolution}[1]
\end{exersolution}

\newpage

\begin{exercise}[2]
  Decide which of the following sets are compact. For those that are not
  compact, show how Definition 3.3.1 breaks down. In other words, give an
  example of a sequence contained in the given set that does not possess a
  subsequence converging to a limit in the set.
  \begin{enumerate}
    \item $\N$.

    \item $Q \cap [0, 1]$.

    \item $\{1, \sfrac{1}{2}, \sfrac{2}{3}, \sfrac{3}{4}, \sfrac{4}{5},
      \dots\}$.
  \end{enumerate}
\end{exercise}

\begin{exersolution}[2]
\end{exersolution}

\newpage

\begin{exercise}[3]
  Prove the converse of Theorem 3.3.4 by showing that if a set $K \subseteq \R$
  is closed and bounded, then it is compact.
\end{exercise}

\begin{exersolution}[3]
\end{exersolution}
