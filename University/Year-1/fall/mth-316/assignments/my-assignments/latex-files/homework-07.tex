\begin{center}
  SECTION 3.2
\end{center}

\medskip

\setcounter{chapter}{3}
\setcounter{section}{2}

\begin{exercise}[4]
  Let $A$ be a nonempty and bounded above so that $s = \sup(A)$ exists.
  \begin{enumerate}
    \item Show that $s \in \bar{A}$.
    \item Can an open set contain its supremum.
  \end{enumerate}
\end{exercise}

\begin{proof}[Solution to (i)]
  Since every $s - \epsilon$ has an $a \in A$ with $a > s - \epsilon$, we can
  find $a \in V_{\epsilon}(s)$, for any $\epsilon > 0$. This implies that $s$ is
  a limit point of $A$. Therefore, $s \in \bar{A}$.
\end{proof}

\begin{proof}[Solution to (ii)]
  No, there doesn't exist any $\epsilon$-neighborhood of $s$ such that
  $V_{\epsilon}(s) \not\subseteq (s, s + \epsilon)$.
\end{proof}

\medskip

\begin{exercise}[5]
  Prove Theorem 3.2.8.
\end{exercise}

\begin{proof}[Solution]
  Let $F \subseteq \R$ be closed. Given an arbitrary Cauchy sequence $(a_n)$
  contained in $F$, we know that $(a_n)$ converges to some $a \in \R$. Since $F$
  is closed and $(a_n)$ is contained in $F$, $a \in F$.

  Suppose that every Cauchy sequence contained in $F$ converges to a point, $l$,
  in $F$. Since $l$ is a limit point of $F$, there exists a sequence $(a_n)$
  contained in $F$ with $\lim_{n \to \infty} (a_n) = l$. Since $(a_n)$
  converges, it must be a Cauchy sequence (by the Cauchy Criterion). Since every
  Cauchy sequence converges to a limit inside $F$, we have $l \in F$, meaning
  that $F$ is closed.
\end{proof}

\medskip

\begin{exercise}[7]
  Given $A \subseteq \R$, let $L$ be the set of all limit points of $A$.
  \begin{enumerate}
    \item Show that the set $L$ is closed.
  \end{enumerate}
\end{exercise}

\begin{proof}[Solution to (i)]
  Let $L'$ be the set of all limit points of $L$. Assume $\ell \in L'$. Then, by
  definition, for every $\epsilon > 0$, we have
  \[%
    L_1 = V_{\epsilon}(\ell) \cap (L - \{\ell\}) \ne \emptyset \subseteq L
  .\]%
  Then,
  \[%
    (\forall y \in L_1)(\forall \epsilon > 0)[L_2 = V_{\delta}(y) \cap (A - \{y\}) \ne \emptyset]
  .\]%
  Now, since $\ell$ is a limit point of $L$, for any $\epsilon > 0$, there
  exists $y \in L - \{\ell\}$ such that $y \in V_{\epsilon}(l)$. By the
  definition of $y \in L$, there exists $z \in A$, distinct from $y$, such that
  $z \in V_{\delta}(y)$.

  Choose $0 < \delta < \epsilon$ so $V_{\delta}(x) \subseteq V_{\epsilon}(l)$.
  Since $z \in V_{\delta}(y) \cap (A - \{y\})$, then $z \in V_{\epsilon}(\ell)$
  and $z \in A$. So, $\ell$ is a limit point of $A$. Therefore, $\ell \in L$ and
  $L$ is closed.
\end{proof}

\medskip

\begin{exercise}[11]
  \begin{enumerate}
    \item Prove that $\overline{A \cup B} = \bar{A} \cup \bar{B}$.
    \item Does this result about closures extend to infinite unions of sets?
  \end{enumerate}
\end{exercise}

\begin{proof}[Solution to (i)]
  Let $L_a$ and $L_b$ be the set of limit points for $A$ and $B$ respectively.
  Let $\ell$ be a limit point for $A \cup B$. Then, $\ell \in L_a$ or $\ell \in
  L_b$. So, $\ell \in L_a \cup L_b$. Then, by definition $\overline{A \cup B} =
  (A \cup B) \cup (L_a \cup L_b) = (A \cup L_a) \cup (B \cup L_b) = \bar{A} \cup
  \bar{B}$.
\end{proof}

\begin{proof}[Solution to (ii)]
  No, the result does not extend to infinite unions of sets. Specifically,
  \[%
    \overline{\bigcup_{i \in I} A_i} \ne \bigcup_{i \in I} \bar{A_i}
  ,\]%
  where $\{A_i\}_{i \in I}$ is a collection of sets, indexed by $I$.

  For a finite union of sets $\overline{A \cup B} = \bar{A} \cup \bar{B}$ holds
  because any limit point of $A \cup B$ must lie in $\bar{A} \cup \bar{B}$, and
  the closure distributes over a finite union.

  For an infinite union, however, the equality may fail because the closure of
  an infinite union may include points that are limit points of the entire
  union, but not limit points of any single set in the collection.

  Let $A_n = \{\sfrac{1}{n} \mid n \in \N\}$ as a counterexample. Then,
  \[%
    \overline{\bigcup_{n=1}^{\infty} A_n} = \left.\left\{\frac{1}{n} \right\rvert n \in \N\right\} \cup \{0\} \ne \left.\left\{\frac{1}{n} \right\rvert n \in \N\right\} = \bigcup_{n=1}^{\infty} \bar{A}_n
  .\qedhere\]%
\end{proof}

\medskip

\begin{center}
  SECTION 3.3
\end{center}

\medskip

\setcounter{section}{3}

\begin{exercise}[1]
  Show that if $K$ is compact and nonempty, then $\sup(K)$ and $\inf(K)$ both
  exist and are elements of $K$.
\end{exercise}

\begin{proof}[Solution]
  By Theorem 3.3.4, $K$ is compact if and only if it is closed and bounded.
  Since $K \neq \emptyset$, it follows that $K$ is bounded above and bounded
  below. By the Axiom of Completeness, $\sup(K)$ exists because $K$ is bounded
  above, and $\inf(K)$ exists because $K$ is bounded below. To show $\sup(K)
  \in K$, note that $\sup(K)$ is a limit point of $K$ (by definition of the
  least upper bound). Since $K $ is closed, it contains all its limit points,
  so $\sup(K) \in K$. Similarly, $\inf(K)$ is a limit point of $K$, and by the
  closedness of $K$, $\inf(K) \in K$.
\end{proof}

\medskip

\begin{exercise}[2]
  Decide which of the following sets are compact. For those that are not
  compact, show how Definition 3.3.1 breaks down. In other words, give an
  example of a sequence contained in the given set that does not possess a
  subsequence converging to a limit in the set.
  \begin{enumerate}
    \item $\N$.

    \item $Q \cap [0, 1]$.

    \item $S = \{1, \sfrac{1}{2}, \sfrac{2}{3}, \sfrac{3}{4}, \sfrac{4}{5},
      \dots\}$.
  \end{enumerate}
\end{exercise}

\begin{proof}[Solution to (i)]
  No, since the sequence $x_n = n$ doesn't diverges.
\end{proof}

\begin{proof}[Solution to (ii)]
  No, since the sequence $x_n = \sfrac{1}{\sqrt{2}} + \frac{1}{n}$ converges to
  $\sfrac{1}{\sqrt{2}} \notin Q \cap [0, 1]$.
\end{proof}

\begin{proof}[Solution to (iii)]
  No, since the sequence $x_n = \sfrac{1}{\sqrt{2}} + \frac{1}{n}$ converges to
  $\sfrac{1}{\sqrt{2}} \notin Q \cap [0, 1]$.
\end{proof}

\begin{proof}[Solution to (iv)]
  Compact, since, it's bounded, closed, and every sequence converges to the
  limit value of $1 \in S$.
\end{proof}

\medskip

\begin{exercise}[3]
  Prove the converse of Theorem 3.3.4 by showing that if a set $K \subseteq \R$
  is closed and bounded, then it is compact.
\end{exercise}

\begin{proof}[Solution]\leavevmode
  \begin{enumerate}
    \item[$\implies$)] Suppose $K \subseteq \R$ is compact. Suppose $K$ isn't
      bounded. Let $(a_n)$ be a sequence in $K$. Since $K$ is unbounded, then
      given any $n \in \N$, we can produce $a_n \in K$ such that $\lvert a_n
      \rvert > n$. Now, since $K$ is compact, there must exist a subsequence
      $(a_{n_k})$ that converges to a limit $a \in K$. But the elements of
      $(a_{n_k})$ must satisfy $\lvert a_{n_k} \rvert > n_k$, which implies that
      $(a_{n_k})$ is unbounded, meaning it doesn't converge. This gives us a
      contradiction since every sequence must contain a converging subsequence
      to be called compact. This implies that $K$ must be bounded.

      Now that we know $K$ is bounded, then suppose we are given a sequence
      $(a_n)$ that's contained in $K$. By the BWT, there exists a converging
      subsequence $(a_{n_k})$ that converges to a limit $a$. By definition of
      compactness, $a \in K$. Suppose that $(a_n)$ converges to a limit point.
      Then, by Theorem 2.5.2, all subsequences converge to the same limit as the
      original sequence. So, $(a_n) \to x$ and $a \in K$, making $a$ a limit
      point. Therefore, $K$ is closed.

    \item[$\impliedby$)] Suppose $K \subseteq \R$ is closed and bounded. Let
      $(a_n)$ be a sequence contained in $K$. Since $K$ is bounded, by the BWT,
      there exists a converging subsequence $(a_{n_k})$ that converges to a
      limit $a$. Since $K$ is closed, $a \in K$. Therefore, $K$ is compact.
      \qedhere
  \end{enumerate}
\end{proof}
