\begin{center}
  SECTION 4.2
\end{center}

\setcounter{chapter}{4}
\setcounter{section}{2}

\medskip

\begin{exercise}[1]
  \begin{enumerate}
    \item Supply the details for how Corollary 4.2.4 part (ii) follows from the
      Sequential Criterion for Functional Limits in Theorem 4.2.3 and the
      Algebraic Limit Theorem for sequences proved in Chapter 2.

    \item Now, write another proof of Corollary 4.2.4 part (ii) directly from
      Definition 4.2.1 without using the sequential criterion in Theorem 4.2.3.
  \end{enumerate}
\end{exercise}

\begin{proof}[Solution to (i)]
  Let $f : A \to \R$ and $g : A \to \R$. By the Sequential Criterion for
  Functional Limits, since $\lim_{x \to c} f(x) = L$, $\lim_{x \to c} g(x) = M$,
  all sequences $(a_n) \to c$, we have $f(x_n) \to L$ and $g(x_n) \to M$. By the
  Algebraic Limit Theorem for sequences, we have $\lim_{n \to \infty} [f(x_n) +
  g(x_n)] = L + M$. Therefore, $\lim_{x \to c} [f(x) + g(x)] = L + M$.
\end{proof}

\begin{proof}[Solution to (ii)]
  Let $\epsilon > 0$. Let $\delta_1$ such that $0 < \lvert x - c \rvert <
  \delta_1 \implies \lvert f(x) - L \rvert < \sfrac{\epsilon}{2}$. Let
  $\delta_2$ such that $0 < \lvert x - c \rvert < \delta_2 \implies \lvert g(x)
  - M \rvert < \sfrac{\epsilon}{2}$. Let $\delta = \min\{\delta_1, \delta_2\}$.
  Let $x$ be arbitrary. Suppose $0 < \lvert x - c \rvert < \delta$. Then, we get
  the following
  \[%
    0 < \lvert x - c \rvert < \delta \implies \lvert [f(x) + g(x)] - [M + L] \rvert < \lvert f(x) - M \rvert + \lvert g(x) - L \rvert < \frac{\epsilon}{2} + \frac{\epsilon}{2} = \epsilon
  .\]%

  Therefore, $0 < \lvert x - c \rvert < \delta \implies \lvert [f(x) + g(x)] -
  [M + L] \rvert < \epsilon$.

  It follows that $(\forall \epsilon > 0)(\exists \delta > 0)(\forall x \in A)[0
  < \lvert x - c \rvert < \delta \implies \lvert [f(x) + g(x)] - [M + L] \rvert
  < \epsilon]$.
\end{proof}

\medskip

\begin{exercise}[3]
  Review the definition of Thomae's function $t(x)$ from Section 4.1.
  \[%
    t(x) = \begin{cases}
      1 & \textrm{if $x = 0$} \\
      1/n & \textrm{if $x = \sfrac{m}{n}$ in lowest terms with $n > 0$} \\
      0 & \textrm{if $x \notin \Q$}
    \end{cases}
  .\]%

  \begin{enumerate}
    \item Construct three different sequences $(x_n)$, $(y_n)$, and $(z_n)$,
      each of which converges to $1$ without using the number $1$ as a term in
      the sequence.

    \item Now, compute $\lim_{n \to \infty} t(x_n)$, $\lim_{n \to \infty}
      t(y_n)$, and $\lim_{n \to \infty} t(z_n)$.

    \item Make an educated conjecture for $\lim_{x \to 1} t(x)$, and use
      Definition 4.2.1 B to verify the claim (Given $\epsilon > 0$, consider the
      set of points $\{x \in \R \mid t(x) \ge \epsilon\}$. Argue that all the
      points in this set are isolated).
  \end{enumerate}
\end{exercise}

\begin{proof}[Solution to (i)]
  Let $(a_n) = (0, \sfrac{1}{2}, \sfrac{2}{3}, \sfrac{3}{4}, \dots)$, $(b_n) =
  (0, \sfrac{1}{3}, \sfrac{2}{4}, \sfrac{3}{5}, \dots)$, and $(c_n) = (0,
  \sfrac{1}{4}, \sfrac{2}{5}, \sfrac{3}{6}, \dots)$.
\end{proof}

\begin{proof}[Solution to (ii)]
  We have $\lim_{n \to \infty} t(a_n) = 1$, $\lim_{n \to \infty} t(b_n) = 1$,
  and $\lim_{n \to \infty} t(c_n) = 1$.
\end{proof}

\begin{proof}[Solution to (iii)]
  I conjecture that $\lim_{x \to 1} t(x) = 0$.

  Let $\epsilon > 0$. From the definition of $t(x)$, the condition $t(x) \geq
  \epsilon$ implies that $x = \frac{m}{n}$ in lowest terms with $n \leq
  \frac{1}{\epsilon}$. Since $m$ is bounded by $\lvert m \rvert \leq n$ and $n$
  is bounded by $\frac{1}{\epsilon}$, there are finitely many such rational
  numbers.

  These rational points $\frac{m}{n}$, where $n \leq \frac{1}{\epsilon}$, form a
  discrete set. Hence, outside a small neighborhood of these points, $t(x) <
  \epsilon$. Define $\delta > 0$ as the minimum distance between $1$ and any of
  these finitely many rational points
  \[%
    \delta = \min\left(\left\{\abs(1 - x) \left\lvert~x = \frac{m}{n}, n \leq \frac{1}{\epsilon}\right\}\right)\right.
  .\]%

  Then, for any $x \in \R$ such that $0 < \lvert x - 1 \rvert < \delta$, it
  follows that $t(x) < \epsilon$.

  Therefore, $(\forall \epsilon > 0)(\exists \delta > 0)(\forall x \in \R)[0 <
  \lvert x - 1 \rvert < \delta \implies t(x) < \epsilon]$.

  It follows that $\lim_{x \to 1} t(x) = 0$.
\end{proof}

\medskip

\begin{exercise}[7]
  Let $g : A \to \R$ and assume that $f$ is a bounded function on $A$ in the
  sense that that there exists $M > 0$ satisfying $\lvert f(x) \rvert \le M$ for
  all $x \in A$.

  Show that $\lim_{x \to \infty} f(x) = 0$, then $\lim_{x \to a} f(x)g(x) = 0$
  for any function $g$ (with domain equal to the domain of $f$).
\end{exercise}

\begin{proof}[Solution]
  Let $g : A \to \R$ and let $f$ be bounded on $A$. Then, there exists an $M >
  0$ such that $(\forall x \in A)[\lvert f(x) \rvert \le M]$. Then, $(\forall x
  \in A)[\lvert f(x) \cdot g(x) \rvert \le M \cdot \lvert g(x) \rvert]$. Set
  $\delta$ small enough such that $\lvert g(x) \rvert < \frac{\epsilon}{M}$.
  Then, we get $\lvert f(x) \cdot g(x) \rvert \le M \cdot \lvert g(x) \rvert < M
  \cdot \frac{\epsilon}{M} = \epsilon$.
\end{proof}

\medskip

\begin{exercise}[8]
  Compute each limit or state that it does not exist. Use the tools developed in
  this section to justify your answer.

  \begin{enumerate}
    \item $\lim_{x \to 2} \frac{\lvert x - 2 \rvert}{x - 2}$.
    \item $\lim_{x \to \frac{7}{4}} \frac{\lvert x - 2 \rvert}{x - 2}$.
    \item $\lim_{x \to 0} (-1)^{\lfloor \sfrac{1}{x} \rfloor}$.
    \item $\lim_{x \to 0} \sqrt[3]{x} (-1)^{\lfloor \sfrac{1}{x} \rfloor}$.
  \end{enumerate}
\end{exercise}

\begin{proof}[Solution to (i)]
  The limit does not exist.

  Let $(x_n) = 2 + \frac{1}{n}$ and $(y_n) = 2 - \frac{1}{n}$. Since $\lim_{n
  \to \infty} a_n = 2 = \lim_{n \to \infty} y_n$ and $\lim_{n \to \infty} f(x_n)
  = 1 \ne -1 = \lim_{n \to \infty} f(y_n)$, then by the Divergence Criterion for
  Functional Limits, the limit does not exist.
\end{proof}

\begin{proof}[Solution to (ii)]
  The limit is $-1$.

  The function $\frac{\lvert x - 2 \rvert}{x - 2}$ can be simplified as
  \[%
    \frac{\lvert x - 2 \rvert}{x - 2} =
    \begin{cases}
      \phantom{-}1, & \text{if } x - 2 > 0 \\
      -1, & \text{if } x - 2 < 0
    \end{cases}
  .\]%
  Notice that when $x = \frac{7}{4}$, we have $x - 2 = \frac{7}{4} - 2 =
  -\frac{1}{4} < 0$. Thus, as $x \to \frac{7}{4}$, $x - 2 < 0$.

  Let $\epsilon > 0$. Suppose $0 < \lvert x - \frac{7}{4} \rvert < \delta$, for
  some $\delta > 0$. Thus, for $x$ sufficiently close to $\frac{7}{4}$,
  $\frac{\lvert x - 2 \rvert}{x - 2} = -1$, and we find
  \[%
    \left\lvert \frac{\lvert x - 2 \rvert}{x - 2} + 1 \right\rvert = \lvert -1 + 1 \rvert = 0 < \epsilon
  .\]%
  We can choose any $\delta > 0$, such as $\delta = 1$, to ensure this
  inequality holds because the function is constant at $-1$ near $x =
  \frac{7}{4}$ when $x - 2 < 0$.

  Therefore, $\lim_{x \to \frac{7}{4}} \frac{\lvert x - 2 \rvert}{x - 2} = -1$.
\end{proof}

\begin{proof}[Solution to (iii)]
  The limit does not exist.

  Let $(x_n) = \frac{1}{2n}$ and $(y_n) = \frac{1}{2n + 1}$. Since $\lim_{n \to
  \infty} x_n = 0 = \lim_{n \to \infty} y_n$ and $\lim_{n \to \infty} f(x_n) = 1
  \ne -1 = \lim_{n \to \infty} f(y_n)$, then by the Divergence Criterion for
  Functional Limits, the limit does not exist.
\end{proof}

\begin{proof}[Solution to (iv)]
  The limit is $0$.

  Let $g(x) = (-1)^{\lfloor \sfrac{1}{x} \rfloor}$ and $g(x) = \sqrt[3]{x}$.
  Then $\lim_{x \to 0} g(x) = 0$ and $f(x)$ is bounded by $M > 0$, then, by
  Exercise 4.2.7, $\lim_{x \to 0} f(x) \cdot g(x) = 0$.
\end{proof}

I don't know if you want me to prove $\lim_{x \to 0} f(x)$, but I will anyway.

\begin{proof}[Proof of $\lim_{x \to 0} f(x) = 0$]
  Let $\epsilon > 0$. Let $\delta = \epsilon^3$. Let $x$ be arbitrary. Suppose
  $0 < \lvert x \rvert < \delta$. Then, we get
  \[%
    0 < \lvert x \rvert < \delta \implies \lvert \sqrt[3]{x} \rvert < \delta^{\sfrac{1}{3}} = \epsilon
  .\]%

  Therefore, $0 < \lvert x \rvert < \delta \implies \lvert \sqrt[3]{x} \rvert <
  \epsilon$.

  It follows that $(\forall \epsilon > 0)(\exists \delta > 0)(\forall x \in A)[0
  < \lvert x \rvert < \delta \implies \lvert \sqrt[3]{x} \rvert < \epsilon]$.
\end{proof}

\medskip

\begin{exercise}[11]
  Let $f$, $g$, and $h$ satisfy $f(x) \le g(x) \le h(x)$ for all $x$ in some
  common domain $A$. If $\lim_{x \to c} f(x) = L$ at some point $c \in A$, show
  $\lim_{x \to c} g(x) = L$ as well.
\end{exercise}

\begin{proof}[Solution]
  Let $\epsilon > 0$. Assume $\lim_{x \to c} f(x) = L = \lim_{x \to c} h(x)$. By
  definition of a limit, there exist $\delta_1 > 0$ and $\delta_2 > 0$ such that
  \[%
    0 < \lvert x - c \rvert < \delta_1 \implies \lvert f(x) - L \rvert < \frac{\epsilon}{2} \aand 0 < \lvert x - c \rvert < \delta_2 \implies \lvert h(x) - L \rvert < \frac{\epsilon}{2}
  .\]%
  Let $\delta = \min\{\delta_1, \delta_2\}$. Then, for $0 < \lvert x - c \rvert
  < \delta$, we have
  \[%
    L - \frac{\epsilon}{2} < f(x) < L + \frac{\epsilon}{2} \aand L - \frac{\epsilon}{2} < h(x) < L + \frac{\epsilon}{2}
  .\]%
  Suppose $0 < \lvert x - c \rvert < \delta$. Then, we get
  \begin{align*}
    \lvert g(x) - L \rvert &\le \lvert g(x) - f(x) \rvert + \lvert f(x) - L \rvert \\
                           &\le \lvert h(x) - f(x) \rvert + \lvert f(x) - L \rvert \\
                           &\le \lvert h(x) - L \rvert + \lvert L - f(x) \rvert + \lvert f(x) - L \rvert \\
                           &\le \lvert h(x) - L \rvert + 2\lvert f(x) - L \rvert \\
                           &< \frac{\epsilon}{2} + 2\frac{\epsilon}{4} = \epsilon
  .\end{align*}

  Therefore, $0 < \lvert x - c \rvert < \delta \implies \lvert g(x) - L \rvert <
  \epsilon$.

  It follows that $(\forall \epsilon > 0)(\exists \delta > 0)(\forall x \in A)[0
  < \lvert x - c \rvert < \delta \implies \lvert g(x) - L \rvert < \epsilon].$
\end{proof}

\medskip

\begin{center}
  SECTION 4.3
\end{center}

\setcounter{section}{3}

\medskip

\begin{exercise}[1]
  Let $g(x) = \sqrt[3]{x}$
  \begin{enumerate}
    \item Prove that $g$ is continuous at $c = 0$.

    \item Prove that $g$ is continuous at a point $c \ne 0$ (The identity $a^3 -
      b^3 = (a - b)(a^2 + ab + b^2)$ will be helpful).
  \end{enumerate}
\end{exercise}

\begin{proof}[Solution to (i)]
  Let $\epsilon > 0$. Let $\delta = \epsilon^3$. Let $x$ be arbitrary. For $c =
  0$, suppose $\lvert x - 0 \rvert < \delta$. Then, we get
  \[%
    \lvert x \rvert < \delta \implies \lvert \sqrt[3]{x} \rvert < \delta^{\sfrac{1}{3}} = \epsilon
  .\]%

  Therefore, $\lvert x \rvert < \delta \implies \lvert \sqrt[3]{x} \rvert <
  \epsilon$.

  It follows that $(\forall \epsilon > 0)(\exists \delta > 0)[\lvert x \rvert <
  \delta \implies \lvert \sqrt[3]{x} \rvert < \epsilon]$.

  Therefore, by the definition of continuity, $g$ is continuous at $c = 0$.
\end{proof}

\begin{proof}[Solution to (ii)]
  Let $\epsilon > 0$. Choose $\delta < \lvert c \rvert$. Let $x$ be arbitrary.
  For $c \ne 0$, suppose $\lvert x - c \rvert < \delta$. Using the given
  inequality, we get
  \[%
    \lvert x^{\sfrac{1}{3}} - c^{\sfrac{1}{3}} \rvert = \lvert x - c \rvert \cdot \lvert x^{\sfrac{2}{3}} + x^{\sfrac{1}{3}} \cdot c^{\sfrac{1}{3}} + c^{\sfrac{2}{3}} \rvert
  .\]%
  Since $\delta < \lvert c \rvert$, we get $\lvert x - c \rvert < \delta
  \implies \lvert x \rvert < \delta < \delta + \lvert c \rvert < 2 \lvert c
  \rvert$. Finding an upper bound for the right-hand side of the first
  inequality gives us
  \begin{align*}
    \lvert x - c \rvert \cdot \lvert x^{\sfrac{2}{3}} + x^{\sfrac{1}{3}} \cdot c^{\sfrac{1}{3}} + c^{\sfrac{2}{3}} \rvert &\le \lvert x^{\sfrac{2}{3}} \rvert + \lvert x^{\sfrac{1}{3}} \cdot c^{\sfrac{1}{3}} \rvert + \lvert c^{\sfrac{2}{3}} \rvert \\
                                                                                                                          &\le 2^{\sfrac{2}{3}} \cdot \lvert c^{\sfrac{2}{3}} \rvert + 2^{\sfrac{1}{3}} \cdot \lvert c^{\sfrac{2}{3}} \rvert + \lvert c^{\sfrac{2}{3}} \rvert = M
  .\end{align*}

  Then, we get $\lvert x^{\sfrac{1}{3}} - c^{\sfrac{1}{3}} \rvert \le \lvert x -
  c \rvert \cdot M$. Let $\delta = \frac{\epsilon}{M}$. Then, we get $\lvert x -
  c \rvert \cdot M < \delta \cdot M = \epsilon$

  Therefore, $\lvert x - c \rvert < \delta \implies \lvert x^{\sfrac{1}{3}} -
  c^{\sfrac{1}{3}} \rvert < \epsilon$.

  It follows that $(\forall \epsilon > 0)(\exists \delta > 0)(\forall x \in
  A)[\lvert x - c \rvert < \delta \implies \lvert x^{\sfrac{1}{3}} -
  c^{\sfrac{1}{3}} \rvert < \epsilon]$.

  Therefore, by the definition of continuity, $g$ is continuous at $c \ne 0$.
  Since $g$ is continuous at $c = 0$ and $c \ne 0$, $g$ is continuous on $A$
  (its domain).
\end{proof}

\medskip

\begin{exercise}[5]
  Show using Definition 4.3.1 that if $c$ is an isolated point of $A \subseteq \R$, then $f : A \to \R$ is continuous at $c$.
\end{exercise}

\begin{proof}[Solution]
  Let $\epsilon > 0$. Since $c$ is an isolated point, then we can choose a
  $\delta > 0$ small enough such that $x \in V_{\delta}(c)$ if and only if $x =
  c$. Then, we get $\lvert x - c \rvert = 0 < \delta$. Then, clearly $\lvert
  f(x) - f(c) \rvert = 0 < \epsilon$.

  Therefore, $(\forall \epsilon > 0)(\exists \delta > 0)[\lvert x - c \rvert <
  \delta \implies \lvert f(x) - f(c) \rvert < \epsilon]$.

  Therefore, by the definition of continuity, $f$ is continuous at $c$.
\end{proof}
