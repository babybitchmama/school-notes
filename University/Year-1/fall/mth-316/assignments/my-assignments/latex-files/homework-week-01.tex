\renewcommand\type{exercise}
\setcounter{chapter}{1}
\setcounter{section}{2}

\begin{exercise}[1]
  \begin{enumerate}
    \item Prove that $\sqrt{3}$ is irrational. Does a similar argument work to
      show $\sqrt{6}$ is irrational?

    \item Where does the proof of Theorem 1.1.1 break down if we try to use it
      to prove $\sqrt{4}$ is irrational?\vspace{-0.5cm}
  \end{enumerate}
\end{exercise}

\begin{exersolution}[1]
  \begin{enumerate}
    \item \textbf{Proof:} Suppose $\sqrt{3}$ is rational. This means that there
      exists $p, q \in \Z$ such that $\frac{p}{q} = \sqrt{3}$. Suppose $p$ and
      $q$ have no common factors. Then we have $p^2 = 3q^2$, which means $p^2$
      is divisible by $3$. This implies that $p$ is divisible by 3, so $p = 3k$
      for some $k \in \Z$. Substituting this back into the equation, we have
      $9k^2 = 3q^2$, which simplifies to $3k^2 = q^2$. This means $q$ is also
      divisible by $3$, which contradicts our assumption that $p$ and $q$ have
      no common factors. Therefore, $\sqrt{3}$ is irrational. \hfill\qedsymbol

      The same argument does show that $\sqrt{6}$ is irrational.

    \item The fact that breaks the proof for Theorem 1.1.1 is that $\sqrt{4}$ is
      a perfect square, meaning that $p^2 = 4q^2$ doesn't imply that $p$ is a
      multiple of $4$. In fact, $p = 2q$ implies that $p$ is a multiple of $2$,
      which isn't a contradiction.
  \end{enumerate}
\end{exersolution}

\newpage

\begin{exercise}[5]
  Let $A$ and $B$ be subsets of $\R$.
  \begin{enumerate}
    \item If $x \in (A \cap B)^c$, explain why $x \in A^c \cup
      B^c$. This shows that $(A \cap B)^c \subseteq A^c \cup B^c$.

    \item Prove the reverse inclusion $(A \cap B)^c \supseteq
      A^c \cup B^c$, and conclude that $(A \cap B)^c = A^c \cup B^c$.

    \item Show $(A \cup B)^c = A^c \cap B^c$ by demonstrating
      inclusion both ways.\vspace{-0.5cm}
  \end{enumerate}
\end{exercise}

\begin{exersolution}[5]
  \begin{enumerate}
    \item \textbf{Proof:} Suppose $x \in (A \cap B)^c$. Then $x \notin (A \cap
      B)$, meaning $x \notin A$ or $x \notin B$, which is equivalent to $x \in
      A^c \cup B^c$. Therefore, $(A \cap B)^c \subseteq A^c \cup B^c$.
      \hfill\qedsymbol

    \item \textbf{Proof:} Suppose $x \in A^c \cup B^c$. Then $x \in A^c$ or $x
      \in B^c$, which is equivalent to $x \notin A$ or $x \notin B$. This means
      $x \notin A \cap B$, so $x \in (A \cap B)^c$. Then, $A^c \cup B^c
      \subseteq (A \cap B)^c$.

      Therefore, since $(A \cap B)^c \subseteq A^c \cup B^c$ and $A^c \cup B^c
      \subseteq (A \cap B)^c$, we have $(A \cap B)^c = A^c \cup B^c$.
      \hfill\qedsymbol

    \item \textbf{Proof:} Suppose $x \in (A \cup B)^c$. Then $x \notin (A \cup
      B)$, meaning $x \notin A$ and $x \notin B$. This is equivalent to $x \in
      A^c$ and $x \in B^c$, so $x \in A^c \cap B^c$. Then, $(A \cup B)^c
      \subseteq A^c \cap B^c$.

      Now suppose $x \in A^c \cap B^c$. Then $x \in A^c$ and $x \in B^c$, which
      is equivalent to $x \notin A$ and $x \notin B$. This means $x \notin A
      \cup B$, so $x \in (A \cup B)^c$. Then, $A^c \cap B^c \subseteq (A \cup
      B)^c$.

      Therefore, since $(A \cup B)^c \subseteq A^c \cap B^c$ and $A^c \cap B^c
      \subseteq (A \cup B)^c$, we have $(A \cup B)^c = A^c \cap B^c$.
      \hfill\qedsymbol
  \end{enumerate}
\end{exersolution}

\newpage

\begin{exercise}[6]
  \begin{enumerate}
    \item Verify the triangle inequality in the special case where $a$ and $b$
      have the same sign.

    \item Find an efficient proof for all the cases at once by first
      demonstrating $(a + b)^2 \le (\lvert a \rvert + \lvert b \rvert)^2$.

    \item Prove $\lvert a - b \rvert \le \lvert a - c \rvert + \lvert c - d
      \rvert + \lvert d - b \rvert$, for all $a$, $b$, $c$, and $d$.

    \item Prove $\lvert \lvert a \rvert - \lvert b \rvert \rvert \le \lvert a -
      b \rvert$. (The unremarkable identity $a = a - b + b$ may be
      useful.)\vspace{-0.5cm}
  \end{enumerate}
\end{exercise}

\begin{exersolution}[6]
  \begin{enumerate}
    \item \textbf{Proof:} Let $a, b \in \R$ such that $a, b > 0$.

      \textbf{Case 1 (Both Positive):} Using the triangle inequality, we have
      $\lvert a + b \rvert \le \lvert a \rvert + \lvert b \rvert$, but since $a,
      b > 0$, we have $\lvert a \rvert = a$ and $\lvert b \rvert = b$.
      Therefore, we have $\lvert a + b \rvert = a + b = \lvert a \rvert + \lvert
      b \rvert$.

      \textbf{Case 2 (Both Negative):} Both $a$ and $b$ will be positive but
      I'll add a negative to make things easier. Using the triangle inequality,
      we have $\lvert (-a) + (-b) \rvert \le \lvert -a \rvert + \lvert -b
      \rvert$. Factoring out the negative, we have $\lvert -(a + b) \rvert \le
      \lvert -a \rvert + \lvert -b \rvert$. Applying the definition of absolute
      value, we have $\lvert -(a + b) \rvert = a + b$, $\lvert -a \rvert = a$,
      and $\lvert -b \rvert = b$. Therefore, we have $\lvert (-a) + (-b) \rvert
      = a + b = \lvert -a \rvert + \lvert -b \rvert$. \hfill\qedsymbol

    \item \textbf{Proof:} Simplifying the expression $(a + b)^2 \le (\lvert a
      \rvert + \lvert b \rvert)^2$ gives us $2ab \le 2\lvert a \rvert \cdot
      \lvert b \rvert$, which is always true, as the left side can be negative,
      but the right side will always be positive. As we've just squared both
      sides which keeps the inequality, the original inequality $a + b \le
      \lvert a \rvert + \lvert b \rvert$ is true. \hfill\qedsymbol

    \item \textbf{Proof:} Notice that $(a - c) + (c - d) + (d - b) = a - b$,
      impliying that $\lvert a - b \rvert = \lvert (a - c) + (c - d) + (d - b)
      \rvert$. Using the triangle inequality, we have $\lvert a - b \rvert =
      \lvert (a - c) + (c - d) + (d - b) \rvert \le \lvert a - c \rvert + \lvert
      c - d \rvert + \lvert d - b \rvert$. \hfill\qedsymbol

    \item \textbf{Proof:} Notice that $\lvert \lvert a \rvert - \lvert b \rvert
      \rvert = \lvert a \rvert - \lvert b \rvert$. Then, $\lvert a \rvert -
      \lvert b \rvert = \lvert (a - b) + b \rvert - \lvert b \rvert \le \lvert a
      - b \rvert$. \hfill\qedsymbol
  \end{enumerate}
\end{exersolution}

\newpage

\begin{exercise}[7]
  Given a function $f$ and a subset $A$ of its domain, let $f(A)$ represent the
  range of $f$ over the set $A$; that is, $f(A) = \left\{f(x) \mid x \in
  A\right\}$.

  \begin{enumerate}
    \item Let $f(x) = x^2$. If $A = [0, 2]$ (the closed interval $\left\{x \in
      \R \mid 0 \le x \le 2\right\}$) and $B = [1, 4]$, find $f(A)$ and $f(B)$.
      Does $f(A \cap B) = f(A) \cap f(B)$ in this case? Does $f(A \cup B) = f(A)
      \cup f(B)$?

    \item Find two sets $A$ and $B$ for which $f(A \cap B) \ne f(A) \cap f(B)$.

    \item Show that, for an arbitrary function $g : \R \to \R$, it is always
      true that $g(A \cap B) \subseteq g(A) \cap g(B)$ for all sets $A, B
      \subseteq \R$.

    \item Form and prove a conjecture about the relationship between $g(A \cup
      B)$ and $g(A) \cup g(B)$ for all sets $A, B \subseteq \R$.\vspace{-0.5cm}
  \end{enumerate}
\end{exercise}

\begin{exersolution}[7]
  \begin{enumerate}
    \item The value of $f(A) = f([0, 2]) = [0, 4]$ and the value of $f(B) =
      f([1, 4]) = [1, 16]$. We have $f(A \cap B) = f([0, 2] \cap [1, 4]) = f([1,
      2]) = [1, 4]$ and $f(A) \cap f(B) = f([0, 2]) \cap f([1, 4]) = [0, 4] \cap
      [1, 16] = [1, 4]$. Therefore, $f(A \cap B) = f(A) \cap f(B)$.

      We have $f(A \cup B) = f([0, 2] \cap [1, 4]) = f([0, 4]) = [0, 16]$ and
      $f(A) \cup f(B) = [0, 4] \cup [1, 16] = [0, 16]$. Therefore, $f(A \cup B)
      = f(A) \cup f(B)$.

    \item Let $A = [-1]$ and $B = [1]$. The value of $f(A) = f([-1]) = [1]$ and
      $f(B) = [1] = [1]$. We have $f(A \cap B) = f([-1] \cap [1]) =
      f(\emptyset)$ and $f(A) \cap f(B) = f([-1]) \cap f([1]) = [1] \cap [1] =
      [1]$.

    \item \textbf{Proof:} Suppose $g(x) \in g(A \cap B)$. Then, $x \in A \cap
      B$, meaning $x \in A$ and $x \in B$.This implies that $g(x) \in g(A)$ and
      $g(x) \in g(B)$. Then, $g(x) \in g(A) \cap g(B)$. Therefore, $g(A \cap B)
      \subseteq g(A) \cap g(B)$. \hfill\qedsymbol

    \item \textbf{Conjecture:} \textit{Given an arbitrary function $g : \R \to
      \R$, it is always true that $g(A \cup B) = g(A) \cup g(B)$ for all sets
      $A, B \subseteq \R$.}

      \textbf{Proof:} Suppose $g(x) \in g(A \cup B)$. Then, $x \in A \cup B$,
      meaning $x \in A$ or $x \in B$. This implies that $g(x) \in g(A)$ or $g(x)
      \in g(B)$. Then, $g(x) \in g(A) \cup g(B)$. Therefore, $g(A \cup B)
      \subseteq g(A) \cup g(B)$.

      Suppose $g(x) \in g(A) \cup g(B)$. Then, $g(x) \in g(A)$ or $g(x) \in
      g(B)$, meaning $x \in A$ or $x \in B$. This implies that $x \in A \cup B$.
      Then, $g(x) \in g(A \cup B)$. Therefore, $g(A) \cup g(B) \subseteq g(A
      \cup B)$.

      Since $g(A \cup B) \subseteq g(A) \cup g(B)$ and $g(A) \cup g(B) \subseteq
      g(A \cup B)$, we have $g(A \cup B) = g(A) \cup g(B)$. \hfill\qedsymbol
  \end{enumerate}
\end{exersolution}

\newpage

\begin{exercise}[11]
  Form the logical negation of each claim. One trivial way to do this is to
  simply add ``It is not the case that $\dots$'' in front of each assertion. To
  make this interesting, fashion the negation into a positive statement that
  avoids using the word ``not'' altogether. In each case, make an intuitive
  guess as to whether the claim or its negation is the true statement.

  \begin{enumerate}
    \item For all real numbers satisfying $a < b$, there exists an $n \in \N$
      such that $\frac{a + 1}{n} < b$.

    \item There exists a real number $x > 0$ such that $x < \frac{1}{n}$ for all
      $n \in \N$.

    \item Between every two distinct real numbers there is a rational
      number.\vspace{-0.5cm}
  \end{enumerate}
\end{exercise}

\begin{exersolution}[11]
  \begin{enumerate}
    \item For all real numbers satisfying $a < b$, there exists an $n \in \N$
      such that $\frac{a + 1}{n} \ge b$. \textit{Intuition:} False.

    \item There exists a real number $x > 0$ such that $x \ge \frac{1}{n}$ for
      all $n \in \N$. \textit{Intuition:} True.

    \item Between every two distinct real numbers there is an irrational number.
      \textit{Intuition:} True.
  \end{enumerate}
\end{exersolution}

\newpage

\begin{exercise}[12]
  Let $y_1 = 6$, and for each $n \in \N$ define $y_{n+1} = \frac{2y_n - 6}{3}$.

  \begin{enumerate}
    \item Use induction to prove that the sequence satisfies $y_n > -6$ for all
      $n \in \N$.

    \item Use another induction argument to show the sequence $(y_1, y_2, y_3,
      \dots)$ is decreasing.\vspace{-0.5cm}
  \end{enumerate}
\end{exercise}

\begin{exersolution}[12]
  \begin{enumerate}
    \item \textbf{Proof:} I'll use mathematical induction.

      Base case: Setting $n = 1$, we get $y_1 = 6 > -6$ as required.

      Induction step: Let $n$ be an arbitrary natural number and suppose that
      $y_{k+1} > -6$. Then
      \[%
        \frac{2y_n - 6}{3} > -6 \implies 2y_n - 6 > -18 \implies 2y_n > -12 \implies y_n > -6
      .\]%
      Therefore, by induction, $y_n > -6$ for all $n \in \N$. \hfill\qedsymbol

    \item \textbf{Proof:} I'll use mathematical induction.

      Base case: Setting $n = 1$, we get $y_1 = 6$. Setting $n = 2$, we get $y_2
      = 2$, giving us $6 \ge 2$ as required.

      Induction step: Let $n$ be an arbitrary natural number and suppose that
      $y_n \ge y_{n+1}$. Then
      \[%
        y_{n+1} - y_{n+2} = \frac{2y_n - 6}{3} - \frac{2y_{n+1} - 6}{3} = \frac{2}{3}\left(y_n - y_{n+1} + 6\right) \ge 0
      .\]%
      Therefore, by induction, $y_n \ge y_{n+1}$ for all $n \in \N$.
      \hfill\qedsymbol
  \end{enumerate}
\end{exersolution}

\newpage

\begin{exercise}[13]
  \begin{enumerate}
    \item Show how induction can be used to conclude that
      \[%
        (A_1 \cup A_2 \cup \cdots \cup A_n)^c = A_1^c \cap A_2^c \cap \cdots \cap A_n^c
      ,\]%
      for any finite $n \in \N$.

    \item It is tempting to appeal to induction to conclude
      \[%
        \left(\bigcup_{i=1}^{\infty} A_i\right)^c = \bigcap_{i=1}^{\infty} A_i^{c}
      ,\]%
      but induction does not apply here. Induction is used to prove that a
      particular statements holds for every real value of $n \in \N$, but this
      does not imply the validity of the infinite case. To illustrate this
      point, find an example of a collection of sets $B_1, B_2, B_3, \dots$
      where $\bigcap_{i=1}^n B_i \ne \emptyset$ is true for every $n \in \N$,
      but $\bigcap_{i=1}^{\infty} B_i \ne \emptyset$ fails.

    \item Nevertheless, the infinite version of De Morgan's Law stated in
      $\circled{2}$ is a valid statement. Provide a proof that does not use
      induction.\vspace{-0.5cm}
  \end{enumerate}
\end{exercise}

\begin{exersolution}[13]
  \begin{enumerate}
    \item \textbf{Proof:} I'll use mathematical induction.

      Base case: Exercise 1.2.5 will be our base case.

      Induction step: Assume that the statement $(A_1 \cup A_2 \cup \cdots \cup
      A_n)^c = A_1^c \cap A_2^c \cap \cdots A_n^c$ is true. Then
      \begin{emptyproof}
        \begin{align*}
          ((A_1 \cup A_2 \cup \cdots \cup A_n) \cup A_{n+1})^c &= (A_1 \cap A_2 \cap \cdots A_n)^c \cap A_{n+1}^c \\
                                                              &= A_1^c \cap A_2^c \cap \cdots A_n^c \cap A_{n+1}^c
        .\qedhere\end{align*}
      \end{emptyproof}

    \item The collection of sets are $B_1 = \{1, 2, \dots\}, B_2 = \{2, 3,
      \dots\}, \dots$. If you take their intersection until $n$, you will always
      get a number, i.e., $n = 100$, you get the singleton set $\{100\}$. But
      the intersection of all the sets as $n \rightarrow \infty$, you get
      $\emptyset$.

    \item \textbf{Proof:} Suppose $x \in \left(\bigcup_{i=1}^{\infty}
      A_i\right)^c$. Then, $x \notin \bigcup_{i=1}^{\infty} A_i$, meaning, $x
      \notin A_i$ for all $i \in \N$. This implies that $x \in A_i^c$ for all $i
      \in \N$, so $x \in \bigcap_{i=1}^{\infty} A_i^c$. Therefore,
      $\left(\bigcup_{i=1}^{\infty} A_i\right)^c \subseteq
      \bigcap_{i=1}^{\infty} A_i^c$.

      Suppose $x \in \bigcap_{i=1}^{\infty} A_i^c$, meaning $i \in \N$, $x
      \notin A_i$, for some $i \in \N$. This implies that $x \in A_i^c$, for
      some $i \in \N$. Therefore, $x \notin \bigcup_{i=1}^{\infty} A_i$, meaning
      $x \in \left(\bigcup_{i=1}^{\infty} A_i\right)^c$. Therefore,
      $\bigcap_{i=1}^{\infty} A_i^c \subseteq \left(\bigcup_{i=1}^{\infty}
      A_i\right)^c$.

      Since $\left(\bigcup_{i=1}^{\infty} A_i\right)^c \subseteq
      \bigcap_{i=1}^{\infty} A_i^c$ and $\bigcap_{i=1}^{\infty} A_i^c \subseteq
      \left(\bigcup_{i=1}^{\infty} A_i\right)^c$, we have
      $\left(\bigcup_{i=1}^{\infty} A_i\right)^c = \bigcap_{i=1}^{\infty}
      A_i^c$. \hfill\qedsymbol
  \end{enumerate}
\end{exersolution}

\newpage

\setcounter{section}{3}

\begin{exercise}[3]
\end{exercise}

\begin{exersolution}[3]
\end{exersolution}

\newpage

\begin{exercise}[8]
\end{exercise}

\begin{exersolution}[8]
\end{exersolution}
