\begin{center}
  SECTION 1.3
\end{center}

\medskip

\setcounter{chapter}{1}
\setcounter{section}{3}

\begin{exercise}[1]
  \begin{enumerate}
    \item Write a formal definition in the style of Definition 1.3.2 for the
      \textit{infimum} or \textit{greatest lower bound} of a set.

    \item Now, state and prove a version of Lemma 1.3.8 for greatest lower
      bound.
  \end{enumerate}
\end{exercise}

\begin{proof}[Solution to (i)]
  A real number $t$ is the \textit{infimum} or \textit{greatest lower bound} of
  a set $S$ if
  \begin{enumerate}
    \item $t$ is a lower bound of $S$, and
    \item if $t'$ is any lower bound of $S$, then $t \leq t'$. \qedhere
  \end{enumerate}
\end{proof}

\begin{proof}[Solution to (ii)]
  Assume $s \in \R$ is a lower bound for a set $A \subseteq \R$. Then, $s =
  \inf(A)$ if and only if for all $\epsilon > 0$, there exists an element $a \in
  A$ such that $s + \epsilon > a$.

  \textit{Proof.} Assume $s = \inf(A)$ and consider $s + \epsilon$ for some
  $\epsilon > 0$. Then, $s + \epsilon$ cannot be a lower bound on $A$ because
  (ii) implies all lower bounds $b$ must be such that $s \leq b$. Therefore,
  there must exist an element $a \in A$ such that $s + \epsilon > a$.

  Conversely, for all $\epsilon > 0$, there exists an $a \in A$ such that $s +
  \epsilon > a$. Then, $s + \epsilon$ is not a lower bound for all $\epsilon$,
  which is the same as saying every lower bound $b$ must have $b \le s$ (ii).
\end{proof}

\medskip

\begin{exercise}[7]
  Prove that if a is an upper bound for $A$, and if $a$ is also an element of
  $A$, then it must be that $a = \sup(A)$.
\end{exercise}

\begin{proof}[Solution]
  If $a$ is an upper bound for the set $A \subseteq \R$ and is an element of
  $A$, then, by definition, $a = \max(A) = \sup(A)$.
\end{proof}

\medskip

\begin{exercise}[9]
  \begin{enumerate}
    \item If $\sup(A) < \sup(B)$, show that there exists an element $b \in B$
      that is an upper bound for $A$.

    \item Give an example to show that this is not always the case if we only
      assume $\sup(A) \le \sup(B)$.
  \end{enumerate}
\end{exercise}

\begin{proof}[Solution to (i)]
  We'll prove this case by case.

  Case 1: If $\sup(B) = \max(B)$, then we can choose $b = \sup(B)$ and $b \in
  B$.

  Case 2: Since $\sup(B) > \sup(A)$, then, by the Theorem 1.4.3 [Density of $\Q$
  in $\R$], there exists some $c$ in between $\sup(A)$ and $\sup(B)$. Let $c =
  \sfrac{\sup(A) + \sup(B)}{2}$. Then, $c \in B$, since $\sup(B) > c$.
  Therefore, $\sup(B) > c > \sup(A)$ and $c \in B$.
\end{proof}

\begin{proof}[Solution to (ii)]
  Let $A = (-\infty, 1]$ and $B = (-\infty, 1)$, then $\sup(A) = 1$ and $\sup(B)
  = 1$. However, there is no element $b \in B$ that is an upper bound for $A$.
\end{proof}

\medskip

\begin{exercise}[11]
  Decide if the following statements about suprema and infima are true or false.
  Give a short proof for those that are true. For any that are false, supply an
  example where the claim in question does not appear to hold.
  \begin{enumerate}
    \item If $A$ and $B$ are nonempty, bounded, and satisfy $A \subseteq B$,
      then $\sup(A) \le \sup(B)$.

    \item If $\sup(A) < \inf(B)$ for sets $A$ and $B$, then there exists a $c
      \in \R$ satisfying $a < c < b$ for all $a \in A$ and $b \in B$.

    \item If there exists a $c \in \R$ satisfying $a < c < b$ for all $a \in A$
      and $b \in B$, then $\sup(A) < \inf(B)$.
  \end{enumerate}
\end{exercise}

\begin{proof}[Solution to (i)]
  True: If $A \subseteq B$, this means that $\forall x \in A \implies x \in B$.
  This means that $\sup(A) \in B$, but not always the case that $\sup(B) \in A$.
  Therefore, $\sup(A) \le \sup(B)$.
\end{proof}

\begin{proof}[Solution to (ii)]
  True: Since $\sup(A) < \sup(B)$, then, by the Theorem 1.4.3 [Density of $\Q$
  in $\R$], there exists some $c$ in between $\sup(A)$ and $\sup(B)$. Let $c =
  \sfrac{\sup(A) + \sup(B)}{2}$. Then, $c \in \R$ and $a < c < b$ for all $a \in
  A$ and $b \in B$.
\end{proof}

\begin{proof}[Solution to (iii)]
  False: Let $A = (-\infty, 1]$ and $B = [1, \infty)$, then $1 < 1$.
\end{proof}

\medskip

\begin{center}
  SECTION 1.4
\end{center}

\medskip

\setcounter{section}{4}

\begin{exercise}[1]
  Recall that $\I$ stands for the set of irrational numbers.
  \begin{enumerate}
    \item Show that if $a, b \in \Q$, then $ab$ and $a + b$ are elements of $\Q$
      as well.

    \item Show that if $a \in \Q$ and $\I$, then $a + t \in \I$ and $at \in \I$
      as long as $a \ne 0$.

    \item Part (i) can be summarized by saying that $\Q$ is closed under
      addition and multiplication. Is $\I$ closed under addition and
      multiplication? Given two irrational numbers $s$ and $t$, what can we say
      about $s + t$ and $st$?
  \end{enumerate}
\end{exercise}

\begin{proof}[Solution to (i)]
  The rational numbers $a$ and $b$ can be expressed as $a = \sfrac{p}{q}$ and $b
  = \sfrac{r}{s}$, where $p, q, r, s \in \Z$ and $q, s \ne 0$. Then, $a + b =
  \sfrac{p}{q} \sfrac{r}{s} = \frac{ps + rq}{qs}$, where $ps + rq \in \Z$ and
  $qs \in \Z$. Therefore, $a + b \in \Q$. Then, $ab = \sfrac{p}{q} \cdot
  \sfrac{r}{s} = \sfrac{pr}{qs}$, where $pr \in \Z$ and $qs \in \Z$. Therefore,
  $ab \in \Q$.
\end{proof}

\begin{proof}[Solution to (ii)]
  Suppose $a + t \in \Q$, then, by (i), $(a + t) - a \in \Q$, contradicting the
  initial assumption that $t \in \I$.
\end{proof}

\begin{proof}[Solution to (iii)]
  No, $\I$ is not closed under addition and multiplication. For example,
  $\sqrt{2} - \sqrt{2} = 0$ and $\sqrt{2} \cdot \sqrt{2} = 2$.
\end{proof}

\medskip

\begin{exercise}[3]
  Prove that $\bigcap_{n=1}^{\infty} (0, \sfrac{1}{n}) = \emptyset$. Notice that
  this demonstrates that the intervals in the Nested Interval Property must be
  closed for the conclusion of the theorem to hold.
\end{exercise}

\begin{proof}[Solution]
  Suppose $x \in \bigcap_{n=1}^{\infty} (0, \sfrac{1}{n})$, then we have $0 < x
  < \sfrac{1}{n}$ for all $n \in \N$. However, this is impossible by the
  Archimedean Property. Therefore, $\bigcap_{n=1}^{\infty} (0, \sfrac{1}{n}) =
  \emptyset$.
\end{proof}

\medskip

\begin{exercise}[4]
  Let $a < b$ be real numbers and consider the set $T = \Q \cap [a, b]$. Show
  $\sup(T) = b$.
\end{exercise}

\begin{proof}[Solution]
  The intersection of $\Q$ and $[a, b]$ is the interval $[a, b]$. Since $b$ is
  an upper bound for $T$, then $b = \sup(T)$.
\end{proof}
