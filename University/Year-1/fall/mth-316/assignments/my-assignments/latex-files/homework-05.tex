\begin{center}
  SECTION 2.4
\end{center}

\medskip

\setcounter{chapter}{2}
\setcounter{section}{4}

\begin{exercise}[7]
  Let $(a_n)$ be a bounded sequence.
  \begin{enumerate}
    \item Prove that the sequence defined by $y_n = \sup(\{a_k \mid k \ge n\})$
      converges.

    \item The \textit{limit superior} of $(a_n)$, or $\lim_{n \to \infty}
      \sup(a_n)$, is defined by
      \[%
        \lim_{n \to \infty} \sup(a_n) = \lim_{n \to \infty} y_n
      ,\]%
      where $y_n$ is the sequence from part (i) of this exercise. Provide a
      reasonable definition for $\lim_{n \to \infty} \inf(a_n)$ and briefly
      explain why it always exists for any bounded sequence.

    \item Prove that $\lim_{n \to \infty} \inf(a_n) \le \lim_{n \to \infty}
      \sup(a_n)$ for every bounded sequence and give an example of a sequence
      for which the inequality is strict.

    \item Show that $\lim_{n \to \infty} \inf(a_n) = \lim_{n \to \infty}
      \sup(a_n)$ if and only if $\lim_{n \to \infty} a_n$ exists. In this case,
      all three share the same value.
  \end{enumerate}
\end{exercise}

\begin{proof}[Solution to (i)]
  Notice that as $n$ increases, the set $\{a_k \mid k \ge n\}$ becomes
  smaller or stays the same. Thus,
  \[%
    \left\{a_k \mid k \ge n + 1\right\} \subseteq \left\{a_k \mid k \ge n\right\}
  .\]%
  Because the supremum of a subset cannot exceed the supremum of the
  larger set containing it, we have
  \[%
    y_{n+1} = \sup\left(\left\{a_k \mid k \ge n + 1\right\}\right) \le \sup\left(\left\{a_k \mid k \ge n\right\}\right) = y_n
  .\]%
  Therefore, $(y_n)$ is a decreasing sequence. Since $(a_n)$ is bounded,
  then there exists an $M$ such that $\lvert a_n \rvert \le M$, for all $n
  \in \N$. This implies that $y_n \le M$, for all $n \in \N$. Thus,
  $(y_n)$ is a decreasing sequence bounded below by $M$. By the Monotone
  Convergence Theorem, $(y_n)$ converges.
\end{proof}

\begin{proof}[Solution to (ii)]
  The \textit{limit inferior} of $(a_n)$, or $\lim_{n \to \infty}
  \inf(a_n)$, is defined by
  \[%
    \lim_{n \to \infty} \inf(a_n) = \lim_{n \to \infty} x_n
  ,\]%
  where $x_n = \inf(\{a_k \mid k \ge n\})$. The limit inferior always exists
  for any bounded sequence because $(x_n)$ is an increasing sequence bounded
  above by $M$, where $M$ is the upper bound of $(a_n)$.
\end{proof}

\begin{proof}[Solution to (iii)]
  Assume $\lim_{n \to \infty} \sup(a_n) = \lim_{n \to \infty} y_n$, where $y_n =
  \sup\left(\left\{a_k \mid k \ge n\right\}\right)$ and $\lim_{n \to \infty}
  \inf(a_n) = \lim_{n \to \infty} z_n$, where $z_n = \inf\left(\left\{a_k \mid k
  \ge n\right\}\right)$. Since $y_n$ is the least upper bound for the set $\{a_k
  \mid k \ge n\}$ and $z_n$ is the greatest lower bound for the set $\{a_k \mid
  k \ge n\}$, then
  \[%
    z_n \le y_n\quad\textrm{for all~}n \in \N
  .\]%
  Taking the limit of both sides, we get
  \[%
    \lim_{n \to \infty} z_n \le \lim_{n \to \infty} y_n
  .\]%
  Thus, we get
  \[%
    \lim_{n \to \infty} \inf(a_n) \le \lim_{n \to \infty} \sup(a_n)
  .\]%

  \textit{Example.} Consider the sequence $(a_n) = (-1)^n + \sfrac{1}{n}$ as $n
  \to \infty$.
  \begin{enumerate}
    \item For an even $n$, $a_n = 1 + \sfrac{1}{n}$.
    \item For an odd $n$, $a_n = -1 + \sfrac{1}{n}$.
  \end{enumerate}
  For $\lim_{n \to \infty} \sup(a_n)$, for large $n$ the supremum of the terms
  $\{a_k \mid k \ge n\}$ will be close to $1$. For $\lim_{n \to \infty}
  \inf(a_n)$, for large $n$ the infimum of the terms $\{a_k \mid k \ge n\}$ will
  be close to $-1$. Thus, $\lim_{n \to \infty} \inf(a_n) < \lim_{n \to \infty}
  \sup(a_n)$.
\end{proof}

\begin{proof}[Solution to (iv)]
  Assume $\lim_{n \to \infty} a_n = a$ exists. Then, for any $\epsilon > 0$,
  there exists $N \in \N$ such that for all $n \ge N$, $\lvert a_n - a \rvert <
  \epsilon$. This implies that for sufficiently large $n$, all terms in the set
  $\{a_k \mid k \ge n\}$ are within $\epsilon$ of $a$, so both $y_n = \sup(\{a_k
  \mid k \ge n\})$ and $z_n = \inf(\{a_k \mid k \ge n\})$ are within $\epsilon$
  of $a$. Thus, $y_n \to a$ and $z_n \to a$ as $n \to \infty$. Therefore,
  $\lim_{n \to \infty} \inf(a_n) = \lim_{n \to \infty} a_n = \lim_{n \to \infty}
  \sup(a_n)$.

  Assume $\lim_{n \to \infty} \inf(a_n) = a = \lim_{n \to \infty} \sup(a_n)$.
  Then, for any $\epsilon > 0$, there exists $N \in \N$ such that for all $n \ge
  N$, $\lvert y_n - a \rvert < \epsilon$ and $\lvert z_n - a \rvert < \epsilon$.
  Since $z_n \le a \le y_n$, then for all $n \ge N$, it follows that $a_n$ is
  squeezed within $\epsilon$ of $a$. By Squeeze Theorem, $\lim_{n \to \infty}
  a_n = a$.

  Therefore, $\lim_{n \to \infty} \inf(a_n) = \lim_{n \to \infty} a_n = \lim_{n
  \to \infty} \sup(a_n)$ if and only if $\lim_{n \to \infty} a_n$ exists.
\end{proof}

\medskip

\begin{center}
  SECTION 2.5
\end{center}

\medskip

\setcounter{section}{5}

\begin{exercise}[1]
  Give an example of each of the following, or argue that such a request is
  impossible.
  \begin{enumerate}
    \item A sequence that has a subsequence that is bounded but contains no
      subsequence that converges.

    \item A sequence that does not contain $0$ or $1$ as a term but contains
      subsequences converging to each of these values.

    \item A sequence that contains subsequences converging to every point in the
      infinite set $\{1, \sfrac{1}{2}, \sfrac{1}{3}, \sfrac{1}{4}, \dots\}$.

    \item A sequence that contains subsequences converging to every point in the
      infinite set $\{1, \sfrac{1}{2}, \sfrac{1}{3}, \sfrac{1}{4}, \dots\}$, and
      no subsequences converging to points outside of this set.
  \end{enumerate}
\end{exercise}

\begin{proof}[Solution to (i)]
  Impossible, as by the Bolzano-Weierstrass theorem, a convergent subsequence of
  that subsequence exists, and that sub-subsequence is also a subsequence of the
  original sequence.
\end{proof}

\begin{proof}[Solution to (ii)]
  Consider the sequence $(2 + \sfrac{1}{n}) \to 2$. The subsequences $(1 +
  \sfrac{1}{n}) \to 1$ and $(\sfrac{1}{n}) \to 0$ converge to $1$ and $0$
  respectfully and the original sequence does not contain $0$ or $1$.
\end{proof}

\begin{proof}[Solution to (iii)]
  We can construct $(a_n)$ by defining it as follows
  \begin{enumerate}
    \item For each $k \in \N$, repeat the terms $1$, $\frac{1}{2}$,
      $\frac{1}{3}$, $\dots$, $\frac{1}{k}$ in that order.

    \item For example, start with $1$, $\frac{1}{2}$, $1$, $\frac{1}{3}$,
      $1$, $\frac{1}{2}$, $\frac{1}{4}$, $\dots$ and continue so that every
      $\frac{1}{k}$ appears infinitely many times.
  \end{enumerate}

  Each point $\frac{1}{k}$ in the set $\{1, \frac{1}{2}, \frac{1}{3},
  \dots\}$ has an associated subsequence of $(a_n)$  consisting entirely of
  terms equal to $\frac{1}{k}$, which converges to $\frac{1}{k}$ itself. For
  example,
  \begin{enumerate}
    \item The subsequence $(a_{n_j})$ with $a_{n_j} = 1$ for all $j$
      converges to $1$.

    \item The subsequence $(a_{n_j}) = \frac{1}{2}$ for all $j$ converges to
      $\frac{1}{2}$.

    \item Similarly, there are subsequences converging to $\frac{1}{3}$,
      $\frac{1}{4}$, and so on. \qedhere
  \end{enumerate}
\end{proof}

\begin{proof}[Solution to (iv)]
  Impossible, the sequence must converge to zero which is not in the set.

  Let $\epsilon > 0$. Choose $N$ large enough that $\sfrac{1}{n} <
  \sfrac{\epsilon}{2}$ for all $n \ge N$. We can find a subsequence $(b_k) \to
  \sfrac{1}{n}$, meaning $\lvert b_k - \sfrac{1}{n} \rvert <
  \sfrac{\epsilon}{2}$ for some $k \in \N$. Using the Triangle Inequality, we
  get $\lvert b_k - 0 \rvert \le \lvert b_n - \sfrac{1}{n} \rvert + \lvert
  \sfrac{1}{n} + 0 \rvert < \sfrac{\epsilon}{2} + \sfrac{\epsilon}{2} =
  \epsilon$.
\end{proof}

\medskip

\begin{exercise}[3]
  \begin{enumerate}
    \item Prove that if an infinite series converges, then the associative
      property holds. Assume $a_1 + a_2 + a_3 + a_4 + a_5 + \cdots$ converges to
      a limit $L$ (i.e., the sequence of partial sums $(s_n) \to L$). Show that
      any regrouping of the terms
      \[%
        (a_1 + a_2 + \cdots + a_{n_1}) + (a_{n_1 + 1} + \cdots + a_{n_2}) + (a_{n_2 + 1} + \cdots + a_{n_3}) + \cdots
      \]%
      leads to a series that also converges to $L$.

    \item Compare this result to the example discussed at the end of Section 2.1
      where infinite addition was shown not to be associate. Why doesn't our
      proof in (i) apply to this example?
  \end{enumerate}
\end{exercise}

\begin{proof}[Solution to (i)]
  Let $s_n$ be the original grouping of the terms and converge to $L$. Let
  $s_n'$ be a subsequence of $s_n$ that has a different grouping of addition. By
  Theorem 2.5.2, since $s_n$, the original sequence, converges, then $s_n'$, the
  subsequence, also converges to $L$.
\end{proof}

\begin{proof}[Solution to (ii)]
  The example discussed at the end of Section 2.1 where infinite addition was
  shown not to be associative was the series $1 - 1 + 1 - 1 + \cdots$. The proof
  in (i) does not apply to this example because the series does not converge.
  The series $1 - 1 + 1 - 1 + \cdots$ does not converge because the sequence of
  partial sums oscillates between $0$ and $1$.
\end{proof}

\medskip

\begin{center}
  SECTION 2.6
\end{center}

\medskip

\setcounter{section}{6}

\begin{exercise}[2]
  Give an example of each of the following, or argue that such a request is
  impossible.
  \begin{enumerate}
    \item A Cauchy sequence that is not monotone.
    \item A Cauchy sequence with an unbounded subsequence.
    \item A divergent monotone sequence with a Cauchy subsequence.
    \item An unbounded sequence containing a subsequence that is Cauchy.
  \end{enumerate}
\end{exercise}

\begin{proof}[Solution to (i)]
  By Theorem 2.6.2, a converging sequence is also Cauchy. Therefore, the
  following sequence $x_n = \frac{(-1)^n}{n}$ is Cauchy.
\end{proof}

\begin{proof}[Solution to (ii)]
  Impossible, since all Cauchy sequences converge, meaning they are bounded.
\end{proof}

\begin{proof}[Solution to (iii)]
  Impossible, if a subsequence was Cauchy, it would converge implying it would
  be bounded. But this would also imply that the original sequence would be
  bounded since the subsequence is a monotone sequence. That would imply that
  the original sequence is monotone and bounded, meaning it would converge.
\end{proof}

\begin{proof}[Solution to (iv)]
  The sequence $a_n = (2, \sfrac{1}{2}, 3, \sfrac{1}{3}, \dots)$ has the
  subsequence $(\sfrac{1}{2}, \sfrac{1}{3}, \dots)$ which is Cauchy.
\end{proof}

\medskip

\begin{exercise}[3]
  If $(x_n)$ and $(y_n)$ are Cauchy sequences, then one easy way to prove that
  $(x_n + y_n)$ is Cauchy is to use the Cauchy Criterion. By Theorem 2.6.4,
  $(x_n)$ and $(y_n)$ must be convergent, and the Algebraic Limit Theorem then
  implies $(x_n + y_n)$ is convergent and hence Cauchy.
  \begin{enumerate}
    \item Give a direct argument that $(x_n + y_n)$ is a Cauchy sequence that
      does not use the Cauchy Criterion or the Algebraic Limit Theorem.

    \item Do the same for the product $(x_ny_n)$.
  \end{enumerate}
\end{exercise}

\begin{proof}
  Let $\epsilon > 0$. Since $(x_n)$ converges, then $(\exists N_1 \in
  \N)(\forall n \ge N_1)[\lvert x_n - x \rvert < \sfrac{\epsilon}{2}]$. Same
  thing for $(y_n)$, $(\exists N_2 \in \N)(\forall n \ge N_2)[\lvert y_n - y
  \rvert < \sfrac{\epsilon}{2}]$. Choose $N = \max(\{N_1, N_2\})$. Then, for all
  $n \ge N$, we have $\lvert (x_n + y_n) - (x - y) \rvert \le \lvert x_n - x
  \rvert + \lvert y_n - y \rvert < \sfrac{\epsilon}{2} + \sfrac{\epsilon}{2} =
  \epsilon$. Therefore, $(x_n + y_n)$ converges, meaning it's a Cauchy sequence.
\end{proof}

\begin{proof}
  Let $\epsilon > 0$. Since $(x_n)$ converges, then $(\exists N_1 \in
  \N)(\forall n \ge N_1)[\lvert x_n - x \rvert < \sfrac{\epsilon}{2M_1}]$. Same
  thing for $(y_n)$, $(\exists N_2 \in \N)(\forall n \ge N_2)[\lvert y_n - y
  \rvert < \sfrac{\epsilon}{2M_2}]$. Choose $N = \max(\{N_1, N_2\})$. Bound
  $\lvert x_n \rvert \le M_1$ and $\lvert y_n \rvert \le M_2$, for all $n \in
  \N$. Then, for all $n \ge N$, we have
  \begin{align*}
    \lvert (x_ny_n) - (xy) \rvert &= \lvert (x_ny_n) - (y_nx) + (y_nx) - (xy) \rvert \\
                                  &\le \lvert x_ny_n - y_nx \rvert + \lvert y_nx - xy \rvert \\
                                  &\le M_1 \cdot \lvert x_n - x \rvert + M_2 \cdot \lvert y_n - x \rvert \\
                                  &< M_1 \cdot \frac{\epsilon}{2M_1} + M_2 \cdot \frac{\epsilon}{2M_2}  = \epsilon
  .\end{align*}
  Therefore, $(x_ny_n)$ converges, meaning it's a Cauchy sequence.
\end{proof}
