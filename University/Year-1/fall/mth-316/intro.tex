{\small
  \noindent\textbf{Rigorous Treatment of Sequential Limits} \\
  We will provide a formal treatment of sequential limits using the $\epsilon$--$N$ definition. Topics include the limit of a sequence, the limit superior, and the limit inferior. We will prove various properties of limits, including uniqueness, algebraic properties, and the squeeze theorem. Additional topics include subsequences and their convergence properties, the Bolzano–Weierstrass theorem, the Monotone Convergence Theorem, and the characterization of Cauchy sequences.

  \vspace{10pt}
  \noindent\textbf{Infinite Series} \\
  We will introduce the concept of infinite series and the notion of convergence. The course will cover several convergence tests, including (but not limited to) the Comparison Test, Ratio Test, Root Test, and Integral Test, with proofs of each. We will also study conditional and absolute convergence, and conclude with the Riemann Series Theorem.

  \vspace{10pt}
  \noindent\textbf{Basic Topology of $\R$} \\
  We will develop the basic topology of $\R$, including open and closed sets, limit points, closure, and interior of sets. We will prove the Heine--Borel Theorem, which characterizes compact subsets of $\R^n$, and study the concept of connectedness. In particular, we will prove that closed and bounded subsets of $\R$ are compact, and that the continuous image of a compact set is compact. We will also discuss perfect sets and connected sets.

  \vspace{10pt}
  \noindent\textbf{Functional Limits and Continuity} \\
  We will extend the concept of limits to functions of one variable, introducing the $\epsilon$--$\delta$ definition of a limit and of continuity. We will prove the algebraic properties of limits for functions, including limits of sums, products, and quotients. Additional results will include the characterization of continuous functions and their behavior on compact sets.

  \vspace{10pt}
  \noindent\textbf{Pre-requisites} \\
  This course is intended for students with a solid understanding of material through MTH~253 (Calculus~III). The primary tools will be limits, sequences, series, and continuity. The following course in the sequence, MTH~317, will treat functional limits, differentiation, and integration in a similarly rigorous framework. Familiarity with mathematical proofs and logic is recommended but not required, as this course will also serve as an introduction to the foundations of mathematical analysis and formal proof writing.
}
