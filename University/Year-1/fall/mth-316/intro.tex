{\small
  \noindent\textbf{Rigorous treatments of sequential limits}\\
  Formal treatment of sequential limits, using the $\epsilon$-N definition. This
  includes the limit of a sequence, the limit superior, and limit inferior.
  Prove varies properties of limits, including uniqueness of limits, algebraic
  properties of limits, and the squeeze theorem. We will also discuss
  subsequences and their convergence properties, including the
  Bolzano-Weierstrass theorem, Monotone Convergence Theorem, and the concept of
  Cauchy sequences.\hspace*{\fill}

  \vspace{10pt}
  \noindent\textbf{Infinite Series}\\
  Introduce the concept of infinite series and convergence of series. We will
  prove varies tests for convergence that was talked about in MTH-253, including
  but not limited to the Comparison Test, Ratio Test, Root Test, and Integral
  Test. We will also discuss conditional and absolute convergence, and the
  Riemann series theorem.\hspace*{\fill}

  \vspace{10pt}
  \noindent\textbf{Basic Topology of $\R$}\\
  Introduce the basic topology of $\R$, including open and closed sets, limit
  points, closure, and interior of sets. We will also discuss the Heine-Borel
  theorem, which characterizes the compact subsets of $\R^n$, and the concept of
  connectedness. This will include proving that closed and bounded subsets of
  $\R$ are compact, and that the continuous image of a compact set is also
  compact. Lastly, we'll talk about perfect and connected sets.

  \vspace{10pt}
  \noindent\textbf{Functional Limits and Continuity}\\
  Lastly, we will extend the concept of limits to functions of one variable. We
  will define the limit of a function at a point, and discuss the
  $\epsilon$-$\delta$ definition of continuity. We will prove the algebraic
  properties of limits for functions, including the limit of sums, products, and
  quotients.\hspace*{\fill}
}
