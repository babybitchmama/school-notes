\begin{problem}[5.2.4a]
  Show that a function $h : A \to \R$ is differentiable at $a \in A$ if and only
  if there exists a function $l : A \to \R$ which is continuous at $a$ and
  satisfies
  \[%
    h(x) - h(a) = l(x)(x - a)\quad\textrm{for all}~x \in A
  .\]%
\end{problem}

\begin{proof}[Solution]
  Assume $h$ is differentiable at $a$. Define
  \[%
    l(x) = \begin{cases}
      \frac{h(x) - h(a)}{x - a} & \textrm{if}~x \neq a \\
      h'(a) & \textrm{if}~x = a
    \end{cases}
  .\]%
  Then $l$ is continuous at $a$ and satisfies the equation $h(x) - h(a) = l(x)(x
  - a)$ for all $x \in A$.

  Conversely, assume there exists a function $l : A \to \R$ which is continuous
  at $a$ and satisfies the equation $h(x) - h(a) = l(x)(x - a)$ for all $x \in
  A$. Rewriting the equation, we have
  \[%
    \frac{h(x) - h(a)}{x - a} = l(x)
  ,\]%
  where $x \ne a$. Taking the limit on both sides as $x \to a$ gives us $h'(a) =
  l(a)$.
\end{proof}

\begin{problem}[5.2.6b]
  Let $g$ be defined on an interval $A$, and let $c \in A$. Assume $A$ is open.
  If $g$ is differentiable at $c \in A$, show
  \[%
    g'(c) = \lim_{h \to 0} \frac{g(c + h) - g(c - h)}{2h}
  .\]%
\end{problem}

\begin{proof}[Solution]
  Rewriting the limit, we have
  \[%
    \lim_{h \to 0} \frac{g(c + h) - g(c) + g(c) - g(c - h)}{2h} = \frac{1}{2} \underbrace{\lim_{h \to 0} \frac{g(c + h) - g(c)}{h}}_{g'(c)} + \frac{1}{2} \underbrace{\lim_{h \to 0} \frac{g(c) - g(c - h)}{h}}_{g'(c)~\textrm{with}~h=c-x} = \frac{1}{2} \cdot 2g'(c) = g'(c)
  .\]%
  Thus, $g'(c) = \lim_{h \to 0} \frac{g(c + h) - g(c - h)}{2h}$.
\end{proof}

\begin{problem}[5.2.7]
  Find a particular (potentially non integer) value for $a$ so that
  \begin{enumerate}
    \item $g_a$ is differentiable on $\R$ but such that $g_a'$ is unbounded on
      $[0, 1]$.

    \item $g_a$ is differentiable on $\R$ with $g_a'$ continuous but not
      differentiable at zero.

    \item $g_a$ is differentiable on $\R$j and $g_a'$ is differentiable on $\R$,
      but such that $g_a''$ is not continuous at zero.
  \end{enumerate}
\end{problem}

\begin{proof}[Solution to (i)]
  Pick $a = 1.5$. The first derivative of $g_a$ is $ax^{a-1}\sin(\sfrac{1}{x}) -
  x^{a-2}\cos(\sfrac{1}{x})$. To make $g_a'$ unbounded on $[0, 1]$, we need the
  second term to be unbounded. This can be achieved by choosing any $1 < a < 2$.
\end{proof}

\begin{proof}[Solution to (ii)]
  Pick $a = 2.5$. The first derivative, $g_a' = 2.5x^{1.5}\sin(\sfrac{1}{x}) -
  x^{0.5}\cos(\sfrac{1}{x})$, is continuous at zero, but isn't differentiable at
  zero, as the second term isn't differentiable at zero, and the sum of two
  differentiable functions is differentiable if and only if both functions are
  differentiable. This can be achieved by choosing any $2 < a < 3$.
\end{proof}

\begin{proof}[Solution to (iii)]
  Pick $a = 3.5$. The first derivative and second derivative are
  \begin{align*}
    g_a' &= -x^{1.5}\cos(\sfrac{1}{x}) + 3.5x^{2.5}\sin(\sfrac{1}{x}) \\
    g_a'' &= -5x^{0.5}\cos(\sfrac{1}{x}) - \frac{\sin(\sfrac{1}{x})}{x^{0.5}} + 8.75x^{1.5}\sin(\sfrac{1}{x})
  .\end{align*}
  Clearly, $g_a'$ is differentiable on $\R$, and $g_a''$ is not continuous at
  zero. This can be achieved by choosing any $3 < a < 4$.
\end{proof}

\begin{problem}[5.2.9]
  Decide whether each conjecture is true or false. Provide an argument for those
  that are true and a counterexample for each one that is false.
  \begin{enumerate}
    \item If $f'$ exists on an interval and is not constant, then $f'$ must take
      on some irrational values.

    \item If $f'$ exists on an open interval and there is some point $c$ where
      $f'(c) > 0$, then there exists a $\delta$-neighborhood $V_\delta(c)$
      around $c$ in which $f'(x) > 0$ for all $x \in V_\delta(c)$.

    \item If $f$ is differentiable on an interval containing zero and if
      $\lim_{x \to 0} f'(x) = L$, then it must be that $L = f'(0)$.
  \end{enumerate}
\end{problem}

\begin{proof}[Solution to (i)]
  It's true. If $f'$ is not constant, then there must exist two points, $x_1$
  and $x_2$, such that $f'(x_1) < f'(x_2)$, or $f'(x_1) > f'(x_2)$. By the
  intermediate value theorem, there must exist a point $x_3$ such that $f'(x_1)
  < f'(x_3) < f'(x_2)$, and $f'(x_3)$ is irrational.
\end{proof}

\begin{proof}[Solution to (ii)]
  It's true if $f'$ is continuous at $c$. If $f'$ is continuous at $c$, then
  there exists a $\delta$-neighborhood $V_\delta(c)$ around $c$ such that $f'(x)
  > 0$ for all $x \in V_\delta(c)$.

  Otherwise, it's false. Assume $f'$ isn't continuous. Consider the function
  $f(x) = \sfrac{x}{2} + x^2\sin(\sfrac{1}{x})$. The derivative of $f$ is
  \[%
    f' = 2x\sin(\sfrac{1}{x}) - \cos(\sfrac{1}{x}) + \frac{1}{2}
  .\]%
  The function $f'$ keeps alternating between positive and negative values as $x
  \to 0$. Define $x_n = \sfrac{1}{2\pi n}$, where $n \in \N$. Then $f'(x_n) =
  -\sfrac{1}{2}$, but $f'(0) = \sfrac{1}{2}$.
\end{proof}

\begin{proof}[Solution to (iii)]
  It's true. Suppose $\lim_{x \to 0} f'(x) = L$ with $f'(0) \ne L$. Let
  \[%
    0 < \epsilon = \frac{|f'(0) - L|}{2}
  .\]%
  There exists a $\delta > 0$ such that $f'(x) \in V_L(\epsilon)$, for $x \in
  V_\delta(0)$. By definition, $f'(0) \in V_L(\epsilon)$. Hence, there's a gap
  between $f'(x)$ and $f'(0)$, which, by Darboux's theorem, is a contradiction.
\end{proof}

\begin{problem}[5.3.1a]
  Recall from Exercise 4.4.9 that a function $f : A \to \R$ is Lipschitz on $A$
  if there exists an $M > 0$ such that
  \[%
    \left\lvert \frac{f(x) - f(y)}{x - y} \right\rvert \le M
  ,\]%
  for all $x \ne y$ in $A$.

  Show that if $f$ is differentiable on a closed interval $[a, b]$ and if $f'$
  is continuous on $[a, b]$, then $f$ is Lipschitz on $[a, b]$.
\end{problem}

\begin{proof}[Solution]
  Every closed interval on $\R$ is compact. Let $(x_n)$ be a sequence
  contained entirely in $[a, b]$. By the Bolzano-Weierstrass theorem, there
  exists a convergent subsequence $(x_{n_k})$ that converges to some limit
  $L$. Since $[a, b]$ is closed, $L \in [a, b]$. Therefore, $[a, b]$ is compact.

  Using that fact, $f'$ is continuous on a compact set $[a, b]$. Let $M$ be the
  maximum of $[f'(a), f'(b)]$ (or $[f'(b), f'(a)]$, as $f'$ is defined to be
  monotone). Pick $x, y \in [a, b]$ with $x < y$. Then, by the Mean Value
  Theorem, there exists $c \in (x, y)$ with
  \[%
    \left\lvert \frac{f(y) - f(x)}{y - x} \right\rvert = \lvert f'(c) \rvert \le M
  .\]%
  Therefore, $f$ is Lipschitz on $[a, b]$.
\end{proof}

\begin{problem}[5.3.2]
  Let $f$ be differentiable on an interval $A$. If $f'(x) \ne 0$ on $A$, show
  that $f$ is one-to-one on $A$. Provide an example to show that the converse
  statement need not be true.
\end{problem}

\begin{proof}[Solution]
  Assume $f'(x) \ne 0$ on $A$. Let $x, y \in A$ with $x \ne y$. Without loss of
  generality, assume $x < y$. Then, by the Mean Value Theorem, there exists $c
  \in (x, y)$ such that
  \[%
    f(y) - f(x) = f'(c)(y - x) \ne 0
  .\]%
  Therefore, $f$ is one-to-one on $A$.

  An example to show that the converse statement need not be true is the
  function $f_a(x) = x^{2a+1}$, where $a \in \R$. The derivative of $f_a$ is
  $f_a'(x) = (2a+1)x^{2a}$. The function $f_a$ is one-to-one on $\R$, but
  $f_a'(0) = 0$.
\end{proof}
