\begin{problem}[5.3.3]
  Let $h$ be a differentiable function defined on the interval $[0, 3]$, and
  assume that $h(0) = 1$, $h(1) = 2$, and $h(3) = 2$.
  \begin{enumerate}
    \item Argue that there exists a point $d \in [0, 3]$ where $h(d) = d$.

    \item Argue that at some point $c$ we have $h'(c) = \sfrac{1}{3}$.

    \item Argue that $h'(x) = \sfrac{1}{4}$ at some point in the domain.
  \end{enumerate}
\end{problem}

\begin{proof}[Solution to (i)]
  Consider $g(x) = h(x) = x$. It's continuous. It's values at the endpoints are
  $g(0) = 1$ and $g(0) = -1$. By the IVT, there exists a point $d \in [0, 3]$
  where $g(d) = 0$, i.e., $h(d) = d$.
\end{proof}

\begin{proof}[Solution to (ii)]
  Just apply the IVT on the interval $[0, 3]$ to get a $c \in (0, 3)$ where
  \[%
    h'(c) = \frac{h(3) - h(0)}{3 - 0} = \frac{2 - 1}{3 - 0} = \frac{1}{3}
  .\qedhere\]%
\end{proof}

\begin{proof}[Solution to (iii)]
  By part (ii), we know there exists a $c \in (0, 3)$ where $h'(c) = 0$. We can
  also find a $d \in (1, 3)$ with $h'(d) = 0$. Then, using Darbox's Theorem,
  there must exist a point $x \in (c, d)$ where $h'(x) = \sfrac{1}{4}$.
\end{proof}

\begin{problem}[5.3.4]
  Let $f$ be differentiable on an interval $A$ containing zero, and assume
  $(x_n)$ is a sequence in $A$ with $(x_n) \to 0$ and $x_n \ne 0$.
  \begin{enumerate}
    \item If $f(x_n) = 0$ for all $n \in \N$, show $f(0) = 0$ and $f'(0) = 0$.

    \item Add the assumption that $f$ is twice-differentiable at zero and show
      that $f''(0) = 0$ as well.
  \end{enumerate}
\end{problem}

\begin{proof}[Solution to (i)]
  Since $f'(0)$ exists and $f(x_n) = 0$ for all $n \in \N$, we have
  \[%
    f'(0) = \lim_{n \to \infty} \frac{f(x_n) - f(0)}{x_n - 0} = \lim_{n \to \infty} \frac{0 - 0}{x_n - 0} = 0
  .\qedhere\]%
\end{proof}

\begin{proof}[Solution to (ii)]
  Using the MVT over $[0, x_n]$, there must exist a $c_n \in (0, x_n)$ such that
  \[%
    f'(c_n) = \frac{f(x_n)}{x_n}
  .\]%
  Then, like we did in part (i), we have
  \[%
    f''(0) = \lim_{n \to \infty} \frac{f'(c_n) - f'(0)}{c_n - 0} = \lim_{n \to \infty} \frac{\frac{f(x_n)}{x_n} - 0}{c_n - 0} = 0
  .\qedhere\]%
\end{proof}

\begin{problem}[5.3.6]\leavevmode
  \begin{enumerate}
    \item Let $g : [0, a] \to \R$ be differentiable, $g(0) = 0$, and $\lvert
      g'(x) \rvert \le M$ for all $x \in [0, a]$. Show that $\lvert g(x) \rvert
      \le Mx$ for all $x \in [0, a]$.

    \item Let $h : [0,a] \to \R$ be twice differentiable, $h'(0) = h(0) = 0$ and
      $\lvert h''(x) \rvert \le M$ for all $x \in [0, a]$. Show $\lvert h(x)
      \rvert \le \sfrac{Mx^2}{2}$ for all $x \in [0, a]$.

    \item Conjecture and prove an analogous result for a function that is
      differentiable three times on $[0, a]$.
  \end{enumerate}
\end{problem}

\begin{proof}[Solution to (i)]
  Since $g(x)$ is differentiable on $(0, a)$ and continuous on $[0, a]$, for
  each $x \in (0, a]$, there exists some $c \in (0, x)$ such that
  \[%
    g(x) = g(0) + g'(c)(x - 0) \implies g(x) = g'(c)x \implies \lvert g(x) \rvert = \lvert g'(c)x \rvert \implies \lvert g(x) \rvert \le Mx
  .\qedhere\]%
\end{proof}

\begin{proof}[Solution to (ii)]
  Applying the MVT to $h'$ on $[0, x]$, there exists some
  $c \in (0, x)$ such that
  \[%
    h'(x) = h'(0) + h''(c)(x - 0) \implies h'(x) = h''(c)x \implies \lvert h(x) \rvert \le Mx
  .\]%
  Now, applying the MVT to $h$ on $[0, x]$, there exists some $d
  \in (0, x)$ such that
  \[%
    h(x) = h(0) + h'(d)(x - 0) \implies h(x) = h'(d)x \implies \lvert h(x) \rvert \le Mx^2
  .\]%
  However, this overestimates the bound. To refine it, observe that the MVT
  picks $d$ so that
  \[%
    \lvert h'(d) \rvert \le Md \le Mx
  .\]%
  Thus, since $d$ is chosen in the middle of the interval, we get the tighter
  bound
  \[%
    \lvert h(x) \rvert \le \frac{Mx^2}{2}
  .\qedhere\]%
\end{proof}

\begin{proof}[Solution to (iii)]
  I conjecture: If $f : [0, a] \to \R$ is three times differentiable with $f(0)
  = f'(0) = f''(0) = 0$ and $\lvert f'''(x) \rvert \le M$ for all $x \in [0,
  a]$, then
  \[%
    \lvert f(x) \rvert \le \frac{Mx^3}{6}
  .\]%

  \noindent\textit{Proof.} Applying the MVT to $f''$ on $[0, x]$,
  then there exists some $c \in (0, x)$ such that
  \[%
    f''(x) = f''(0) + f'''(c)x \implies \lvert f''(x) \rvert = \lvert f'''(c)x \rvert \le Mx
  .\]%
  Applying the MVT to $f'$ on $[0, x]$, there exists some $d \in
  (0, x)$ such that
  \[%
    f'(x) = f'(0) + f''(d)x \implies \lvert f'(x) \rvert = \lvert f''(d)x \rvert \le Mdx \le \frac{Mx^2}{2}
  .\]%
  Finally, applying the MVT to $f$ on $[0, x]$, there exists some
  $\epsilon \in (0, x)$ such that
  \[%
    f(x) = f(0) + f'(\epsilon)x \implies \lvert f(x) \rvert = \lvert f(\epsilon)x \rvert
  .\]%
  Using the bound $\lvert f'(\epsilon) \rvert \le \sfrac{M\epsilon^2}{2}$, we
  get
  \[%
    \lvert f(x) \rvert \le \frac{M\epsilon^2}{2}x
  .\]%
  Since $\epsilon \in (0, x)$ we refine the bound as before to get
  \[%
    \lvert f(x) \rvert \le \frac{Mx^3}{6}
  .\qedhere\]%
\end{proof}

\begin{problem}[5.3.8]
  Assume $f$ is continuous on an interval containing zero and differentiable for
  all $x \ne 0$. If $\lim_{x \to 0} f'(x) = L$, show $f'(0)$ exists and equals
  $L$.
\end{problem}

\begin{proof}[Solution]
  Define $g(x) = f(x) - f(0)$. Then, $g(x)$ is continuous on an interval
  containing $0$ and differentiable for $x \ne 0$, with $g(0) = 0$. Then, using
  l'Hospital's Rule, we have
  \[%
    \lim_{x \to 0} \frac{g(x)}{x} = \lim_{x \to 0} \frac{g'(x)}{1} = L
  .\qedhere\]%
\end{proof}

\begin{problem}[7.2.1]
  Let $f$ be a bounded function on $[a, b]$, and let $P$ be an arbitrary
  partition of $[a, b]$. First, explain why $U(f) \ge L(f, P)$. Now, prove Lemma
  7.2.6.
\end{problem}

\begin{proof}[Solution]
  Since for each subinterval $[x_{i-1}, x_i]$, we have
  \[%
    m_i \le f(x) \le M_i~\textrm{for all}~x \in [x_{i-1}, x_i]
  .\]%
  It follows that $m_i \le M_i~\textrm{for each}~i$.
  Multiplying by $\delta x_i$, which is always positive, and summing over all
  subintervals, we get
  \[%
    L(f, P) = \sum_{i=1}^n m_i \Delta x_i \le \sum_{i=1}^n M_i \Delta x_i = U(f, P)
  .\]%
  Thus, $U(f) \ge L(f, P)$.

  The upper integral and lower integral are defined as
  \[%
    U(f) = \inf_P U(f, P) \aand L(f) = \sup_P L(f, P)
  .\]%
  Since for any partition $P$, we established that $U(f, P) \ge L(f, P)$, it
  follows that
  \[%
    \inf_P U(f, P) \ge \sup_P L(f, P) \implies U(f) \ge L(f) \iff U(f) \ge L(f, P)
  .\]%

  This proves that the upper integral is always at least the lower integral for
  any bounded function $f$ on $[a, b]$.
\end{proof}

\begin{problem}[7.2.2]
  Consider $f(x) = \sfrac{1}{x}$ over the interval $[1, 4]$. Let $P$ be the
  partition consisting of the points $\{1, \sfrac{3}{2}, 2, 4\}$.
  \begin{enumerate}
    \item Compute $L(f, P)$, $U(f, P)$, and $U(f, P) - L(f, P)$.

    \item What happens to the value of $U(f, P) - L(f, P)$ when we add the point
      $3$ to the partition?

    \item Find a partition $P'$ of $[1, 4]$ for which $U(f, P') - L(f, P') <
      \sfrac{2}{5}$.
  \end{enumerate}
\end{problem}

\begin{proof}[Solution to (i)]
  This creates the subintervals
  \[%
    \left[1, \frac{3}{2}\right],\quad \left[\frac{3}{2}, 2\right],\aand [2, 4]
  .\]%
  \begin{align*}
    M_i &= \sup f(x)~\textrm{on}~[x_{i-1}, x_i] \\
    m_i &= \inf f(x)~\textrm{on}~[x_{i-1}, x_i] \\
    \Delta x_i &= x_i - x_{i-1}
  .\end{align*}
  Therefore, computing each of these values, we get
  \[%
    \begin{rcases}
      L(f, P) &= \sum_{i=1}^n m_i \Delta x_i = \frac{1}{3} + \frac{1}{4} + \frac{1}{2} = \frac{13}{12} \\
      U(f, P) &= \sum_{i=1}^n \frac{1}{2} + \frac{1}{3} + 1 = \frac{11}{6}
    \end{rcases} \implies U(f, P) - L(f, P) = \frac{11}{6} - \frac{13}{12} = \frac{3}{4}
  .\qedhere\]%
\end{proof}

\begin{proof}[Solution to (ii)]
  If we add $x = 3$, we create a new subinterval $[2, 3]$ and $[3, 4]$. This
  reduces the maximum and minimum function values over each subinterval, leading
  to a smaller difference $U(f, P) - L(f, P)$. Since we are refining the
  partition, $U(f, P)$ decreases and $L(f, P)$ increases, making $U(f, P) - L(f,
  P)$ smaller.
\end{proof}

\begin{proof}[Solution to (iii)]
  Let $P' = \{1, \sfrac{5}{4}, \sfrac{3}{2}, 2, 3, 4\}$. Then, we have
  \[%
    \begin{rcases}
      L(f, P') &= \sum_{i=1}^n m_i \Delta x_i = \frac{12}{60} + \frac{10}{60} + \frac{15}{60} + \frac{20}{60} + \frac{15}{60} = \frac{6}{5}\\
      U(f, P') &= \sum_{i=1}^n M_i \Delta x_i = \frac{15}{60} + \frac{12}{60} + \frac{20}{60} + \frac{30}{60} + \frac{20}{60} = \frac{97}{60}
    \end{rcases} \implies U(f, P') - L(f, P') = \frac{97}{60} - \frac{6}{5} = \frac{5}{12} < \frac{1}{2}
  .\qedhere\]%
\end{proof}
