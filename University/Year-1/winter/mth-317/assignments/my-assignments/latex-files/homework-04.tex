\begin{problem}[5.3.3]
  Let $h$ be a differentiable function defined on the interval $[0, 3]$, and
  assume that $h(0)  1$, $h(1) = 2$, and $h(3) = 2$.
  \begin{enumerate}
    \item Argue that there exists a point $d \in [0, 3]$ where $h(d) = d$.

    \item Argue that at some point $c$ we have $h'(c) = \sfrac{1}{3}$.

    \item Argue that $h'(x) = \sfrac{1}{4}$ at some point in the domain.
  \end{enumerate}
\end{problem}

\begin{proof}[Solution to (i)]
\end{proof}

\begin{proof}[Solution to (ii)]
\end{proof}

\begin{proof}[Solution to (iii)]
\end{proof}


\begin{problem}[5.3.4]
  Let $f$ be differentiable on an interval $A$ containing zero, and assume
  $(x_n)$ is a sequence in $A$ with $(x_n) \to 0$ and $x_n \ne 0$.
  \begin{enumerate}
    \item If $f(x_n) = 0$ for all $n \in \N$, show $f(0) = 0$ and $f'(0) = 0$.

    \item Add the assumption that $f$ is twice-differentiable at zero and show
      that $f''(0) = 0$ as well.
  \end{enumerate}
\end{problem}

\begin{proof}[Solution to (i)]
\end{proof}

\begin{proof}[Solution to (ii)]
\end{proof}

\begin{problem}[5.3.6]\leavevmode
  \begin{enumerate}
    \item Let $g : [0, a] \to \R$ be differentiable, $g(0) = 0$, and $\lvert
      g'(x) \rvert \le M$ for all $x \in [0, a]$. Show that $\lvert g(x) \rvert
      \le Mx$ for all $x \in [0, a]$.

    \item Let $h : [0,a] \to \R$ be twice differentiable, $h'(0) = h(0) = 0$ and
      $\lvert h''(x) \rvert \le M$ for all $x \in [0, a]$. Show $\lvert h(x)
      \rvert \le \sfrac{Mx^2}{2}$ for all $x \in [0, a]$.

    \item Conjecture and prove an analogous result for a function that is
      differentiable three times on $[0, a]$.
  \end{enumerate}
\end{problem}

\begin{proof}[Solution to (i)]
\end{proof}

\begin{proof}[Solution to (ii)]
\end{proof}

\begin{proof}[Solution to (iii)]
\end{proof}

\begin{problem}[5.3.8]
  Assume $f$ is continuous on an interval containing zero and differentiable for
  all $x \ne 0$. If $\lim_{x \to 0} f'(x) = L$, show $f'(0)$ exists and equals
  $L$.
\end{problem}

\begin{proof}[Solution]
\end{proof}

\begin{problem}[7.2.1]
  Let $f$ be a bounded function on $[a, b]$, and let $P$ be an arbitrary
  partition of $[a, b]$. First, explain why $U(f) \ge L(f, P)$. Now, prove Lemma
  7.2.6.
\end{problem}

\begin{proof}[Solution]
\end{proof}

\begin{problem}[7.2.2]
  Consider $f(x) = \sfrac{1}{x}$ over the interval $[1, 4]$. Let $P$ be the
  partition consisting of the points $\{1, \sfrac{3}{2}, 2, 4\}$.
  \begin{enumerate}
    \item Compute $L(f, P)$, $U(f, P)$, and $U(f, P) - L(f, P)$.

    \item What happens to the value of $U(f, P) - L(f, P)$ when we add the point
      $3$ to the partition?

    \item Find a partition $P'$ of $[1, 4]$ for which $U(f, P') - L(f, P') <
      \sfrac{2}{5}$.
  \end{enumerate}
\end{problem}

\begin{proof}[Solution to (i)]
\end{proof}

\begin{proof}[Solution to (ii)]
\end{proof}

\begin{proof}[Solution to (iii)]
\end{proof}
