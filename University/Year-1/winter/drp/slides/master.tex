\documentclass[aspectratio=169,xcolor=dvipsnames]{beamer}
\usetheme{SimpleDarkBlue}

\usepackage{hyperref}
\usepackage{graphicx}
\usepackage{booktabs}

\usepackage{derivative}
\usepackage{amsmath}
\usepackage{amsfonts}
\usepackage{mathtools}
\usepackage{amsthm}
\usepackage{amssymb}
\usepackage{mathrsfs}
\usepackage{breqn}
\usepackage{xifthen}
\usepackage{xfrac}

\usefonttheme[onlymath]{serif}

\let\to\rightarrow
\let\implies\Rightarrow
\let\impliedby\Leftarrow
\let\iff\Leftrightarrow
\let\epsilon\varepsilon
\let\phi\varphi
\let\tau\uptau

\newcommand\adX{\text{ad}X}

\renewcommand\a{\mathbf{a}}
\renewcommand\b{\mathbf{b}}
\renewcommand\c{\mathbf{c}}
\renewcommand\d{\mathbf{d}}
\newcommand\e{\mathbf{e}}
\newcommand\f{\mathbf{f}}
\newcommand\FF{\mathbf{F}}
\newcommand\g{\mathbf{g}}
\newcommand\n{\mathbf{n}}
\newcommand\p{\mathbf{p}}
\renewcommand\r{\mathbf{r}}
\newcommand\Ta{\mathbf{T}}
\renewcommand\u{\mathbf{u}}
\renewcommand\v{\mathbf{v}}
\newcommand\w{\mathbf{w}}
\newcommand\x{\mathbf{x}}
\newcommand\y{\mathbf{y}}
\newcommand\z{\mathbf{z}}
\newcommand\zero{\mathbf{0}}
\newcommand\ad{\operatorname{ad}}
\newcommand\Ad{\operatorname{Ad}}
\newcommand\hb{\hbar}

% Hat vectors
\newcommand\ah{\mathbf{a}}
\newcommand\bh{\mathbf{b}}
\newcommand\ch{\mathbf{c}}
\renewcommand\dh{\mathbf{d}}
\newcommand\eh{\mathbf{e}}
\newcommand\ph{\mathbf{p}}
\newcommand\uh{\mathbf{u}}
\newcommand\vh{\mathbf{v}}
\newcommand\wh{\mathbf{w}}
\newcommand\xh{\mathbf{x}}
\newcommand\yh{\mathbf{y}}
\newcommand\zh{\mathbf{z}}

% Unit vectors
\newcommand\ui{\boldsymbol{\imath}}
\newcommand\uj{\boldsymbol{\jmath}}
\newcommand\uk{\mathbf{k}}

% Matrix groups
\newcommand\M{\mathrm{M}}
\newcommand\Ec{\mathrm{E}}
\newcommand\GL{\mathrm{GL}}
\newcommand\Or{\mathrm{O}}
\newcommand\PGL{\mathrm{PGL}}
\newcommand\PSL{\mathrm{PSL}}
\newcommand\PSO{\mathrm{PSO}}
\newcommand\PSU{\mathrm{PSU}}
\newcommand\SL{\mathrm{SL}}
\newcommand\SO{\mathrm{SO}}
\newcommand\Spin{\mathrm{Spin}}
\newcommand\Sp{\mathrm{Sp}}
\newcommand\SU{\mathrm{SU}}
\newcommand\U{\mathrm{U}}
\newcommand\Mat{\mathrm{Mat}}
\renewcommand\P{\mathbb{P}}

% Lie algebras
\newcommand\ec{\mathrm{e}}
\newcommand\gl{\mathrm{gl}}
\renewcommand\sl{\mathrm{sl}}
\newcommand\so{\mathrm{so}}
\newcommand\su{\mathrm{su}}
\newcommand\ggl{\mathfrak{g}}
\newcommand\hl{\mathfrak{h}}

% Subspaces
\newcommand\B{\mathcal{B}}
\newcommand\C{\mathbb{C}}
\newcommand\D{\mathbb{D}}
\newcommand\E{\mathbb{E}}
\newcommand\F{\mathbb{F}}
\newcommand\I{\mathbb{I}}
\newcommand\N{\mathbb{N}}
\newcommand\Q{\mathbb{Q}}
\newcommand\R{\mathbb{R}}
\newcommand\Z{\mathbb{Z}}

% Helpful text in math mode
\newcommand\aand{\quad\text{and}\quad}
\newcommand\oor{\quad\text{or}\quad}
\renewcommand\and{\text{and}}
\newcommand\qtq[1]{\quad\text{#1}\quad}
\newcommand\Col{\text{Col}}
\newcommand\Nul{\text{Null}}
\newcommand\Range{\text{Range}}
\newcommand\Ker{\text{Ker}}
\newcommand\Tr{\text{Tr}}
\newcommand\Rank{\text{Rank}}
\newcommand\proj{\text{proj}}
\newcommand\comp{\text{comp}}
\newcommand\Sspan{\text{Span}}
\renewcommand\Re{\text{Re}}
\renewcommand\Im{\text{Im}}

\title{Classifications of all Irreducible Representations of $\sl_2(\C)$}
\subtitle{up to Isomorphism}

\author{Hashem A. Damrah}

\institute{
  Department of Mathematics \\
  University of Oregon
}
\date{April 25, 2025}

\begin{document}
  \begin{frame}
    \titlepage
  \end{frame}

  \begin{frame}{Lie groups}
    \textbf{Definition:} A \textit{Lie group} is a differentiable manifold $G$,
    which is also a group, and such that the group product
    \[%
      G \times G \to G
    ,\]%
    and the inverse map $g \to g^{-1}$ are differentiable.
  \end{frame}

  \begin{frame}{Matrix Lie groups}
    \textbf{Definition:} A \textit{Matrix Lie group} is a subgroup $G \subseteq
    \GL_n(\C)$ that's topologically closed, i.e., any matrix sequence that
    converges, converges to a matrix in $G$.

    \vspace{1em}

    \textbf{Example:} The \textit{Special Linear Group} $\SL_n(\C)$ is the set
    \[%
      \SL_n(\C) = \left\{X \in \GL_n(\C) \rvert \det(X) = 1\right\}
    .\]%
    Clearly, this is a subgroup of $\GL_n(\C)$ since
    \begin{enumerate}
      \item $\SL_n(\C)$ is closed under matrix multiplication, since $\det(XY) =
        \det(X)\det(Y) = 1$.

      \item $\SL_n(\C)$ has an identity element, since $\det(I_n) = 1$.

      \item $\SL_n(\C)$ is closed under inverses, since the inverse is the
        matrix itself.

      \item $\SL_n(\C)$ is associative, since matrix multiplication is
        associative.
    \end{enumerate}
    And it's also topologically closed, since the determinant is a continuous
    function, and the set $\det^{-1}(1)$ is closed in $\C$.
  \end{frame}

  \begin{frame}{Lie algebra}
    \textbf{Definition:} A \textit{finite-dimensional real or complex Lie
    algebra} is a finite-dimensional real or complex vector space $\ggl$,
    together with a map $[\cdot, \cdot] : \ggl \times \ggl \to \ggl$ with the
    following properties:
    \begin{enumerate}
      \item $[\cdot, \cdot]$ is bilinear, i.e.,
        \[%
          [W + X, Y + Z] = [W, Y + Z] + [X, Y + Z] = [W, Y] + [W, Z] + [X, Y] + [X, Z]
        .\]%

      \item $[X, Y] = -[Y, X]$.

      \item Satisfies the Jacobi Identity: $[X, [Y, Z]] + [Y, [Z, X]] + [Z, [X,
        Y]] = 0$.
    \end{enumerate}
  \end{frame}

  \begin{frame}{Lie algebra of a Matrix Lie group}
    \textbf{Definition:} Let $G$ be a matrix Lie group. Then the \textit{Lie
    algebra} of $G$, denoted by $\ggl$, is the set of all matrices $X \in
    \M_n(\C)$ such that $e^{Xt} \in G$, for all $t \in \R$.

    \vspace{1em}

    \textbf{Example:} The Lie algebra of $\SL_2(\C)$, denoted $\sl_2(\C)$, is
    the set of all $X \in \M_2(\C)$ such that
    \[%
      \sl_2(\C) = \left\{
        \left.X = \begin{pmatrix}
          a & b \\
          c & d \\
        \end{pmatrix}\right\rvert
        \Tr(X) = a + d = 0
      \right\}
    ,\]%
    since $\det\left(e^{Xt}\right) = e^{\Tr(X)t}$. Therefore, $\dim(\sl_2(\C)) =
    3$, since we can construct the basis
    \[%
      \left\{
        h = \begin{pmatrix}
          1 & 0 \\
          0 & -1 \\
        \end{pmatrix},
        e = \begin{pmatrix}
          0 & 1 \\
          0 & 0 \\
        \end{pmatrix},
        f = \begin{pmatrix}
          0 & 0 \\
          1 & 0 \\
        \end{pmatrix}
      \right\}
    .\]%
    Since $\sl_2(\C) \subset \gl_2(\C)$, the Lie Bracket induced on $\sl_2(\C)$,
    comes from $\gl_2(\C)$, defined as $[X, Y] = XY - YX$. It's easy to see that
    the operation is bilinear, skew-symmetric, closed, and satisfies the Jacobi
    identity. Computing the Lie bracket of the basis elements, we get
    \[%
      [h, e] = 2e, \quad [h, f] = -2f, \aand [e, f] = h
    .\]%
  \end{frame}

  \begin{frame}{Complexification of a Lie algebra}
    \textbf{Definition:} Let $\ggl$ be a real Lie algebra. Then, the
    \textit{complexification} of $\ggl$, denoted by $\ggl_\C$, is the complex
    vector space $\ggl \otimes_\R \C$ with the Lie bracket defined by
    \[%
      [X \otimes z, Y \otimes w] = [X, Y] \otimes zw
    ,\]%
    extended by $\C$-bilinearity, where $X, Y \in \ggl$ and $z, w \in \C$.

    \vspace{1em}

    \textbf{Example:} Let $\ggl$ be a real Lie algebra of the matrix Lie group
    $G$. Then, the complexification of $\ggl$, denoted by $\ggl_\C$, is the set
    \[%
      \ggl_\C = \left\{
        X + iY \mid X, Y \in \ggl
      \right\}
    .\]%
    The Lie bracket, $[\cdot, \cdot] : \ggl_\C \times \ggl_\C \to \ggl_\C$, then
    becomes
    \begin{align*}
      [X + iY, W + iZ] &= [X, W + iZ] + [iY, W + iZ] \\
                       &= [X, W] + [X, iZ] + [iY, W] + [iY, iZ] \\
                       &= ([X, W] - [Y, Z]) + i([X, Z] + [Y, W])
    .\end{align*}

    \vspace{1em}

    \textbf{Note:} The complexification of a real Lie algebra is a complex Lie
    algebra.
  \end{frame}

  \begin{frame}{Homomorphisms of Lie groups and Lie algebras}
    \textbf{Definition:} Let $(G_1, \cdot)$ and $(G_2, \times)$ be two Lie
    groups. Then, a \textit{homomorphism} of Lie groups is a map $\Phi : G_1 \to
    G_2$ such that
    \[%
      \Phi(g_1 \cdot g_2) = \Phi(g_1) \times \Phi(g_2)
    ,\]%
    where $g_1, g_2 \in G_1$ and $\Phi(g_1), \Phi(g_2) \in G_2$.

    \vspace{1em}

    \textbf{Definition:} A \textit{homomorphism} of Lie algebras is a linear map
    $\Phi : \ggl_1 \to \ggl_2$, where $\ggl_1$ and $\ggl_2$ are Lie algebras, such
    that
    \[%
      \Phi([X, Y]) = [\Phi(X), \Phi(Y)]
    ,\]%
    where $X, Y \in \ggl_1$ and $\Phi(X), \Phi(Y) \in \ggl_2$.
  \end{frame}

  \begin{frame}{Representations}
    Let $G$ be a matrix Lie group, $\ggl$ a Lie algebra, and $V$ a
    finite-dimensional vector space.

    \vspace{1em}

    \textbf{Definition:} A \textit{representation} of $G$= is a Lie group
    homomorphism                                         =
    \[%
      \Pi : G \to \GL(V)
    ,\]%
    where $\GL(V)$ is the set of all automorphisms of $V$.

    \vspace{1em}

    \textbf{Definition:} A representation of $\ggl$ is a Lie algebra
    homomorphism
    \[%
      \pi :\ggl \to \gl(V)
    ,\]%
    where $\gl(V)$ is the set of all one-to-one mappings of $V$ onto itself.

    \vspace{1em}

    \textbf{Note:} To denote a representation, $\Pi$, of a Lie group $G$ on a
    vector space $V$, I'll write $(\Pi, G, V)$. Same thing for a representation
    $\pi$ of a Lie algebra $\ggl$ on a vector space $V$, I'll write $(\pi, \ggl,
    V)$.
  \end{frame}

  \begin{frame}{Irreducible and Isomorphic representations}
    \textbf{Definition:} Define $(\Pi, V)$ and $(\Sigma, W)$ on a Lie group $G$.
    A linear map $\Phi : V \to W$ is called a \textit{morphism} of
    representations if
    \[%
      \Phi(\Pi(A)v) = \Sigma(A)\Phi(v)
    ,\]%
    for all $A \in G$ and $v \in V$. If $\Phi$ is invertible, then it's called
    an \textit{isomorphism} of representations.

    \vspace{1em}

    \textbf{Definition:} Define $(\Pi, V)$ on a Lie group $G$.
    A subspace $W$ of $V$ is called \textit{invariant} under the representation
    $(\Pi, V)$ if
    \[%
      \Pi(A)w \in W
    ,\]%
    for all $A \in G$ and $w \in W$. An invariant subspace, $W$, is called
    \textit{non-trivial} if $W \ne \{0\}$ and $W \ne V$. A representation with
    no non-trivial invariant subspaces is called \textit{irreducible}. A
    representation $(\pi, \ggl, V)$ of a Lie algebra $\ggl$ is irreducible if it
    has no nontrivial invariant subspaces under the action of $\pi(X)$ for all
    $X \in \ggl$.
  \end{frame}

  \begin{frame}{References}
    \footnotesize
    \nocite*{}
    \bibliography{reference.bib}
    \bibliographystyle{apalike}
  \end{frame}
\end{document}
