\documentclass[notitlepage]{report}

\title{Several-Variab Calc II}
\date{January 6, 2025}

\usepackage{xcolor}
\usepackage{amsmath}
\usepackage{amsfonts}
\usepackage{mathtools}
\usepackage{amsthm}
\usepackage{amssymb}
\usepackage{mathrsfs}
\usepackage{breqn}
\usepackage{xifthen}
\usepackage{geometry}
\usepackage{xfrac}
\geometry{
  top=1in,
  bottom=1.5in,
  right=1in,
  left=1in,
}

\renewcommand\a{\mathbf{a}}
\renewcommand\b{\mathbf{b}}
\renewcommand\c{\mathbf{c}}
\renewcommand\d{\mathbf{d}}
\newcommand\e{\mathbf{e}}
\newcommand\f{\mathbf{f}}
\newcommand\FF{\mathbf{F}}
\newcommand\g{\mathbf{g}}
\newcommand\n{\mathbf{n}}
\newcommand\p{\mathbf{p}}
\renewcommand\r{\mathbf{r}}
\newcommand\Ta{\mathbf{T}}
\renewcommand\u{\mathbf{u}}
\renewcommand\v{\mathbf{v}}
\newcommand\w{\mathbf{w}}
\newcommand\x{\mathbf{x}}
\newcommand\y{\mathbf{y}}
\newcommand\z{\mathbf{z}}
\newcommand\zero{\mathbf{0}}

% Hat vectors
\newcommand\ah{\mathbf{a}}
\newcommand\bh{\mathbf{b}}
\newcommand\ch{\mathbf{c}}
\renewcommand\dh{\mathbf{d}}
\newcommand\eh{\mathbf{e}}
\newcommand\ph{\mathbf{p}}
\newcommand\uh{\mathbf{u}}
\newcommand\vh{\mathbf{v}}
\newcommand\wh{\mathbf{w}}
\newcommand\xh{\mathbf{x}}
\newcommand\yh{\mathbf{y}}
\newcommand\zh{\mathbf{z}}

% Unit vectors
\newcommand\ui{\boldsymbol{\imath}}
\newcommand\uj{\boldsymbol{\jmath}}
\newcommand\uk{\mathbf{k}}

% Matrix groups
\newcommand\Ec{\mathrm{E}}
\newcommand\GL{\mathrm{GL}}
\newcommand\Or{\mathrm{O}}
\newcommand\PGL{\mathrm{PGL}}
\newcommand\PSL{\mathrm{PSL}}
\newcommand\PSO{\mathrm{PSO}}
\newcommand\PSU{\mathrm{PSU}}
\newcommand\SL{\mathrm{SL}}
\newcommand\SO{\mathrm{SO}}
\newcommand\Spin{\mathrm{Spin}}
\newcommand\Sp{\mathrm{Sp}}
\newcommand\SU{\mathrm{SU}}
\newcommand\U{\mathrm{U}}
\newcommand\Mat{\mathrm{Mat}}
\renewcommand\P{\mathbb{P}}

% Subspaces
\newcommand\B{\mathcal{B}}
\newcommand\C{\mathbb{C}}
\newcommand\D{\mathbb{D}}
\newcommand\E{\mathbb{E}}
\newcommand\F{\mathbb{F}}
\newcommand\I{\mathbb{I}}
\newcommand\N{\mathbb{N}}
\newcommand\Q{\mathbb{Q}}
\newcommand\R{\mathbb{R}}
\newcommand\Z{\mathbb{Z}}

% Helpful text in math mode
\newcommand\aand{\quad\text{and}\quad}
\newcommand\oor{\quad\text{or}\quad}
\renewcommand\and{\text{and}}
\newcommand\qtq[1]{\quad\textrm{#1}\quad}
\newcommand\Col{\textrm{Col}}
\newcommand\Nul{\textrm{Null}}
\newcommand\Range{\textrm{Range}}
\newcommand\Ker{\textrm{Ker}}
\newcommand\Tr{\textrm{Tr}}
\newcommand\Rank{\textrm{Rank}}
\newcommand\proj{\text{proj}}
\newcommand\comp{\text{comp}}
\newcommand\Sspan{\text{Span}}
\renewcommand\Re{\text{Re}}
\renewcommand\Im{\text{Im}}

\newenvironment{exercise}[1][-1]{\noindent\textbf{Exercise #1}.}

\begin{document}
  \chapter{Matrix Lie Groups Solutions}

  \begin{exercise}[11]
    \textit{Connectedness of} $\SO(n)$. Show that $\SO(n)$ is connected,
    following the outline below.

    For the $n = 1$ case, there is not much to show, since a $1 \times 1$ matrix
    with determinant one must be 1. Assume, then, that $n \ge 2$ Let $\e_1$
    denote the vector
    \[%
      \begin{pmatrix}
        1 \\
        0 \\
        \vdots \\
        0
      \end{pmatrix} \in \R^n
    .\]%
    Given any unit vector $\v \in \R^n$, show that there exists a continuous
    path $R(t)$ in $\SO(n)$ such that $R(0) = I$ and $R(1)\e_1 = \v$. (Thus any
    unit vector can be ``continuously rotated'' to $\e_1$.)

    Now show that any element $R$ of $\SO(n)$ can be connected to an element of
    $\SO(n - 1)$, and proceed by induction.
  \end{exercise}

  \begin{proof}[Solution]
    Let $\B = \{\v, \e_1\}$ be a basis for a two-dimensional plane. By the
    Gram-Schmitt process, we can construct an orthonormal basis $\B' = \{\u_1,
    \u_2\}$ for the same plane. Let $\u_1 = \v$ and $\u_2$ be defined as
    \[%
      \u_2 = \frac{\e_1 - \langle \e_1, \v \rangle \v}{\|\e_1 - \langle \e_1, \v \rangle \v\|}
    .\]%
    Let $\theta(t) : [0, 1] \to \R$, defined by
    \[%
      \theta(0) = 0 \aand \theta(1) = \arccos\left(\frac{\v \cdot \e_1}{\lVert \v \rVert \lVert \e_1 \rVert}\right)
    \]%
    Then, we can construct a rotation $R(t) \in \SO(n)$ as a block matrix that
    acts as a rotation by $\theta(t)$ in the plane spanned by $\{\u_1, \u_2\}$,
    and as the identity on the orthogonal complement. This defines a continuous
    path with $R(0) = I$ and $R(1) \v = \e_1$.

    We'll use induction to show that $\SO(n)$ is connected for all $n \ge 1$.
    The base case is trivial, which is trivially connected.

    Assume $\SO(n)$ is connected for some $n \geq 1$. We need to show that
    $\SO(n+1)$ is connected. Let $R \in \SO(n+1)$. Consider the first column of
    $R$, which is a unit vector $\v \in \R^{n+1}$.

    {\color{red}I'm not sure how to proceed from here.}
  \end{proof}

  \begin{exercise}[12]
    \textit{The polar decomposition of $\SL(n, \R)$.} Show that every element
    $A$ of $\SL(n, \R)$ can be written uniquely in the form $A = RH$, where $R
    \in \SO(n)$, and $H$ is a symmetric, positive-definite matrix with
    determinant one (That is, $H^T = H$, and $\langle \x, H\x \rangle \ge 0$ for
    all $\x \in \R^n$).

    \textit{Hint:} If $A$ could be written in this form, then we would have
    \[%
      A^TA = H^TR^TRH = HR^{-1}RH = H^2
    .\]%
    Thus $H$ would have to be the unique positive-definite symmetric square root
    of $A^TA$.

    \textit{Note:} A similar argument gives polar decompositions for $\GL(n,
    \R)$, $\SL(n, \C)$, and $\GL(n, \C)$. For example, every element $A$ of
    $\SL(n, \C)$ can be written uniquely as $A = UH$, with $U \in \SU(n)$, and
    $H$ is a self-adjoint positive definite matrix with determinant one.
  \end{exercise}

  \begin{proof}[Solution]
    Consider the matrix $A^T A$, which is symmetric and positive definite
    because for any nonzero vector $\x \in \R^n$, we have
    \[%
      (A^T A)^T = (A)^T(A^T)^T = A^T A \aand \x^T A^T A\x = (A\x)^T (A\x) = \lVert A\x \rVert^2 > 0
    .\]%
    Since $A \in \SL(n, \R)$, we have $\det(A) = 1$, implying $\det(A^T A) =
    \det(A^T) \cdot \det(A) = 1 \cdot 1 = 1$ since determinant is
    multiplicative. By the spectral theorem, $A^T A$ has an orthonormal
    eigenbasis with positive eigenvalues, so it admits a unique
    positive-definite square root, denoted $H$, such that
    \[%
      H = \sqrt{A^T A} \implies H^2 = A^T A
    .\]%
    Define $R = AH^{-1}$. We check that $R$ is orthogonal
    \[%
      R^T R = (H^{-1} A^T) (A H^{-1}) = H^{-1} A^T A H^{-1} = H^{-1} H^2 H^{-1} = I
    .\]%
    Thus, $R \in \SO(n)$ since $\det(R) = \sfrac{\det(A)}{\det(H)} =
    \sfrac{1}{1} = 1$, proving existence.

    Suppose $A = R_1 H_1 = R_2 H_2$ are two such decompositions. Then,
    \[%
      H_1^{-1} R_1^{-1} R_2 H_2 = I
    .\]%
    Multiplying on the right by $H_2^{-1}$ and on the left by $H_1$, we obtain
    \[%
      H_1 H_1^{-1} R_1^{-1} R_2 H_2 H_2^{-1} = H_1 H_2^{-1} = I
    ,\]%
    so $H_1 = H_2$. This implies $R_1 = R_2$, proving uniqueness.

    Thus, every element of $\SL(n, \mathbb{R})$ has a unique polar
    decomposition.
  \end{proof}

  \begin{exercise}[13]
    \textit{The connectedness of $\SL(n, \R)$.} Using the polar decomposition of
    $\SL(n, \R)$ and the connectedness of $\SO(n)$, show that $\SL(n, \R)$ is
    connected.

    \textit{Hint:} Recall that if $H$ is a real, symmetric matrix, then there
    exists a \textit{real} orthogonal matrix $R_1$ such that $H = R_1DR_1^{-1}$,
    where $D$ is diagonal.
  \end{exercise}

  \begin{proof}[Solution]
    Since we are dealing with $\SL(n, \R)$, we add the restriction that $H$ is
    of determinant one. By the polar decomposition, we can write $A = RH$, where
    $R \in \SO(n)$ and $H$ is a symmetric, positive-definite matrix with
    determinant one.

    Also, by the hint, we can write $H = R_1 D R_1^{-1}$, where $R_1 \in \Or(n)$
    and $D$ is a diagonal matrix. The space of symmetric matrices with
    determinant 1 that are also positive definite forms a connected space. This
    follows because the space of positive-definite diagonal matrices with
    determinant 1 is connected, and conjugation by an orthogonal matrix does not
    change connectivity.

    By exercise 11, we know that $\SO(n)$ is connected. Since each element in
    $\SL(n, \R)$ can be written as $RH$, where $R \in \SO(n)$ and $H$ belongs to
    a connected space, and the product of connected spaces is connected, we
    conclude that $\SL(n, \R)$ is connected.
  \end{proof}

  \begin{exercise}[14]
    \textit{The connectedness of $\GL(n, \R)^+$.} Show that $\GL(n, \R)^+$ is
    connected.
  \end{exercise}

  \begin{proof}[Solution]
    Consider any matrix $A \in \GL(n, \R)^+$.
  \end{proof}

  \begin{exercise}[15]
    Show that the set of translations is a normal subgroup of the Euclidean
    group, and also of the Poincar\'e group. Show that
    $(\Ec(n)/\textrm{translations}) \cong \Or(n)$.
  \end{exercise}

  \begin{proof}[Solution]
  \end{proof}

  \begin{exercise}[16]
    \textit{Harder.} Show that every Lie group homomorphism $\phi : \R \to S^1$
    is of the form $\phi(x) = e^{iax}$ for some $a \in \R$. In particular, every
    such homomorphism is smooth.
  \end{exercise}

  \begin{proof}[Solution]
  \end{proof}
\end{document}
