\begin{problem}[1]
  The \textit{trace} of an $n \times n$ matrix $A = (a_{ij})$ is defined as the
  sum of the diagonal elements of $A$, i.e., $\Tr(A) = \sum_{j=1}^n a_{jj}$.
  Prove that $\Tr(AB) = \Tr(BA)$ for any $n \times n$ matrices $A$ and $B$.
\end{problem}

\begin{proof}[Solution]
  Let $A = (a_{ij})$ and $B = (b_{ij})$ be $n \times n$ matrices. The $(i,
  i)$-th element of $AB$ and $BA$ are given as follows
  \[%
    (AB)_{ii} = \sum_{k=1}^n a_{ik} b_{ki} \aand (BA)_{ii} = \sum_{k=1}^n b_{ik} a_{ki}
  .\]%
  Substituting both of these into the definition of the trace gives us
  \[%
    \Tr(AB) = \sum_{i=1}^n \sum_{k=1}^n a_{ik} b_{ki} \aand \Tr(BA) = \sum_{i=1}^n \sum_{k=1}^n b_{ik} a_{ki}
  .\]%
  Notice that both expressions involve summing over all pairs $(i, k)$. Since
  addition is commutative, the order of summation does not matter, and thus
  $\Tr(AB) = \Tr(BA)$.
\end{proof}

\begin{problem}[2]
  State the replacement theorem.
\end{problem}

\begin{proof}[Solution]
  Let $V$ be a vector space, and let $\{\v_1, \v_2, \dots, \v_n\}$ be a linearly
  independent set of vectors in $V$. If $\{\w_1, \w_2, \dots, \w_m\}$ is a spanning
  set of $V$, then
  \begin{enumerate}
    \item $n \leq m$, and

    \item It is possible to replace $n$ vectors in $\{\w_1, \w_2, \dots, \w_m\}$
      with $\{\v_1, \v_2, \dots, \v_n\}$ such that the resulting set still spans
      $V$.\qedhere
  \end{enumerate}
\end{proof}

\begin{problem}[3]
  Let $V$ be a vector space. Prove that the zero vector in $V$ is unique.
\end{problem}

\begin{proof}[Solution]
  \renewcommand\e{\mathbf{e}}\renewcommand\f{\mathbf{f}}
  Let $\e$ and $\f$ be zero vectors in $V$. Since $(\forall \a \in V)[\e \cdot
  \a = \a]$, then, we have $\e \cdot \f = \f$. Since $(\forall \b \in V)[\b
  \cdot \f = \b]$, then, we have $\e \cdot \f = \e$. Therefore, $\e = \f$.
\end{proof}

\begin{problem}[4]
  Let $V = \{(a_1, a_2) \mid a_1, a_2 \in \C\}$. Define
  \[%
    (a_1, a_2) + (b_1, b_2) = (a_1 + b_1, a_2 + b_2) \aand c(a_1, a_2) = (a_1, 0)
  .\]%
  Determine whether or not $V$ is a vector space over $\C$ with these
  operations. Justify your answer.
\end{problem}

\begin{proof}[Solution]
  It isn't a vector space over $\C$ as it doesn't have an identity scalar
  element, as $(\forall c \in \C)[c(a_1, a_2) = (a_1, 0)]$, but in order for $V$
  to be a vector space over $\C$, $(\exists c \in \C)[c(a_1, a_2) = (a_1,
  a_2)]$.
\end{proof}

\begin{problem}[5]
  Let $V = \{(a_1, a_2) \mid a_1, a_2 \in \C\}$. Define
  \[%
    (a_1, a_2) + (b_1, b_2) = (a_1 + 2b_1, a_2 + 3b_2) \aand c(a_1, a_2) = (ca_1, ca_2)
  .\]%
  Determine whether or not $V$ is a vector space over $\C$ with these
  operations. Justify your answer.
\end{problem}

\begin{proof}[Solution]
  It isn't a vector space over $\C$ since it isn't commutative, as $(a_1, a_2) +
  (b_1, b_2) \ne (b_1, b_2) + (a_1, a_2)$.
\end{proof}

\begin{problem}[6]
  If $W_1$ and $W_2$ are subspaces of a vector space $V$, prove that $W_1 \cap
  W_2$ is a subspace of $V$.
\end{problem}

\begin{proof}[Solution]
  Let $\x, \y \in W_1 \cap W_2$. This means that $\x, \y \in W_1$ and $\x, \y
  \in W_2$. Since they both are subspaces of the vector space $V$, then
  $(\forall c \in \F)[c\x + \y \in W_1]$ and $(\forall c \in \F)[c\x + \y \in
  W_2]$. Then, by definition, we get, $(\forall c \in \F)[c\x + \y \in W_1 \cap
  W_2]$. Therefore, $W_1 \cap W_2$ must be a subspace of the vector space $V$.
\end{proof}

\begin{problem}[7]
  Consider the following subsets in $\C^n$:
  \[%
    W_1 = \left\{
      \begin{bmatrix}
        a_1 \\
        a_2 \\
        \vdots \\
        a_n \\
      \end{bmatrix} \in \C^n \mid a_1 + \cdots + a_n = 0
    \right\},
    W_2 = \left\{
      \begin{bmatrix}
        a_1 \\
        a_2 \\
        \vdots \\
        a_n \\
      \end{bmatrix} \in \C^n \mid a_1 + \cdots + a_n = c~\textrm{where}~c \ne 0
    \right\}
  .\]%
  Prove that $W_1$ is a subspace of $\C^n$, but $W_2$ is not a subspace of
  $\C^n$.
\end{problem}

\begin{proof}[Solution]
  The set $W_1$ is a subspace of $\C^n$ as $(\forall c \in \F)(\forall \x, \y
  \in W_1)[c\x + \y \in W_1]$, since $c(a_1 + \cdots + a_n) + (b_1 + \cdots +
  b_n) = c \zero + \zero = \zero \in W_1$.

  The set $W_2$ is not a subspace of $\C^n$ as $(\nexists \e \in W_2)(\forall \v
  \in W_2)[\v + \e = \v = \e + \v]$, since $a_1 + \cdots + a_n \ne 0$.
\end{proof}

\begin{problem}[8]
  Let $S$ be the subset of all symmetric matrices in $\R^{n \times n}$, i.e.,
  $\displaystyle S = \left.\left\{A \in \R^{n \times n} \right\rvert A^T =
  A\right\}$. Prove that $S$ is a subspace of $\R^{n \times n}$.
\end{problem}

\begin{proof}[Solution]
  The set $S$ is a subspace of $\R^{n \times n}$ since $(\forall c \in
  \F)(\forall A, B \in S)[cA + B \in S]$, since $cA + B = (cA + B)^T = (cA)^T +
  (B)^T = cA^T + B^T = cA + B$.
\end{proof}

\begin{problem}[9]
  Let $V$ be a finite-dimensional vector space over $\R$ or $\C$. Let $\BB =
  \{\x_1,\x_2, \cdots, \x_n\}$ be an ordered basis for $V$. Prove that for any
  $\x \in V$, there exists a unique set of scalars $\{a_1, a_2, \cdots, a_n\}$
  such that
  \[%
    \x = a_1\x_1 + a_2\x_2 + \cdots + a_n\x_n
  .\]%
\end{problem}

\begin{proof}[Solution]
  Let $\x \in V$. Since $\BB$ is a basis for $V$, then $\x$ can be written as a
  linear combination of the vectors in $\BB$. By the spanning property, there
  exist scalars $\{a_1, \cdots, a_n\}$ such that $\x = \sum_{i=1}^n a_i\x_i$.
  Suppose there's another set of scalars $\{b_1, \cdots, b_n\}$ such that $\x =
  \sum_{i=1}^n b_i\x_i$. Subtracting the two equations gives us $0 =
  \sum_{i=1}^n (a_i - b_i)\x_i$. Let $c_i = a_i - b_i$. Since $\BB$ is a basis,
  the vectors $\{\x_1, \cdots, \x_n\}$ are linearly independent. Therefore, $c_i
  = 0$ for all $i$. Thus, $a_i = b_i$ for all $i$, and the set of scalars is
  unique.
\end{proof}

\begin{problem}[10]
  True or False. (No explanation needed).
  \begin{enumerate}
    \item A vector space may have more than one zero vector.

    \item If $f$ and $g$ are polynomials of degree $n$, then $f + g$ is a polynomial of degree $n$.

    \item If $V$ is a vector space and $W$ is a subset of $V$ that is a vector space, then $W$ is a subspace.

    \item If $W$ and $U$ are subspaces of $V$, then $W \cup U$ is a subspace of
      $V$.
  \end{enumerate}
\end{problem}

\begin{proof}[Solution to (i)]
  False, since by problem 3, I've shown that the zero vector in a vector space
  is unique.
\end{proof}

\begin{proof}[Solution to (ii)]
  No, take $f(x) = x^2 + x$ and $g(x) = -x^2 + x$. Then, $(f + g)(x) = 2x$.
\end{proof}

\begin{proof}[Solution to (iii)]
  Yes, by definition, a subset $W$ of a vector space $V$ if it's not empty and
  closed under addition and scalar multiplication. But since $W$ is already a
  vector space, that means it's not empty and closed under addition and scalar
  multiplication, so it's a subspace of $V$.
\end{proof}

\begin{proof}[Solution to (iv)]
  No, as it might not be closed under addition or scalar multiplication. For
  example, let $W = \{(x, 0) \mid x \in \R\}$ and $U = \{(0, y) \mid y \in
  \R\}$. Then, $(1, 0) + (0, 1) = (1, 1) \notin W \cup U$.
\end{proof}
