\begin{problem}[1]
  Classify by Jordan Canonical form (i.e. up to similarity and up to the order
  of the Jordan blocks) for all $6 \times 6$ matrices which have characteristic
  polynomial $(x - 2)^2(x + 3)^4$.
\end{problem}

\begin{proof}[Solution]
  We first need to identify the eigenvalues and their algebraic multiplicities
  \begin{enumerate}
    \item $\lambda = 2$ with algebraic multiplicity $2$.
    \item $\lambda = -3$ with algebraic multiplicity $4$.
  \end{enumerate}

  Now, we determine the possible Jordan block structures.
  Since the algebraic multiplicity of $\lambda = 2$ is $2$, the only possible Jordan forms are
  \begin{enumerate}
    \item $J_1(2) \oplus J_1(2)$, both blocks are size $1$,

    \item $J_2(2)$, a single block of size $2$.
  \end{enumerate}

  Since the algebraic multiplicity of $\lambda = -3$ is $4$, the possible Jordan
  block structures are
  \begin{enumerate}
    \item $J_1(-3) \oplus J_1(-3) \oplus J_1(-3) \oplus J_1(-3)$, all blocks are
      size $1$.

    \item $J_2(-3) \oplus J_1(-3) \oplus J_1(-3)$, one block of size $2$, two
      blocks of size $1$.

    \item $J_3(-3) \oplus J_1(-3)$, one block of size $3$, one block of size
      $1$.

    \item $J_2(-3) \oplus J_2(-3)$, two blocks of size $2$.

    \item $J_4(-3)$, a single block of size $4$.
  \end{enumerate}

  Each Jordan form consists of a combination of the Jordan blocks for $\lambda = 2$ and $\lambda = -3$:

  \begin{enumerate}
    \item $J_1(2) \oplus J_1(2) \oplus J_1(-3) \oplus J_1(-3) \oplus J_1(-3)
      \oplus J_1(-3)$

    \item $J_1(2) \oplus J_1(2) \oplus J_2(-3) \oplus J_1(-3) \oplus J_1(-3)$

    \item $J_1(2) \oplus J_1(2) \oplus J_3(-3) \oplus J_1(-3)$

    \item $J_1(2) \oplus J_1(2) \oplus J_2(-3) \oplus J_2(-3)$

    \item $J_1(2) \oplus J_1(2) \oplus J_4(-3)$

    \item $J_2(2) \oplus J_1(-3) \oplus J_1(-3) \oplus J_1(-3) \oplus J_1(-3)$

    \item $J_2(2) \oplus J_2(-3) \oplus J_1(-3) \oplus J_1(-3)$

    \item $J_2(2) \oplus J_3(-3) \oplus J_1(-3)$

    \item $J_2(2) \oplus J_2(-3) \oplus J_2(-3)$

    \item $J_2(2) \oplus J_4(-3)$
  \end{enumerate}

  Therefore, there are 10 distinct Jordan forms for matrices with the given
  characteristic polynomial.
\end{proof}

\begin{problem}[2]
  Find the Jordan canonical form of the matrix $\begin{pmatrix}
    a & 0 & 1 & 0 & \cdots & 0 & 0 & 0 & 0 \\
    0 & a & 0 & 1 & \cdots & 0 & 0 & 0 & 0 \\
    0 & 0 & a & 0 & \cdots & 0 & 0 & 0 & 0 \\
    \vdots & \vdots & \vdots & \vdots & \ddots & \vdots & \vdots & \vdots & \vdots \\
    0 & 0 & 0 & 0 & \cdots & a & 0 & 1 & 0 \\
    0 & 0 & 0 & 0 & \cdots & 0 & a & 0 & 1 \\
    0 & 0 & 0 & 0 & \cdots & 0 & 0 & a & 0 \\
    0 & 0 & 0 & 0 & \cdots & 0 & 0 & 0 & a \\
  \end{pmatrix}$.
\end{problem}

\begin{proof}[Solution]
  We first need to compute the characteristic polynomial. The matrix $A$ has $a$
  along the main diagonal and ones along the second superdiagonal. This
  structure suggests that $A$ is a companion-like matrix shifted by $aI$.

  To determine the characteristic polynomial, consider the determinant of $A -
  xI$
  \[%
    A - xI = \begin{pmatrix}
      a - x & 0 & 1 & 0 & \cdots & 0 & 0 & 0 & 0 \\
      0 & a - x & 0 & 1 & \cdots & 0 & 0 & 0 & 0 \\
      0 & 0 & a - x & 0 & \cdots & 0 & 0 & 0 & 0 \\
      \vdots & \vdots & \vdots & \vdots & \ddots & \vdots & \vdots & \vdots & \vdots \\
      0 & 0 & 0 & 0 & \cdots & a - x & 0 & 1 & 0 \\
      0 & 0 & 0 & 0 & \cdots & 0 & a - x & 0 & 1 \\
      0 & 0 & 0 & 0 & \cdots & 0 & 0 & a - x & 0 \\
      0 & 0 & 0 & 0 & \cdots & 0 & 0 & 0 & a - x \\
    \end{pmatrix}
  .\]%
  By expanding the determinant recursively, we see that the characteristic
  polynomial satisfies $\chi_A(x) = (a - x)^n$ This means the only eigenvalue of
  $A$ is $a$ with algebraic multiplicity $n$.

  To determine the Jordan form, we examine the generalized eigenspaces. Since
  there is a single eigenvalue $a$ and the superdiagonal structure extends
  throughout the matrix, we check the size of the largest Jordan block by
  computing the rank of successive powers of $A - aI$.

  The matrix $A - aI$ is nilpotent. The key observation is that $(A - aI)^{n-1}$
  has only a single nonzero entry, meaning the generalized eigenspace has a
  single block of size $n$.

  Thus, the Jordan form of $A$ is
  \[%
    J(A) = J_n(a) = \begin{pmatrix}
      a & 1 & 0 & \cdots & 0 \\
      0 & a & 1 & \cdots & 0 \\
      0 & 0 & a & \cdots & 0 \\
      \vdots & \vdots & \vdots & \ddots & \vdots \\
      0 & 0 & 0 & \cdots & a \\
    \end{pmatrix}
  .\qedhere\]%
\end{proof}

\begin{problem}[3]
  Let $A = \begin{pmatrix}
    11 & -4 & -5 \\
    21 & -8 & -11 \\
    3 & -1 & 0 \\
  \end{pmatrix}$.
  \begin{enumerate}
    \item Find a basis for each generalized eigenspace of $A$ consisting of a
      union of disjoint cycles of generalized eigenvectors.

    \item Find a Jordan canonical form $J$ of $A$ using your basis in Part (i).

    \item Find the minimal polynomial of $A$.
  \end{enumerate}
\end{problem}

\begin{proof}[Solution to (i)]
  Using Wolfram Alpha, we get the characteristic polynomial
  \[%
    f_A(\lambda) = -(\lambda - 2)^2(\lambda + 1)
  ,\]%
  giving us eigenvalues of $\lambda_1 = -1$ with algebraic multiplicity $1$ and
  $\lambda_2 = 2$ with algebraic multiplicity $2$.

  An eigenvalue will have a generalized eigenvector if its algebraic
  multiplicity is greater than the geometric multiplicity. If so, then the
  eigenvalue will have $n$ generalized eigenvectors, where $n$ is the difference
  between the algebraic multiplicity and the geometric multiplicity.

  First, we work with $\lambda_1 = -1$. We need to first compute the
  eigenvectors, giving us
  \begin{alignat*}{5}
    \underline{\lambda = -1}: \quad& A - \lambda_1 I
    &&= \begin{pmatrix}
      12 & -4 & -5 \\
      21 & -9 & -11 \\
      3 & -1 & 1 \\
    \end{pmatrix}
    &&= \begin{pmatrix}
      1 & -\sfrac{1}{3} & 0 \\
      0 & 0 & 1 \\
      0 & 0 & 0 \\
    \end{pmatrix} \\
    &\implies \ker(A - \lambda_2 I)
    &&= \begin{pmatrix}
      1 & -\sfrac{1}{3} & 0 \\
      0 & 0 & 1 \\
      0 & 0 & 0 \\
    \end{pmatrix}\begin{pmatrix}
      x_1 \\
      x_2 \\
      x_3 \\
    \end{pmatrix}
    &&= \begin{pmatrix}
      0 \\
      0 \\
      0 \\
    \end{pmatrix} \\
    &\implies \v_1 = \begin{pmatrix}
      1 \\
      3 \\
      0 \\
    \end{pmatrix}
  .\end{alignat*}
  Since the algebraic multiplicity of $\lambda_1$ is $1$, there are no
  generalized eigenvectors.

  Now, we find the generalized eigenvectors for $\lambda_2 = 2$ with
  multiplicity of $2$.
  \begin{alignat*}{5}
    \underline{\lambda = -2}: \quad& A - \lambda_2 I
    &&= \begin{pmatrix}
      9 & -4 & -5 \\
      21 & -10 & -11 \\
      3 & -1 & -2 \\
    \end{pmatrix}
    &&= \begin{pmatrix}
      1 & 0 & -1 \\
      0 & 1 & -1 \\
      0 & 0 & 0 \\
    \end{pmatrix} \\
    &\implies \ker(A - \lambda I)
    &&= \begin{pmatrix}
      1 & 0 & -1 \\
      0 & 1 & -1 \\
      0 & 0 & 0 \\
    \end{pmatrix}\begin{pmatrix}
      x_1 \\
      x_2 \\
      x_3 \\
    \end{pmatrix}
    &&= \begin{pmatrix}
      0 \\
      0 \\
      0 \\
    \end{pmatrix} \\
    &\implies
    \v_1 = \begin{pmatrix}
      1 \\
      1 \\
      1 \\
    \end{pmatrix}
  .\end{alignat*}
  Since the algebraic multiplicity of $\lambda_2$ is $2$, there exists a single
  generalized eigenvector. To find it, we solve $(A - \lambda_2I)\v_2 = \v_1$.
  Therefore, we have
  \[%
    (A - \lambda_2I)\v_2 = \v_1 \implies
    \begin{pmatrix}
      9 & -4 & -5 \\
      21 & -10 & -11 \\
      3 & -1 & -2 \\
    \end{pmatrix}\begin{pmatrix}
      x_1 \\
      x_2 \\
      x_3 \\
    \end{pmatrix}
    = \begin{pmatrix}
      1 \\
      1 \\
      1 \\
    \end{pmatrix}
    \implies \v_2 = \begin{pmatrix}
      1 \\
      2 \\
      0 \\
    \end{pmatrix}
  .\]%

  Therefore, the generalized eigenspaces are given by
  \[%
    G_{\lambda_1}(A) = \Sspan\left\{
      \begin{pmatrix}
        1 \\
        3 \\
        0 \\
      \end{pmatrix}
    \right\}
    \aand
    G_{\lambda_2}(A) = \Sspan\left\{
      \begin{pmatrix}
        1 \\
        2 \\
        0 \\
      \end{pmatrix},
      \begin{pmatrix}
        0 \\
        0 \\
        1 \\
      \end{pmatrix}
    \right\}
  .\qedhere\]%
\end{proof}

\begin{proof}[Solution to (ii)]
  The generalized eigenspace for $\lambda_1 = -1$ has dimension $1$, so there
  will be a $1 \times 1$ Jordan block for $-1$.

  The generalized eigenspace for $\lambda_2 = 2$ has dimension $2$. Since the
  geometric multiplicity is $1$ (only one eigenvector for $2$), there will be a
  single $2 \times 2$ Jordan block for $2$.

  The matrix $P$ is constructed from the combinations of the basis for the
  generalized eigenspaces for $\lambda_1$ and $\lambda_2$. Therefore, $A$ can be
  written as
  \[%
    A = PJP^{-1} = \begin{pmatrix}
      1 & 1 & 1 \\
      3 & 1 & 2 \\
      0 & 1 & 0 \\
    \end{pmatrix}
    \begin{pmatrix}
      -1 & 0 & 0 \\
      0 & 2 & 1 \\
      0 & 0 & 2 \\
    \end{pmatrix}
    \begin{pmatrix}
      -2 & 1 & 1 \\
      0 & 0 & 1 \\
      3 & -1 & -2 \\
    \end{pmatrix}
  .\qedhere\]%
\end{proof}

\begin{proof}[Solution to (iii)]
  The eigenvalue $\lambda_1 = -1$ has a $1 \times 1$ Jordan block, so it appears
  in the minimal polynomial as $(x + 1)$. The eigenvalue $\lambda_2 = 2$ has a 2
  × 2 Jordan block, so it appears in the minimal polynomial as $(x - 2)^2$.
  Thus, the minimal polynomial of $A$ is
  \[%
    m_A(x) = (x + 1)(x - 2)^2
  .\qedhere\]%
\end{proof}

\begin{problem}[4]
  Let $D$ be the differential linear operator on the vector space $V =
  \Sspan\{1, t, t^2, e^t, te^t\}$, i.e.,
  \[%
    D(f(x)) = \odv{}{x} f(x),\qtq{for any} f(x) \in V
  .\]%
  \begin{enumerate}
    \item Let $\BB = \{1, t, t^2, e^t, te^t\}$. Find the matrix representation
      $A = [D]_\BB$.

    \item Find a basis for each generalized eigenspace of $D$ consisting of a
      union of disjoint cycles of generalized eigenvectors.

    \item Find a Jordan canonical form $J$ of $D$ using your basis in Part (ii).

    \item Find the minimal polynomial of $D$.
  \end{enumerate}
\end{problem}

\begin{proof}[Solution to (i)]
  To find the matrix representation of $D$ with respect to the basis $\BB$, we
  need to compute $D$ applied to each basis element and express the result as a
  linear combination of the basis elements.

  \begin{align*}
    D(1) = 0, \quad D(t) = 1, \quad D(t^2) = 2t, \quad D(e^t) = e^t, \aand D(te^t) = e^t + te^t
  .\end{align*}

  Now, we can express these results in terms of the basis $\BB$:
  \begin{align*}
    D(1) &= 0 \cdot 1 + 0 \cdot t + 0 \cdot t^2 + 0 \cdot e^t + 0 \cdot te^t \\
    D(t) &= 1 \cdot 1 + 0 \cdot t + 0 \cdot t^2 + 0 \cdot e^t + 0 \cdot te^t \\
    D(t^2) &= 0 \cdot 1 + 2 \cdot t + 0 \cdot t^2 + 0 \cdot e^t + 0 \cdot te^t \\
    D(e^t) &= 0 \cdot 1 + 0 \cdot t + 0 \cdot t^2 + 1 \cdot e^t + 0 \cdot te^t \\
    D(te^t) &= 0 \cdot 1 + 0 \cdot t + 0 \cdot t^2 + 1 \cdot e^t + 1 \cdot te^t
  .\end{align*}
  Therefore, the matrix representation $A = [D]_\BB$ is given by
  \[%
    A = \begin{pmatrix}
      0 & 1 & 0 & 0 & 0 \\
      0 & 0 & 2 & 0 & 0 \\
      0 & 0 & 0 & 0 & 0 \\
      0 & 0 & 0 & 1 & 1 \\
      0 & 0 & 0 & 0 & 1 \\
    \end{pmatrix}
  .\qedhere\]%
\end{proof}

\begin{proof}[Solution to (ii)]
  First, we need to find the characteristic polynomial of $A$, giving us
  \[%
    \det(A - \lambda I) = \det\begin{pmatrix}
      -\lambda & 1 & 0 & 0 & 0 \\
      0 & -\lambda & 2 & 0 & 0 \\
      0 & 0 & -\lambda & 0 & 0 \\
      0 & 0 & 0 & 1 - \lambda & 1 \\
      0 & 0 & 0 & 0 & 1 - \lambda \\
    \end{pmatrix} = -\lambda^3(\lambda - 1)^2
  .\]%
  The eigenvalues are $\lambda_1 = 0$ with algebraic multiplicity $3$ and
  $\lambda_2 = 1$ with algebraic multiplicity $2$. Next, we find the
  eigenvectors and generalized eigenvectors for each eigenvalue.

  For $\lambda_1 = 0$, we have
  \begin{alignat*}{4}
    A\v_1 &= 0 &&\implies
    \begin{pmatrix}
      0 & 1 & 0 & 0 & 0 \\
      0 & 0 & 2 & 0 & 0 \\
      0 & 0 & 0 & 0 & 0 \\
      0 & 0 & 0 & 1 & 1 \\
      0 & 0 & 0 & 0 & 1 \\
    \end{pmatrix}\begin{pmatrix}
      x_1 \\
      x_2 \\
      x_3 \\
      x_4 \\
      x_5 \\
    \end{pmatrix} &&= \begin{pmatrix}
      0 \\
      0 \\
      0 \\
      0 \\
      0 \\
    \end{pmatrix} \\
    &&&\implies \begin{pmatrix}
      0 & 1 & 0 & 0 & 0 \\
      0 & 0 & 1 & 0 & 0 \\
      0 & 0 & 0 & 1 & 0 \\
      0 & 0 & 0 & 0 & 1 \\
      0 & 0 & 0 & 0 & 0 \\
    \end{pmatrix}\begin{pmatrix}
      x_1 \\
      x_2 \\
      x_3 \\
      x_4 \\
      x_5 \\
    \end{pmatrix} &&= \begin{pmatrix}
      0 \\
      0 \\
      0 \\
      0 \\
      0 \\
    \end{pmatrix}
    \implies\v_1 = \begin{pmatrix}
      1 \\
      0 \\
      0 \\
      0 \\
      0 \\
    \end{pmatrix}
  \end{alignat*}
  Since the algebraic multiplicity of $\lambda_1$ is $3$ and the geometric
  multiplicity is $1$, there are $2$ generalized eigenvectors. To find them like
  we did in problem 3, we solve $(A - \lambda_1I)\v_2 = \v_1$ and $(A -
  \lambda_1I)\v_3 = \v_2$.
  \begin{alignat*}{3}
    (A - \lambda_1I)\v_2 &= \v_1 &&\implies
    \begin{pmatrix}
      0 & 1 & 0 & 0 & 0 \\
      0 & 0 & 2 & 0 & 0 \\
      0 & 0 & 0 & 0 & 0 \\
      0 & 0 & 0 & 1 & 1 \\
      0 & 0 & 0 & 0 & 1 \\
    \end{pmatrix}\begin{pmatrix}
      x_1 \\
      x_2 \\
      x_3 \\
      x_4 \\
      x_5 \\
      \end{pmatrix} &&= \begin{pmatrix}
      1 \\
      0 \\
      0 \\
      0 \\
      0 \\
    \end{pmatrix} \implies \v_2 = \begin{pmatrix}
      0 \\
      1 \\
      0 \\
      0 \\
      0 \\
    \end{pmatrix} \\
    (A - \lambda_1I)\v_3 &= \v_2 &&\implies
    \begin{pmatrix}
      0 & 1 & 0 & 0 & 0 \\
      0 & 0 & 2 & 0 & 0 \\
      0 & 0 & 0 & 0 & 0 \\
      0 & 0 & 0 & 1 & 1 \\
      0 & 0 & 0 & 0 & 1 \\
    \end{pmatrix}\begin{pmatrix}
      x_1 \\
      x_2 \\
      x_3 \\
      x_4 \\
      x_5 \\
    \end{pmatrix} &&= \begin{pmatrix}
      0 \\
      1 \\
      0 \\
      0 \\
      0 \\
    \end{pmatrix} \implies\v_3 = \begin{pmatrix}
      0 \\
      0 \\
      \sfrac{1}{2} \\
      0 \\
      0 \\
    \end{pmatrix}
  .\end{alignat*}

  For $\lambda_2 = 1$, we have
  \begin{alignat*}{3}
    (A - \lambda_2 I) = \v_1 &\implies
    \begin{pmatrix}
      -1 & 1 & 0 & 0 & 0 \\
      0 & -1 & 2 & 0 & 0 \\
      0 & 0 & -1 & 0 & 0 \\
      0 & 0 & 0 & 0 & 1 \\
      0 & 0 & 0 & 0 & 0 \\
    \end{pmatrix}\begin{pmatrix}
      x_1 \\
      x_2 \\
      x_3 \\
      x_4 \\
      x_5 \\
    \end{pmatrix} &&= \begin{pmatrix}
      0 \\
      0 \\
      0 \\
      0 \\
      0 \\
    \end{pmatrix} \\
    &\implies \begin{pmatrix}
      1 & 0 & 0 & 0 & 0 \\
      0 & 1 & 0 & 0 & 0 \\
      0 & 0 & 1 & 0 & 0 \\
      0 & 0 & 0 & 0 & 1 \\
      0 & 0 & 0 & 0 & 0 \\
    \end{pmatrix}
    \begin{pmatrix}
      x_1 \\
      x_2 \\
      x_3 \\
      x_4 \\
      x_5 \\
    \end{pmatrix} &&= \begin{pmatrix}
      0 \\
      0 \\
      0 \\
      0 \\
      0 \\
    \end{pmatrix} \implies \v_1 = \begin{pmatrix}
      0 \\
      0 \\
      0 \\
      1 \\
      0 \\
    \end{pmatrix}
  .\end{alignat*}
  Since the algebraic multiplicity of $\lambda_2$ is $2$ and the geometric
  multiplicity is $1$, there is $1$ generalized eigenvector. To find it, we
  solve $(A - \lambda_2I)\v_2 = \v_1$.
  \begin{alignat*}{3}
    (A - \lambda_2I)\v_2 &= \v_1 \implies
    \begin{pmatrix}
      -1 & 1 & 0 & 0 & 0 \\
      0 & -1 & 2 & 0 & 0 \\
      0 & 0 & -1 & 0 & 0 \\
      0 & 0 & 0 & 0 & 1 \\
      0 & 0 & 0 & 0 & 0 \\
    \end{pmatrix}\begin{pmatrix}
      x_1 \\
      x_2 \\
      x_3 \\
      x_4 \\
      x_5 \\
    \end{pmatrix} &&= \begin{pmatrix}
      0 \\
      0 \\
      0 \\
      1 \\
      0 \\
    \end{pmatrix} \implies
    \begin{pmatrix}
      0 \\
      0 \\
      0 \\
      0 \\
      1 \\
    \end{pmatrix}
  .\end{alignat*}

  Therefore, the generalized eigenspaces are given by
  \[%
    G_{\lambda_1}(D) = \Sspan\left\{
      \begin{pmatrix}
        1 \\
        0 \\
        0 \\
        0 \\
        0 \\
      \end{pmatrix},
      \begin{pmatrix}
        0 \\
        1 \\
        0 \\
        0 \\
        0 \\
      \end{pmatrix},
      \begin{pmatrix}
        0 \\
        0 \\
        \sfrac{1}{2} \\
        0 \\
        0 \\
      \end{pmatrix}
    \right\}
    \aand
    G_{\lambda_2}(D) = \Sspan\left\{
      \begin{pmatrix}
        0 \\
        0 \\
        0 \\
        1 \\
        0
      \end{pmatrix},
      \begin{pmatrix}
        0 \\
        0 \\
        0 \\
        0 \\
        1
      \end{pmatrix}
    \right\}
  .\qedhere\]%
\end{proof}

\begin{proof}[Solution to (iii)]
  The generalized eigenspace for $\lambda_1 = 0$ has dimension $3$, so there
  will be a single $3 \times 3$ Jordan block for $0$.

  The generalized eigenspace for $\lambda_2 = 1$ has dimension $2$. Since the
  geometric multiplicity is $1$ (only one eigenvector for $1$), there will be a
  single $2 \times 2$ Jordan block for $1$.

  The matrix $P$ is constructed from the combinations of the basis for the
  generalized eigenspaces for $\lambda_1$ and $\lambda_2$. Therefore, $A$ can be
  written as
  \[%
    A = PJP^{-1} = \begin{pmatrix}
      1 & 0 & 0 & 0 & 0 \\
      0 & 1 & 0 & 0 & 0 \\
      0 & 0 & \sfrac{1}{2} & 0 & 0 \\
      0 & 0 & 0 & -1 & 0 \\
      0 & 0 & 0 & 0 & -1 \\
    \end{pmatrix}
    \begin{pmatrix}
      0 & 1 & 0 & 0 & 0 \\
      0 & 0 & 1 & 0 & 0 \\
      0 & 0 & 0 & 0 & 0 \\
      0 & 0 & 0 & 1 & 1 \\
      0 & 0 & 0 & 0 & 1 \\
    \end{pmatrix}
    \begin{pmatrix}
      1 & 0 & 0 & 0 & 0 \\
      0 & 1 & 0 & 0 & 0 \\
      0 & 0 & 2 & 0 & 0 \\
      0 & 0 & 0 & -1 & 0 \\
      0 & 0 & 0 & 0 & -1 \\
    \end{pmatrix}
  .\qedhere\]%
\end{proof}

\begin{proof}[Solution to (iv)]
  The eigenvalue $\lambda_1 = 0$ has a $3 \times 3$ Jordan block, so it appears
  in the minimal polynomial as $x^3$. The eigenvalue $\lambda_2 = 1$ has a $2 \times
  2$ Jordan block, so it appears in the minimal polynomial as $(x - 1)^2$. Thus,
  the minimal polynomial of $D$ is
  \[%
    m_D(x) = x^3(x - 1)^2
  .\qedhere\]%
\end{proof}

\begin{problem}[5]
  Let $A \in \C^{n \times n}$. Assume $(A + I_n)^m = 0$. Prove that $A$ is
  invertible and find $\det(A)$
\end{problem}

\begin{proof}[Solution]
  To show that $A$ is invertible, we must prove that $\det(A) \neq 0$, or
  equivalently, that $A$ is non-singular. The given condition $(A + I_n)^m = 0$
  means that $A + I_n$ is nilpotent, i.e., all its eigenvalues are zero. That
  is, for every eigenvalue $\lambda$ of $A + I_n$, we have $\lambda^m = 0$,
  implying $\lambda = 0$. This means that the only eigenvalue of $A + I_n$ is 0,
  so the eigenvalues of $A$ satisfy
  \[%
    \mu + 1 = 0 \implies \mu = -1
  .\]%
  Thus, all eigenvalues of $A$ are $-1$, which means the matrix $A$ is
  diagonalizable (if it is already diagonalizable) or similar to a Jordan block
  form with eigenvalues $-1$. Since $-1$ is never zero, all eigenvalues of $A$
  are nonzero, which implies that $A$ is invertible.

  Since the determinant of a matrix is the product of its eigenvalues and all
  eigenvalues of $A$ are $-1$, we obtain $\det(A) = (-1)^n$.
\end{proof}

\begin{problem}[6]
  Let $V$ be the vector space of all polynomials over $\R$. Let $D$ be the
  differentiation operator on the vector space $V$. Find the minimal polynomial
  of $D$ on $V$ or prove that $D$ has no minimal polynomial on $V$.
\end{problem}

\begin{proof}[Solution]
  We consider the minimal polynomial $m_D(x)$, which is the monic polynomial of
  least degree such that $m_D(D) = 0$. This means there must exist a nonzero
  polynomial $m_D(x) = a_0 + a_1x + \dots + a_kx^k$ such that
  \[%
    m_D(D) = a_0 I + a_1 D + \cdots + a_k D^k = 0
  .\]%
  In other words, applying this polynomial expression of $D$ to any polynomial
  $p(x)$ must yield $0$.

  Now, we check if $D$ Has a Minimal Polynomial. If $D$ had a minimal polynomial
  of degree $k$, then for every polynomial $p(x)$, applying $D$ repeatedly up to
  the $k$-th derivative should result in zero. However, differentiation does not
  satisfy such a condition for all polynomials. Specifically, the space of
  polynomials is infinite-dimensional. We also know that there exist polynomials
  of arbitrarily high degree, and differentiation reduces the degree by one at
  each step. Therefore, no finite power $D^k$ annihilates all polynomials; for
  any $k$, there exist polynomials of degree higher than $k$, whose $k$-th
  derivative is nonzero.

  Thus, there is no nonzero polynomial $m_D(x)$ such that $m_D(D)$ annihilates
  all polynomials.

  Since $D$ does not satisfy a minimal polynomial equation for all polynomials
  in $V$, we conclude that $D$ has no minimal polynomial on $V$.
\end{proof}

\begin{problem}[7]
  Suppose $A \in \C^{n \times n}$ satisfies $A^2 = A$. Prove that $A$ is
  diagonalizable.
\end{problem}

\begin{proof}[Solution]
  Since $A$ satisfies $A^2 - A = 0$, the minimal polynomial of $A$, which is the
  monic polynomial of least degree that annihilates $A$, must divide $x^2 - x$.
  The polynomial $x^2 - x = x(x - 1)$. Thus, the minimal polynomial of $A$ must
  be one of
  \[%
    x, \quad x - 1, \oor x(x - 1)
  .\]%
  This means that the only possible eigenvalues of $A$ are $0$ and $1$.

  A matrix is diagonalizable if and only if its minimal polynomial splits
  completely into distinct linear factors. We have already determined that the
  minimal polynomial of $A$ is one of
  \[%
    x, \quad x - 1, \oor x(x - 1)
  ,\]%
  all of which are products of distinct linear factors. This guarantees that $A$
  is diagonalizable.

  Since the only possible eigenvalues are $0$ and $1$, we decompose the space
  $\C^n$ as a direct sum of the eigenspaces corresponding to $0$ and $1$
  \[%
    \C^n = \ker(A) \oplus \Im(A)
  .\]%
  From $A^2 = A$, it follows that if $\v$ is in the image of $A$, then $A\v =
  \v$, meaning all vectors in $\Im(A)$ are eigenvectors corresponding to
  eigenvalue $1$. Similarly, if $\v \in \ker(A)$, then $A\v = \zero$, meaning
  all vectors in $\ker(A)$ are eigenvectors corresponding to eigenvalue $0$.

  Since the generalized eigenspaces span $\C^n$ and contain only true
  eigenvectors (i.e., there are no nontrivial Jordan blocks), $A$ is
  diagonalizable.
\end{proof}

\begin{problem}[8]
  Give an example of two matrices who have the same characteristic polynomial
  but distinct minimal polynomials.
\end{problem}

\begin{proof}[Solution]
  Define $A$ and $B$ as the following matrices
  \[%
    A = \begin{pmatrix}
      2 & 0 & 0 \\
      0 & 2 & 0 \\
      0 & 0 & 3 \\
    \end{pmatrix}
    \aand
    B = \begin{pmatrix}
      2 & 1 & 0 \\
      0 & 2 & 0 \\
      0 & 0 & 3 \\
    \end{pmatrix}
  .\]%
  Their characteristic polynomials are
  \[%
    \det(A) = -(\lambda - 3)(\lambda - 2)^2 = \det(B)
  ,\]%
  but their characteristic polynomials are
  \[%
    m_A(x) = (x - 2)(x - 3)
    \aand
    m_B(x) = (x - 3)(x - 2)^2
  .\]%
\end{proof}
