\begin{problem}[1]
  State the rank-nullity theorem for an $m \times n$ matrix with real entries.
\end{problem}

\begin{proof}[Solution]
  Let $A$ be an $m \times n$ matrix with real entries. Then the rank-nullity theorem states that
  \[%
    \rank(A) + \Nul(A) = n \aand \rank(A^T) + \Nul(A) = m
  .\qedhere\]%
\end{proof}

\begin{problem}[2]
  Given that $\displaystyle A = \begin{bmatrix}
    1 & 2 & -5 & 11 & -3 \\
    2 & 4 & -5 & 15 & 2 \\
    1 & 2 & 0 & 4 & 5 \\
    3 & 6 & -5 & 19 & -2 \\
  \end{bmatrix}$
  is row equivalent to $B = \begin{bmatrix}
    1 & 2 & 0 & 4 & 5 \\
    0 & 0 & 5 & -7 & 8 \\
    0 & 0 & 0 & 0 & -9 \\
    0 & 0 & 0 & 0 & 0 \\
  \end{bmatrix}$, find a basis for $\Nul(A)$ and a basis for $\range(A)$.
\end{problem}

\begin{proof}[Solution]
  The null space of $A$ consists of all solutions to the equation $A\x = \zero$.
  We need to convert $B$ into reduced row echelon form to find the basis for the
  null space of $A$. The reduced row echelon form of $B$ is
  \[%
    \begin{bmatrix}
      1 & 2 & 0 & 4 & 0 \\
      0 & 0 & 1 & -\frac{7}{5} & 0 \\
      0 & 0 & 0 & 0 & 1 \\
      0 & 0 & 0 & 0 & 0 \\
    \end{bmatrix} \rightarrow
    \begin{aligned}
      x_1 &= -2x_2 - 4x_4 \\
      x_2 &= x_2 \\
      x_3 &= \frac{7}{5}x_4 \\
      x_4 &= x_4 \\
      x_5 &= 0
    \end{aligned}
  .\]%
  Parametrizing the solution, we have
  \[%
     \x = \begin{bmatrix}
       -2x_2 - 4x_4 \\
       x_2 \\
       \frac{7}{5}x_4 \\
       x_4 \\
       0 \\
     \end{bmatrix} =
     x_2\begin{bmatrix}
       -2 \\
       1 \\
       0 \\
       0 \\
       0 \\
     \end{bmatrix} +
     x_4\begin{bmatrix}
       -4 \\
       0 \\
       \frac{7}{5} \\
       1 \\
       0 \\
     \end{bmatrix}
  .\]%
  Therefore, a basis for $\Nul(A)$ is
  \[%
    \left\{
     \begin{bmatrix}
       -2 \\
       1 \\
       0 \\
       0 \\
       0 \\
     \end{bmatrix},
     \begin{bmatrix}
       -4 \\
       0 \\
       \frac{7}{5} \\
       1 \\
       0 \\
     \end{bmatrix}
    \right\}
  .\]%

  The basis for the range of $A$ (or $\Col(A)$) is the set of pivot columns of
  $A$. The pivot columns of $A$ are the first, third, and fourth columns.
  Therefore, a basis for $\range(A)$ is
  \[%
    \left\{
      \begin{bmatrix}
        1 \\
        2 \\
        1 \\
        3 \\
      \end{bmatrix},
      \begin{bmatrix}
        -5 \\
        -5 \\
        0 \\
        -5 \\
      \end{bmatrix},
      \begin{bmatrix}
        -3 \\
        2 \\
        5 \\
        -2 \\
      \end{bmatrix}
    \right\}
  .\qedhere\]%
\end{proof}

\begin{problem}[3]
  Let $A$ be an $m \times n$ matrix with real entires. Prove that $A = 0$ if and
  only if $\Tr(A^TA) = 0$.
\end{problem}

\begin{proof}[Solution] $ $
  \begin{enumerate}
    \item[$\implies)$] Assume $A = 0$, where $0$ is the $m \times n$
      zero matrix. Then,
      \[%
        A^T A = 0^T 0 = 0
      ,\]%
      and the trace of the zero matrix is
      \[%
        \Tr(A^T A) = \Tr(0) = 0
      .\]%
      Thus, if $A = 0$, we have $\Tr(A^T A) = 0$.

    \item[$\impliedby)$] Assume $\Tr(A^T A) = 0$. By the definition of $A^T A$,
      \[%
        A^T A = \begin{bmatrix}
          \langle \mathbf{a}_1, \mathbf{a}_1 \rangle & \langle \mathbf{a}_1, \mathbf{a}_2 \rangle & \cdots & \langle \mathbf{a}_1, \mathbf{a}_n \rangle \\
          \langle \mathbf{a}_2, \mathbf{a}_1 \rangle & \langle \mathbf{a}_2, \mathbf{a}_2 \rangle & \cdots & \langle \mathbf{a}_2, \mathbf{a}_n \rangle \\
          \vdots & \vdots & \ddots & \vdots \\
          \langle \mathbf{a}_n, \mathbf{a}_1 \rangle & \langle \mathbf{a}_n, \mathbf{a}_2 \rangle & \cdots & \langle \mathbf{a}_n, \mathbf{a}_n \rangle
        \end{bmatrix}
      ,\]%
      where $\a_i$ are the column vectors of $A$. The trace of $A^T A$ is
      \[%
        \Tr(A^T A) = \sum_{i=1}^n \langle \a_i, \a_i \rangle = \sum_{i=1}^n \lVert \a_i \rVert^2
      .\]%
      \[%
        \text{Tr}(A^T A) = \sum_{i=1}^n \langle \mathbf{a}_i, \mathbf{a}_i \rangle = \sum_{i=1}^n \lVert \a_i \rVert^2
      .\]%
      Since $\Tr(A^T A) = 0$, it follows that
      \[%
        \sum_{i=1}^n \lVert \a_i \rVert^2 = 0
      .\]%
      Each term $\lVert \a_i \rVert^2$ is a sum of squares of the entries of
      $\a_i$ and is non-negative. Therefore, $\lVert \a_i \rVert^2 = 0$ for all
      $i$, which implies $\a_i = 0$ for all $i$.

      Thus, $A = 0$. \qedhere
  \end{enumerate}
\end{proof}

\begin{problem}[4]
  True or False. No explanation needed.
  \begin{enumerate}
    \item If $A$ is a $3 \times 3$ matrix, then $\det(3A) = 9 \det(A)$.

    \item If $A$, $B$ are invertible $n \times n$ matrices, then $[(AB)^T]^{-1} = (B^T)^{-1}(A^T)^{-1}$
  \end{enumerate}
\end{problem}

\begin{proof}[Solution] $ $
  \begin{enumerate}
    \item False. If $A$ is a $3 \times 3$ matrix, then $\det(3A) = 3^3 \det(A) =
      27 \det(A)$, not $9 \det(A)$.

    \item True. The inverse of the transpose of a product of matrices is the
      product of the transposes of the inverses, i.e., $[(AB)^T]^{-1} =
      (B^T)^{-1}(A^T)^{-1}$.
  \end{enumerate}
\end{proof}
