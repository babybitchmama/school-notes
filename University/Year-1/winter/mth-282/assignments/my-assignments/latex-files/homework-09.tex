% \begin{problem}[1]
%   Evaluate $\displaystyle\oint_C \FF \cdot \drr$ where $\FF = \left\langle 2xy,
%   z^3 + 3x^2, 3yz^2 - x \right\rangle$, $C$ is the boundary of the triangle with
%   vertices $P = (2, 0, 0)$, $Q = (0, 3, 0)$, and $R = (0, 0, 5)$ oriented from
%   $P$ to $Q$ to $R$ and back to $P$.
% \end{problem}

% \begin{proof}[Solution]
%   Since we're evaluating a closed integral over a positive orientated curve, we
%   can use Stokes' Theorem to convert the line integral into a surface integral
%   \[%
%     \oint_C \FF \cdot \drr = \iint_S \curl(\FF) \cdot \dSS
%   ,\]%
%   where $S$ is the surface bounded by $C$ and $\curl(\FF) = \nabla \times \FF$.
%   We first compute the curl of $\FF$
%   \begin{align*}
%     \curl(\FF) = \nabla \times \FF &= \begin{vmatrix}
%       \ui & \uj & \uk \\
%       \pdv{}/{x} & \pdv[style-var-/=multiple]{}/{y} & \pdv[style-var-/=multiple]{}/{z} \\
%       2xy & z^3 + 3x^2 & 3yz^2 - x \\
%     \end{vmatrix} \\
%     &= \left\langle 3z^2 - 3z^2, 1, 6x - 2x \right\rangle \\
%     &= \left\langle 0, 1, 4x \right\rangle
%   .\end{align*}
%   To find the normal to the surface $S$, we can use the cross product of two
%   vectors,
%   \begin{alignat*}{10}
%     &\overrightarrow{PQ} &&= Q - P &&= \langle 0, 3, 0 \rangle - \langle 2, 0, 0 \rangle &&= \langle -2, 3, 0 \rangle \\
%     \aand&\overrightarrow{PR} &&= R - P &&= \langle 0, 0, 5 \rangle - \langle 2, 0, 0 \rangle &&= \langle -2, 0, 5 \rangle
%   ,\end{alignat*}
%   to find
%   \[%
%     \n = \overrightarrow{PQ} \times \overrightarrow{PR} = \begin{vmatrix}
%       \ui & \uj & \uk \\
%       -2 & 3 & 0 \\
%       -2 & 0 & 5 \\
%     \end{vmatrix} = \langle 15, 10, 6 \rangle
%   .\]%
%   But, for the surface integral, we need the unit normal vector
%   \[%
%     \hat{\n} = \frac{\overrightarrow{PQ} \times \overrightarrow{PR}}{\left\lVert \overrightarrow{PQ} \times \overrightarrow{PR} \right\rVert} = \frac{\langle 15, 10, 6 \rangle}{\sqrt{15^2 + 10^2 + 6^2}} = \frac{\langle 15, 10, 6 \rangle}{19} = \left\langle \frac{15}{19}, \frac{10}{19}, \frac{6}{19} \right\rangle
%   .\]%
%   The projection of $S$ onto the $xy$-plane is a triangle with vertices $P = (2,
%   0)$, $Q = (0, 3)$, and $R = (0, 0)$. The equation of the line from $P$ to $Q$
%   is $y = -\sfrac{3}{2} x + 3$. Therefore, the bounds are $0 \le x \le 2$ and $0
%   \le y \le -\sfrac{3}{2} x + 3$. The surface integral becomes
%   \begin{align*}
%     \oint_C \FF \cdot \dd{\r} &= \iint_S \curl(\FF) \cdot \dSS \\
%                              &= \iint_S \left\langle 0, 1, 4x \right\rangle \cdot \left\langle \frac{15}{19}, \frac{10}{19}, \frac{6}{19} \right\rangle \dSS \\
%                              &= \iint_D 0 + 1\cdot\frac{10}{19} + 4x\cdot\frac{6}{19} \dA \\
%                              &= \int_0^2 \int_0^{-\sfrac{3}{2} x + 3} \left(\frac{10}{19} + \frac{24x}{19}\right) \dyx \\
%                              &= \int_0^2 \left(\frac{10}{19} + \frac{24x}{19}\right)\left(-\frac{3}{2}x + 3\right) \dx \\
%                              &= -\frac{3}{19} \cdot \int_0^2 12x^2 - 19x - 10 \dx \\
%                              &= -\frac{3}{19} \cdot \left[4x^3 - \frac{19x^2}{2} - 10x\right]_0^2 \\
%                              &= -\frac{3}{19} \cdot \left[4(2)^3 - \frac{19(2)^2}{2} - 10(2)\right] = \frac{78}{19}
%   .\qedhere\end{align*}
% \end{proof}

% \begin{problem}[2]
%   Let $S$ be the portion of the ellipsoid $4x^2 + y^2 + 16z^2 = 64$ above the
%   $xy$-plane oriented upward. Use Stokes' Theorem to evaluate
%   $\displaystyle\iint_S \curl(\FF) \cdot \dSS$ where $\FF = \left\langle xz, x^3
%   + 2y, e^{x^2-y^2} \right\rangle$.
% \end{problem}

% \begin{proof}[Solution]
%   The boundary $C$ occurs when $z = 0$, which gives the ellipse $4x^2 + y^2 =
%   64$. This also gives us a simpler form of $\FF$,
%   \[%
%     \FF = \left\langle 0, x^3 + 2y, e^{x^2-y^2} \right\rangle
%   .\]%
%   Since
%   \[%
%     \frac{x^2}{16} + \frac{y^2}{64} = 1
%   ,\]%
%   we can use the parametrization
%   \[%
%     \r(t) = \langle 4\cos(t), 8\sin(t), 0 \rangle
%   ,\]%
%   where $0 \le t \le 2\pi$. Therefore, we have
%   \[%
%     \drr = \langle \dx, \dy, \dz \rangle = \langle -4\sin(t), 8\cos(t), 0 \rangle \dt
%   .\]%
%   Computing the dot product, we have
%   \begin{align*}
%     \FF \cdot \drr &= \left\langle 0, x^3 + 2y, e^{x^2-y^2} \right\rangle \cdot \langle \dx, \dy, 0 \rangle \\
%                    &= (x^3 + 2y) \dy \\
%                    &= \left((4\cos(t))^3 + 2(8\sin(t))\right) \cdot 8\cos(t) \dt \\
%                    &= \left(64\cos^3(t) + 16\sin(t)\right) \cdot 8\cos(t) \dt \\
%                    &= 512\cos^4(t) + 128\sin(t)\cos(t) \dt
%   .\end{align*}
%   By Stokes' Theorem, we have
%   \begin{align*}
%     \iint_S \curl(\FF) \cdot \dSS &= \oint_C \FF \cdot \drr \\
%                                  &= \int_0^{2\pi} 512\cos^4(t) + 128\sin(t)\cos(t) \dt \\
%                                  &= 512\int_0^{2\pi} \cos^4(t) \dt + 128\int_0^{2\pi} \sin(t)\cos(t) \dt \\
%                                  &= 512\cdot\frac{3\pi}{4} + 128\cdot 0 = 384\pi
%   .\qedhere\end{align*}
% \end{proof}

% \begin{problem}[3]
%   Verify Stokes' Theorem for the vector field $\FF = \langle y, z, x \rangle$
%   and the hemisphere $y = \sqrt{1 - x^2 - z^2}$ orientated in the direction of
%   the positive $y$-axis.

%   \noindent(That means, evaluate both $\displaystyle\oint_C \FF \cdot \dd{\r}$
%   and $\displaystyle\iint_S \curl(\FF) \cdot \dSS$ showing that they are equal
%   for the given field and surface.)
% \end{problem}

% \begin{proof}[Solution]
%   We're first going to compute the line integral over $C$. We're given that the
%   surface is defined as $y = \sqrt{1 - x^2 - y^2}$. This means that $x^2 + y^2
%   \le 1$. The boundary $C$ occurs when $y = 0$, which gives us the circle $x^2 +
%   z^2 = 1$. Therefore, using polar, we get the parameterization
%   \[%
%     \r(t) = \langle \cos(t), 0, \sin(t) \rangle
%   ,\]%
%   where $r = 1$ and $0 \le t \le 2\pi$. The differential is
%   \[%
%     \drr = \langle \dx, \dy, \dz \rangle = \langle -\sin(t), 0, \cos(t) \rangle \dt
%   .\]%
%   Computing the dot product, we have
%   \[%
%     \FF \cdot \drr = \langle 0, z, x \rangle \cdot \langle \dx, 0, \dz \rangle = x \cdot \dz = \cos^2(t) \dt
%   .\]%
%   Therefore, we have
%   \begin{align*}
%     \oint_C \FF \cdot \drr &= \int_0^{2\pi} \cos^2(t) \dt \\
%                           &= \int_0^{2\pi} \frac{1 + \cos(2t)}{2} \dt \\
%                           &= \frac{1}{2} \cdot \int_0^{2\pi} 1 + \cos(2t) \dt \\
%                           &= \frac{t}{2} + \frac{\sin(2t)}{4} \bigg|_0^{2\pi} \\
%                           &= \frac{2\pi}{2} + 0 - 0 = \pi
%   .\end{align*}

%   Now, we compute the double integral over $S$. We first compute the curl of
%   $\FF$
%   \begin{align*}
%     \curl(\FF) = \nabla \times \FF &= \begin{vmatrix}
%       \ui & \uj & \uk \\
%       \pdv{}/{x} & \pdv[style-var-/=multiple]{}/{y} & \pdv[style-var-/=multiple]{}/{z} \\
%       y & z & x \\
%     \end{vmatrix} \\
%     &= \left\langle 0 - 1, -(1 - 0), 0 - 1 \right\rangle \\
%     &= \langle -1, -1, -1 \rangle
%   .\end{align*}
%   Using polar coordinates, we have
%   \[%
%     \r(\theta, r) = \langle r\cos(\theta), \sqrt{1 - r^2}, r\sin(\theta) \rangle
%   .\]%
%   The Jacobian determinant for this transformation is $r$. The limits of
%   integration are $0 \le r \le 1$ and $0 \le \theta \le 2\pi$. We know that
%   \[%
%     \r_\theta = \langle -r\sin(\theta), 0, r\cos(\theta) \rangle \aand \r_r = \left\langle \cos(\theta), -\frac{2r}{\sqrt{1 - r^2}}, \sin(\theta) \right\rangle
%   .\]%
%   Therefore, the normal vector is
%   \begin{align*}
%     \n &= \r_\theta \times \r_r = \begin{vmatrix}
%       \ui & \uj & \uk \\
%       -r\sin(\theta) & 0 & r\cos(\theta) \\
%       \cos(\theta) & -\frac{2r}{\sqrt{1 - r^2}} & \sin(\theta) \\
%     \end{vmatrix} \\
%     &= \left\langle \frac{2r^2\cos(\theta)}{\sqrt{1 - r^2}}, r\sin^2(\theta) + r\cos^2(\theta), \frac{2r^2\sin(\theta)}{\sqrt{1 - r^2}} \right\rangle \\
%     &= \left\langle \frac{2r^2\cos(\theta)}{\sqrt{1 - r^2}}, r, \frac{2r^2\sin(\theta)}{\sqrt{1 - r^2}} \right\rangle
%   .\end{align*}
% \end{proof}

% \begin{problem}[4]
%   Evaluate $\displaystyle\oiint_S \FF \cdot \dd{\S}$ where $\FF = \langle x^3 +
%   zy^2, 4x^2z + 2yz, 4 - z^2 \rangle$ and $S$ is the sphere $x^2 + y^2 + z^2 =
%   9$ orientated outward.
% \end{problem}

% \begin{proof}[Solution]
%   By the Divergence Theorem, we have
%   \[%
%     \oiint_S \FF \cdot \dd{\S} = \iiint_V \div(\FF) \dd{V}
%   ,\]%
%   where $\div(\FF) = \nabla \cdot \FF$ and $V$ is the volume of the sphere $x^2
%   + y^2 + z^2 = 9$. We first compute the divergence of $\FF$
%   \[%
%     \div(\FF) = \nabla \cdot \FF = \pdv{}{x}\left(x^3 + zy^2\right) + \pdv{}{y}\left(4x^2z + 2yz\right) + \pdv{}{z}\left(4 - z^2\right) = 3x^2 + 2z - 2z = 3x^2
%   .\]%
%   Using spherical coordinates, we have
%   \[%
%     x = \rho\sin(\phi)\cos(\theta), \quad y = \rho\sin(\phi)\sin(\theta), \aand z = \rho\cos(\phi)
%   .\]%
%   Therefore, the divergence becomes
%   \[%
%     \div(\FF) = 3x^2 = 3\rho^2\sin^2(\phi)\cos^2(\theta)
%   .\]%
%   The Jacobian determinant for this transformation is $\rho^2\sin(\phi)$. The
%   limits of integration are $0 \le \rho \le 3$, $0 \le \phi \le \pi$, and $0 \le
%   \theta \le 2\pi$. Thus, we have
%   \begin{align*}
%     \oiint_S \FF \cdot \dd{\S} &= \iiint_V \nabla \cdot \FF \dd{V} \\
%                                &= \int_0^{2\pi} \int_0^{\pi} \int_0^3 3\rho^2\sin^2(\phi)\cos^2(\theta) \cdot \rho^2\sin(\phi) \dd{\rho,\phi,\theta} \\
%                                &= 3 \cdot \int_0^{2\pi} \cos^2(\theta) \dd{\theta} \cdot \int_0^{\pi} \sin^3(\phi) \dd{\phi} \cdot \int_0^3 \rho^4 \dd{\rho} \\
%                                &= 3 \cdot \left[\frac{1}{2}(\theta - \sin(\theta)\cos(\theta))\right]_0^{2\pi} \cdot \int_0^\pi (1 - \cos^2(\phi))\sin(\phi) \dd{\phi} \cdot \left[\frac{\rho^5}{5}\right]_0^3 \\
%                                &= 3 \cdot \frac{2\pi}{2} \cdot \left[\frac{1}{12}\left(\cos(3\phi) - 9\cos(\phi)\right)\right]_0^\pi \cdot \frac{243}{5} \\
%                                &= 3\pi \cdot \frac{4}{3} \cdot \frac{243}{5} = \frac{972\pi}{5}
%   .\qedhere\end{align*}
% \end{proof}

\begin{problem}[5]
  Verify the Divergence Theorem for the vector field $\FF = \langle -x, y, z
  \rangle$ and the surface, $S$, is the boundary of the solid enclosed by the
  parabolic cylinder $y = 4 - x^2$ and the planes $y + 2z = 4$ and $z = 2$ with
  positive orientation.

  \noindent(That means, evaluate both $\displaystyle\oiint_S \FF \cdot \dSS$ and
  $\displaystyle\iiint_V \div(\FF) \dV$ showing that they are equal for the
  given field and surface.)
\end{problem}

\begin{proof}[Solution]
  We first compute the divergence of $\FF$
  \[%
    \div(\FF) = \nabla \cdot \FF = \pdv{}{x}(-x) + \pdv{}{y}(y) + \pdv{}{z}(z) = -1 + 1 + 1 = 1
  .\]%
  Therefore, we get
  \[%
    \iiint_V \div(\FF) \dV = \iiint_V 1 \dV = \iiint_V \dV
  .\]%
  The volume limits are $-2 \le x \le 2$, since $y = 4 - x^2$ defines a parabola
  opening down from $y = 4$, $0 \le y \le 4 - x^2$, and $0 \le z \le \frac{4 -
  y}{2}$. Thus, the integral becomes
  \begin{align*}
    \iiint_V \dV &= \int_{-2}^2 \int_0^{4 - x^2} \int_0^{\frac{4 - y}{2}} \dzyx \\
                 &= \int_{-2}^2 \int_0^{4 - x^2} \frac{4 - y}{2} \dyx \\
                 &= \int_{-2}^2 \left[2y - \frac{y^2}{4}\right]_0^{4 - x^2} \dx \\
                 &= \int_{-2}^2 2(4 - x^2) - \frac{(4 - x^2)^2}{4} \dx \\
  .\end{align*}
\end{proof}
