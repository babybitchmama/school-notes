\begin{problem}[1]
  Evaluate $\displaystyle\oint_{\pd{S}} \FF \cdot \drr$ where $\FF =
  \left\langle 2xy, z^3 + 3x^2, 3yz^2 - x \right\rangle$, $\pd{S}$ is the
  boundary of the triangle with vertices $P = (2, 0, 0)$, $Q = (0, 3, 0)$, and
  $R = (0, 0, 5)$ oriented from $P$ to $Q$ to $R$ and back to $P$.
\end{problem}

\begin{proof}[Solution]
  By the orientation induced on the surface, using the right hand rule, the
  orientation of the normal vector is pointing outwards. We can use Stokes'
  Theorem to convert the line integral into a surface integral,
  \[%
    \oint_{\pd{S}} \FF \cdot \drr = \iint_S \curl(\FF) \cdot \dSS
  ,\]%
  where $S$ is the surface bounded by $\pd{S} = C$ and $\curl(\FF) = \nabla
  \times \FF$. We first compute the curl of $\FF$
  \begin{align*}
    \curl(\FF) = \nabla \times \FF &= \begin{vmatrix}
      \ui & \uj & \uk \\
      \pdv{}/{x} & \pdv[style-var-/=multiple]{}/{y} & \pdv[style-var-/=multiple]{}/{z} \\
      2xy & z^3 + 3x^2 & 3yz^2 - x \\
    \end{vmatrix} \\
    &= \left\langle 3z^2 - 3z^2, 1, 6x - 2x \right\rangle \\
    &= \left\langle 0, 1, 4x \right\rangle
  .\end{align*}
  Finding the equation of the plane containing the triangle, we have
  \[%
    15x + 10y + 6z = 30 \implies z = 5 - \frac{5}{2}x - \frac{5}{3}y
  .\]%
  The normal vector to this plane is
  \[%
    \n = \left\langle \frac{5}{2}, \frac{5}{3}, 1 \right\rangle
  .\]%
  Computing the dot product, we have
  \[%
    \curl(\FF) \cdot \n = \left\langle 0, 1, 4x \right\rangle \cdot \left\langle \frac{5}{2}, \frac{5}{3}, 1 \right\rangle = \frac{5}{3} + 4x
  .\]%
  The projection onto the $xy$-plane is the triangle with vertices $P' = (2,
  0)$, $Q' = (0, 3)$, and $R' = (0, 0)$. Thus, the boundaries are $0 \le x \le
  2$ and $0 \le y \le -\frac{3}{2}x + 3$. Therefore, we have
  \begin{align*}
    \oint_{\pd{S}} \FF \cdot \drr &= \iint_S \curl(\FF) \cdot \dSS \\
                           &= \iint_D \left(\frac{5}{3} + 4x\right) \dA \\
                           &= \int_0^2 \int_0^{-\sfrac{3}{2}x+3} \left(\frac{5}{3} + 4x\right) \dyx \\
                           &= \int_0^2 \frac{5}{3}y + 4xy\bigg\vert_0^{-\sfrac{3}{2}x+3} \dx \\
                           &= \int_0^2 \frac{5}{3}\left(-\frac{3}{2}x + 3\right) + 4x\left(-\frac{3}{2}x + 3\right) \dx \\
                           &= \int_0^2 -6x^2 + \frac{19x}{2} + 5 \dx \\
                           &= \left[-2x^3 + \frac{19x^2}{4} + 5x\right]_0^2 \\
                           &= -16 + 19 + 10 = 13
  .\qedhere\end{align*}
\end{proof}

\begin{problem}[2]
  Let $S$ be the portion of the ellipsoid $4x^2 + y^2 + 16z^2 = 64$ above the
  $xy$-plane oriented upward. Use Stokes' Theorem to evaluate
  $\displaystyle\iint_S \curl(\FF) \cdot \dSS$ where $\FF = \left\langle xz, x^3
  + 2y, e^{x^2-y^2} \right\rangle$.
\end{problem}

\begin{proof}[Solution]
  By the orientation induced on the surface, using the right hand rule, the
  orientation of the line integral is counterclockwise. We can use Stokes'
  Theorem to convert the surface integral into a line integral.

  The boundary $\pd{S}$ occurs when $z = 0$, which gives the ellipse $4x^2 +
  y^2 = 64$. This also gives us a simpler form of $\FF$,
  \[%
    \FF = \left\langle 0, x^3 + 2y, e^{x^2-y^2} \right\rangle
  .\]%
  Since
  \[%
    \frac{x^2}{16} + \frac{y^2}{64} = 1
  ,\]%
  we can use the parametrization
  \[%
    \r(t) = \langle 4\cos(t), 8\sin(t), 0 \rangle
  ,\]%
  where $0 \le t \le 2\pi$. Therefore, we have
  \[%
    \drr = \langle \dx, \dy, \dz \rangle = \langle -4\sin(t), 8\cos(t), 0 \rangle \dt
  .\]%
  Computing the dot product, we have
  \begin{align*}
    \FF \cdot \drr &= \left\langle 0, x^3 + 2y, e^{x^2-y^2} \right\rangle \cdot \langle \dx, \dy, 0 \rangle \\
                   &= (x^3 + 2y) \dy \\
                   &= \left((4\cos(t))^3 + 2(8\sin(t))\right) \cdot 8\cos(t) \dt \\
                   &= \left(64\cos^3(t) + 16\sin(t)\right) \cdot 8\cos(t) \dt \\
                   &= 512\cos^4(t) + 128\sin(t)\cos(t) \dt
  .\end{align*}
  By Stokes' Theorem, we have
  \begin{align*}
    \iint_S \curl(\FF) \cdot \dSS &= \oint_{\pd{S}} \FF \cdot \drr \\
                                 &= \int_0^{2\pi} 512\cos^4(t) + 128\sin(t)\cos(t) \dt \\
                                 &= 512\int_0^{2\pi} \cos^4(t) \dt + 128\int_0^{2\pi} \sin(t)\cos(t) \dt \\
                                 &= 512\cdot\frac{3\pi}{4} + 128\cdot 0 = 384\pi
  .\qedhere\end{align*}
\end{proof}

\begin{problem}[3]
  Verify Stokes' Theorem for the vector field $\FF = \langle y, z, x \rangle$
  and the hemisphere $y = \sqrt{1 - x^2 - z^2}$ orientated in the direction of
  the positive $y$-axis.

  \noindent(That means, evaluate both $\displaystyle\oint_{\pd{S}} \FF \cdot
  \dd{\r}$ and $\displaystyle\iint_S \curl(\FF) \cdot \dSS$ showing that they
  are equal for the given field and surface.)
\end{problem}

\begin{proof}[Solution]
  We're first going to compute the line integral over $\pd{S}$. Given the
  orientation of the surface, using the right hand rule, the orientation of the
  line integral is clockwise in the $xz$-plane.

  We're given that the surface is defined as $y = \sqrt{1 - x^2 - y^2}$. This
  means that $x^2 + y^2 \le 1$. The boundary $\pd{S}$ occurs when $y = 0$, which
  gives us the circle $x^2 + z^2 = 1$. Therefore, using polar, we get the
  parameterization
  \[%
    \r(t) = \langle \sin(t), 0, \cos(t) \rangle
  ,\]%
  where $r = 1$ and $0 \le t \le 2\pi$. We need to flip the parameter as we need
  to go clockwise, as polar goes counterclockwise. The differential is
  \[%
    \drr = \r'(t) \dt = \langle \dx, \dy, \dz \rangle \dt = \langle \cos(t), 0, -\sin(t) \rangle \dt
  .\]%
  Evaluating $\FF(\r(t))$, we get
  \[%
    \FF(\r(t)) = \langle 0, \cos(t), \sin(t) \rangle
  .\]%
  Computing the dot product, we have
  \[%
    \FF \cdot \drr = \langle 0, \cos(t), \sin(t) \rangle \cdot \langle \cos(t), 0, -\sin(t) \rangle = -\sin^2(t)
  .\]%
  Therefore, we have
  \begin{align*}
    \oint_{\pd{S}} \FF \cdot \drr &= \int_0^{2\pi} -\sin^2(t) \dt \\
                          &= -\frac{1}{2} \cdot \int_0^{2\pi} 1 - \sin(2t) \dt \\
                          &= -\frac{1}{2} \cdot \left[t + \frac{1}{2}\cos(2t)\right]_0^{2\pi} \\
                          &= -\frac{1}{2} \cdot \left[\left(2\pi + \frac{1}{2}\right) - \left(0 - \frac{1}{2}\right)\right] \\
                          &= -\frac{2\pi}{2} = -\pi
  .\end{align*}

  Now, we compute the double integral over $S$. Computing the curl of $\FF$, we
  have
  \begin{align*}
    \curl(\FF) = \nabla \times \FF &= \begin{vmatrix}
      \ui & \uj & \uk \\
      \pdv{}/{x} & \pdv[style-var-/=multiple]{}/{y} & \pdv[style-var-/=multiple]{}/{z} \\
      y & z & x \\
    \end{vmatrix} \\
    &= \left\langle 0 - 1, -(1 - 0), 0 - 1 \right\rangle \\
    &= \langle -1, -1, -1 \rangle
  .\end{align*}
  Using spherical coordinates, we have
  \[%
    \r(\theta, \phi) = \langle \sin(\phi)\cos(\theta), \cos(\phi), \sin(\phi)\sin(\theta) \rangle
  ,\]%
  where $\rho = 1$. We know that
  \begin{align*}
    \r_\theta &= \langle -\sin(\phi)\sin(\theta), 0, \sin(\phi)\cos(\theta) \rangle \\
    \r_\phi &= \langle \cos(\phi)\cos(\theta), -\sin(\phi), \cos(\phi)\sin(\theta) \rangle
  .\end{align*}
  Since the orientation induced on the surface is in the direction of the
  positive $y$-axis, we have need to have the $\uj$-component of the normal
  positive. Therefore, the normal vector is
  \begin{align*}
    \n &= \r_\theta \times \r_\phi = \begin{vmatrix}
      \ui & \uj & \uk \\
      -\sin(\phi)\sin(\theta) & 0 & \sin(\phi)\cos(\theta) \\
      \cos(\phi)\cos(\theta) & -\sin(\phi) & \cos(\phi)\sin(\theta) \\
    \end{vmatrix} \\
    &= \langle \sin^2(\phi)\cos(\theta), \sin(\phi)\cos(\phi)\cos^2(\theta), \sin^2(\phi)\sin(\theta) \rangle
  .\end{align*}
  The dot product is
  \begin{align*}
    \curl(\FF) \cdot \n &= \langle -1, -1, -1 \rangle \cdot \langle \sin^2(\phi)\cos(\theta), \sin(\phi)\cos(\phi)\cos^2(\theta), \sin^2(\phi)\sin(\theta) \rangle \\
                        &= -\left(\sin^2(\phi)\cos(\theta) + \sin(\phi)\cos(\phi)\cos^2(\theta) + \sin^2(\phi)\sin(\theta)\right)
  .\end{align*}
  Therefore, the surface integral becomes
  \begin{align*}
    \iint_S \curl(\FF) \cdot \dSS &= \iint_D -\left(\sin^2(\phi)\cos(\theta) + \sin(\phi)\cos(\phi)\cos^2(\theta) + \sin^2(\phi)\sin(\theta)\right) \dA \\
                                  &= -\left[\int_0^\pi \int_0^\pi \sin^2(\phi)\cos(\theta) + \sin(\phi)\cos(\phi)\cos^2(\theta) + \sin^2(\phi)\sin(\theta) \dd{\phi,\theta}\right] \\
                                  &= -\int_0^\pi \cos(\theta) \dd{\theta} \cdot \int_0^\pi \sin^2(\phi) \dd{\phi} - \int_0^\pi \cos^2(\theta) \dd{\theta} \cdot \int_0^\pi \sin(\phi)\cos(\phi) \dd{\phi} \\
                                  &\qquad- \int_0^\pi \sin(\theta) \dd{\theta} \cdot \int_0^\pi \sin^2(\phi) \dd{\phi} \\
                                  &= 0 - \frac{\pi}{2} \cdot 0 - 2 \cdot \frac{\pi}{2} = -\pi
  .\end{align*}

  Therefore, we have
  \[%
    \oint_{\pd{S}} \FF \cdot \drr = \iint_S \curl(\FF) \cdot \dSS = -\pi
  .\qedhere\]%
\end{proof}

\begin{problem}[4]
  Evaluate $\displaystyle\oiint_S \FF \cdot \dd{\S}$ where $\FF = \langle x^3 +
  zy^2, 4x^2z + 2yz, 4 - z^2 \rangle$ and $S$ is the sphere $x^2 + y^2 + z^2 =
  9$ orientated outward.
\end{problem}

\begin{proof}[Solution]
  By the Divergence Theorem, we have
  \[%
    \oiint_S \FF \cdot \dd{\S} = \iiint_V \div(\FF) \dd{V}
  ,\]%
  where $\div(\FF) = \nabla \cdot \FF$ and $V$ is the volume of the sphere $x^2
  + y^2 + z^2 = 9$. We first compute the divergence of $\FF$
  \[%
    \div(\FF) = \nabla \cdot \FF = \pdv{}{x}\left(x^3 + zy^2\right) + \pdv{}{y}\left(4x^2z + 2yz\right) + \pdv{}{z}\left(4 - z^2\right) = 3x^2 + 2z - 2z = 3x^2
  .\]%
  Using spherical coordinates, we have
  \[%
    x = \rho\sin(\phi)\cos(\theta), \quad y = \rho\sin(\phi)\sin(\theta), \aand z = \rho\cos(\phi)
  .\]%
  Therefore, the divergence becomes
  \[%
    \div(\FF) = 3x^2 = 3\rho^2\sin^2(\phi)\cos^2(\theta)
  .\]%
  The Jacobian determinant for this transformation is $\rho^2\sin(\phi)$. The
  limits of integration are $0 \le \rho \le 3$, $0 \le \phi \le \pi$, and $0 \le
  \theta \le 2\pi$. Thus, we have
  \begin{align*}
    \oiint_S \FF \cdot \dd{\S} &= \iiint_V \nabla \cdot \FF \dd{V} \\
                               &= \int_0^{2\pi} \int_0^{\pi} \int_0^3 3\rho^2\sin^2(\phi)\cos^2(\theta) \cdot \rho^2\sin(\phi) \dd{\rho,\phi,\theta} \\
                               &= 3 \cdot \int_0^{2\pi} \cos^2(\theta) \dd{\theta} \cdot \int_0^{\pi} \sin^3(\phi) \dd{\phi} \cdot \int_0^3 \rho^4 \dd{\rho} \\
                               &= 3 \cdot \left[\frac{1}{2}(\theta - \sin(\theta)\cos(\theta))\right]_0^{2\pi} \cdot \int_0^\pi (1 - \cos^2(\phi))\sin(\phi) \dd{\phi} \cdot \left[\frac{\rho^5}{5}\right]_0^3 \\
                               &= 3 \cdot \frac{2\pi}{2} \cdot \left[\frac{1}{12}\left(\cos(3\phi) - 9\cos(\phi)\right)\right]_0^\pi \cdot \frac{243}{5} \\
                               &= 3\pi \cdot \frac{4}{3} \cdot \frac{243}{5} = \frac{972\pi}{5}
  .\qedhere\end{align*}
\end{proof}

\begin{problem}[5]
  Verify the Divergence Theorem for the vector field $\FF = \langle -x, y, z
  \rangle$ and the surface, $S$, is the boundary of the solid enclosed by the
  parabolic cylinder $y = 4 - x^2$ and the planes $y + 2z = 4$ and $z = 2$ with
  positive orientation.

  \noindent(That means, evaluate both $\displaystyle\oiint_S \FF \cdot \dSS$ and
  $\displaystyle\iiint_V \div(\FF) \dV$ showing that they are equal for the
  given field and surface.)
\end{problem}

\begin{proof}[Solution]
  First, we compute the double closed surface integral over $S$.
  Notice that
  \[%
    \oiint_S \FF \cdot \dSS = \iint_{S_1} \FF \cdot \dSS + \iint_{S_2} \FF \cdot \dSS + \iint_{S_3} \FF \cdot \dSS
  ,\]%
  where $S_1$ is the parabolic cylinder, $S_2$ is the plane $y + 2z = 4$, and
  $S_3$ is the plane $z = 2$.

  $S_1$: Parametrizing the surface, we get $\r(x, z) = \langle x, 4 - x^2, z
  \rangle$. We know that
  \begin{align*}
    \r_z &= \langle 0, 0, 1 \rangle \\
    \r_x &= \langle 1, -2x, 0 \rangle
  .\end{align*}
  Therefore, the normal vector is
  \[%
    \n = \r_z \times \r_x = \begin{vmatrix}
      \ui & \uj & \uk \\
      0 & 0 & 1 \\
      1 & -2x & 0 \\
    \end{vmatrix} = \langle 2x, 1, 0 \rangle
  .\]%
  Evaluating $\FF(\r(x, z))$, we get
  \[%
    \FF(\r(x, z)) = \langle -x, 4 - x^2, z \rangle
  .\]%
  Computing the dot product, we have
  \[%
    \FF \cdot \n = \langle -x, 4 - x^2, z \rangle \cdot \langle 2x, 1, 0 \rangle = -2x^2 + 4 - x^2 = 4 - 3x^2
  .\]%
  The intersection of the parabolic cylinder and the plane $y + 2z = 4$ is
  \[%
    4 - 2z = 4 - x^2 \implies z = \frac{x^2}{2}
  .\]%
  Projecting the surface onto the $xz$-plane, we get the bounds $-2 \le x \le 2$
  and $\sfrac{x^2}{2} \le z \le 2$. Thus, we have
  \begin{align*}
    \iint_{S_1} \FF \cdot \dSS &= \iint_{D_1} 4 - 3x^2 \dA \\
                               &= \int_{-2}^2 \int_{\sfrac{x^2}{2}}^2 4 - 3x^2 \dzx \\
                               &= \int_{-2}^2 4z - x^2z\bigg\vert_{\sfrac{x^2}{2}}^2 \dx \\
                               &= \int_{-2}^2 \frac{3x^4}{2} - 8x^2 + 8 \dx \\
                               &= \frac{3x^5}{10} - \frac{8x^3}{3} + 8x\bigg\vert_{-2}^2 \\
                               &= \frac{128}{15}
  .\end{align*}

  $S_2$: Parametrizing the surface, we get $\r(x, y) = \left\langle x, y, 2 -
  \frac{4 - y}{2} \right\rangle$. We know that
  \begin{align*}
    \r_y &= \left\langle 0, 1, -\frac{1}{2} \right\rangle \\
    \r_x &= \langle 1, 0, 0 \rangle
  .\end{align*}
  Therefore, the normal vector is
  \[%
    \n = \r_y \times \r_x = \begin{vmatrix}
      \ui & \uj & \uk \\
      0 & 1 & -\frac{1}{2} \\
      1 & 0 & 0 \\
    \end{vmatrix} = \left\langle 0, -\frac{1}{2}, -1 \right\rangle
  .\]%
  But we need the normal vector to be pointing outwards. Therefore, we have
  \[%
    \n = \left\langle 0, \frac{1}{2}, 1 \right\rangle
  .\]%
  Evaluating $\FF(\r(x, y))$, we get
  \[%
    \FF(\r(x, y)) = \left\langle -x, y, \frac{4 - y}{2} \right\rangle
  .\]%
  Computing the dot product, we have
  \[%
    \FF \cdot \n = \left\langle -x, y, \frac{4 - y}{2} \right\rangle \cdot \left\langle 0, \frac{1}{2}, 1 \right\rangle = 2
  .\]%
  The bounds for this surface are $-2 \le x \le 2$ and $0 \le y \le 4 - x^2$.
  Thus, we have
  \begin{align*}
    \iint_{S_2} \FF \cdot \dSS &= \iint_{D_2} 2 \dA \\
                              &= 2 \cdot \int_{-2}^2 \int_0^{4-x^2} \dyx \\
                              &= 2 \cdot \int_{-2}^2 4 - x^2 \dx \\
                              &= 2 \cdot \left[4x - \frac{x^3}{3}\right]_{-2}^2 \\
                              &= 2 \cdot \left[\left(8 - \frac{8}{3}\right) - \left(-8 + \frac{8}{3}\right)\right] \\
                              &= \frac{64}{3}
  .\end{align*}

  $S_3$: Parametrizing the surface, we get $\r(x, z) = \langle x, y, 2 \rangle$.
  We know that
  \begin{align*}
    \r_x &= \langle 1, 0, 0 \rangle \\
    \r_y &= \langle 0, 1, 0 \rangle
  .\end{align*}
  Therefore, the normal vector is
  \[%
    \n = \r_x \times \r_y = \begin{vmatrix}
      \ui & \uj & \uk \\
      1 & 0 & 0 \\
      0 & 1 & 0 \\
    \end{vmatrix} = \langle 0, 0, 1 \rangle
  .\]%
  But we need the normal vector to be pointing outwards. Therefore, we have
  \[%
    \n = \langle 0, 0, -1 \rangle
  .\]%
  Evaluating $\FF(\r(x, y))$, we get
  \[%
    \FF(\r(x, y)) = \langle -x, y, 2 \rangle
  .\]%
  Computing the dot product, we have
  \[%
    \FF \cdot \n = \langle -x, y, 2 \rangle \cdot \langle 0, 0, 1 \rangle = -2
  .\]%
  The bounds for this surface are $-2 \le x \le 2$ and $0 \le y \le 4 - x^2$.
  Thus, we have
  \begin{align*}
    \iint_{S_3} \FF \cdot \dSS &= \iint_{D_3} -2 \dA \\
                              &= \int_{-2}^2 \int_0^{4-x^2} -2 \dyx \\
                              &= -2 \cdot \int_{-2}^2 4 - x^2 \dx \\
                              &= -2 \cdot \left[4x - \frac{x^3}{3}\right]_{-2}^2 \\
                              &= -2 \cdot \left[\left(8 - \frac{8}{3}\right) - \left(-8 + \frac{8}{3}\right)\right] \\
                              &= -\frac{64}{3}
  .\end{align*}

  Therefore, we have
  \[%
    \oiint_S \FF \cdot \dSS = \frac{128}{15} + \frac{64}{3} - \frac{64}{3} = \frac{128}{15}
  .\]%

  Now, we compute the triple integral over the volume. We first compute the
  divergence of $\FF$
  \[%
    \div(\FF) = \nabla \cdot \FF = \pdv{}{x}(-x) + \pdv{}{y}(y) + \pdv{}{z}(z) = -1 + 1 + 1 = 1
  .\]%
  Therefore, we get
  \[%
    \iiint_V \div(\FF) \dV = \iiint_V 1 \dV = \iiint_V \dV
  .\]%
  The volume limits are $-2 \le x \le 2$, since $y = 4 - x^2$ defines a parabola
  opening down from $y = 4$, $0 \le y \le 4 - x^2$, and $2 - \sfrac{y}{2} \le z
  \le 2$. Thus, the integral becomes
  \begin{align*}
    \iiint_V \dV &= \int_{-2}^2 \int_0^{4-x^2} \int_{2-\sfrac{y}{2}}^2 \dz \dyx \\
                 &= \int_{-2}^2 \int_0^{4-x^2} \frac{y}{2} \dyx \\
                 &= \int_{-2}^2 \frac{y^2}{4}\bigg\vert_0^{4-x^2} \dx \\
                 &= \int_{-2}^2 \frac{(4 - x^2)^2}{4} \dx \\
                 &= \int_{-2}^2 \frac{x^4}{4} - 2x^2 + 4 \dx \\
                 &= \frac{x^5}{20} - \frac{2x^3}{3} + 4x\bigg\vert_{-2}^2 \\
                 &= \frac{128}{15}
  .\end{align*}

  Therefore, we get
  \[%
    \oiint_S \FF \cdot \dSS = \iiint_V \div(\FF) \dV
  .\qedhere\]%
\end{proof}
