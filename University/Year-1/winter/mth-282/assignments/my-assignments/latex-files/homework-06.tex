\begin{problem}[1]
  Evaluate $\displaystyle \int_C \FF \cdot \dd{\r}$ for the vector field $\FF = (x + z)\ui + 2y\uj + (y + 2x)\uk$ where $C$ is
  \begin{enumerate}
    \item the line segment from $(-1, 0, 1)$ and $(0, -1, 2)$.

    \item the curve $\r(t) = \left\langle t - 1, t^2 - 2t, t^3 + 1
      \right\rangle$ from $0 \le t \le 1$.
  \end{enumerate}
\end{problem}

\begin{proof}[Solution to (i)]
  Let $C$ be the line segment from $P(-1, 0, 1)$ to $Q(0, -1, 2)$. Then, we have
  \begin{align*}
    \phantom{\implies}&~\r(t) = (1 - t)P + tQ = (1 - t)(-1, 0, 1) + t(0, -1, 2) = \langle -1 + t, -t, 1 + t \rangle \\
    \implies&~\r'(t) = \langle 1, -1, 1 \rangle
  .\end{align*}
  Converting $\FF$ from a vector function of $x$ and $y$ to a vector function of
  $t$, we have
  \[%
    \FF(x, y) = \langle x + z, 2y, y + 2x \rangle = \langle (-1 + t) + (-1 + t), -2t, -t + (-1 + t) \rangle = \langle 2t, -2t, -2 + t \rangle = \FF(t)
  .\]%
  Therefore, the line integral over the vector field $\FF$ along the line
  segment $C$ is
  \begin{align*}
    \int_C \FF \cdot \dd{\r} &= \int_0^1 \FF(t) \cdot \r'(t) \,\dt \\
                             &= \int_0^1 (1) \cdot (2t) + (-1) \cdot (-2t) + (1) \cdot (-2 + t) \,\dt \\
                             &= \int_0^1 2t + 2t -2 + t \,\dt \\
                             &= \int_0^1 5t - 2 \,\dt = \left(\frac{5}{2} - 2\right) = \frac{1}{2}
  .\qedhere\end{align*}
\end{proof}

\begin{proof}[Solution to (ii)]
  Let $C$ be the curve $\r(t) = \langle t - 1, t^2 - 2t, t^3 + 1 \rangle$ where
  $0 \le t \le 1$. Then, we have
  \[%
    \r'(t) = \left\langle 1, 2t - 2, 3t^2 \right\rangle
  .\]%
  Converting $\FF$ from a vector function of $x$ and $y$ to a vector function of
  $t$, we have
  \begin{align*}
    \FF(x, y) = \left\langle x + z, 2y, y + 2x \right\rangle &= \left\langle (t - 1) + \left(t^3 + 1\right), 2\left(t^2 - 2t\right), \left(t^2 - 2t\right) + 2(t - 1) \right\rangle \\
                                                  &= \left\langle t^3 + t, 2t^2 - 4t, t^2 - 2 \right\rangle
  .\end{align*}
  Evaluating the dot product of $\FF$ and $\r'$, we have
  \begin{align*}
    \FF(t) \cdot \r'(t) &= \left\langle t^3 + t, 2t^2 - 4t, t^2 - 2 \right\rangle \cdot \left\langle 1, 2t - 2, 3t^2 \right\rangle \\
                        &= \left(t^3 + t\right) \cdot (1) + \left(2t^2 - 4t\right) \cdot (2t - 2) + \left(t^2 - 2\right) \cdot \left(3t^2\right) \\
                        &= t^3 + t + 4t^3 - 4t^2 - 8t^2 + 8t + 3t^4 - 6t^2 \\
                        &= 3t^4 + 5t^3 - 18t^2 + 9t
  .\end{align*}
  Therefore, the line integral over the vector field $\FF$ along the curve $C$
  is
  \begin{align*}
    \int_C \FF \cdot \dd{\r} &= \int_0^1 \FF(t) \cdot \r'(t) \,\dt \\
                             &= \int_0^1 3t^4 + 5t^3 - 18t^2 + 9t \,\dt \\
                             &= \frac{3}{5} + \frac{5}{4} - \frac{18}{3} + \frac{9}{2} = \frac{7}{20}
  .\qedhere\end{align*}
\end{proof}

\begin{problem}[2]
  Compute the amount of work done by the vector field $\FF = \langle -y, x, x -
  z \rangle$ moving a particle along the curve of intersection of the surfaces
  $x^2 + y^2 + z^2 = 9$ and $y - x + z = 1$ oriented in the counterclockwise
  direction about the cylinder.

  Note: Do not use a method we haven’t discussed yet. The use of anything other
  than directly evaluating the line integral will not be accepted.
\end{problem}

\begin{proof}[Solution]
  Let $C$ be the curve of intersection of the surfaces $x^2 + y^2 + z^2 = 9$ and
  $y - x + z = 1$. Converting to cylindrical coordinates, we have
  \begin{align*}
    \begin{cases}
      x^2 + y^2 + z^2 = 9 \\
      y - x + z = 1
    \end{cases}
    &\implies
    \begin{cases}
      r^2 + z^2 = 9 \\
      r\sin\theta - r\cos\theta + z = 1
    \end{cases}
    \implies
    \begin{cases}
      r^2 + z^2 = 9 \\
      r(\sin\theta - \cos\theta) + z = 1
    \end{cases} \\
    &\implies z = 1 - r(\sin(\theta) - \cos(\theta))
  .\end{align*}
  To parameterize $C$, notice that $x^2 + y^2 + z^2 = 9$ is just a sphere.
  Therefore, we have $r = 3$, $x = 3\cos(\theta)$, and $y = 3\sin(\theta)$.
  Using $r = 3$, we have the following parameterization for $C$
  \[%
    \r(t) = \left\langle 3\cos(\theta), 3\sin(\theta), 1 - 3(\sin(\theta) - \cos(\theta)) \right\rangle
  .\]%
  Evaluating $\r'(t)$ gives us
  \[%
    \r'(t) = \langle -3\sin(\theta), 3\cos(\theta), 3(\cos(\theta) + \sin(\theta)) \rangle
  .\]%
  The bounds for $\theta$ is $0 \le \theta \le 2\pi$. Converting $\FF$ from a
  vector function of $x$ and $y$ to a vector function of $t$, we have
  \begin{align*}
    \FF(t) = \langle -y(t), x(t), x(t) - z(t) \rangle &= \langle -3\sin(\theta), 3\cos(\theta), 3\cos(\theta) - \left(1 - 3\left(\sin(\theta) - \cos(\theta)\right)\right) \\
                                                      &= \langle -3\sin(\theta), 3\cos(\theta), 3\sin(\theta) - 1 \rangle
  .\end{align*}
  Evaluating the dot product of $\FF$ and $\r'$, we have
  \begin{align*}
    \FF(t) \cdot \r'(t) &= \left\langle -3\sin(\theta), 3\cos(\theta), 3\sin(\theta) - 1 \right\rangle \cdot \left\langle -3\sin(\theta), 3\cos(\theta), 3(\cos(\theta) + \sin(\theta)) \right\rangle \\
                        &= \left(-3\sin(\theta)\right) \cdot \left(-3\sin(\theta)\right) + \left(3\cos(\theta)\right) \cdot \left(3\cos(\theta)\right) + \left(3\sin(\theta) - 1\right) \cdot \left(3(\cos(\theta) + \sin(\theta))\right) \\
                        &= 9(\sin^2(\theta) + \cos^2(\theta)) + (-9\sin^2(\theta) - 9\sin(\theta)\cos(\theta) + 3\cos(\theta) + 3\sin(\theta)) \\
                        &= 9 - 9\sin^2(\theta) - 9\sin(\theta)\cos(\theta) + 3\cos(\theta) + 3\sin(\theta)
  .\end{align*}
  Computing the work integral over the vector field $\FF$ along the curve $C$,
  we have
  \begin{align*}
    W = \int_C \FF \cdot \dd{\r} &= \int_0^{2\pi} \FF(t) \cdot \r'(t) \,\dt \\
                                 &= \int_0^{2\pi} 9 - 9\sin^2(\theta) - 9\sin(\theta)\cos(\theta) + 3\cos(\theta) + 3\sin(\theta) \,\dt \\
                                 &= \int 9 \,\dd{\theta} - 9\int \sin^2(\theta) \,\dd{\theta} - 9\int \sin(\theta)\cos(\theta) \,\dd{\theta} + 3\int \cos(\theta) \,\dd{\theta} + 3\int \sin(\theta) \,\dd{\theta} \bigg\vert_0^{2\pi} \\
                                 &= 9\theta - \frac{9}{2}(\theta - \sin(\theta)\cos(\theta)) - \frac{9}{2}\cos^2(\theta) + 3\sin(\theta) - 3\cos(\theta) \bigg\vert_0^{2\pi} \\
                                 &= \left[9(2\pi) - \frac{9}{2}(2\pi - \sin(2\pi)\cos(2\pi)) - \frac{9}{2}\cos^2(2\pi) + 3\sin(2\pi) - 3\cos(2\pi)\right] \\
                                 &\phantom{=} -\left[9(0) - \frac{9}{2}(0 - \sin(0)\cos(0)) - \frac{9}{2}\cos^2(0) + 3\sin(0) - 3\cos(0)\right] \\
                                 &= \left[18\pi - \frac{9}{2}(2\pi - (0)(-1)) - \frac{9}{2}(1) + 3(0) - 3(1)\right] \\\
                                 &\phantom{=} -\left[0 - \frac{9}{2}(0 - (0)(1)) - \frac{9}{2}(1) + 3(0) - 3(1)\right] \\
                                 &= 18\pi - 9\pi - \frac{9}{2} - 3 + \frac{9}{2} + 3 = 9\pi
  .\qedhere\end{align*}
\end{proof}

\begin{problem}[3]
  Evaluate $\displaystyle\int_C \FF \cdot \dd{\r}$. Use the fundamental theorem
  for line integrals whenever it applies.
  \begin{enumerate}
    \item $\FF = \langle 2xe^{xy} + x^2ye^{xy} + 3x^2, x^3e^{xy} + 2\sin(y)
      \rangle$, $C$ is the line segment from $(-1, 0)$ to $(0, 3)$.

    \item $\FF = \langle y^3 - 2x, 3xy^2 + \sin(\pi y) \rangle$, $C$ is the path
      $y = \sqrt{x}$ from $(1,1 )$ to $(4, 2)$.

    \item $\FF = \langle 6xy - z^2, 3x^2 + 6y^2, 1 - 2xz \rangle$, $C$ is the
      circular helix $\r(t) = \langle t, 2\cos(t), 2\sin(t) \rangle$, $0 \le t
      \le \pi$.

    \item $\FF = \langle y + z, x - 2z, x + 2y \rangle$, $C$ is the intersection
      of sphere $x^2 + y^2 + z^2 = 4$ and the plane $x = 1$ in the first octant
      oriented upward.
  \end{enumerate}
\end{problem}

\begin{proof}[Solution to (i)]
\end{proof}

\begin{proof}[Solution to (ii)]
\end{proof}

\begin{proof}[Solution to (iii)]
\end{proof}

\begin{proof}[Solution to (iv)]
\end{proof}

\begin{problem}[4]
  Consider $\FF = \langle P, Q \rangle$ where $\displaystyle P(x, y) =
  \frac{-y}{x^2 + y^2}$ and $\displaystyle Q(x, y) = \frac{x}{x^2 + y^2}$.
  \begin{enumerate}
    \item Show $\displaystyle\pdv{P}{y} = \pdv{Q}{x}$ on the domain of $\FF$.

    \item Use the definition of the line integral to show that
      $\displaystyle\int_C \FF \cdot \dd{\r} = 2\pi$ where $C$ is the circle
      $x^2 + y^2 = a^2$, counterclockwise orientation, for any constant $a > 0$.
      Is $\FF$ conservative?
  \end{enumerate}
\end{problem}

\begin{proof}[Solution to (i)]
  We have
  \begin{align*}
    \pdv{P}{y} = \pdv{}{y}\left(\frac{-y}{x^2 + y^2}\right) &= \frac{-(x^2 + y^2) - (-y)(2y)}{(x^2 + y^2)^2} = \frac{-x^2 - y^2 + 2y^2}{(x^2 + y^2)^2} = \frac{y^2 - x^2}{(x^2 + y^2)^2} \\
    \pdv{Q}{x} = \pdv{}{x}\left(\frac{x}{x^2 + y^2}\right) &= \frac{(x^2 + y^2) - x(2x)}{(x^2 + y^2)^2} = \frac{x^2 + y^2 - 2x^2}{(x^2 + y^2)^2} = \frac{y^2 - x^2}{(x^2 + y^2)^2}
  .\end{align*}
  Therefore, $\displaystyle\pdv{P}{y} = \pdv{Q}{x}$ on the domain of $\FF$.
\end{proof}

\begin{proof}[Solution to (ii)]
  We parameterize the circle $C$
  \[%
    \r(t) = \langle a\cos(t), a\sin(t) \rangle
  .\]%
  Evaluating $\r'(t)$ gives us
  \[%
    \r'(t) = \langle -a\sin(t), a\cos(t) \rangle
  ,\]%
  where $0 \le t \le 2\pi$. Converting $\FF$ from a vector function of $x$ and
  $y$ to a vector function of $t$, we have
  \[%
    \FF(t) = \left\langle \frac{-a\sin(t)}{a^2}, \frac{a\cos(t)}{a^2} \right\rangle = \left\langle \frac{-\sin(t)}{a}, \frac{\cos(t)}{a} \right\rangle
  .\]%
  Evaluating the dot product of $\FF$ and $\r'$, we have
  \[%
    \FF(t) \cdot \r'(t) = \left\langle \frac{-\sin(t)}{a}, \frac{\cos(t)}{a} \right\rangle \cdot \left\langle -a\sin(t), a\cos(t) \right\rangle = \frac{-\sin(t)}{a} \cdot (-a\sin(t)) + \frac{\cos(t)}{a} \cdot (a\cos(t)) = 1
  .\]%
  Thus, the line integral evaluates to
  \[%
    \int_C \FF \cdot \dd{\r} = \int_0^{2\pi} 1 \,\dt = 2\pi
  .\]%

  A vector field $\FF = \langle P, Q \rangle$ is conservative if and only if
  there exists a function $f(x, y)$ such that $\nabla f = \langle P, Q \rangle$.
  A necessary condition for $\FF$ to be conservative is that
  $\displaystyle\pdv{P}{y} = \pdv{Q}{x}$, which we verified in part (i).

  However, $\FF$ is not defined at $(0, 0)$, and the domain $\R^2 - \{(0, 0)\}$
  is not simply connected, since any loop enclosing the origin cannot be
  continuously shrunk to a point without leaving the domain.  A conservative
  vector field must be path-independent, meaning that the line integral around
  any closed curve should be zero.

  Since we computed
  \[%
    \oint_C \FF \cdot \dd{\r} = 2\pi \neq 0
  ,\]%
  we conclude that $\FF$ is not conservative.
\end{proof}
