\begin{problem}[1]
  Evaluate the line integral $\displaystyle\oint_C \FF \cdot \dd{\r}$
  \begin{enumerate}
    \item $\FF = \langle xy^2 + 4xy, 2y + 2x^2 \rangle$ and $C$ is the path $y =
      x^2$ from $(-2, 4)$ to $(1, 1)$ and the line segment from $(1, 1)$ to
      $(-2, 4)$.

    \item $\FF = \langle y\sin(x) - y^4, y^2 - \cos(x) \rangle$ and $C$ is the
      union of the half circle $y = \sqrt{4 - x^2}$ from $(2, 0)$ to $(-2, 0)$
      and the line segment from $(-2, 0)$ to $(2, 0)$.
  \end{enumerate}
\end{problem}

\begin{proof}[Solution to (i)]
  The region $D$ is the region enclosed by the path $y = x^2$ from $(-2, 4)$ to
  $(1, 1)$ and the line segment from $(1, 1)$ to $(-2, 4)$. The region $D$ is
  shown in Figure~\ref{fig:hw_07_1_i}.
  \begin{figure}[H]
    \centering
    \begin{tikzpicture}
      \begin{axis}[
        axis lines = center,
        xlabel = $x$,
        ylabel = $y$,
        xmin=-2.5, xmax=1.5,
        ymin=-0.5, ymax=4.5,
        xtick = {-2, 1},
        ytick = {1, 4},
        xticklabels = {$-2$, $1$},
        yticklabels = {$1$, $4$},
        legend pos = outer north east,
      ]
        \addplot[domain=-2:1, samples=1000] {x^2};
        \addplot[domain=1:-2, samples=1000] {x^2};
        \addplot[mark=*] coordinates {(-2, 4) (1, 1)};
        \addplot[mark=*] coordinates {(1, 1) (-2, 4)};
      \end{axis}
    \end{tikzpicture}
    \caption{Region $D$ for Problem~1(i)}
    \label{fig:hw_07_1_i}
  \end{figure}
  Since $D$ is positively oriented, piecewise-smooth, and simply connected, we
  can use Green's Theorem to evaluate the line integral, which states that
  \[%
    \oint_C \FF \cdot \dd{\r} = \iint_D \left(\pdv{Q}{x} - \pdv{P}{y}\right) \,\dA
  ,\]%
  where $P = xy^2 + 4xy$ and $Q = 2y + x^2$. We have
  \[%
    \pdv{Q}{x} - \pdv{P}{y} = \left(4x\right) - \left(2xy + 4x\right) = -2xy
  .\]%

  Finding the equation of the line segment from $(1, 1)$ to $(-2, 4)$, we have
  $y = -x + 2$. Therefore, we get the bounds for $x$ as $-2 \leq x \leq 1$ and
  $x^2 \leq y \leq -x + 2$. Therefore, we have
  \begin{align*}
    \oint_C \FF \cdot \dd{\r} &= \iint_D \left(\pdv{Q}{x} - \pdv{P}{y}\right) \,\dA \\
                              &= \int_{-2}^1 \int_{x^2}^{-x+2} -2xy \,\dy \,\dx \\
                              &= \int_{-2}^1 -xy^2\bigg\vert_{x^2}^{-x+2} \,\dx \\
                              &= \int_{-2}^1 \left[-x(-x + 2)^2\right] - \left[-x(x^2)^2\right] \,\dx \\
                              &= \int_{-2}^1 x^5 - x^3 + 4x^2 - 4x \,\dx \\
                              &= \frac{x^6}{6} - \frac{x^4}{4} + \frac{4x^3}{3} - 2x^2\bigg\vert_{-2}^1 \\
                              &= \left[\frac{1}{6} - \frac{1}{4} + \frac{4}{3} - 2\right] - \left[\frac{64}{6} - 4 - \frac{32}{3} - 8\right] = \frac{45}{4}
  .\qedhere\end{align*}
\end{proof}

\begin{proof}[Solution to (ii)]
  The region $D$ is the region enclosed by the half circle $y = \sqrt{4 - x^2}$
  from $(2, 0)$ to $(-2, 0)$ and the line segment from $(-2, 0)$ to $(2, 0)$.
  The region $D$ is shown in Figure~\ref{fig:hw_07_1_ii}.
  \begin{figure}[H]
    \centering
    \begin{tikzpicture}
      \begin{axis}[
        axis lines = center,
        xlabel = $x$,
        ylabel = $y$,
        xmin=-2.5, xmax=2.5,
        ymin=-0.5, ymax=2.5,
        xtick = {-2, 2},
        ytick = {2},
        xticklabels = {$-2$, $2$},
        yticklabels = {$2$},
        legend pos = outer north east,
      ]
        \addplot[domain=-2:2, samples=1000] {sqrt(4 - x^2)};
        \addplot[domain=2:-2, samples=1000] {sqrt(4 - x^2)};
        \addplot[mark=*] coordinates {(2, 0) (-2, 0)};
      \end{axis}
    \end{tikzpicture}
    \caption{Region $D$ for Problem~1(ii)}
    \label{fig:hw_07_1_ii}
  \end{figure}
  Since $D$ is positively oriented, piecewise-smooth, and simply connected, we
  can use Green's Theorem to evaluate the line integral, which states that
  \[%
    \oint_C \FF \cdot \dd{\r} = \iint_D \left(\pdv{Q}{x} - \pdv{P}{y}\right) \,\dA
  ,\]%
  where $P = y\sin(x) - y^4$ and $Q = y^2 - \cos(x)$. We have
  \[%
    \pdv{Q}{x} - \pdv{P}{y} = \left(-\sin(x)\right) - \left(\sin(x) - 4y^3\right) = 4y^3
  .\]%

  Converting to polar coordinates, we get the bounds $0 \le r \le 2$ and $0 \le
  \theta \le \pi$.
  \begin{align*}
    \oint_C \FF \cdot \dd{\r} &= \iint_D \left(\pdv{Q}{x} - \pdv{P}{y}\right) \,\dA \\
                              &= \iint_D 4y^3 \,\dA \\
                              &= \int_0^\pi \int_0^2 4(r\sin(\theta))^3 \cdot r \,\dr \,\dd{\theta} \\
                              &= 4\int_0^\pi \sin^3(\theta) \,\dd{\theta} \cdot \int_0^2 r^4 \,\dr \\
                              &= \left(4 \cdot \frac{32}{5}\right) \cdot \int_0^\pi \sin(\theta)(1 - \cos^2(\theta)) \,\dd{\theta}
  .\end{align*}
  Using the substitution $u = \cos(\theta)$, we get $\du = -\sin(\theta) \,\dt$.
  The bounds for $u$ are $u(0) = 1$ and $u(\pi) = -1$. Therefore, we have
  \begin{align*}
    \oint_C \FF \cdot \dd{\r} &= \left(4 \cdot \frac{32}{5}\right) \cdot \int_0^\pi \sin(\theta)(1 - \cos^2(\theta)) \,\dd{\theta} \\
                              &= \frac{128}{5} \cdot \int_1^{-1} - 1 + u^2 \,\du \\
                              &= \frac{128}{5} \cdot \int_{-1}^1 1 - u^2 \,\du \\
                              &= \frac{128}{5} \cdot \left(u - \frac{u^3}{3} \bigg\vert_{-1}^1\right) \\
                              &= \frac{128}{5} \cdot \left(\left[1 - \frac{1}{3}\right] - \left[-1 + \frac{1}{3}\right]\right) \\
                              &= \frac{128}{5} \cdot \left(2 - \frac{2}{3}\right) = \frac{512}{15}
  .\qedhere\end{align*}
\end{proof}

\begin{problem}[2]
  If a closed and bounded region, $D$, has a constant density, $\rho$, then the
  center of mass is called the centroid.
  \begin{enumerate}
    \item Use Green’s Theorem to show that the centroid of $D$ has coordinates
      \[%
        \bar{x} = \frac{1}{2A}\oint_C x^2 \,\dy \aand \bar{y} = -\frac{1}{2A}\oint_C y^2 \,\dx
      ,\]%
      where $A$ is the area of $D$ and $C$ is the closed boundary of $D$ with
      positive orientation.

    \item Use these line integrals to find the centroid of the quarter circular
      region $D = \{(x, y) \mid x^2 + y^2 \le a^2, x \ge 0, y \ge 0\}$.
  \end{enumerate}
\end{problem}

\begin{proof}[Solution to (i)]
  The centroid $(\bar{x}, \bar{y})$ of a region $D$ with uniform density is
  given by by
  \[%
    \bar{x} = \frac{1}{A} \iint_D x \,\dA \aand \bar{y} = \frac{1}{A} \iint_D y \,\dA
  .\]%
  Converting the double integral for $\bar{x}$ to a line integral, we set $P =
  0$ and $Q = \sfrac{x^2}{2}$. Therefore, we have
  \[%
    \pdv{Q}{x} - \pdv{P}{y} = x
  .\]%
  Therefore,
  \[%
    \bar{x} = \frac{1}{A} \iint_D x \,\dA = \frac{1}{2A} \oint_C x^2 \,\dy
  .\]%

  Similarly, converting the double integral for $\bar{y}$ to a line integral, we
  set $P = \sfrac{y^2}{2}$ and $Q = 0$. Therefore, we have
  \[%
    \pdv{Q}{x} - \pdv{P}{y} = -y
  .\]%
  Therefore, we have
  \[%
    \bar{y} = \frac{1}{A} \iint_D y \,\dA = -\frac{1}{2A} \oint_C y^2 \,\dx
  .\]%

  Therefore, the centroid of $D$ has coordinates
  \[%
    \bar{x} = \frac{1}{A} \iint_D x \,\dA = \frac{1}{2A}\oint_C x^2 \,\dy \aand \bar{y} = \frac{1}{A} \iint_D y \,\dA = -\frac{1}{2A}\oint_C y^2 \,\dx
  .\qedhere\]%
\end{proof}

\begin{proof}[Solution to (ii)]
  The region $D$ is the quarter circular region $D = \{(x, y) \mid x^2 + y^2
  \le a^2, x \ge 0, y \ge 0\}$. The region $D$ is shown in
  Figure~\ref{fig:hw_07_2_ii}.
  \begin{figure}[H]
    \centering
    \begin{tikzpicture}
      \begin{axis}[
        axis lines = center,
        xlabel = $x$,
        ylabel = $y$,
        xmin=-0.5, xmax=2.5,
        ymin=-0.5, ymax=2.5,
        xtick = {2},
        ytick = {2},
        xticklabels = {$a$},
        yticklabels = {$a$},
        legend pos = outer north east,
      ]
        \addplot[domain=0:2, samples=1000] {sqrt(4 - x^2)};
      \end{axis}
    \end{tikzpicture}
    \caption{Region $D$ for Problem~2(ii)}
    \label{fig:hw_07_2_ii}
  \end{figure}

  The area of $D$ is $A = \sfrac{\pi a^2}{4}$.

  We break the boundary of $D$ into three parts: $C_1$ is the circular arc $r =
  a$, $C_2$ is the line segment $y = 0$, and $C_3$ is the line segment $x = 0$.
  Note that the paths $C_2$ and $C_3$ don't contribute to the line integrals for
  the centroid. We know that $x = a\cos(\theta)$ and $y = a\sin(\theta)$, giving
  us the differential $\dy = a\cos(\theta) \,\dd{\theta}$. The bounds are
  clearly $0 \le \theta \le \sfrac{\pi}{2}$. Substituting these into the line
  integrals for the centroid, we get
  \begin{align*}
    \oint_{C_1} x^2 \,\dy &= \int_0^{\sfrac{\pi}{2}} (a\cos(\theta))^2 \cdot a\cos(\theta) \,\dd{\theta} \\
                          &= a^3 \int_0^{\sfrac{\pi}{2}} \cos^3(\theta) \,\dd{\theta} \\
                          &= a^3 \int_0^{\sfrac{\pi}{2}} \cos(\theta)(1 - \sin^2(\theta)) \,\dd{\theta}
  .\end{align*}
  Using the substitution $u = \sin(\theta)$, we get $\du = \cos(\theta) \,\dt$.
  The bounds for $u$ are $u(0) = 0$ and $u\left(\sfrac{\pi}{2}\right) = 1$.
  Therefore, we have
  \begin{align*}
    \oint_{C_1} x^2 \,\dy &= a^3 \int_0^{\sfrac{\pi}{2}} \cos(\theta)(1 - \sin^2(\theta)) \,\dd{\theta} \\
                          &= a^3 \int_0^1 (1 - u^2) \,\du \\
                          &= a^3 \left(u - \frac{u^3}{3} \bigg\vert_0^1\right) \\
                          &= a^3 \left(1 - \frac{1}{3}\right) = \frac{2a^3}{3}
  .\end{align*}
  Therefore, the center for the $x$-coordinate is
  \[%
    \bar{x} = \frac{1}{2A} \oint_C x^2 \,\dy = \frac{1}{2 \cdot \sfrac{\pi a^2}{4}} \cdot \frac{2a^3}{3} = \frac{4a}{3\pi}
  .\]%
  Similarly, we have
  \begin{align*}
    \oint_{C_1} y^2 \,\dx &= \int_0^{\sfrac{\pi}{2}} (a\sin(\theta))^2 \cdot -a\sin(\theta) \,\dd{\theta} \\
                          &= -a^3 \int_0^{\sfrac{\pi}{2}} \sin^3(\theta) \,\dd{\theta} \\
                          &= -a^3 \int_0^{\sfrac{\pi}{2}} \sin(\theta)(1 - \cos^2(\theta)) \,\dd{\theta}
  .\end{align*}
  Using the substitution $u = \cos(\theta)$, we get $\du = -\sin(\theta) \,\dt$.
  The bounds for $u$ are $u(0) = 1$ and $u\left(\sfrac{\pi}{2}\right) = 0$.
  Therefore, we have
  \begin{align*}
    \oint_{C_1} y^2 \,\dx &= -a^3 \int_0^{\sfrac{\pi}{2}} \sin(\theta)(1 - \cos^2(\theta)) \,\dd{\theta} \\
                          &= -a^3 \int_1^0 1 - u^2 \,\du \\
                          &= -a^3 \left(u - \frac{u^3}{3} \bigg\vert_1^0\right) \\
                          &= -a^3 \left(0 - \frac{0^3}{3} - 1 + \frac{1^3}{3}\right) \\
                          &= -a^3 \left(-\frac{1}{3}\right) = \frac{a^3}{3}
  .\end{align*}
  Therefore, the center for the $y$-coordinate is
  \[%
    \bar{y} = -\frac{1}{2A} \oint_C y^2 \,\dx = -\frac{1}{2 \cdot \sfrac{\pi a^2}{4}} \cdot \left(-\frac{a^3}{3}\right) = \frac{4a}{3\pi}
  .\]%
  Therefore, the centroid of the quarter circular region $D$ is
  \[%
    \left(\frac{4a}{3\pi}, \frac{4a}{3\pi}\right)
  .\qedhere\]%
\end{proof}

\begin{problem}[3]
  Find parametric equations and the parameter domain that define the following
  surfaces. Then use the parametric equations to find the surface area of the
  surfaces.

  Note: All steps used for computing the normal vector must be included.
  \begin{enumerate}
    \item The portion of the sphere $x^2 + y^2 + z^2 = 4$ between the planes $z
      = -1$ and $z = \sqrt{3}$.

    \item The portion of the paraboloid $x = y^2 + z^2$ that lies inside the cylinder $y^2 + z^2 = 16$.

    \item The portion of the cylinder $x^2 + z^2 = 4$ that is above $z = 0$ and
      inside the cylinder $x^2 + y^2 = 4$.
  \end{enumerate}
\end{problem}

\begin{proof}[Solution to (i)]
  The equation has a radius of $r = 2$ centered at the origin and the given
  constraints are $z = -1$ and $z = \sqrt{3}$, which define a spherical cap.
  Using spherical coordinates, we have
  \[%
    x = 2 \sin(\theta)\cos(\phi), \quad y = 2 \sin(\theta)\sin(\phi), \aand z = 2 \cos(\theta)
  .\]%
  From $z = 2 \cos(\theta)$, we set bounds for $\theta$ using
  \begin{alignat*}{5}
    -1 &= 2 \cos(\theta) &&\implies \cos(\theta) &&= -\frac{1}{2} &&\implies \theta &&= \frac{2\pi}{3} \\
    \sqrt{3} &= 2 \cos(\theta) &&\implies \cos(\theta) &&= \frac{\sqrt{3}}{2} &&\implies \theta &&= \frac{\pi}{6}
  .\end{alignat*}
  Thus, we get the bounds $\sfrac{\pi}{6} \le \theta \le \sfrac{2\pi}{3}$ and $0
  \le \phi \le 2\pi$.

  We know that
  \begin{align*}
    \r_\theta &= \left\langle \pdv{x}{\theta}, \pdv{y}{\theta}, \pdv{z}{\theta} \right\rangle = \langle 2\cos(\theta)\cos(\phi), 2\cos(\theta)\sin(\phi), -2\sin(\theta) \rangle \\
    \r_\phi &= \left\langle \pdv{x}{\phi}, \pdv{y}{\phi}, \pdv{z}{\phi} \right\rangle = \langle -2\sin(\theta)\sin(\phi), 2\sin(\theta)\cos(\phi), 0 \rangle
  .\end{align*}
  The normal vector is
  \begin{align*}
    \mathbf{N} &= \r_\theta \times \r_\phi = \begin{vmatrix}
      \mathbf{i} & \mathbf{j} & \mathbf{k} \\
      2\cos(\theta)\cos(\phi) & 2\cos(\theta)\sin(\phi) & -2\sin(\theta) \\
      -2\sin(\theta)\sin(\phi) & 2\sin(\theta)\cos(\phi) & 0
    \end{vmatrix} \\
               &= \langle (2\cos(\theta)\sin(\phi))(0) - (-2\sin(\theta))(2\sin(\theta)\cos(\phi)), \\
               &\quad-\left[(2\cos(\theta)\cos(\phi))(0) - (-2\sin(\theta))(-2\sin(\theta)\sin(\phi))\right], \\
               &\quad(2\cos(\theta)\cos(\phi))(2\sin(\theta)\cos(\phi)) - (2\cos(\theta)\sin(\phi))(-2\sin(\theta)\sin(\phi)) \rangle \\
               &= \langle 4\sin^2(\theta)\cos(\phi), 4\sin^2(\theta)\sin(\phi), 4\cos(\theta)\sin(\theta)\cos^2(\phi) + 4\cos(\theta)\sin(\theta)\sin^2(\phi) \rangle
  .\end{align*}
  The magnitude of the normal vector is
  \begin{align*}
    \lvert \r_\theta \times \r_\phi \rvert &= \sqrt{\left(4\sin^2(\theta)\cos(\phi)\right)^2 + \left(4\sin^2(\theta)\sin(\phi)\right)^2 + \left(4\cos(\theta)\sin(\theta)\cos^2(\phi) + 4\cos(\theta)\sin(\theta)\sin^2(\phi)\right)^2} \\
                                           &= \sqrt{16\sin^4(\theta)\cos^2(\phi) + 16\sin^4(\theta)\sin^2(\phi) + 16\cos^2(\theta)\sin^2(\theta)} \\
                                           &= 4\sqrt{\sin^2(\theta)\left(\sin^2(\theta)\cos^2(\phi) + \sin^2(\theta)\sin^2(\phi) + \cos^2(\theta)\right)} \\
                                           &= 4\sin(\theta)\sqrt{\sin^2(\theta)\cos^2(\phi) + \sin^2(\theta)\sin^2(\phi) + \cos^2(\theta)} \\
                                           &= 4\sin(\theta)\sqrt{\sin^2(\theta)\left(\cos^2(\phi) + \sin^2(\phi)\right) + \cos^2(\theta)} \\
                                           &= 4\sin(\theta)\sqrt{\sin^2(\theta) + \cos^2(\theta)} \\
                                           &= 4\sin(\theta)
  .\end{align*}
  Therefore, the surface area element is
  \[%
    \dS = \lvert \r_\theta \times \r_\phi \rvert \,\dA = 4\sin(\theta) \,\dd{\theta} \,\dd{\phi}
  .\]%
  The surface area integral is
  \begin{align*}
    A &= \int_{\pi/6}^{2\pi/3} \int_0^{2\pi} 4\sin(\theta) \,\dd{\theta} \,\dd{\phi} \\
      &= 4 \int_{\pi/6}^{2\pi/3} \sin(\theta) \,\dd{\theta} \int_0^{2\pi} \,\dd{\phi} \\
      &= -8\pi \cos(\theta)\bigg\vert_{\pi/6}^{2\pi/3} \\
      &= -8\pi \left(\cos\left(\frac{2\pi}{3}\right) - \cos\left(\frac{\pi}{6}\right)\right) \\
      &= -8\pi \left(-\frac{1}{2} - \frac{\sqrt{3}}{2}\right) \\
      &= 4\pi + 4\sqrt{3}\pi
  .\qedhere\end{align*}
\end{proof}

\begin{proof}[Solution to (ii)]
  Using cylindrical coordinates, we have
  \[%
    x = r^2, \quad y = r\cos(\theta), \aand z = r\sin(\theta)
  ,\]%
  we get the parametric equation for the surface area as
  \[%
    \r = \langle r^2, r\cos(\theta), r\sin(\theta) \rangle
  .\]%
  The bounds for $r$ are $0 \le r \le 4$ and $0 \le \theta \le 2\pi$.

  We know that
  \begin{align*}
    \r_r &= \left\langle \pdv{x}{r}, \pdv{y}{r}, \pdv{z}{r} \right\rangle = \langle 2r, \cos(\theta), \sin(\theta) \rangle \\
    \r_\theta &= \left\langle \pdv{x}{\theta}, \pdv{y}{\theta}, \pdv{z}{\theta} \right\rangle = \langle 0, -r\sin(\theta), r\cos(\theta) \rangle
  .\end{align*}
  The normal vector is
  \begin{align*}
    \mathbf{N} &= \r_r \times \r_\theta = \begin{vmatrix}
      \mathbf{i} & \mathbf{j} & \mathbf{k} \\
      2r & \cos(\theta) & \sin(\theta) \\
      0 & -r\sin(\theta) & r\cos(\theta)
    \end{vmatrix} \\
               &= \langle (\cos(\theta))(r\cos(\theta)) - (\sin(\theta))(-r\sin(\theta)), -\left[(2r)(r\cos(\theta)) - (\sin(\theta))(0)\right], 2r(-r\sin(\theta)) - (\sin(\theta))(0) \rangle \\
               &= \langle r\cos^2(\theta) + r\sin^2(\theta), -2r^2\cos(\theta), 2r^2\sin(\theta) \rangle \\
               &= \langle r, -2r^2\cos(\theta), 2r^2\sin(\theta) \rangle
  \end{align*}
  The magnitude of the normal vector is
  \begin{align*}
    \lvert \r_r \times \r_\theta \rvert &= \sqrt{r^2 + 4r^4\cos^2(\theta) + 4r^4\sin^2(\theta)} \\
                                         &= \sqrt{r^2 + 4r^4} = r\sqrt{1 + 4r^2}
  .\end{align*}
  Therefore, the surface area element is
  \[%
    \dS = \lvert \r_r \times \r_\theta \rvert \,\dA = r\sqrt{1 + 4r^2} \,\dr \,\dd{\theta}
  .\]%

  The surface area integral is
  \[%
    A = \int_0^{2\pi} \int_0^4 r\sqrt{1 + 4r^2} \,\dr \,\dd{\theta}
  .\]%
  Let $u = 1 + 4r^2$, then $\du = 8r \,\dr$. The bounds for $u$ are $u(0) = 1$
  and $u(4) = 65$. Therefore, we have
  \begin{align*}
    A &= \int_0^{2\pi} \int_1^{65} \frac{1}{8} \sqrt{u} \,\du \,\dd{\theta} \\
      &= \frac{1}{8} \left(\frac{2}{3}u\sqrt{u}\bigg\vert_1^{65}\right) \cdot 2\pi \\
      &= \frac{1}{6}\left(65\sqrt{65} - 1\right)\pi
  .\qedhere\end{align*}
\end{proof}

\begin{proof}[Solution to (iii)]
  Using cylindrical coordinates, we have
  \[%
    x = 2\cos(\theta), \quad y = y, \aand z = 2\sin(\theta)
  ,\]%
  where $r = 2$. Therefore, we get the parametric equation for the surface area
  as $\r = \langle 2\cos(\theta), y, 2\sin(\theta) \rangle$. The bounds for
  $\theta$ are $0 \le \theta \le \pi$ and $0 \le y \le 2$. Since $x =
  2\cos(\theta)$ and $4\cos^2(\theta) + y^2 \le 4$, we have
  \[%
    y^2 \le 4 - 4\cos^2(\theta) = 4\sin^2(\theta)
  .\]%
  Therefore, the bounds for $y$ are $-2\lvert \sin(\theta) \rvert \le y \le
  2\lvert \sin(\theta) \rvert$.

  We know that
  \begin{align*}
    \r_\theta &= \left\langle \pdv{x}{\theta}, \pdv{y}{\theta}, \pdv{z}{\theta} \right\rangle = \langle -2\sin(\theta), 0, 2\cos(\theta) \rangle \\
    \r_y &= \left\langle \pdv{x}{y}, \pdv{y}{y}, \pdv{z}{y} \right\rangle = \langle 0, 1, 0 \rangle
  .\end{align*}
  The normal vector is
  \begin{align*}
    \mathbf{N} &= \r_\theta \times \r_y = \begin{vmatrix}
      \mathbf{i} & \mathbf{j} & \mathbf{k} \\
      -2\sin(\theta) & 0 & 2\cos(\theta) \\
      0 & 1 & 0
    \end{vmatrix} \\
               &= \langle -2\cos(\theta), 0, -2\sin(\theta) \rangle
  .\end{align*}
  The magnitude of the normal vector is
  \[%
    \lvert \r_\theta \times \r_y \rvert = \sqrt{4\cos^2(\theta) + 4\sin^2(\theta)} = 2
  .\]%
  Therefore, the surface area element is
  \[%
    \dS = \lvert \r_\theta \times \r_y \rvert \,\dA = 2 \,\dy \,\dd{\theta}
  .\]%

  The surface area integral is
  \begin{align*}
    A &= \int_0^\pi \int_{-2\lvert \sin(\theta) \rvert}^{2\lvert \sin(\theta) \rvert} 2 \,\dy \,\dd{\theta} \\
      &= \int_0^\pi 2 \cdot 4\lvert \sin(\theta) \rvert \,\dd{\theta} \\
      &= -8\cos(\theta)\bigg\vert_0^\pi = 16
  .\qedhere\end{align*}
\end{proof}
