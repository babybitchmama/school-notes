\section{Semisimple Lie Algebras}

\subsection{Theorems of Lie and Cartan}

\begin{exercise}[1.2.1.1]\label{exc:1.2.1.1}
  Let $L = \sl(V)$. Use Lie's Theorem to prove that $\Rad L = Z(L)$; conclude that $L$ is semisimple (cf. Exercise \ref{exc:1.1.2.3}). [Observe that $\Rad L$ lies in each maximal solvable subalgebra $B$ of $L$. Select a basis of $V$ so that $B = L \cap \t(n, \F)$, and notice that the transpose of $B$ is also a maximal solvable subalgebra of $L$. Conclude that $\Rad L \subset L \cap \d(n, \F)$, then that $\Rad L = Z(L)$.]
\end{exercise}

\begin{solution}[1.2.1.1]\label{sol:1.2.1.1}
\end{solution}

\begin{exercise}[1.2.1.2]\label{exc:1.2.1.2}
  Show that the proof of Theorem 4.1 still goes through in prime characteristic, provided $\dim V$ is less than char $\F$.
\end{exercise}

\begin{solution}[1.2.1.2]\label{sol:1.2.1.2}
\end{solution}

\begin{exercise}[1.2.1.3]\label{exc:1.2.1.3}
  This exercise illustrates the failure of Lie's Theorem when $\F$ is allowed to have prime characteristic $p$. Consider the $p \times p$ matrices:
  \[%
    x = \begin{bmatrix}
      0 & 1 & 0 & \cdots & 0 & 0 \\
      0 & 0 & 1 & \cdots & 0 & 0 \\
      \vdots & \vdots & \ddots & \ddots & \vdots & \vdots \\
      0 & 0 & \cdots & 0 & 1 & 0 \\
      0 & 0 & \cdots & 0 & 0 & 1 \\
      1 & 0 & \cdots & 0 & 0 & 0 \\
    \end{bmatrix},
    \quad
    y = \diag(0,1,2,3, \cdots, p-1) .
  .\]%
  Check that $[x, y]=x$, hence that $x$ and $y$ span a two dimensional solvable subalgebra $L$ of $\gl(p, \F)$. Verify that $x, y$ have no common eigenvector.
\end{exercise}

\begin{solution}[1.2.1.3]\label{sol:1.2.1.3}
\end{solution}

\begin{exercise}[1.2.1.4]\label{exc:1.2.1.4}
  Exercise \ref{exc:1.2.1.3} shows that a solvable Lie algebra of endomorphisms over a field of prime characteristic $p$ need not have derived algebra consisting of nilpotent endomorphisms. For arbitrary $p$, construct a counterexample to Corollary C of Theorem 4.1 as follows: Start with $L \subset \gl(p, \F)$ as in Exercise \ref{exc:1.2.1.3}. Form the vector space direct sum $M = L + \F^p$, and make $M$ a Lie algebra by decreeing that $\F^p$ is abelian, while $L$ has its usual product and acts on $\F^p$ in the given way. Verify that $M$ is solvable, but that its derived algebra ($= \F x + \F^p$) fails to be nilpotent.
\end{exercise}

\begin{solution}[1.2.1.4]\label{sol:1.2.1.4}
\end{solution}

\begin{exercise}[1.2.1.5]\label{exc:1.2.1.5}
  If $x, y \in \End V$ commute, prove that $(x + y)_s = x_s + y_s$, and $(x + y)_n=$ $x_n + y_n$. Show by example that this can fail if $x, y$ fail to commute. [Show first that $x, y$ semisimple (resp. nilpotent) implies $x + y$ semisimple (resp. nilpotent).]
\end{exercise}

\begin{solution}[1.2.1.5]\label{sol:1.2.1.5}
\end{solution}

\begin{exercise}[1.2.1.6]\label{exc:1.2.1.6}
  Check formula (*) at the end of (4.2).
\end{exercise}

\begin{solution}[1.2.1.6]\label{sol:1.2.1.6}
\end{solution}

\begin{exercise}[1.2.1.7]\label{exc:1.2.1.7}
  Prove the converse of Theorem 4.3.
\end{exercise}

\begin{solution}[1.2.1.7]\label{sol:1.2.1.7}
\end{solution}

\begin{exercise}[1.2.1.8]\label{exc:1.2.1.8}
  Note that it suffices to check the hypothesis of Theorem 4.3 (or its corollary) for $x, y$ ranging over a basis of $[LL]$, resp. $L$. For the example given in Exercise \ref{exc:1.1.1.2}, verify solvability by using Cartan's Criterion.
\end{exercise}

\begin{solution}[1.2.1.8]\label{sol:1.2.1.8}
\end{solution}

\subsection{Killing form}

\begin{exercise}[1.2.2.1]\label{exc:1.2.2.1}
  Prove that if $L$ is nilpotent, the Killing form of $L$ is identically zero.
\end{exercise}

\begin{solution}[1.2.2.1]\label{sol:1.2.2.1}
\end{solution}

\begin{exercise}[1.2.2.2]\label{exc:1.2.2.2}
  Prove that $L$ is solvable if and only if $[LL]$ lies in the radical of the Killing form.
\end{exercise}

\begin{solution}[1.2.2.2]\label{sol:1.2.2.2}
\end{solution}

\begin{exercise}[1.2.2.3]\label{exc:1.2.2.3}
  Let $L$ be the two dimensional nonabelian Lie algebra (1.4), which is solvable. Prove that $L$ has nontrivial Killing form.
\end{exercise}

\begin{solution}[1.2.2.3]\label{sol:1.2.2.3}
\end{solution}

\begin{exercise}[1.2.2.4]\label{exc:1.2.2.4}
  Let $L$ be the three dimensional solvable Lie algebra of Exercise \ref{exc:1.1.1.2}. Compute the radical of its Killing form.
\end{exercise}

\begin{solution}[1.2.2.4]\label{sol:1.2.2.4}
\end{solution}

\begin{exercise}[1.2.2.5]\label{exc:1.2.2.5}
  Let $L = \sl(2, \F)$. Compute the basis of $L$ dual to the standard basis, relative to the Killing form.
\end{exercise}

\begin{solution}[1.2.2.5]\label{sol:1.2.2.5}
\end{solution}

\begin{exercise}[1.2.2.6]\label{exc:1.2.2.6}
  Let char $\F = p \neq 0$. Prove that $L$ is semisimple if its Killing form is nondegenerate. Show by example that the converse fails. [Look at $\sl[(3, \F)$ modulo its center, when char $\F = 3$.]
\end{exercise}

\begin{solution}[1.2.2.6]\label{sol:1.2.2.6}
\end{solution}

\begin{exercise}[1.2.2.7]\label{exc:1.2.2.7}
  Relative to the standard basis of $\sl(3, \F)$, compute the determinant of $\kappa$. Which primes divide it?
\end{exercise}

\begin{solution}[1.2.2.7]\label{sol:1.2.2.7}
\end{solution}

\begin{exercise}[1.2.2.8]\label{exc:1.2.2.8}
  Let $L = L_1 \oplus \cdots \oplus L_t$ be the decomposition of a semisimple Lie algebra $L$ into its simple ideals. Show that the semisimple and nilpotent parts of $x \in L$ are the sums of the semisimple and nilpotent parts in the various $L_i$ of the components of $x$.
\end{exercise}

\begin{solution}[1.2.2.8]\label{sol:1.2.2.8}
\end{solution}

\subsection{Complete reducibility of representations}

\begin{exercise}[1.2.3.1]\label{exc:1.2.3.1}
  Using the standard basis for $L = \sl(2, \F)$, write down the Casimir element of the adjoint representation of $L$ (cf. Exercise \ref{exc:1.2.2.5}). Do the same thing for the usual (3-dimensional) representation of $\sl(3, \F)$, first computing dual bases relative to the trace form.
\end{exercise}

\begin{solution}[1.2.3.1]\label{sol:1.2.3.1}
\end{solution}

\begin{exercise}[1.2.3.2]\label{exc:1.2.3.2}
  Let $V$ be an $L$-module. Prove that $V$ is a direct sum of irreducible submodules if and only if each $L$-submodule of $V$ possesses a complement.
\end{exercise}

\begin{solution}[1.2.3.2]\label{sol:1.2.3.2}
\end{solution}

\begin{exercise}[1.2.3.3]\label{exc:1.2.3.3}
  If $L$ is solvable, every irreducible representation of $L$ is one dimensional.
\end{exercise}

\begin{solution}[1.2.3.3]\label{sol:1.2.3.3}
\end{solution}

\begin{exercise}[1.2.3.4]\label{exc:1.2.3.4}
  Use Weyl's Theorem to give another proof that for $L$ semisimple, $\ad L= \Der L$ (Theorem 5.3). [If $\delta \in \Der L$, make the direct sum $\F + L$ into an $L$-module via the rule $x.(a, y) = (0, a\delta(x) + [xy])$. Then consider a complement to the submodule $L$.]
\end{exercise}

\begin{solution}[1.2.3.4]\label{sol:1.2.3.4}
\end{solution}

\begin{exercise}[1.2.3.5]\label{exc:1.2.3.5}
  A Lie algebra $L$ for which $\Rad L = Z(L)$ is called reductive. (Examples: $L$ abelian, $L$ semisimple, $L = \gl(n, \F)$.)
  \begin{enumerate}
    \item If $L$ is reductive, then $L$ is a completely reducible $\ad L$-module. [If $\ad L \neq 0$, use Weyl's Theorem.] In particular, $L$ is the direct sum of $Z(L)$ and $[LL]$, with $[LL]$ semisimple.

    \item If $L$ is a classical linear Lie algebra (1.2), then $L$ is semisimple. [Cf. Exercise \ref{exc:1.1.1.9}.]

    \item If $L$ is a completely reducible $\ad L$-module, then $L$ is reductive.

    \item If $L$ is reductive, then all finite dimensional representations of $L$ in which $Z(L)$ is represented by semisimple endomorphisms are completely reducible.
  \end{enumerate}
\end{exercise}

\begin{solution}[1.2.3.5(i)]\label{sol:1.2.3.5(i)}
\end{solution}

\begin{solution}[1.2.3.5(ii)]\label{sol:1.2.3.5(ii)}
\end{solution}

\begin{solution}[1.2.3.5(iii)]\label{sol:1.2.3.5(iii)}
\end{solution}

\begin{solution}[1.2.3.5(iv)]\label{sol:1.2.3.5(iv)}
\end{solution}

\begin{exercise}[1.2.3.6]\label{exc:1.2.3.6}
  Let $L$ be a simple Lie algebra. Let $\beta(x, y)$ and $\gamma(x, y)$ be two symmetric associative bilinear forms on $L$. If $\beta, \gamma$ are nondegenerate, prove that $\beta$ and $\gamma$ are proportional. [Use Schur's Lemma.]
\end{exercise}

\begin{solution}[1.2.3.6]\label{sol:1.2.3.6}
\end{solution}

\begin{exercise}[1.2.3.7]\label{exc:1.2.3.7}
  It will be seen later on that $\sl(n, \F)$ is actually simple. Assuming this and using Exercise \ref{exc:1.2.3.6}, prove that the Killing form $\kappa$ on $\sl(n, \F)$ is related to the ordinary trace form by $\kappa(x, y) = 2n \Tr(x y)$.
\end{exercise}

\begin{solution}[1.2.3.7]\label{sol:1.2.3.7}
\end{solution}

\begin{exercise}[1.2.3.8]\label{exc:1.2.3.8}
  If $L$ is a Lie algebra, then $L$ acts (via ad) on $(L \otimes L)^*$, which may be identified with the space of all bilinear forms $\beta$ on $L$. Prove that $\beta$ is associative if and only if $L.\beta = 0$.
\end{exercise}

\begin{solution}[1.2.3.8]\label{sol:1.2.3.8}
\end{solution}

\begin{exercise}[1.2.3.9]\label{exc:1.2.3.9}
  Let $L'$ be a semisimple subalgebra of a semisimple Lie algebra $L$. If $x \in L'$, its Jordan decomposition in $L'$ is also its Jordan decomposition in $L$.
\end{exercise}

\begin{solution}[1.2.3.9]\label{sol:1.2.3.9}
\end{solution}

\subsection{Representations of $\sl(2, \F)$}

\begin{exercise}[1.2.4.1]\label{exc:1.2.4.1}
  Use Lie's Theorem to prove the existence of a maximal vector in an arbitrary finite dimensional $L$-module. [Look at the subalgebra $B$ spanned by $h$ and $x$.]
\end{exercise}

\begin{solution}[1.2.4.1]\label{sol:1.2.4.1}
\end{solution}

\begin{exercise}[1.2.4.2]\label{exc:1.2.4.2}
  $M = \sl(3, \F)$ contains a copy of $L$ in its upper left-hand $2 \times 2$ position. Write $M$ as direct sum of irreducible $L$-submodules ($M$ viewed as $L$ module via the adjoint representation): $V(0) \oplus V(1) \oplus V(1) \oplus V(2)$.
\end{exercise}

\begin{solution}[1.2.4.2]\label{sol:1.2.4.2}
\end{solution}

\begin{exercise}[1.2.4.3]\label{exc:1.2.4.3}
  Verify that formulas (a)-(c) of Lemma 7.2 do define an irreducible representation of $L$. [To show that they define a representation, it suffices to show that the matrices corresponding to $x, y, h$ satisfy the same structural equations as $x, y, h$.]
\end{exercise}

\begin{solution}[1.2.4.3]\label{sol:1.2.4.3}
\end{solution}

\begin{exercise}[1.2.4.4]\label{exc:1.2.4.4}
  The irreducible representation of $L$ of highest weight $m$ can also be realized ``naturally'', as follows. Let $X, Y$ be a basis for the two dimensional vector space $\F^2$, on which $L$ acts as usual. Let $\mathscr{R} = \F[X, Y]$ be the polynomial algebra in two variables, and extend the action of $L$ to $\mathscr{R}$ by the derivation rule: $z . f g=(z . f) g+f(z . g)$, for $z \in L, f, g \in \mathscr{R}$. Show that this extension is well defined and that $\mathscr{R}$ becomes an $L$-module. Then show that the subspace of homogeneous polynomials of degree $m$, with basis $X^m$, $X^{m-1} Y, \cdots, X Y^{m-1}, Y^m$, is invariant under $L$ and irreducible of highest weight $m$.
\end{exercise}

\begin{solution}[1.2.4.4]\label{sol:1.2.4.4}
\end{solution}

\begin{exercise}[1.2.4.5]\label{exc:1.2.4.5}
  Suppose char $\F = p > 0, L = \sl(2, \F)$. Prove that the representation $V(m)$ of $L$ constructed as in Exercise \ref{exc:1.2.4.3} or \ref{exc:1.2.4.4} is irreducible so long as the highest weight $m$ is strictly less than $p$, but reducible when $m = p$.
\end{exercise}

\begin{solution}[1.2.4.5]\label{sol:1.2.4.5}
\end{solution}

\begin{exercise}[1.2.4.6]\label{exc:1.2.4.6}
  Decompose the tensor product of the two $L$-modules $V(3), V(7)$ into the sum of irreducible submodules: $V(4) \oplus V(6) \oplus V(8) \oplus V(10)$. Try to develop a general formula for the decomposition of $V(m) \otimes V(n)$.
\end{exercise}

\begin{solution}[1.2.4.6]\label{sol:1.2.4.6}
\end{solution}

\begin{exercise}[1.2.4.7]\label{exc:1.2.4.7}
  In this exercise we construct certain infinite dimensional $L$-modules. Let $\lambda \in \F$ be an arbitrary scalar. Let $Z(\lambda)$ be a vector space over F with countably infinite basis $\left(v_0, v_1, v_2, \cdots\right)$.
  \begin{enumerate}
    \item Prove that formulas (a) -- (c) of Lemma 7.2 define an $L$-module structure on $Z(\lambda)$, and that every nonzero $L$-submodule of $Z(\lambda)$ contains at least one maximal vector.

    \item Suppose $\lambda + 1 = i$ is a positive integer. Prove that $v_i$ is a maximal vector. This induces an $L$-module homomorphism $Z(\mu) \xrightarrow{\phi} Z(\lambda), \mu = \lambda - 2i$, sending $v_0$ to $v_i$. Show that $\phi$ is a monomorphism, and that $\operatorname{Im} \phi$, $Z(\lambda) / \Im \phi$ are both irreducible $L$-modules (but $Z(\lambda)$ fails to be completely reducible).

    \item Suppose $\lambda+1$ is not a positive integer. Prove that $Z(\lambda)$ is irreducible.
  \end{enumerate}
\end{exercise}

\begin{solution}[1.2.4.7(i)]\label{sol:1.2.4.7(i)}
\end{solution}

\begin{solution}[1.2.4.7(ii)]\label{exc:1.2.4.7(ii)}
\end{solution}

\begin{solution}[1.2.4.7(iii)]\label{sol:1.2.4.7(iii)}
\end{solution}

\subsection{Root space decomposition}

\begin{exercise}[1.2.5.1]\label{exc:1.2.5.1}
  If $L$ is a classical linear Lie algebra of type $\Al, \Bl, \Cl$, or $\Dl$ (see (1.2)), prove that the set of all diagonal matrices in $L$ is a maximal toral subalgebra, of dimension $\ell$. (Cf. Exercise \ref{exc:1.1.2.8}.)
\end{exercise}

\begin{solution}[1.2.5.1]\label{sol:1.2.5.1}
\end{solution}

\begin{exercise}[1.2.5.2]\label{exc:1.2.5.2}
  For each algebra in Exercise \ref{exc:1.2.5.1}, determine the roots and root spaces. How are the various $h_\alpha$ expressed in terms of the basis for $H$ given in (1.2)?
\end{exercise}

\begin{solution}[1.2.5.2]\label{sol:1.2.5.2}
\end{solution}

\begin{exercise}[1.2.5.3]\label{exc:1.2.5.3}
  If $L$ is of classical type, compute explicitly the restriction of the Killing form to the maximal toral subalgebra described in Exercise \ref{exc:1.2.5.1}.
\end{exercise}

\begin{solution}[1.2.5.3]\label{sol:1.2.5.3}
\end{solution}

\begin{exercise}[1.2.5.4]\label{exc:1.2.5.4}
  If $L = \sl(2, \F)$, prove that each maximal toral subalgebra is one dimensional.
\end{exercise}

\begin{solution}[1.2.5.4]\label{sol:1.2.5.4}
\end{solution}

\begin{exercise}[1.2.5.5]\label{exc:1.2.5.5}
  If $L$ is semisimple, $H$ a maximal toral subalgebra, prove that $H$ is selfnormalizing (i.e., $H = N_L(H)$).
\end{exercise}

\begin{solution}[1.2.5.5]\label{sol:1.2.5.5}
\end{solution}

\begin{exercise}[1.2.5.6]\label{exc:1.2.5.6}
  Compute the basis of $\sl(n, \F)$ which is dual (via the Killing form) to the standard basis. (Cf. Exercise \ref{exc:1.2.3.5}.)
\end{exercise}

\begin{solution}[1.2.5.6]\label{sol:1.2.5.6}
\end{solution}

\begin{exercise}[1.2.5.7]\label{exc:1.2.5.7}
  Let $L$ be semisimple, $H$ a maximal toral subalgebra. If $h \in H$, prove that $C_L(h)$ is reductive (in the sense of Exercise \ref{exc:1.2.4.5}). Prove that $H$ contains elements $h$ for which $C_L(h) = H$; for which $h$ in $\sl(n, \F)$ is this true?
\end{exercise}

\begin{solution}[1.2.5.7]\label{sol:1.2.5.7}
\end{solution}

\begin{exercise}[1.2.5.8]\label{exc:1.2.5.8}
  For $\sl(n, \F)$ (and other classical algebras), calculate explicitly the root strings and Cartan integers. In particular, prove that all Cartan integers $2(\alpha, \beta) / (\beta, \beta), \alpha \neq \pm \beta$, for $\sl(n, \F)$ are $0, \pm 1$.
\end{exercise}

\begin{solution}[1.2.5.8]\label{sol:1.2.5.8}
\end{solution}

\begin{exercise}[1.2.5.9]\label{exc:1.2.5.9}
  Prove that every three dimensional semisimple Lie algebra has the same root system as $\sl(2, \F)$, hence is isomorphic to $\sl(2, \F)$.
\end{exercise}

\begin{solution}[1.2.5.9]\label{sol:1.2.5.9}
\end{solution}

\begin{exercise}[1.2.5.10]\label{exc:1.2.5.10}
  Prove that no four, five or seven dimensional semisimple Lie algebras exist.
\end{exercise}

\begin{solution}[1.2.5.10]\label{sol:1.2.5.10}
\end{solution}

\begin{exercise}[1.2.5.11]\label{exc:1.2.5.11}
  If $(\alpha, \beta)>0$, prove that $\alpha-\beta \in \Phi(\alpha, \beta \in \Phi)$. Is the converse true?
\end{exercise}

\begin{solution}[1.2.5.11]\label{sol:1.2.5.11}
\end{solution}
