\section{Root Systems}

\subsection{Axiomatics}

\begin{note}
  Unless otherwise specified, $\Phi$ denotes a root system in $\E$, with Weyl group $\W$.
\end{note}

\begin{exercise}[1.3.1.1]\label{exc:1.3.1.1}
  Let $\E'$ be a subspace of $\E$. If a reflection $\sigma_\alpha$ leaves $\E'$ invariant, prove that either $\alpha \in \E'$ or else $\E' \in P_\alpha$.
\end{exercise}

\begin{solution}[1.3.1.1]\label{sol:1.3.1.1}
\end{solution}

\begin{exercise}[1.3.1.2]\label{exc:1.3.1.2}
  Prove that $\Phi^\vee$ is a root system in $\E$, whose Weyl group is naturally isomorphic to $\W$; show also that $\bra{\alpha^\lor, \beta^\lor} = \bra{\beta, \alpha}$, and draw a picture of $\Phi^\vee$ in the cases $\A_1, \A_2, \B_2, \G_2$.
\end{exercise}

\begin{solution}[1.3.1.2]\label{sol:1.3.1.2}
\end{solution}

\begin{exercise}[1.3.1.3]\label{exc:1.3.1.3}
  In Table 1, show that the order of $\sigma_\alpha \sigma_\beta$ in $\W$ is (respectively) $2$, $3$, $4$, $6$ when $\theta = \pi/2$, $\pi/3$ (or $2\pi/ 3$ ), $\pi/4$ (or $3\pi/4$ ), $\pi/6$ (or $5\pi/6$ ). [Note that $\sigma_\alpha \sigma_\beta =$ rotation through $2\theta$.]
\end{exercise}

\begin{solution}[1.3.1.3]\label{sol:1.3.1.3}
\end{solution}

\begin{exercise}[1.3.1.4]\label{exc:1.3.1.4}
  Prove that the respective Weyl groups of $\A_1 \times \A_1, \A_2, \B _2, \G_2$ are dihedral of order $4$, $6$, $8$, $12$. If $\Phi$ is any root system of rank $2$, prove that its Weyl group must be one of these.
\end{exercise}

\begin{solution}[1.3.1.4]\label{sol:1.3.1.4}
\end{solution}

\begin{exercise}[1.3.1.5]\label{exc:1.3.1.5}
  Show by example that $\alpha - \beta$ may be a root even when $(\alpha, \beta) \leq 0$ (cf. Lemma 9.4).
\end{exercise}

\begin{solution}[1.3.1.5]\label{sol:1.3.1.5}
\end{solution}

\begin{exercise}[1.3.1.6]\label{exc:1.3.1.6}
  Prove that $\W$ is a normal subgroup of $\Aut \Phi$ ($=$ group of all isomorphisms of $\Phi$ onto itself).
\end{exercise}

\begin{solution}[1.3.1.6]\label{sol:1.3.1.6}
\end{solution}

\begin{exercise}[1.3.1.7]\label{exc:1.3.1.7}
  Let $\alpha, \beta \in \Phi$ span a subspace $\E'$ of $\E$. Prove that $\E' \cap \Phi$ is a root system in $\E'$. Prove similarly that $\Phi \cap(\Z \alpha + \Z\beta)$ is a root system in $\E'$ (must this coincide with $\E' \cap \Phi$?). More generally, let $\Phi'$ be a nonempty subset of $\Phi$ such that $\Phi' = -\Phi'$, and such that $\alpha, \beta \in \Phi', \alpha + \beta \in \Phi$ implies $\alpha + \beta \in \Phi'$. Prove that $\Phi'$ is a root system in the subspace of $\E$ it spans. [Use Table 1].
\end{exercise}

\begin{solution}[1.3.1.7]\label{sol:1.3.1.7}
\end{solution}

\begin{exercise}[1.3.1.8]\label{exc:1.3.1.8}
  Compute root strings in $\G_2$ to verify the relation $r - q = \bra{\beta, \alpha}$.
\end{exercise}

\begin{solution}[1.3.1.8]\label{sol:1.3.1.8}
\end{solution}

\begin{exercise}[1.3.1.9]\label{exc:1.3.1.9}
  Let $\Phi$ be a set of vectors in a euclidean space $\E$, satisfying only (R1), (R3), (R4). Prove that the only possible multiples of $\alpha \in \Phi$ which can be in $\Phi$ are $\pm 1/2\alpha, \pm \alpha, \pm 2\alpha$. Verify that $\{\alpha \in \Phi \mid 2\alpha \notin \Phi\}$ is a root system.
\end{exercise}

\begin{solution}[1.3.1.9]\label{sol:1.3.1.9}
\end{solution}

\subsection{Simple roots and Weyl group}

\begin{exercise}[1.3.2.1]\label{exc:1.3.2.1}
  Let $\Phi^\vee$ be the dual system of $\Phi$, $\Delta^\vee=\left\{\alpha^\vee \mid \alpha \in \Delta\right\}$. Prove that $\Delta^\vee$ is a base of $\Phi^\vee$. [Compare Weyl chambers of $\Phi$ and $\Phi^\vee$.]
\end{exercise}

\begin{solution}[1.3.2.1]\label{sol:1.3.2.1}
\end{solution}

\begin{exercise}[1.3.2.2]\label{exc:1.3.2.2}
  If $\Delta$ is a base of $\Phi$, prove that the set $\left(\Z\alpha + \Z\beta\right) \cap \Phi$ $(\alpha \neq \beta$ in $\Delta)$ is a root system of rank $2$ in the subspace of $\E$ spanned by $\alpha, \beta$ (cf. Exercise \ref{exc:1.3.1.7}). Generalize to an arbitrary subset of $\Delta$.
\end{exercise}

\begin{solution}[1.3.2.2]\label{sol:1.3.2.2}
\end{solution}

\begin{exercise}[1.3.2.3]\label{exc:1.3.2.3}
  Prove that each root system of rank 2 is isomorphic to one of those listed in (9.3).
\end{exercise}

\begin{solution}[1.3.2.3]\label{sol:1.3.2.3}
\end{solution}

\begin{exercise}[1.3.2.4]\label{exc:1.3.2.4}
  Verify the Corollary of Lemma 10.2A directly for $\G_2$.
\end{exercise}

\begin{solution}[1.3.2.4]\label{sol:1.3.2.4}
\end{solution}

\begin{exercise}[1.3.2.5]\label{exc:1.3.2.5}
  If $\sigma \in \W$ can be written as a product of $t$ simple reflections, prove that $t$ has the same parity as $\ell(\sigma)$.
\end{exercise}

\begin{solution}[1.3.2.5]\label{sol:1.3.2.5}
\end{solution}

\begin{exercise}[1.3.2.6]\label{exc:1.3.2.6}
  Define a function $sn : \W \to \{\pm1\}$ by $sn(\sigma) = (-1)^{\ell(\sigma)}$. Prove that $sn$ is a homomorphism (cf. the case $\A_2$, where $\W$ is isomorphic to the symmetric group $\mathscr{S}_3$ ).
\end{exercise}

\begin{solution}[1.3.2.6]\label{sol:1.3.2.6}
\end{solution}

\begin{exercise}[1.3.2.7]\label{exc:1.3.2.7}
  Prove that the intersection of ``positive'' open half-spaces associated with any basis $\gamma_1, \cdots, \gamma_l$ of $\E$ is nonvoid. [If $\delta_i$ is the projection of $\gamma_i$ on the orthogonal complement of the subspace spanned by all basis vectors except $\gamma_i$, consider $\gamma = \sum r_i \delta_i$ when all $r_i > 0$.]
\end{exercise}

\begin{solution}[1.3.2.7]\label{sol:1.3.2.7}
\end{solution}

\begin{exercise}[1.3.2.8]\label{exc:1.3.2.8}
  Let $\Delta$ be a base of $\Phi$, $\alpha \neq \beta$, simple roots, $\Phi_{\alpha \beta}$ the rank 2 root system in $\E_{\alpha \beta} = \R\alpha + \R\beta$ (see Exercise ref{exc:1.3.2.2} above). The Weyl group $\W_{\alpha\beta}$ of $\Phi_{\alpha \beta}$ is generated by the restrictions $\tau_\alpha, \tau_\beta$ to $\E_{\alpha \beta}$ of $\sigma_\alpha, \sigma_\beta$, and $\W_{a \beta}$ may be viewed as a subgroup of $\W$. Prove that the ``length'' of an element of $\W_{\alpha\beta}$ (relative to $\tau_\alpha$, $\tau_\beta$) coincides with the length of the corresponding clement of $\W$.
\end{exercise}

\begin{solution}[1.3.2.8]\label{sol:1.3.2.8}
\end{solution}

\begin{exercise}[1.3.2.9]\label{exc:1.3.2.9}
  Prove that there is a unique element $\sigma$ in $\W$ sending $\Psi^+$ to $\Phi^{-}$(relative to $\Delta$ ). Prove that any reduced expression for $\sigma$ must involve all $\sigma_\alpha$ ($\alpha \in \Delta$). Discuss $\ell(\sigma)$.
\end{exercise}

\begin{solution}[1.3.2.9]\label{sol:1.3.2.9}
\end{solution}

\begin{exercise}[1.3.2.10]\label{exc:1.3.2.10}
  Given $\Delta = \left\{\alpha_1, \cdots, \alpha_l\right\}$ in $\Phi$, let $\lambda = \sum_{i=1}^l k_i\alpha_i$ ($k_i \in \Z$, for all $k_i \ge 0$ or all $k_i \le 0$). Prove that either $\lambda$ is a multiple (possibly 0) of a root, or else there exists $\sigma \in \W$ such that $\sigma\lambda = \sum_{i=1}^l k_i'\alpha_i$, with some $k_i' > 0$ and some $k_i' < 0$. [Sketch of proof: If $\lambda$ is not a multiple of any root, then the hyperplane $P_\lambda$ orthogonal to $\lambda$ is not included in $\bigcup_{\alpha \in \Phi} P_\alpha$. Take $\mu \in P_\lambda - \bigcup_{\alpha \in \Phi} P_\alpha$. Then find $\sigma \in \W$ for which all $(\alpha_i, \sigma\mu) > 0$. It follows that $0 = (\lambda, \mu) = (\sigma\lambda, \sigma\mu) = \sum k(\alpha_i, \sigma\mu)$.]
\end{exercise}

\begin{solution}[1.3.2.10]\label{sol:1.3.2.10}
\end{solution}

\begin{exercise}[1.3.2.11]\label{exc:1.3.2.11}
  Let $\Phi$ be irreducible. Prove that $\Phi^\vee$ is also irreducible. If $\Phi$ has all roots of equal length, so does $\Phi^\vee$ (and then $\Phi^\vee$ is isomorphic to $\Phi$ ). On the other hand, if $\Phi$ has two root lengths, then so does $\Phi^\vee$; but if $\alpha$ is long, then $\alpha^\vee$ is short (and vice versa). Use this fact to prove that $\Phi$ has a unique maximal short root (relative to the partial order $\prec$ defined by $\Delta$ ).
\end{exercise}

\begin{solution}[1.3.2.11]\label{sol:1.3.2.11}
\end{solution}

\begin{exercise}[1.3.2.12]\label{exc:1.3.2.12}
  Let $\lambda \in \C(\Delta)$. If $\sigma\lambda = \lambda$ for some $\sigma \in \W$, then $\sigma = 1$.
\end{exercise}

\begin{solution}[1.3.2.12]\label{sol:1.3.2.12}
\end{solution}

\begin{exercise}[1.3.2.13]\label{exc:1.3.2.13}
  The only reflections in $\W$ are those of the form $\sigma_\alpha$ $(\alpha \in \Phi)$. [A vector in the reflecting hyperplane would, if orthogonal to no root, be fixed only by the identity in $\W$.]
\end{exercise}

\begin{solution}[1.3.2.13]\label{sol:1.3.2.13}
\end{solution}

\begin{exercise}[1.3.2.14]\label{exc:1.3.2.14}
  Prove that each point of $\E$ is $\W$-conjugate to a point in the closure of the fundamental Weyl chamber relative to a base $\Delta$. [Enlarge the partial order on $\E$ by defining $\mu \prec \lambda$ iff $\lambda - \mu$ is a nonnegative $\R$-linear combination of simple roots. If $\mu \in \E$, choose $\sigma \in \W$ for which $\lambda = \sigma\mu$ is maximal in this partial order.]
\end{exercise}

\begin{solution}[1.3.2.14]\label{sol:1.3.2.14}
\end{solution}

\subsection{Classification}

\begin{exercise}[1.3.3.1]\label{exc:1.3.3.1}
  Verify the Cartan matrices (Table 1).
\end{exercise}

\begin{solution}[1.3.3.1]\label{sol:1.3.3.1}
\end{solution}

\begin{exercise}[1.3.3.2]\label{exc:1.3.3.2}
  Calculate the determinants of the Cartan matrices (using induction on $\ell$ for types $\Al - \Dl$), which are as follows:
  \[%
    \Al: \ell + 1;~\Bl: 2;~\Cl: 2;~\Dl: 4;~\E_6: 3;~\E_7: 2;~\E_8,~\F_4~\text{and}~\G_2: 1
  .\]%
\end{exercise}

\begin{solution}[1.3.3.2]\label{sol:1.3.3.2}
\end{solution}

\begin{exercise}[1.3.3.3]\label{exc:1.3.3.3}
  Use the algorithm of (11.1) to write down all roots for $\G_2$. Do the same for $\CC_3:
  \begin{pmatrix}
    2 & -1 & 0 \\
    -1 & 2 & -1 \\
    0 & -2 & 2 \\
  \end{pmatrix}$.
\end{exercise}

\begin{solution}[1.3.3.3]\label{sol:1.3.3.3}
\end{solution}

\begin{exercise}[1.3.3.4]\label{exc:1.3.3.4}
  Prove that the Weyl group of a root system $\Phi$ is isomorphic to the direct product of the respective Weyl groups of its irreducible components.
\end{exercise}

\begin{solution}[1.3.3.4]\label{sol:1.3.3.4}
\end{solution}

\begin{exercise}[1.3.3.5]\label{exc:1.3.3.5}
  Prove that each irreducible root system is isomorphic to its dual, except that $\Bl$, $\Cl$ are dual to each other.
\end{exercise}

\begin{solution}[1.3.3.5]\label{sol:1.3.3.5}
\end{solution}

\begin{exercise}[1.3.3.6]\label{exc:1.3.3.6}
  Prove that an inclusion of one Dynkin diagram in another (e.g., $\E_6$ in $\E_7$ or $\E_7$ in $\E_8$) induces an inclusion of the corresponding root systems.
\end{exercise}

\begin{solution}[1.3.3.6]\label{sol:1.3.3.6}
\end{solution}

\subsection{Construction of root systems and automorphisms}

\begin{exercise}[1.3.4.1]\label{exc:1.3.4.1}
  Verify the details of the constructions in (12.1).
\end{exercise}

\begin{solution}[1.3.4.1]\label{sol:1.3.4.1}
\end{solution}

\begin{exercise}[1.3.4.2]\label{exc:1.3.4.2}
  Verify Table 2.
\end{exercise}

\begin{solution}[1.3.4.2]\label{sol:1.3.4.2}
\end{solution}

\begin{exercise}[1.3.4.3]\label{exc:1.3.4.3}
  Let $\Phi \subset E$ satisfy (R1), (R3), (R4), but not (R2), cf. Exercise \ref{exc:1.3.2.9}. Suppose moreover that $\Phi$ is irreducible, in the sense of $\S$11. Prove that $\Phi$ is the union of root systems of type $\B_n, \C_n$ in $\E$ (if $\dim \E = n > 1$), where the long roots of $\B_n$ are also the short roots of $\C_n$. (This is called the non-reduced root system of type $\mathrm{BC}_n$ in the literature). See table \ref{tab:highest_long_and_short_roots}.
  \begin{table}
    \centering
    \caption{Highest long and short roots}
    \begin{tabular}{lll}
      \hline
      Type & Long & Short \\
      \hline
      $\Al$ & $\alpha_1 + \alpha_2 + \cdots + \alpha_\ell$ & \\
      $\Bl$ & $\alpha_1 + 2\alpha_2 + 2\alpha_3 + \cdots + 2\alpha_\ell$ & $\alpha_1 + \alpha_2 + \cdots + \alpha_\ell$ \\
      $\Cl$ & $2\alpha_1 + 2 \alpha_2 + \cdots + 2 \alpha_{\ell-1} + \alpha_\ell$ & $\alpha_1 + 2\alpha_2 + \cdots + 2\alpha_{\ell-1} + \alpha_\ell$ \\
      $\Dl$ & $\alpha_1 + 2 \alpha_2 + \cdots + 2 \alpha_{l-2} + \alpha_{l-1} + \alpha_l$ & \\
      $\E_6$ & $\alpha_1 + 2\alpha_2 + 2\alpha_3 + 3\alpha_4 + 2\alpha_5 + \alpha_6$ & \\
      $\E_7$ & $2\alpha_1 + 2\alpha_2 + 3\alpha_3 + 4\alpha_4 + 3\alpha_5 + 2\alpha_6 + \alpha_7$ & \\
      $\E_8$ & $2 \alpha_1 + 3 \alpha_2 + 4 \alpha_3 + 6 \alpha_4 + 5 \alpha_5 + 4 \alpha_6 + 3 \alpha_7 + 2 \alpha_8$ & \\
      $\F_4$ & $2 \alpha_1 + 3 \alpha_2 + 4 \alpha_3 + 2 \alpha_4$ & $\alpha_1 + 2 \alpha_2 + 3 \alpha_3 + 2 \alpha_4$ \\
      $\G_2$ & $3 \alpha_1 + 2 \alpha_2$ & $2 \alpha_1 + \alpha_2$ \\
      \hline
    \end{tabular}
    \label{tab:highest_long_and_short_roots}
  \end{table}
\end{exercise}

\begin{solution}[1.3.4.3]\label{sol:1.3.4.3}
\end{solution}

\begin{exercise}[1.3.4.4]\label{exc:1.3.4.4}
  Prove that the long roots in $\G_2$ form a root system in $\E$ of type $\A_2$.
\end{exercise}

\begin{solution}[1.3.4.4]\label{sol:1.3.4.4}
\end{solution}

\begin{exercise}[1.3.4.5]\label{exc:1.3.4.5}
  In constructing $\Cl$, would it be correct to characterize $\Phi$ as the set of all vectors in $I$ of squared length 2 or 4? Explain.
\end{exercise}

\begin{solution}[1.3.4.5]\label{sol:1.3.4.5}
\end{solution}

\begin{exercise}[1.3.4.6]\label{exc:1.3.4.6}
  Prove that the map $\alpha \mapsto -\alpha$ is an automorphism of $\Phi$. Try to decide for which irreducible $\Phi$ this belongs to the Weyl group.
\end{exercise}

\begin{solution}[1.3.4.6]\label{sol:1.3.4.6}
\end{solution}

\begin{exercise}[1.3.4.7]\label{exc:1.3.4.7}
  Describe $\Aut \Phi$ when $\Phi$ is not irreducible.
\end{exercise}

\begin{solution}[1.3.4.7]\label{sol:1.3.4.7}
\end{solution}

\subsection{Abstract theory of weights}

\begin{exercise}[1.3.5.1]\label{exc:1.3.5.1}
  Let $\Phi = \Phi_1 \cup \cdots \cup \Phi_t$ be the decomposition of $\Phi$ into its irreducible components, with $\Delta=\Delta_1 \cup \cdots \cup \Delta_t$. Prove that $\Lambda$ decomposes into a direct sum $\Lambda_1 \oplus \cdots \oplus \Lambda_t$; what about $\Lambda^+$?
\end{exercise}

\begin{solution}[1.3.5.1]\label{sol:1.3.5.1}
\end{solution}

\begin{exercise}[1.3.5.2]\label{exc:1.3.5.2}
  Show by example (e.g., for $\A_2$) that $\lambda \notin \Lambda^+$, $\alpha \in \Delta$, $\lambda - \alpha \in \Lambda^+$ is possible.
\end{exercise}

\begin{solution}[1.3.5.2]\label{sol:1.3.5.2}
\end{solution}

\begin{exercise}[1.3.5.3]\label{exc:1.3.5.3}
  Verify some of the data in Table 1, e.g., for $\F_4$.
\end{exercise}

\begin{solution}[1.3.5.3]\label{sol:1.3.5.3}
\end{solution}

\begin{exercise}[1.3.5.4]\label{exc:1.3.5.4}
  Using Table 1, show that the fundamental group of $\Al$ is cyclic of order $\ell + 1$, while that of $\Dl$ is isomorphic to $\Z/4\Z$ ($\ell$ odd), or $\Z/2\Z \times \Z/2\Z$ ($\ell$ even). (It is easy to remember which is which, since $\A_3 = \D_3$.)
\end{exercise}

\begin{solution}[1.3.5.4]\label{sol:1.3.5.4}
\end{solution}

\begin{exercise}[1.3.5.5]\label{exc:1.3.5.5}
  If $\Lambda'$ is any subgroup of $\Lambda$ which includes $\Lambda_r$, prove that $\Lambda'$ is $\W$-invariant. Therefore, we obtain a homomorphism $\phi : \Aut \Phi/\W \to \Aut(\Lambda/\Lambda_r)$. Prove that $\phi$ is injective, then deduce that $-1 \in \W$ if and only if $\Lambda_r \supset 2\Lambda$ (cf. Exercise \ref{exc:1.3.4.6}). Show that $-1 \in \W$ for precisely the irreducible root systems $\A_1, \Bl, \Cl, \Dl$ ($\ell$ even), $\E_7, \E_8, \F_4, \G_2$.
\end{exercise}

\begin{solution}[1.3.5.5]\label{sol:1.3.5.5}
\end{solution}

\begin{exercise}[1.3.5.6]\label{exc:1.3.5.6}
  Prove that the roots in $\Phi$ which are dominant weights are precisely the highest long root and (if two root lengths occur) the highest short root (cf. (10.4) and Exercise \ref{exc:1.3.2.11}), when $\Phi$ is irreducible.
\end{exercise}

\begin{solution}[1.3.5.6]\label{sol:1.3.5.6}
\end{solution}

\begin{exercise}[1.3.5.7]\label{exc:1.3.5.7}
  If $\epsilon_1, \cdots, \epsilon_{\ell}$ is an \emph{obtuse} basis of the euclidean space $\E$ (i.e., all $(\epsilon_i, \epsilon_j) \leq 0$ for $i \neq j$ ), prove that the dual basis is acute (i.e., all $\left(\epsilon_i^*, \epsilon_j^*\right) \geq 0$ for $i \neq j$). [Reduce to the case $\ell = 2$.]
\end{exercise}

\begin{solution}[1.3.5.7]\label{sol:1.3.5.7}
\end{solution}

\begin{exercise}[1.3.5.8]\label{exc:1.3.5.8}
  Let $\Phi$ be irreducible. Without using the data in Table 1, prove that each $\lambda_i$ is of the form $\sum_j q_{ij} \alpha_j$, where all $q_{ij}$ are positive rational numbers. [Deduce from Exercise \ref{exc:1.3.5.7} that all $q_{ij}$ are nonnegative. From $\left(\lambda_i, \lambda_i\right) > 0$ obtain $q_{ii} > 0$. Then show that if $q_{ij} > 0$ and $\left(\alpha_j, \alpha_k\right) < 0$, then $q_{ik} > 0$.]
\end{exercise}

\begin{solution}[1.3.5.8]\label{sol:1.3.5.8}
\end{solution}

\begin{exercise}[1.3.5.9]\label{exc:1.3.5.9}
  Let $\lambda \in \Lambda^+$. Prove that $\sigma(\lambda+\delta)-\delta$ is dominant only for $\sigma = 1$.
\end{exercise}

\begin{solution}[1.3.5.9]\label{sol:1.3.5.9}
\end{solution}

\begin{exercise}[1.3.5.10]\label{exc:1.3.5.10}
  If $\lambda \in \Lambda^+$, prove that the set $\Pi$ consisting of all dominant weights $\mu \prec \lambda$ and their $\W$-conjugates is saturated, as asserted in (13.4).
\end{exercise}

\begin{solution}[1.3.5.10]\label{sol:1.3.5.10}
\end{solution}

\begin{exercise}[1.3.5.11]\label{exc:1.3.5.11}
  Prove that each subset of $\Lambda$ is contained in a unique smallest saturated set, which is finite if the subset in question is finite.
\end{exercise}

\begin{solution}[1.3.5.11]\label{sol:1.3.5.11}
\end{solution}

\begin{exercise}[1.3.5.12]\label{exc:1.3.5.12}
  For the root system of type $\A_2$, write down the effect of each element of the Weyl group on each of $\lambda_1, \lambda_2$. Using this data, determine which weights belong to the saturated set having highest weight $\lambda_1 + 3\lambda_2$. Do the same for type $\G_2$ and highest weight $\lambda_1 + 2\lambda_2$.
\end{exercise}

\begin{solution}[1.3.5.12]\label{sol:1.3.5.12}
\end{solution}

\begin{exercise}[1.3.5.13]\label{exc:1.3.5.13}
  Call $\lambda \in \Lambda^+$ \textbf{minimal} if $\mu \in \Lambda^+$, $\mu \prec \lambda$ implies that $\mu = \lambda$. Show that each coset of $\Lambda_r$ in $\Lambda$ contains precisely one minimal $\lambda$. Prove that $\lambda$ is minimal if and only if the $\W$-orbit of $\lambda$ is saturated (with highest weight $\lambda$), if and only if $\lambda \in \Lambda^+$ and $\bra{\lambda, \alpha} = 0, 1, -1$ for all roots $\alpha$. Determine (using Table 1) the nonzero minimal $\lambda$ for each irreducible $\Phi$, as follows:
  \[%
    \begin{aligned}
      &\Al: \lambda_1, \cdots, \lambda_l \\
      &\Bl: \lambda_\ell \\
      &\Cl: \lambda_1 \\
      &\Dl: \lambda_1, \lambda_{\ell-1}, \lambda_\ell \\
      &\E_6: \lambda_1, \lambda_6 \\
      &\E_7: \lambda_7
    \end{aligned}
  .\]%
\end{exercise}

\begin{solution}[1.3.5.13]\label{sol:1.3.5.13}
\end{solution}
