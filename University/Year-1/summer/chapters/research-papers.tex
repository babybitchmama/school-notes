\begin{exercise}[1]
  Compute the value of a bigon at $q = 1$ or at general $q$.
\end{exercise}

\begin{solution}[1]
\end{solution}

\begin{exercise}[2]
  Look at (2.9). Can you find associativity and coassociativity inside? Use only these relations and (2.4) to prove (2.9).
\end{exercise}

\begin{solution}[2]
\end{solution}

\begin{exercise}[3]
  Write down what (2.10) means explicitly for some small values of k, l, r, s, until you get a feeling for how it works. You'll definitely want an exampel where k-l+r-s is at least 2 eventually. Then try to verify it using vectors for small values.
\end{exercise}

\begin{solution}[3]
\end{solution}

\begin{exercise}[4]
  Try to prove Lemma 2.9 from \href{https://arxiv.org/pdf/1510.06840}{Light Ladders and Clasp Conjectures}
\end{exercise}

\begin{solution}[4]
\end{solution}

\begin{exercise}[5]
  Remember how for the Temperley-Lieb algebra you described the "Crossing" $v \otimes w \mapsto w \otimes v$ as a linear combination of other maps. Let's do this again, but with webs this time. You're going to have to use $q = 1$ do this exercise, so forget about the q-deformation.

  Consider the map $\Lambda^1 V \otimes \Lambda^2 V \to \Lambda^2 V \otimes \Lambda^1 V$ which just swaps the tensor factors. This is a linear combination of:
  \begin{enumerate}
    \item The web which merges $1,2$ into $3$ and then splits $3$ into $2,1$.

    \item The web which splits $1,2$ into $1,1,1$ and then merges $1,1,1$ into $2,1$.
  \end{enumerate}
  Find the linear combo.
\end{exercise}

\begin{solution}[5(i)]
\end{solution}

\begin{solution}[5(ii)]
\end{solution}

\begin{exercise}[6]
  Consider the map $\Lambda^2 V \otimes \Lambda^2 V \to \Lambda^2 V \otimes \Lambda^2 V$ which just swaps the tensor factors. This is a linear combination of:
  \begin{enumerate}
    \item the web which merges $2,2$ into $4$ and then splits $4$ into $2,2$.

    \item the web which splits $2,2$ into $2,1,1$ and then merges $2,1,1$ into $3,1$ and then splits back to $2,1,1$ and marges back to $2,2$.

    \item the identity of $2,2$.
  \end{enumerate}
  Find the linear combo.
\end{exercise}

\begin{solution}[6(i)]
\end{solution}

\begin{solution}[6(ii)]
\end{solution}

\begin{solution}[6(iii)]
\end{solution}
