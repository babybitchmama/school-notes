\documentclass[notitlepage]{report}

\title{Several-Variab Calc II}
\date{January 6, 2025}

\usepackage{hyperref}
\usepackage{derivative}
\usepackage{xcolor}
\usepackage{amsmath}
\usepackage{amsfonts}
\usepackage{mathtools}
\usepackage{amsthm}
\usepackage{amssymb}
\usepackage{mathrsfs}
\usepackage{breqn}
\usepackage{ifthen}
\usepackage{xifthen}
\usepackage{geometry}
\usepackage{xfrac}
\geometry{
  top=1in,
  bottom=1.5in,
  right=1in,
  left=1in,
}

\renewcommand\a{\mathbf{a}}
\renewcommand\b{\mathbf{b}}
\renewcommand\c{\mathbf{c}}
\renewcommand\d{\mathbf{d}}
\newcommand\e{\mathbf{e}}
\newcommand\f{\mathbf{f}}
\newcommand\FF{\mathbf{F}}
\newcommand\g{\mathbf{g}}
\newcommand\n{\mathbf{n}}
\newcommand\p{\mathbf{p}}
\renewcommand\r{\mathbf{r}}
\newcommand\Ta{\mathbf{T}}
\renewcommand\u{\mathbf{u}}
\renewcommand\v{\mathbf{v}}
\newcommand\w{\mathbf{w}}
\newcommand\x{\mathbf{x}}
\newcommand\y{\mathbf{y}}
\newcommand\z{\mathbf{z}}
\newcommand\zero{\mathbf{0}}
\newcommand\hb{\hbar}

% Hat vectors
\newcommand\ah{\mathbf{a}}
\newcommand\bh{\mathbf{b}}
\newcommand\ch{\mathbf{c}}
\renewcommand\dh{\mathbf{d}}
\newcommand\eh{\mathbf{e}}
\newcommand\ph{\mathbf{p}}
\newcommand\uh{\mathbf{u}}
\newcommand\vh{\mathbf{v}}
\newcommand\wh{\mathbf{w}}
\newcommand\xh{\mathbf{x}}
\newcommand\yh{\mathbf{y}}
\newcommand\zh{\mathbf{z}}

% Unit vectors
\newcommand\ui{\boldsymbol{\imath}}
\newcommand\uj{\boldsymbol{\jmath}}
\newcommand\uk{\mathbf{k}}

% Matrix groups
\newcommand\GL{\operatorname{GL}}
\newcommand\Or{\operatorname{O}}
\newcommand\PGL{\operatorname{PGL}}
\newcommand\PSL{\operatorname{PSL}}
\newcommand\PSO{\operatorname{PSO}}
\newcommand\PSU{\operatorname{PSU}}
\newcommand\SL{\operatorname{SL}}
\newcommand\SO{\operatorname{SO}}
\newcommand\Spin{\operatorname{Spin}}
\newcommand\Sp{\operatorname{Sp}}
\newcommand\SU{\operatorname{SU}}
\newcommand\U{\operatorname{U}}
\newcommand\Mat{\operatorname{Mat}}
\renewcommand\P{\operatorname{P}}

% Random functions
\newcommand\sech{\operatorname{sech}}
\newcommand\fff{\mathrm{I}}
\newcommand\sff{\mathrm{II}}

% Lie algebras
\newcommand\ec{\mathrm{e}}
\newcommand\gl{\mathrm{gl}}
\newcommand\so{\mathrm{so}}
\newcommand\su{\mathrm{su}}
\newcommand\ggl{\mathfrak{g}}
\newcommand\hl{\mathfrak{h}}

% Subspaces
\newcommand\B{\mathbb{B}}
\newcommand\C{\mathbb{C}}
\newcommand\D{\mathbb{D}}
\newcommand\E{\mathbb{E}}
\newcommand\F{\mathbb{F}}
\newcommand\I{\mathbb{I}}
\newcommand\N{\mathbb{N}}
\newcommand\Q{\mathbb{Q}}
\newcommand\R{\mathbb{R}}
\newcommand\Z{\mathbb{Z}}

% Math operators
\let\div\relax
\let\Re\relax
\let\Im\relax

% Operators
% \DeclareMathOperator\arg{arg}
\DeclareMathOperator\Arg{Arg}
\DeclareMathOperator\curl{curl}
\DeclareMathOperator\div{div}
\DeclareMathOperator\Col{Col}
\DeclareMathOperator\Null{Null}
\DeclareMathOperator\Range{Range}
\DeclareMathOperator\Ker{Ker}
\DeclareMathOperator\Tr{Tr}
\DeclareMathOperator\Rank{Rank}
\DeclareMathOperator\proj{proj}
\DeclareMathOperator\comp{comp}
\DeclareMathOperator\Sspan{Span}
\DeclareMathOperator\Re{Re}
\DeclareMathOperator*\Res{Res}
\DeclareMathOperator\Im{Im}
\DeclareMathOperator\ad{ad}
\DeclareMathOperator\Ad{Ad}
\DeclareMathOperator\Log{Log}
\DeclareMathOperator\Hom{Hom}

\newcommand\aand{\quad\text{and}\quad}
\newcommand\oor{\quad\text{or}\quad}
\renewcommand\and{\text{and}}
\newcommand\qtq[1]{\quad\text{#1}\quad}

% Fancy letters
\newcommand\BB{\mathcal{B}}
\newcommand\DD{\mathcal{D}}

\newenvironment{exercise}[1][-1]{\noindent\textbf{Exercise #1}.}

\newenvironment{solution}[1][]{%
  \begin{proof}[Solution\ifthenelse{\equal{#1}{}}{}{~to~#1}]
}{%
  \end{proof}
}

\begin{document}
  \chapter{Introduction to Soergel Bimodules}

  \begin{exercise}[7.8]
    Show that the axioms of a 2-category imply the following equalities.
  \end{exercise}

  \begin{solution}[7.8]
  \end{solution}

  \begin{exercise}[7.16]
    We can view the algebra $A = \R[x]/(x^2)$ as an object in the monoidal category of $\R$-vector spaces. Let $\cap : A \otimes A \to \R$ denote the linear map which sends $f \otimes g$ to the coefficient of $x$ in $fg$. Let $\cup : \R \to A \otimes A$ denote the map which sends $1$ to $x \otimes 1 + 1 \otimes x$.

    \begin{enumerate}
      \item We wish to encode these maps diagrammatically, drawing $\cap$ as a cap and $\cup$ as a cup. Justify this diagrammatic notation, by checking the isotopy relations.

      \item Draw a sequence of nested circles, as in an archery target. Evaluate this morphism.
    \end{enumerate}
  \end{exercise}

  \begin{solution}[7.16(i)]
  \end{solution}

  \begin{solution}[7.16(i)]
  \end{solution}

  \begin{exercise}[7.17]
    This question is about the Temperley–Lieb category.
    \begin{enumerate}
      \item Finish the proof that the isotopy relation holds in vector spaces.

      \item There is a map $V \otimes V \to V \otimes V$ which sends $x \otimes y \mapsto y \otimes x$. Draw this as an element of the Temperley–Lieb category (a linear combination of diagrams).

      \item Find an endomorphism of $2$ strands which is killed by placing a cap on top. Can you find one which is an idempotent? Also find an endomorphism killed by putting a cup on bottom.

      \item (Harder) Find an idempotent endomorphism of $3$ strands which is killed by a cap on top (for either of the two placements of the cap). \end{enumerate}
  \end{exercise}

  \begin{solution}[7.17(i)]
  \end{solution}

  \begin{solution}[7.17(ii)]
  \end{solution}

  \begin{solution}[7.17(iii)]
  \end{solution}

  \begin{exercise}[7.19]
    One can think about the right mate and the left mate as ``twisting'' or ``rotating'' $\alpha$ by $180^{\circ}$ to the right or to the left. Visualize what it would mean to twist $\alpha$ by $360^{\circ}$ to the right, yielding another $2$-morphism $\alpha^{\lor\lor} : E \to F$. Verify that $^{\lor}\alpha = \alpha^{\lor}$, if and only if $\alpha = \alpha^{\lor\lor}$. Thus cyclicity is the same as ``$360$ degree rotation invariance,'' which one might expect from any planar picture.
  \end{exercise}

  \begin{solution}[7.19]
  \end{solution}

  \begin{exercise}[7.20]
    Suppose that $B$ is an object in a monoidal category with biadjoints, and $\Phi : B \otimes B \otimes B \to \mathbb{1}$ is a cyclic morphism. What should it mean to ``rotate'' $\Phi$ by $120^{\circ}$? Suppose that $\Hom(B \otimes B \otimes B, \mathbb{1})$ is one-dimensional over $\C$. What can you say about the $120^{\circ}$ rotation of $\Phi$, vis a vis $\Phi$? What if $\Hom(B \otimes B \otimes B, \mathbb{1})$ is one-dimensional over $\R$?
  \end{exercise}

  \begin{solution}[7.20]
  \end{solution}

  \begin{exercise}[9.25]
  \end{exercise}

  \begin{solution}[9.25]
  \end{solution}

  \begin{exercise}[9.26]
  \end{exercise}

  \begin{solution}[9.26]
  \end{solution}

  \begin{exercise}[9.27]
  \end{exercise}

  \begin{solution}[9.27]
  \end{solution}

  \begin{exercise}[9.28]
  \end{exercise}

  \begin{solution}[9.28]
  \end{solution}

  \chapter{Research Papers}

  \begin{exercise}[1]
    Compute the value of a bigon at $q = 1$ or at general $q$.
  \end{exercise}

  \begin{solution}[1]
  \end{solution}

  \begin{exercise}[2]
    Look at (2.9). Can you find associativity and coassociativity inside? Use only these relations and (2.4) to prove (2.9).
  \end{exercise}

  \begin{solution}[2]
  \end{solution}

  \begin{exercise}[3]
    Write down what (2.10) means explicitly for some small values of k, l, r, s, until you get a feeling for how it works. You'll definitely want an exampel where k-l+r-s is at least 2 eventually. Then try to verify it using vectors for small values.
  \end{exercise}

  \begin{solution}[3]
  \end{solution}

  \begin{exercise}[4]
    Try to prove Lemma 2.9 from \href{https://arxiv.org/pdf/1510.06840}{Light Ladders and Clasp Conjectures}
  \end{exercise}

  \begin{solution}[4]
  \end{solution}

  \begin{exercise}[5]
    Remember how for the Temperley-Lieb algebra you described the "Crossing" $v \otimes w \mapsto w \otimes v$ as a linear combination of other maps. Let's do this again, but with webs this time. You're going to have to use $q = 1$ do this exercise, so forget about the q-deformation.

    Consider the map $\Lambda^1 V \otimes \Lambda^2 V \to \Lambda^2 V \otimes \Lambda^1 V$ which just swaps the tensor factors. This is a linear combination of:
    \begin{enumerate}
      \item The web which merges $1,2$ into $3$ and then splits $3$ into $2,1$.

      \item The web which splits $1,2$ into $1,1,1$ and then merges $1,1,1$ into $2,1$.
    \end{enumerate}
    Find the linear combo.
  \end{exercise}

  \begin{solution}[5(i)]
  \end{solution}

  \begin{solution}[5(ii)]
  \end{solution}

  \begin{exercise}[6]
    Consider the map $\Lambda^2 V \otimes \Lambda^2 V \to \Lambda^2 V \otimes \Lambda^2 V$ which just swaps the tensor factors. This is a linear combination of:
    \begin{enumerate}
      \item the web which merges $2,2$ into $4$ and then splits $4$ into $2,2$.

      \item the web which splits $2,2$ into $2,1,1$ and then merges $2,1,1$ into $3,1$ and then splits back to $2,1,1$ and marges back to $2,2$.

      \item the identity of $2,2$.
    \end{enumerate}
    Find the linear combo.
  \end{exercise}

  \begin{solution}[6(i)]
  \end{solution}

  \begin{solution}[6(ii)]
  \end{solution}

  \begin{solution}[6(iii)]
  \end{solution}

  \chapter{Misc}

  \begin{exercise}[1]
    Find a formula for the product $[n][3]$ when $n \ge 3$ and $[n][4]$ when $n \ge 4$. Generalize this.
  \end{exercise}

  \begin{solution}[1]
  \end{solution}

  \begin{exercise}[2]
    What is $[n][n] - [n + 1][n - 1]$?
  \end{exercise}

  \begin{solution}[2]
  \end{solution}

  \begin{exercise}[3]
    What is $[n][k] - [n + 1][k - 1]$ for $k < n$?
  \end{exercise}

  \begin{solution}[3]
  \end{solution}
\end{document}
