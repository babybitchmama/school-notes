%%%%%%%%%%%%%%%%%%%%%%%%%%%%%%%%%%%%%%%%%%%%%%%%%%%%%%%%%%%%%%%%%%%%%%%%%%%%%%%%
%                                                                              %
%                              Required Packages                               %
%                                                                              %
%%%%%%%%%%%%%%%%%%%%%%%%%%%%%%%%%%%%%%%%%%%%%%%%%%%%%%%%%%%%%%%%%%%%%%%%%%%%%%%%

% Required for creating documents
\usepackage[utf8]{inputenc}
\usepackage[T1]{fontenc}
\RequirePackage{etex}

\usepackage{etoolbox}

% Required math packages
\usepackage{amsmath}
\usepackage{amsfonts}
\usepackage{mathtools}
\usepackage{amsthm}
\usepackage{amssymb}
\usepackage{mathrsfs}
\usepackage{breqn}

\usepackage[usenames,dvipsnames,pdftex]{color} % Required for nicer colors
\usepackage{hyperref} % Required for hyperlinks
\usepackage{graphicx} % Required for including images
\usepackage{enumitem} % Required for customizing lists
\usepackage{float} % Required for positioning figures and tables
\usepackage{array} % Required for customizing tables
\usepackage{cancel} % Required for \cancel
\usepackage{derivative} % Required for \odv and \pdv
\usepackage{geometry} % Required for customizing page layout
\usepackage{textgreek} % Required for greek letters in text mode
\usepackage{multirow} % Required for multirow in tables
\usepackage{nicematrix} % Required for better matrices
\usepackage{cellspace} % Required for better spacing in tables
\usepackage{xfrac} % Required for extra fraction options
\usepackage{xifthen} % Required for if-then-else statements
\usepackage{fancyhdr} % Required for customizing headers and footers
\usepackage{import} % Required for importing pdf_tex files
\usepackage[font=bf]{caption} % Required for customizing captions
\usepackage{subcaption} % Required for creating subfigures
\usepackage{siunitx} % Required for SI units
\usepackage{titletoc} % Required for customizing table of contents
\usepackage[most,many,breakable]{tcolorbox} % Required for creating boxes

% Required for drawing figures
\usepackage{tikz}
\usepackage{pgffor}
\usepackage{tkz-euclide}
\usepackage{tikz-cd}
\usepackage{tikz-3dplot}

% Required for creating plots
\usepackage{pgfplots}
\usepackage{pgfplotstable}

\makeatletter


%%%%%%%%%%%%%%%%%%%%%%%%%%%%%%%%%%%%%%%%%%%%%%%%%%%%%%%%%%%%%%%%%%%%%%%%%%%%%%%%
%                                                                              %
%                                Basic Settings                                %
%                                                                              %
%%%%%%%%%%%%%%%%%%%%%%%%%%%%%%%%%%%%%%%%%%%%%%%%%%%%%%%%%%%%%%%%%%%%%%%%%%%%%%%%

\setcounter{tocdepth}{2}
\allowdisplaybreaks

\ifx\nauthor\undefined
  \def\nauthor{Hashem A. Damrah}
\else
\fi

\newcommand\globalcolor[1]{%
  \color{#1}\global\let\default@color\current@color
}

% Symbols
\let\oldlimsymbol\lim\renewcommand\lim{\displaystyle\oldlimsymbol}
\let\oldsumsymbol\sum\renewcommand\sum{\displaystyle\oldsumsymbol}
\let\oldlimsupsymbol\limsup\renewcommand\limsup{\displaystyle\oldlimsupsymbol}
\let\oldliminfsymbol\liminf\renewcommand\liminf{\displaystyle\oldliminfsymbol}
\let\to\rightarrow
\let\implies\Rightarrow
\let\impliedby\Leftarrow
\let\iff\Leftrightarrow
\let\epsilon\varepsilon
\let\phi\varphi

% Geometry
\geometry{
  top=1in,
  bottom=1.5in,
  right=1in,
  left=1in,
}

% Tables
\newcolumntype{C}{>{\Centering\arraybackslash}X}
\setlength{\tabcolsep}{5pt}
\renewcommand\arraystretch{1.5}
\renewcommand\thetable{\Roman{table}}
\captionsetup[figure]{font=small}
\captionsetup{justification=centering}
\setlength\cellspacetoplimit{6pt}
\setlength\cellspacebottomlimit{6pt}

% Lists
\renewcommand{\labelitemi}{--}
\renewcommand{\labelitemii}{$\circ$}
\renewcommand{\labelenumi}{\textnormal{(\roman{*})}}

% Modify Links Color
\hypersetup{
  colorlinks,
  linkcolor=black!90,
  citecolor=black,
  urlcolor=cyan!70!black,
}

%%%%%%%%%%
%  TikZ  %
%%%%%%%%%%

\usetikzlibrary{
  shadings,
  intersections,
  angles,
  quotes,
  calc,
  positioning,
  3d,
  perspective,
  arrows,
  arrows.meta,
  patterns,
  decorations.markings,
  bending,
  decorations.pathreplacing,
  calligraphy,
  backgrounds,
}

\tikzoption{canvas is xy plane at z}[]{%
	\def\tikz@plane@origin{\pgfpointxyz{0}{0}{#1}}%
	\def\tikz@plane@x{\pgfpointxyz{1}{0}{#1}}%
	\def\tikz@plane@y{\pgfpointxyz{0}{1}{#1}}%
	\tikz@canvas@is@plane}

\usetikzlibrary{shapes.arrows}
\newcommand\graphslopefield{
  \pgfmathsetmacro{\hx}{(\xmax-\xmin)/\nx}
  \pgfmathsetmacro{\hy}{(\ymax-\ymin)/\ny}
  \foreach \i in {0,...,\nx}
  \foreach \j in {0,...,\ny}{
    \pgfmathsetmacro{\yprime}{f({\xmin+\i*\hx},{\ymin+\j*\hy})}
    \draw[black,shift={({\xmin+\i*\hx},{\ymin+\j*\hy})}]
    (0,0)--($(0,0)!2mm!(.1,.1*\yprime)$);
  }

  \draw[->] (\xmin-.5,0)--(\xmax+.5,0) node[below right] {$x$};
  \draw[->] (0,\ymin-.5)--(0,\ymax+.5) node[above left] {$y$};
}

\tikzset{derivative/.style={color=gray,mark=none,line width=0.5pt,solid}}
\tikzset{asymptote/.style={color=gray,mark=none,line width=1pt,<->,dashed}}
\tikzset{soldot/.style={color=black,fill=black,only marks,mark=*}}
\tikzset{holdot/.style={color=black,fill=white,only marks,mark=*}}

\tikzset{>=stealth}
\tikzset{->-/.style={decoration={markings,mark=at position .5 with {\arrow{>}}},postaction={decorate}}}

%%%%%%%%%%%%%%
%  PgfPlots  %
%%%%%%%%%%%%%%

\pgfplotsset{width=7cm,compat=1.8}
\pgfplotsset{compat=newest}

\usepgfplotslibrary{patchplots}
\usepgfplotslibrary{fillbetween}
\usetikzlibrary{intersections}

\pgfplotsset{plot/.style={color=red,mark=none,line width=1pt,<->,solid}}
\pgfplotsset{asymptote/.style={color=gray,mark=none,line width=1pt,<->,dashed}}
\pgfplotsset{soldot/.style={color=red,only marks,mark=*}}
\pgfplotsset{holdot/.style={color=red,fill=white,only marks,mark=*}}
\pgfplotsset{blankgraph/.style={xmin=-10,xmax=10,ymin=-10,ymax=10,axis line style= {-, draw opacity=0 },axis lines=box,major tick length=0mm,xtick={-10,-9,...,10},ytick={-10,-9,...,10},grid=major,yticklabels={,,},xticklabels={,,},minor xtick=,minor ytick=,xlabel={},ylabel={},width=0.75\textwidth,grid style={solid,gray!40}}}

\pgfplotscreateplotcyclelist{stylelist}{
  plot \\
}

\def\axisdefaultwidth{175pt}
\def\axisdefaultheight{\axisdefaultwidth}

\pgfplotsset{every axis/.append style={
    axis x line=middle,
    axis y line=middle,
    axis line style={<->},
    xlabel={$x$},
    ylabel={$y$},
    xmin=-7,xmax=7,
    ymin=-7,ymax=7,
    yticklabel style={inner sep=0.333ex},
    minor xtick={-7,-6,...,7},
    minor ytick={-7,-6,...,7},
    scale only axis,
    cycle list name=stylelist,
    tick label style={font=\footnotesize},
    legend cell align=left,
    grid=minor,
    grid style={solid,gray!40},
    try min ticks=6,
  },
  framed/.style={axis background/.style={draw=gray}}
}

\pgfplotsset{axis background/.style={draw=gray}}


%%%%%%%%%%%%%%%%%%%%%%%%%%%%%%%%%%%%%%%%%%%%%%%%%%%%%%%%%%%%%%%%%%%%%%%%%%%%%%%%
%                                                                              %
%                           School Specific Commands                           %
%                                                                              %
%%%%%%%%%%%%%%%%%%%%%%%%%%%%%%%%%%%%%%%%%%%%%%%%%%%%%%%%%%%%%%%%%%%%%%%%%%%%%%%%

%%%%%%%%%%%%%%%%%%%%%%
%  Helpful Commands  %
%%%%%%%%%%%%%%%%%%%%%%

\newcommand\resetcounters{
  \setcounter{section}{0}
  \setcounter{subsection}{0}
  \setcounter{subsubsection}{0}
  \setcounter{paragraph}{0}
  \setcounter{subparagraph}{0}
}

%%%%%%%%%%%%%%%%%%%%%%%%%%%%%
%  Lecture/Chapter Command  %
%%%%%%%%%%%%%%%%%%%%%%%%%%%%%

\def\notenum{}
\def\@note{}%
\newcommand\lecture[3]{
  \ifthenelse{\isempty{#3}}{%
    \def\@note{Lecture #1}%
  }{%
    \def\@note{Lecture #1: #3}%
  }%
  \section*{\@note\hfill{\small\textnormal{#2}}}
}

\patchcmd{\maketitle}
  {\end{titlepage}}
  {\thispagestyle{titlepagestyle}\end{titlepage}}
  {}{}

\fancypagestyle{titlepagestyle}{%
   \fancyhf{}%
    \fancyfoot[C]{%
      \textit{For more notes like this, visit \href{\othernotes}{\othernotes}}. \\%
      \noindent\makebox[\linewidth]{\rule{0.8\textwidth}{0.4pt}}
      \nterm: \nyear, \\%
      Last Update: \today, \\%
      \faculty, \location.%
    }%
    \renewcommand{\headrulewidth}{0 mm}
}%

% Intro
\pagestyle{plain}
\newcommand\createintro{
  \author{Based on lectures by \nlecturer \\\small Notes taken by \nauthor}

  \pagenumbering{roman}
  \renewcommand\headrulewidth{0pt}
  \maketitle
  \thispagestyle{titlepagestyle}

  \begin{center}
    \begin{minipage}[c]{0.9\textwidth}
      \centering\footnotesize \textit{Disclaimer:} This document will inevitably
      contain some mistakes--both simple typos and legitimate errors. Keep in
      mind that these are the notes of an undergraduate student in the process
      of learning the material himself, so take what you read with a grain of
      salt. If you find mistakes and feel like telling me, I will be grateful
      and happy to hear from you, even for the most trivial of errors. You can
      reach me by email, in English, Arabic, Hebrew, or Spanish at
      \href{mailto:singularisartt@gmail.com}{singularisartt@gmail.com}.
    \end{minipage}
  \end{center}

  \begingroup
  \IfFileExists{./intro.tex}{
    \setlength{\parindent}{1cm}
    Lecture notes from the course \MyTitle, given by professor Victor Ostrik at the \faculty~at \location~in the academic year \academicyear, during the \term term. This course covers symbolic logic, basic set theory, analyzing functions and their properties, modular arithmetic, counting and other problems in discrete mathematics, induction, and convergence of sequences and continuity of functions. Credit for the material in these notes is due to professor Victor, while the structure is loosely taken from the \href{https://www.amazon.com/Mathematical-Reasoning-Writing-Proof-2nd/dp/0131877186}{Mathematical Reasoning: Writing and Proof} textbook. The credit for the typesetting is my own.

\textit{Disclaimer:} This document will inevitably contain some mistakes--both simple typos and legitimate errors. Keep in mind that these are the notes of an undergraduate student in the process of learning the material himself, so take what you read with a grain of salt. If you find mistakes and feel like telling me, I will be grateful and happy to hear from you, even for the most trivial of errors. You can reach me by email, in English, Arabic, Hebrew, or Spanish at \href{mailto:singularisartt@gmail.com}{singularisartt@gmail.com}.

  }{}
  \endgroup

  \newpage

  \newgeometry{
    top=1in,
    bottom=1in,
    right=1in,
    left=1in,
  }

  \tableofcontents

  \pagestyle{fancy}
  \pagenumbering{arabic}
  \setcounter{page}{1}

  \renewcommand\headrulewidth{0.4pt}
  \fancyhead[R]{\@note}
  \fancyhead[L]{\nauthor}
  \fancyfoot[C]{\thepage}
}

%%%%%%%%%%%%%%%%%%%%%
%  Random Commands  %
%%%%%%%%%%%%%%%%%%%%%

% Circle
\newcommand*\circled[1]{
  \tikz[baseline=(char.base)] {
    \node[shape=circle,draw,inner sep=1pt] (char) {#1};
  }
}

% Import Figures
\newcommand\incimg[2][1]{%
  \includegraphics[width=#1\columnwidth]{figures/#2}%
}

\newcommand\incfig[2][1]{
  \def\svgwidth{#1\columnwidth}
  \import{figures}{#2.pdf_tex}
}

% For diagonal strikeout in red
\newcommand\rcancel[1]{\renewcommand\CancelColor{\color{red}}\cancel{#1}}

% Logic
\renewcommand\nmid{\not|~}

%%%%%%%%%%%%%%%%%%
%  Math Symbols  %
%%%%%%%%%%%%%%%%%%

\let\U\relax
\let\C\relax
\let\G\relax

% Brackets
\newcommand\abs[1]{\left\lvert #1\right\rvert}
\newcommand\bket[1]{\left\lvert #1\right\rangle}
\newcommand\brak[1]{\left\langle #1 \right\rvert}
\newcommand\braket[2]{\left\langle #1\middle\vert #2 \right\rangle}
\newcommand\bra{\langle}
\newcommand\ket{\rangle}

% For differentials
\newcommand\dd[1]{\textrm{d}#1}
\newcommand\dA{\textrm{d}A}
\newcommand\dm{\textrm{d}m}
\newcommand\dt{\textrm{d}t}
\newcommand\dr{\textrm{d}r}
\newcommand\dT{\textrm{d}T}
\newcommand\du{\textrm{d}u}
\newcommand\dv{\textrm{d}v}
\newcommand\dx{\textrm{d}x}
\newcommand\dy{\textrm{d}y}
\newcommand\dz{\textrm{d}z}

% Helpful text in math
\newcommand\aand{\quad\text{and}\quad}
\newcommand\oor{\quad\text{or}\quad}
\renewcommand\and{\text{and}}
\newcommand\qtq[1]{\quad\textrm{#1}\quad}
\newcommand\Col{\textrm{Col}}
\newcommand\Nul{\textrm{Null}}
\newcommand\Range{\textrm{Range}}
\newcommand\Ker{\textrm{Ker}}
\newcommand\Tr{\textrm{Tr}}
\newcommand\Rank{\textrm{Rank}}
\newcommand\proj{\text{proj}}
\newcommand\comp{\text{comp}}
\newcommand\Sspan{\text{Span}}
\renewcommand\Re{\text{Re}}
\renewcommand\Im{\text{Im}}

% Vectors
\renewcommand\a{\mathbf{a}}
\renewcommand\b{\mathbf{b}}
\renewcommand\c{\mathbf{c}}
\renewcommand\d{\mathbf{d}}
\newcommand\e{\mathbf{e}}
\newcommand\f{\mathbf{f}}
\newcommand\g{\mathbf{g}}
\newcommand\n{\mathbf{n}}
\newcommand\p{\mathbf{p}}
\renewcommand\r{\mathbf{r}}
\newcommand\Ta{\mathbf{T}}
\renewcommand\u{\mathbf{u}}
\renewcommand\v{\mathbf{v}}
\newcommand\w{\mathbf{w}}
\newcommand\x{\mathbf{x}}
\newcommand\y{\mathbf{y}}
\newcommand\z{\mathbf{z}}
\newcommand\zero{\mathbf{0}}

% Hat vectors
\newcommand\ah{\mathbf{a}}
\newcommand\bh{\mathbf{b}}
\newcommand\ch{\mathbf{c}}
\renewcommand\dh{\mathbf{d}}
\newcommand\eh{\mathbf{e}}
\newcommand\ph{\mathbf{p}}
\newcommand\uh{\mathbf{u}}
\newcommand\vh{\mathbf{v}}
\newcommand\wh{\mathbf{w}}
\newcommand\xh{\mathbf{x}}
\newcommand\yh{\mathbf{y}}
\newcommand\zh{\mathbf{z}}

% Unit vectors
\newcommand\ui{\mathbf{\imath}}
\newcommand\uj{\mathbf{\jmath}}
\newcommand\uk{\mathbf{k}}

% Matrix groups
\newcommand\GL{\mathrm{GL}}
\newcommand\Or{\mathrm{O}}
\newcommand\PGL{\mathrm{PGL}}
\newcommand\PSL{\mathrm{PSL}}
\newcommand\PSO{\mathrm{PSO}}
\newcommand\PSU{\mathrm{PSU}}
\newcommand\SL{\mathrm{SL}}
\newcommand\SO{\mathrm{SO}}
\newcommand\Spin{\mathrm{Spin}}
\newcommand\Sp{\mathrm{Sp}}
\newcommand\SU{\mathrm{SU}}
\newcommand\U{\mathrm{U}}
\newcommand\Mat{\mathrm{Mat}}
\renewcommand\P{\mathbb{P}}

% Matrix algebras
\newcommand\gl{\mathfrak{gl}}
\newcommand\ort{\mathfrak{o}}
\newcommand\so{\mathfrak{so}}
\newcommand\su{\mathfrak{su}}
\newcommand\uu{\mathfrak{u}}
\renewcommand\sl{\mathfrak{sl}}

% Subspaces
\newcommand\B{\mathbb{B}}
\newcommand\C{\mathbb{C}}
\newcommand\D{\mathbb{D}}
\newcommand\E{\mathbb{E}}
\newcommand\F{\mathbb{F}}
\newcommand\I{\mathbb{I}}
\newcommand\N{\mathbb{N}}
\newcommand\Q{\mathbb{Q}}
\newcommand\R{\mathbb{R}}
\newcommand\Z{\mathbb{Z}}

% Fancy letters
\newcommand\BB{\mathcal{B}}

% Create command to box equations
\newcommand*\colorboxed{}
\def\colorboxed#1#{%
  \colorboxedAux{#1}%
}
\newcommand*\colorboxedAux[3]{%
  \begingroup
    \colorlet{cb@saved}{.}%
    \color#1{#2}%
    \boxed{%
      \color{cb@saved}%
      #3%
    }%
  \endgroup
}
\newcommand\empheq[1]{%
  \colorboxed{black}{%
    \begin{aligned}[b]
      #1
    \end{aligned}%
  }%
}


%%%%%%%%%%%%%%%%%%%%%%%%%%%%%%%%%%%%%%%%%%%%%%%%%%%%%%%%%%%%%%%%%%%%%%%%%%%%%%%%
%                                                                              %
%                                 Environments                                 %
%                                                                              %
%%%%%%%%%%%%%%%%%%%%%%%%%%%%%%%%%%%%%%%%%%%%%%%%%%%%%%%%%%%%%%%%%%%%%%%%%%%%%%%%

\theoremstyle{definition}
\newtheorem*{assumption}{Assumption}
\newtheorem*{claim}{Claim}
\newtheorem*{conjecture}{Conjecture}
\newtheorem*{definition}{Definition}
\newtheorem*{example}{Example}
\newtheorem*{notation}{Notation}
\newtheorem*{proposition}{Proposition}
\newtheorem*{question}{Question}
\newtheorem*{problem}{Problem}
\newtheorem*{rrule}{Rule}

\newtheorem*{remark}{Remark}
\newtheorem*{note}{Note}
\newtheorem*{warning}{Warning}

\theoremstyle{plain}
\newtheorem*{corollary}{Corollary}
\newtheorem*{lemma}{Lemma}
\newtheorem*{theorem}{Theorem}
\newtheorem*{worksheet}{Worksheet}

\newcommand\frmebox[1][]{%
  \begin{tcolorbox}[%
    title={\sffamily\color{black}#1},
    colback=white,
    enhanced,
    attach boxed title to top center={
      yshift=-3mm,
      yshifttext=-1mm,
    },
    boxed title style={
      size=small,
      colback=white,
      frame code={},
    },
    coltext=black,
    frame hidden,
    borderline east={0.5pt}{0pt}{black},
    borderline west={0.5pt}{0pt}{black},
    borderline north={0.5pt}{0pt}{black},
    borderline south={0.5pt}{0pt}{black},
    breakable,
    parskip=0pt,
  ]
}

\renewenvironment{frame}[1][]{%
  \frmebox[#1]%
}{%
  \end{tcolorbox}%
}

\makeatother
