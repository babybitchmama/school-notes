\renewcommand\S{\mathbb{S}}
\newcommand\M{\mathcal{M}}

\begin{problem}[1]
  Let $M$ be $\R^2$, and let $x_0 \in \R^2$ and $f \in C^\infty(M)$. Given a metric $g$ we can define a gradient flow for $f$, call this flow $\gamma_{g,x_0}(t)$, which satisfies
  \begin{gather*}
    \gamma_{g,x_0}(0) = x_0 \\
    \odv{}{t} \gamma_{g,x_0}(t) = -\nabla_g f
  .\end{gather*}
  Suppose we have two different metrics $g_1$ and $g_2$. Let
  \begin{align*}
    a_1 &= \lim_{t \to \infty} \gamma_{g_1,x_0}(t) \\
    a_2 &= \lim_{t \to \infty} \gamma_{g_2,x_0}(t)
  .\end{align*}
  Give an example of a function $f$ and two metrics such that $a_1$ and $a_2$ exist, but $a_1 \neq a_2$.
\end{problem}

\begin{solution}
  We can consider the function
  \[%
    f(x, y) = (x^2 + y^2 - 1)^2 + (x - 1)^2
  .\]%
  This function has two local minima: one at $(1, 0)$ and another at approximately $(-0.5, 0)$. Now, we can define two different metrics on $\R^2$:
  \[%
    g_1 = \begin{pmatrix}
      1 & 0 \\
      0 & 1
    \end{pmatrix}, \quad g_2 = \begin{pmatrix}
      10 & 0 \\
      0 & 1
    \end{pmatrix}
  .\]%
  The first metric $g_1$ is the standard Euclidean metric, while the second metric $g_2$ stretches the $x$-direction by a factor of 10. Now, let's consider the gradient flows starting from the point $x_0 = (0, 0)$. Under the metric $g_1$, the gradient flow will move towards the local minimum at $(1, 0)$, so we have
  \[%
    a_1 = (1, 0)
  .\]%
  On the other hand, under the metric $g_2$, the gradient flow will be influenced more heavily in the $x$-direction, causing it to move towards the local minimum at approximately $(-0.5, 0)$. Thus, we have
  \[%
    a_2 \approx (-0.5, 0)
  .\]%
  Therefore, we have constructed a function $f$ and two metrics $g_1$ and $g_2$ such that the limits of the gradient flows starting from the same point $x_0$ are different $a_1 \neq a_2$.
\end{solution}

\begin{problem}[2]
  Show that every smooth manifold admits a Riemmanian metric.
\end{problem}

\begin{solution}
  Assume that $M$ is a smooth manifold. By definition, $M$ is locally diffeomorphic to $\R^n$. Therefore, for each point $p \in M$, there exists an open neighborhood $U_p$ of $p$ and a diffeomorphism $\phi_p : U_p \to V_p \subset \R^n$. Since $\R^n$ has a standard Euclidean metric $g_{\text{std}}$, we can pull back this metric to $U_p$ via the diffeomorphism $\phi_p$. Specifically, we define a metric $g_p$ on $U_p$ by
  \[%
    g_p(X, Y) = g_{\text{std}}(D\phi_p(X), D\phi_p(Y))
  ,\]%
  for any tangent vectors $X, Y \in T_q M$ with $q \in U_p$. Now, the collection $\{U_p\}_{p \in M}$ forms an open cover of $M$. Since $M$ is a smooth manifold, it is paracompact, which means that there exists a partition of unity subordinate to this open cover. Let $\{\psi_p\}_{p \in M}$ be such a partition of unity, where each $\psi_p : M \to [0, 1]$ is a smooth function with support contained in $U_p$ and $\sum_{p \in M} \psi_p(q) = 1$ for all $q \in M$. We can now define a global Riemannian metric
  \[%
    g(X, Y) = \sum_{p \in M} \psi_p(q) g_p(X, Y)
  ,\]%
  for any tangent vectors $X, Y \in T_q M$. This metric $g$ is smooth because it is a finite sum of smooth functions (due to the local finiteness of the partition of unity) and is positive-definite since each $g_p$ is positive-definite and the weights $\psi_p(q)$ are non-negative and sum to 1. Therefore, we have constructed a Riemannian metric on the smooth manifold $M$. Thus, every smooth manifold admits a Riemannian metric.
\end{solution}

\begin{problem}[3]
  Consider the function on $L^2(\R)$
  \[%
    F(u) = \int_\R u^2 \dx
  .\]%
  Given a function $u_0 \in L^2$, explicitly write down the gradient flow for $F$ starting at $u_0$.
\end{problem}

\begin{solution}
  Let $H = L^2(\R)$ and $F(u) = \int_\R u^2 \dx$. Fix $u, v \in L^2(\R)$ and compute
  \[%
    DF(u)[v] = \odv{}{\epsilon} \bigg|_{\epsilon=0} F(u + \epsilon v) = \odv{}{\epsilon} \bigg|_{\epsilon=0} \int_\R (u + \epsilon v)^2 \dx = \int_\R 2u v \dx = 2\bra{u, v}_{L^2}
  .\]%
  Therefore, $DF(u)$ is a linear functional that sends $v \mapsto 2\bra{u, v}_{L^2}$.

  By the Riesz representation theorem, there exists a unique element $\nabla_{L^2} F(u) \in L^2(\R)$ such that
  \[%
    DF(u)[v] = \bra{\nabla_{L^2} F(u), v}_{L^2}~\text{for all}~v \in L^2(\R)
  ,\]%
  which is just simply $\nabla_{L^2} F(u) = 2u$.

  Therefore, the gradient flow starting at $u_0$ is given by the ODE
  \begin{gather*}
    \gamma_{u_0}(0) = u_0 \\
    \odv{}{t} \gamma_{u_0}(t) = -\nabla_{L^2} F(\gamma_{u_0}(t)) = -2 \gamma_{u_0}(t)
  .\end{gather*}
  This is a simple ODE with solution
  \[%
    \gamma_{u_0}(t) = u_0 e^{-2t}
  .\]%

  Now, we can verify that this is indeed the gradient flow:
  \[%
    \odv{}{t} F(u(t)) = \bra{\nabla F(u(t)), \dot u(t)}_{L^2} = -\|\nabla F(u(t))\|_{L^2}^2 = -4\|u(t)\|_{L^2}^2 = -4F(u(t)) \le 0
  ,\]%
  which shows that $F$ is decreasing along the flow.
\end{solution}

\begin{problem}[4]
  Let
  \[%
    \M = \{F : \S^1 \to \R^2, F~\text{is a smooth immersion}\}
  .\]%
  \begin{enumerate}
    \item Given $F$ in $\M$, show that there is a metric on the tangent space defined as follows
      \[%
        g_F(V, W) = \int \bra{V(\theta), W(\theta)} \left\lvert \odv{F}{\theta} \right\rvert \dd{\theta}
      ,\]%
      where $\bra{\cdot, \cdot}$ is the standard inner product on $\R^2$ (In this case, $V$, $W$ are smooth functions $\S^1 : \R^2$ so that there is a path defined by
      \[%
        \gamma(t) = F(\theta) + tV(\theta)
      ,\]%
      which gives a velocity vector
      \[%
        \odv{}{t} \gamma(0) = V(\theta)
      .\]%

    \item Show that given any path $\gamma(t)$ in $\M$, that is, a path of maps
      \[%
        F(\theta, t) \to \R^2
      ,\]%
      there is (at least for small $t$) a family of diffeomorphisms $\Phi_t : \S^1 \times (-\epsilon, \epsilon) \to \S^1$ such that
      \[%
        \odv{}{t} \dd{F}(\Phi(\theta, t), t) \perp \odv{F}{\theta}~\text{for all}~\theta~\text{when}~t = 0
      ,\]%
      and $\Phi(\theta, 0) = \theta$.

    \item Consider the arclength function on $\M$ given by
      \[%
        L(F) = \int_{\S^1} \left\lvert \odv{F}{\theta} \right\rvert \dd{\theta}
      .\]%
      Assume that $F$ is a constant speed map,
      \[%
        \left\lvert \odv{F}{\theta} \right\rvert = C~\text{for some constant}~C
      ,\]%
      and $V$ is a normal tangent vector,
      \[%
        V \perp \odv{F}{\theta}~\text{for all}~\theta
      .\]%
      Show that
      \[%
        VL(F) = -\int \bra{\odv[2]{F}{s}, V} \dd{\theta}
      ,\]%
      where $s$ is the arclength parameter: Hint: you may want to use the fact that
      \[%
        \odv{}{\theta} \bra{\odv{F}{\theta}, \odv{F}{t}} = 0 = \bra{\odv[2]{F}{\theta}, \odv{F}{t}} + \bra{\odv{F}{\theta}, \odv{}{\theta}\odv{F}{t}}
      .\]%
  \end{enumerate}
\end{problem}

\begin{solution}[(i)]
  We need to show that $g_F$ is a metric on the tangent space $T_F \M$. First, we show that $g_F$ is bilinear. Let $V_1, V_2, W \in T_F \M$ and $a, b \in \R$. Then,
  \begin{align*}
    g_F(aV_1 + bV_2, W) &= \int \bra{aV_1(\theta) + bV_2(\theta), W(\theta)} \left\lvert \odv{F}{\theta} \right\rvert \dd{\theta} \\
                        &= a \int \bra{V_1(\theta), W(\theta)} \left\lvert \odv{F}{\theta} \right\rvert \dd{\theta} + b \int \bra{V_2(\theta), W(\theta)} \left\lvert \odv{F}{\theta} \right\rvert \dd{\theta} \\
                        &= a g_F(V_1, W) + b g_F(V_2, W)
  .\end{align*}%
  Similarly, we can show linearity in the second argument.

  Next, we show symmetry. Let $V, W \in T_F \M$. Then,
  \begin{align*}
    g_F(V, W) &= \int \bra{V(\theta), W(\theta)} \left\lvert \odv{F}{\theta} \right\rvert \dd{\theta} \\
              &= \int \bra{W(\theta), V(\theta)} \left\lvert \odv{F}{\theta} \right\rvert \dd{\theta} = g_F(W, V)
  .\end{align*}%

  Finally, we show positive-definiteness. Let $V \in T_F \M$. Then,
  \[%
    g_F(V, V) = \int \bra{V(\theta), V(\theta)} \left\lvert \odv{F}{\theta} \right\rvert \dd{\theta} = \int |V(\theta)|^2 \left\lvert \odv{F}{\theta} \right\rvert \dd{\theta} \geq 0
  .\]%
  Moreover, if $g_F(V, V) = 0$, then $|V(\theta)|^2 = 0$ for all $\theta$, which implies that $V(\theta) = 0$ for all $\theta$. Thus, $V$ is the zero vector in $T_F \M$.

  Therefore, $g_F$ is a metric on the tangent space $T_F \M$.
\end{solution}

\begin{solution}[(ii)]
\end{solution}

\begin{solution}[(iii)]
  We have
  \[%
    L(F) = \int_{\S^1} \left\lvert \odv{F}{\theta} \right\rvert \dd{\theta}
  .\]%
  Therefore,
  \[%
    VL(F) = \odv{}{\epsilon} \bigg|_{\epsilon=0} L(F + \epsilon V) = \odv{}{\epsilon} \bigg|_{\epsilon=0} \int_{\S^1} \left\lvert \odv{}{\theta} (F + \epsilon V) \right\rvert \dd{\theta}
  .\]%
  Computing the derivative inside the integral, we get
  \begin{align*}
    VL(F) &= \int_{\S^1} \odv{}{\epsilon} \bigg|_{\epsilon=0} \left( \bra{\odv{F}{\theta} + \epsilon \odv{V}{\theta}, \odv{F}{\theta} + \epsilon \odv{V}{\theta}}^{1/2} \right) \dd{\theta} \\
    &= \int_{\S^1} \frac{\bra{\odv{F}{\theta}, \odv{V}{\theta}}}{\left\lvert \odv{F}{\theta} \right\rvert} \dd{\theta}
  .\end{align*}

  Since $F$ is a constant speed map, we have $\left\lvert \odv{F}{\theta} \right\rvert = C$. Thus,
  \[%
    VL(F) = \frac{1}{C} \int_{\S^1} \bra{\odv{F}{\theta}, \odv{V}{\theta}} \dd{\theta}
  .\]%

  Integrating by parts and using periodicity on $\S^1$ (so the boundary term vanishes) gives
  \begin{align*}
    VL(F) &= \frac{1}{C} \left[\bra{\odv{F}{\theta}, V} \bigg|_{\S^1} - \int_{\S^1} \bra{\odv[2]{F}{\theta}, V} \dd{\theta}\right] \\
          &= -\frac{1}{C} \int \bra{\odv[2]{F}{\theta}, V} \dd{\theta}
  .\end{align*}
  The boundary term vanishes since $\S^1$ is a closed curve, and since $\ds = C\dd{\theta}$, those cancel out. Therefore, we have the desired result
  \[%
    VL(F) = -\int \bra{\odv[2]{F}{s}, V} \ds
  .\qedhere\]%
\end{solution}
