\renewcommand\S{\mathbb{S}}

\begin{problem}
  Suppose that $M^n$ is a compact smooth $n$-manifold. Show that there exists a smooth map $F : M \to \R^k$, for some $k$ such that
  \begin{enumerate}
    \item $F$ is injective at each point,
    \item and
      \[%
        \dd{F}_p : T_p M \to T_{F(p)} \R^k
      ,\]%
      is injective at each point.
  \end{enumerate}
\end{problem}

\begin{solution}
  Since $M$ is compact, it emits a finite atlas, say $\{(U_i, \phi_i)\}_{i=1}^r$, where each $\phi_i : U_i \to \R^n$ is a smooth chart.

  Let $\{\rho\}^r_i$ be a smooth partition of unity subordinate to this cover. Define the map $G : M \to \R^N$, where
  \[%
    p \mapsto (\rho_1(p)\phi_1(p), \rho_1(p), \rho_2(p)\phi_2(p), \rho_2(p), \cdots, \rho_r(p)\phi_r(p), \rho_r(p))
  ,\]%
  where $N = r(n + 1)$. This map is an immersion and is locally injective on $M$.

  Now, view $G(M) \subseteq \R^N$. By the Whitney Embedding Theorem, for a generic linear map $L : \R^N \to \R^{2n}$, the composition $F = L \circ G : M \to \R^{2n}$ is an embedding. Thus, $F$ is injective and $\dd{F}_p$ is injective at each point.
\end{solution}

\begin{problem}
  Let $X \subset L^2[0, 1]$ be defined by
  \[%
    X = \left\{f \in L^2[0, 1] \left\vert \int_0^1 f^2 \dx = 1\right\}\right.
  .\]%
  What makes most sense as to define as $T_hX$, for $h = \sqrt{3}x$?
\end{problem}

\begin{solution}
  Notice that $X$ is the unit sphere in the Hilbert space $L^2[0, 1]$. Define $F$ such that $F : L^2[0, 1] \to \R$ where
  \[%
    F(f) = \int_0^1 f(x)^2 \dx \implies X = F^{-1}(1)
  .\]%
  Then, the differential of $F$ at $h$ is given by
  \[%
    \dd{F}_h(g) = 2\int_0^1 g(x)h(x) \dx = 2\bra{g, h}_{L^2}
  .\]%
  We want the differential to be tangent to $F = 1$, which gives us the condition that $\dd{F}_h(g) = 0$. We first must check that $\sqrt{3}x$ is in $X$
  \[%
    \int_0^1 (\sqrt{3}x)^2 \dx = 3\int_0^1 x^2 \dx = 1
  .\]%
  Then, $T_hX$ is given by
  \[%
    T_hX = \left\{g \in L^2[0, 1] \left\lvert \int_0^1 xg(x) \dx = 0\right\}\right.~\text{or}~\left\{g \in L^2[0, 1] \left\lvert \bra{x, g}_{L^2} = 0\right\}\right.
  .\qedhere\]%
\end{solution}

\begin{problem}[3.1]
  Suppose $M$ and $N$ are smooth manifolds with or without boundary, and $F : M \to N$ is a smooth map. Show that $\dd{F}_p : T_pM \to F_{F(p)}N$ is the zero map for each $p \in M$ if and only if $F$ is constant on each component of $M$.
\end{problem}

\begin{solution}
  Assume $\dd{F}_p$ is the zero map for each $p \in M$. Let $p, q \in M$ be in the same component, and let $\gamma : [0, 1] \to M$ be a smooth curve such that $\gamma(0) = p$ and $\gamma(1) = q$. Then, the composition $F \circ \gamma : [0, 1] \to N$ is a smooth curve in $N$, and its derivative at any $t \in [0, 1]$ is given by
  \[%
    (F \circ \gamma)'(t) = \dd{F}_{\gamma(t)}(\gamma'(t)) = 0
  .\]%
  Thus, $F \circ \gamma$ is constant, and in particular,
  \[%
    F(p) = (F \circ \gamma)(0) = (F \circ \gamma)(1) = F(q)
  .\]%
  Therefore, $F$ is constant on each component of $M$.

  Conversely, assume the converse. Let $F$ be constant in each component of $M$. Then, for any vector $v = v^j \pdv{}{x^j}\mid_p~\in T_pM$, we have
  \[%
    \dd{F}_p(v) = v^j \left.\pdv{F}{x^j}\right\rvert_p \left.\pdv{}{y^i}\right\rvert_{F(p)} = 0
  .\]%
  Thus, $\dd{F}_p$ is the zero map for each $p \in M$.

  Therefore, $\dd{F}_p : T_pM \to T_{F(p)}N$ is the zero map for each $p \in M$ if and only if $F$ is constant on each component of $M$.
\end{solution}

\begin{problem}[3.4]
  Show that $T\S^1$ is diffeomorphic to $\S^1 \times \R$.
\end{problem}

\begin{solution}
  Define the map $\Phi : \S^1 \times \R \to T\S^1$ by
  \[%
    \Phi\left((x, y), t\right) = ((x, y), tu_{(x,y)}) = \left((x, y), t(-y, x)\right)
  .\]%
  This is well defined since $u_{(x,y)} = (-y, x)$ is a unit tangent vector at the point $(x, y) \in \S^1$. For each fixed $p$, the mapping $t \mapsto tu_p$ is a linear bijection $\R \to T_p\S^1$.

  Let $(p, v) \in T\S^1$. Write $p = (x, y)$. Because $v$ is tangent, $v \cdot p = 0$, so $v$ is a scalar multiple of the tangent direction $u_p$. Thus there exists $t \in \R$ with $v = t u_p$. Then $\Phi((p, t)) = (p, v)$.

  If $\Phi((p, t)) = \Phi((p', t'))$ then the basepoints must match $p = p'$ and $tu_p = t'u_p$. Since $u_p \neq 0$, this forces $t = t'$. So $\Phi$ is injective.

  Thus $\Phi$ is a bijection. Its inverse $\Psi : T\S^1 \to \S^1 \times \R$ is explicitly
  \[%
    \Psi((x, y), v) = ((x,y), t),\qquad t = v\cdot u_{(x,y)}
  .\]%
  Indeed $v$ lies in the span of $u_p$ so taking the inner product with $u_p$ recovers the scalar $t$.

  Both $\Phi$ and $\Psi$ are given by polynomial (hence smooth) formulas in the coordinates $(x, y, t)$ or $(x, y, v_1, v_2)$. Therefore $\Phi$ and $\Psi$ are smooth and are inverses of each other, so $\Phi$ is a diffeomorphism. Thus, $T\S^1$ is diffeomorphic to $\S^1 \times \R$.
\end{solution}

\begin{problem}[3.5]
  Let $\S^1 \subseteq \R^2$ be the unit circle, and let $K \subseteq \R^2$ be the boundary of the square of side 2 centered at the origin: $K = \{(x, y) \mid \max(|x|, |y|) = 1\}$. Show that there is a homeomorphism $F : \R^2 \to \R^2$ such that $F(\S^1) = K$, but there is no \emph{diffeomorphism} with the same property. [Hint: let $\gamma$ be a smooth curve whose image lies in $\S^1$, and consider the action of $\dd{F}(\gamma'(t))$ on the coordinate functions $x$ and $y$.] \emph{(Used on p. 123.)}
\end{problem}

\begin{solution}
  Define a map $k : \S^1 \to K$ by
  \[%
    k(u_1, u_2) = \frac{(u_1, u_2)}{\max(|u_1|, |u_2|)}
  .\]%
  This is well-defined, continuous, bijective, and has a continuous inverse $k^{-1} : K \to \S^1$ given by $k^{-1}(y) = y/\|y\|$. Extend this map radially to all of $\R^2$ by defining $F : \R^2 \to \R^2$ as
  \[%
    F(0) = 0, \qquad F(r, u) = rk(u)~\text{for}~r > 0~\text{and}~u \in \S^1
  .\]%
  Clearly, $F$ is continuous because $k$ is continuous and the radial extension is continuous in polar coordinates and is bijective with inverse
  \[%
    F^{-1}(0) = 0, \qquad F^{-1}(y) = \|y\|\, k^{-1}\!\Big(\frac{y}{\|y\|}\Big) \text{ for } y \neq 0,
  \]%
  which is continuous. It's also on the circle, $F(u) = k(u) \in K$, so $F(\S^1) = K$.

  Hence $F$ is a homeomorphism of $\R^2$ sending $\S^1$ to $K$. 

  Now, suppose for contradiction, that there exists a diffeomorphism $G : \R^2 \to \R^2$ with $G(\S^1) = K$. Since $\S^1$ is a smooth embedded $1$-manifold in $\R^2$, the image $G(\S^1)$ would also be a smooth embedded $1$-manifold in $\R^2$. However, $K$ has corner points (e.g., $(1,1)$) where no well-defined tangent line exists. A smooth embedded $1$-manifold cannot have such singular points. Therefore, no such diffeomorphism $G$ can exist.
\end{solution}

\begin{problem}[3.8]
  Let $M$ be a smooth manifold with or without boundary and $p \in M$. Let $\mathcal{V}_pM$ denote the set of equivalence classes of smooth curves starting at $p$ under the relation $\gamma_1 ~ \gamma_2$ if $(f \circ \gamma_1)'(0) = (f \circ \gamma_2)'(0)$ for every smooth real-valued function $f$ defined in a neighborhood of $p$. Show that the map $\Psi : \mathcal{V}_pM \to T_pM$ defined by $\Psi[\gamma] = \gamma'(0)$ is well defined and bijective. \emph{(Used on p. 72.)}
\end{problem}

\begin{solution}
  Let $\gamma_1$ and $\gamma_2$ be two smooth curves starting at $p$ such that $\gamma_1 \sim \gamma_2$. By definition, this means that for every smooth real-valued function $f$ defined in a neighborhood of $p$, we have
  \[%
    (f \circ \gamma_1)'(0) = (f \circ \gamma_2)'(0)
  .\]%
  The derivative of the composition can be expressed using the chain rule:
  \[%
    (f \circ \gamma_i)'(0) = \dd{f}_p(\gamma_i'(0)) \quad \text{for } i = 1, 2
  .\]%
  Therefore, we have
  \[%
    \dd{f}_p(\gamma_1'(0)) = \dd{f}_p(\gamma_2'(0))
  .\]%
  Since this holds for all smooth functions $f$, it follows that $\gamma_1'(0) = \gamma_2'(0)$. Thus, the map $\Psi$ is well defined.

  To show that $\Psi$ is bijective, we first prove injectivity. Suppose $\Psi[\gamma_1] = \Psi[\gamma_2]$. This means that $\gamma_1'(0) = \gamma_2'(0)$. For any smooth function $f$ defined in a neighborhood of $p$, we have
  \[%
    (f \circ \gamma_1)'(0) = \dd{f}_p(\gamma_1'(0)) = \dd{f}_p(\gamma_2'(0)) = (f \circ \gamma_2)'(0)
  .\]%
  Thus, $\gamma_1 \sim \gamma_2$, and hence $[\gamma_1] = [\gamma_2]$. Therefore, $\Psi$ is injective.

  Next, we prove surjectivity. Let $v \in T_pM$. We need to find a smooth curve $\gamma$ starting at $p$ such that $\gamma'(0) = v$. Choose a coordinate chart $(U, \phi)$ around $p$ such that $\phi(p) = 0$. In these coordinates, we can define a smooth curve $\tilde{\gamma} : (-\epsilon, \epsilon) \to \R^n$ by
  \[%
    \tilde{\gamma}(t) = tv
  .\]%
  Then, we can define $\gamma : (-\epsilon, \epsilon) \to M$ by
  \[%
    \gamma(t) = \phi^{-1}(\tilde{\gamma}(t))
  .\]%
  This curve is smooth, starts at $p$, and its derivative at $0$ is
  \[%
    \gamma'(0) = \dd{\phi^{-1}}_0(v) = v
  .\]%
  Thus, $\Psi[\gamma] = v$, proving that $\Psi$ is surjective.

  Since $\Psi$ is both injective and surjective, it is a bijection. Therefore, the map $\Psi : \mathcal{V}_pM \to T_pM$ defined by $\Psi[\gamma] = \gamma'(0)$ is well defined and bijective.
\end{solution}
