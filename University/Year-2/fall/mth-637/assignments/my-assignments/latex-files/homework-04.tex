\renewcommand\P{\mathbb{P}}
\renewcommand\T{\mathbb{T}}
\renewcommand\S{\mathbb{S}}

\begin{problem}[4.4]
  Let $\gamma : \R \to \T^2$ be the curve of Example 4.20. Show that the image set $\gamma(\R)$ is dense in $\T^2$. (\emph{Used on pp. 502, 542.})
\end{problem}

\begin{solution}
\end{solution}

\begin{problem}[4.5]
  Let $\C\P^n$ denote the $n$-dimensional complex projective space, as defined in Problem 1-9.
  \begin{enumerate}
    \item Show that the quotient map $\pi : \C^{n+1} \setminus \{0\} \to \C\P^n$ is a surjective smooth submersion.

    \item Show that $\C\P^1$ is diffeomorphic to $\S^2$.
  \end{enumerate}
\end{problem}

\begin{solution}[(i)]
  Smoothness and surjectivity is rather straightforward. Every point in $\C\P^n$ is an equivalence class $[z]$ of some nonzero $z \in \C^{n+1}$, so $\pi(z) = [z]$ is surjective. To prove smoothness, fix an index $j \in \{0, \cdots, n\}$ and consider the chart
  \[%
    U_j = \{[z] \in \C\P^n \mid z_j \neq 0\}
  .\]%
  The usual coordinate chart $\phi_j : U_j \to \C^n$ is
  \[%
    \phi_j([z]) = \left(z_0/z_j, \cdots, z_{j-1}/z_j, z_{j+1}/z_j, \cdots, z_n/z_j\right)
  ,\]%
  where $z_0 ~ \cdots ~ z_n$ are the coordinates of any representative of $[z]$ with $z_j \neq 0$. On the open subset $V_j := \{z \in \C^{n+1} \setminus \{0\} \mid z_j \neq 0\}$ of the domain we have the map
  \begin{gather*}
    \phi_j \circ \pi : V_j \to \C^n \\
    z \mapsto \left(\frac{z_0}{z_j}, \cdots, \widehat{\frac{z_j}{z_j}}, \cdots, \frac{z_n}{z_j}\right)
  ,\end{gather*}
  where the hat indicates omission. This map is smooth and since the charts, $\{(U_j, \phi_j)\}$, cover the codomain. Thus, $\pi$ is smooth. Hence, $\pi$ is a surjective smooth map.

  Now, we show that $\pi$ is a submersion. Define a local smooth section $\sigma_j : \C^n \to V_j \subset \C^{n+1} \setminus \{0\}$ by
  \[%
    \sigma_j(w_0, \cdots, w_{n-1}) = (w_0, \cdots, w_{j-1}, 1, w_j, \cdots, w_{n-1})
  ,\]%
  i.e., the representative with $j$-th coordinate $1$. Then for every point of $U_j$, we have
  \[%
    (\phi_j \circ \pi) \circ \sigma_j = \Id_{\C^n}
  .\]%
  Differentiate this identity to get $D(\phi_j \circ \pi) \circ D \sigma_j = I$. Thus $D(\phi_j \circ \pi)$ has a right inverse $D\sigma_j$, so it is surjective. But $D(\phi_j \circ \pi)$ is the coordinate representation of $d\pi$ on $V_j$, so $d\pi$ is surjective at every point of $V_j$. Since the $V_j$ cover the domain, $d\pi$ is surjective everywhere: $\pi$ is a submersion. Conclude
  \[%
    \Rank(d\pi) = \dim_\R \C^n = 2n
  ,\]%
  at every point.
\end{solution}

\begin{solution}[(ii)]
\end{solution}

\begin{problem}[4.6]
  Let $M$ be a nonempty smooth compact manifold. Show that there is no smooth submersion $F : M \to \R^k$ for any $k > 0$.
\end{problem}

\begin{solution}
\end{solution}

\begin{problem}[4.8]
  This problem shows that the converse of Theorem 4.29 is false. Let $\pi : \R^2 \to \R$ be defined by $\pi(x, y) = xy$. Show that $\pi$ is surjective and smooth, and for each smooth manifold $P$, a map $F : \R \to P$ is smooth if and only if $F \circ \pi$ is smooth, but $\pi$ is not a smooth submersion.
\end{problem}

\begin{solution}
\end{solution}

\begin{problem}[4.10]
  Show that the map $q : \S^n \to \R\P^n$ defined in Example 2.13(f) is a smooth covering map. (\emph{Used on p. 550.})
\end{problem}

\begin{solution}
\end{solution}

\begin{problem}[4.13]
  Define a map $F : \S^2 \to \R^4$ by $F(x, y, z) = (x^2 - y^2, xy, xz, yz)$. Using the smooth covering map of Example 2.13(f) and Problem 4-10, show that $F$ descends to a smooth embedding of $\R\P^2$ into $\R^4$.
\end{problem}

\begin{solution}
\end{solution}
