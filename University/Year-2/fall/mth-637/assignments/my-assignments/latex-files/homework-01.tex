\renewcommand\S{\mathbb{S}}

\begin{problem}[1.1]
  Let $X$ be the set of all points $(x, y) \in \R^2$ such that $y = \pm 1$, and let $M$ be the quotient of $X$ by the equivalence relation generated by $(x, -1) \sim (x, 1)$ for all $x \ne 0$. Show that $M$ is locally Euclidean and second-countable, but not Hausdorff. (This space is called the \emph{\textbf{line with two origins}}.)
\end{problem}

\begin{solution}
  Define
  \[%
    X = \{(x, 1) \mid x \in \R\} \cup \{(x, -1) \mid x \in \R\} \cong \R \times \{\pm 1\}
  ,\]%
  with the subspace topology from $\R^2$. Let $q : X \to M$ be the quotient map under the equivalence relation $(x, 1) \sim (x, -1)$ for all $x \neq 0$. Denote $0_+ = q(0, 1)$ and $0_- = q(0, -1)$.

  Let $p \in M$. If $p = q(x_0, y_0)$ with $x_0 \neq 0$, then there exists $\varepsilon > 0$ such that $(x_0 - \varepsilon, x_0 + \varepsilon)$ does not contain $0$. Define
  \[%
    U = q\!\left( (x_0 - \varepsilon, x_0 + \varepsilon) \times \{1, -1\} \right)
  .\]%
  The map $\Phi : U \to (x_0 - \varepsilon, x_0 + \varepsilon)$ defined by $\Phi(q(x, y)) = x$ is well-defined, bijective, continuous, and its inverse $\Phi^{-1}(x) = q(x, 1)$ is continuous. Hence $U$ is homeomorphic to an open interval in $\R$.

  But if $p = 0_+$, choose $\varepsilon > 0$ and define
  \[%
    U_+ = q\!\left((-\varepsilon, \varepsilon) \times \{1\} \right) = \{0_+\} \cup \{ q(x, 1) \mid 0 < |x| < \varepsilon \}
  .\]%
  Define $\Phi_+ : U_+ \to (-\varepsilon, \varepsilon)$ by
  \[%
    \Phi_+(0_+) = 0, \quad \Phi_+(q(x, 1)) = x \text{ for } x \ne 0
  .\]%
  Then $\Phi_+$ is bijective, continuous, and has a continuous inverse $\Phi_+^{-1}(x) = q(x, 1)$. Thus, $U_+$ is homeomorphic to an open interval. The same argument applies for $p = 0_-$ using
  \[%
    U_- = q\!\left( (-\varepsilon, \varepsilon) \times \{-1\} \right), \quad \Phi_-(0_-) = 0, \ \Phi_-(q(x, -1)) = x
  .\]%
  Therefore, every point of $M$ has a neighborhood homeomorphic to an open subset of $\R$, and hence $M$ is locally Euclidean of dimension $1$.

  Now, we show that $M$ is second-countable. For each $a \in \Q$ and $r \in \Q^+$, define
  \[%
    U_{a, r}^+ = q\!\left( (a - r, a + r) \times \{1\} \right), \quad U_{a, r}^- = q\!\left( (a - r, a + r) \times \{-1\} \right)
  ,\]%
  and for the two origins, define for each $\epsilon \in \Q^+$,
  \[%
    U_+(\epsilon) = q\!\left( (-\epsilon, \epsilon) \times \{1\} \right), \quad U_-(\epsilon) = q\!\left( (-\epsilon, \epsilon) \times \{-1\} \right)
  .\]%
  Let
  \[%
    \mathcal{B} = \{U_{a, r}^+, U_{a, r}^- \mid a \in \Q, r \in \Q^+\} \cup \{U_+(\epsilon), U_-(\epsilon) \mid \epsilon \in \Q^+\}
  .\]%
  Since $\Q$ and $\Q^+$ are countable, $\mathcal{B}$ is countable. Each $B \in \mathcal{B}$ is open in $M$ because $q^{-1}(B)$ is open in $X$. To see that $\mathcal{B}$ is a basis, take any open $O \subset M$ and any $p \in O$. Then $q^{-1}(O)$ is open in $X$, so there exists an open interval $(a - r, a + r)$ (with $a, r \in \Q$) such that $q((a - r, a + r) \times \{y_0\}) \subset O$. Hence $O$ is a union of elements of $\mathcal{B}$, and $\mathcal{B}$ is a countable basis. Thus, $M$ is second-countable.

  Lastly, we show that $M$ is not Hausdorff. Suppose $U$ and $V$ are disjoint open neighborhoods of $0_+$ and $0_-$, respectively. Then $q^{-1}(U)$ is an open subset of $X$ containing $(0, 1)$, so there exists $\delta > 0$ such that $(-\delta, \delta) \times \{1\} \subset q^{-1}(U)$. Similarly, there exists $\varepsilon > 0$ such that $(-\varepsilon, \varepsilon) \times \{-1\} \subset q^{-1}(V)$. Let $0 < |x| < \min(\delta, \varepsilon)$. Then
  \[%
    q(x, 1) = q(x, -1) \in U \cap V
  ,\]%
  contradicting that $U$ and $V$ are disjoint. Hence, no disjoint open neighborhoods can separate $0_+$ and $0_-$, and $M$ fails to be Hausdorff.

  Therefore, the space $M$ is locally Euclidean and second-countable, but not Hausdorff.
\end{solution}

\begin{problem}[1.2]
  Show that a disjoint union of uncountably many copies of $\R$ is locally Euclidean and Hausdorff, but not second-countable.
\end{problem}

\begin{solution}
  Define the space
  \[%
    X = \bigsqcup_{\alpha \in A} \R_\alpha
  ,\]%
  where $A$ is an uncountable index set and each $\R_\alpha$ is a copy of the real line. The topology on $X$ is the disjoint union topology, where a set $U \subseteq X$ is open if and only if $U \cap \R_\alpha$ is open in $\R_\alpha$ for each $\alpha \in A$. To show that $X$ is locally Euclidean, take any point $p \in X$. Then, there exists a unique $\alpha_0 \in A$ such that $p \in \R_{\alpha_0}$. Notice that each $\R_\alpha$ is open in $X$, since for any open set $U \subseteq \R_\alpha$, we have $U = U \cap \R_\alpha$ and $U \cap \R_\beta = \emptyset$ for all $\beta \ne \alpha$. Thus, $U$ is open in $X$ by the definition of the disjoint union topology. Therefore, $\R_{\alpha_0}$ is an open neighborhood of $p$ in $X$. Since there exists a homeomorphism
  \[%
    \Phi : \R_{\alpha_0} \to \R, \quad \Phi(x) = x
  ,\]%
  the set $\R_{\alpha_0}$ is homeomorphic to $\R$. Hence, $X$ is locally homeomorphic to $\R$. Therefore, $X$ is locally Euclidean of dimension $1$.

  To show that $X$ is Hausdorff, take any two distinct points $p, q \in X$. Then, there exist unique $\alpha_1, \alpha_2 \in A$ such that $p \in \R_{\alpha_1}$ and $q \in \R_{\alpha_2}$. If $\alpha_1 \ne \alpha_2$, then $\R_{\alpha_1}$ and $\R_{\alpha_2}$ are disjoint open neighborhoods of $p$ and $q$, respectively. If $\alpha_1 = \alpha_2$, then $p, q \in \R_{\alpha_1}$. Since $\R_{\alpha_1}$ is homeomorphic to $\R$, which is Hausdorff, there exist disjoint open neighborhoods $U_p$ and $U_q$ of $p$ and $q$ in $\R_{\alpha_1}$. Since $\R_{\alpha_1}$ is open in $X$, both $U_p$ and $U_q$ are open in $X$. Thus, in either case, we can find disjoint open neighborhoods of any two distinct points in $X$. Therefore, $X$ is Hausdorff.

  Finally, we show that $X$ is not second-countable. Suppose for contradiction that $X$ has a countable basis $\mathcal{B} = \{B_i\}_{i \in I}$, where $I$ is a countable index. That means, for every $\alpha \in A$, if $p_\alpha \in \R_\alpha$, then there exists $i \in I$ such that $p_\alpha \in B_i$ and $B_i \subseteq \R_\alpha$. But since $A$ is uncountable and $I$ is countable, by the pigeonhole principle, there exists some $i_0 \in I$ such that $B_{i_0} \subseteq \R_\alpha$ for uncountably many $\alpha \in A$. This is a contradiction since $B_{i_0}$ can only be a subset of one $\R_\alpha$. Therefore, $X$ cannot have a countable basis, and hence is not second-countable.

  In conclusion, the disjoint union of uncountably many copies of $\R$ is locally Euclidean and Hausdorff, but not second-countable.
\end{solution}

\begin{problem}[1.3]
  A topological manifold is said to be \emph{\textbf{$\sigma$-compact}} if it can be expressed as a union of countably many compact subspaces. Show that a locally Euclidean Hausdorff space is a topological manifold if and only if it is $\sigma$-compact.
\end{problem}

\begin{solution}
  Assume $M$ is a locally Euclidean Hausdorff topological manifold. Then, there exists a countable amount of charts $\{(U_\alpha, \Phi_\alpha)\}_{\alpha=1}$ such that
  \[%
    M = \bigcup_{\alpha=1}^\infty U_\alpha
  ,\]%
  where each $\Phi_\alpha : U_\alpha \subset M \to \R^n$ is a homeomorphism, and $U_\alpha$ is an open subset of $M$. Since $M$ is Hausdorff, for any point $p_{\alpha_1} \in U_\alpha$, there exists an open ball $B_r(p)$ that's fully contained in $U_\alpha$ such that $\overline{B_r(p_{\alpha_1})}$ is compact.
  Continuing this process, we can find a countable collection of open balls $\{B_{r_i}(p_{\alpha_i})\}_{i=1}^\infty$ such that
  \[%
    U_\alpha = \bigcup_{i=1}^\infty B_{r_i}(p_{\alpha_i})
  .\]%
  We do this for every $\alpha$, and we get a countable collection of open balls $\{B_{r_i}(p_{\alpha_i})\}_{i=1}^\infty$ such that
  \[%
    M = \bigcup_{\alpha=1}^\infty \bigcup_{i=1}^\infty B_{r_i}(p_{\alpha_i})
  .\]%
  Since each $\overline{B_{r_i}(p_{\alpha_i})}$ is compact, $M$ is the union of countably many compact subsets. Thus, $M$ is $\sigma$-compact.

  Now, for the converse, suppose $M$ is a locally Euclidean Hausdorff space that is $\sigma$-compact. Then, there exists a countable collection of compact subsets $\{K_n\}_{n=1}^\infty$ such that
  \[%
    M = \bigcup_{n=1}^\infty K_n
  .\]%
  To show that $M$ is second-countable, we will construct a countable basis for its topology. Since each $K_n$ is compact, there exists a finite collection of charts $\{(U_{\alpha_1}, \Phi_{\alpha_1}), \cdots, (U_{\alpha_{m_n}}, \Phi_{\alpha_{m_n}})\}$ such that for every $i \le n$, $\Phi_{\alpha_i} : U_{\alpha_i} \subset M \to \R^n$ is a homeomorphism. Define $B_{\alpha_i}$ to be the collection of the pre-image of all countably many open balls with rational radii and centers with rational coordinates in $\R^n$. Then, the collection
  \[%
    \mathcal{B}_n = \bigcup_{i=1}^{m_n} \{\Phi_{\alpha_i}^{-1}(B) \mid B \in B_{\alpha_i}\}
  ,\]%
  is a countable basis for the subspace topology on $K_n$. Since $M$ is the union of the $K_n$, the collection
  \[%
    \mathcal{B} = \bigcup_{n=1}^\infty \mathcal{B}_n
  .\]%
  is a countable basis for the topology on $M$. Thus, $M$ is second-countable.

  Therefore, a locally Euclidean Hausdorff space is a topological manifold if and only if it is $\sigma$-compact.
\end{solution}

\begin{problem}[1.7]
  Let $N$ denote the \emph{\textbf{north pole}} $(0, \cdots, 0, 1) \in \S^n \subseteq \R^{n+1}$, and let $S$ denote the \emph{\textbf{south pole}} $(0, \cdots, 0, -1)$. Define the \emph{\textbf{stereographic projection}} $\sigma : \S^n \setminus \{N\} \to \R^n$ by
  \[%
    \sigma(x^1, \cdots, x^{n+1}) = \frac{(x^1, \cdots, x^n)}{1 - x^{n+1}}
  .\]%
  Let $\widetilde{\sigma} = -\sigma(-x)$ for $x \in \S^n \setminus \{S\}$.
  \begin{enumerate}
    \item For any $x \in \S^n \setminus \{N\}$, show that $\sigma(x) = u$, where $(u, 0)$ is the point where the line through $N$ and $x$ intersects the linear subspace where $x^{n+1} = 0$. Similarly, show that $\widetilde{\sigma}(x)$ is the point where the line through $S$ and $x$ intersects the same subspace. (For this reason, $\widetilde{\sigma}$ is called \emph{\textbf{stereographic projection from the south pole}}.)

    \item Show that $\sigma$ is bijective, and
      \[%
        \sigma^{-1}(u^1, \cdots, u^n) = \frac{(2u^1, \cdots, 2u^n, |u|^2 - 1)}{|u^2| + 1}
      .\]%

    \item Compute the transition map $\widetilde{\sigma} \circ \sigma^{-1}$ and verify that the atlas consisting of two charts ($\S^n \setminus \{N\}, \sigma)$ and $(\S^n \setminus \{S\}, \sigma)$ defines a smooth structure on $\S^n$. (The coordinates defined by $\sigma$ or $\widetilde{\sigma}$ are called \emph{\textbf{stereographic coordinates}}.)

    \item Show that this smooth structure is the same as the one defined in Example 1.31.
  \end{enumerate}
\end{problem}

\begin{solution}[(i)]
  Let $x = (x^1, \cdots, x^n, x^{n+1}) \in \S^n \setminus \{N\}$. The line from $N$ to $x$ can be written as $\ell(t) = N + t(x - N)$. We seek the point $\ell(t)$ whose last coordinate is zero, i.e., the intersection with the hyperplane $x^{n+1} = 0$. Setting the last coordinate of $\ell(t)$ to zero gives:
  \[%
    1 + t(x^{n+1} - 1) = 0 \implies t = \frac{1}{1 - x^{n+1}}
  .\]%
  Substituting this back into the first $n$ coordinates of $\ell(t)$, we have:
  \[%
    \ell(t) = \left(\frac{x^1}{1 - x^{n+1}}, \cdots, \frac{x^n}{1 - x^{n+1}}, 0\right) = \frac{(x^1, \cdots, x^n, 0)}{1 - x^{n+1}} = (\sigma(x), 0)
  .\]%
  Thus, $\sigma(x) = u$ where $(u, 0)$ is the intersection point.


  To find $\widetilde{\sigma}(x)$, note that it is defined by $\widetilde{\sigma}(x) = -\sigma(-x)$. Applying $\sigma$ to the antipodal point $-x$ gives
  \[%
    \sigma(-x) = \frac{(-x^1, \cdots, -x^n)}{1 + x^{n+1}} = -\frac{(x^1, \cdots, x^n)}{1 + x^{n+1}}
  .\]%
  But $\widetilde{\sigma}(x) = -\sigma(-x)$, giving us
  \[%
    \widetilde{\sigma}(x) = -\sigma(-x) = -\left(-\frac{(x^1, \cdots, x^n)}{1 + x^{n+1}}\right) = \frac{(x^1, \cdots, x^n)}{1 + x^{n+1}}
  .\]%
  Therefore, $\widetilde{\sigma}(x) = u$ where $(u, 0)$ is the intersection of the line through $S$ and $x$ with the hyperplane $x^{n+1} = 0$.
\end{solution}

\begin{solution}[(ii)]
  If $\sigma$ is a bijection, there its inverse $\sigma^{-1}$ is well-defined. Solving for each coordinate of $u$, we get
  \[%
    u^i = \frac{x^i}{1 - x^{n+1}}
  .\]%
  Since the radius of $\S^n$ is $1$, we have
  \begin{alignat*}{3}
      \phantom{\implies}\quad&\sum_{i=1}^n (x^i)^2 + \left(x^{n+1}\right)^2 &&= 1 \\
      \implies\quad&\sum_{i=1}^n (u^i)^2\left(1 - x^{n+1}\right)^2 + \left(x^{n+1}\right)^2 &&= 1 \\
      \implies\quad&\left(1 - x^{n+1}\right)^2 \sum_{i=1}^n (u^i)^2 + \left(x^{n+1}\right)^2 &&= 1 \\
      \implies\quad&\left(1 - x^{n+1}\right)^2 |u|^2 + \left(x^{n+1}\right)^2 &&= 1 \\
      \implies\quad&|u|^2\left(1 - 2x^{n+1} + (x^{n+1})^2\right) + \left(x^{n+1}\right)^2 &&= 1 \\
      \implies\quad&|u|^2 - 2|u|^2x^{n+1} + |u|^2\left(x^{n+1}\right)^2 + \left(x^{n+1}\right)^2 &&= 1 \\
      \implies\quad&\left(|u|^2 + 1\right)\left(x^{n+1}\right)^2 - 2|u|^2x^{n+1} + (|u|^2 - 1) &&= 0
  .\end{alignat*}
  This is a quadratic equation in $x^{n+1}$. From the equation, we know $A = |u|^2 + 1$, $B = -2|u|^2$, and $C = |u|^2 - 1$. There are two solutions to this equation, but we want the one less than $1$ since $x^{n+1} \ne 1$ (as $N$ is not in the domain of $\sigma$). Let $r^2 = |u|^2$ and $t = x^{n+1}$. From this, we get the following quadratic
  \[%
    (r^2 + 1)t^2 - 2r^2 t + (r^2 - 1) = 0
  .\]%
  Using the quadratic formula, we find
  \begin{align*}
    t &= \frac{2r^2 \pm \sqrt{(-2r^2)^2 - 4(r^2 + 1)(r^2 - 1)}}{2(r^2 + 1)} \\
      &= \frac{2r^2 \pm \sqrt{4r^4 - 4(r^4 - 1)}}{2(r^2 + 1)} \\
      &= \frac{2r^2\pm 2}{2(r^2 + 1)} \\
      &= \frac{r^2\pm 1}{r^2 + 1}
  .\end{align*}
  Therefore, the two roots are
  \[%
    x^{n+1} = 1 \aand x^{n+1} = \frac{|u|^2 - 1}{|u|^2 + 1}
  .\]%
  Taking the second root, we can substitute back to find $x^i$ for $i = 1, \cdots, n$:
  \[%
    1 - x^{n+1} = 1 - \frac{|u|^2 - 1}{|u|^2 + 1} = \frac{2}{|u|^2 + 1}
  .\]%
  Plugging that into the equation for $x^i$, we get
  \[%
    x^i = u^i\left(1 - x^{n+1}\right) = u^i \frac{2}{|u|^2 + 1} = \frac{2u^i}{|u|^2 + 1}
  .\]%
  Thus, we have
  \[%
    \sigma^{-1}(u^1, \cdots, u^n) = \left(\frac{2u^1}{|u|^2 + 1}, \cdots, \frac{2u^n}{|u|^2 + 1}, \frac{|u|^2 - 1}{|u|^2 + 1}\right) = \frac{(2u^1, \cdots, 2u^n, |u|^2 - 1)}{|u|^2 + 1}
  .\]%

  Since $\sigma^{-1}$ exists and is well-defined, $\sigma$ is bijective.
\end{solution}

\begin{solution}[(iii)]
  We compute the transition map
  \[%
    \Phi = \sigma \circ \widetilde{\sigma}^{-1} : \R^n \setminus \{0\} \to \R^n \setminus \{0\}
  .\]%
  Let $u = (u^1, \cdots, u^n) \in \R^n \setminus \{0\}$ and set $r^2 = |u|^2$. Recall that
  \[%
    \widetilde{\sigma}^{-1}(u) = \left(\frac{2u}{1 + r^2}, \frac{1 - r^2}{1 + r^2}\right)
  .\]%
  Applying $\sigma$ to $\widetilde{\sigma}^{-1}(u)$ gives
  \[%
    \Phi(u) = \sigma(\widetilde{\sigma}^{-1}(u)) = \frac{\dfrac{2u}{1 + r^2}}{1 - \dfrac{1 - r^2}{1 + r^2}} = \frac{\dfrac{2u}{1 + r^2}}{\dfrac{2r^2}{1 + r^2}} = \frac{u}{r^2} = \frac{u}{|u|^2}
  .\]%
  Therefore, $\Phi(u)$ is the \emph{inversion in the unit sphere}. Note that this map is its own inverse and smooth on $\R^n \setminus \{0\}$, so the stereographic charts $\sigma$ and $\widetilde{\sigma}$ are smoothly compatible.
\end{solution}

\begin{solution}[(iv)]
  Recall that in Example 1.31, the smooth structure on $\S^n$ was defined using the charts
  \[%
    \varphi_i^\pm : U_i^\pm \cap \S^n \to \B^n, \quad \varphi_i^\pm(x^1, \ldots, x^{n+1}) = (x^1, \ldots, \widehat{x^i}, \ldots, x^{n+1})
  ,\]%
  where $U_i^\pm = \{ (x^1, \ldots, x^{n+1}) \in \R^{n+1} : \pm x^i > 0 \}$.
  Each $\varphi_i^\pm$ is a homeomorphism with smooth inverse
  \[%
    (\varphi_i^\pm)^{-1}(u^1, \ldots, u^n) = (u^1, \ldots, u^{i-1}, \pm\sqrt{1 - |u|^2}, u^i, \ldots, u^n) ,\]%
  so that $\{(U_i^\pm \cap \S^n, \varphi_i^\pm)\}$ defines the standard smooth structure on $\S^n$.

  Now, the smooth structure defined by the stereographic projections
  \[%
    \sigma : \S^n \setminus \{N\} \to \R^n, \quad \widetilde{\sigma} : \S^n \setminus \{S\} \to \R^n
  ,\]%
  is generated by the charts $(\S^n \setminus \{N\}, \sigma)$ and $(\S^n \setminus \{S\}, \widetilde{\sigma})$.
  To show that this smooth structure coincides with the standard one, it suffices to check that the transition maps between these charts and the $\varphi_i^\pm$ are smooth.

  For instance, consider $\sigma \circ (\varphi_{n+1}^+)^{-1}$. For $u \in \B^n$,
  \[%
    (\varphi_{n+1}^+)^{-1}(u) = (u^1, \ldots, u^n, \sqrt{1 - |u|^2})
  ,\]%
  and thus
  \[%
    \sigma((\varphi_{n+1}^+)^{-1}(u)) = \frac{(u^1, \ldots, u^n)}{1 - \sqrt{1 - |u|^2}}
  ,\]%
  which is a smooth map on $\B^n \setminus \{0\}$. A similar computation shows that $\varphi_{n+1}^+ \circ \sigma^{-1}$ and the corresponding maps for $\widetilde{\sigma}$ are smooth as well.

  Since all transition maps between the stereographic charts and the charts $\varphi_i^\pm$ are smooth, the two atlases are smoothly compatible. Hence, they determine the same maximal smooth structure on $\S^n$.
\end{solution}

\begin{problem}[2.3]
  For each of the following maps between spheres, compute sufficiently many coordinate representations to prove that it is smooth.
  \begin{enumerate}
    \item $p_n : \S^1 \to \S^1$ is the \emph{\textbf{nth power map}} for $n \in \Z$, given in complex notation by $p_n(z) = z^n$.

    \item $\alpha : \S^n \to \S^n$ is the \emph{\textbf{antipodal map}} $\alpha(x) = -x$.

    \item $F : \S^3 \to \S^2$ is given by $F(w, z) = (z\overline{w} + w\overline{z}, iw\overline{z} - iz\overline{w}, z\overline{z} - w\overline{w})$ where we think of $\S^3$ as the subset $\{(w, z) \mid |w|^2 + |z|^2 = 1\}$ of $\C^2$.
  \end{enumerate}
\end{problem}

\begin{solution}[(i)]
\end{solution}

\begin{solution}[(ii)]
\end{solution}

\begin{solution}[(iii)]
\end{solution}

\begin{problem}[2.4]
  Show that the inclusion map $\overline{\B}^n \hookrightarrow \R^n$ is smooth when $\overline{\B}^n$ is regarded as a smooth manifold with boundary.
\end{problem}

\begin{solution}
\end{solution}
