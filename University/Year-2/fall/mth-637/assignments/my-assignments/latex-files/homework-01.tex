\renewcommand\S{\mathbb{S}}

\begin{problem}[1.1]
  Let $X$ be the set of all points $(x, y) \in \R^2$ such that $y = \pm 1$, and let $M$ be the quotient of $X$ by the equivalence relation generated by $(x, -1) \sim (x, 1)$ for all $x \ne 0$. Show that $M$ is locally Euclidean and second-countable, but not Hausdorff. (This space is called the \emph{\textbf{line with two origins}}.)
\end{problem}

\begin{solution}
\end{solution}

\begin{problem}[1.2]
  Show that a disjoint union of uncountably many copies of $\R$ is locally Euclidean and Hausdorff, but not second-countable.
\end{problem}

\begin{solution}
\end{solution}

\begin{problem}[1.3]
  A topological manifold is said to be \emph{\textbf{$\sigma$-compact}} if it can be expressed as a union of countably many compact subspaces. Show that a locally Euclidean Hausdorff space is a topological manifold if and only if it is $\sigma$-compact.
\end{problem}

\begin{solution}
\end{solution}

\begin{problem}[1.7]
  Let $N$ denote the \emph{\textbf{north pole}} $(0, \cdots, 0, 1) \in \S^n \subseteq \R^{n+1}$, and let $S$ denote the \emph{\textbf{south pole}} $(0, \cdots, 0, -1)$. Define the \emph{\textbf{stereographic projection}} $\sigma : \S^n \setminus \{N\} \to \R^n$ by
  \[%
    \sigma(x^1, \cdots, x^{n+1}) = \frac{(x^1, \cdots, x^n}{1 - x^{n+1}}
  .\]%
  Let $\widetilde{\sigma} = -\sigma(-x)$ for $x \in \S^n \setminus \{S\}$.
  \begin{enumerate}
    \item For any $x \in \S^n \setminus \{N\}$, show that $\sigma(x) = u$, where $(u, 0)$ is the point where the line through $N$ and $x$ intersects the linear subspace where $x^{n+1} = 0$. Similarly, show that $\widetilde{\sigma}(x)$ is the point where the line through $S$ and $x$ intersects the same subspace. (For this reason, $\widetilde{\sigma}$ is called \emph{\textbf{stereographic projection from the south pole}}.)

    \item Show that $\sigma$ is bijective, and
      \[%
        \sigma^{-1}(u^1, \cdots, u^n) = \frac{(2u^1, \cdots, 2u^n, |u|^2 - 1)}{|u^2| + 1}
      .\]%

    \item Compute the transition map $\widetilde{\sigma} \circ \sigma^{-1}$ and verify that the atlas consisting of two charts ($\S^n \setminus \{N\}, \sigma)$ and $(\S^n \setminus \{S\}, \sigma)$ defines a smooth structure on $\S^n$. (The coordinates defined by $\sigma$ or $\widetilde{\sigma}$ are called \emph{\textbf{stereographic coordinates}}.)

    \item Show that this smooth structure is the same as the one defined in Example 1.31.
  \end{enumerate}
\end{problem}

\begin{solution}[(i)]
\end{solution}

\begin{solution}[(ii)]
\end{solution}

\begin{solution}[(iii)]
\end{solution}

\begin{solution}[(iv)]
\end{solution}

\begin{problem}[2.3]
  For each of the following maps between spheres, compute sufficiently many coordinate representations to prove that it is smooth.
  \begin{enumerate}
    \item $p_n : \S^1 \to \S^1$ is the \emph{\textbf{nth power map}} for $n \in \Z$, given in complex notation by $p_n(z) = z^n$.

    \item $\alpha : \S^n \to \S^n$ is the \emph{\textbf{antipodal map}} $\alpha(x) = -x$.

    \item $F : \S^3 \to \S^2$ is given by $F(w, z) = (z\overline{w} + w\overline{z}, iw\overline{z} - iz\overline{w}, z\overline{z} - w\overline{w})$ where we think of $\S^3$ as the subset $\{(w, z) \mid |w|^2 + |z|^2 = 1\}$ of $\C^2$.
  \end{enumerate}
\end{problem}

\begin{solution}[(i)]
\end{solution}

\begin{solution}[(ii)]
\end{solution}

\begin{solution}[(iii)]
\end{solution}

\begin{problem}[2.4]
  Show that the inclusion map $\overline{\B}^n \hookrightarrow \R^n$ is smooth when $\overline{\B}^n$ is regarded as a smooth manifold with boundary.
\end{problem}

\begin{solution}
\end{solution}
