
---

# 2) Submersion (rank full = (\dim_{\mathbb R}\C\P^n=2n))

There are two equivalent, standard ways to see (d\pi) has maximal rank everywhere.

### a) Local section (chart) argument — the cleanest

Define a local smooth section (\sigma_j:\C^n\to V_j\subset\C^{n+1}\setminus{0}) by
[
\sigma_j(w_0,\dots,w_{n-1})=(w_0,\dots,w_{j-1},1,w_j,\dots,w_{n-1}),
]
i.e. the representative with (j)-th coordinate (1). Then for every point of (U_j),
[
(\varphi_j\circ\pi)\circ\sigma_j = \mathrm{id}*{\C^n}.
]
Differentiate this identity: (D(\varphi_j\circ\pi)\circ D\sigma_j = I). Thus (D(\varphi_j\circ\pi)) has a right inverse (D\sigma_j), so it is surjective. But (D(\varphi_j\circ\pi)) is the coordinate representation of (d\pi) on (V_j), so (d\pi) is surjective at every point of (V_j). Since the (V_j) cover the domain, (d\pi) is surjective everywhere: (\pi) is a submersion. Conclude
[
\operatorname{rank}(d\pi)=\dim*{\mathbb R}\C^n=2n
]
at every point.

(Reasoning principle used: if a smooth map has a smooth local right inverse on an open neighbourhood then its differential is surjective there.)

### b) Tangent / kernel viewpoint — alternative

The action of the Lie group (\C^\times=\C\setminus{0}) on (\C^{n+1}\setminus{0}) by scalar multiplication has orbits equal to the equivalence classes. The tangent space to the orbit through (z) is the 2-real-dimensional subspace spanned by the real and imaginary directions of (z) (equivalently ({ \lambda z : \lambda\in\C}) viewed as a real 2-dim subspace). This orbit is exactly the kernel of (d\pi_z). Thus
[
\dim_{\mathbb R}\ker d\pi_z = 2,
]
and since the domain has real dimension (2n+2),
[
\operatorname{rank}(d\pi_z) = (2n+2)-2 = 2n,
]
so (d\pi_z) is surjective. (Same conclusion; this view ties the submersion property to the quotient-by-a-free-action picture.)

---

# Example of the same approach (simpler case)

Do the same argument for real projective space:
[
\pi:\R^{n+1}\setminus{0}\to\R\P^n.
]
For each (j) with (x_j\neq0) you have a chart (U_j\cong\R^n) and a local section (\sigma_j:\R^n\to\R^{n+1}) sending coordinates to the representative with (j)-th entry (1). The same calculation shows (\pi) is a surjective smooth submersion. This is exactly the real analogue and is slightly easier to think about if you want an example to keep in mind.

---

If you want, I can now:
• write the above more rigorously with formulas for (D(\varphi_j\circ\pi)) at a specific point, **or**
• move on to part (2) and show the explicit diffeomorphism (\C\P^1\cong S^2) (via the stereographic / Hopf coordinate description). Which would you prefer?
