\begin{problem}[1]
  Suppose that A and B are mutually exclusive events for which $P(A) = 0.2$ and $P(B) = 0.6$.
  \begin{enumerate}
    \item What is the probability that either A or B occurs?

    \item What is the probability that A and B occur?

    \item What is the probability that A occurs but B does not?
  \end{enumerate}
\end{problem}

\begin{solution}[(i)]
\end{solution}

\begin{solution}[(ii)]
\end{solution}

\begin{solution}[(iii)]
\end{solution}

\begin{problem}[2]
  The following data were given in a study of a group of 1000 subscribers to a certain magazine: In reference to job, marital status, and education, there are 310 professionals, 475 married persons, 525 college graduates, 45 professional college graduates, 145 married college graduates, 86 married professionals, and 27 married professional college graduates. Show that the numbers reported in the study must be incorrect.
\end{problem}

\begin{solution}
\end{solution}

\begin{problem}[3]
  Consider a standard 52 card playing deck that consists of four suits (hearts, diamonds, spades, and clubs) and 13 cards per suit ($2, 3, \cdots, 10$, Jack, Queen, King, Ace). Assume that all $\binom{52}{5}$ poker hands are equally likely.
  \begin{enumerate}
    \item What is the probability of being dealt a flush? (A flush is a hand consisting of 5 cards of one suit. For example, 2589 J of hearts)

    \item What is the probability of being dealt two pairs? (Two pairs is two different pairs and a fifth unique card. For example, 3377Q)

    \item What is the probability of being dealt four of a kind? (A four of a kind is all four cards from one denomination and one more card. For example, 55552)
  \end{enumerate}
\end{problem}

\begin{solution}[(i)]
\end{solution}

\begin{solution}[(ii)]
\end{solution}

\begin{solution}[(iii)]
\end{solution}

\begin{problem}[4]
  A pair of dice is rolled until a sum of either 5 or 7 appears. Find the probability that a 5 occurs first. Hint : Let $E_n$ denote the event that a 5 occurs on the $n$th roll and no 5 or 7 occurs on the first $n-1$ rolls. Compute $P\left(E_n\right)$ and argue that $\sum_{n=1}^{\infty} P\left(E_n\right)$ is the desired probability.

  Note: You might need to use a formula for a sum of a geometric series.
\end{problem}

\begin{solution}
\end{solution}

\begin{problem}[5]
  Seven balls are randomly withdrawn from an urn that contains 12 red, 16 blue, and 18 green balls.
  \begin{enumerate}
    \item Find the probability that 3 red, 2 blue, and 2 green balls are withdrawn.

    \item Find the probability that at least 2 red balls are withdrawn.

    \item Find the probability that all withdrawn balls are the same color.

    \item Find the probability that either exactly 3 red balls or exactly 3 blue balls are withdrawn.
  \end{enumerate}
\end{problem}

\begin{solution}[(i)]
\end{solution}

\begin{solution}[(ii)]
\end{solution}

\begin{solution}[(iii)]
\end{solution}

\begin{solution}[(iv)]
\end{solution}

\begin{problem}[6]
  Compute the probability that a bridge hand is void in at least one suit.

  Note that the answer is not
  \[%
    \frac{\binom{4}{1} \binom{39}{13}}{\binom{52}{13}}
  .\]%
  Note: In bridge, all 52 cards in a standard deck are dealt to four players where each player has 13 cards. A bridge hand is void in a suit (hearts, diamonds, spades, clubs) if the hand does not contain any of the 13 cards in the suit.
\end{problem}

\begin{solution}
\end{solution}

\begin{problem}[7]
  Consider two people being randomly selected. For simplicity of this problem, ignore leap years.
  \begin{enumerate}
    \item Find the probability that two people have a birthday on the 6th of any month.

    \item What is the probability that two people have a birthday on the same day of the same month?
  \end{enumerate}
\end{problem}

\begin{solution}[(i)]
\end{solution}

\begin{solution}[(ii)]
\end{solution}
