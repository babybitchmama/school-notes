\begin{problem}[1]
  Suppose that A and B are mutually exclusive events for which $P(A) = 0.2$ and $P(B) = 0.6$.
  \begin{enumerate}
    \item What is the probability that either A or B occurs?

    \item What is the probability that A and B occur?

    \item What is the probability that A occurs but B does not?
  \end{enumerate}
\end{problem}

\begin{solution}[(i)]
  Using the formula for the union of two events, we have
  \[%
    P(A \cup B) = P(A) + P(B) - P(A \cap B) = 0.2 + 0.6 - 0 = 0.8
  .\]%
  Therefore, the probability that either A or B occurs is 0.8.
\end{solution}

\begin{solution}[(ii)]
  Since A and B are mutually exclusive events, the probability that both A and B occur is
  \[%
    P(A \cap B) = 0
  .\]%
  Therefore, the probability that A and B occur is 0.
\end{solution}

\begin{solution}[(iii)]
  The probability that A occurs but B does not is given by
  \[%
    P(A \cap B^c) = P(A) - P(A \cap B) = 0.2 - 0 = 0.2
  .\]%
  Therefore, the probability that A occurs but B does not is 0.2.
\end{solution}

\begin{problem}[2]
  The following data were given in a study of a group of 1000 subscribers to a certain magazine: In reference to job, marital status, and education, there are 310 professionals, 475 married persons, 525 college graduates, 45 professional college graduates, 145 married college graduates, 86 married professionals, and 27 married professional college graduates. Show that the numbers reported in the study must be incorrect.
\end{problem}

\begin{solution}
  Let $P$ be the set of professionals, $M$ be the set of married persons, and $C$ be the set of college graduates. We are given the following information:
  \begin{align*}
    |P| & = 310 \\
    |M| & = 475 \\
    |C| & = 525 \\
    |P \cap C| & = 45 \\
    |M \cap C| & = 145 \\
    |M \cap P| & = 86 \\
    |M \cap P \cap C| & = 27
  .\end{align*}
  We can use the principle of inclusion-exclusion to find the total number of subscribers:
  \[%
    |P \cup M \cup C| = |P| + |M| + |C| - |P \cap M| - |P \cap C| - |M \cap C| + |P \cap M \cap C|
  .\]%
  Substituting the given values, we have
  \[%
    |P \cup M \cup C|  = 310 + 475 + 525 - 86 - 45 - 145 + 27 = 1061
  .\]%
  However, this contradicts the fact that there are a total of 1000 subscribers. Therefore, the numbers reported in the study must be incorrect.
\end{solution}

\begin{problem}[3]
  Consider a standard 52 card playing deck that consists of four suits (hearts, diamonds, spades, and clubs) and 13 cards per suit ($2, 3, \cdots, 10$, Jack, Queen, King, Ace). Assume that all $\binom{52}{5}$ poker hands are equally likely.
  \begin{enumerate}
    \item What is the probability of being dealt a flush? (A flush is a hand consisting of 5 cards of one suit. For example, 2589 J of hearts)

    \item What is the probability of being dealt two pairs? (Two pairs is two different pairs and a fifth unique card. For example, 3377Q)

    \item What is the probability of being dealt four of a kind? (A four of a kind is all four cards from one denomination and one more card. For example, 55552)
  \end{enumerate}
\end{problem}

\begin{solution}[(i)]
  To calculate the probability of being dealt a flush, we first determine the number of ways to choose 5 cards from a single suit. There are 4 suits, and for each suit, we can choose 5 cards from the 13 available cards in that suit. The number of ways to choose 5 cards from 13 is given by $\binom{13}{5}$. Therefore, the total number of flush hands is
  \[%
    4 \times \binom{13}{5}
  .\]%
  The total number of possible poker hands is $\binom{52}{5}$. Thus, the probability of being dealt a flush is
  \[%
    P(\text{flush}) = \frac{4 \times \binom{13}{5}}{\binom{52}{5}} = \frac{4 \times 1287}{2598960} = \frac{5148}{2598960} \approx 0.001981
  .\]%
  Therefore, the probability of being dealt a flush is approximately 0.1981 percent.
\end{solution}

\begin{solution}[(ii)]
  To calculate the probability of being dealt two pairs, we first determine the number of ways to choose 2 different ranks for the pairs from the 13 available ranks. This can be done in $\binom{13}{2}$ ways. For each chosen rank, we can select 2 cards from the 4 available cards of that rank, which can be done in $\binom{4}{2}$ ways. Therefore, the number of ways to choose the two pairs is
  \[%
    \binom{13}{2} \times \left(\binom{4}{2}\right)^2
  .\]%
  Next, we need to choose a fifth card that is not of the same rank as either of the pairs. There are 11 remaining ranks (since we have already chosen 2 ranks for the pairs), and for each rank, there are 4 available cards. Thus, the number of ways to choose the fifth card is $11 \times 4$. Therefore, the total number of two pair hands is
  \[%
    \binom{13}{2} \times \left(\binom{4}{2}\right)^2 \times (11 \times 4)
  .\]%
  The total number of possible poker hands is $\binom{52}{5}$. Thus, the probability of being dealt two pairs is
  \[%
    P(\text{two pairs}) = \frac{\binom{13}{2} \times \left(\binom{4}{2}\right)^2 \times (11 \times 4)}{\binom{52}{5}} = \frac{78 \times 6^2 \times 44}{2598960} = \frac{123552}{2598960} \approx 0.04754
  .\]%
  Therefore, the probability of being dealt two pairs is approximately 4.754 percent.
\end{solution}

\begin{solution}[(iii)]
  To calculate the probability of being dealt four of a kind, we first determine the number of ways to choose 1 rank for the four of a kind from the 13 available ranks. This can be done in $\binom{13}{1}$ ways. For the chosen rank, we take all 4 cards of that rank. Next, we need to choose a fifth card that is not of the same rank as the four of a kind. There are 12 remaining ranks (since we have already chosen 1 rank for the four of a kind), and for each rank, there are 4 available cards. Thus, the number of ways to choose the fifth card is $12 \times 4$. Therefore, the total number of four of a kind hands is
  \[%
    \binom{13}{1} \times (12 \times 4)
  .\]%
  The total number of possible poker hands is $\binom{52}{5}$. Thus, the probability of being dealt four of a kind is
  \[%
    P(\text{four of a kind}) = \frac{\binom{13}{1} \times (12 \times 4)}{\binom{52}{5}} = \frac{13 \times 48}{2598960} = \frac{624}{2598960} \approx 0.0002401
  .\]%
  Therefore, the probability of being dealt four of a kind is approximately 0.02401 percent.
\end{solution}

\begin{problem}[4]
  A pair of dice is rolled until a sum of either 5 or 7 appears. Find the probability that a 5 occurs first. Hint : Let $E_n$ denote the event that a 5 occurs on the $n$th roll and no 5 or 7 occurs on the first $n-1$ rolls. Compute $P\left(E_n\right)$ and argue that $\sum_{n=1}^{\infty} P\left(E_n\right)$ is the desired probability.

  Note: You might need to use a formula for a sum of a geometric series.
\end{problem}

\begin{solution}
  The sample space for a roll of two dice has $36$ equally likely outcomes. The outcomes summing to $5$ are 4 in number, and the outcomes summing to $7$ are 6 in number. Hence
  \[%
    P(5) = \frac{4}{36} = \frac{1}{9}, \qquad P(7) = \frac{6}{36} = \frac{1}{6}
  .\]%
  The probability that a single roll is neither a 5 nor a 7 is
  \[%
    P(\text{neither}) = 1 - P(5) - P(7) = 1 - \frac{1}{9} - \frac{1}{6} = \frac{13}{18}
  .\]%

  Let $E_n$ be the event that the first $n-1$ rolls are neither 5 nor 7 and the $n$th roll is a 5. Then
  \[%
    P(E_n) = \left(\frac{13}{18}\right)^{n-1} \cdot \frac{1}{9}
  .\]%
  The desired probability (a 5 occurs before a 7) is
  \[%
    \sum_{n=1}^\infty P(E_n) = \sum_{n=1}^\infty \left(\frac{13}{18}\right)^{n-1} \frac{1}{9}
  .\]%
  This is an infinite geometric series with first term $a = 1/9$ and ratio $r = 13/18$, so
  \[%
    \sum_{n=1}^\infty P(E_n) = \frac{a}{1-r} = \frac{\frac{1}{9}}{1 - \frac{13}{18}} = \frac{\frac{1}{9}}{\frac{5}{18}} = \frac{2}{5}
  .\]%
  Therefore, the probability that a 5 occurs before a 7 is $2/5$, or 40\%.
\end{solution}

\begin{problem}[5]
  Seven balls are randomly withdrawn from an urn that contains 12 red, 16 blue, and 18 green balls.
  \begin{enumerate}
    \item Find the probability that 3 red, 2 blue, and 2 green balls are withdrawn.

    \item Find the probability that at least 2 red balls are withdrawn.

    \item Find the probability that all withdrawn balls are the same color.

    \item Find the probability that either exactly 3 red balls or exactly 3 blue balls are withdrawn.
  \end{enumerate}
\end{problem}

\begin{solution}[(i)]
  The total number of balls in the urn is $12+16+18=46$. The total number of ways to choose 7 balls from 46 is $\binom{46}{7}$. The number of ways to choose 3 red, 2 blue, and 2 green is
  \[%
    \binom{12}{3} \binom{16}{2} \binom{18}{2}
  .\]%
  Hence
  \[%
    P(3R, 2B, 2G) = \frac{\binom{12}{3} \binom{16}{2} \binom{18}{2}}{\binom{46}{7}} = \frac{4,039,200}{53,524,680} \approx 0.0754643
  .\]%
  Therefore $P(3R, 2B, 2G) \approx 0.07546$, or about 7.546\%.
\end{solution}

\begin{solution}[(ii)]
  ``At least 2 red'' means exactly $k$ red for $k = 2,3,4,5,6,7$. For each $k$ the count is
  \[%
    \binom{12}{k} \binom{34}{7-k}
  ,\]%
  so
  \[%
    P(\text{at least 2 red}) = \frac{\sum_{k=2}^{7} \binom{12}{k} \binom{34}{7-k}}{\binom{46}{7}} = \frac{32,006,216}{53,524,680} \approx 0.5979712
  .\]%
  Therefore $P(\text{at least 2 red}) \approx 0.59797$, or about 59.797\%.
\end{solution}

\begin{solution}[(iii)]
  All withdrawn balls the same color can be red, blue, or green:
  \[%
    \binom{12}{7} + \binom{16}{7} + \binom{18}{7}
  .\]%
  Hence
  \[%
    P(\text{all same color}) = \frac{\binom{12}{7} + \binom{16}{7} + \binom{18}{7}}{\binom{46}{7}} = \frac{44,056}{53,524,680} \approx 0.00082310
  .\]%
  Therefore $P(\text{all same color}) \approx 0.0008231$, or about 0.08231\%.
\end{solution}

\begin{solution}[(iv)]
  We have two cases:
  \begin{enumerate}
    \item Exactly 3 red: $\binom{12}{3} \binom{34}{4}$.
    \item Exactly 3 blue: $\binom{16}{3} \binom{30}{4}$.
  \end{enumerate}
  Sum them and divide by $\binom{46}{7}$:
  \[%
    P(\text{exactly 3 red or exactly 3 blue}) = \frac{\binom{12}{3} \binom{34}{4} + \binom{16}{3} \binom{30}{4}}{\binom{46}{7}} = \frac{25,549,520}{53,524,680} \approx 0.4773409
  .\]%
  Therefore $P(\text{exactly 3 red or exactly 3 blue})\approx 0.47734$, or about 47.734\%.
\end{solution}

\begin{problem}[6]
  Compute the probability that a bridge hand is void in at least one suit.

  Note that the answer is not
  \[%
    \frac{\binom{4}{1} \binom{39}{13}}{\binom{52}{13}}
  .\]%
  Note: In bridge, all 52 cards in a standard deck are dealt to four players where each player has 13 cards. A bridge hand is void in a suit (hearts, diamonds, spades, clubs) if the hand does not contain any of the 13 cards in the suit.
\end{problem}

\begin{solution}
  Let $A_i$ be the event that the hand is void in the $i$th suit. We want
  \[%
    P\left(\bigcup_{i=1}^4 A_i\right)
  .\]%
  By inclusion–exclusion,
  \[%
    P\left(\bigcup_{i=1}^4 A_i\right) = \sum_{i} P(A_i) - \sum_{i<j} P(A_i\cap A_j) + \sum_{i<j<k} P(A_i \cap A_j \cap A_k) - P(A_1 \cap A_2 \cap A_3 \cap A_4)
  .\]%
  For any single suit void,
  \[%
    P(A_i) = \frac{\binom{39}{13}}{\binom{52}{13}}
  ,\]%
  for any two suits void,
  \[%
    P(A_i\cap A_j) = \frac{\binom{26}{13}}{\binom{52}{13}}
  ,\]%
  for any three suits void,
  \[%
    P(A_i \cap A_j \cap A_k) = \frac{\binom{13}{13}}{\binom{52}{13}} = \frac{1}{\binom{52}{13}}
  ,\]%
  and all four void is impossible. Hence
  \[%
    P(\text{void in at least one suit}) = 4\frac{\binom{39}{13}}{\binom{52}{13}} - 6\frac{\binom{26}{13}}{\binom{52}{13}} + 4\frac{1}{\binom{52}{13}}
  .\]%
  Substituting exact values,
  \[%
    P(\text{void in at least one suit}) = \frac{4\cdot 8,122,425,444 - 6\cdot 10,400,600 + 4}{635,013,559,600} = \frac{32,427,298,180}{635,013,559,600} \approx 0.05106552
  .\]%
  Therefore, the probability that a bridge hand is void in at least one suit is approximately $0.05107$, or about 5.1066\%.
\end{solution}

\begin{problem}[7]
  Consider two people being randomly selected. For simplicity of this problem, ignore leap years.
  \begin{enumerate}
    \item Find the probability that two people have a birthday on the 6th of any month.

    \item What is the probability that two people have a birthday on the same day of the same month?
  \end{enumerate}
\end{problem}

\begin{solution}[(i)]
  The probability that one person has a birthday on the 6th of any month is $12/365$, since there are 12 months in a year. Assuming birthdays are independent, the probability that both people have birthdays on the 6th of any month is
  \[%
    P(\text{both on 6th}) = \left(\frac{12}{365}\right)^2 = \frac{144}{133,225} \approx 0.00108
  .\]%
  Therefore, the probability that two people have birthdays on the 6th of any month is approximately $0.00108$, or about $0.108\%$.
\end{solution}

\begin{solution}[(ii)]
  The probability that one person has a birthday on a specific day of the year is $1/365$. For two people, the probability that both have birthdays on the same day of the same month is
  \[%
    P(\text{same birthday}) = \frac{1}{365} \approx 0.00274
  .\]%
  Therefore, the probability that two people share the same birthday is approximately $0.00274$, or about $0.274\%$.
\end{solution}
