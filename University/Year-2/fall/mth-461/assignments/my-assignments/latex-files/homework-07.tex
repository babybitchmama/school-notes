\begin{problem}[1]
  The joint probability density function of $X$ and $Y$ is given by
  \[%
    f(x, y) = \begin{cases}
      c(y^2 - x^2)e^{-y} & -y \le x \le y, 0 < y < \infty, \\
      0 & \text{otherwise}.
    \end{cases}
  \]%
  \begin{enumerate}
    \item Find $c$.
    \item Find the marginal densities of $X$ and $Y$.
    \item Find $E[X]$.
  \end{enumerate}
\end{problem}

\begin{solution}[(i)]
  In order to find $c$, we use the fact that the total probability must equal 1. Thus, we have
  \begin{align*}
    1 &= \iint_{-y \le x \le y,\, 0 < y < \infty} c(y^2 - x^2)e^{-y} \dx \dy \\
      &= \int_0^\infty c e^{-y} \int_{x=-y}^y (y^2 - x^2) \dx \dy \\
      &= \int_0^\infty c e^{-y} \left.\left(y^2 x - \frac{x^3}{3}\right)\right\rvert_{x=-y}^{x=y} \dy \\
      &= \frac{4c}{3} \int_0^\infty y^3 e^{-y} \dy \\
      &= \frac{4c}{3} \cdot \Gamma(4) = 8c
  .\end{align*}
  Thus, we have $c = 1/8$.
\end{solution}

\begin{solution}[(ii)]
  The marginal densities of $X$ and $Y$ are given by

  \noindent\begin{minipage}{.5\linewidth}
    \begin{align*}
      f_X(x) &= \int_{y=\lvert x\rvert}^\infty f(x,y) \dy \\
        &= \int_{\lvert x\rvert}^\infty \frac{1}{8}(y^2 - x^2)e^{-y} \dy \\
        &= \frac{1}{8} e^{-\lvert x\rvert} \left( (\lvert x\rvert^2 - x^2) + 2\lvert x\rvert + 2 \right) \\
        &= \frac{1}{4} (\lvert x\rvert + 1)e^{-\lvert x\rvert}
    ,\end{align*}
  \end{minipage}
  \begin{minipage}{.495\linewidth}
    \begin{align*}
      f_Y(y) &= \int_{x=-y}^y f(x,y) \dx \\
        &= \int_{-y}^y \frac{1}{8}(y^2 - x^2)e^{-y} \dx \\
        &= \frac{1}{8} e^{-y} \left( y^3 - \frac{y^3}{3} - (-y^3 + \frac{y^3}{3}) \right) \\
        &= \frac{1}{3} y^3 e^{-y}
    .\qedhere\end{align*}
  \end{minipage}
\end{solution}

\begin{solution}[(iii)]
  To find $E[X]$, we compute
  \begin{align*}
    E[X] &= \int_{-\infty}^\infty x f_X(x) \dx \\
      &= \int_{-\infty}^\infty x \cdot \frac{1}{4} (|x| + 1)e^{-|x|} \dx \\
      &= 0,
  \end{align*}
  since the integrand is an odd function.
\end{solution}

\begin{problem}[2]
  The random vector $(X, Y)$ is said to be uniformly distributed over a region $R$ in the plane if its joint probability density is
  \[%
    f(x, y) = \begin{cases}
      \frac{1}{A} & (x, y) \in R, \\
      0 & \text{otherwise},
    \end{cases}
  \]%
  where $A$ is the area of region $R$. Therefore if $B$ is any subset of R with area $a$, then $P\{(X, Y) \in B\} = a/A$ and all regions of equal area are equally likely to contain a randomly selected point, $(x, y)$

  Suppose $(X, Y)$ is uniformly distributed over the square centered at $(0, 0)$ and with sides of length 2.
  \begin{enumerate}
    \item Show that $X$ and $Y$ are independent, with each being distributed uniformly over $(-1, 1)$.
    \item What is the probability that $(X, Y)$ lies in the circle of radius 1 centered at the origin? That is, find $P\{X^2 + Y^2 \le 1\}$.
  \end{enumerate}
\end{problem}

\begin{solution}[(i)]
  The square is centered at $(0, 0)$ with side length $2$, so the region $R$ is
  \[%
    R = \{(x, y) \mid -1 \le x \le 1,\ -1 \le y \le 1\}
  .\]%
  Its area is $A = 2 \cdot 2 = 4$. Therefore the joint density of $(X, Y)$ is
  \[%
    f(x, y) = \frac{1}{4}
  ,\]%
  for $-1 \le x \le 1$ and $-1 \le y \le 1$, and $f(x, y) = 0$ otherwise.

  The marginal densities of $X$ and $Y$ are given by
  \begin{align*}
    f_X(x) &= \int_{-1}^1 f(x, y) \dy = \int_{-1}^1 \frac{1}{4} \dy = \frac{1}{2} \\
    f_Y(y) &= \int_{-1}^1 f(x, y) \dx = \int_{-1}^1 \frac{1}{4} \dx = \frac{1}{2}
  .\end{align*}
  Since
  \[%
    f(x,y) = \frac{1}{4} = \left(\frac{1}{2}\right)\left(\frac{1}{2}\right) = f_X(x)\,f_Y(y)
  ,\]%
  the joint density factors as the product of the marginals. Therefore $X$ and $Y$ are independent, each uniformly distributed over $(-1, 1)$.
\end{solution}

\begin{solution}[(ii)]
  We want the probability that $(X, Y)$ lies inside the circle of radius $1$ centered at the origin. Since $(X, Y)$ is uniformly distributed over the square of area $4$, the probability that it falls in any region is the area of that region divided by $4$. The set $\{(x, y) \mid X^2 + Y^2 \le 1\}$ is the disk of radius $1$, whose area is $\pi(1)^2 = \pi$.
\end{solution}

\begin{problem}[3]
  Let $\displaystyle f(x, y) = \begin{cases}
    24xy & 0 \leq x \leq 1, 0 \leq y \leq 1, 0 \le x + y \le 1, \\
    0 & \text{otherwise}.
  \end{cases}$
  \begin{enumerate}
    \item Show that $f(x, y)$ is a joint probability density function.
    \item Find $E[Y]$.
  \end{enumerate}
\end{problem}

\begin{solution}[(i)]
  For the first condition, clearly $f(x, y) \ge 0$ over the given bounds.

  For the second condition, we need to verify that the total integral of $f(x, y)$ over its bounds, $R$, equals $1$
  \begin{align*}
    \iint_R f(x, y) \dx \dy &= \int_{y=0}^1 \int_{x=0}^{1-y} 24xy \dx \dy \\
                          &= \int_0^1 12(1-y)^2 y \dy \\
                          &= 12 \int_0^1 (y - 2y^2 + y^3) \dy \\
                          &= 12\!\left[\frac{y^2}{2} - \frac{2y^3}{3} + \frac{y^4}{4}\right]_0^1 = 12\!\left(\tfrac{1}{2} - \tfrac{2}{3} + \tfrac{1}{4}\right) = 1
  .\end{align*}
  Therefore, $f(x, y)$ is a valid joint probability density function.
\end{solution}

\begin{solution}[(ii)]
  First, we need to find the marginal density of $Y$
  \[%
    f_Y(y) = \int_0^{1-y} 24xy \dx = 12(1 - y)^2 y
  .\]%
  Then, we can compute $E[Y]$ as follows
  \begin{align*}
    E[Y] &= \int_0^1 y f_Y(y) \dy \\
         &= \int_0^1 y \cdot 12(1 - y)^2 y \dy \\
         &= 12 \int_0^1 (y^2 - 2y^3 + y^4) \dy \\
         &= 12\!\left[\frac{y^3}{3} - \frac{2y^4}{4} + \frac{y^5}{5}\right]_0^1 = 12\!\left(\tfrac{1}{3} - \tfrac{1}{2} + \tfrac{1}{5}\right) = \frac{2}{5}
  .\qedhere\end{align*}
\end{solution}

\begin{problem}[4]
  The joint density function of $X$ and $Y$ is $\displaystyle f(x, y) =
  \begin{cases}
    x + y & 0 < x < 1, 0 < y < 1, \\
    0 & \text{otherwise}.
  \end{cases}$
  \begin{enumerate}
    \item Are $X$ and $Y$ independent?
    \item Find $P\{X + Y < 1\}$.
  \end{enumerate}
\end{problem}

\begin{solution}[(i)]
  To determine if $X$ and $Y$ are independent, we need to find their marginal densities and see if the joint density factors as the product of the marginals.

  Their marginal densities are given by
  \begin{align*}
    f_X(x) &= \int_0^1 (x + y) \dy = \left[xy + \frac{y^2}{2}\right]_0^1 = x + \frac{1}{2} \\
    f_Y(y) &= \int_0^1 (x + y) \dx = \left[\frac{x^2}{2} + yx\right]_0^1 = \frac{1}{2} + y
  .\end{align*}

  Notice that
  \[%
    f_X(x) f_Y(y) = \left(x + \frac{1}{2}\right)\left(\frac{1}{2} + y\right) = xy + \frac{x}{2} + \frac{y}{2} + \frac{1}{4} \ne f(x, y)
  .\]%
  Therefore, $X$ and $Y$ are not independent.
\end{solution}

\begin{solution}[(ii)]
  The region $0 < x < 1$, $0 < y < 1$, and $x + y < 1$ is the triangle with vertices at $(0, 0)$, $(1, 0)$, and $(0, 1)$. Integrate in $x$ first, we have
  \begin{align*}
    \int_{y=0}^1 \int_{x=0}^{1-y} (x + y) \dxy &= \int_0^1 \left[\frac{x^2}{2} + yx\right]_{x=0}^{x=1-y} \dy \\
                                               &= \int_0^1 \frac{1 - y^2}{2} \dy \\
                                               &= \left[\frac{y}{2} - \frac{y^3}{6}\right]_0^1 = \frac{1}{2} - \frac{1}{6} = \frac{1}{3}
  .\qedhere\end{align*}
\end{solution}

\begin{problem}[5]
  If $X_1$ and $X_2$ are independent exponential random variables with respective parameters $\lambda_1$ and $\lambda_2$, find the distribution of $Z = X_1/X_2$. Then, compute $P\{X_1 < X_2\}$.
\end{problem}

\begin{solution}
  Let $X_1$ and $X_2$ be independent exponential random variables with parameters $\lambda_1$ and $\lambda_2$, respectively. We define $Z = X_1 / X_2$ and first find its cumulative distribution function. For $z \ge 0$, we have
  \[%
    F_Z(z) = P(Z \le z) = P\left(\frac{X_1}{X_2} \le z\right) = P(X_1 \le z X_2)
  .\]%
  Using independence of $X_1$ and $X_2$, this probability can be written as
  \[%
    F_Z(z) = \int_0^\infty P(X_1 \le z x_2) f_{X_2}(x_2) \dx_2
  .\]%
  The CDF of $X_1$ is $P(X_1 \le t) = 1 - e^{-\lambda_1 t}$, so
  \[%
    F_Z(z) = \int_0^\infty \bigl(1 - e^{-\lambda_1 z x_2}\bigr) \lambda_2 e^{-\lambda_2 x_2} \dx_2 = \int_0^\infty \lambda_2 e^{-\lambda_2 x_2} \dx_2 - \int_0^\infty \lambda_2 e^{-(\lambda_2 + \lambda_1 z)x_2} \dx_2
  .\]%
  Evaluating these integrals gives
  \[%
    \int_0^\infty \lambda_2 e^{-\lambda_2 x_2} \dx_2 = 1 \aand \int_0^\infty \lambda_2 e^{-(\lambda_2 + \lambda_1 z)x_2} \dx_2 = \frac{\lambda_2}{\lambda_2 + \lambda_1 z}
  .\]%
  Therefore, the CDF of $Z$ is
  \[%
    F_Z(z) = 1 - \frac{\lambda_2}{\lambda_2 + \lambda_1 z} = \frac{\lambda_1 z}{\lambda_2 + \lambda_1 z}
  ,\]%
  where $z \ge 0$. Differentiating this CDF with respect to $z$ gives the probability density function of $Z$:
  \[%
    f_Z(z) = \frac{\lambda_1 \lambda_2}{(\lambda_2 + \lambda_1 z)^2}
  ,\]%
  where $z \ge 0$. To compute $P(X_1 < X_2)$, observe that
  \[%
    P(X_1 < X_2) = P\left(\frac{X_1}{X_2} < 1\right) = F_Z(1) = \frac{\lambda_1}{\lambda_1 + \lambda_2}
  .\]%
  In summary, the distribution of $Z$ has PDF $f_Z(z) = (\lambda_1\lambda_2)/(\lambda_2 + \lambda_1 z)^2$ for $z \ge 0$, and the probability that $X_1$ is less than $X_2$ is $\lambda_1/(\lambda_1 + \lambda_2)$.
\end{solution}

\begin{problem}[6]
  The gross weekly sales at a certain restaurant area normal random variable with mean \$2200 and standard deviation \$230. What is the probability the total gross sales over the next 2 weeks exceeds \$5000?
\end{problem}

\begin{solution}
  Let $X_1$ and $X_2$ denote the gross weekly sales for the next two weeks. Since each week is normally distributed with mean $2200$ and standard deviation $230$, we have
  \[%
    X_1, X_2 \sim N(2200, 230^2) \aand X_1 + X_2 \sim N(4400,\, 2\cdot 230^2)
  .\]%
  The total sales over two weeks is therefore
  \[%
    T = X_1 + X_2 \sim N(4400,\, (230\sqrt{2})^2)
  .\]%
  We compute
  \[%
    P(T > 5000) = P\!\left( Z > \frac{5000 - 4400}{230\sqrt{2}} \right) = P\!\left( Z > \frac{600}{230\sqrt{2}} \right) = P(Z > 1.844\ldots)
  .\]%
  Using standard normal tables,
  \[%
    P(Z > 1.844) \approx 0.033
  .\]%
  Therefore, the probability that the total gross sales exceed \$5000 is approximately $0.033$.
\end{solution}

\begin{problem}[7]
  The monthly worldwide average number of airplane crashes of commercial airlines is 2.2 What is the probability that there will be
  \begin{enumerate}
    \item more than 2 such accidents in the next month?
    \item more than 4 such accidents in the next 2 months?
  \end{enumerate}
\end{problem}

\begin{solution}[(i)]
  The number of airplane crashes in a month is modeled as a Poisson random variable with parameter $\lambda = 2.2$. We want to find
  \[%
    P(X > 2) = 1 - P(X \le 2)
  ,\]%
  where $X \sim \mathrm{Poisson}(2.2)$. Compute the partial sum
  \[%
    P(X \le 2) = e^{-2.2} \left(\frac{2.2^0}{0!} + \frac{2.2^1}{1!} + \frac{2.2^2}{2!}\right) = e^{-2.2}(5.62) \approx 0.1108
  .\]%
  Therefore, $P(X > 2) = 1 - 0.6227 \approx 0.3773$.
\end{solution}

\begin{solution}[(ii)]
  Over 2 months, the Poisson parameter doubles, so the number of crashes in two months is $Y \sim \mathrm{Poisson}(4.4)$. We want $P(Y > 4) = 1 - P(Y \le 4)$. We compute
  \[%
    P(Y \le 4) = e^{-4.4} \left(\frac{4.4^0}{0!} + \frac{4.4^1}{1!} + \frac{4.4^2}{2!} + \frac{4.4^3}{3!} + \frac{4.4^4}{4!}\right) = 44.894 \cdot e^{-4.4} \approx 0.5522
  .\]%
  Therefore, $P(Y > 4) = 1 - 0.5522 \approx 0.4478$.
\end{solution}
