\begin{problem}[1]
  How many 4-digit numbers, divisible by five, are there?
\end{problem}

\begin{solution}
  A number is divisible by 5 iff its last digit is either 0 or 5. The first 3 digits can be any digit from 0 to 9 (with the exception that the first digit cannot be 0, since we want a 4-digit number). Therefore, we have
  \[%
    9 \times 10 \times 10 \times 2 = 1,800
  .\]%
  Thus, there are 1,800 4-digit numbers divisible by 5.
\end{solution}

\begin{problem}[2]
  For years, telephone area codes in the United States and Canada consisted of a sequence of three digits. The first digit was an integer between 2 and 9, the second digit was either 0 or 1, and the third digit was an integer between 1 and 9.
  \begin{enumerate}
    \item How many area codes were possible?
    \item How many area codes starting with a 5 were possible?
  \end{enumerate}
\end{problem}

\begin{solution}[(i)]
  The first digit can be any integer from 2 to 9, so there are 8 choices for the first digit. The second digit can be either 0 or 1, so there are 2 choices for the second digit. The third digit can be any integer from 1 to 9, so there are 9 choices for the third digit. Therefore, the total number of possible area codes is
  \[%
    8 \times 2 \times 9 = 144
  .\]%
  Thus, there are 144 possible area codes.
\end{solution}

\begin{solution}[(ii)]
  If the area code starts with a 5, then there is only 1 choice for the first digit. All the other choices remain the same as in part (i). Therefore, the total number of possible area codes starting with a 5 is
  \[%
    1 \times 2 \times 9 = 18
  .\]%
  Thus, there are 18 possible area codes starting with a 5.
\end{solution}

\begin{problem}[3]
  How many different letter arrangements can be made from the letters
  \begin{enumerate}
    \item COMMITTEE
    \item PROPOSITION
    \item BARRACUDA
  \end{enumerate}
\end{problem}

\begin{solution}[(i)]
  The word \textit{COMMITTEE} has 9 letters in total. The letters and their frequencies are
  \[%
    \text{C:}~1,\quad \text{O:}~1,\quad \text{M:}~2,\quad \text{I:}~1,\quad \text{T:}~2,\quad \text{E:}~2
  .\]%
  Therefore, the number of distinct arrangements is
  \[%
    \frac{9!}{2!\,2!\,2!} = \frac{362,880}{8} = 45,360
  .\]%
  Thus, there are 45,360 different letter arrangements of the word \textit{COMMITTEE}.
\end{solution}

\begin{solution}[(ii)]
  The word \textit{PROPOSITION} has 1 letters in total. The letters and their frequencies are as follows:
  \[%
    \text{P:}~1,\quad \text{R:}~1,\quad \text{O:}~3,\quad \text{S:}~1,\quad \text{I:}~2,\quad \text{T:}~1,\quad \text{N:}~1
  .\]%
  Therefore, the number of distinct arrangements is
  \[%
    \frac{11!}{3!\,2!} = \frac{39,916,800}{12} = 3,326,400
  .\]%
  Thus, there are 3,326,400 different letter arrangements of the word \textit{PROPOSITION}.
\end{solution}

\begin{solution}[(iii)]
  The word \textit{BARRACUDA} has 9 letters in total. The letters and their frequencies are
  \[%
    \text{B:}~1,\quad \text{A:}~3,\quad \text{R:}~2,\quad \text{C:}~1,\quad \text{U:}~1,\quad \text{D:}~1
  .\]%
  Therefore, the number of distinct arrangements is
  \[%
    \frac{9!}{3!\,2!} = \frac{362,880}{12} = 30,240
  .\]%
  Thus, there are 30,240 different letter arrangements of the word \textit{BARRACUDA}.
\end{solution}

\begin{problem}[4]
  In how many ways can 10 people be seated in a row if
  \begin{enumerate}
    \item there are no restrictions on the seating arrangement?
    \item persons A and B must sit next to each other?
    \item there are 5 men and 5 women and no 2 men or 2 women can sit next to each other?
    \item there are 6 men and they must sit next to one another?
    \item there are 5 married couples and each couple must sit together?
  \end{enumerate}
\end{problem}

\begin{solution}[(i)]
  With no restrictions, all 10 people are distinct and can be arranged in a row in $10! = 3,628,800$ ways.
\end{solution}

\begin{solution}[(ii)]
  If persons A and B must sit next to each other, treat \{A, B\} as a single block. Then we have the block plus the other 8 people, i.e., 9 items to arrange, and inside the block A and B can be ordered in 2! ways. Thus the total number is $9! \times 2! = 362,880 \times 2 = 725,760$.
\end{solution}

\begin{solution}[(iii)]
  To ensure no two men or two women sit next to each other with 5 men and 5 women, the seats must alternate man-woman-man-$\cdots$ or woman-man-woman-$\cdots$. There are 2 possible patterns (start with a man or start with a woman). For a fixed pattern, the 5 men can be arranged among the men-positions in 5! ways and the 5 women in 5! ways. Hence the total number is $2 \times 5! \times 5! = 2 \times 120 \times 120 = 28,800$.
\end{solution}

\begin{solution}[(iv)]
  If the 6 men must sit next to one another, treat the block of 6 men as a single item. Together with the remaining 4 people this gives 5 items to arrange in a row, so 5! orderings of the items. Inside the men-block the 6 men may be arranged in 6! ways. Thus the total number is $5! \times 6! = 120 \times 720 = 86,400$.
\end{solution}

\begin{solution}[(v)]
  If each of the 5 married couples must sit together, treat each couple as a block. Then there are 5 blocks to arrange, giving 5! orderings, and within each couple the two spouses may be arranged in 2! ways independently. Therefore the total number is $5! \times (2!)^{5} = 120 \times 32 = 3,840$.
\end{solution}

\begin{problem}[5]
  A student has to sell 3 books from a collection of 10 math, 4 science, and 3 economics books. How many choices are possible if
  \begin{enumerate}
    \item all three books are to be on the same subject?
    \item the books are to be on different subjects?
  \end{enumerate}
\end{problem}

\begin{solution}[(i)]
  If all three books are on the same subject, then the student can choose
  \[%
    \binom{10}{3}~\text{from math,} \quad \binom{4}{3}~\text{from science, or}\quad \binom{3}{3}~\text{from economics}
  .\]%
  Therefore, the total number of ways is
  \[
    \binom{10}{3} + \binom{4}{3} + \binom{3}{3} = 120 + 4 + 1 = 125
  .\]%
  Thus, there are 125 possible choices if all three books are on the same subject.
\end{solution}

\begin{solution}[(ii)]
  If the three books are to be on different subjects, then one must be chosen from each subject category. The number of choices is therefore
  \[
    \binom{10}{1} \times \binom{4}{1} \times \binom{3}{1} = 10 \times 4 \times 3 = 120.
  \]
  Thus, there are 120 possible choices if the three books are on different subjects.
\end{solution}

\begin{problem}[6]
  Six different gifts are to be distributed among 12 children. How many distinct results are possible if no child is to receive more than one gift?
\end{problem}

\begin{solution}
  Since no child can receive more than one gift, we must select 6 children from the 12 to receive the gifts, and then assign one distinct gift to each of them.

  First, choose which 6 children will receive gifts,
  \[%
    \binom{12}{6}~\text{ways}
  .\]%
  Then, distribute the 6 distinct gifts among these 6 children, 6! ways. Therefore, the total number of distinct results is
  \[%
    \binom{12}{6} \times 6! = \frac{12!}{6!}
  .\]%
  Simplifying,
  \[%
    \frac{12!}{6!} = 665,280
  .\]%
  Thus, there are 665,280 distinct ways to distribute the 6 gifts among 12 children if no child receives more than one gift.
\end{solution}

\begin{problem}[7]
  A person has 10 friends, of whom 7 will be invited to a party.
  \begin{enumerate}
    \item How many choices are there if 2 of the friends are feuding and will not attend together?
    \item How many choices if 2 of the friends will only attend together?
  \end{enumerate}
\end{problem}

\begin{solution}[(i)]
  Let the two feuding friends be \(A\) and \(B\). We must count the number of ways to invite \(7\) friends such that \(A\) and \(B\) are not both invited.

  There are two cases:
  \begin{enumerate}
    \item A is invited and B is not: choose 6 more from the remaining 8 friends, giving $\binom{8}{6}$ ways.
    \item B is invited and A is not: again, $\binom{8}{6}$ ways.
    \item Neither A nor B is invited: choose all 7 from the remaining 8 friends, giving $\binom{8}{7}$ ways.
  \end{enumerate}
  Therefore, the total number of choices is
  \[%
    \binom{8}{6} + \binom{8}{6} + \binom{8}{7} = 2 \times 28 + 8 = 64
  .\]%
  Thus, there are 64 possible choices if the two feuding friends will not attend together.
\end{solution}

\begin{solution}[(ii)]
  Let the two friends who will only attend together be \(A\) and \(B\). We can treat them as a single unit, so that \(A\) and \(B\) either both attend or both do not.

  If they both attend, then this pair counts as one “block,” so we need 5 additional friends from the remaining 8,
  \[%
    \binom{8}{5} = 56~\text{ways}
  .\]%
  If they both do not attend, we must choose all 7 from the remaining 8 friends:
  \[%
    \binom{8}{7} = 8~\text{ways}
  .\]%
  Therefore, the total number of possible invitations is
  \[%
    \binom{8}{5} + \binom{8}{7} = 56 + 8 = 64
  .\]%
  Thus, there are 64 possible choices if the two friends will only attend together.
\end{solution}
