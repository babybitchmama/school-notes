\begin{problem}[1]
  Five men and 5 women are ranked according to their scores on an examination. Assume that no two scores are alike and all 10! possible rankings are equally likely. Let $X$ denote the highest ranking achieved by a woman. (For example, $X = 1$ if the top–ranked person of the group of 10 is female). Find $P\{X = i\}$, for $i = 1, 2, \cdots, 10$.
\end{problem}

\begin{solution}
  Let $X$ denote the highest ranking achieved by a woman. The event $\{X = i\}$ means that all of the first $(i-1)$ positions are occupied by men and the $i$th position is occupied by a woman. Since there are $5$ men and $5$ women, there can be at most $5$ men before the first woman, so $i$ can range only from $1$ to $6$. 

  To count the number of favorable rankings, we first choose which $(i-1)$ men occupy the top $(i-1)$ positions in $\binom{5}{i-1}$ ways and arrange them in $(i-1)!$ ways. Then we choose one of the $5$ women to occupy position $i$, and finally arrange the remaining $(10 - i)$ people in $(10 - i)!$ ways. Thus the number of favorable rankings is 
  \[%
    N_i = \binom{5}{i-1}(i-1)! \cdot 5 \cdot (10 - i)!
  .\]%
  Since all $10!$ rankings are equally likely, 
  \[%
    P\{X = i\} = \frac{N_i}{10!} = \frac{\binom{5}{i-1}(i-1)! \cdot 5 \cdot (10 - i)!}{10!}
  .\]%
  Simplifying gives
  \[%
    P\{X = i\} = \frac{\binom{10 - i}{4}}{\binom{10}{5}}, \quad i = 1, 2, \cdots, 6
  ,\]%
  and $P\{X = i\} = 0$ for $i = 7, 8, 9, 10$.
\end{solution}

\begin{problem}[2]
  Suppose that a fair die is rolled twice. What are the possible values that the following random variables can take on? Calculate the probabilities associated with the random variables and define the probability mass function.
  \begin{enumerate}
    \item he maximum value to appear in the two rolls
    \item the minimum value to appear in the two rolls
    \item the sum of the two rolls
    \item the value of the first roll minus the value of the second roll
  \end{enumerate}
\end{problem}

\begin{solution}[(i)]
  Let $X$ denote the maximum of the two rolls. The possible values are $1, 2, \cdots, 6$. We have
  \[%
    P(X \le i) = \left(\frac{i}{6}\right)^2
  ,\]%
  so
  \[%
    P(X = i) = P(X \le i) - P(X \le i-1) = \left(\frac{i}{6}\right)^2 - \left(\frac{i-1}{6}\right)^2 = \frac{2i - 1}{36}, \quad i = 1, 2, \cdots, 6
  .\]%
  Hence the pmf is
  \[%
    p_X(i) = \frac{2i - 1}{36}, \quad i = 1, 2, \cdots, 6
  .\qedhere\]%
\end{solution}

\begin{solution}[(ii)]
  Let $Y$ denote the minimum of the two rolls. The possible values are again $1, 2, \cdots, 6$. We have
  \[%
    P(Y \ge i) = \left(\frac{7 - i}{6}\right)^2
  ,\]%
  so
  \[%
    P(Y = i) = P(Y \ge i) - P(Y \ge i+1) = \left(\frac{7 - i}{6}\right)^2 - \left(\frac{6 - i}{6}\right)^2 = \frac{13 - 2i}{36}, \quad i = 1, 2, \cdots, 6
  .\]%
  Hence the pmf is
  \[%
    p_Y(i) = \frac{13 - 2i}{36}, \quad i = 1, 2, \cdots, 6
  .\qedhere\]%
\end{solution}

\begin{solution}[(iii)]
  Let $Z$ denote the sum of the two rolls. The possible values are $2, 3, \cdots, 12$. The number of ways to obtain each sum $z$ is:
  \[%
    \begin{array}{c|ccccccccccc}
      z & 2 & 3 & 4 & 5 & 6 & 7 & 8 & 9 & 10 & 11 & 12 \\
      \hline
      \text{\# of outcomes} & 1 & 2 & 3 & 4 & 5 & 6 & 5 & 4 & 3 & 2 & 1
    \end{array}
  \]%
  Since each of the $36$ outcomes is equally likely,
  \[%
    p_Z(z) = \frac{\text{\# of outcomes}}{36}, \quad z = 2, 3, \cdots, 12
  .\qedhere\]%
\end{solution}

\begin{solution}[(iv)]
  Let $W$ denote the value of the first roll minus the value of the second roll. The possible values are $-5, -4, -3, -2, -1, 0, 1, 2, 3, 4, 5$.

  For a given difference $w$, the number of outcomes satisfying $\text{first} - \text{second} = w$ is $(6 - |w|)$ for $|w| \le 5$. Thus,
  \[%
    p_W(w) = \frac{6 - |w|}{36}, \quad w = -5, -4, -3, -2, -1, 0, 1, 2, 3, 4, 5
  .\qedhere\]%
\end{solution}

\begin{problem}[3]
  A gambling book recommends the following winning strategy for roulette : Bet \$1 on red. If red appears (which has probability $18/38$), then take the \$1 profit and quit. If red does not appear and you lose this bet (which has probability $20/38$ of occurring), make additional \$1 bets on red on each of the next two spins of the roulette wheel and then quit. Let $X$ denote your winnings when you quit
  \begin{enumerate}
    \item Find $P\{X > 0\}$
    \item Find $E[X]$
    \item Now, look at your previous calculations. Are you convinced that the strategy is indeed a“winning” strategy? Briefly justify.
  \end{enumerate}
\end{problem}

\begin{solution}[(i)]
  Let $p = \frac{18}{38} = \frac{9}{19}$ denote the probability that red appears on any given spin, and $q = 1 - p = \frac{10}{19}$ the probability that it does not.

  The random variable $X$ (total winnings) can take only three possible values:
  \[%
    X \in \{1, -1, -3\}
  .\]%
  \begin{itemize}
    \item $X = 1$: occurs if you win the first bet, or if you lose the first but win both of the next two.
    \item $X = -1$: occurs if you lose the first and win exactly one of the next two.
    \item $X = -3$: occurs if you lose all three bets.
  \end{itemize}

  Hence,
  \[%
    \begin{aligned}
      P(X = 1) &= p + q p^2 \\
      P(X = -1) &= q \cdot 2p(1 - p) \\
      P(X = -3) &= q (1 - p)^2
    \end{aligned}
  .\]%

  Substituting $p = 9/19$ and $q = 10/19$ gives
  \[%
    \begin{aligned}
      P(X = 1) &= \frac{4059}{6859} \approx 0.5918 \\
      P(X = -1) &= \frac{1800}{6859} \approx 0.2624 \\
      P(X = -3) &= \frac{1000}{6859} \approx 0.1458
    \end{aligned}
  .\]%

  Therefore,
  \[%
    P\{X > 0\} = P(X = 1) = p + q p^2 = \frac{4059}{6859} \approx 0.5918
  .\qedhere\]%
\end{solution}

\begin{solution}[(ii)]
  The expected winnings are
  \[%
    E[X] = (1)P(X = 1) + (-1)P(X = -1) + (-3)P(X = -3)
  .\]%
  Substituting the expressions above,
  \[%
    \begin{aligned}
      E[X]
      &= (p + qp^2) - q \cdot 2p(1 - p) - 3q(1 - p)^2 \\
      &= -\frac{39}{361} \approx -0.108
    \end{aligned}
  .\]%

  Alternatively, note that each \$1 bet has expected gain $2p - 1 = -\tfrac{1}{19}$. The expected number of bets is $1$ if you win the first, or $3$ if you lose the first:
  \[%
    E[\text{\# bets}] = p(1) + q(3) = \frac{39}{19}
  .\]%
  Hence,
  \[%
    E[X] = E[\text{\# bets}] \cdot (-\tfrac{1}{19}) = -\frac{39}{361} \approx -0.108
  .\qedhere\]%
\end{solution}

\begin{solution}[(iii)]
  Although the probability of winning money is fairly high ($P\{X > 0\} \approx 0.592$), the expected value of your winnings is negative ($E[X] \approx -0.108$). This means that on average, you lose about 10.8 cents per play.

  The strategy therefore is \emph{not} a winning strategy. The reason it appears attractive is that you win small amounts often but lose larger amounts less frequently — resulting in a negative expected return overall.
\end{solution}

\begin{problem}[4]
  A person tosses a fair coin until a tail appears for the first time. If the tail appears on the $n$-th flip, the person wins $2n$ dollars. Let $X$ denote the player's winnings. Show that $E[X] = \infty$. This problem is known as the St. Petersburg paradox.
  \begin{enumerate}
    \item Would you be willing to pay \$1 million to play this game once?
    \item Would you be willing to pay \$1 million for each game if you could play for as long as you liked and only had to settle up when you stopped playing?
  \end{enumerate}
\end{problem}

\begin{solution}[(i)]
  Let $X$ be the payoff. The event ``first tail on the $n$-th flip'' has probability
  \[%
    P(\text{first tail at}~n) = \left(\tfrac12\right)^{n}, \qquad n=1,2,\cdots
  .\]%
  (we need $n-1$ heads followed by a tail). If the payoff is $2^{n}$, then
  \[%
    E[X] = \sum_{n=1}^{\infty} 2^{n} \cdot P(\text{first tail at}~n) = \sum_{n=1}^{\infty} 2^{n} \left(\tfrac12\right)^{n} = \sum_{n=1}^{\infty} 1 = \infty
  .\]%
  Thus the expected payoff is infinite.
\end{solution}

\begin{solution}[(ii)]
  I believe most people would \emph{not} pay \$1 million to play once since real people have finite wealth and cannot afford to risk such a large amount for a game with highly skewed outcomes. The distribution places very large probability on relatively small payouts and extremely tiny probability on astronomically large payouts that drive the expectation to infinity. Paying \$1M up front exposes you to a very large downside relative to the likely small rewards.
\end{solution}

\begin{solution}[(iii)]
  If I could play for as long as I liked and only had to settle up when I stopped playing, I might be more willing to pay \$1 million. This is because I could choose to stop playing after a few rounds if I had already won a substantial amount, thereby limiting my downside risk. However, the infinite expected value still suggests that over many plays, the average payoff would be very high, so I would need to carefully consider my risk tolerance and financial situation before committing to such a strategy.
\end{solution}

\begin{problem}[5]
  A box contains 5 red and 5 blue marbles. Two marbles are withdrawn randomly. If they are the same color,
  then you win \$1.50; if they are different colors, then you win -\$1.00. (That is, you lose \$1.00)
  \begin{enumerate}
    \item Calculate the expected value of the amount you win.
    \item Calculate the variance of the amount you win.
  \end{enumerate}
\end{problem}

\begin{solution}[(i)]
  Let $X$ denote your winnings from the game. There are $\binom{10}{2} = 45$ equally likely pairs of marbles.
  \begin{gather*}
    P(\text{same color}) = \frac{\binom{5}{2} + \binom{5}{2}}{\binom{10}{2}} = \frac{20}{45} = \frac{4}{9} \\
    P(\text{different colors}) = 1 - \frac{4}{9} = \frac{5}{9}
  .\end{gather*}

  The possible values of $X$ are:
  \[%
    X = \begin{cases}
      1.50, & \text{with probability}~\frac{4}{9} ,\\[3pt]
      -1.00, & \text{with probability}~\frac{5}{9}.
    \end{cases}
  \]%
  Hence the expected value is
  \[%
    E[X] = (1.50)\left(\frac{4}{9}\right) + (-1.00)\left(\frac{5}{9}\right) = \frac{6 - 5}{9} = \frac{1}{9} \approx 0.111~\text{dollars}
  .\]%
  Thus, on average you win about \$0.11 per play.
\end{solution}

\begin{solution}[(ii)]
  First compute $E[X^2]$:
  \[%
    E[X^2] = (1.50)^2\left(\frac{4}{9}\right) + (-1.00)^2\left(\frac{5}{9}\right) = \frac{2.25(4) + 5}{9} = \frac{9 + 5}{9} = \frac{14}{9}
  .\]%
  Then
  \[%
    \operatorname{Var}(X) = E[X^2] - (E[X])^2 = \frac{14}{9} - \left(\frac{1}{9}\right)^2 = \frac{14}{9} - \frac{1}{81} = \frac{125}{81} \approx 1.543
  .\]%
  Hence the variance of your winnings is $\displaystyle \frac{125}{81}$, and the standard deviation is $\sqrt{125/81} \approx 1.242.$
\end{solution}

\begin{problem}[6]
  Let $X$ be a random variable such that $\displaystyle P\{X = n\} = \left(\frac{1}{2}\right)^n$; $n = 1, 2, \cdots$. Find $E[X]$.
\end{problem}

\begin{solution}
  First, note that the given probabilities must sum to 1 for a valid probability mass function:
  \[%
    \sum_{n=1}^{\infty} P\{X = n\} = \sum_{n=1}^{\infty} \left(\frac{1}{2}\right)^n = 1
  .\]%
  Thus the pmf is already normalized.

  Now compute the expected value:
  \[%
    E[X] = \sum_{n=1}^{\infty} n \left(\frac{1}{2}\right)^n
  .\]%
  Recall the power series identity for $|r| < 1$:
  \[%
    \sum_{n=1}^{\infty} n r^{n} = \frac{r}{(1 - r)^2}
  .\]%
  Substituting $r = 1/2$ gives
  \[%
    E[X] = \frac{\tfrac{1}{2}}{(1 - \tfrac{1}{2})^2} = \frac{\tfrac{1}{2}}{(\tfrac{1}{2})^2} = 2
  .\]%
  Therefore, the expected value of $X$ is $E[X] = 2$.
\end{solution}
