In analysis, we defined properties such as \emph{convergence of sequences} and \emph{continuity of functions}. For example, $(x_n) \to x$ means
\[%
  (\forall \epsilon > 0)(\exists N \in \N)(\forall n \ge N)[|x_n - x| < \epsilon]
.\]%
A function $f : A \to B$ is continuous at $x_0 \in A$ if
\[%
  (\forall \epsilon > 0)(\exists \delta > 0)(\forall x \in A)[|x - x_0| < \delta \implies |f(x) - f(x_0)| < \epsilon]
.\]%
However, these definitions are not very useful when we want to generalize them to other settings, such as metric spaces or topological spaces, because what does $|x - x_n|$ mean if $x$ and $x_n$ aren’t numbers? To define convergence more generally, we don’t need to rely on the traditional distance in Euclidean space. We just need a sensible notion of distance between two points in a set. This leads us to the concept of a \emph{metric space}.

Given a set $X$, a \emph{metric} on $X$ is a function $\metric_X : X \times X \to \R$, where $\metric_X(x, y)$ represents the distance between two arbitrary points $x, y \in X$. Now we can replace the traditional distance $|x - y|$ with $\metric_X(x, y)$ in our definitions. So we get
\[%
  (x_n) \to x \iff (\forall \epsilon > 0)(\exists N \in \N)(\forall n \ge N)[\metric_X(x_n, x) < \epsilon]
.\]%
Similarly, a function $f : A \to B$ is continuous at $x_0 \in A$ if
\[%
  (\forall \epsilon > 0)(\exists \delta > 0)(\forall x \in A)[\metric_A(x, x_0) < \delta \implies \metric_B(f(x), f(x_0)) < \epsilon]
,\]%
where $\metric_A$ and $\metric_B$ are metrics on $A$ and $B$, respectively. Of course, these metrics must satisfy certain axioms to be a valid notion of distance.

Even though metric spaces let us talk about ideas like convergence and continuity in settings far beyond $\R^n$, the notion of distance can still be too restrictive. Sometimes, we only care about which subsets of a space are “open,” not about how far apart points are. This motivates the study of \emph{topological spaces}, which capture the essence of continuity and convergence without referring to distance at all.

Topology can be thought of as the study of the underlying shape or structure of spaces—what remains unchanged when we bend, stretch, or deform them without tearing. It provides a unified language for understanding continuity, connectedness, compactness, and many other concepts that appear across analysis, geometry, and beyond.

These notes will follow that progression: starting with metric spaces, where we still have a notion of distance, and then moving to topological spaces, where we generalize further. Along the way, we’ll revisit familiar analytical ideas in a new light and see how they fit into a broader mathematical framework.

This is the first half of a sequence. In the second half, Introduction to Topology II, we will study topological spaces in more depth, as this course mainly focuses on metric spaces.
