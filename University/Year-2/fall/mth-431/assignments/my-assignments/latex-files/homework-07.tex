\begin{problem}[7.3]
  Let $(X, \metric)$ be a metric space. A function $f : X \to \R$ is called \emph{lower semi-continuous} at a point $p \in X$ if $\lim_{q \to p} f(q) \ge f(p)$, or equivalently, for every $\epsilon > 0$ there is a $\delta > 0$ such that $\metric(p, q) < \delta$ implies $f(q) > f(p) - \epsilon$. The idea is that in the limit, $f$ can only jump down. You can guess what \emph{upper semi-continuous means}.
  \begin{enumerate}
    \item Let $X = \R$ with the usual metric, and consider the floor function $\lfloor x \rfloor$, which returns the greatest integer $\le x$, and the ceiling function $\lceil x \rceil$, which returns the least integer $\ge x$. Which one is lower semi-continuous, and which one is upper semi-continuous?

      (You don’t have to prove it.)

    \item Prove that $f : X \to \R$ is lower semi-continuous (at every point) if and only if it is continuous as a map of topological spaces when the codomain $\R$ is given the lower semi-continuous topology from Example 7.3(c).
  \end{enumerate}
\end{problem}

\begin{solution}[(i)]
\end{solution}

\begin{solution}[(ii)]
\end{solution}

\begin{problem}[7.4]
  Let $(X, \metric)$ be a metric space. A function $f : \R \to X$ is called \emph{continuous from the right} at a point $p \in \R$ if $\lim_{q \to p^+} f(q) = f(p)$: that is, for every $\epsilon > 0$ there is a $\delta > 0$ such that $p \le q < p + \delta$ implies $\metric(f(p), f(q)) < \epsilon$. You can guess what continuous from the left means.

  \begin{enumerate}
    \item Again take $X = \R$ in the usual metric, and consider the floor and ceiling functions. Which one is continuous from the right, and which one is continuous from the left?

      (You don’t have to prove it.)

    \item Prove that a map $f : \R \to X$ is continuous from the right (at every point) if and only if it is continuous as a map of topological spaces when the domain $\R$ is given the lower limit topology from Example 7.3(d).
  \end{enumerate}
\end{problem}

\begin{solution}[(i)]
\end{solution}

\begin{solution}[(ii)]
\end{solution}

\begin{problem}[7.5]
  The countable complement topology on $\R$ is like the finite complement topology, but we say that a subset $U \subset \R$ is open if either the complement $\R \setminus U$ is countable, or $U = \emptyset$.
  \begin{enumerate}
    \item Prove that in this topology, a sequence $p_1, p_2, p_3, \ldots \in \R$ converges to a limit $\ell \in \R$ if and only if it is eventually constant, meaning that there is an $N$ such that for all $n \ge N$ we have $p_n = \ell$.
    \item Give an example of a subset $A \subset \R$ that is not closed in this topology, but nonetheless for every sequence $p_1, p_2, \ldots \in A$ converging to a limit $\ell \in \R$, we have $\ell \in A$. (Thus Proposition 2.10 does not hold in a general topological space.)
    \item Give an example of a metric space $(Y, \metric)$ and a map $f : \R \to Y$ that is not continuous with respect to the co-countable topology on $\R$ and the metric topology on $Y$, but nonetheless for every convergent $p_n \to \ell$ in $\R$, we have $f(p_n) \to f(\ell)$ in $Y$ . (Thus Proposition 1.10 does not hold in a general topological space.)
  \end{enumerate}
\end{problem}

\begin{solution}[(i)]
\end{solution}

\begin{solution}[(ii)]
\end{solution}

\begin{solution}[(iii)]
\end{solution}

\begin{problem}[8.1]
  Let $X$ be a topological space. Let $Y \subset X$, and give $Y$ the subspace topology. Let $A \subset Y$.
  \begin{enumerate}
    \item Prove that if $A$ is closed in $Y$ and $Y$ is closed in $X$, then $A$ is closed in $X$.
    \item Give two examples to show that if $A$ is closed in $Y$ and $Y$ is not closed in $X$, then $A$ may or may not be closed in $X$.
    \item Prove that if $A$ is open in $Y$ and $Y$ is open in $X$, then $A$ is open in $X$.
    \item Give two examples to show that if $A$ is open in $Y$ and $Y$ is not open in $X$, then $A$ may or may not be open in $X$.
    \item Let $A \subset Y$, let $\cl_Y(A)$ denote the closure of $A$ in $X$, and let $\cl_Y(A)$ denote the closure of $A$ in $Y$ . Prove that
      \[%
        \cl_Y(A) = \cl_X(A) \cap Y
      .\]%
    \item Let $A \subset Y$, let $\sint_X(A)$ denote the interior of $A$ as a subset of $X$, and let $\sint_Y(A)$ denote the interior of $A$ as a subset of $Y$. Prove that
      \[%
        \sint_X(A) \subset \sint_Y(A)
      .\]%
    \item Give an example where the inclusion in part (vi) is strict.
  \end{enumerate}
\end{problem}

\begin{solution}[(i)]
\end{solution}

\begin{solution}[(ii)]
\end{solution}

\begin{solution}[(iii)]
\end{solution}

\begin{solution}[(iv)]
\end{solution}

\begin{solution}[(v)]
\end{solution}

\begin{solution}[(vi)]
\end{solution}

\begin{solution}[(vii)]
\end{solution}

\begin{problem}[8.2]
  \begin{enumerate}
    \item Let $X$ be a topological space, and suppose that we can write $X = F_1 \cup \cdots \cup F_n$ , where each $F_i$ is closed. Let $Y$ be another topological space, let $f : X \to Y$, and let $f_i : F_i \to Y$ be the restriction of $f$ to $F_i$: that is, for $x \in F_i$ we set $f_i(x) = f(x)$. Prove that $f$ is continuous if and only if $f_i$ is continuous for all $i$.

      Hint: Use the fact that a map is continuous if and only if the preimage of every closed set is closed.

    \item This is usually applied to show that a piecewise function is continuous. Consider the function $f : [0, 1] \to \R$ defined by
      \[%
        f(x) = \begin{cases}
          0 & \text{if}~x \le 1/3, \\
          3x - 1 & \text{if}~1/3 \le x \le 2/3, \\
          1 & \text{if}~x \ge 2/3.
        \end{cases}
      \]%
      If we wanted to apply part (i) to show that $f$ is continuous, which should sets should we take for the $F_i$?

    \item Give an example of how part (i) can fail if we allow countably many closed sets $F_i$.
  \end{enumerate}
\end{problem}

\begin{solution}[(i)]
\end{solution}

\begin{solution}[(ii)]
\end{solution}

\begin{solution}[(iii)]
\end{solution}

\begin{problem}[9.3]
  \begin{enumerate}
    \item Let $X$ and $Y$ be topological spaces, and let $p : X \times Y \to X$ be the map given by $p(x, y) = x$. Prove that $p$ is continuous.

    \item Write ``Similarly, the map $q : X \times Y → Y$ be the map given by $q(x, y) = y$ is continuous.''
  \end{enumerate}
\end{problem}

\begin{solution}[(i)]
\end{solution}

\begin{solution}[(ii)]
\end{solution}
