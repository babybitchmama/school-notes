\begin{problem}[2.2]
  Let $(X, \metric)$ be a metric space and $A \subset X$. Use Proposition 2.6 to prove that
  \[%
    \partial A = \partial(X \setminus A) = \overline A \cap \overline{X \setminus A}
  .\]%
\end{problem}

\begin{solution}
  Assume $x \in \partial A$. By definition, that means $x \in \overline A \setminus \sint A$. Then, $x \in \overline A$ and $x \notin \sint A$. By Proposition~2.6, we have $x \notin \sint A$ if and only if $x \in \overline{X \setminus A}$. Therefore, $x \in \overline A$ and $x \in \overline{X \setminus A}$, which implies
  \[%
    x \in \overline A \cap \overline{X \setminus A}
  .\]%
  Hence, $\partial A \subset \overline A \cap \overline{X \setminus A}$.

  Now, for the converse. Assume $x \in \overline A \cap \overline{X \setminus A}$. Then, $x \in \overline A$ and $x \in \overline{X \setminus A}$. By Proposition~2.6 again, $x \in \overline{X \setminus A}$ if and only if $x \notin \sint A$. Therefore, $x \in \overline A$ and $x \notin \sint A$, which means $x \in \partial A$. Hence,
  \[%
    \overline A \cap \overline{X \setminus A} \subset \partial A
  .\]%

  Combining both inclusions, we conclude that
  \[%
    \partial A = \overline A \cap \overline{X \setminus A}
  .\]%
  Finally, since the expression is symmetric in $A$ and $X \setminus A$, it follows that
  \[%
    \partial A = \partial(X \setminus A)
  .\qedhere\]%
\end{solution}

\begin{problem}[2.4]
  Define the closed ball
  \[%
    \bar{B}_r(p) = \{q \in X \mid \metric(p, q) \le r\}
  .\]%
  \begin{enumerate}
    \item Prove that $\bar{B}_r(p)$ equals its own closure.

    \item Prove that $\overline{B_r(p)} \subset \bar{B}_r(p)$: that is, the closure of the open ball is contained in the closed ball. But give an example to show that the inclusion can be strict.

      Hint: For the proof, you may quote Exercise 2.3(a). For the example, you might take $X = \Z$ with the usual metric inherited from $\R$, or any set with a discrete metric (Exercise 1.4).
  \end{enumerate}
\end{problem}

\begin{solution}[(i)]
  If $q \in \overline{\bar B_r(p)}$ then for every $s > 0$ there exists $x \in \bar B_r(p)$ with $\metric(q, x) < s$. Thus $\metric(p, q) \le \metric(p, x) + \metric(x, q) \le r + s$ for every $s > 0$, so $\metric(p,q)\le r$ and $q \in \bar B_r(p)$. Conversely, if $q \in \bar B_r(p)$ then for every $s > 0$ we have $q \in B_s(q) \cap \bar B_r(p)$, so $q \in \overline{\bar B_r(p)}$. Hence $\overline{\bar B_r(p)} = \bar B_r(p)$.
\end{solution}

\begin{solution}[(ii)]
  Let $q \in \overline{B_r(p)}$. By definition of closure (or Exercise 2.3(a)), every open neighbourhood of $q$ meets $B_r(p)$. Hence for each $s>0$ there exists a point $x_s \in B_r(p)$ with
  \[%
    \metric(q, x_s) < s \aand \metric(p, x_s) < r
  .\]%
  By the triangle inequality,
  \[%
    \metric(p, q) \le \metric(p, x_s) + \metric(x_s, q) < r + s
  .\]%
  This inequality holds for every $s > 0$. If $d(p, q) > r$ then choosing $s < \metric(p, q) - r$ would contradict $\metric(p, q) < r + s$. Thus we must have $\metric(p + q) \le r$, so $q \in \bar B_r(p)$. Since $q$ was arbitrary in $\overline{B_r(p)}$, we obtain $\overline{B_r(p)} \subset \bar B_r(p)$.

  As for the example where the inclusion is strict, take $X = \Z$ with the metric inherited from $\R$ let $p = 0$ and $r = 1$.
  \[%
    B_1(0) = \{n \in \Z \mid |n| < 1\} = \{0\}
  ,\]%
  so $\overline{B_1(0)} = \{0\}$ (closure in $X$ is still $\{0\}$). But
  \[%
    \bar B_1(0) = \{n \in \Z \mid |n| \le 1\} = \{-1,0,1\}
  .\]%
  Hence $\overline{B_1(0)} = \{0\}\subsetneq\{-1,0,1\} = \bar B_1(0)$, so the inclusion can be strict.
\end{solution}

\begin{problem}[3.1]
  In Example 1.4 we saw three different metrics on $\R^2$. Prove one of the following:
  \begin{enumerate}
    \item A subset $A \subset \R^2$ is open in the Euclidean metric if and only if it is open in the taxicab metric.

    \item A subset $A \subset \R^2$ is open in the Euclidean metric if and only if it is open in the square metric.

    \item A subset $A \subset \R^2$ is open in the taxicab metric if and only if it is open in the square metric.
  \end{enumerate}
\end{problem}

\begin{solution}
  We show that a subset $A \subset \R^2$ is open in the Euclidean metric if and only if it is open in the taxicab metric. Notice that we have
  \[%
    \metric_2(x, y) \le \metric_1(x, y) \le \sqrt{2}\, \metric_2(x, y)
  ,\]%
  which holds for all $x, y \in \R^2$. We will use these inequalities to show that openness in one metric implies openness in the other.

  Suppose that $A$ is open in the Euclidean metric. Let $x \in A$. Then, by definition of openness, there exists $r > 0$ such that the Euclidean open ball
  \[%
    B_2(x, r) = \{ y \in \R^2 : \metric_2(x, y) < r \}
  ,\]%
  is contained in $A$. Now consider any point $y \in B_1(x, r)$, where
  \[%
    B_1(x, r) = \{ y \in \R^2 : \metric_1(x, y) < r \}
  .\]%
  Using the inequality $\metric_2(x, y) \le \metric_1(x, y)$, we have $\metric_2(x, y) < r$ whenever $\metric_1(x, y) < r$. Hence $B_1(x, r) \subseteq B_2(x, r) \subseteq A$. This shows that for every $x \in A$, there exists an $r > 0$ such that $B_1(x, r) \subset A$. Therefore, $A$ is open in the taxicab metric.

  Conversely, suppose that $A$ is open in the taxicab metric. Let $x \in A$. Then there exists $r > 0$ such that
  \[%
    B_1(x, r) = \{ y \in \R^2 : \metric_1(x, y) < r \} \subseteq A
  .\]%
  Using the inequality $\metric_1(x, y) \le \sqrt{2}\, \metric_2(x, y)$, we see that if $\metric_2(x, y) < r / \sqrt{2}$, then $\metric_1(x, y) < r$, and hence $y \in B_1(x, r) \subseteq A$. Therefore,
  \[%
    B_2\left(x, \frac{r}{\sqrt{2}}\right) \subseteq B_1(x, r) \subseteq A
  .\]%
  This shows that each point $x \in A$ has a Euclidean neighborhood contained in $A$, so $A$ is open in the Euclidean metric.

  Thus a subset $A \subset \R^2$ is open in the Euclidean metric if and only if it is open in the taxicab metric.
\end{solution}

\begin{problem}[3.2]
  Let $X = \Q$ with the metric induced from the usual one on $\R$: that is, $\metric(x, y) = |x - y|$ for all $x, y \in \Q$, but we're thinking about $\Q$ in itself and forgetting about the rest of $\R$.
  \begin{enumerate}
    \item Prove that the subset
      \[%
        \{x \in \Q \mid x^2 < 1\}
      ,\]%
      is open but not closed.

    \item Prove that the subset
      \[%
        \{x \in \Q \mid x^2 \le 2\}
      ,\]%
      is both open and closed

      (You may use the fact that $\sqrt{2}$ is irrational without proving it.)
  \end{enumerate}
\end{problem}

\begin{solution}[(i)]
  Let $A = \{x \in \Q \mid x^2 < 1\}$. We will show that $A$ is open in $\Q$.

  Clearly, $A = (-1, 1) \cap \Q$. Let $p \in A$. Then, $p \in (-1, 1)$, so there exists an $\epsilon > 0$ such that the open interval $(p - \epsilon, p + \epsilon)$ is contained in $(-1, 1)$. Since $\Q$ is dense in $\R$, we can choose $\epsilon$ small enough so that $(p - \epsilon, p + \epsilon) \cap \Q \subset A$. Therefore, for every $p \in A$, there exists an $\epsilon > 0$ such that the open ball $B_\epsilon(p) \cap \Q \subset A$. This shows that $A$ is open in $\Q$.

  Next, we will show that $A$ is not closed in $\Q$. The closure of $A$ in $\Q$ is given by
  \[%
    \overline{A} = \{x \in \Q \mid x^2 \le 1\}
  .\]%
  This is because the points $-1$ and $1$ are limit points of $A$ in $\R$, but they are not in $\Q$. Since $-1, 1 \notin A$, we have $\overline{A} \neq A$. Therefore, $A$ is not closed in $\Q$.

  Hence, we conclude that $A$ is open but not closed in $\Q$.
\end{solution}

\begin{solution}[(ii)]
  Let $B = \{x \in \Q \mid x^2 \le 2\}$. We will show that $B$ is both open and closed in $\Q$.

  First, we show that $B$ is closed in $\Q$. The closure of $B$ in $\Q$ is given by
  \[%
    \overline{B} = \{x \in \Q \mid x^2 \le 2\}
  .\]%
  This is because the points $-\sqrt{2}$ and $\sqrt{2}$ are limit points of $B$ in $\R$, but they are not in $\Q$. Since $-\sqrt{2}, \sqrt{2} \notin B$, we have $\overline{B} = B$. Therefore, $B$ is closed in $\Q$.

  Next, we show that $B$ is open in $\Q$. Let $p \in B$. Then, $p^2 \le 2$, so there exists an $\epsilon > 0$ such that the open interval $(p - \epsilon, p + \epsilon)$ is contained in the interval $(-\sqrt{2}, \sqrt{2})$. Since $\Q$ is dense in $\R$, we can choose $\epsilon$ small enough so that $(p - \epsilon, p + \epsilon) \cap \Q \subset B$. Therefore, for every $p \in B$, there exists an $\epsilon > 0$ such that the open ball $B_\epsilon(p) \cap \Q \subset B$. This shows that $B$ is open in $\Q$.

  Hence, we conclude that $B$ is both open and closed in $\Q$.
\end{solution}

\begin{problem}[3.3]
  Let $A \subset C^1([0, 1])$ be the set of functions with simple roots as in Example 3.3. Prove that $A$ is open in the $C^1$ metric.

  Hint: For a given $f \in A$, take the ball of radius
  \[%
    r = \inf_{x\in[0,1]} (|f(x)| + |f'(x)|))
  .\]%
\end{problem}

\begin{solution}
  We show that the set $A \subset C^1([0, 1])$ consisting of all functions with simple roots is open in the $C^1$ metric. Recall that for $f, g \in C^1([0, 1])$, the $C^1$ metric is given by
  \[%
    \metric(f, g) = \max_{x \in [0, 1]} |f(x) - g(x)| + \max_{x \in [0, 1]} |f'(x) - g'(x)|
  .\]%
  A function $f \in C^1([0, 1])$ has a \emph{simple root} at $x_0$ if $f(x_0) = 0$ and $f'(x_0) \ne 0$. The set $A$ is defined as
  \[%
    A = \{ f \in C^1([0, 1]) : f(x) = 0 \implies f'(x) \ne 0 \}
  .\]%

  Let $f \in A$. We must show that there exists $r > 0$ such that if $g \in C^1([0, 1])$ satisfies $\metric(f, g) < r$, then $g \in A$. Taking
  \[%
    r = \inf_{x \in [0, 1]} (|f(x)| + |f'(x)|)
  ,\]%
  notice that $r > 0$ must hold. This is because $f$ is continuously differentiable, implying both $|f(x)|$ and $|f'(x)|$ are continuous on $[0, 1]$, meaning their sum is continuous and attains a minimum. If this minimum were zero, then there would exist $x_0 \in [0, 1]$ such that $f(x_0) = 0$ and $f'(x_0) = 0$, contradicting the assumption that all roots of $f$ are simple. Hence $r > 0$.

  Now suppose $g \in C^1([0, 1])$ satisfies $\metric(f, g) < r$. This means that for all $x \in [0, 1]$,
  \[%
    |f(x) - g(x)| < r \quad \text{and} \quad |f'(x) - g'(x)| < r
  .\]%
  We claim that $g$ also has only simple roots. Suppose, for contradiction, that there exists $x_0 \in [0, 1]$ such that $g(x_0) = 0$ and $g'(x_0) = 0$. Then
  \[%
    |f(x_0)| + |f'(x_0)| = |f(x_0) - g(x_0)| + |f'(x_0) - g'(x_0)|
  ,\]%
  Since both $|f(x_0) - g(x_0)|$ and $|f'(x_0) - g'(x_0)|$ are strictly less than $r$, we obtain
  \[%
    |f(x_0)| + |f'(x_0)| < 2r
  .\]%
  However, by the definition of $r$ as the infimum of $|f(x)| + |f'(x)|$, we must have $|f(x_0)| + |f'(x_0)| \ge r$, giving us
  \[%
    r \le |f(x_0)| + |f'(x_0)| < 2r
  ,\]%
  which is not a contradiction in itself. But notice that if $\metric(f, g) < r/2$, then both $|f(x_0) - g(x_0)|$ and $|f'(x_0) - g'(x_0)|$ are less than $r/2$, giving
  \[%
    |f(x_0)| + |f'(x_0)| \le |f(x_0) - g(x_0)| + |f'(x_0) - g'(x_0)| < r
  ,\]%
  contradicting the definition of $r$ as the infimum of $|f(x)| + |f'(x)|$. Therefore, for all $g$ with $\metric(f, g) < r/2$, no such $x_0$ can exist, and $g$ must have only simple roots.

  Hence, for each $f \in A$, the $C^1$ ball
  \[%
    B\left(f, \frac{r}{2}\right) = \{ g \in C^1([0, 1]) : \metric(f, g) < r/2 \}
  ,\]%
  is contained in $A$. Therefore, $A$ is open in the $C^1$ metric.
\end{solution}

\begin{problem}[3.5]
  Without using Proposition 3.9,
  \begin{enumerate}
    \item Prove that if $U, V \subset X$ are open, then the intersection is again open.

    \item Give an example of a metric space $(X, \metric)$ and countably many open sets $U_1, U_2, U_3, \cdots \subset X$ such that their intersection $U_1 \cap U_2 \cap U_3 \cap \cdots$ is not open.

    \item Let $I$ be a set, and suppose that for each $i \in I$, we have an open set $U_i \subset X$. Prove that the union $\bigcup_{i\in I} U_i$ is again open.

      (Don't assume that the index set $I$ is countable!)
  \end{enumerate}
\end{problem}

\begin{solution}[(i)]
  Assume $U, V \subset X$ are open sets. Let $p \in U \cap V$. Since $U$ is open, there exists an $\epsilon_1 > 0$ such that the open ball $B_{\epsilon_1}(p) \subset U$. Similarly, since $V$ is open, there exists an $\epsilon_2 > 0$ such that the open ball $B_{\epsilon_2}(p) \subset V$. Let $\epsilon = \min(\epsilon_1, \epsilon_2)$. Then, the open ball $B_{\epsilon}(p)$ is contained in both $U$ and $V$, i.e., $B_{\epsilon}(p) \subset U \cap V$. Therefore, for every point $p \in U \cap V$, there exists an $\epsilon > 0$ such that the open ball $B_{\epsilon}(p) \subset U \cap V$. This shows that $U \cap V$ is open.
\end{solution}

\begin{solution}[(ii)]
  Consider the metric space $(\R, \metric)$, where $d$ is the standard Euclidean metric. For each $n \in \N$, define
  \[%
    U_n = \left(-\frac{1}{n}, \frac{1}{n}\right)
  .\]%
  Each $U_n$ is open in $\R$ because it is an open interval. Observe that the sequence of sets is nested, i.e.,
  \[%
    U_1 \supset U_2 \supset U_3 \supset \cdots
  .\]%
  Now consider their intersection:
  \[%
    \bigcap_{n = 1}^\infty U_n = \{0\}
  .\]%
  The set $\{0\}$ is not open in $\R$, since for any $\epsilon > 0$, the open ball $B_\epsilon(0) = (-\epsilon, \epsilon)$ contains points other than $0$. Therefore, no open ball centered at $0$ is contained in $\{0\}$.
\end{solution}

\begin{solution}[(iii)]
  Let $I$ be an index set, and for each $i \in I$, let $U_i \subset X$ be an open set. Let $p \in \bigcup_{i \in I} U_i$. Then, there exists some index $j \in I$ such that $p \in U_j$. Since $U_j$ is open, there exists an $\epsilon > 0$ such that the open ball $B_{\epsilon}(p) \subset U_j$. Since $U_j \subset \bigcup_{i \in I} U_i$, it follows that $B_{\epsilon}(p) \subset \bigcup_{i \in I} U_i$. Therefore, for every point $p \in \bigcup_{i \in I} U_i$, there exists an $\epsilon > 0$ such that the open ball $B_{\epsilon}(p) \subset \bigcup_{i \in I} U_i$. This shows that $\bigcup_{i \in I} U_i$ is open.
\end{solution}
