\begin{problem}[2.2]
  Let $(X, \metric)$ be a metric space and $A \subset X$. Use Proposition 2.6 to prove that
  \[%
    \partial A = \partial(X \setminus A) = \overline A \cap \overline{X \setminus A}
  .\]%
\end{problem}

\begin{solution}
  Assume $x \in \partial A$. By definition, that means $x \in \overline A \setminus \sint A$. Then, $x \in \overline A$ and $x \notin \sint A$. By Proposition~2.6, we have $x \notin \sint A$ if and only if $x \in \overline{X \setminus A}$. Therefore, $x \in \overline A$ and $x \in \overline{X \setminus A}$, which implies
  \[%
    x \in \overline A \cap \overline{X \setminus A}
  .\]%
  Hence, $\partial A \subset \overline A \cap \overline{X \setminus A}$.

  Now, for the converse. Assume $x \in \overline A \cap \overline{X \setminus A}$. Then, $x \in \overline A$ and $x \in \overline{X \setminus A}$. By Proposition~2.6 again, $x \in \overline{X \setminus A}$ if and only if $x \notin \sint A$. Therefore, $x \in \overline A$ and $x \notin \sint A$, which means $x \in \partial A$. Hence,
  \[%
    \overline A \cap \overline{X \setminus A} \subset \partial A
  .\]%

  Combining both inclusions, we conclude that
  \[%
    \partial A = \overline A \cap \overline{X \setminus A}
  .\]%
  Finally, since the expression is symmetric in $A$ and $X \setminus A$, it follows that
  \[%
    \partial A = \partial(X \setminus A)
  .\qedhere\]%
\end{solution}

\begin{problem}[2.3]
  Prove the analogue of Proposition 2.7 for closures, without appealing to Proposition 2.6.
  \begin{enumerate}
    \item If $A \subset B$, then $\bar{A} \subset \bar{B}$.

    \item $\bar{A} \cup \bar{B} = \overline{A \cup B}$.

    \item $\overline{A \cap B} \subset \bar{A} \cap \bar{B}$

      Give an example to show that the inclusion can be strict.

    \item $\bar{\bar{A}} = \bar{A}$.
  \end{enumerate}
\end{problem}

\begin{solution}[(i)]
  Assume $A \subset B$. Let $p \in \bar{A}$. By definition of closure, for every $r > 0$, the open ball $B_r(p)$ intersects $A$. Since $A \subset B$, it follows that $B_r(p)$ also intersects $B$. Therefore, for every $r > 0$, the open ball $B_r(p)$ intersects $B$, which means $p \in \bar{B}$. Hence, $\bar{A} \subset \bar{B}$.
\end{solution}

\begin{solution}[(ii)]
  We will prove both inclusions to show that $\bar{A} \cup \bar{B} = \overline{A \cup B}$.

  First, let $p \in \bar{A} \cup \bar{B}$. Without loss of generality, assume $p \in \bar{A}$. By definition of closure, for every $r > 0$, the open ball $B_r(p)$ intersects $A$. Since $A \subset A \cup B$, it follows that $B_r(p)$ also intersects $A \cup B$. Therefore, for every $r > 0$, the open ball $B_r(p)$ intersects $A \cup B$, which means $p \in \overline{A \cup B}$. Hence, $\bar{A} \cup \bar{B} \subset \overline{A \cup B}$.

  Now, let $p \in \overline{A \cup B}$. By definition of closure, for every $r > 0$, the open ball $B_r(p)$ intersects $A \cup B$. This means that for each $r > 0$, there exists a point in either $A$ or $B$ that lies within the ball. If there are infinitely many such points in $A$, then $p$ is a limit point of $A$ and thus belongs to $\bar{A}$. Similarly, if there are infinitely many such points in $B$, then $p$ is a limit point of $B$ and thus belongs to $\bar{B}$. In either case, we have $p \in \bar{A} \cup \bar{B}$. Hence, $\overline{A \cup B} \subset \bar{A} \cup \bar{B}$.

  Combining both inclusions, we conclude that $\bar{A} \cup \bar{B} = \overline{A \cup B}$.
\end{solution}

\begin{solution}[(iii)]
  We will prove the inclusion $\overline{A \cap B} \subset \bar{A} \cap \bar{B}$.

  Let $p \in \overline{A \cap B}$. By definition of closure, for every $r > 0$, the open ball $B_r(p)$ intersects $A \cap B$. This means that for each $r > 0$, there exists a point in both $A$ and $B$ that lies within the ball. Therefore, for every $r > 0$, the open ball $B_r(p)$ intersects $A$ and also intersects $B$. This implies that $p \in \bar{A}$ and $p \in \bar{B}$. Hence, $p \in \bar{A} \cap \bar{B}$, and we conclude that
  \[%
    \overline{A \cap B} \subset \bar{A} \cap \bar{B}
  .\]%

  To show that the inclusion can be strict, consider the metric space $(\R, \metric)$ with the usual metric. Let
  \[%
    A = (0, 1) \quad\text{and}\quad B = (1, 2)
  .\]%
  Then,
  \[%
    A \cap B = (0, 1) \cap (1, 2) = \emptyset
  ,\]%
  so
  \[%
    \overline{A \cap B} = \overline{\emptyset} = \emptyset
  .\]%
  However,
  \[%
    \bar{A} = [0, 1] \quad\text{and}\quad \bar{B} = [1, 2]
  ,\]%
  so
  \[%
    \bar{A} \cap \bar{B} = [0, 1] \cap [1, 2] = \{1\}
  .\]%
  Thus, $\overline{A \cap B} = \emptyset \subsetneq \{1\} = \bar{A} \cap \bar{B}$.
\end{solution}

\begin{solution}[(iv)]
  We will prove that $\bar{\bar{A}} = \bar{A}$.

  First, let $p \in \bar{\bar{A}}$. By definition of closure, for every $r > 0$, the open ball $B_r(p)$ intersects $\bar{A}$. This means that for each $r > 0$, there exists a point in $\bar{A}$ that lies within the ball. Since $\bar{A}$ is the closure of $A$, it follows that for every $r > 0$, the open ball $B_r(p)$ also intersects $A$. Therefore, $p \in \bar{A}$. Hence, $\bar{\bar{A}} \subset \bar{A}$.

  Now, let $p \in \bar{A}$. By definition of closure, for every $r > 0$, the open ball $B_r(p)$ intersects $A$. Since $A \subset \bar{A}$, it follows that for every $r > 0$, the open ball $B_r(p)$ also intersects $\bar{A}$. Therefore, $p \in \bar{\bar{A}}$. Hence, $\bar{A} \subset \bar{\bar{A}}$.

  Combining both inclusions, we conclude that $\bar{\bar{A}} = \bar{A}$.
\end{solution}

\begin{problem}[2.4]
  Define the closed ball
  \[%
    \bar{B}_r(p) = \{q \in X \mid \metric(p, q) \le r\}
  .\]%
  \begin{enumerate}
    \item Prove that $\bar{B}_r(p)$ equals its own closure.

    \item Prove that $\overline{B_r(p)} \subset \bar{B}_r(p)$: that is, the closure of the open ball is contained in the closed ball. But give an example to show that the inclusion can be strict.

      Hint: For the proof, you may quote Exercise 2.3(a). For the example, you might take $X = \Z$ with the usual metric inherited from $\R$, or any set with a discrete metric (Exercise 1.4).
  \end{enumerate}
\end{problem}

\begin{solution}[(i)]
  We will prove that $\bar{B}_r(p)$ equals its own closure, i.e., $\overline{\bar{B}_r(p)} = \bar{B}_r(p)$.

  First, let $q \in \overline{\bar{B}_r(p)}$. By definition of closure, for every $s > 0$, the open ball $B_s(q)$ intersects $\bar{B}_r(p)$. This means that for each $s > 0$, there exists a point $x \in \bar{B}_r(p)$ such that $\metric(q, x) < s$. Since $x \in \bar{B}_r(p)$, we have $\metric(p, x) \le r$. Now, consider the triangle inequality:
  \[%
    \metric(p, q) \le \metric(p, x) + \metric(x, q) < r + s
  .\]%
  Since this holds for every $s > 0$, we can let $s$ approach $0$, which gives us $\metric(p, q) \le r$. Therefore, $q \in \bar{B}_r(p)$. Hence, we conclude that $\overline{\bar{B}_r(p)} \subset \bar{B}_r(p)$.

  Next, let $q \in \bar{B}_r(p)$. By definition of closed ball, we have $\metric(p, q) \le r$. For every $s > 0$, the open ball $B_s(q)$ contains points that are arbitrarily close to $q$. Since $\metric(p, q) \le r$, it follows that for every $s > 0$, there exists a point in $B_s(q)$ that lies within $\bar{B}_r(p)$. Therefore, for every $s > 0$, the open ball $B_s(q)$ intersects $\bar{B}_r(p)$, which means $q \in \overline{\bar{B}_r(p)}$. Hence, we conclude that $\bar{B}_r(p) \subset \overline{\bar{B}_r(p)}$.

  Combining both inclusions, we conclude that $\overline{\bar{B}_r(p)} = \bar{B}_r(p)$.
\end{solution}

\begin{solution}[(ii)]
  Let $p \in \overline{B_r(p)}$. By definition of closure, for every $s > 0$, the open ball $B_s(p)$ intersects $B_r(p)$. This means that for each $s > 0$, there exists a point $q \in B_r(p)$ such that $\metric(p, q) < s$. Since $q \in B_r(p)$, we have $\metric(p, q) < r$. 

  Now, consider the limit as $s$ approaches $0$. As $s$ gets smaller and smaller, the points $q$ that lie within the open ball $B_s(p)$ get closer and closer to $p$. Since $\metric(p, q) < r$ for all such points, it follows that $\metric(p, p) = 0 \le r$. Therefore, $p \in \bar{B}_r(p)$.

  Hence, we conclude that $\overline{B_r(p)} \subset \bar{B}_r(p)$.

  Example: TODO!!!!!!!!!!!!!!!!!!!!!!!!!!!!!!!!!
\end{solution}

\begin{problem}[3.1]
  In Example 1.4 we saw three different metrics on $\R^2$. Prove one of the following:
  \begin{enumerate}
    \item A subset $A \subset \R^2$ is open in the Euclidean metric if and only if it is open in the taxicab metric.

    \item A subset $A \subset \R^2$ is open in the Euclidean metric if and only if it is open in the square metric.

    \item A subset $A \subset \R^2$ is open in the taxicab metric if and only if it is open in the square metric.
  \end{enumerate}
\end{problem}

\begin{solution}[(i)]
\end{solution}

\begin{solution}[(ii)]
\end{solution}

\begin{solution}[(iii)]
\end{solution}

\begin{problem}[3.2]
  Let $X = \Q$ with the metric induced from the usual one on $\R$: that is, $\metric(x, y) = |x - y|$ for all $x, y \in \Q$, but we're thinking about $\Q$ in itself and forgetting about the rest of $\R$.
  \begin{enumerate}
    \item Prove that the subset
      \[%
        \{x \in \Q \mid x^2 < 1\}
      ,\]%
      is open but not closed.

    \item Prove that the subset
      \[%
        \{x \in \Q \mid x^2 \le 2\}
      ,\]%
      is both open and closed

      (You may use the fact that $\sqrt{2}$ is irrational without proving it.)
  \end{enumerate}
\end{problem}

\begin{solution}[(i)]
  Let $A = \{x \in \Q \mid x^2 < 1\}$. We will show that $A$ is open in $\Q$.

  Clearly, $A = (-1, 1) \cap \Q$. Let $p \in A$. Then, $p \in (-1, 1)$, so there exists an $\epsilon > 0$ such that the open interval $(p - \epsilon, p + \epsilon)$ is contained in $(-1, 1)$. Since $\Q$ is dense in $\R$, we can choose $\epsilon$ small enough so that $(p - \epsilon, p + \epsilon) \cap \Q \subset A$. Therefore, for every $p \in A$, there exists an $\epsilon > 0$ such that the open ball $B_\epsilon(p) \cap \Q \subset A$. This shows that $A$ is open in $\Q$.

  Next, we will show that $A$ is not closed in $\Q$. The closure of $A$ in $\Q$ is given by
  \[%
    \overline{A} = \{x \in \Q \mid x^2 \le 1\}
  .\]%
  This is because the points $-1$ and $1$ are limit points of $A$ in $\R$, but they are not in $\Q$. Since $-1, 1 \notin A$, we have $\overline{A} \neq A$. Therefore, $A$ is not closed in $\Q$.

  Hence, we conclude that $A$ is open but not closed in $\Q$.
\end{solution}

\begin{solution}[(ii)]
  Let $B = \{x \in \Q \mid x^2 \le 2\}$. We will show that $B$ is both open and closed in $\Q$.

  First, we show that $B$ is closed in $\Q$. The closure of $B$ in $\Q$ is given by
  \[%
    \overline{B} = \{x \in \Q \mid x^2 \le 2\}
  .\]%
  This is because the points $-\sqrt{2}$ and $\sqrt{2}$ are limit points of $B$ in $\R$, but they are not in $\Q$. Since $-\sqrt{2}, \sqrt{2} \notin B$, we have $\overline{B} = B$. Therefore, $B$ is closed in $\Q$.

  Next, we show that $B$ is open in $\Q$. Let $p \in B$. Then, $p^2 \le 2$, so there exists an $\epsilon > 0$ such that the open interval $(p - \epsilon, p + \epsilon)$ is contained in the interval $(-\sqrt{2}, \sqrt{2})$. Since $\Q$ is dense in $\R$, we can choose $\epsilon$ small enough so that $(p - \epsilon, p + \epsilon) \cap \Q \subset B$. Therefore, for every $p \in B$, there exists an $\epsilon > 0$ such that the open ball $B_\epsilon(p) \cap \Q \subset B$. This shows that $B$ is open in $\Q$.

  Hence, we conclude that $B$ is both open and closed in $\Q$.
\end{solution}

\begin{problem}[3.3]
  Let $A \subset C^1([0, 1])$ be the set of functions with simple roots as in Example 3.3. Prove that $A$ is open in the $C^1$ metric.

  Hint: For a given $f \in A$, take the ball of radius
  \[%
    r = \inf_{x\in[0,1]} (|f(x)| + |f'(x)|))
  .\]%
\end{problem}

\begin{solution}
\end{solution}

\begin{problem}[3.5]
  Without using Proposition 3.9,
  \begin{enumerate}
    \item Prove that if $U, V \subset X$ are open, then the intersection is again open.

    \item Give an example of a metric space $(X, \metric)$ and countably many open sets $U_1, U_2, U_3, \cdots \subset X$ such that their intersection $U_1 \cap U_2 \cap U_3 \cap \cdots$ is not open.

    \item Let $I$ be a set, and suppose that for each $i \in I$, we have an open set $U_i \subset X$. Prove that the union $\bigcup_{i\in I} U_i$ is again open.

      (Don't assume that the index set $I$ is countable!)
  \end{enumerate}
\end{problem}

\begin{solution}[(i)]
  Assume $U, V \subset X$ are open sets. We will show that their intersection $U \cap V$ is also open.

  Let $p \in U \cap V$. Since $U$ is open, there exists an $\epsilon_1 > 0$ such that the open ball $B_{\epsilon_1}(p) \subset U$. Similarly, since $V$ is open, there exists an $\epsilon_2 > 0$ such that the open ball $B_{\epsilon_2}(p) \subset V$. Let $\epsilon = \min(\epsilon_1, \epsilon_2)$. Then, the open ball $B_{\epsilon}(p)$ is contained in both $U$ and $V$, i.e., $B_{\epsilon}(p) \subset U \cap V$. Therefore, for every point $p \in U \cap V$, there exists an $\epsilon > 0$ such that the open ball $B_{\epsilon}(p) \subset U \cap V$. This shows that $U \cap V$ is open.
\end{solution}

\begin{solution}[(ii)]
\end{solution}

\begin{solution}[(iii)]
  Let $I$ be an index set, and for each $i \in I$, let $U_i \subset X$ be an open set. We will show that the union $\bigcup_{i \in I} U_i$ is also open.

  Let $p \in \bigcup_{i \in I} U_i$. Then, there exists some index $j \in I$ such that $p \in U_j$. Since $U_j$ is open, there exists an $\epsilon > 0$ such that the open ball $B_{\epsilon}(p) \subset U_j$. Since $U_j \subset \bigcup_{i \in I} U_i$, it follows that $B_{\epsilon}(p) \subset \bigcup_{i \in I} U_i$. Therefore, for every point $p \in \bigcup_{i \in I} U_i$, there exists an $\epsilon > 0$ such that the open ball $B_{\epsilon}(p) \subset \bigcup_{i \in I} U_i$. This shows that $\bigcup_{i \in I} U_i$ is open.
\end{solution}
