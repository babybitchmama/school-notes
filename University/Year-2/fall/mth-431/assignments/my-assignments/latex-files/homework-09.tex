\begin{problem}[11.2]
  Prove that the subset $[0, 1] \subset \R$ is not compact in the lower limit topology from Example 7.3(d).
\end{problem}

\begin{solution}
  Let $U = [0, 1]$. Take the open cover of $U$ given by $V_x = [x, x + \epsilon)$, for some $\epsilon > 0$. Clearly, each $V_x$ is open in the lower limit topology, and
  \[%
    U \subset \bigcup_{x \in [0, 1]} V_x
  .\]%
  Take any finite subcover $\{V_{x_1}, \cdots, V_{x_n}\}$. Pick any two points, $x_i$ and $x_j$ such that $x_i < x_j$. Make $\epsilon > 0$ small enough such that $x_i + \epsilon < x_j$. Then, the there exists a point $p \in (x_i + \epsilon, x_j)$ that is not covered by any of the $V_{x_k}$'s. Thus, no finite subcover exists, and $[0, 1]$ is not compact in the lower limit topology.
\end{solution}

\begin{problem}[11.4]
  Let $X$ be a topological space, and let $\Phi$ be a set of continuous functions $X \to [0, \infty)$ such that for every $x \in X$ there is some $f \in \Phi$ with $f(x) > 0$.
  \begin{enumerate}
    \item Prove that if $X$ is compact then there are $f_1, \cdots, f_n \in \Phi$ such that $f_1(x) + \cdots + f_n(x) > 0$ for all $x \in X$.

    \item Give a counterexample when $X$ is not compact.

      (If you know something about rings and ideals, you can consider the ring of continuous functions on a topological space $X$; then for any $p \in X$, the functions that vanish at $p$ turn out to form a maximal ideal. This problem can be used to prove that if $X$ is compact, then every maximal ideal comes from some point $p$ in this way. But if $X$ is not compact then there are more maximal ideals.)
  \end{enumerate}
\end{problem}

\begin{solution}[(i)]
  Assume $X$ is compact. For each $x \in X$, there exists some $f_x \in \Phi$ such that $f_x(x) > 0$. Since $f_x$ is continuous, there exists an open neighborhood $U_x$ of $x$ such that for all $y \in U_x$, $f_x(y) > 0$. The collection $\{U_x \mid x \in X\}$ forms an open cover of $X$. By compactness, there exists a finite subcover $\{U_{x_1}, U_{x_2}, \ldots, U_{x_n}\}$. Correspondingly, we have functions $f_{x_1}, f_{x_2}, \ldots, f_{x_n} \in \Phi$. Now, for any $y \in X$, there exists some $U_{x_i}$ in the finite subcover such that $y \in U_{x_i}$. Therefore, $f_{x_i}(y) > 0$. Then, we have
  \[%
    f_{x_1}(y) + f_{x_2}(y) + \cdots + f_{x_n}(y) \ge f_{x_i}(y) > 0
  .\qedhere\]%
\end{solution}

\begin{solution}[(ii)]
\end{solution}

\begin{problem}[11.6]
  \begin{enumerate}
    \item Let $X$ be a Hausdorff space. By definition, distinct points of $X$ have disjoint neighborhoods. Proposition 11.6 proved that a compact subset $K \subset X$ and a point $p \not\in K$ have disjoint neighborhoods. Now prove that two disjoint compact sets $C$, $K \subset X$ have disjoint neighborhoods: that is, if $C \cap K = \emptyset$ then there are open sets $U, V \subset X$ with $C \subset X$, $K \subset V$, and $U \cap V = \emptyset$.

    \item Give an example of a non-Hausdorff space $X$, a compact subset $K \subset X$, and a point $p \in X \setminus K$ such that every neighborhood of $p$ meets every neighborhood of $K$.
  \end{enumerate}
\end{problem}

\begin{solution}[(i)]
  Let $C$ and $K$ be two disjoint compact subsets of a Hausdorff space $X$. For each point $c \in C$, since $X$ is Hausdorff, for each point $k \in K$, there exist disjoint open neighborhoods $U_{c,k}$ of $c$ and $V_{c,k}$ of $k$. The collection $\{V_{c,k} \mid k \in K\}$ forms an open cover of $K$. By compactness of $K$, there exists a finite subcover $\{V_{c,k_1}, V_{c,k_2}, \ldots, V_{c,k_m}\}$. Correspondingly, we have open neighborhoods $U_{c,k_1}, U_{c,k_2}, \ldots, U_{c,k_m}$ of $c$. Define
  \[%
    U_c = \bigcap_{i=1}^m U_{c,k_i} \aand V_c = \bigcup_{i=1}^m V_{c,k_i}
  .\]%
  Then, $U_c$ is an open neighborhood of $c$ and $V_c$ is an open neighborhood of $K$, with $U_c \cap V_c = \emptyset$. The collection $\{U_c \mid c \in C\}$ forms an open cover of $C$. By compactness of $C$, there exists a finite subcover $\{U_{c_1}, U_{c_2}, \ldots, U_{c_n}\}$. Correspondingly, we have open neighborhoods $V_{c_1}, V_{c_2}, \ldots, V_{c_n}$ of $K$. Define
  \[%
    U = \bigcup_{j=1}^n U_{c_j} \aand V = \bigcap_{j=1}^n V_{c_j}
  .\]%
  Then, $U$ is an open neighborhood of $C$ and $V$ is an open neighborhood of $K$, with $U \cap V = \emptyset$.
\end{solution}

\begin{solution}[(ii)]
\end{solution}

\begin{problem}[11.7]
  \begin{enumerate}
    \item Prove that a non-empty subset $A \subset \R$ is compact in the lower semi-continuous topology from Example 7.3(c) if and only if $A$ is bounded below and contains its infimum.

    \item Let X be a compact space, and let $f : X \to \R$ be lower semicontinuous, that is, continuous with respect to the lower semicontinuous topology on $\R$. Prove that $f$ attains its minimum: there is a point $p \in X$ such that $f(p) \le f(x)$ for all $x \in X$.
  \end{enumerate}
\end{problem}

\begin{solution}[(i)]
  Assume $A$ is compact in the lower semi-continuous topology. Since $A$ is non-empty, let $m = \inf A$. Assume for contradiction that $m \notin A$. Consider the open cover of $A$ given by
  \[%
    U_a = \left(\frac{m + a}{2}, \infty\right)
  .\]%
  Clearly, the collection $\{U_a \mid a \in A\}$ is an open cover for $A$. Now, since $A$ is compact, there exists a finite subcover $\{U_{a_1}, U_{a_2}, \cdots, U_{a_n}\}$ that covers $A$. Let
  \[%
    b = \min_{i\in[1,n]} \frac{m + a_i}{2}
  .\]%
  Clearly, $m < b$. However, pick any point $c \in (m, b)$. Then, $c \notin U_{a_i}$ for all $i \in [1, n]$, which contradicts the fact that $\{U_{a_1}, U_{a_2}, \cdots, U_{a_n}\}$ is a cover for $A$. Thus, $m \in A$. Also, since $A$ is non-empty and compact, it must be bounded below.

  Now, conversely, assume $A$ is bounded below and contains its infimum $m$. Since $m \in A$, it's contained in any open cover, $U_m$, of $A$, which has the form $(a, \infty)$, where $a < m$. But then, $A \subset (a, \infty) = U_m$, since every $x \in A$ satisfies $x \ge m > a$. Thus, any open cover of $A$ has a finite subcover (in fact, just one set suffices), and $A$ is compact.
\end{solution}

\begin{solution}[(ii)]
\end{solution}

\begin{problem}[11.8]
  Let $(X, \metric)$ be a metric space. A map $f : X \to X$ is called a \emph{contraction mapping} if there is an $r \in [0, 1)$ such that
  \[%
    \metric(f(p), f(q)) \le r \cdot \metric(p, q)
  ,\]%
  for all $p, q \in X$, while it is called a \emph{weak contraction mapping} if we just have
  \[%
    \metric(f(p), f(q)) < \metric(p, q)
  ,\]%
  whenever $p \ne q$. If $X$ is complete then the Banach fixed-point theorem (Theorem 4.13) stated that a contraction mapping has a fixed point, while Exercise 4.5(a) asked you to show that a weak contraction mapping need not have a fixed point.
  \begin{enumerate}
    \item Prove that if $X$ is compact and $f : X \to X$ is a weak contraction mapping, then $f$ has fixed point.

      Hint: Prove that the function $g : X \to \R$ given by $g(p) = \metric(f(p), p)$ is continuous; thus it achieves its minimum by the extreme value theorem, and if this minimum is not zero then you get a contradiction.

    \item Let $X = [0, 1/2]$ with the usual metric, which is compact. Prove that the map $f : X \to X$ given by $f(x) = x^2$ is a weak contraction mapping, but not a contraction mapping.

    \item Give an example of a compact metric space $(X, \metric)$ and a map $f : X \to X$ that satisfies $\metric(f(p), f(q)) \le \metric(p, q)$ but has no fixed point.
  \end{enumerate}
\end{problem}

\begin{solution}[(i)]
\end{solution}

\begin{solution}[(ii)]
\end{solution}

\begin{solution}[(iii)]
\end{solution}

\begin{problem}[11.10]
  A continuous map $f : X \to Y$ is called proper if the preimage of any compact set $K \subset Y$ is compact.
  \begin{enumerate}
    \item Prove that the map $f : \R^2 \to \R$ given by $f(x, y) = x^2 + y^2$ is proper.
    \item Prove that the map $f : \R^2 \to \R$ given by $f(x, y) = x^2 - y^2$ is not proper.
    \item Prove that if $f$ is proper then the preimage of every point is compact.
    \item Give an example of a continuous map $f: X \to Y$ for which the preimage of every point is compact, but nonetheless $f$ is not proper.
    \item Prove that if $X$ is compact and $Y$ is Hausdorff then any continuous map $f : X \to Y$ is proper.
    \item Let $X$ and $Y$ be topological spaces. Prove that the projection $p : X \times Y \to X$ is proper if and only if $Y$ is compact.
  \end{enumerate}
\end{problem}

\begin{solution}[(i)]
\end{solution}

\begin{solution}[(ii)]
\end{solution}

\begin{solution}[(iii)]
\end{solution}

\begin{solution}[(iv)]
\end{solution}

\begin{solution}[(v)]
\end{solution}

\begin{solution}[(vi)]
\end{solution}
