\begin{problem}[11.2]
  Prove that the subset $[0, 1] \subset \R$ is not compact in the lower limit topology from Example 7.3(d).
\end{problem}

\begin{solution}
  Let $U = [0, 1]$. Take the open cover of $U$ given by $V_x = [x, x + \epsilon)$, for some $\epsilon > 0$. Clearly, each $V_x$ is open in the lower limit topology, and
  \[%
    U \subset \bigcup_{x \in [0, 1]} V_x
  .\]%
  Take any finite subcover $\{V_{x_1}, \cdots, V_{x_n}\}$. Pick any two points, $x_i$ and $x_j$ such that $x_i < x_j$. Make $\epsilon > 0$ small enough such that $x_i + \epsilon < x_j$. Then, the there exists a point $p \in (x_i + \epsilon, x_j)$ that is not covered by any of the $V_{x_k}$'s. Thus, no finite subcover exists, and $[0, 1]$ is not compact in the lower limit topology.
\end{solution}

\begin{problem}[11.4]
  Let $X$ be a topological space, and let $\Phi$ be a set of continuous functions $X \to [0, \infty)$ such that for every $x \in X$ there is some $f \in \Phi$ with $f(x) > 0$.
  \begin{enumerate}
    \item Prove that if $X$ is compact then there are $f_1, \cdots, f_n \in \Phi$ such that $f_1(x) + \cdots + f_n(x) > 0$ for all $x \in X$.

    \item Give a counterexample when $X$ is not compact.

      (If you know something about rings and ideals, you can consider the ring of continuous functions on a topological space $X$; then for any $p \in X$, the functions that vanish at $p$ turn out to form a maximal ideal. This problem can be used to prove that if $X$ is compact, then every maximal ideal comes from some point $p$ in this way. But if $X$ is not compact then there are more maximal ideals.)
  \end{enumerate}
\end{problem}

\begin{solution}[(i)]
  Assume $X$ is compact. For each $x \in X$, there exists some $f_x \in \Phi$ such that $f_x(x) > 0$. Since $f_x$ is continuous, there exists an open neighborhood $U_x$ of $x$ such that for all $y \in U_x$, $f_x(y) > 0$. The collection $\{U_x \mid x \in X\}$ forms an open cover of $X$. By compactness, there exists a finite subcover $\{U_{x_1}, U_{x_2}, \ldots, U_{x_n}\}$. Correspondingly, we have functions $f_{x_1}, f_{x_2}, \ldots, f_{x_n} \in \Phi$. Now, for any $y \in X$, there exists some $U_{x_i}$ in the finite subcover such that $y \in U_{x_i}$. Therefore, $f_{x_i}(y) > 0$. Then, we have
  \[%
    f_{x_1}(y) + f_{x_2}(y) + \cdots + f_{x_n}(y) \ge f_{x_i}(y) > 0
  .\qedhere\]%
\end{solution}

\begin{solution}[(ii)]
\end{solution}

\begin{problem}[11.6]
  \begin{enumerate}
    \item Let $X$ be a Hausdorff space. By definition, distinct points of $X$ have disjoint neighborhoods. Proposition 11.6 proved that a compact subset $K \subset X$ and a point $p \not\in K$ have disjoint neighborhoods. Now prove that two disjoint compact sets $C$, $K \subset X$ have disjoint neighborhoods: that is, if $C \cap K = \emptyset$ then there are open sets $U, V \subset X$ with $C \subset X$, $K \subset V$, and $U \cap V = \emptyset$.

    \item Give an example of a non-Hausdorff space $X$, a compact subset $K \subset X$, and a point $p \in X \setminus K$ such that every neighborhood of $p$ meets every neighborhood of $K$.
  \end{enumerate}
\end{problem}

\begin{solution}[(i)]
  Let $C$ and $K$ be two disjoint compact subsets of a Hausdorff space $X$. For each point $c \in C$, since $X$ is Hausdorff, for each point $k \in K$, there exist disjoint open neighborhoods $U_{c,k}$ of $c$ and $V_{c,k}$ of $k$. The collection $\{V_{c,k} \mid k \in K\}$ forms an open cover of $K$. By compactness of $K$, there exists a finite subcover $\{V_{c,k_1}, V_{c,k_2}, \ldots, V_{c,k_m}\}$. Correspondingly, we have open neighborhoods $U_{c,k_1}, U_{c,k_2}, \ldots, U_{c,k_m}$ of $c$. Define
  \[%
    U_c = \bigcap_{i=1}^m U_{c,k_i} \aand V_c = \bigcup_{i=1}^m V_{c,k_i}
  .\]%
  Then, $U_c$ is an open neighborhood of $c$ and $V_c$ is an open neighborhood of $K$, with $U_c \cap V_c = \emptyset$. The collection $\{U_c \mid c \in C\}$ forms an open cover of $C$. By compactness of $C$, there exists a finite subcover $\{U_{c_1}, U_{c_2}, \ldots, U_{c_n}\}$. Correspondingly, we have open neighborhoods $V_{c_1}, V_{c_2}, \ldots, V_{c_n}$ of $K$. Define
  \[%
    U = \bigcup_{j=1}^n U_{c_j} \aand V = \bigcap_{j=1}^n V_{c_j}
  .\]%
  Then, $U$ is an open neighborhood of $C$ and $V$ is an open neighborhood of $K$, with $U \cap V = \emptyset$.
\end{solution}

\begin{solution}[(ii)]
\end{solution}

\begin{problem}[11.7]
  \begin{enumerate}
    \item Prove that a non-empty subset $A \subset \R$ is compact in the lower semi-continuous topology from Example 7.3(c) if and only if $A$ is bounded below and contains its infimum.

    \item Let X be a compact space, and let $f : X \to \R$ be lower semicontinuous, that is, continuous with respect to the lower semicontinuous topology on $\R$. Prove that $f$ attains its minimum: there is a point $p \in X$ such that $f(p) \le f(x)$ for all $x \in X$.
  \end{enumerate}
\end{problem}

\begin{solution}[(i)]
  Assume $A$ is compact in the lower semi-continuous topology. Since $A$ is non-empty, let $m = \inf A$. Assume for contradiction that $m \notin A$. Consider the open cover of $A$ given by
  \[%
    U_a = \left(\frac{m + a}{2}, \infty\right)
  .\]%
  Clearly, the collection $\{U_a \mid a \in A\}$ is an open cover for $A$. Now, since $A$ is compact, there exists a finite subcover $\{U_{a_1}, U_{a_2}, \cdots, U_{a_n}\}$ that covers $A$. Let
  \[%
    b = \min_{i\in[1,n]} \frac{m + a_i}{2}
  .\]%
  Clearly, $m < b$. However, pick any point $c \in (m, b)$. Then, $c \notin U_{a_i}$ for all $i \in [1, n]$, which contradicts the fact that $\{U_{a_1}, U_{a_2}, \cdots, U_{a_n}\}$ is a cover for $A$. Thus, $m \in A$. Also, since $A$ is non-empty and compact, it must be bounded below.

  Now, conversely, assume $A$ is bounded below and contains its infimum $m$. Since $m \in A$, it's contained in any open cover, $U_m$, of $A$, which has the form $(a, \infty)$, where $a < m$. But then, $A \subset (a, \infty) = U_m$, since every $x \in A$ satisfies $x \ge m > a$. Thus, any open cover of $A$ has a finite subcover (in fact, just one set suffices), and $A$ is compact.
\end{solution}

\begin{solution}[(ii)]
  Assume $X$ is compact and $f : X \to \R$ is lower semicontinuous. Let $\displaystyle m = \inf_{x \in X} f(x)$. Consider the open cover of $X$ given by
  \[%
    U_x = \left(f(x), \infty\right)
  .\]%
  Clearly, the collection $\{U_x \mid x \in X\}$ is an open cover for $X$. Since $X$ is compact, there exists a finite subcover $\{U_{x_1}, U_{x_2}, \cdots, U_{x_n}\}$ that covers $X$. Let
  \[%
    b = \min_{i\in[1,n]} f(x_i)
  .\]%
  Clearly, $b \ge m$. Now, for any $x \in X$, there exists some $U_{x_i}$ in the finite subcover such that $x \in U_{x_i}$. Therefore, $f(x) > f(x_i) \ge b$. Thus, for all $x \in X$, $f(x) \ge b$. Since $b$ is a lower bound for $f(X)$ and $m$ is the greatest lower bound, we have $m \ge b$. Combining this with the earlier inequality, we get $m \le b \le m$, which implies $b = m$. Therefore, there exists some $x_j$ such that $f(x_j) = m$, or equivalently, $f(x_j) = m$. Thus, $f$ attains its minimum at the point $x_j \in X$.
\end{solution}

\begin{problem}[11.8]
  Let $(X, \metric)$ be a metric space. A map $f : X \to X$ is called a \emph{contraction mapping} if there is an $r \in [0, 1)$ such that
  \[%
    \metric(f(p), f(q)) \le r \cdot \metric(p, q)
  ,\]%
  for all $p, q \in X$, while it is called a \emph{weak contraction mapping} if we just have
  \[%
    \metric(f(p), f(q)) < \metric(p, q)
  ,\]%
  whenever $p \ne q$. If $X$ is complete then the Banach fixed-point theorem (Theorem 4.13) stated that a contraction mapping has a fixed point, while Exercise 4.5(a) asked you to show that a weak contraction mapping need not have a fixed point.
  \begin{enumerate}
    \item Prove that if $X$ is compact and $f : X \to X$ is a weak contraction mapping, then $f$ has fixed point.

      Hint: Prove that the function $g : X \to \R$ given by $g(p) = \metric(f(p), p)$ is continuous; thus it achieves its minimum by the extreme value theorem, and if this minimum is not zero then you get a contradiction.

    \item Let $X = [0, 1/2]$ with the usual metric, which is compact. Prove that the map $f : X \to X$ given by $f(x) = x^2$ is a weak contraction mapping, but not a contraction mapping.

    \item Give an example of a compact metric space $(X, \metric)$ and a map $f : X \to X$ that satisfies $\metric(f(p), f(q)) \le \metric(p, q)$ but has no fixed point.
  \end{enumerate}
\end{problem}

\begin{solution}[(i)]
  Assume $X$ is compact and $f : X \to X$ is a weak contraction mapping. Define the mapping $g : X \to \R$ by $g(p) = \metric(f(p), p)$. Pick two points $p, q \in X$. By the triangle inequality, we have
  \[%
    \metric(f(p), p) \le \metric(f(p), f(q)) + \metric(f(q), q) + \metric(q, p)
  .\]%
  But since $f$ is a weak contraction mapping, we have
  \[%
    \metric(f(p), p) < \metric(p, q) + \metric(f(q), q) + \metric(q, p) = 2\metric(p, q) + \metric(f(q), q)
  .\]%
  Therefore, we have
  \[%
    g(p) - g(q) = \metric(f(p), p) - \metric(f(q), q) < 2\metric(p, q)
  .\]%
  By symmetry, we also have $g(q) - g(p) < 2\metric(p, q)$. Therefore, we have $|g(p) - g(q)| < 2\metric(p, q)$. This shows that $g$ is continuous. Since $X$ is compact, $g$ attains a minimum at some $p_0 \in X$. If $g(p_0) = 0$, then $f(p_0) = p_0$, and we have found a fixed point. Now, assume for contradiction that $g(p_0) > 0$. Then, we have
  \[%
    g(f(p_0)) = \metric(f(f(p_0)), f(p_0)) < \metric(f(p_0), p_0) = g(p_0)
  .\]%
  This contradicts the fact that $g(p_0)$ is the minimum value of $g$. Thus, we must have $g(p_0) = 0$, and $f$ has a fixed point at $p_0$.
\end{solution}

\begin{solution}[(ii)]
  Let $X = [0, 1/2]$ with the usual metric $\metric(x, y) = |x - y|$. Consider the map $f : X \to X$ given by $f(x) = x^2$. For any two points $x, y \in X$, without loss of generality, assume $x < y$. Then, we have
  \[%
    \metric(f(x), f(y)) = |x^2 - y^2| = |x - y||x + y| \le |x - y| \cdot 1 = \metric(x, y)
  .\]%
  Since $x + y \le 1$ for all $x, y \in [0, 1/2]$, we have $\metric(f(x), f(y)) < \metric(x, y)$ whenever $x \ne y$. Thus, $f$ is a weak contraction mapping. However, to show that $f$ is not a contraction mapping, assume for contradiction that there exists some $r \in [0, 1)$ such that
  \[%
    \metric(f(x), f(y)) \le r \cdot \metric(x, y)
  ,\]%
  for all $x, y \in X$. Pick $x = 0$ and $y = 1/2$. Then, we have
  \[%
    \metric(f(0), f(1/2)) = |0 - (1/2)^2| = 1/4 \le r \cdot |0 - 1/2| = r/2
  .\]%
  This implies that $r \ge 1/2$. Now, pick $x = 1/4$ and $y = 1/2$. Then, we have
  \[%
    \metric(f(1/4), f(1/2)) = |(1/4)^2 - (1/2)^2| = 3/16 \le r \cdot |1/4 - 1/2| = r/4
  .\]%
  This implies that $r \ge 3/4$. Continuing this process, we can see that for any $\epsilon > 0$, we can find points $x, y \in X$
\end{solution}

\begin{solution}[(iii)]
\end{solution}

\begin{problem}[11.10]
  A continuous map $f : X \to Y$ is called proper if the preimage of any compact set $K \subset Y$ is compact.
  \begin{enumerate}
    \item Prove that the map $f : \R^2 \to \R$ given by $f(x, y) = x^2 + y^2$ is proper.
    \item Prove that the map $f : \R^2 \to \R$ given by $f(x, y) = x^2 - y^2$ is not proper.
    \item Prove that if $f$ is proper then the preimage of every point is compact.
    \item Give an example of a continuous map $f: X \to Y$ for which the preimage of every point is compact, but nonetheless $f$ is not proper.
    \item Prove that if $X$ is compact and $Y$ is Hausdorff then any continuous map $f : X \to Y$ is proper.
    \item Let $X$ and $Y$ be topological spaces. Prove that the projection $p : X \times Y \to X$ is proper if and only if $Y$ is compact.
  \end{enumerate}
\end{problem}

\begin{solution}[(i)]
  The pre-image of some number $a$ under the map $f$ is given by
  \[%
    f^{-1}(a) = \{(x, y) \in \R^2 \mid x^2 + y^2 = a\}
  .\]%
  If $a < 0$, then $f^{-1}(a) = \emptyset$, which is compact. If $a = 0$, then $f^{-1}(a) = \{(0, 0)\}$, which is also compact. If $a > 0$, then $f^{-1}(a)$ is the circle of radius $\sqrt{a}$ centered at the origin, which is closed and bounded in $\R^2$, and hence compact by the Heine-Borel theorem. Now, consider any compact set $K \subset \R$. Since $K$ is compact, it is closed and bounded, so there exist real numbers $m$ and $M$ such that
  \[%
    K \subset [m, M]
  .\]%
  Therefore, we have
  \[%
    f^{-1}(K) \subset f^{-1}([m, M]) = \bigcup_{a \in [m, M]} f^{-1}(a)
  .\]%
  Since $[m, M]$ is closed and bounded, it is compact. The union of the pre-images of all points in a compact set is also compact, as each pre-image is compact. Thus, $f^{-1}(K)$ is compact, and $f$ is a proper map.
\end{solution}

\begin{solution}[(ii)]
  The equation $x^2 - y^2 = 0$ is true if and only if $x = y$. Thus, the pre-image of $0$ under the map $f$ is given by
  \[%
    f^{-1}(0) = \{(x, y) \in \R^2 \mid x = y\} = \{(t, t) \mid t \in \R\}
  ,\]%
  which is the line $y = x$ in $\R^2$. This set is not bounded, and hence not compact. Therefore, $f$ is not a proper map.
\end{solution}

\begin{solution}[(iii)]
  Let $f : X \to Y$ be a proper map. Consider any point $y \in Y$. The set $\{y\}$ is compact in $Y$ since we can take the finite subcover consisting of just the set $\{y\}$ itself. Since $f$ is proper, the pre-image $f^{-1}(\{y\})$ is compact in $X$. Thus, the pre-image of every point under a proper map is compact.
\end{solution}

\begin{solution}[(iv)]
\end{solution}

\begin{solution}[(v)]
  Assume that $X$ is compact, $Y$ is Hausdorff, and $f : X \to Y$ is continuous. Consider any compact set $K \subset Y$. Since $Y$ is Hausdorff, $K$ is closed in $Y$. Since $f$ is continuous, the pre-image $f^{-1}(K)$ is closed in $X$. Since $X$ is compact, any closed subset of $X$ is also compact. Therefore, $f^{-1}(K)$ is compact in $X$, and $f$ is a proper map.
\end{solution}

\begin{solution}[(vi)]
  Assume that the projection $p : X \times Y \to X$ is proper. Consider any compact set $K \subset X$. Since $p$ is proper, the pre-image
  \[%
    p^{-1}(K) = K \times Y
  .\]%
  is compact in $X \times Y$. By the Heine-Borel theorem, this implies that $Y$ is compact.

  Now, conversely, assume that $Y$ is compact. Consider any compact set $K \subset X$. The pre-image
  \[%
    p^{-1}(K) = K \times Y
  .\]%
  Since both $K$ and $Y$ are compact, their product $K \times Y$ is also compact by the Tychonoff theorem. Thus, $p$ is a proper map.
\end{solution}
