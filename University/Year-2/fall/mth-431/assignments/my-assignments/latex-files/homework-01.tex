\begin{problem}[1.1]\leavevmode
  \begin{enumerate}
    \item For each of the three metrics in Example 1.4, sketch the open ball of some radius $r > 0$ around the origin in $\R^2$:
      \[%
        B_r(0) = \{(x, y) \in \R^2 \mid \metric{((x, y), 0)} < r\}
      .\]%

    \item For one of the three metrics (your choice), prove or give a counterexample to the following statement: a sequence of points $(x_1, y_1), (x_2, y_2), \cdots \in \R^2$ converges to a limit $(x, y)$ if and only if $x_n \to x$ and $y_n \to y$ separately, as sequences in $\R$ with the usual metric.

    \item Why is
      \[%
        \metric(\x, \y) = \min(\{|x_1 - y_1|, |x_2 - y_2|, \cdots, |x_n - y_n|\})
      ,\]%
      not a metric on $\R^n$?
  \end{enumerate}
\end{problem}

\begin{solution}[i]
\end{solution}

\begin{solution}[ii]
\end{solution}

\begin{solution}[iii]
  Clearly, $\metric(\x, \y)$ satisfies the first two properties of a metric. However, it does not satisfy the triangle inequality.
\end{solution}

\begin{problem}[1.3]
  Consider the following silly metric on $\R^2$:
  \[%
    \metric((x_1,y_1), (x_2, y_2)) = \begin{cases}
      |y_1 - y_2| & \text{if}~x_1 = x_2 \\
      |y_1 - y_2| + 1 & \text{if}~x_1 \neq x_2 \\
    \end{cases}
  .\]%
  \begin{enumerate}
    \item Prove that $\metric$ is a metric, that is, it has the three properties listed in Definition 1.2.

    \item Sketch the open balls of radius $1/2$, $1$, and $2$ around the origin in this metric.

    \item Give an example of a sequence that converges in the Euclidean metric $\metric_2$ but not in our silly metric $\metric$.

    \item Prove that every sequence that converges in $\metric$ also converges $\metric_2$.
  \end{enumerate}
\end{problem}

\begin{solution}[i]
  Clearly, $\metric((x_1, y_1), (x_2, y_2)) = \metric((x_2, y_2), (x_1, y_1))$. Also $\metric((x_1, y_1), (x_2, y_2)) = 0$ if and only if $(x_1, y_1) = (x_2, y_2)$.

  Let $(x_1, y_1), (x_2, y_2), (x_3, y_3) \in \R^2$. For the triangle inequality, observe that we may write
  \[%
    \metric((x_i, y_i), (x_j, y_j)) = |y_i - y_j| + \delta_{x_i}^{x_j}
  ,\]%
  where $\delta_a^b$ is the indicator function that is $0$ if $a = b$ and $1$ if $a \neq b$. The usual triangle inequality in $\R$ gives $|y_1 - y_3| \le |y_1 - y_2| + |y_2 - y_3|$, and the indicator satisfies
  \[%
    \delta_{x_1}^{x_3} \le \delta_{x_1}^{x_2} + \delta_{x_2}^{x_3}
  .\]%
  Adding these inequalities yields
  \[%
    \metric((x_1,y_1),(x_3,y_3)) \le \metric((x_1, y_1), (x_2, y_2)) + \metric((x_2, y_2), (x_3, y_3))
  ,\]%
  so the triangle inequality holds. Therefore $\metric$ is a metric on $\R^2$.
\end{solution}

\begin{solution}[ii]
\end{solution}

\begin{solution}[iii]
\end{solution}

\begin{solution}[iv]
\end{solution}

\begin{problem}[1.4]
  Let $X$ be any set, and let $\metric_X$ be the \emph{discrete metric}
  \[%
    \metric_X(p, q) = \begin{cases}
      0 & \text{if}~p = q \\
      1 & \text{if}~p \neq q \\
    \end{cases}
  .\]%
  \begin{enumerate}
    \item Prove that $\metric_X$ is a metric

    \item Let $(Y, \metric_Y)$ be another metric space (not necessarily discrete). Prove that every map $f : X \to Y$ is continuous.

    \item Prove that a sequence $p_1, p_2, p_3, \cdots \in X$ converges in the discrete metric if and only if it is eventually constant.
  \end{enumerate}
\end{problem}

\begin{solution}[i]
  Clearly, $\metric_X(p, q) = \metric_X(q, p)$ for all $p, q \in X$. Also, $\metric_X(p, q) = 0$ if and only if $p = q$. We now prove that the metric satisfies the triangle inequality. Let $p, q, r \in X$. If any two of the points are equal, then the triangle inequality hold trivially. So, assume that $p, q, r$ are all distinct. Then,
  \[%
    \metric_X(p, r) = 1 \le \metric_X(p, q) + \metric_X(q, r) = 1 + 1 = 2
  .\]%
  Thus, $\metric_X$ is a metric on $X$.
\end{solution}

\begin{solution}[ii]
  Let $p \in X$ and $\epsilon > 0$. Choose $\delta = 1/2$. Then $B(p,\delta) = \{p\}$. Thus, if $q \in B(p,\delta)$, we must have $q = p$, and so
  \[%
    d_Y(f(p), f(q)) = d_Y(f(p), f(p)) = 0 < \epsilon
  .\]%
  Hence $f$ is continuous at $p$. Since $p \in X$ was arbitrary, $f$ is continuous on $X$. 
\end{solution}

\begin{solution}[iii]
  Assume that the sequence $(p_n) = (p_1, p_2, p_3, \dots)$ converges to $p \in X$. By definition, for every $\epsilon > 0$, there exists $N \in \N$ such that for all $n \geq N$, $\metric_X(p_n, p) < \epsilon$. Choose $\epsilon = 1/2$. Then, for all $n \geq N$, $\metric_X(p_n, p) < 1/2$. Since the distance between any two distinct points in $X$ is $1$, this implies that $p_n = p$ for all $n \geq N$. Thus, the sequence is eventually constant.

  Conversely, assume that the sequence $(p_n)$ is eventually constant. Then there exists $N \in \N$ and $p \in X$ such that $p_n = p$ for all $n \geq N$. Let $\epsilon > 0$ be arbitrary. For this $N$, we have $\metric_X(p_n, p) = 0 < \epsilon$ whenever $n \geq N$. Hence, by definition, $(p_n)$ converges to $p$.
\end{solution}

\begin{problem}[1.11]
  Let $(X, \metric_X)$ and $(Y, \metric_Y)$ be metric spaces, let $(p_n) = (p_1, p_2, p_3, \cdots)$ be a sequence that converges to a point $\ell$ in $X$, and let $f : X \to Y$ be continuous at $\ell$. Prove that the sequence $f(p_n) = f(p_1), f(p_2), f(p_3), \cdots$ converges to $f(\ell)$ in $Y$.
\end{problem}

\begin{solution}
  Since $f$ is continuous at $\ell$ (since $\ell \in X$), for every $\epsilon > 0$, there exists $\delta > 0$ such that for all $x \in X$ with $\metric_X(x, \ell) < \delta$, we have $\metric_Y(f(x), f(\ell)) < \epsilon$. Since $(p_n)$ converges to $\ell$, for this $\delta > 0$, there exists $N \in \N$ such that for all $n \geq N$, $\metric_X(p_n, \ell) < \delta$. Therefore, for all $n \geq N$, we have $\metric_Y(f(p_n), f(\ell)) < \epsilon$. This shows that the sequence $(f(p_n))$ converges to $f(\ell)$ in $Y$.
\end{solution}
