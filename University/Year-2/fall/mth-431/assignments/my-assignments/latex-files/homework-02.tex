\begin{problem}[1.5]
  In Example 1.8(a) we saw a sequence in $C([0, 1])$ that converges in the $L^1$ metric but not in the sup metric. Prove that the reverse cannot happen: every sequence that converges in the sup metric converges in the $L^1$ metric.
\end{problem}

\begin{solution}
\end{solution}

\begin{problem}[1.6]
  Let $(X, \metric)$ be a metric space. Prove the reverse \emph{triangle inequality}:
\end{problem}

\begin{solution}
\end{solution}

\begin{problem}[1.7]
  Let $(X, \metric_X)$ and $(Y, \metric_Y)$ and $(Z, \metric_Z)$ be metric spaces. Let $f : X \to Y$ be continuous at a point $p \in X$, and let $g : Y \to Z$ be continuous at $f(p)$. Prove that $g \circ f$ is continuous at $p$.
\end{problem}

\begin{solution}
\end{solution}

\begin{problem}[2.1]
  Sketch each subset of $\R^2$ and find its closure, interior, and boundary in the Euclidean metric:
  \begin{enumerate}
    \item $A_1 = \{(x, y) \mid 0 < x \le 1 0 \le y < 1\}$.
    \item $A_2 = \{(x, y) \mid 0 < x \le 1, y = 0\}$.
    \item $A_3 = \{(x, y) \mid x \in \Q~\text{or}~y \in \Q\}$.
    \item $A_4 = \{(x, y) \mid x \ne 0~\text{or}~y = 0\}$.
  \end{enumerate}
\end{problem}

\begin{solution}[(i)]
\end{solution}

\begin{solution}[(ii)]
\end{solution}

\begin{solution}[(iii)]
\end{solution}

\begin{solution}[(iv)]
\end{solution}

\begin{problem}[2.4]
  Prove the analogue of Proposition 2.7 for closures, without appealing to Proposition 2.6.
  \begin{enumerate}
    \item $\bar{\bar{A}} = \bar{A}$.
    \item If $A \subset B$, then $\bar{A} \subset \bar{B}$.
    \item $\bar{A} \cup \bar{B} = \overline{A \cup B}$.
    \item $\overline{A \cap B} \subset \bar{A} \cap \bar{B}$.

      Give an example to show that the inclusion can be strict.
  \end{enumerate}
\end{problem}

\begin{solution}[(i)]
\end{solution}

\begin{solution}[(ii)]
\end{solution}

\begin{solution}[(iii)]
\end{solution}

\begin{solution}[(iv)]
\end{solution}

\begin{problem}[2.6]
  We saw in Example 2.8 that the inclusion $\sint A \cup \sint B \subset \sint(A \cup B)$ can be strict. Prove however that if $\bar{A} \cup \bar{B} = \emptyset$ then $\sint A \cup \sint B = \sint(A \cup B)$.
\end{problem}

\begin{solution}
\end{solution}
