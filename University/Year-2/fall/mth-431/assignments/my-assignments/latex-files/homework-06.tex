\begin{problem}[6.1]
  \begin{enumerate}
    \item Let $f : \R \to \R$ be defined by $f(x) = x^2$. Find $f(A)$ for the following subsets $A \subset \R$: the intervals $[-1,1]$, $[-1,1)$, $(-1,1)$, $[0,1]$, $[0,1)$, $(0,1)$, and the singletons $\{-1\}$, $\{0\}$, and $\{1\}$.

    \item Now let $f : X \to Y$ be arbitrary, and let $A, B \subset X$. Prove that if $A \subset B$ then $f(A) \subset f(B)$. Prove that $f(A \cup B) = f(A) \cup f(B)$. Prove that $f(A \cap B) \subset f(A) \cap f(B)$, but give an example where they are not equal.
  \end{enumerate}
\end{problem}

\begin{solution}[(i)]
  For $f(x) = x^2$ we have:
  \begin{gather*}
    f([-1,1]) = [0,1], \quad f([-1,1)) = [0,1], \quad f((-1,1)) = [0,1), \quad f([0,1]) = [0,1] \\
    f([0,1)) = [0,1), \quad f((0,1)) = (0,1), \quad f(\{-1\}) = \{1\}, \quad f(\{0\}) = \{0\}, \quad f(\{1\}) = \{1\}
  .\qedhere\end{gather*}
\end{solution}

\begin{solution}[(ii)]
  Let $f : X \to Y$ and let $A, B \subset X$.

  If $A \subset B$, then for every $x \in A$ we have $x \in B$. Hence $f(x) \in f(B)$ for all $x \in A$, so $f(A) \subset f(B)$.

  For the union,
  \[%
    f(A \cup B) = \{f(x) \mid x \in A \cup B\} = \{f(x) \mid x \in A\} \cup \{f(x) : x \in B\} = f(A) \cup f(B)
  .\]%

  For the intersection, if $x \in A \cap B$ then $x \in A$ and $x \in B$, hence $f(x) \in f(A)$ and $f(x) \in f(B)$. Therefore $f(A \cap B) \subset f(A) \cap f(B)$.

  Equality doesn't have to hold. Take, for instance, $f : \R \to \R$ be given by $f(x) = x^2$, and $A = \{-1\}$, $B = \{1\}$. Then, we have that $A \cap B = \emptyset$. Therefore, $f(A \cap B) = \emptyset$, but $f(A) = \{1\} = f(B)$, so
  \[%
    f(A) \cap f(B) = \{1\} \neq \emptyset = f(A \cap B)
  .\]%
  Thus, in this case, we have $f(A \cap B) \neq f(A) \cap f(B)$.
\end{solution}

\begin{problem}[6.2]
  \begin{enumerate}
    \item Let $f : \R \to \R$ be defined by $f(x) = x^2$. Find $f^{-1}(B)$ for the following subsets $B \subset \R$: the intervals $[-1,1]$, $[-1,1)$, $(-1,1)$, $[0,1]$, $[0,1)$, $(0,1)$, and the singletons $\{-1\}$, $\{0\}$, and $\{1\}$.

    \item Now let $f : X \to Y$ be arbitrary, and let $A, B \subset Y$. Prove that if $A \subset B$ then $f^{-1}(A) \subset f^{-1}(B)$. Prove that $f^{-1}(A \cup B) = f^{-1}(A) \cup f^{-1}(B)$. Prove that $f^{-1}(A \cap B) = f^{-1}(A) \cap f^{-1}(B)$.
  \end{enumerate}
\end{problem}

\begin{solution}[(i)]
  For $f(x) = x^2$ we have:
  \begin{gather*}
    f^{-1}([-1,1]) = [-1,1],
    \quad f^{-1}([-1,1)) = (-1,1),
    \quad f^{-1}((-1,1)) = (-1,1) \\
    \quad f^{-1}([0,1]) = [-1,1],
    \quad f^{-1}([0,1)) = (-1,1),
    \quad f^{-1}((0,1)) = (-1,0) \cup (0,1) \\
    \quad f^{-1}(\{-1\}) = \emptyset,
    \quad f^{-1}(\{0\}) = \{0\},
    \quad f^{-1}(\{1\}) = \{-1,1\}
  .\qedhere\end{gather*}
\end{solution}

\begin{solution}[(ii)]
  Let $f : X \to Y$ and let $A,B \subset Y$.

  If $A \subset B$ and $x \in f^{-1}(A)$, then $f(x) \in A \subset B$, so $x \in f^{-1}(B)$. Thus $f^{-1}(A) \subset f^{-1}(B)$. For the union, we have
  \[%
    f^{-1}(A \cup B) = \{x \in X \mid f(x) \in A \cup B\} = f^{-1}(A) \cup f^{-1}(B)
  ,\]%
  and for the intersection, we have
  \[%
    f^{-1}(A \cap B) = \{x \mid f(x) \in A \cap B\} = \{x \mid f(x) \in A~\text{and}~f(x) \in B\} = f^{-1}(A) \cap f^{-1}(B)
  .\qedhere\]%
\end{solution}

\begin{problem}[6.5]
  The function $f : \R \to \R$ defined by
  \[%
    f(x) = \begin{cases}
      x & \text{if}~x \le 0 \\
      x + 1 & \text{if}~x > 0
    \end{cases}
  ,\]%
  is discontinuous (with respect to the usual metric). Find an open set $V \subset \R$ such that $f^{-1}(V)$ is not open.
\end{problem}

\begin{solution}
  Take the open set $V = (-1/2, 1/2) \subset \R$. Now, we compute $f^{-1}(V)$. If $x \le 0$ then $f(x) = x$, so $x \in f^{-1}(V)$ precisely when $-1/2 < x \le 0$, i.e. this contribution is $(-1/2, 0]$. If $x > 0$ then $f(x) = x +1$, and $x +1 \in (-1/2, 1/2)$ would force $-3/2 < x < -1/2$, which is impossible for $x > 0$. Hence the $x > 0$ part contributes nothing. Therefore
  \[%
    f^{-1}(V) = \left(-\frac{1}{2}, 0\right]
  .\]%
  The set $(-1/2, 0]$ is not open in $\R$ because it contains the point $0$ but no open interval around $0$ is contained. Thus $f^{-1}(V)$ is not open.
\end{solution}

\begin{problem}[6.6]
  Use Proposition 6.6 to prove that the function $f : \R \to \R$ defined by $f(x) = x^2$ is continuous with respect to the usual metric, as follows.
  \begin{enumerate}
    \item Let $A$ be an open interval of the form $(a, \infty) \subset \R$. Find $f^{-1}(A)$ and observe that it is open.
    \item Let $B$ be an open interval of the form $(-\infty, b) \subset \R$. Find $f^{-1}(B)$ and observe that it is open.
    \item Let $C$ be an open interval of the form $(a, b) \subset \R$. Prove from the previous two parts that $f^{-1}(C)$ is open.
    \item Let $V \subset \R$ be an arbitrary open set. Prove that $V$ is a union of open intervals. Conclude that $f^{-1}(V)$ is open.
    \item Look up a $\delta$-$\epsilon$ proof that $f$ is continuous and reproduce it here, or write one yourself. Which proof would you say is more straightforward?
  \end{enumerate}
\end{problem}

\begin{solution}[(i)]
\end{solution}

\begin{solution}[(ii)]
\end{solution}

\begin{solution}[(iii)]
\end{solution}

\begin{solution}[(iv)]
\end{solution}

\begin{solution}[(v)]
\end{solution}

\begin{problem}[7.1]
  Prove that each of the four topologies on $\R$ given in Example 7.3 is a topology, that is, it satisfies the three conditions in Definition 7.1.
\end{problem}

\renewcommand\B{\mathcal{B}}
\renewcommand\S{\mathcal{S}}
\renewcommand\T{\mathcal{T}}
\begin{solution}
  Let $\T_1 = \{U \subseteq \R \mid U^c~\text{is finite}\} \cup \{\emptyset\}$. Then, $\emptyset, \R \in \T_1$. If $U, V \in \T_1$, then notice that $(U \cap V)^c = U^c \cup V^c$, which is a union of finite sets. Hence, $U \cap V \in \T_1$. For any arbitrary collection $\{U_i\}_{i \in I} \subseteq \T_1$, for any arbitrary index set $I$, we have
  \[%
    \left(\bigcup_{i \in I} U_i\right)^c = \bigcap_{i \in I} U_i^c
  ,\]%
  which is an intersection of finite sets, and hence finite. Therefore, $\bigcup_{i \in I} U_i \in \T_1$. Thus, the finite complement topology is indeed a topology.

  Fix $p \in \R$, and let $\T_2 = \{U \subseteq \R \mid p \in U\} \cup \{\emptyset\}$. Then $\emptyset, \R \in \T_2$. If $U,V \in \T_2$, then $p \in U$ and $p \in V$, so $p \in U \cap V$, hence $U \cap V \in \T_2$. For any collection $\{U_i\}_{i \in I} \subseteq \T_2$, we have $p \in U_i$ for all $i$, so $p \in \bigcup_{i \in I} U_i$, hence $\bigcup_{i \in I} U_i \in \T_2$. Thus $\T_2$ is a topology.

  Let $\S = \{(-\infty, a) \mid a \in \R\}$ and let $\T_3$ be the topology generated by $\S$, i.e., the smallest topology containing all sets of the form $(-\infty, a)$. Since $\emptyset, \R$ are unions of sets in $\S$, they lie in $\T_3$. Let $U, V \in \T_3$. Then $U$ and $V$ are unions of sets from $\S$, and for any $a,b \in \R$,
  \[%
    (-\infty, a) \cap (-\infty, b) = (-\infty, \min\{a, b\})
  ,\]%
  which is again in $\S$. Thus $U \cap V$ is a union of sets in $\S$, hence in $\T_3$. Arbitrary unions of unions of sets in $\S$ remain unions of sets in $\S$, so $\T_3$ is a topology.

  Let $\B = \{[a, b) \mid a < b, a, b \in \R\}$, and let $\T_4$ be the topology generated by $\B$. Then $\emptyset, \R$ are unions of sets in $\B$, so they belong to $\T_4$. For $[a, b), [c, d) \in \B$, we have
  \[%
    [a, b) \cap [c, d) = [\max\{a, c\}, \min\{b, d\})
  ,\]%
  which is either empty or again in $\B$. Hence the intersection of any two basis elements is a union of basis elements. Arbitrary unions of basis elements remain in $\T_4$, so $\T_4$ is a topology.
\end{solution}

\begin{problem}[7.2]
  Find the interiors, closures, and boundaries of the following subsets $A \subset \R$ in the topologies from Example 7.3:
  \begin{enumerate}
    \item $\Z \subset \R$ in the finite complement topology.
    \item $\{0\} \subset \R$ and $\{1\} \subset \R$ in the particular point topology.
    \item $(0, 1) \subset \R$ in the lower semi-continuous topology.
    \item $(0, 1) \subset \R$ in the lower limit topology.
  \end{enumerate}
\end{problem}

\begin{solution}[(i)]
  The interior of $\Z$ is the largest open set contained in $\Z$. Any nonempty open set is cofinite and thus contains non-integer real numbers, so no nonempty open set is contained in $\Z$. Hence $\sint(\Z) = \emptyset$.

  Closed sets are exactly $\R$ and all finite subsets. Since $\Z$ is infinite, the only closed set containing $\Z$ is $\R$. Thus $\overline{\Z} = \R$.

  The boundary is just given by $\partial \Z = \overline{\Z} \setminus \sint(\Z) = \R \setminus \emptyset = \R$.
\end{solution}

\begin{solution}[(ii)]
  First consider $\{0\}$. If $p = 0$, then $\{0\}$ contains $p$, so it is open. Therefore $\sint(\{0\}) = \{0\}$. To compute the closure, note that a closed set contains $\{0\}$ only if it is $\mathbb{R}$ (since any closed set other than $\mathbb{R}$ must omit $p$, but here $p = 0$). Hence $\overline{\{0\}} = \mathbb{R}$, and $\partial\{0\} = \mathbb{R} \setminus \{0\}$.

  Now assume $p \neq 0$. Then $\{0\}$ does not contain $p$, so it is closed. The largest open subset of $\{0\}$ is $\emptyset$, so $\sint(\{0\}) = \emptyset$. Since $\{0\}$ is closed, $\overline{\{0\}} = \{0\}$, and therefore $\partial\{0\} = \{0\}$.

  Next consider $\{1\}$. If $p = 1$, then $\{1\}$ is open, so $\sint(\{1\}) = \{1\}$. The only closed set containing $\{1\}$ is $\mathbb{R}$, so $\overline{\{1\}} = \mathbb{R}$, and thus $\partial\{1\} = \mathbb{R} \setminus \{1\}$.

  Finally, if $p \neq 1$, then $\{1\}$ does not contain $p$, hence it is closed. Its interior is $\emptyset$, its closure is $\{1\}$, and therefore $\partial\{1\} = \{1\}$.
\end{solution}

\begin{solution}[(iii)]
  To find the interior, note that any basic open set $(-\infty,a)$ that is contained in $(0,1)$ must satisfy $a \le 0$, which forces $(-\infty,a) \subseteq (-\infty,0]$. But no such set lies inside $(0,1)$ except the empty set. Hence the interior of $(0,1)$ is empty, so $\sint(A) = \emptyset$.

  To find the closure, suppose $x \in \mathbb{R}$. Then every basic open neighborhood of $x$ is of the form $(-\infty,a)$ with $a > x$, and such a neighborhood always intersects $(0,1)$ whenever $x < 1$, since $(0,1) \subset (-\infty,a)$ for all $a > 1$. If $x \ge 1$, then every basic neighborhood $(-\infty,a)$ of $x$ with $a > x \ge 1$ also contains $(0,1)$. Thus every point $x \in \mathbb{R}$ satisfies the condition for being in the closure, and therefore $\overline{A} = \mathbb{R}$.

  Since $\partial A = \overline{A} \setminus \sint(A)$, we conclude that $\partial(0,1) = \mathbb{R}$.
\end{solution}

\begin{solution}[(iv)]
  To find the interior, observe that for any $x \in (0,1)$, we may choose a basis element $[x, x+\epsilon)$ for some $\epsilon > 0$, and this set is contained in $(0,1)$ provided $x + \epsilon \le 1$. Therefore each $x \in (0,1)$ has a basis neighborhood contained in $(0,1)$, and hence $\sint(A) = (0,1)$.

  To find the closure, let $x < 0$. Then every basis neighborhood $[x, x+\epsilon)$ contains negative points only, and thus cannot intersect $(0,1)$. Hence such $x$ are not in the closure. If $0 \le x \le 1$, then every basis neighborhood $[x, x+\epsilon)$ intersects $(0,1)$, since it contains points in $(0,1)$ whenever $\epsilon > 0$. If $x > 1$, then we may take $\epsilon$ small enough so that $[x, x+\epsilon)$ lies entirely to the right of $1$, hence such neighborhoods do not intersect $(0,1)$. Therefore $\overline{A} = [0,1]$.

  Finally, since $\partial A = \overline{A} \setminus \sint(A)$, we conclude that $\partial(0,1) = [0,1] \setminus (0,1) = \{0,1\}$.
\end{solution}
