\begin{problem}[6.1]
  \begin{enumerate}
    \item Let $f : \R \to \R$ be defined by $f(x) = x^2$. Find $f(A)$ for the following subsets $A \subset \R$: the intervals $[-1,1]$, $[-1,1)$, $(-1,1)$, $[0,1]$, $[0,1)$, $(0,1)$, and the singletons $\{-1\}$, $\{0\}$, and $\{1\}$.

    \item Now let $f : X \to Y$ be arbitrary, and let $A, B \subset X$. Prove that if $A \subset B$ then $f(A) \subset f(B)$. Prove that $f(A \cup B) = f(A) \cup f(B)$. Prove that $f(A \cap B) \subset f(A) \cap f(B)$, but give an example where they are not equal.
  \end{enumerate}
\end{problem}

\begin{solution}[(i)]
\end{solution}

\begin{solution}[(ii)]
\end{solution}

\begin{problem}[6.2]
  \begin{enumerate}
    \item Let $f : \R \to \R$ be defined by $f(x) = x^2$. Find $f^{-1}(B)$ for the following subsets $B \subset \R$: the intervals $[-1,1]$, $[-1,1)$, $(-1,1)$, $[0,1]$, $[0,1)$, $(0,1)$, and the singletons $\{-1\}$, $\{0\}$, and $\{1\}$.

    \item Now let $f : X \to Y$ be arbitrary, and let $A, B \subset Y$. Prove that if $A \subset B$ then $f^{-1}(A) \subset f^{-1}(B)$. Prove that $f^{-1}(A \cup B) = f^{-1}(A) \cup f^{-1}(B)$. Prove that $f^{-1}(A \cap B) \subset f^{-1}(A) \cap f^{-1}(B)$, but give an example where they are not equal.
  \end{enumerate}
\end{problem}

\begin{solution}[(i)]
\end{solution}

\begin{solution}[(ii)]
\end{solution}

7.1, 7.2

\begin{problem}[6.5]
  The function $f : \R \to \R$ defined by
  \[%
    f(x) = \begin{cases}
      x & \text{if}~x \le 0 \\
      x + 1 & \text{if}~x > 0
    \end{cases}
  ,\]%
  is discontinuous (with respect to the usual metric). Find an open set $V \subset \R$ such that $f^{-1}(V)$ is not open.
\end{problem}

\begin{solution}
\end{solution}

\begin{problem}[6.6]
  Use Proposition 6.6 to prove that the function $f : \R \to \R$ defined by $f(x) = x^2$ is continuous with respect to the usual metric, as follows.
  \begin{enumerate}
    \item Let $A$ be an open interval of the form $(a, \infty) \subset \R$. Find $f^{-1}(A)$ and observe that it is open.
    \item Let $B$ be an open interval of the form $(-\infty, b) \subset \R$. Find $f^{-1}(B)$ and observe that it is open.
    \item Let $C$ be an open interval of the form $(a, b) \subset \R$. Prove from the previous two parts that $f^{-1}(C)$ is open.
    \item Let $V \subset \R$ be an arbitrary open set. Prove that $V$ is a union of open intervals. Conclude that $f^{-1}(V)$ is open.
    \item Look up a $\delta$-$\epsilon$ proof that $f$ is continuous and reproduce it here, or write one yourself. Which proof would you say is more straightforward?
  \end{enumerate}
\end{problem}

\begin{solution}[(i)]
\end{solution}

\begin{solution}[(ii)]
\end{solution}

\begin{solution}[(iii)]
\end{solution}

\begin{solution}[(iv)]
\end{solution}

\begin{solution}[(v)]
\end{solution}

\begin{problem}[7.1]
  Prove that each of the four topologies on $\R$ given in Example 7.3 is a topology, that is, it satisfies the three conditions in Definition 7.1.
\end{problem}

\begin{solution}
\end{solution}

\begin{problem}[7.2]
  Find the interiors, closures, and boundaries of the following subsets $A \subset \R$ in the topologies from Example 7.3:
  \begin{enumerate}
    \item $\Z \subset \R$ in the finite complement topology.
    \item $\{0\} \subset \R$ and $\{1\} \subset \R$ in the particular point topology.
    \item $(0, 1) \subset \R$ in the lower semi-continuous topology.
    \item $(0, 1) \subset \R$ in the lower limit topology.
  \end{enumerate}
\end{problem}

\begin{solution}[(i)]
\end{solution}

\begin{solution}[(ii)]
\end{solution}

\begin{solution}[(iii)]
\end{solution}

\begin{solution}[(iv)]
\end{solution}
