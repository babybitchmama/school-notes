\documentclass{report}

% \newcommand\hmwkTitle{Homework 1}
% \newcommand\hmwkDueDate{October 8, 2025}

% \newcommand\hmwkTitle{Homework 2}
% \newcommand\hmwkDueDate{October 15, 2025}

% \newcommand\hmwkTitle{Homework 3}
% \newcommand\hmwkDueDate{October 22, 2025}

% \newcommand\hmwkTitle{Homework 4}
% \newcommand\hmwkDueDate{October 29, 2025}

% \newcommand\hmwkTitle{Homework 5}
% \newcommand\hmwkDueDate{November 5, 2025}

% \newcommand\hmwkTitle{Homework 6}
% \newcommand\hmwkDueDate{November 12, 2025}

\newcommand\hmwkTitle{Homework 7}
\newcommand\hmwkDueDate{November 19, 2025}

% \newcommand\hmwkTitle{Optional Problems}
% \newcommand\hmwkDueDate{December ???, 2025}

\newcommand\uoID{952102243}
\newcommand\hmwkDueTime{23:59}
\newcommand\hmwkClass{Introduction to Abstract Algebra I}
\newcommand\hwkClassShort{MTH-444}
\newcommand\hmwkClassTime{}
\newcommand\hmwkClassInstructor{Victor Ostrik}
\newcommand\hmwkClassInstructorShortName{Victor}
\newcommand\hmwkAuthorName{\textbf{Hashem A. Damrah}}

%%%%%%%%%%%%%%%%%%%%%%%%%%%%%%%%%%%%%%%%%%%%%%%%%%%%%%%%%%%%%%%%%%%%%%%%%%%%%%%%
%                                                                              %
%                              Required Packages                               %
%                                                                              %
%%%%%%%%%%%%%%%%%%%%%%%%%%%%%%%%%%%%%%%%%%%%%%%%%%%%%%%%%%%%%%%%%%%%%%%%%%%%%%%%

% Required for creating documents
\usepackage[utf8]{inputenc}
\usepackage[T1]{fontenc}
\RequirePackage{etex}

% Required math packages
\usepackage{amsmath}
\usepackage{amsfonts}
\usepackage{mathtools}
\usepackage{amsthm}
\usepackage{amssymb}
\usepackage{mathrsfs}

\usepackage{multicol} % for multiple columns
\usepackage[usenames,dvipsnames,pdftex]{color} % Required for nicer colors
\usepackage{hyperref} % Required for hyperlinks
\usepackage{xparse} % Required for \NewDocumentCommand
\usepackage{graphicx} % Required for including images
\usepackage{enumitem} % Required for customizing lists
\usepackage{float} % Required for positioning figures and tables
\usepackage{array} % Required for customizing tables
\usepackage{systeme} % Required for \systeme
\usepackage{cancel} % Required for \cancel
\usepackage{derivative} % Required for \odv and \pdv
\usepackage{authoraftertitle} % Required for \MyTitle
\usepackage{geometry} % Required for customizing page layout
\usepackage{textgreek} % Required for greek letters in text mode
\usepackage{multirow} % Required for multirow in tables
\usepackage{emptypage} % Required for removing page numbers on empty pages
\usepackage{nicematrix} % Required for better matrices
\usepackage{booktabs} % Required for better tables
\usepackage{cellspace} % Required for better spacing in tables
\usepackage{longtable} % Required for long tables
\usepackage{xfrac} % Required for extra fraction options
\usepackage{diagbox} % Required for creating diagonal boxes
\usepackage{polynom} % Required for polynomial long division
\usepackage{setspace} % For line spacing

\usepackage[font=bf]{caption} % Required for customizing captions
\usepackage{subcaption} % Required for creating subfigures
\usepackage{siunitx} % Required for SI units
\usepackage{titletoc} % Required for customizing table of contents
\usepackage{braket} % Required for creating braket notation

% Required for drawing figures
\usepackage{tikz}
\usepackage{pgffor}
\usepackage{tkz-euclide}
\usepackage{tikz-cd}
\usepackage{tikz-3dplot}
\usepackage{circuitikz}

% Required for creating plots
\usepackage{pgfplots}
\usepackage{pgfplotstable}

\usepackage{titling} % Required for customizing title page
\usepackage{ifthen} % Required for if-then-else statements
\usepackage{xifthen} % Required for if-then-else statements
\usepackage{fancyhdr} % Required for customizing headers and footers
\usepackage{import} % Required for importing pdf_tex files
\usepackage{titlesec} % Required for customizing sectioning commands
\usepackage{etex} % Required for more registers

% Required for creating boxes
\usepackage[most,many,breakable]{tcolorbox}

\makeatletter


%%%%%%%%%%%%%%%%%%%%%%%%%%%%%%%%%%%%%%%%%%%%%%%%%%%%%%%%%%%%%%%%%%%%%%%%%%%%%%%%
%                                                                              %
%                                Basic Settings                                %
%                                                                              %
%%%%%%%%%%%%%%%%%%%%%%%%%%%%%%%%%%%%%%%%%%%%%%%%%%%%%%%%%%%%%%%%%%%%%%%%%%%%%%%%

\ifx\nauthor\undefined
  \def\nauthor{Hashem A. Damrah}
\else
\fi

\ifx\class\undefined
  \def\class{report}
\else
\fi

\newcommand\globalcolor[1]{%
  \color{#1}\global\let\default@color\current@color
}

% Symbols
\let\oldlimsymbol\lim\renewcommand\lim{\displaystyle\oldlimsymbol}
\let\oldsumsymbol\sum\renewcommand\sum{\displaystyle\oldsumsymbol}
\let\oldlimsupsymbol\limsup\renewcommand\limsup{\displaystyle\oldlimsupsymbol}
\let\oldliminfsymbol\liminf\renewcommand\liminf{\displaystyle\oldliminfsymbol}
\let\implies\Rightarrow
\let\impliedby\Leftarrow
\let\iff\Leftrightarrow
\let\epsilon\varepsilon

% Geometry
\geometry{
  top=1in,
  bottom=1in,
  right=1in,
  left=1in,
}

% Tables
\newcolumntype{C}{>{\Centering\arraybackslash}X}
\setlength{\tabcolsep}{5pt}
\renewcommand\arraystretch{1.5}
\renewcommand\thetable{\Roman{table}}
\captionsetup[figure]{font=small}
\captionsetup{justification=centering}
\setlength\cellspacetoplimit{6pt}
\setlength\cellspacebottomlimit{6pt}

% Lists
\renewcommand{\labelitemi}{--}
\renewcommand{\labelitemii}{$\circ$}
\renewcommand{\labelenumi}{\textnormal{(\roman{*})}}

% Center Title Page
\let\@real@maketitle\maketitle
% \renewcommand\maketitle{
%   \@real@maketitle
%   \begin{center}
%     \begin{minipage}[c]{0.9\textwidth}
%       \centering\footnotesize
%       These notes are not endorsed by the lecturers, and
%       I have modified them (often significantly) after lectures. They are
%       nowhere near accurate representations of what was actually lectured, and
%       in particular, all errors are almost surely mine.

%       \textit{Disclaimer:} This document will inevitably contain some
%       mistakes--both simple typos and legitimate errors. Keep in mind that these
%       are the notes of an undergraduate student in the process of learning the
%       material himself, so take what you read with a grain of salt. If you find
%       mistakes and feel like telling me, I will be grateful and happy to hear
%       from you, even for the most trivial of errors. You can reach me by email,
%       in English, Arabic, Hebrew, or Spanish at
%       \href{mailto:singularisartt@gmail.com}{singularisartt@gmail.com}.
%     \end{minipage}
%   \end{center}
% }
\renewcommand\maketitlehooka{\null\mbox{}\vfill}
\renewcommand\maketitlehookd{\vfill\null}

% Footnote Line
\renewcommand\footnoterule{\hrule\vspace{0.1cm}}

% Modify Links Color
\hypersetup{
  colorlinks,
  linkcolor=black!90,
  citecolor=black,
  urlcolor=cyan!70!black,
}

%%%%%%%%%%
%  TikZ  %
%%%%%%%%%%

\usetikzlibrary{
  shadings,
  intersections,
  angles,
  quotes,
  calc,
  positioning,
  3d,
  perspective,
  arrows,
  arrows.meta,
  patterns,
  decorations.markings,
  bending,
  decorations.pathreplacing,
  calligraphy,
  backgrounds,
}

\tikzoption{canvas is xy plane at z}[]{%
	\def\tikz@plane@origin{\pgfpointxyz{0}{0}{#1}}%
	\def\tikz@plane@x{\pgfpointxyz{1}{0}{#1}}%
	\def\tikz@plane@y{\pgfpointxyz{0}{1}{#1}}%
	\tikz@canvas@is@plane}

\usetikzlibrary{shapes.arrows}
\newcommand\graphslopefield{
  \pgfmathsetmacro{\hx}{(\xmax-\xmin)/\nx}
  \pgfmathsetmacro{\hy}{(\ymax-\ymin)/\ny}
  \foreach \i in {0,...,\nx}
  \foreach \j in {0,...,\ny}{
    \pgfmathsetmacro{\yprime}{f({\xmin+\i*\hx},{\ymin+\j*\hy})}
    \draw[black,shift={({\xmin+\i*\hx},{\ymin+\j*\hy})}]
    (0,0)--($(0,0)!2mm!(.1,.1*\yprime)$);
  }

  \draw[->] (\xmin-.5,0)--(\xmax+.5,0) node[below right] {$x$};
  \draw[->] (0,\ymin-.5)--(0,\ymax+.5) node[above left] {$y$};
}

\tikzset{derivative/.style={color=gray,mark=none,line width=0.5pt,solid}}
\tikzset{asymptote/.style={color=gray,mark=none,line width=1pt,<->,dashed}}
\tikzset{soldot/.style={color=black,fill=black,only marks,mark=*}}
\tikzset{holdot/.style={color=black,fill=white,only marks,mark=*}}

\tikzset{>=stealth}
\tikzset{->-/.style={decoration={markings,mark=at position .5 with {\arrow{>}}},postaction={decorate}}}

%%%%%%%%%%%%%%
%  PgfPlots  %
%%%%%%%%%%%%%%

\pgfplotsset{width=7cm,compat=1.8}
\pgfplotsset{compat=newest}

\usepgfplotslibrary{patchplots}
\usepgfplotslibrary{fillbetween}
\usetikzlibrary{intersections}

\pgfplotsset{plot/.style={color=red,mark=none,line width=1pt,<->,solid}}
\pgfplotsset{asymptote/.style={color=gray,mark=none,line width=1pt,<->,dashed}}
\pgfplotsset{soldot/.style={color=red,only marks,mark=*}}
\pgfplotsset{holdot/.style={color=red,fill=white,only marks,mark=*}}
\pgfplotsset{blankgraph/.style={xmin=-10,xmax=10,ymin=-10,ymax=10,axis line style= {-, draw opacity=0 },axis lines=box,major tick length=0mm,xtick={-10,-9,...,10},ytick={-10,-9,...,10},grid=major,yticklabels={,,},xticklabels={,,},minor xtick=,minor ytick=,xlabel={},ylabel={},width=0.75\textwidth,grid style={solid,gray!40}}}

\pgfplotscreateplotcyclelist{stylelist}{
  plot \\
}

\def\axisdefaultwidth{175pt}
\def\axisdefaultheight{\axisdefaultwidth}

\pgfplotsset{every axis/.append style={
    axis x line=middle,
    axis y line=middle,
    axis line style={<->},
    xlabel={$x$},
    ylabel={$y$},
    xmin=-7,xmax=7,
    ymin=-7,ymax=7,
    yticklabel style={inner sep=0.333ex},
    minor xtick={-7,-6,...,7},
    minor ytick={-7,-6,...,7},
    scale only axis,
    cycle list name=stylelist,
    tick label style={font=\footnotesize},
    legend cell align=left,
    grid=minor,
    grid style={solid,gray!40},
    try min ticks=6,
  },
  framed/.style={axis background/.style={draw=gray}}
}

\pgfplotsset{axis background/.style={draw=gray}}


%%%%%%%%%%%%%%%%%%%%%%%%%%%%%%%%%%%%%%%%%%%%%%%%%%%%%%%%%%%%%%%%%%%%%%%%%%%%%%%%
%                                                                              %
%                           School Specific Commands                           %
%                                                                              %
%%%%%%%%%%%%%%%%%%%%%%%%%%%%%%%%%%%%%%%%%%%%%%%%%%%%%%%%%%%%%%%%%%%%%%%%%%%%%%%%

%%%%%%%%%%%%%%%%%%%%%%
%  Helpful Commands  %
%%%%%%%%%%%%%%%%%%%%%%

\newcommand\resetcounters{
  \setcounter{section}{0}
  \setcounter{subsection}{0}
  \setcounter{subsubsection}{0}
  \setcounter{paragraph}{0}
  \setcounter{subparagraph}{0}
}

\newcommand*\cleartoleftpage{%
  \clearpage
  \thispagestyle{empty}
  \ifodd\value{page}\else\hbox{}\newpage\fi
}

%%%%%%%%%%%%%%%%%%%%%%%%%%%%%
%  Lecture/Chapter Command  %
%%%%%%%%%%%%%%%%%%%%%%%%%%%%%

\def\notenum{}
\def\@note{}%
\newcommand\lecture[3]{
  \ifthenelse{\isempty{#3}}{%
    \def\@note{Lecture #1}%
  }{%
    \def\@note{Lecture #1: #3}%
  }%
  \section*{\@note\hfill{\small\textnormal{#2}}}
}

% Intro
\newcommand\createintro{
  \maketitle

  \pagenumbering{roman}
  \begin{center}
    \textbf{{\LARGE Introduction}}
    \phantomsection\addcontentsline{toc}{chapter}{0\hspace{0.8em}Introduction}
  \end{center}

  \begingroup
  \IfFileExists{./intro.tex}{
    \setlength{\parindent}{1cm}
    Lecture notes from the course \MyTitle, given by professor Victor Ostrik at the \faculty~at \location~in the academic year \academicyear, during the \term term. This course covers symbolic logic, basic set theory, analyzing functions and their properties, modular arithmetic, counting and other problems in discrete mathematics, induction, and convergence of sequences and continuity of functions. Credit for the material in these notes is due to professor Victor, while the structure is loosely taken from the \href{https://www.amazon.com/Mathematical-Reasoning-Writing-Proof-2nd/dp/0131877186}{Mathematical Reasoning: Writing and Proof} textbook. The credit for the typesetting is my own.

\textit{Disclaimer:} This document will inevitably contain some mistakes--both simple typos and legitimate errors. Keep in mind that these are the notes of an undergraduate student in the process of learning the material himself, so take what you read with a grain of salt. If you find mistakes and feel like telling me, I will be grateful and happy to hear from you, even for the most trivial of errors. You can reach me by email, in English, Arabic, Hebrew, or Spanish at \href{mailto:singularisartt@gmail.com}{singularisartt@gmail.com}.

  }{}
  \endgroup

  \pagestyle{fancy}
  \renewcommand\headrulewidth{0pt}

  \fancyhead{}
  \fancyfoot[C]{%
    \textit{For more notes like this, visit \href{\linktootherpages}{\shortlinkname}}.%
  }%

  \begin{tcolorbox}[
      enhanced,
      colback=white,
      center upper,
      size=fbox,
      drop shadow southwest,
      sharp corners,
    ]
    \term: \academicyear, \\
    Last Update: \today, \\
    \faculty, \location.
  \end{tcolorbox}

  \newpage
  \tableofcontents

  \pagenumbering{arabic}
  \setcounter{page}{1}

  \renewcommand\headrulewidth{0.4pt}
  \fancyhead[R]{\@note}
  \fancyhead[L]{\nauthor}
  \fancyfoot[C]{\thepage}
}

%%%%%%%%%%%%%%%%%%%%%
%  Random Commands  %
%%%%%%%%%%%%%%%%%%%%%

% Circle
\newcommand*\circled[1]{
  \tikz[baseline=(char.base)] {
    \node[shape=circle,draw,inner sep=1pt] (char) {#1};
  }
}

% Import Figures
\newcommand\incimg[2][1]{%
  \includegraphics[width=#1\columnwidth]{figures/#2}%
}

\newcommand\incfig[2][1]{
  \def\svgwidth{#1\columnwidth}
  \import{figures}{#2.pdf_tex}
}

% Correct
\newcommand\correct[1]{\textcolor{correct}{#1}}
\newcommand\incorrect[1]{{\color{incorrect}#1}}
\newcommand\inctocor[2]{\incorrect{#1} \ensuremath{\to} \correct{#2}}

% Bracket
\renewcommand\bra[1]{\left\langle#1\right|}
\renewcommand\ket[1]{\left|#1\right\rangle}
\renewcommand\braket[2]{\left\langle#1\middle|#2\right\rangle}
\renewcommand\ang[1]{\left\langle#1\right\rangle}

% For diagonal strikeout in red
\newcommand\rcancel[1]{\renewcommand\CancelColor{\color{red}}\cancel{#1}}

% Logic
\renewcommand\nmid{\not|~}

% For differentials
\newcommand\dd[1]{\textrm{d}#1}
\newcommand\dA{\textrm{d}A}
\newcommand\dm{\textrm{d}m}
\newcommand\dt{\textrm{d}t}
\newcommand\dT{\textrm{d}T}
\newcommand\du{\textrm{d}u}
\newcommand\dv{\textrm{d}v}
\newcommand\dx{\textrm{d}x}
\newcommand\dy{\textrm{d}y}
\newcommand\dz{\textrm{d}z}

% Helpful text in math
\newcommand\abs{\text{abs}}
\newcommand\echelon{\underrightarrow{\textrm{ echelon form }}}
\newcommand\rref{\underrightarrow{\textrm{ rref }}}
\newcommand\pick[1]{\xrightarrow[#1]{\textrm{ pick }}}
\newcommand\generalsol{\xrightarrow[\textrm{solution}]{\textrm{ general }}}
\newcommand\ngeneralsol{\parbox{4em}{general \\ solution}\textrm{:}}
\renewcommand\and{\text{and}}
\newcommand\aand{\quad\text{and}\quad}
\newcommand\adj{\text{adj}}
\newcommand\qtq[1]{\quad\textrm{#1}\quad}
\newcommand\oor{\quad\text{or}\quad}
\newcommand\Col{\textrm{Col}}
\newcommand\Nul{\textrm{Null}}
\newcommand\range{\textrm{Range}}
\newcommand\Tr{\textrm{Tr}}
\newcommand\Row{\textrm{Row}}
\newcommand\rank{\textrm{rank}}
\newcommand\dist{\text{dist}}
\newcommand\sz{\stackrel{\textrm{set}}{=}}
\newcommand\ce{\overset{\checkmark}{=}}
\newcommand\proj{\text{proj}}
\newcommand\comp{\text{comp}}
\newcommand\Card{\text{Card}}
\renewcommand\mod{\text{mod}}
\newcommand\st{\text{such that}}
\newcommand\geogebra{\textsf{GeoGebra}}
\newcommand\Sspan{\text{Span}}
\renewcommand\Re{\text{Re}}
\renewcommand\Im{\text{Im}}

% Laplace
\newcommand\laplace[1]{\mathscr{L}\left\{#1\right\}}
\newcommand\ilaplace[1]{\mathscr{L}^{-1}\left\{#1\right\}}

% Vectors
\renewcommand\a{\mathbf{a}}
\renewcommand\b{\mathbf{b}}
\renewcommand\c{\mathbf{c}}
\renewcommand\d{\mathbf{d}}
\newcommand\e{\mathbf{e}}
\newcommand\f{\mathbf{f}}
\newcommand\g{\mathbf{g}}
\newcommand\n{\mathbf{n}}
\newcommand\p{\mathbf{p}}

\newcommand\RR{\mathbf{R}}
\newcommand\FF{\mathbf{F}}

\renewcommand\r{\mathbf{r}}
\newcommand\rr{\mathbf{r}^{\prime}}
\newcommand\rrr{\mathbf{r}^{\prime\prime}}

\newcommand\Ta{\mathbf{T}}
\newcommand\Taa{\mathbf{T}^{}}

\renewcommand\u{\mathbf{u}}
\newcommand\uu{\mathbf{u}^{\prime}}
\newcommand\uuu{\mathbf{u}^{\prime\prime}}

\renewcommand\v{\mathbf{v}}
\newcommand\vv{\mathbf{v}^{\prime}}
\newcommand\vvv{\mathbf{v}^{\prime\prime}}

\newcommand\w{\mathbf{w}}
\newcommand\x{\mathbf{x}}
\newcommand\y{\mathbf{y}}
\newcommand\z{\mathbf{z}}

\newcommand\zero{\mathbf{0}}

% Hat vectors
\newcommand\ah{\hat{\mathbf{a}}}
\newcommand\bh{\hat{\mathbf{b}}}
\newcommand\ch{\hat{\mathbf{c}}}
\renewcommand\dh{\hat{\mathbf{d}}}
\newcommand\eh{\hat{\mathbf{e}}}
\newcommand\ph{\hat{\mathbf{p}}}
\newcommand\uh{\hat{\mathbf{u}}}
\newcommand\vh{\hat{\mathbf{v}}}
\newcommand\wh{\hat{\mathbf{w}}}
\newcommand\xh{\hat{\mathbf{x}}}
\newcommand\yh{\hat{\mathbf{y}}}
\newcommand\zh{\hat{\mathbf{z}}}

% Unit vectors
\newcommand\ui{\mathbf{i}}
\newcommand\uj{\mathbf{j}}
\newcommand\uk{\mathbf{k}}

% Subspaces
\newcommand\B{\mathcal{B}}
\newcommand\CC{\mathbb{C}}
\newcommand\C{\mathcal{C}}
\newcommand\D{\mathcal{D}}
\newcommand\E{\mathcal{E}}
\newcommand\F{\mathcal{F}}
\newcommand\I{\mathbb{I}}
\newcommand\N{\mathbb{N}}
\newcommand\Q{\mathbb{Q}}
\newcommand\R{\mathbb{R}}
\newcommand\U{\mathcal{U}}
\newcommand\W{\mathbf{w}}
\newcommand\X{\mathcal{X}}
\newcommand\Y{\mathcal{Y}}
\newcommand\Z{\mathbb{Z}}

% Create command to box equations
\newcommand*\colorboxed{}
\def\colorboxed#1#{%
  \colorboxedAux{#1}%
}
\newcommand*\colorboxedAux[3]{%
  \begingroup
    \colorlet{cb@saved}{.}%
    \color#1{#2}%
    \boxed{%
      \color{cb@saved}%
      #3%
    }%
  \endgroup
}
\newcommand\empheq[1]{%
  \colorboxed{black}{%
    \begin{aligned}[b]
      #1
    \end{aligned}%
  }%
}


%%%%%%%%%%%%%%%%%%%%%%%%%%%%%%%%%%%%%%%%%%%%%%%%%%%%%%%%%%%%%%%%%%%%%%%%%%%%%%%%
%                                                                              %
%                                 Environments                                 %
%                                                                              %
%%%%%%%%%%%%%%%%%%%%%%%%%%%%%%%%%%%%%%%%%%%%%%%%%%%%%%%%%%%%%%%%%%%%%%%%%%%%%%%%

\theoremstyle{definition}
\newtheorem*{assumption}{Assumption}
\newtheorem*{claim}{Claim}
\newtheorem*{conjecture}{Conjecture}
\newtheorem*{definition}{Definition}
\newtheorem*{example}{Example}
\newtheorem*{notation}{Notation}
\newtheorem*{proposition}{Proposition}
\newtheorem*{question}{Question}
\newtheorem*{problem}{Problem}
\newtheorem*{rrule}{Rule}

\newtheorem*{remark}{Remark}
\newtheorem*{note}{Note}
\newtheorem*{warning}{Warning}

\theoremstyle{plain}
\newtheorem*{corollary}{Corollary}
\newtheorem*{lemma}{Lemma}
\newtheorem*{theorem}{Theorem}
\newtheorem*{worksheet}{Worksheet}

\newcommand\frmebox[1][]{%
  \begin{tcolorbox}[%
    title={\sffamily\color{black}#1},
    colback=white,
    enhanced,
    attach boxed title to top center={
      yshift=-3mm,
      yshifttext=-1mm,
    },
    boxed title style={
      size=small,
      colback=white,
      frame code={},
    },
    coltext=black,
    frame hidden,
    borderline east={0.5pt}{0pt}{black},
    borderline west={0.5pt}{0pt}{black},
    borderline north={0.5pt}{0pt}{black},
    borderline south={0.5pt}{0pt}{black},
    breakable,
    parskip=0pt,
  ]
}

\renewenvironment{frame}[1][]{%
  \frmebox[#1]%
}{%
  \end{tcolorbox}%
}

\makeatother

\def\type{exercise}
\DeclareMathOperator{\sgn}{sgn}

\begin{document}
  \maketitle
  \pagestyle{plain}
  \pagenumbering{gobble}
  \mbox{}\newpage
  \pagestyle{fancy}
  \pagenumbering{arabic}
  \setcounter{page}{1}
  % \begin{problem}[1.1]\leavevmode
  \begin{enumerate}
    \item For each of the three metrics in Example 1.4, sketch the open ball of some radius $r > 0$ around the origin in $\R^2$:
      \[%
        B_r(0) = \{(x, y) \in \R^2 \mid \metric{((x, y), 0)} < r\}
      .\]%

    \item For one of the three metrics (your choice), prove or give a counterexample to the following statement: a sequence of points $(x_1, y_1), (x_2, y_2), \cdots \in \R^2$ converges to a limit $(x, y)$ if and only if $x_n \to x$ and $y_n \to y$ separately, as sequences in $\R$ with the usual metric.

    \item Why is
      \[%
        \metric(\x, \y) = \min(\{|x_1 - y_1|, |x_2 - y_2|, \cdots, |x_n - y_n|\})
      ,\]%
      not a metric on $\R^n$?
  \end{enumerate}
\end{problem}

\begin{solution}[i]
\end{solution}

\begin{solution}[ii]
\end{solution}

\begin{solution}[iii]
\end{solution}

\begin{problem}[1.3]
  Consider the following silly metric on $\R^2$:
  \[%
    \metric((x_1,y_1), (x_2, y_2)) = \begin{cases}
      |y_1 - y_2| & \text{if}~x_1 = x_2 \\
      |y_1 - y_2| + 1 & \text{if}~x_1 \neq x_2 \\
    \end{cases}
  .\]%
  \begin{enumerate}
    \item Prove that $\metric$ is a metric, that is, it has the three properties listed in Definition 1.2.

    \item Sketch the open balls of radius $1/2$, $1$, and $2$ around the origin in this metric.

    \item Give an example of a sequence that converges in the Euclidean metric $\metric_2$ but not in our silly metric $\metric$.

    \item Prove that every sequence that converges in $\metric$ also converges $\metric_2$.
  \end{enumerate}
\end{problem}

\begin{solution}[i]
\end{solution}

\begin{solution}[ii]
\end{solution}

\begin{solution}[iii]
\end{solution}

\begin{solution}[iv]
\end{solution}

\begin{problem}[1.4]
  Let $X$ be any set, and let $\metric_X$ be the \emph{discrete metric}
  \[%
    \metric_X(p, q) = \begin{cases}
      0 & \text{if}~p = q \\
      1 & \text{if}~p \neq q \\
    \end{cases}
  .\]%
  \begin{enumerate}
    \item Prove that $\metric_X$ is a metric

    \item Let $(Y, \metric_Y)$ be another metric space (not necessarily discrete). Prove that every map $f : X \to Y$ is continuous.

    \item Prove that a sequence $p_1, p_2, p_3, \cdots \in X$ converges in the discrete metric if and only if it is eventually constant.
  \end{enumerate}
\end{problem}

\begin{solution}[i]
\end{solution}

\begin{solution}[ii]
\end{solution}

\begin{solution}[iii]
\end{solution}

\begin{problem}[1.11]
  Let $(X, \metric_X)$ and $(Y, \metric_Y)$ be metric spaces, let $p_1, p_2, p_3, \cdots$ be a sequence that converges to a point $\ell$ in $X$, and let $f : X \to Y$ be continuous at $\ell$. Prove that the sequence $f(p_1), f(p_2), f(p_3), \cdots$ converges to $f(\ell)$ in $Y$.
\end{problem}

\begin{solution}
\end{solution}

  % \begin{problem}[2.10]
  Let $n$ be a positive integer and let $n\Z = \{nm \mid m \in \Z\}$.
  \begin{enumerate}
    \item Show that $\bra{n\Z, +}$ is a group.

    \item Show that $\bra{n\Z, +} \cong \bra{\Z, +}$.
  \end{enumerate}
\end{problem}

\begin{solution}[(i)]
\end{solution}

\begin{solution}[(ii)]
\end{solution}

\begin{problem}[2.19]
  Let $S$ be the set of all real numbers except $-1$. Define $*$ on $S$ by
  \[%
    a * b = a + b + ab
  .\]%
  \begin{enumerate}
    \item Show that $*$ gives a binary operation on $S$.

    \item Show that $\bra{S, *}$ is a group.

    \item Find the solution of the equation $2 * x * 3 = 7$ in $S$.
  \end{enumerate}
\end{problem}

\begin{solution}[(i)]
\end{solution}

\begin{solution}[(ii)]
\end{solution}

\begin{solution}[(iii)]
\end{solution}

\begin{problem}[2.28]
  An element $a \ne e$ in a group is said to have order 2 if $a * a = e$. Prove that if $G$ is a group and $a \in G$ has order 2, then for any $b \in G$, $b' * a * b$ also has order $2$.
\end{problem}

\begin{solution}
\end{solution}

\begin{problem}[2.29]
  Show that if $G$ is a finite group with identity $e$ and with an even number of elements, then there is $a \ne e$ in G such that $a * a = e$.
\end{problem}

\begin{solution}
\end{solution}

\begin{problem}[2.30]
  Let $\R^*$ be the set of all real numbers except 0. Define $*$ on $\R^*$ by letting $a * b = |a|b$.
  \begin{enumerate}
    \item Show that $*$ gives an associative binary operation on $\R^*$.

    \item Show that there is a left identity for $*$ and a right inverse for each element in $\R^*$.

    \item Is $\R^*$ with this binary operation a group?

    \item Explain the significance of this exercise.
  \end{enumerate}
\end{problem}

\begin{solution}[(i)]
\end{solution}

\begin{solution}[(ii)]
\end{solution}

\begin{solution}[(iii)]
\end{solution}

\begin{solution}[(iv)]
\end{solution}

\begin{problem}[2.31]
  If $*$ is a binary operation on a set $S$, an element $x$ of $S$ is an \emph{idempotent for} $*$ if $x * x = x$. Prove that a group has exactly one idempotent element. (You may use any theorems proved so far in the text.)
\end{problem}

\begin{solution}
\end{solution}

\begin{problem}[2.32]
  Show that every group $G$ with identity $e$ and such that $x * x = e$ for all $x \in G$ is abelian. [Hint: Consider $(a * b) * (a * b)$.]
\end{problem}

\begin{solution}
\end{solution}

\begin{problem}[2.33]
  Let $G$ be an abelian group and let $c^n = c * c * \cdots * c$ for $n$ factors $c$, where $c \in G$ and $n \in \Z^+$. Give a mathematical induction proof that $(a * b)^n = (a^n) * (b^n)$ for all $a, b \in G$.
\end{problem}

\begin{solution}
\end{solution}

\begin{problem}[2.34]
  Suppose that $G$ is a group and $a, b \in G$ satisfy $a * b = b * a'$ where as usual, $a'$ is the inverse for $a$. Prove that $b * a = a' * b$.
\end{problem}

\begin{solution}
\end{solution}

\begin{problem}[2.36]
  Let $G$ be a group with finite number of elements. Show that for any $a \in G$, there exists $n \in \Z^+$ such that $a^n = e$. See Exercise 33 for the meaning of $a^n$. [Hint: Consider $e, a, a^2, a^3, \cdots, a^m$, where $m$ is the number of elements in $G$, and use the cancellation laws.]
\end{problem}

\begin{solution}
\end{solution}

  % \renewcommand\type{exercise}
\setcounter{chapter}{2}
\setcounter{section}{2}

\begin{exercise}[1]
\end{exercise}

\begin{exersolution}[1]
\end{exersolution}

\newpage

\begin{exercise}[2]
  Verify, using the definition of convergence of a sequence, that the following
  sequences converge to the proposed limit.
  \begin{enumerate}
    \item $\lim_{n \to \infty} \frac{2n + 1}{5n + 4} = \frac{2}{5}$.
    \item $\lim_{n \to \infty} \frac{2n^2}{n^3 + 3} = 0$.
    \item $\lim_{n \to \infty} \frac{\sin(n^2)}{\sqrt[3]{n}} = 0$.
  \end{enumerate}
\end{exercise}

\begin{exersolution}[2]
  The definition of convergence of a sequence is as follows
  \[%
    \empheq{\lim_{n \to \infty} a_n = a \iff \forall \epsilon > 0, \exists N~\st~\forall n > N \implies \lvert a_n - a \rvert < \epsilon}
  \]%

  \begin{enumerate}
    \item \begin{proof}
        Given $\epsilon > 0$, choose
        \[%
          N = \frac{3}{25\epsilon} - \frac{4}{5}
        .\]%
        Suppose $n > N > 0$. Then,
        \[%
          \lvert a_n - a \rvert = \left\lvert \frac{2n + 1}{5n + 4} - \frac{2}{5} \right\rvert = \frac{3}{5(5n + 4)} < \frac{3}{5(5N + 4)} < \frac{3}{5\left(5\left(\frac{3}{25\epsilon} - \frac{4}{5}\right) + 4\right)} = \frac{3}{5\left(\frac{3}{5\epsilon}\right)} = \epsilon
        .\qedhere\]%
      \end{proof}

    \item \begin{proof}
        Given $\epsilon > 0$, choose
        \[%
          N = \frac{1}{\epsilon}
        .\]%
        Suppose $n > N > 0$. Then,
        \[%
          \lvert a_n - a \rvert = \left\lvert \frac{2n^2}{n^3 + 3} - 0 \right\rvert = \frac{2n^2}{n^3 + 3} < \frac{2n^2}{n^3} = \frac{1}{n} < \frac{1}{N} < \epsilon
        .\qedhere\]%
      \end{proof}

    \item \begin{proof}
        Given $\epsilon > 0$, choose
        \[%
          N = \frac{1}{\epsilon^3}
        .\]%
        Suppose $n > N > 0$. Then,
        \[%
          \lvert a_n - a \rvert = \left\lvert \frac{\sin(n^2)}{\sqrt[3]{n}} \right\rvert \le \frac{1}{\sqrt[3]{n}} < \frac{1}{\sqrt[3]{N}} < \epsilon
        .\qedhere\]%
      \end{proof}
  \end{enumerate}
\end{exersolution}

\newpage

\begin{exercise}[3]
\end{exercise}

\begin{exersolution}[3]
\end{exersolution}

\newpage

\begin{exercise}[4]
\end{exercise}

\begin{exersolution}[4]
\end{exersolution}

\newpage

\begin{exercise}[6]
\end{exercise}

\begin{exersolution}[6]
\end{exersolution}

\newpage

\setcounter{section}{3}

\begin{exercise}[1]
\end{exercise}

\begin{exersolution}[1]
\end{exersolution}

\newpage

\begin{exercise}[3]
\end{exercise}

\begin{exersolution}[3]
\end{exersolution}

  % \begin{problem}[5.30]
  Find the order of the cyclic subgroup $\Z_{16}$ generated by 12.
\end{problem}

\begin{solution}
  Notice that the cyclic subgroup of $\Z_{16}$ generated by 12 is given by
  \[%
    \bra{12} = \{12 \cdot n \pmod{16} \mid n \in \Z\} = \{0, 4, 8, 12\}
  .\]%
  Thus, the order of the cyclic subgroup is 4.
\end{solution}

\begin{problem}[5.32]
  Find the order of the cyclic subgroup $S_8$ generated by $(2, 4, 6, 9)(3, 5, 7)$.
\end{problem}

\begin{solution}
  Notice that the permutation $(2, 4, 6, 9)(3, 5, 7)$ is the product of two disjoint cycles: a 4-cycle $(2, 4, 6, 9)$ and a 3-cycle $(3, 5, 7)$. The order of a permutation is the least common multiple (LCM) of the lengths of its disjoint cycles. Therefore, the order of the permutation is
  \[%
    \text{lcm}(4, 3) = 12
  .\]%
  Thus, the order of the cyclic subgroup generated by $(2, 4, 6, 9)(3, 5, 7)$ is 12.
\end{solution}

\begin{problem}[5.40]
  Show by means of an example that it is possible for the quadratic equation $x^2 = e$ have more than two solutions in some group with identity $e$.
\end{problem}

\begin{solution}
  Take the algebraic group $\bra{V_4, *}$, where $V_4$ is the Klein four-group defined as $V_4 = \{e, a, b, c\}$ with the operation $*$ defined by the following Cayley table:
  \[%
    \begin{array}{c|cccc}
      * & e & a & b & c \\
      \hline
      e & e & a & b & c \\
      a & a & e & c & b \\
      b & b & c & e & a \\
      c & c & b & a & e \\
    \end{array}
  \]%
  In this group, we can see that:
  \[%
    a * a = e, \quad b * b = e, \quad c * c = e
  .\]%
  Thus, the equation $x^2 = e$ has four solutions: $e$, $a$, $b$, and $c$. This shows that it is possible for the quadratic equation $x^2 = e$ to have more than two solutions in a group.
\end{solution}

\begin{problem}[5.41]
  Let $B$ be a subset of $A$, and let $b$ be a particular element of $B$. Determine whether the subset, $\{\sigma \in S_A \mid \sigma(b) = b\}$, of the symmetric group $S_A$ is a subgroup of $S_A$ under the induced operation.
\end{problem}

\begin{solution}
  Define the set $B_S = \{\sigma \in S_A \mid \sigma(b) = b\}$. It's clearly non-empty, since the identity permutation $e$ is in $B_S$, as $e(b) = b$. Notice that, for any two permutations $\sigma_1, \sigma_2 \in B_S$, we have
  \[%
    (\sigma_1 \circ \sigma_2)(b) = \sigma_1(\sigma_2(b)) = \sigma_1(b) = b
  .\]%
  Thus, the composition $\sigma_1 \circ \sigma_2$ is also in $B_S$, showing closure under the group operation. Finally, if $\sigma \in B_S$, then $\sigma \in S_A$. Since $S_A$ is a group, the inverse permutation $\sigma^{-1}$ also exists in $S_A$. Moreover, $\sigma^{-1}(b) = b$. Thus, $\sigma^{-1} \in B_S$. Since $B_S$ is closed under the group operation, contains the identity element, and contains inverses, we conclude that $B_S$ is a subgroup of $S_A$ under the induced operation.
\end{solution}

\begin{problem}[5.42]
  Let $B$ be a subset of $A$, and let $b$ be a particular element of $B$. Determine whether the subset, $\{\sigma \in S_A \mid \sigma(b) = B\}$, of the symmetric group $S_A$ is a subgroup of $S_A$ under the induced operation.
\end{problem}

\begin{solution}
  Define the set $B_S = \{\sigma \in S_A \mid \sigma(b) = B\}$. Again, it's clearly non-empty, since the identity permutation $e$ is in $B_S$, as $e(b) = b \in B$. However, consider two permutations $\sigma_1, \sigma_2 \in B_S$. We have
  \[%
    (\sigma_1 \circ \sigma_2)(b) = \sigma_1(\sigma_2(b)) = \sigma_1(B)
  ,\]%
  which may not equal $B$ unless $\sigma_1$ maps all elements of $B$ back to $B$. Thus, the composition $\sigma_1 \circ \sigma_2$ may not be in $B_S$, showing that $B_S$ is not closed under the group operation. Therefore, we conclude that $B_S$ is not a subgroup of $S_A$ under the induced operation.
\end{solution}

\begin{problem}[5.45]
  Let $\Phi : G \to G'$ be an isomorphism of a group $\bra{G, *}$ with a group $\bra{G', *'}$. Prove that if $H$ is a subgroup of $G$, then $\Phi[H] = \{\Phi(h) \mid h \in H\}$ is a subgroup of $G'$. That is, an isomorphism carries subgroups into subgroups
\end{problem}

\begin{solution}
  Let $H$ be a subgroup of $G$. Since $H$ is a subgroup of $G$, it contains the identity element $e$. Therefore, $\Phi(e_G) = e'$ is in $\Phi[H]$, so $\Phi[H]$ is non-empty.

  Let $\Phi(h_1), \Phi(h_2) \in \Phi[H]$ for some $h_1, h_2 \in H$. Since $\Phi$ is an isomorphism, we have
  \[%
    \Phi(h_1 * h_2) = \Phi(h_1) *' \Phi(h_2)
  .\]%
  We have $h_1 * h_2 \in H$ because $H$ is a subgroup of $G$. Thus, $\Phi(h_1 * h_2) \in \Phi[H]$. Therefore, $\Phi[H]$ is closed under the operation $*'$.

  Let $\Phi(h) \in \Phi[H]$ for some $h \in H$. Since $\Phi$ is an isomorphism, we have
  \[%
    \Phi(h^{-1}) = (\Phi(h))^{-1}
  .\]%
  Since $H$ is a subgroup, $h^{-1} \in H$. Thus, $\Phi(h^{-1}) \in \Phi[H]$. Therefore, $\Phi[H]$ contains inverses.

  Hence, $\Phi[H]$ is non-empty, closed under the operation $*'$, and contains inverses. Thus, $\Phi[H]$ is a subgroup of $G'$.
\end{solution}

\begin{problem}[5.46]
  Let $\Phi : G \to G'$ be an isomorphism of a group $\bra{G, *}$ with a group $\bra{G', *'}$. Prove that if there is an $a \in G$ such that $\bra{a} = G$, then $G'$ is cyclic.
\end{problem}

\begin{solution}
  Since $\bra{a} = G$, every element $g \in G$ can be expressed as $g = a^n$ for some integer $n$. Consider the element $\Phi(a) \in G'$. We will show that $\bra{\Phi(a)} = G'$.

  Let $g' \in G'$. Since $\Phi$ is an isomorphism, there exists a unique $g \in G$ such that $\Phi(g) = g'$. Since $g \in G$, we can write $g = a^n$ for some integer $n$. Therefore,
  \[%
    g' = \Phi(g) = \Phi(a^n) = (\Phi(a))^n
  .\]%
  This shows that every element $g' \in G'$ can be expressed as a power of $\Phi(a)$.

  Thus, $\bra{\Phi(a)} = G'$, and hence $G'$ is cyclic.
\end{solution}

\begin{problem}[5.48]
  Find an example of a group $G$ and two subgroups $H$ and $K$ such that the set in Exercise 47 is not a subgroup of $G$.
\end{problem}

\begin{solution}
  Let $G = S_3$, the symmetric group on $\{1, 2, 3\}$. Define two subgroups
  \[%
    H = \langle (1\ 2) \rangle = \{e, (1\ 2)\}~\text{and}~ K = \langle (1\ 3) \rangle = \{e, (1\ 3)\}
  .\]%
  Both $H$ and $K$ are subgroups of $S_3$ of order $2$.

  Consider the set $HK = \{hk \mid h \in H,\, k \in K\}$. Its elements consist of $\{e, (1\ 2), (1\ 3), (1\ 3\ 2)\}$.

  This set has $4$ elements. By Lagrange's theorem, any subgroup of $S_3$ must have an order dividing $|S_3| = 6$.  Since there is no subgroup of order $4$ in $S_3$, it follows that $HK$ is \emph{not} a subgroup of $G$.
\end{solution}

\begin{problem}[5.53]
  Prove that if $G$ is an abelian group, written multiplicatively, with identity element $e$, then all elements $x$ of $G$ satisfying the equation $x^2 = e$ form a subgroup $H$ of $G$.
\end{problem}

\begin{solution}
  Let $H = \{x \in G \mid x^2 = e\}$. The identity element $e$ of $G$ satisfies $e^2 = e$, so $e \in H$. Thus, $H$ is non-empty. Let $x, y \in H$. Then $x^2 = e$ and $y^2 = e$. We have
  \[%
    (xy)^2 = xyxy = x(yx)y = x(xy)y = (xx)(yy) = e \cdot e = e
  .\]%
  Thus, $xy \in H$. Let $x \in H$. We have
  \[%
    (x^{-1})^2 = x^{-1}x^{-1} = (xx)^{-1} = e^{-1} = e
  .\]%
  Thus, $x^{-1} \in H$.

  Since $H$ is non-empty, closed under the group operation, and contains inverses, we conclude that $H$ is a subgroup of $G$.
\end{solution}

\begin{problem}[5.55]
  Find a counterexample to Exercise 53 if the assumption of abelian is dropped.
\end{problem}

\begin{solution}
  Consider the non-abelian group $G = S_3$, the symmetric group on $\{1,2,3\}$. The elements of $S_3$ are
  \[%
    S_3 = \{e,\ (1\ 2),\ (1\ 3),\ (2\ 3),\ (1\ 2\ 3),\ (1\ 3\ 2)\}
  .\]%
  The elements satisfying $x^2 = e$ are exactly the identity and the transpositions, so
  \[%
    H = \{x\in S_3 \mid x^2 = e\} = \{e, (1\ 2), (1\ 3), (2\ 3)\}
  .\]%
  To see that $H$ is not a subgroup, note that it is not closed under the group operation. For instance,
  \[%
    (1\ 2)(1\ 3) = (1\ 3\ 2) \not\in H
  ,\]%
  because $(1\ 3\ 2)^2=(1\ 2\ 3) \neq e$. Thus $H$ fails to be closed, so it is not a subgroup of $S_3$.
\end{solution}

  % \renewcommand\T{\mathbb{T}}
\renewcommand\S{\mathbb{S}}

\begin{problem}[5.1]
  Consider the map $\Phi : \R^4 \to \R^2$ defined by
  \[%
    \Phi(x, y, s, t) = (x^2 + y, x^2 + y^2 + s^2 + t^2 + y)
  .\]%
  Show that $(0, 1)$ is a regular value of $\Phi$, and that the level set $\Phi^{-1}(0, 1)$ is diffeomorphic to $\S^2$.
\end{problem}

\begin{solution}
\end{solution}

\begin{problem}[5.2]
  Prove Theorem 5.11 (the boundary of a manifold with boundary is an embedded submanifold).
\end{problem}

\begin{solution}
\end{solution}

\begin{problem}[5.3]
  Prove Proposition 5.21 (sufficient conditions for immersed submanifolds to be embedded).
\end{problem}

\begin{solution}
\end{solution}

\begin{problem}[5.4]
  Show that the image of the curve $\beta : (-\pi, \pi) \to \R^2$ of Example 4.19 is not an embedded submanifold of $\R^2$. [Be careful: this is not the same as showing that $\beta$ is not an embedding.]
\end{problem}

\begin{solution}
\end{solution}

\begin{problem}[5.5]
  Let $\gamma : \R \to \T^2$ be the curve of Example 4.20. Show that $\gamma(\R)$ is not an embedded submanifold of the torus. [Remark: the warning in Problem 5-4 applies in this case as well.]
\end{problem}

\begin{solution}
\end{solution}

  % \begin{problem}[1]
  Evaluate $\displaystyle \int_C \FF \cdot \dd{\r}$ for the vector field $\FF = (x + z)\ui + 2y\uj + (y + 2x)\uk$ where $C$ is
  \begin{enumerate}
    \item the line segment from $(-1, 0, 1)$ and $(0, -1, 2)$.

    \item the curve $\r(t) = \left\langle t - 1, t^2 - 2t, t^3 + 1
      \right\rangle$ from $0 \le t \le 1$.
  \end{enumerate}
\end{problem}

\begin{proof}[Solution to (i)]
  Let $C$ be the line segment from $P(-1, 0, 1)$ to $Q(0, -1, 2)$. Then, we have
  \begin{align*}
    \phantom{\implies}&~\r(t) = (1 - t)P + tQ = (1 - t)(-1, 0, 1) + t(0, -1, 2) = \langle -1 + t, -t, 1 + t \rangle \\
    \implies&~\r'(t) = \langle 1, -1, 1 \rangle
  .\end{align*}
  Converting $\FF$ from a vector function of $x$ and $y$ to a vector function of
  $t$, we have
  \[%
    \FF(x, y) = \langle x + z, 2y, y + 2x \rangle = \langle (-1 + t) + (-1 + t), -2t, -t + (-1 + t) \rangle = \langle 2t, -2t, -2 + t \rangle = \FF(t)
  .\]%
  Therefore, the line integral over the vector field $\FF$ along the line
  segment $C$ is
  \begin{align*}
    \int_C \FF \cdot \dd{\r} &= \int_0^1 \FF(t) \cdot \r'(t) \,\dt \\
                             &= \int_0^1 (1) \cdot (2t) + (-1) \cdot (-2t) + (1) \cdot (-2 + t) \,\dt \\
                             &= \int_0^1 2t + 2t -2 + t \,\dt \\
                             &= \int_0^1 5t - 2 \,\dt = \left(\frac{5}{2} - 2\right) = \frac{1}{2}
  .\qedhere\end{align*}
\end{proof}

\begin{proof}[Solution to (ii)]
  Let $C$ be the curve $\r(t) = \langle t - 1, t^2 - 2t, t^3 + 1 \rangle$ where
  $0 \le t \le 1$. Then, we have
  \[%
    \r'(t) = \left\langle 1, 2t - 2, 3t^2 \right\rangle
  .\]%
  Converting $\FF$ from a vector function of $x$ and $y$ to a vector function of
  $t$, we have
  \begin{align*}
    \FF(x, y) = \left\langle x + z, 2y, y + 2x \right\rangle &= \left\langle (t - 1) + \left(t^3 + 1\right), 2\left(t^2 - 2t\right), \left(t^2 - 2t\right) + 2(t - 1) \right\rangle \\
                                                  &= \left\langle t^3 + t, 2t^2 - 4t, t^2 - 2 \right\rangle
  .\end{align*}
  Evaluating the dot product of $\FF$ and $\r'$, we have
  \begin{align*}
    \FF(t) \cdot \r'(t) &= \left\langle t^3 + t, 2t^2 - 4t, t^2 - 2 \right\rangle \cdot \left\langle 1, 2t - 2, 3t^2 \right\rangle \\
                        &= \left(t^3 + t\right) \cdot (1) + \left(2t^2 - 4t\right) \cdot (2t - 2) + \left(t^2 - 2\right) \cdot \left(3t^2\right) \\
                        &= t^3 + t + 4t^3 - 4t^2 - 8t^2 + 8t + 3t^4 - 6t^2 \\
                        &= 3t^4 + 5t^3 - 18t^2 + 9t
  .\end{align*}
  Therefore, the line integral over the vector field $\FF$ along the curve $C$
  is
  \begin{align*}
    \int_C \FF \cdot \dd{\r} &= \int_0^1 \FF(t) \cdot \r'(t) \,\dt \\
                             &= \int_0^1 3t^4 + 5t^3 - 18t^2 + 9t \,\dt \\
                             &= \frac{3}{5} + \frac{5}{4} - \frac{18}{3} + \frac{9}{2} = \frac{7}{20}
  .\qedhere\end{align*}
\end{proof}

\begin{problem}[2]
  Compute the amount of work done by the vector field $\FF = \langle -y, x, x -
  z \rangle$ moving a particle along the curve of intersection of the surfaces
  $x^2 + y^2 + z^2 = 9$ and $y - x + z = 1$ oriented in the counterclockwise
  direction about the cylinder.

  Note: Do not use a method we haven’t discussed yet. The use of anything other
  than directly evaluating the line integral will not be accepted.
\end{problem}

\begin{proof}[Solution]
  Let $C$ be the curve of intersection of the surfaces $x^2 + y^2 + z^2 = 9$ and
  $y - x + z = 1$. Converting to cylindrical coordinates, we have
  \begin{align*}
    \begin{cases}
      x^2 + y^2 + z^2 = 9 \\
      y - x + z = 1
    \end{cases}
    &\implies
    \begin{cases}
      r^2 + z^2 = 9 \\
      r\sin\theta - r\cos\theta + z = 1
    \end{cases}
    \implies
    \begin{cases}
      r^2 + z^2 = 9 \\
      r(\sin\theta - \cos\theta) + z = 1
    \end{cases} \\
    &\implies z = 1 - r(\sin(\theta) - \cos(\theta))
  .\end{align*}
  To parameterize $C$, notice that $x^2 + y^2 + z^2 = 9$ is just a sphere.
  Therefore, we have $r = 3$, $x = 3\cos(\theta)$, and $y = 3\sin(\theta)$.
  Using $r = 3$, we have the following parameterization for $C$
  \[%
    \r(t) = \left\langle 3\cos(\theta), 3\sin(\theta), 1 - 3(\sin(\theta) - \cos(\theta)) \right\rangle
  .\]%
  Evaluating $\r'(t)$ gives us
  \[%
    \r'(t) = \langle -3\sin(\theta), 3\cos(\theta), 3(\cos(\theta) + \sin(\theta)) \rangle
  .\]%
  The bounds for $\theta$ is $0 \le \theta \le 2\pi$. Converting $\FF$ from a
  vector function of $x$ and $y$ to a vector function of $t$, we have
  \begin{align*}
    \FF(t) = \langle -y(t), x(t), x(t) - z(t) \rangle &= \langle -3\sin(\theta), 3\cos(\theta), 3\cos(\theta) - \left(1 - 3\left(\sin(\theta) - \cos(\theta)\right)\right) \\
                                                      &= \langle -3\sin(\theta), 3\cos(\theta), 3\sin(\theta) - 1 \rangle
  .\end{align*}
  Evaluating the dot product of $\FF$ and $\r'$, we have
  \begin{align*}
    \FF(t) \cdot \r'(t) &= \left\langle -3\sin(\theta), 3\cos(\theta), 3\sin(\theta) - 1 \right\rangle \cdot \left\langle -3\sin(\theta), 3\cos(\theta), 3(\cos(\theta) + \sin(\theta)) \right\rangle \\
                        &= \left(-3\sin(\theta)\right) \cdot \left(-3\sin(\theta)\right) + \left(3\cos(\theta)\right) \cdot \left(3\cos(\theta)\right) + \left(3\sin(\theta) - 1\right) \cdot \left(3(\cos(\theta) + \sin(\theta))\right) \\
                        &= 9(\sin^2(\theta) + \cos^2(\theta)) + (-9\sin^2(\theta) - 9\sin(\theta)\cos(\theta) + 3\cos(\theta) + 3\sin(\theta)) \\
                        &= 9 - 9\sin^2(\theta) - 9\sin(\theta)\cos(\theta) + 3\cos(\theta) + 3\sin(\theta)
  .\end{align*}
  Computing the work integral over the vector field $\FF$ along the curve $C$,
  we have
  \begin{align*}
    W = \int_C \FF \cdot \dd{\r} &= \int_0^{2\pi} \FF(t) \cdot \r'(t) \,\dt \\
                                 &= \int_0^{2\pi} 9 - 9\sin^2(\theta) - 9\sin(\theta)\cos(\theta) + 3\cos(\theta) + 3\sin(\theta) \,\dt \\
                                 &= \int 9 \,\dd{\theta} - 9\int \sin^2(\theta) \,\dd{\theta} - 9\int \sin(\theta)\cos(\theta) \,\dd{\theta} + 3\int \cos(\theta) \,\dd{\theta} + 3\int \sin(\theta) \,\dd{\theta} \bigg\vert_0^{2\pi} \\
                                 &= 9\theta - \frac{9}{2}(\theta - \sin(\theta)\cos(\theta)) - \frac{9}{2}\cos^2(\theta) + 3\sin(\theta) - 3\cos(\theta) \bigg\vert_0^{2\pi} \\
                                 &= \left[9(2\pi) - \frac{9}{2}(2\pi - \sin(2\pi)\cos(2\pi)) - \frac{9}{2}\cos^2(2\pi) + 3\sin(2\pi) - 3\cos(2\pi)\right] \\
                                 &\phantom{=} -\left[9(0) - \frac{9}{2}(0 - \sin(0)\cos(0)) - \frac{9}{2}\cos^2(0) + 3\sin(0) - 3\cos(0)\right] \\
                                 &= \left[18\pi - \frac{9}{2}(2\pi - (0)(-1)) - \frac{9}{2}(1) + 3(0) - 3(1)\right] \\\
                                 &\phantom{=} -\left[0 - \frac{9}{2}(0 - (0)(1)) - \frac{9}{2}(1) + 3(0) - 3(1)\right] \\
                                 &= 18\pi - 9\pi - \frac{9}{2} - 3 + \frac{9}{2} + 3 = 9\pi
  .\qedhere\end{align*}
\end{proof}

\begin{problem}[3]
  Evaluate $\displaystyle\int_C \FF \cdot \dd{\r}$. Use the fundamental theorem
  for line integrals whenever it applies.
  \begin{enumerate}
    \item $\FF = \langle 2xe^{xy} + x^2ye^{xy} + 3x^2, x^3e^{xy} + 2\sin(y)
      \rangle$, $C$ is the line segment from $(-1, 0)$ to $(0, 3)$.

    \item $\FF = \langle y^3 - 2x, 3xy^2 + \sin(\pi y) \rangle$, $C$ is the path
      $y = \sqrt{x}$ from $(1,1 )$ to $(4, 2)$.

    \item $\FF = \langle 6xy - z^2, 3x^2 + 6y^2, 1 - 2xz \rangle$, $C$ is the
      circular helix $\r(t) = \langle t, 2\cos(t), 2\sin(t) \rangle$, $0 \le t
      \le \pi$.

    \item $\FF = \langle y + z, x - 2z, x + 2y \rangle$, $C$ is the intersection
      of sphere $x^2 + y^2 + z^2 = 4$ and the plane $x = 1$ in the first octant
      oriented upward.
  \end{enumerate}
\end{problem}

\begin{proof}[Solution to (i)]
\end{proof}

\begin{proof}[Solution to (ii)]
\end{proof}

\begin{proof}[Solution to (iii)]
\end{proof}

\begin{proof}[Solution to (iv)]
\end{proof}

\begin{problem}[4]
  Consider $\FF = \langle P, Q \rangle$ where $\displaystyle P(x, y) =
  \frac{-y}{x^2 + y^2}$ and $\displaystyle Q(x, y) = \frac{x}{x^2 + y^2}$.
  \begin{enumerate}
    \item Show $\displaystyle\pdv{P}{y} = \pdv{Q}{x}$ on the domain of $\FF$.

    \item Use the definition of the line integral to show that
      $\displaystyle\int_C \FF \cdot \dd{\r} = 2\pi$ where $C$ is the circle
      $x^2 + y^2 = a^2$, counterclockwise orientation, for any constant $a > 0$.
      Is $\FF$ conservative?
  \end{enumerate}
\end{problem}

\begin{proof}[Solution to (i)]
  We have
  \begin{align*}
    \pdv{P}{y} = \pdv{}{y}\left(\frac{-y}{x^2 + y^2}\right) &= \frac{-(x^2 + y^2) - (-y)(2y)}{(x^2 + y^2)^2} = \frac{-x^2 - y^2 + 2y^2}{(x^2 + y^2)^2} = \frac{y^2 - x^2}{(x^2 + y^2)^2} \\
    \pdv{Q}{x} = \pdv{}{x}\left(\frac{x}{x^2 + y^2}\right) &= \frac{(x^2 + y^2) - x(2x)}{(x^2 + y^2)^2} = \frac{x^2 + y^2 - 2x^2}{(x^2 + y^2)^2} = \frac{y^2 - x^2}{(x^2 + y^2)^2}
  .\end{align*}
  Therefore, $\displaystyle\pdv{P}{y} = \pdv{Q}{x}$ on the domain of $\FF$.
\end{proof}

\begin{proof}[Solution to (ii)]
  We parameterize the circle $C$
  \[%
    \r(t) = \langle a\cos(t), a\sin(t) \rangle
  .\]%
  Evaluating $\r'(t)$ gives us
  \[%
    \r'(t) = \langle -a\sin(t), a\cos(t) \rangle
  ,\]%
  where $0 \le t \le 2\pi$. Converting $\FF$ from a vector function of $x$ and
  $y$ to a vector function of $t$, we have
  \[%
    \FF(t) = \left\langle \frac{-a\sin(t)}{a^2}, \frac{a\cos(t)}{a^2} \right\rangle = \left\langle \frac{-\sin(t)}{a}, \frac{\cos(t)}{a} \right\rangle
  .\]%
  Evaluating the dot product of $\FF$ and $\r'$, we have
  \[%
    \FF(t) \cdot \r'(t) = \left\langle \frac{-\sin(t)}{a}, \frac{\cos(t)}{a} \right\rangle \cdot \left\langle -a\sin(t), a\cos(t) \right\rangle = \frac{-\sin(t)}{a} \cdot (-a\sin(t)) + \frac{\cos(t)}{a} \cdot (a\cos(t)) = 1
  .\]%
  Thus, the line integral evaluates to
  \[%
    \int_C \FF \cdot \dd{\r} = \int_0^{2\pi} 1 \,\dt = 2\pi
  .\]%

  A vector field $\FF = \langle P, Q \rangle$ is conservative if and only if
  there exists a function $f(x, y)$ such that $\nabla f = \langle P, Q \rangle$.
  A necessary condition for $\FF$ to be conservative is that
  $\displaystyle\pdv{P}{y} = \pdv{Q}{x}$, which we verified in part (i).

  However, $\FF$ is not defined at $(0, 0)$, and the domain $\R^2 - \{(0, 0)\}$
  is not simply connected, since any loop enclosing the origin cannot be
  continuously shrunk to a point without leaving the domain.  A conservative
  vector field must be path-independent, meaning that the line integral around
  any closed curve should be zero.

  Since we computed
  \[%
    \oint_C \FF \cdot \dd{\r} = 2\pi \neq 0
  ,\]%
  we conclude that $\FF$ is not conservative.
\end{proof}

  \begin{center}
  SECTION 3.2
\end{center}

\medskip

\setcounter{chapter}{3}
\setcounter{section}{2}

\begin{exercise}[4]
  Let $A$ be a nonempty and bounded above so that $s = \sup(A)$ exists.
  \begin{enumerate}
    \item Show that $s \in \bar{A}$.
    \item Can an open set contain its supremum.
  \end{enumerate}
\end{exercise}

\begin{proof}[Solution to (i)]
  Since every $s - \epsilon$ has an $a \in A$ with $a > s - \epsilon$, we can
  find $a \in V_{\epsilon}(s)$, for any $\epsilon > 0$. This implies that $s$ is
  a limit point of $A$. Therefore, $s \in \bar{A}$.
\end{proof}

\begin{proof}[Solution to (ii)]
  No, there doesn't exist any $\epsilon$-neighborhood of $s$ such that
  $V_{\epsilon}(s) \not\subseteq (s, s + \epsilon)$.
\end{proof}

\medskip

\begin{exercise}[5]
  Prove Theorem 3.2.8.
\end{exercise}

\begin{proof}[Solution]
  Let $F \subseteq \R$ be closed. Given an arbitrary Cauchy sequence $(a_n)$
  contained in $F$, we know that $(a_n)$ converges to some $a \in \R$. Since $F$
  is closed and $(a_n)$ is contained in $F$, $a \in F$.

  Suppose that every Cauchy sequence contained in $F$ converges to a point, $l$,
  in $F$. Since $l$ is a limit point of $F$, there exists a sequence $(a_n)$
  contained in $F$ with $\lim_{n \to \infty} (a_n) = l$. Since $(a_n)$
  converges, it must be a Cauchy sequence (by the Cauchy Criterion). Since every
  Cauchy sequence converges to a limit inside $F$, we have $l \in F$, meaning
  that $F$ is closed.
\end{proof}

\medskip

\begin{exercise}[7]
  Given $A \subseteq \R$, let $L$ be the set of all limit points of $A$.
  \begin{enumerate}
    \item Show that the set $L$ is closed.
  \end{enumerate}
\end{exercise}

\begin{proof}[Solution to (i)]
  Let $L'$ be the set of all limit points of $L$. Assume $\ell \in L'$. Then, by
  definition, for every $\epsilon > 0$, we have
  \[%
    L_1 = V_{\epsilon}(\ell) \cap (L - \{\ell\}) \ne \emptyset \subseteq L
  .\]%
  Then,
  \[%
    (\forall y \in L_1)(\forall \epsilon > 0)[L_2 = V_{\delta}(y) \cap (A - \{y\}) \ne \emptyset]
  .\]%
  Now, since $\ell$ is a limit point of $L$, for any $\epsilon > 0$, there
  exists $y \in L - \{\ell\}$ such that $y \in V_{\epsilon}(l)$. By the
  definition of $y \in L$, there exists $z \in A$, distinct from $y$, such that
  $z \in V_{\delta}(y)$.

  Choose $0 < \delta < \epsilon$ so $V_{\delta}(x) \subseteq V_{\epsilon}(l)$.
  Since $z \in V_{\delta}(y) \cap (A - \{y\})$, then $z \in V_{\epsilon}(\ell)$
  and $z \in A$. So, $\ell$ is a limit point of $A$. Therefore, $\ell \in L$ and
  $L$ is closed.
\end{proof}

\medskip

\begin{exercise}[11]
  \begin{enumerate}
    \item Prove that $\overline{A \cup B} = \bar{A} \cup \bar{B}$.
    \item Does this result about closures extend to infinite unions of sets?
  \end{enumerate}
\end{exercise}

\begin{proof}[Solution to (i)]
  Let $L_a$ and $L_b$ be the set of limit points for $A$ and $B$ respectively.
  Let $\ell$ be a limit point for $A \cup B$. Then, $\ell \in L_a$ or $\ell \in
  L_b$. So, $\ell \in L_a \cup L_b$. Then, by definition $\overline{A \cup B} =
  (A \cup B) \cup (L_a \cup L_b) = (A \cup L_a) \cup (B \cup L_b) = \bar{A} \cup
  \bar{B}$.
\end{proof}

\begin{proof}[Solution to (ii)]
  No, the result does not extend to infinite unions of sets. Specifically,
  \[%
    \overline{\bigcup_{i \in I} A_i} \ne \bigcup_{i \in I} \bar{A_i}
  ,\]%
  where $\{A_i\}_{i \in I}$ is a collection of sets, indexed by $I$.

  For a finite union of sets $\overline{A \cup B} = \bar{A} \cup \bar{B}$ holds
  because any limit point of $A \cup B$ must lie in $\bar{A} \cup \bar{B}$, and
  the closure distributes over a finite union.

  For an infinite union, however, the equality may fail because the closure of
  an infinite union may include points that are limit points of the entire
  union, but not limit points of any single set in the collection.

  Let $A_n = \{\sfrac{1}{n} \mid n \in \N\}$ as a counterexample. Then,
  \[%
    \overline{\bigcup_{n=1}^{\infty} A_n} = \left.\left\{\frac{1}{n} \right\rvert n \in \N\right\} \cup \{0\} \ne \left.\left\{\frac{1}{n} \right\rvert n \in \N\right\} = \bigcup_{n=1}^{\infty} \bar{A}_n
  .\qedhere\]%
\end{proof}

\medskip

\begin{center}
  SECTION 3.3
\end{center}

\medskip

\setcounter{section}{3}

\begin{exercise}[1]
  Show that if $K$ is compact and nonempty, then $\sup(K)$ and $\inf(K)$ both
  exist and are elements of $K$.
\end{exercise}

\begin{proof}[Solution]
  By Theorem 3.3.4, $K$ is compact if and only if it is closed and bounded.
  Since $K \neq \emptyset$, it follows that $K$ is bounded above and bounded
  below. By the Axiom of Completeness, $\sup(K)$ exists because $K$ is bounded
  above, and $\inf(K)$ exists because $K$ is bounded below. To show $\sup(K)
  \in K$, note that $\sup(K)$ is a limit point of $K$ (by definition of the
  least upper bound). Since $K $ is closed, it contains all its limit points,
  so $\sup(K) \in K$. Similarly, $\inf(K)$ is a limit point of $K$, and by the
  closedness of $K$, $\inf(K) \in K$.
\end{proof}

\medskip

\begin{exercise}[2]
  Decide which of the following sets are compact. For those that are not
  compact, show how Definition 3.3.1 breaks down. In other words, give an
  example of a sequence contained in the given set that does not possess a
  subsequence converging to a limit in the set.
  \begin{enumerate}
    \item $\N$.

    \item $Q \cap [0, 1]$.

    \item $S = \{1, \sfrac{1}{2}, \sfrac{2}{3}, \sfrac{3}{4}, \sfrac{4}{5},
      \dots\}$.
  \end{enumerate}
\end{exercise}

\begin{proof}[Solution to (i)]
  No, since the sequence $x_n = n$ doesn't diverges.
\end{proof}

\begin{proof}[Solution to (ii)]
  No, since the sequence $x_n = \sfrac{1}{\sqrt{2}} + \frac{1}{n}$ converges to
  $\sfrac{1}{\sqrt{2}} \notin Q \cap [0, 1]$.
\end{proof}

\begin{proof}[Solution to (iii)]
  No, since the sequence $x_n = \sfrac{1}{\sqrt{2}} + \frac{1}{n}$ converges to
  $\sfrac{1}{\sqrt{2}} \notin Q \cap [0, 1]$.
\end{proof}

\begin{proof}[Solution to (iv)]
  Compact, since, it's bounded, closed, and every sequence converges to the
  limit value of $1 \in S$.
\end{proof}

\medskip

\begin{exercise}[3]
  Prove the converse of Theorem 3.3.4 by showing that if a set $K \subseteq \R$
  is closed and bounded, then it is compact.
\end{exercise}

\begin{proof}[Solution]\leavevmode
  \begin{enumerate}
    \item[$\implies$)] Suppose $K \subseteq \R$ is compact. Suppose $K$ isn't
      bounded. Let $(a_n)$ be a sequence in $K$. Since $K$ is unbounded, then
      given any $n \in \N$, we can produce $a_n \in K$ such that $\lvert a_n
      \rvert > n$. Now, since $K$ is compact, there must exist a subsequence
      $(a_{n_k})$ that converges to a limit $a \in K$. But the elements of
      $(a_{n_k})$ must satisfy $\lvert a_{n_k} \rvert > n_k$, which implies that
      $(a_{n_k})$ is unbounded, meaning it doesn't converge. This gives us a
      contradiction since every sequence must contain a converging subsequence
      to be called compact. This implies that $K$ must be bounded.

      Now that we know $K$ is bounded, then suppose we are given a sequence
      $(a_n)$ that's contained in $K$. By the BWT, there exists a converging
      subsequence $(a_{n_k})$ that converges to a limit $a$. By definition of
      compactness, $a \in K$. Suppose that $(a_n)$ converges to a limit point.
      Then, by Theorem 2.5.2, all subsequences converge to the same limit as the
      original sequence. So, $(a_n) \to x$ and $a \in K$, making $a$ a limit
      point. Therefore, $K$ is closed.

    \item[$\impliedby$)] Suppose $K \subseteq \R$ is closed and bounded. Let
      $(a_n)$ be a sequence contained in $K$. Since $K$ is bounded, by the BWT,
      there exists a converging subsequence $(a_{n_k})$ that converges to a
      limit $a$. Since $K$ is closed, $a \in K$. Therefore, $K$ is compact.
      \qedhere
  \end{enumerate}
\end{proof}

  % \begin{problem}[0.17]
  Let $A$ be a finite set, and let $|A| = s$. Based on the preceding exercise, make a conjecture about the value of $|\Pow(A)|$ Then try to prove your conjecture.
\end{problem}

\begin{solution}
\end{solution}

\begin{problem}[0.18]
  For any set $A$, finite or infinite, let $B^A$ be the set of all functions mapping $A$ into the set $B = \{0, 1\}$. Show that the cardinality of $B^A$ is the same as the cardinality of the set $\Pow(A)$.
  [Hint: Each element of $B^A$ determines a subset of $A$ in a natural way.]
\end{problem}

\begin{solution}
\end{solution}

\begin{problem}[0.19]
  Show that the power set of a set A, finite or infinite, has too many elements to be able to be put in a one-to-one correspondence with A. Explain why this intuitively means that there are an infinite number of infinite cardinal numbers.
  [Hint: Imagine a one-to-one function $\phi$ mapping A into $\Pow(A)$ to be given. Show that $\phi$ cannot be onto $\Pow(A)$ by considering, for each $x \in A$, whether $x \in \phi(x)$ and using this idea to define a subset $S$ of $A$ that is not in the range of $\phi$.]
  Is the set of everything a logically acceptable concept? Why or why not?
\end{problem}

\begin{solution}
\end{solution}

\begin{problem}[1.44]
  Let $S$ be a set and let $*$ be a binary operation on $S$ satisfying the two laws
  \begin{itemize}
    \item $x * x = x$ for all $x \in S$

    \item $(x * y) * z = (y * z) * x$ for all $x, y, z \in S$.
  \end{itemize}
  Show that $*$ is associative and commutative. (This is problem B-1 on the 1971 Putnam Competition.)
\end{problem}

\begin{solution}
\end{solution}

\begin{problem}[2.20]
  This exercise shows that there are two non-isomorphic group structures on a set of 4 elements.

  Let the set be $\{e, a, b, c\}$, with $e$ the identity element for the group operation. A group table would then have to start in the manner shown in Table 2.29. The square indicated by the question mark cannot be filled in with $a$. It must be filled in either with the identity element $e$ or with an element different from both $e$ and $a$. In this latter case, it is no loss of generality to assume that this element is $b$. If this square is filled in with $e$, the table can then be completed in two ways to give a group. Find these two tables. (You need not check the associative law.) If this square is filled in with $b$, then the table can only be completed in one way to give a group. Find this table. (Again, you need not check the associative law.) Of the three tables you now have, two give isomorphic groups. Determine which two tables these are, and give the one-to-one onto relabeling function which is an isomorphism.
  \begin{enumerate}
    \item Are all groups of 4 elements commutative?

    \item Find a way to relabel the four matrices

      \[%
        \left\{
          \begin{bmatrix}
            1 & 0 \\
            0 & 1 \\
          \end{bmatrix},
          \begin{bmatrix}
            0 & -1 \\
            1 & 0 \\
          \end{bmatrix},
          \begin{bmatrix}
            -1 & 0 \\
            0 & -1 \\
          \end{bmatrix},
          \begin{bmatrix}
            0 & 1 \\
            -1 & 0 \\
          \end{bmatrix}
        \right\}
      ,\]%
      so the matrix multiplication table is identical to one you constructed. This shows that the table you constructed defines an associative operation and therefore gives a group.

    \item Show that for a particular value of n , the group elements given in Exercise 14 can be relabeled so their group table is identical to one you constructed. This implies the operation in the table is also associative.
  \end{enumerate}
\end{problem}

\begin{solution}[(i)]
\end{solution}

\begin{solution}[(ii)]
\end{solution}

\begin{solution}[(iii)]
\end{solution}

\begin{problem}[2.38]
  Let $G$ be a group and let $a, b \in G$. Show that $(a * b)' = a' * b'$ if and only if $a * b = b * a$.
\end{problem}

\begin{solution}
\end{solution}

\begin{problem}[2.40]
  Prove that a set $G$, together with a binary operation $*$ on $G$ satisfying the left axioms 1, 2, and 3 given after Corollary 2.19, is a group.
\end{problem}

\begin{solution}
\end{solution}

\begin{problem}[2.41]
  Prove that a nonempty set $G$, together with an associative binary operation $*$ on $G$ such that
  \[%
    a * x = b~\text{and}~y * a = b~\text{have solutions in}~G~\text{for all}~a, b \in G
  ,\]%
  is a group. [Hint: Use Exercise 40.]
\end{problem}

\begin{solution}
\end{solution}

\begin{problem}[2.45]
  Suppose that $G$ is a group with $n$ elements and $A \subseteq G$ has more than $n/2$ elements. Prove that for every $g \in G$, there exists $a, b \in A$ such that $a * b = g$. (This was Problem B-2 on the 1968 Putnam exam.)
\end{problem}

\begin{solution}
\end{solution}

\begin{problem}[3.44]
  Prove that for any $n \in \Z^+$, $\bra{\Z_n, +_n}$ is associative without using the fact that $U_n$ is associative.
\end{problem}

\begin{solution}
\end{solution}

\begin{problem}[4.34]
  (See the warning following Theorem 4.8.) Let $G$ be a group with binary operation $*$. Let $G'$ be the same set as $G$, and define a binary operation $*'$ on $G'$ by $x *' y=y * x$ for all $x, y \in G'$.
  \begin{enumerate}
    \item (Intuitive argument that $G'$ under $*'$ is a group.) Suppose the front wall of your classroom were made of transparent glass, and that all possible products $a * b=c$ and all possible instances $a *(b * c)= (a * b) * c$ of the associative property for $G$ under $*$ were written on the wall with a magic marker. What would a person see when looking at the other side of the wall from the next room in front of yours?

    \item Show from the mathematical definition of $*'$ that $G'$ is a group under $*'$.
  \end{enumerate}
\end{problem}

\begin{solution}
\end{solution}



  % %%%%%%%%%%%%%%%%%%%%%%%%%%%%%%%%%%%%%%%%%%%%%%%%%%%%%%%%%%%%%%%%%%%%%%%%%%%%%
  % %                                Cheat Sheet                                %
  % %%%%%%%%%%%%%%%%%%%%%%%%%%%%%%%%%%%%%%%%%%%%%%%%%%%%%%%%%%%%%%%%%%%%%%%%%%%%%

  % \noindent\emph{Division Algorithm for $\Z$:} If $m$ is a positive integer and $n$ is any integer, then there exists unique integers $q$ and $r$ such that $n = mq + r$ and $0 \le r \le m$.

  % \noindent\emph{Theorem 6.15:} $G = \bra{a}$, where $|G| = n$ and let $H = \bra{a^s = b}$. Then $|H| = n/\gcd(n, s)$.

  % \noindent\emph{Cayley's Theorem:} Every group is isomorphic to a subgroup of a symmetric group.

  % \noindent\emph{Even and odd permutations:} A permutation is even if it can be written as a product of an even number of transpositions, and similarly for odd permutations.

  % \noindent\emph{Permutations:} A permutation is made from disjoint cycles. A 2 cycle is called a transposition.

  % \noindent\emph{Alternating Group:} The set of all even permutations on $n$ letters is called the alternating group on $n$ letters, denoted by $A_n$. Similarly with $B_n$.

  % \noindent\emph{Theorem 9.5:} $\Z_m \times \Z_n \cong \Z_{mn}$ and is cyclic iff $\gcd(m, n) = 1$.

  % \noindent\emph{Primary Factor Version of the Fundamental Theorem of Finite Abelian Groups:} ...

  % \noindent\emph{Invariant Factor Version of the Fundamental Theorem of Finite Abelian Groups:} ...

  % \noindent\emph{Lagrange's Theorem:} If $G$ is a finite group and $H$ is a subgroup of $G$, then $|H|$ divides $|G|$.

  % \noindent\emph{Corollary 10.8:} Every group of prime order is cyclic.
\end{document}
