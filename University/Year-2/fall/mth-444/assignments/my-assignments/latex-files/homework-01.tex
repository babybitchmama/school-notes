\begin{problem}[0.1]
  Describe the set by listing its elements: $\{x \in \R \mid x^2 = 3\}$.
\end{problem}

\begin{solution}
  The set can be described by listing its elements as $\{\sqrt{3}, -\sqrt{3}\}$.
\end{solution}

\begin{problem}[0.2]
  Describe the set by listing its elements: $\{m \in \Z \mid m^2 + m = 6\}$.
\end{problem}

\begin{solution}
  The equation $m^2 + m - 6 = 0$ can be factored as $(m - 2)(m + 3) = 0$. Therefore, the solutions are $m = 2$ and $m = -3$. Thus, the set can be described by listing its elements as $\{2, -3\}$.
\end{solution}

\begin{problem}[0.4]
  Describe the set by listing its elements: $\{x \in \Z \mid x^2 - 10x + 16 \le 0\}$.
\end{problem}

\begin{solution}
  The inequality $x^2 - 10x + 16 \le 0$ can be factored as $(x - 2)(x - 8) \le 0$. The solutions to the equation $(x - 2)(x - 8) = 0$ are $x = 2$ and $x = 8$. The inequality holds for values of $x$ between $2$ and $8$, inclusive. Therefore, the integer solutions are $x = 2, 3, 4, 5, 6, 7, 8$. Thus, the set can be described by listing its elements as $\{2, 3, 4, 5, 6, 7, 8\}$.
\end{solution}

\begin{problem}[0.15]
  Show that $S = \{x \in \R \mid 0 < x < 1\}$ has the same cardinality as $\R$. [Hint: Find an elementary function of calculus that maps an interval one-to-one onto $\R$, and then translate and scale appropriately to make the domain the set $S$.]
\end{problem}

\begin{solution}
  Take the function $f : S \to \R$, defined by $f(x) = \tan(\pi x - \pi/2)$. As $x$ approaches $0$ from the right, $f(x)$ approaches $-\infty$, and as $x$ approaches $1$ from the left, $f(x)$ approaches $\infty$. The function is continuous and strictly increasing on the interval $(0, 1)$. Therefore, $f$ is a bijection from $S$ to $\R$, showing that $S$ has the same cardinality as $\R$.
\end{solution}

\begin{problem}[0.24]
  Find the number of different partitions of a set having $2$ elements.
\end{problem}

\begin{solution}
  A set with 2 elements, say $\{a, b\}$, can be partitioned in the following ways:
  \begin{enumerate}
    \item $\{\{a\}, \{b\}\}$
    \item $\{\{a, b\}\}$
  \end{enumerate}

  Therefore, there are a total of 2 different partitions of a set having 2 elements.
\end{solution}

\begin{problem}[0.26]
  Find the number of different partitions of a set having $4$ elements.
\end{problem}

\begin{solution}
  A set with 4 elements, say $\{a, b, c, d\}$, can be partitioned in the following ways:
  \begin{enumerate}
    \item $\{\{a\}, \{b\}, \{c\}, \{d\}\}$
    \item $\{\{a, b\}, \{c\}, \{d\}\}$ (and all permutations of this form)
    \item $\{\{a, b, c\}, \{d\}\}$ (and all permutations of this form)
    \item $\{\{a, b\}, \{c, d\}\}$ (and all permutations of this form)
    \item $\{\{a, b, c, d\}\}$
  \end{enumerate}

  Counting all unique partitions, we find there are a total of 15 different partitions of a set having 4 elements.
\end{solution}

\begin{problem}[0.30]
  Determine whether $x \mathscr{R} y$ in $\R$ if $x \ge y$ is an equivalence relation on the set. Describe the partition arising from the relation.
\end{problem}

\begin{solution}
  The relation $x \mathscr{R} y$ defined by $x \ge y$ is not an equivalence relation because it fails the symmetry property. For example, if $x = 3$ and $y = 2$, then $3 \ge 2$ (so $3 \mathscr{R} 2$), but $2 \not\ge 3$.

  Since it is not an equivalence relation, there is no partition arising from this relation.
\end{solution}

\begin{problem}[0.32]
  Determine whether $(x_1, y_1) \mathscr{R} (x_2, y_2)$ in $\R \times \R$ if $x_1^2 + y_1^2 = x_2^2 + y_2^2$ is an equivalence relation on the set. Describe the partition arising from the relation.
\end{problem}

\begin{solution}
  The relation $(x_1, y_1) \mathscr{R} (x_2, y_2)$ defined by $x_1^2 + y_1^2 = x_2^2 + y_2^2$ is an equivalence relation because it satisfies the following properties:
  \begin{enumerate}
    \item Reflexivity: For any point $(x, y)$, we have $x^2 + y^2 = x^2 + y^2$, so $(x, y) \mathscr{R} (x, y)$.
    \item Symmetry: If $(x_1, y_1) \mathscr{R} (x_2, y_2)$, then $x_1^2 + y_1^2 = x_2^2 + y_2^2$. This implies that $x_2^2 + y_2^2 = x_1^2 + y_1^2$, so $(x_2, y_2) \mathscr{R} (x_1, y_1)$.
    \item Transitivity: If $(x_1, y_1) \mathscr{R} (x_2, y_2)$ and $(x_2, y_2) \mathscr{R} (x_3, y_3)$, then $x_1^2 + y_1^2 = x_2^2 + y_2^2$ and $x_2^2 + y_2^2 = x_3^2 + y_3^2$. Therefore, $x_1^2 + y_1^2 = x_3^2 + y_3^2$, so $(x_1, y_1) \mathscr{R} (x_3, y_3)$.
  \end{enumerate}

  The partition arising from this equivalence relation consists of sets of points in $\R \times \R$ that lie on circles centered at the origin with radius $\sqrt{r}$ for each non-negative real number $r$. Each equivalence class corresponds to a circle defined by the equation $x^2 + y^2 = \sqrt{r}$ for some fixed $r \geq 0$.
\end{solution}

\begin{problem}[0.34]
  Determine whether $n \mathscr{R} m$ in $\Z^+$ if $n$ and $m$ have the same final digit in the usual base ten notation is an equivalence relation on the set. Describe the partition arising from the relation.
\end{problem}

\begin{solution}
  The relation $n \mathscr{R} m$ defined by $n$ and $m$ having the same final digit in base ten is an equivalence relation because it satisfies the following properties:
  \begin{enumerate}
    \item Reflexivity: For any positive integer $n$, the final digit of $n$ is the same as itself, so $n \mathscr{R} n$.
    \item Symmetry: If $n \mathscr{R} m$, then $n$ and $m$ have the same final digit. This implies that $m$ and $n$ also have the same final digit, so $m \mathscr{R} n$.
    \item Transitivity: If $n \mathscr{R} m$ and $m \mathscr{R} p$, then $n$ and $m$ have the same final digit, and $m$ and $p$ have the same final digit. Therefore, $n$ and $p$ must also have the same final digit, so $n \mathscr{R} p$.
  \end{enumerate}

  The partition arising from this equivalence relation consists of sets of positive integers that share the same final digit. There are ten equivalence classes corresponding to the final digits 0 through 9:
  \begin{enumerate}
    \item Class for final digit 0: $\{0, 10, 20, 30, \cdots\}$.
    \item Class for final digit 1: $\{1, 11, 21, 31, \cdots\}$.
    \item Class for final digit 2: $\{2, 12, 22, 32, \cdots\}$.
    \item Class for final digit 3: $\{3, 13, 23, 33, \cdots\}$.
    \item Class for final digit 4: $\{4, 14, 24, 34, \cdots\}$.
    \item Class for final digit 5: $\{5, 15, 25, 35, \cdots\}$.
    \item Class for final digit 6: $\{6, 16, 26, 36, \cdots\}$.
    \item Class for final digit 7: $\{7, 17, 27, 37, \cdots\}$.
    \item Class for final digit 8: $\{8, 18, 28, 38, \cdots\}$.
    \item Class for final digit 9: $\{9, 19, 29, 39, \cdots\}$. \qedhere
  \end{enumerate}
\end{solution}

\begin{problem}[1.7]
  Determine whether the operation *, defined on $\Z$ by letting $a * b = a - b$, is associative, whether the operation is commutative, and whether the set has an identity element.
\end{problem}

\begin{solution}
  The operation is not associative since $(3 * 2) * 1 = 1 * 1 = 0$, while $3 * (2 * 1) = 3 * 1 = 2$.

  The operation is also not commutative since $3 * 2 = 3 - 2 = 1$, while $2 * 3 = 2 - 3 = -1$.

  Finally, the set does not have an identity element for this operation. An identity element $e$ would need to satisfy $a * e = a$ for all $a \in \Z$. This implies $a - e = a \implies e = 0$. However, checking with $e = 0$, we have $a * 0 = a - 0 = a$, which holds true, but we also need to check if $0 * a = -a$, which does not equal $a$ unless $a = 0$. Therefore, there is no identity element that works for all integers.
\end{solution}

\begin{problem}[1.8]
  Determine whether the operation *, defined on $\Q$ by letting $a * b = 2ab + 3$, is associative, whether the operation is commutative, and whether the set has an identity element.
\end{problem}

\begin{solution}
  The operation is not associative since $(a*b)*c = (2ab + 3)*c = 2(2ab + 3)c + 3 = 4abc + 6c + 3$, while $a*(b*c) = a*(2bc + 3) = 2a(2bc + 3) + 3 = 4abc + 6a + 3$.

  The operation is commutative since $a * b = 2ab + 3 = b * a$.

  To find the identity element $e$, we need $a * e = a$ for all $a \in \Q$. This gives $a * e = 2ae + 3 = a$. Rearranging, we get $2ae - a + 3 = 0 \implies a(2e - 1) + 3 = 0$. For this to hold for all $a$, we must have $2e - 1 = 0$ and $3 = 0$, which is impossible. Therefore, there is no identity element in $\Q$ for this operation.
\end{solution}

\begin{problem}[1.10]
  Determine whether the operation *, defined on $\Z^+$ by letting $a * b = 2^{ab}$, is associative, whether the operation is commutative, and whether the set has an identity element.
\end{problem}

\begin{solution}
  The operation is not associative since $(a * b) * c = 2^{(2^{ab})c} = 2^{c \cdot 2^{ab}}$, while $a * (b * c) = 2^{a(2^{bc})} = 2^{a \cdot 2^{bc}}$.

  The operation is commutative since $a * b = 2^{ab} = 2^{ba} = b * a$.

  If an identity $e$ existed, we would need $a * e = a$ for all $a$, i.e. $2^{ae} = a$. Taking $\log_2$ gives $ae = \log_2(a)$, which has no single solution $e$ valid for all $a$. Therefore, no identity element exists.
\end{solution}
