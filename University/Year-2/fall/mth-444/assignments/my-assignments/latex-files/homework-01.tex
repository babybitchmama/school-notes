\begin{problem}[0.1]
  Describe the set by listing its elements: $\{x \in \R \mid x^2 = 3\}$.
\end{problem}

\begin{solution}
  The set can be described by listing its elements as $\{\sqrt{3}, -\sqrt{3}\}$.
\end{solution}

\begin{problem}[0.2]
  Describe the set by listing its elements: $\{m \in \Z \mid m^2 + m = 6\}$.
\end{problem}

\begin{solution}
  The equation $m^2 + m - 6 = 0$ can be factored as $(m - 2)(m + 3) = 0$. Therefore, the solutions are $m = 2$ and $m = -3$. Thus, the set can be described by listing its elements as $\{2, -3\}$.
\end{solution}

\begin{problem}[0.4]
  Describe the set by listing its elements: $\{x \in \Z \mid x^2 - 10x + 16 \le 0\}$.
\end{problem}

\begin{solution}
  The inequality $x^2 - 10x + 16 \le 0$ can be factored as $(x - 2)(x - 8) \le 0$. The solutions to the equation $(x - 2)(x - 8) = 0$ are $x = 2$ and $x = 8$. The inequality holds for values of $x$ between $2$ and $8$, inclusive. Therefore, the integer solutions are $x = 2, 3, 4, 5, 6, 7, 8$. Thus, the set can be described by listing its elements as $\{2, 3, 4, 5, 6, 7, 8\}$.
\end{solution}

\begin{problem}[0.15]
  Show that $S = \{x \in \R \mid 0 < x < 1\}$ has the same cardinality as $\R$. [Hint: Find an elementary function of calculus that maps an interval one-to-one onto $\R$, and then translate and scale appropriately to make the domain the set $S$.]
\end{problem}

\begin{solution}
  Take the function $f : S \to \R$, defined by $f(x) = \tan(\pi x - \pi/2)$. As $x$ approaches $0$ from the right, $f(x)$ approaches $-\infty$, and as $x$ approaches $1$ from the left, $f(x)$ approaches $\infty$. The function is continuous and strictly increasing on the interval $(0, 1)$. Therefore, $f$ is a bijection from $S$ to $\R$, showing that $S$ has the same cardinality as $\R$.
\end{solution}

\begin{problem}[0.24]
  Find the number of different partitions of a set having $2$ elements.
\end{problem}

\begin{solution}
  A set with 2 elements, say $\{a, b\}$, can be partitioned in the following ways:
  \begin{enumerate}
    \item $\{\{a\}, \{b\}\}$
    \item $\{\{a, b\}\}$
  \end{enumerate}

  Therefore, there are a total of 2 different partitions of a set having 2 elements.
\end{solution}

\begin{problem}[0.26]
  Find the number of different partitions of a set having $4$ elements.
\end{problem}

\begin{solution}
  A set with 4 elements, say $\{a, b, c, d\}$, can be partitioned in the following ways:
  \begin{enumerate}
    \item $\{\{a\}, \{b\}, \{c\}, \{d\}\}$
    \item $\{\{a, b\}, \{c\}, \{d\}\}$ (and all permutations of this form)
    \item $\{\{a, b, c\}, \{d\}\}$ (and all permutations of this form)
    \item $\{\{a, b\}, \{c, d\}\}$ (and all permutations of this form)
    \item $\{\{a, b, c, d\}\}$
  \end{enumerate}

  Counting all unique partitions, we find there are a total of 15 different partitions of a set having 4 elements.
\end{solution}

\begin{problem}[0.30]
  Determine whether $x \mathscr{R} y$ in $\R$ if $x \ge y$ is an equivalence relation on the set. Describe the partition arising from each equivalence relation.
\end{problem}

\begin{solution}
\end{solution}

\begin{problem}[0.32]
  Determine whether $(x_1, y_1) \mathscr{R} (x_2, y_2)$ in $\R \times \R$ if $x_1^2 + y_1^2 = x_2^2 + y_2^2$ is an equivalence relation on the set. Describe the partition arising from each equivalence relation.
\end{problem}

\begin{solution}
\end{solution}

\begin{problem}[0.34]
  Determine whether $n \mathscr{R} m$ in $\Z$ if $x_1^2 + y_1^2 = x_2^2 + y_2^2$ is an equivalence relation on the set. Describe the partition arising from each equivalence relation.
\end{problem}

\begin{solution}
\end{solution}

\begin{problem}[1.7]
  Determine whether the operation *, defined on $\Z$ by letting $a * b = a - b$, is associative, whether the operation is commutative, and whether the set has an identity element.
\end{problem}

\begin{solution}
\end{solution}

\begin{problem}[1.8]
  Determine whether the operation *, defined on $\Q$ by letting $a * b = 2ab + 3$, is associative, whether the operation is commutative, and whether the set has an identity element.
\end{problem}

\begin{solution}
\end{solution}

\begin{problem}[1.10]
  Determine whether the operation *, defined on $\Z^+$ by letting $a * b = 2^{ab}$, is associative, whether the operation is commutative, and whether the set has an identity element.
\end{problem}

\begin{solution}
\end{solution}
