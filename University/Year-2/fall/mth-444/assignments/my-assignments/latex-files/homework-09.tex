\begin{problem}[12.32]
  Let $H$ be a normal subgroup of a group $G$, and let $m = (G : H)$. Show that $a^m \in H$ for every $a \in G$.
\end{problem}

\begin{solution}
  Let $a \in G$. Since $H$ is normal in $G$, the left cosets of $H$ in $G$ are the same as the right cosets. The index $m = (G : H)$ represents the number of distinct cosets of $H$ in $G$. Therefore, the cosets can be represented as $H, aH, a^2H, \ldots, a^{m-1}H$. Since there are $m$ distinct cosets, we have $a^mH = H$. This implies that $a^m \in H$. Thus, for every $a \in G$, we have $a^m \in H$.
\end{solution}

\begin{problem}[12.37]
  Show that if $H$ and $N$ are subgroups of a group $G$, and $N$ is normal in $G$, then $H \cap N$ is normal in $H$. Show by an example that $H \cap N$ need not be normal in $G$.
\end{problem}

\begin{solution}
  Let $H$ and $N$ be subgroups of a group $G$, and let $N$ be normal in $G$. We need to show that $H \cap N$ is normal in $H$. Take any $h \in H$ and any $x \in H \cap N$. Since $x \in N$ and $N$ is normal in $G$, we have
  \[%
    h x h^{-1} \in N
  .\]%
  Additionally, since $h, x \in H$ and $H$ is a subgroup, we have
  \[%
    h x h^{-1} \in H
  .\]%
  Therefore, we have $h x h^{-1} \in H \cap N$. Thus, $H \cap N$ is normal in $H$.

  For an example where $H \cap N$ is not normal in $G$, consider the group $G = S_3$, the symmetric group on 3 elements. Let $H = \bra{(1), (12)}$ and let $N = A_3 = \bra{(1), (123), (132)}$. Here, $N$ is normal in $G$, but the intersection $H \cap N = \bra{(1)}$ is not normal in $G$, since conjugating $(12)$ by $(123)$ gives $(13)$, which is not in $H$. Thus, this example shows that $H \cap N$ need not be normal in $G$.
\end{solution}

\begin{problem}[12.39]
  \begin{enumerate}
    \item Show that all automorphisms of a group $G$ form a group under function composition.

    \item Show that the inner automorphisms of a group $G$ form a normal subgroup of the group of all automorphisms of $G$ under function composition. [Warning: Be sure to show that the inner automorphisms do form a subgroup.]
  \end{enumerate}
\end{problem}

\begin{solution}[(i)]
  The identity automorphism $\iota_G$ defined by $\iota_G(a) = a$ for all $a \in G$ is in $\Aut(G)$, serving as the identity element. For any $\phi \in \Aut(G)$, its inverse $\phi^{-1}$ is also an automorphism since it is a bijection and satisfies the homomorphism property. Therefore, every element in $\Aut(G)$ has an inverse in $\Aut(G)$.

  Next, we show closure. Let $\Aut(G)$ be the collection of all automorphisms of $G$. Take $\phi, \psi \in \Aut(G)$. We need to show that the composition $\phi \circ \psi \in \Aut(G)$. By function composition, we have
  \[%
    (\phi \circ \psi)(ab) = \phi(\psi(ab)) = \phi(\psi(a)\psi(b)) = \phi(\psi(a))\phi(\psi(b)) = (\phi \circ \psi)(a)(\phi \circ \psi)(b)
  .\]%
  Thus, $\phi \circ \psi$ is a homomorphism. Since both $\phi$ and $\psi$ are bijections, their composition is also a bijection. Therefore, $\phi \circ \psi$ is an automorphism of $G$. Thus, $\Aut(G)$ is closed under function composition.

  Next, we show that it contains inverses. Since $\phi \in \Aut(G)$ is an automorphism (an isomorphism from $G$ to $G$), there exists an inverse, $\phi^{-1}$ that's also an automorphism. From this, we get
  \[%
    (\phi \circ \phi^{-1})(x) = \phi(\phi^{-1}(x)) = \iota_G(x)
  .\]%
  Therefore, $\Aut(G)$ contains all its inverses.

  Lastly, we show associativity. Given $\phi, \psi, \rho \in \Aut(G)$, we have
  \[%
    (\phi \circ \psi) \circ \rho = \phi \circ (\psi \circ \rho)
  .\]%
  Therefore, $\Aut(G)$ is associative.

  Thus, $\Aut(G)$ is a group under function composition.
\end{solution}

\begin{solution}[(ii)]
  Clearly, $\Inn(G)$ is non-empty, since $\iota_e(x) = exe^{-1} = e$ for all $x \in G$, where $e$ is the identity element of $G$. Thus, $\iota_e \in \Inn(G)$.

  Next, we show that $\Inn(G) \subseteq \Aut(G)$ is closed. Take $\iota_a, \iota_b \in \Inn(G)$. Then, we have
  \[%
    (\iota_a \circ \iota_b)(x) = \iota_a(\iota_b(x)) = \iota_a(bxb^{-1}) = \iota_a(b) \iota_a(x) \iota_a(b^{-1}) = aba^{-1} x a b^{-1} a^{-1} = \iota_{ab}(x)
  .\]%
  Therefore, $\iota_a \circ \iota_b \in \Inn(G)$, showing closure.

  Lastly, we show that $\Inn(G)$ contains inverses. Take $\iota_a \in \Inn(G)$. Then, we have
  \[%
    \iota_a^{-1}(x) = a^{-1} x a = \iota_{a^{-1}}(x)
  .\]%
  Therefore, $\iota_a^{-1} \in \Inn(G)$, showing that $\Inn(G)$ contains inverses.

  Thus, $\Inn(G)$ is a subgroup of $\Aut(G)$.

  Lastly, we show that $\Inn(G)$ is normal in $\Aut(G)$. Take $\phi \in \Aut(G)$ and $\iota_a \in \Inn(G)$. Then, we have
  \[%
    (\phi \circ \iota_a \circ \phi^{-1})(x) = \phi(\iota_a(\phi^{-1}(x))) = \phi(a \phi^{-1}(x) a^{-1}) = \phi(a) x \phi(a)^{-1} = \iota_{\phi(a)}(x)
  .\]%
  Therefore, $\phi \circ \iota_a \circ \phi^{-1} \in \Inn(G)$. Thus, $\Inn(G)$ is normal in $\Aut(G)$.
\end{solution}

\begin{problem}[13.12]
  Classify the group $(\Z \times \Z \times \Z) / \bra{(3, 3, 3)}$ according to the fundamental theorem of finitely generated abelian groups.
\end{problem}

\begin{solution}
  Here, we have the group $G = \Z \times \Z \times \Z$ and the subgroup $N = \bra{(3, 3, 3)}$. The subgroup $N$ is generated by the element $(3, 3, 3)$, which can be expressed as $N = \{(3k, 3k, 3k) \mid k \in \Z\}$. Take the homomorphism $\phi : \Z \times \Z \times \Z \to \Z \times \Z \times \Z_3$ defined by
  \[%
    \phi(a, b, c) = (a - c, b - c, c \mod 3)
  .\]%
  Clearly, $\phi$ is a surjective homomorphism. The kernel of $\phi$ is given by
  \[%
    \ker(\phi) = \{(a, b, c) \in \Z \times \Z \times \Z \mid a - c = 0, b - c = 0, c \equiv 0 \mod 3\} = \{(3k, 3k, 3k) \mid k \in \Z\} = N
  .\]%
  By the First Isomorphism Theorem, we have
  \[%
    (\Z \times \Z \times \Z) / \bra{(3, 3, 3)} \cong \Z \times \Z \times \Z_3
  .\qedhere\]%
\end{solution}

\begin{problem}[13.14]
  Classify the group $(\Z \times \Z \times \Z_2) / \bra{(1, 1, 1)}$ according to the fundamental theorem of finitely generated abelian groups.
\end{problem}

\begin{solution}
  Here, we have the group $G = \Z \times \Z \times \Z_2$ and the subgroup $N = \bra{(1, 1, 1)}$. The subgroup $N$ is generated by the element $(1, 1, 1)$, which can be expressed as $N = \{(k, k, k \mod 2) \mid k \in \Z\}$. Take the homomorphism $\phi : \Z \times \Z \times \Z_2 \to \Z \times \Z_2$ defined by
  \[%
    \phi(a, b, c) = (a - b, (b - c) \mod 2)
  .\]%
  Clearly, $\phi$ is a surjective homomorphism. The kernel of $\phi$ is given by
  \[%
    \ker(\phi) = \{(a, b, c) \in \Z \times \Z \times \Z_2 \mid a - b = 0, b - c \equiv 0 \mod 2\} = \{(k, k, k \mod 2) \mid k \in \Z\} = N
  .\]%
  By the First Isomorphism Theorem, we have
  \[%
    (\Z \times \Z \times \Z_2) / \bra{(1, 1, 1)} \cong \Z \times \Z_2
  .\qedhere\]%
\end{solution}

\begin{problem}[13.16]
  Find both the center and the commutator subgroup of $\Z_3 \times S_3$.
\end{problem}

\begin{solution}
  The center of the group $\Z_3 \times S_3$ is given by
  \[%
    Z(\Z_3 \times S_3) = Z(\Z_3) \times Z(S_3) = \Z_3 \times \{\iota\} \cong \Z_3
  .\]%
  The commutator of the group $\Z_3 \times S_3$ is given by
  \[%
    (\Z_3 \times S_3)' = \Z_3' \times S_3' = \{e\} \times A_3 \cong A_3
  .\qedhere\]%
\end{solution}

\begin{problem}[13.17]
  Find both the center and the commutator subgroup of $S_3 \times D_4$.
\end{problem}

\begin{solution}
  The center of the group $S_3 \times D_4$ is given by
  \[%
    Z(S_3 \times D_4) = Z(S_3) \times Z(D_4) = \{\iota\} \times \{e, r^2\} \cong \Z_2
  .\]%
  The commutator of the group $S_3 \times D_4$ is given by
  \[%
    (S_3 \times D_4)' = S_3' \times D_4' = A_3 \times \{e, r^2\} \cong \Z_3 \times \Z_2
  .\qedhere\]%
\end{solution}

\begin{problem}[13.37]
  Let $\phi : G \to G'$ be a group homomorphism, and let $N$ be a normal subgroup of $G$. Show that $\phi[N]$ is a normal subgroup of $\phi[G]$.
\end{problem}

\begin{solution}
  Let $\phi : G \to G'$ be a group homomorphism, and let $N$ be a normal subgroup of $G$. First, we show that $\phi[N]$ is a subgroup of $\phi[G]$. Since $N$ is a subgroup of $G$, for any $n_1, n_2 \in N$, we have $n_1 n_2^{-1} \in N$. Applying the homomorphism $\phi$, we get
  \[%
    \phi(n_1 n_2^{-1}) = \phi(n_1) \phi(n_2)^{-1} \in \phi[N]
  .\]%
  Thus, $\phi[N]$ is closed under the group operation and contains inverses, making it a subgroup of $\phi[G]$.

  Next, we show that $\phi[N]$ is normal in $\phi[G]$. Take any $g' \in \phi[G]$ and $n' \in \phi[N]$. There exist $g \in G$ and $n \in N$ such that $\phi(g) = g'$ and $\phi(n) = n'$. Since $N$ is normal in $G$, we have $g n g^{-1} \in N$. Applying the homomorphism $\phi$, we get
  \[%
    \phi(g n g^{-1}) = \phi(g) \phi(n) \phi(g)^{-1} = g' n' (g')^{-1} \in \phi[N]
  .\]%
  Thus, $\phi[N]$ is normal in $\phi[G]$.
\end{solution}

\begin{problem}[13.38]
  Let $\phi : G \to G'$ be a group homomorphism, and let $N'$ be a normal subgroup of $G'$. Show that $\phi^{-1}[N']$ is a normal subgroup of $G$.
\end{problem}

\begin{solution}
  Let $\phi : G \to G'$ be a group homomorphism, and let $N'$ be a normal subgroup of $G'$. First, we show that $\phi^{-1}[N']$ is a subgroup of $G$. For any $a, b \in \phi^{-1}[N']$, we have $\phi(a), \phi(b) \in N'$. Since $N'$ is a subgroup of $G'$, we have $\phi(a) \phi(b)^{-1} \in N'$. Applying the inverse homomorphism, we get
  \[%
    \phi^{-1}(\phi(a) \phi(b)^{-1}) = ab^{-1} \in \phi^{-1}[N']
  .\]%
  Thus, $\phi^{-1}[N']$ is closed under the group operation and contains inverses, making it a subgroup of $G$.

  Next, we show that $\phi^{-1}[N']$ is normal in $G$. Take any $g \in G$ and $n \in \phi^{-1}[N']$. There exists $n' \in N'$ such that $\phi(n) = n'$. Since $N'$ is normal in $G'$, we have $\phi(g) n' \phi(g)^{-1} \in N'$. Applying the inverse homomorphism, we get
  \[%
    \phi^{-1}(\phi(g) n' \phi(g)^{-1}) = g n g^{-1} \in \phi^{-1}[N']
  .\]%
  Thus, $\phi^{-1}[N']$ is normal in $G$.
\end{solution}

\begin{problem}[13.39]
  Show that if $G$ is nonabelian, then the factor group $G/Z(G)$ is not cyclic. [Hint: Show the equivalent contrapositive, namely, that if $G/Z(G)$ is cyclic then $G$ is abelian (and hence $Z(G) = G$).]
\end{problem}

\begin{solution}
  Assume $G/Z(G)$ is cyclic. Then, there exists an element $gZ(G) \in G/Z(G)$ such that every element of $G/Z(G)$ can be written as $(gZ(G))^n$ for some integer $n$. This means that for any $a \in G$, there exists an integer $n$ such that
  \[%
    aZ(G) = (gZ(G))^n = g^nZ(G)
  .\]%
  Therefore, we can express $a$ as $a = g^nz$, for some $z \in Z(G)$. Now, take any two elements $a, b \in G$. We can write them as $a = g^nz_1$ and $b = g^mz_2$ for some integers $n, m$ and $z_1, z_2 \in Z(G)$. Then, we have
  \[%
    ab = (g^nz_1)(g^mz_2) = g^{n+m}z_1z_2
  .\]%
  Similarly, we have
  \[%
    ba = (g^mz_2)(g^nz_1) = g^{m+n}z_2z_1
  .\]%
  Since $z_1, z_2 \in Z(G)$, they commute with all elements of $G$, including each other. Thus, we have $z_1z_2 = z_2z_1$. Therefore, we get
  \[%
    ab = g^{n+m}z_1z_2 = g^{m+n}z_2z_1 = ba
  .\]%
  Hence, $G$ is abelian. Thus, if $G/Z(G)$ is cyclic, then $G$ is abelian. The contrapositive statement is that if $G$ is nonabelian, then $G/Z(G)$ is not cyclic.
\end{solution}
