\begin{problem}[5.30]
  Find the order of the cyclic subgroup $\Z_{16}$ generated by 12.
\end{problem}

\begin{solution}
  Notice that the cyclic subgroup of $\Z_{16}$ generated by 12 is given by
  \[%
    \bra{12} = \{12 \cdot n \pmod{16} \mid n \in \Z\} = \{0, 4, 8, 12\}
  .\]%
  Thus, the order of the cyclic subgroup is 4.
\end{solution}

\begin{problem}[5.32]
  Find the order of the cyclic subgroup $S_8$ generated by $(2, 4, 6, 9)(3, 5, 7)$.
\end{problem}

\begin{solution}
  Notice that the permutation $(2, 4, 6, 9)(3, 5, 7)$ is the product of two disjoint cycles: a 4-cycle $(2, 4, 6, 9)$ and a 3-cycle $(3, 5, 7)$. The order of a permutation is the least common multiple (LCM) of the lengths of its disjoint cycles. Therefore, the order of the permutation is
  \[%
    \text{lcm}(4, 3) = 12
  .\]%
  Thus, the order of the cyclic subgroup generated by $(2, 4, 6, 9)(3, 5, 7)$ is 12.
\end{solution}

\begin{problem}[5.40]
  Show by means of an example that it is possible for the quadratic equation $x^2 = e$ have more than two solutions in some group with identity $e$.
\end{problem}

\begin{solution}
  Take the algebraic group $\bra{V_4, *}$, where $V_4$ is the Klein four-group defined as $V_4 = \{e, a, b, c\}$ with the operation $*$ defined by the following Cayley table:
  \[%
    \begin{array}{c|cccc}
      * & e & a & b & c \\
      \hline
      e & e & a & b & c \\
      a & a & e & c & b \\
      b & b & c & e & a \\
      c & c & b & a & e \\
    \end{array}
  \]%
  In this group, we can see that:
  \[%
    a * a = e, \quad b * b = e, \quad c * c = e
  .\]%
  Thus, the equation $x^2 = e$ has four solutions: $e$, $a$, $b$, and $c$. This shows that it is possible for the quadratic equation $x^2 = e$ to have more than two solutions in a group.
\end{solution}

\begin{problem}[5.41]
  Let $B$ be a subset of $A$, and let $b$ be a particular element of $B$. Determine whether the subset, $\{\sigma \in S_A \mid \sigma(b) = b\}$, of the symmetric group $S_A$ is a subgroup of $S_A$ under the induced operation.
\end{problem}

\begin{solution}
  Define the set $B_S = \{\sigma \in S_A \mid \sigma(b) = b\}$. It's clearly non-empty, since the identity permutation $e$ is in $B_S$, as $e(b) = b$. Notice that, for any two permutations $\sigma_1, \sigma_2 \in B_S$, we have
  \[%
    (\sigma_1 \circ \sigma_2)(b) = \sigma_1(\sigma_2(b)) = \sigma_1(b) = b
  .\]%
  Thus, the composition $\sigma_1 \circ \sigma_2$ is also in $B_S$, showing closure under the group operation. Finally, if $\sigma \in B_S$, then $\sigma \in S_A$. Since $S_A$ is a group, the inverse permutation $\sigma^{-1}$ also exists in $S_A$. Moreover, $\sigma^{-1}(b) = b$. Thus, $\sigma^{-1} \in B_S$. Since $B_S$ is closed under the group operation, contains the identity element, and contains inverses, we conclude that $B_S$ is a subgroup of $S_A$ under the induced operation.
\end{solution}

\begin{problem}[5.42]
  Let $B$ be a subset of $A$, and let $b$ be a particular element of $B$. Determine whether the subset, $\{\sigma \in S_A \mid \sigma(b) = B\}$, of the symmetric group $S_A$ is a subgroup of $S_A$ under the induced operation.
\end{problem}

\begin{solution}
  Define the set $B_S = \{\sigma \in S_A \mid \sigma(b) = B\}$. Again, it's clearly non-empty, since the identity permutation $e$ is in $B_S$, as $e(b) = b \in B$. However, consider two permutations $\sigma_1, \sigma_2 \in B_S$. We have
  \[%
    (\sigma_1 \circ \sigma_2)(b) = \sigma_1(\sigma_2(b)) = \sigma_1(B)
  ,\]%
  which may not equal $B$ unless $\sigma_1$ maps all elements of $B$ back to $B$. Thus, the composition $\sigma_1 \circ \sigma_2$ may not be in $B_S$, showing that $B_S$ is not closed under the group operation. Therefore, we conclude that $B_S$ is not a subgroup of $S_A$ under the induced operation.
\end{solution}

\begin{problem}[5.45]
  Let $\Phi : G \to G'$ be an isomorphism of a group $\bra{G, *}$ with a group $\bra{G', *'}$. Prove that if $H$ is a subgroup of $G$, then $\Phi[H] = \{\Phi(h) \mid h \in H\}$ is a subgroup of $G'$. That is, an isomorphism carries subgroups into subgroups
\end{problem}

\begin{solution}
  Let $H$ be a subgroup of $G$. Since $H$ is a subgroup of $G$, it contains the identity element $e$. Therefore, $\Phi(e_G) = e'$ is in $\Phi[H]$, so $\Phi[H]$ is non-empty.

  Let $\Phi(h_1), \Phi(h_2) \in \Phi[H]$ for some $h_1, h_2 \in H$. Since $\Phi$ is an isomorphism, we have
  \[%
    \Phi(h_1 * h_2) = \Phi(h_1) *' \Phi(h_2)
  .\]%
  We have $h_1 * h_2 \in H$ because $H$ is a subgroup of $G$. Thus, $\Phi(h_1 * h_2) \in \Phi[H]$. Therefore, $\Phi[H]$ is closed under the operation $*'$.

  Let $\Phi(h) \in \Phi[H]$ for some $h \in H$. Since $\Phi$ is an isomorphism, we have
  \[%
    \Phi(h^{-1}) = (\Phi(h))^{-1}
  .\]%
  Since $H$ is a subgroup, $h^{-1} \in H$. Thus, $\Phi(h^{-1}) \in \Phi[H]$. Therefore, $\Phi[H]$ contains inverses.

  Hence, $\Phi[H]$ is non-empty, closed under the operation $*'$, and contains inverses. Thus, $\Phi[H]$ is a subgroup of $G'$.
\end{solution}

\begin{problem}[5.46]
  Let $\Phi : G \to G'$ be an isomorphism of a group $\bra{G, *}$ with a group $\bra{G', *'}$. Prove that if there is an $a \in G$ such that $\bra{a} = G$, then $G'$ is cyclic.
\end{problem}

\begin{solution}
  Since $\bra{a} = G$, every element $g \in G$ can be expressed as $g = a^n$ for some integer $n$. Consider the element $\Phi(a) \in G'$. We will show that $\bra{\Phi(a)} = G'$.

  Let $g' \in G'$. Since $\Phi$ is an isomorphism, there exists a unique $g \in G$ such that $\Phi(g) = g'$. Since $g \in G$, we can write $g = a^n$ for some integer $n$. Therefore,
  \[%
    g' = \Phi(g) = \Phi(a^n) = (\Phi(a))^n
  .\]%
  This shows that every element $g' \in G'$ can be expressed as a power of $\Phi(a)$.

  Thus, $\bra{\Phi(a)} = G'$, and hence $G'$ is cyclic.
\end{solution}

\begin{problem}[5.48]
  Find an example of a group $G$ and two subgroups $H$ and $K$ such that the set in Exercise 47 is not a subgroup of $G$.
\end{problem}

\begin{solution}
  Let $G = S_3$, the symmetric group on $\{1, 2, 3\}$. Define two subgroups
  \[%
    H = \langle (1\ 2) \rangle = \{e, (1\ 2)\}~\text{and}~ K = \langle (1\ 3) \rangle = \{e, (1\ 3)\}
  .\]%
  Both $H$ and $K$ are subgroups of $S_3$ of order $2$.

  Consider the set $HK = \{hk \mid h \in H,\, k \in K\}$. Its elements consist of $\{e, (1\ 2), (1\ 3), (1\ 3\ 2)\}$.

  This set has $4$ elements. By Lagrange's theorem, any subgroup of $S_3$ must have an order dividing $|S_3| = 6$.  Since there is no subgroup of order $4$ in $S_3$, it follows that $HK$ is \emph{not} a subgroup of $G$.
\end{solution}

\begin{problem}[5.53]
  Prove that if $G$ is an abelian group, written multiplicatively, with identity element $e$, then all elements $x$ of $G$ satisfying the equation $x^2 = e$ form a subgroup $H$ of $G$.
\end{problem}

\begin{solution}
  Let $H = \{x \in G \mid x^2 = e\}$. The identity element $e$ of $G$ satisfies $e^2 = e$, so $e \in H$. Thus, $H$ is non-empty. Let $x, y \in H$. Then $x^2 = e$ and $y^2 = e$. We have
  \[%
    (xy)^2 = xyxy = x(yx)y = x(xy)y = (xx)(yy) = e \cdot e = e
  .\]%
  Thus, $xy \in H$. Let $x \in H$. We have
  \[%
    (x^{-1})^2 = x^{-1}x^{-1} = (xx)^{-1} = e^{-1} = e
  .\]%
  Thus, $x^{-1} \in H$.

  Since $H$ is non-empty, closed under the group operation, and contains inverses, we conclude that $H$ is a subgroup of $G$.
\end{solution}

\begin{problem}[5.55]
  Find a counterexample to Exercise 53 if the assumption of abelian is dropped.
\end{problem}

\begin{solution}
  Consider the non-abelian group $G = S_3$, the symmetric group on $\{1,2,3\}$. The elements of $S_3$ are
  \[%
    S_3 = \{e,\ (1\ 2),\ (1\ 3),\ (2\ 3),\ (1\ 2\ 3),\ (1\ 3\ 2)\}
  .\]%
  The elements satisfying $x^2 = e$ are exactly the identity and the transpositions, so
  \[%
    H = \{x\in S_3 \mid x^2 = e\} = \{e, (1\ 2), (1\ 3), (2\ 3)\}
  .\]%
  To see that $H$ is not a subgroup, note that it is not closed under the group operation. For instance,
  \[%
    (1\ 2)(1\ 3) = (1\ 3\ 2) \not\in H
  ,\]%
  because $(1\ 3\ 2)^2=(1\ 2\ 3) \neq e$. Thus $H$ fails to be closed, so it is not a subgroup of $S_3$.
\end{solution}
