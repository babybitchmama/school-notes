\begin{problem}[5.30]
  Find the order of the cyclic subgroup $\Z_{16}$ generated by 12.
\end{problem}

\begin{solution}
  Notice that the cyclic subgroup of $\Z_{16}$ generated by 12 is given by
  \[%
    \bra{12} = \{12 \cdot n \pmod{16} \mid n \in \Z\} = \{0, 4, 8, 12\}
  .\]%
  Thus, the order of the cyclic subgroup is 4.
\end{solution}

\begin{problem}[5.32]
  Find the order of the cyclic subgroup $S_8$ generated by $(2, 4, 6, 9)(3, 5, 7)$.
\end{problem}

\begin{solution}
  Notice that the permutation $(2, 4, 6, 9)(3, 5, 7)$ is the product of two disjoint cycles: a 4-cycle $(2, 4, 6, 9)$ and a 3-cycle $(3, 5, 7)$. The order of a permutation is the least common multiple (LCM) of the lengths of its disjoint cycles. Therefore, the order of the permutation is
  \[%
    \text{lcm}(4, 3) = 12
  .\]%
  Thus, the order of the cyclic subgroup generated by $(2, 4, 6, 9)(3, 5, 7)$ is 12.
\end{solution}

\begin{problem}[5.40]
  Show by means of an example that it is possible for the quadratic equation $x^2 = e$ have more than two solutions in some group with identity $e$.
\end{problem}

\begin{solution}
  Take the algebraic group $\bra{V_4, *}$, where $V_4$ is the Klein four-group defined as $V_4 = \{e, a, b, c\}$ with the operation $*$ defined by the following Cayley table:
  \[%
    \begin{array}{c|cccc}
      * & e & a & b & c \\
      \hline
      e & e & a & b & c \\
      a & a & e & c & b \\
      b & b & c & e & a \\
      c & c & b & a & e \\
    \end{array}
  \]%
  In this group, we can see that:
  \[%
    a * a = e, \quad b * b = e, \quad c * c = e
  .\]%
  Thus, the equation $x^2 = e$ has four solutions: $e$, $a$, $b$, and $c$. This shows that it is possible for the quadratic equation $x^2 = e$ to have more than two solutions in a group.
\end{solution}

\begin{problem}[5.41]
  Let $B$ be a subset of $A$, and let $b$ be a particular element of $B$. Determine whether the subset, $\{\sigma \in S_A \mid \sigma(b) = b\}$, of the symmetric group $S_A$ is a subgroup of $S_A$ under the induced operation.
\end{problem}

\begin{solution}
\end{solution}

\begin{problem}[5.42]
  Let $B$ be a subset of $A$, and let $b$ be a particular element of $B$. Determine whether the subset, $\{\sigma \in S_A \mid \sigma(b) = B\}$, of the symmetric group $S_A$ is a subgroup of $S_A$ under the induced operation.
\end{problem}

\begin{solution}
\end{solution}

\begin{problem}[5.45]
  Let $\Phi : G \to G'$ be an isomorphism of a group $\bra{G, *}$ with a group $\bra{G', *'}$. Prove that if $H$ is a subgroup of $G$, then $\Phi[H] = \{\Phi(h) \mid h \in H\}$ is a subgroup of $G'$. That is, an isomorphism carries subgroups into subgroups
\end{problem}

\begin{solution}
\end{solution}

\begin{problem}[5.46]
  Let $\Phi : G \to G'$ be an isomorphism of a group $\bra{G, *}$ with a group $\bra{G', *'}$. Prove that if there is an $a \in G$ such that $\bra{a} = G$, then $G'$ is cyclic.
\end{problem}

\begin{solution}
\end{solution}

\begin{problem}[5.48]
  Find an example of a group $G$ and two subgroups $H$ and $K$ such that the set in Exercise 47 is not a subgroup of $G$.
\end{problem}

\begin{solution}
\end{solution}

\begin{problem}[5.53]
  Prove that if $G$ is an abelian group, written multiplicatively, with identity element $e$, then all elements $x$ of $G$ satisfying the equation $x^2 = e$ form a subgroup $H$ of $G$.
\end{problem}

\begin{solution}
\end{solution}

\begin{problem}[5.55]
  Find a counterexample to Exercise 53 if the assumption of abelian is dropped.
\end{problem}

\begin{solution}
\end{solution}
