\begin{problem}[7.4]
  List the elements of the subgroup generated by the subset $\{12, 30\}$ of $\Z_{36}$.
\end{problem}

\begin{solution}
  The subgroup generated by the subset $\{12, 30\}$ of $\Z_{36}$ consists of all integer combinations of 12 and 30 modulo 36. We can find the elements by calculating the greatest common divisor (gcd) of 12 and 30, which is 6. Therefore, the subgroup generated by $\{12, 30\}$ is the set of all multiples of 6 modulo 36.
  The elements of this subgroup are:
  \[%
    \{0, 6, 12, 18, 24, 30\}
  .\qedhere\]%
\end{solution}

\begin{problem}[7.6]
  List the elements of the subgroup generated by the subset $\{18, 24, 39\}$ of $\Z$.
\end{problem}

\begin{solution}
  The subgroup generated by the subset $\{18, 24, 39\}$ of $\Z$ consists of all integer combinations of 18, 24, and 39. We can find the elements by calculating the greatest common divisor (gcd) of 18, 24, and 39, which is 3. Therefore, the subgroup generated by $\{18, 24, 39\}$ is the set of all multiples of 3.
  The elements of this subgroup are:
  \[%
    \{\cdots, -6, -3, 0, 3, 6, 9, 12, \cdots\}
  .\qedhere\]%
\end{solution}

\begin{problem}[7.18]
  Draw a Cayley digraph for $\Z_8$ with generating set $S = \{2, 5\}$.
\end{problem}

\begin{solution}
  I'm too tired to draw the picture.
\end{solution}

\begin{problem}[8.12]
  Compute the kernel for the given homomorphism $\Phi : \Z \to \Z$ such that $\Phi(1) = 12$.
\end{problem}

\begin{solution}
  The kernel of the homomorphism $\Phi : \Z \to \Z$ defined by $\Phi(1) = 12$ consists of all integers $n$ such that $\Phi(n) = 0$. Since $\Phi(n) = 12n$, we need to solve the equation:
  \[%
    12n = 0
  .\]%
  The only solution to this equation is $n = 0$. Therefore, the kernel of the homomorphism is:
  \[%
    \Ker(\Phi) = \{0\}
  .\qedhere\]%
\end{solution}

\begin{problem}[8.14]
  Compute the kernel for the given homomorphism $\Phi : \Z \times \Z \to \Z$ where $\Phi(1, 0) = 6$ and $\Phi(0, 1) = 9$.
\end{problem}

\begin{solution}
  The kernel of the homomorphism $\Phi : \Z \times \Z \to \Z$ defined by $\Phi(1, 0) = 6$ and $\Phi(0, 1) = 9$ consists of all pairs $(m, n) \in \Z \times \Z$ such that $\Phi(m, n) = 0$. We can express $\Phi(m, n)$ as:
  \[%
    \Phi(m, n) = 6m + 9n
  .\]%
  We need to solve the equation:
  \[%
    6m + 9n = 0
  .\]%
  Dividing the entire equation by 3 gives:
  \[%
    2m + 3n = 0
  .\]%
  Rearranging this equation, we find:
  \[%
    2m = -3n \implies m = -\frac{3}{2}n
  .\]%
  For $m$ to be an integer, $n$ must be even. Letting $n = 2k$ for some integer $k$, we have:
  \[%
    m = -3k
  .\]%
  Therefore, the elements of the kernel are of the form:
  \[%
    (m, n) = (-3k, 2k) \text{ for } k \in \Z
  .\]%
  Thus, the kernel can be expressed as:
  \[%
    \Ker(\Phi) = \{(-3k, 2k) \mid k \in \Z\}
  .\qedhere\]%
\end{solution}

\begin{problem}[8.24]
  Express the permutation of $\{1, 2, 3, 4, 5, 6, 7, 8\}$ as a product of disjoint cycles, and then as a product of transpositions
  \[%
    \begin{pmatrix}
      1 & 2 & 3 & 4 & 5 & 6 & 7 & 8 \\
      3 & 6 & 4 & 1 & 8 & 2 & 5 & 7 \\
    \end{pmatrix}
  .\]%
\end{problem}

\begin{solution}
  To express the given permutation as a product of disjoint cycles, we start with the first element and follow its mapping.
  \begin{enumerate}
    \item Start with $1 \mapsto 3 \mapsto 4 \mapsto 1$, giving the cycle $(1 \, 3 \, 4)$.
    \item Next, we have $2 \mapsto 6 \mapsto 2$, giving the cycle $(2 \, 6)$.
    \item Lastly, we have $5 \mapsto 8 \mapsto 7 \mapsto 5$, giving the cycle $(5 \, 8 \, 7)$.
  \end{enumerate}
  Combining all cycles, we have
  \[%
    (1 \, 3 \, 4)(2 \, 6)(5 \, 8 \, 7)
  .\]%
  Now, to express this as a product of transpositions. The cycle $(1 \, 3 \, 4)$ can be expressed as $(1 \, 4)(1 \, 3)$. The cycle $(2 \, 6)$ can be expressed as $(2 \, 6)$. The cycle $(5 \, 8 \, 7)$ can be expressed as $(5 \, 7)(5 \, 8)$.

  Therefore, the entire permutation can be expressed as:
  \[%
    (1 \, 4)(1 \, 3)(2 \, 6)(5 \, 7)(5 \, 8)
  .\qedhere\]%
\end{solution}

\begin{problem}[8.26]
  Figure 8.26 shows a Cayley digraph for the alternating group $A_4$ using the generating set $S = \{(1, 2, 3), (1, 2)(3, 4)\}$. Continue labeling the other nine vertices with the elements of $A_4$, expressed as a product of disjoint cycles.
\end{problem}

\begin{solution}
  I'm too tired to draw the picture.
\end{solution}

\begin{problem}[8.35]
  Suppose that $\Phi : G \to G'$ is a group homomorphism and $a \in \Ker(\Phi)$. Show that for any $g \in G$, $gag^{-1} \in \Ker(\Phi)$.
\end{problem}

\begin{solution}
  Let $g \in G$ and $a \in \Ker(\Phi)$. By definition of the kernel, we have $\Phi(a) = e'$, where $e'$ is the identity element in $G'$. We need to show that $gag^{-1} \in \Ker(\Phi)$, which means we need to show that $\Phi(gag^{-1}) = e'$.

  Using the properties of homomorphisms, we have:
  \[%
    \Phi(gag^{-1}) = \Phi(g) \Phi(a) \Phi(g^{-1})
  .\]%
  Since $\Phi(a) = e'$, this simplifies to:
  \[%
    \Phi(gag^{-1}) = \Phi(g) e' \Phi(g^{-1}) = \Phi(g) \Phi(g^{-1})
  .\]%
  By the property of homomorphisms, we know that $\Phi(g^{-1}) = (\Phi(g))^{-1}$. Therefore:
  \[%
    \Phi(gag^{-1}) = \Phi(g) (\Phi(g))^{-1} = e'
  .\]%
  Thus, we have shown that $\Phi(gag^{-1}) = e'$, which means that $gag^{-1} \in \Ker(\Phi)$.

  Therefore, for any $g \in G$, if $a \in \Ker(\Phi)$, then $gag^{-1} \in \Ker(\Phi)$ as well.
\end{solution}

\begin{problem}[8.36]
  Prove that a homomorphism $\Phi : G \to G'$ is one-to-one if and only if $\Ker(\Phi)$ is the trivial subgroup of $G$.
\end{problem}

\begin{solution}
  Suppose that $\Phi : G \to G'$ is one-to-one. By definition, the kernel of $\Phi$ is given by $\Ker(\Phi) = \{g \in G \mid \Phi(g) = e'\}$, where $e'$ is the identity element in $G'$. Since $\Phi$ is one-to-one, the only element in $G$ that maps to $e'$ is the identity element $e$ in $G$. Therefore, we have:
  \[%
    \Ker(\Phi) = \{e\}
  .\]%
  Thus, $\Ker(\Phi)$ is the trivial subgroup of $G$.

  Now suppose that $\Ker(\Phi)$ is the trivial subgroup of $G$, i.e., $\Ker(\Phi) = \{e\}$. We need to show that $\Phi$ is one-to-one. Assume that $\Phi(g_1) = \Phi(g_2)$ for some $g_1, g_2 \in G$. We want to show that $g_1 = g_2$.

  Applying the properties of homomorphisms on the assumptions, we can rewrite this as
  \[%
    \Phi(g_1 g_2^{-1}) = e'
  .\]%
  This implies that $g_1 g_2^{-1} \in \Ker(\Phi)$. Since we assumed that $\Ker(\Phi) = \{e\}$, it follows that
  \[%
    g_1 g_2^{-1} = e
  .\]%
  Therefore, we have $g_1 = g_2$. Therefore, $\Phi$ is one-to-one.

  Thus, $\Phi$ is one-to-one if and only if $\Ker(\Phi)$ is the trivial subgroup of $G$.
\end{solution}

\begin{problem}[8.37]
  Let $\Phi : G \to G$ be a group homomorphism. Show that $\Phi(a) = \Phi(b)$ if and only if $a^{-1}b \in \Ker(\Phi)$.
\end{problem}

\begin{solution}
  Suppose that $\Phi(a) = \Phi(b)$ for some $a, b \in G$. We need to show that $a^{-1}b \in \Ker(\Phi)$. Using the properties of homomorphisms, we can rewrite this as:
  \[%
    \Phi(a^{-1}b) = \Phi(a^{-1})\Phi(b)
  .\]%
  Since $\Phi(a) = \Phi(b)$, we have $\Phi(a^{-1}) = (\Phi(a))^{-1} = (\Phi(b))^{-1}$. Therefore:
  \[%
    \Phi(a^{-1}b) = (\Phi(b))^{-1}\Phi(b) = e'
  .\]%
  This implies that $a^{-1}b \in \Ker(\Phi)$.

  Now suppose that $a^{-1}b \in \Ker(\Phi)$. We need to show that $\Phi(a) = \Phi(b)$. Using the properties of homomorphisms, we have:
  \[%
    \Phi(a^{-1}b) = e'
  .\]%
  This implies that:
  \[%
    \Phi(a^{-1})\Phi(b) = e'
  .\]%
  Therefore, we have:
  \[%
    (\Phi(a))^{-1}\Phi(b) = e' \implies \Phi(b) = \Phi(a)
  .\]%
  Thus, we have shown that $\Phi(a) = \Phi(b)$ if and only if $a^{-1}b \in \Ker(\Phi)$.
\end{solution}
