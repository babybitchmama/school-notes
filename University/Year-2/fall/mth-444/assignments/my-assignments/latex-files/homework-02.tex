\begin{problem}[2.10]
  Let $n$ be a positive integer and let $n\Z = \{nm \mid m \in \Z\}$.
  \begin{enumerate}
    \item Show that $\bra{n\Z, +}$ is a group.

    \item Show that $\bra{n\Z, +} \cong \bra{\Z, +}$.
  \end{enumerate}
\end{problem}

\begin{solution}[(i)]
  We verify the group axioms for $\bra{n\Z, +}$.

  Let $a, b \in n\Z$. Then $a = nm$ and $b = nk$ for some $m, k \in \Z$. Hence,
  \[%
    a + b = nm + nk = n(m + k)
  .\]%
  Since $m + k \in \Z$, it follows that $a + b \in n\Z$.

  Addition in $\Z$ is associative, and $n\Z$ inherits this property. Thus, for all $a, b, c \in n\Z$,
  \[%
    (a + b) + c = a + (b + c)
  .\]%

  The additive identity is $0$, since for any $a \in n\Z$,
  \[%
    a + 0 = a
  .\]%

  For any $a = nm \in n\Z$, the additive inverse is $-a = n(-m)$, because
  \[%
    a + (-a) = nm + n(-m) = n(m - m) = 0
  .\]%

  Therefore, all group axioms are satisfied, and $\bra{n\Z, +}$ is a group.
\end{solution}

\begin{solution}[(ii)]
  Define a map $\phi : \Z \to n\Z$ by $\phi(m) = nm$, for all $m \in \Z$. We show that $\phi$ is an isomorphism.

  For any $m_1, m_2 \in \Z$,
  \[
    \phi(m_1 + m_2) = n(m_1 + m_2) = nm_1 + nm_2 = \phi(m_1) + \phi(m_2).
  \]
  Hence, $\phi$ preserves addition.

  Suppose $\phi(m_1) = \phi(m_2)$. Then $nm_1 = nm_2$, and since $n \neq 0$, we may divide by $n$ to obtain $m_1 = m_2$. Thus, $\phi$ is injective.

  For any $a \in n\Z$, there exists $k \in \Z$ such that $a = nk$. Then $\phi(k) = nk = a$, so $\phi$ is surjective.

  Since $\phi$ is an isomorphism, it follows that $\bra{n\Z, +} \cong \bra{\Z, +}$.
\end{solution}

\begin{problem}[2.19]
  Let $S$ be the set of all real numbers except $-1$. Define $*$ on $S$ by
  \[%
    a * b = a + b + ab
  .\]%
  \begin{enumerate}
    \item Show that $*$ gives a binary operation on $S$.

    \item Show that $\bra{S, *}$ is a group.

    \item Find the solution of the equation $2 * x * 3 = 7$ in $S$.
  \end{enumerate}
\end{problem}

\begin{solution}[(i)]
  We need to show that for any $a, b \in S$, the result of the operation $a * b$ is also in $S$. 

  Let $a, b \in S$. Then, $a * b = a + b + ab$. We need to check that $a * b \neq -1$. Suppose for contradiction that $a * b = -1$. Then, $a + b + ab = -1$. Rearranging gives $ab + a + b + 1 = 0$, which can be factored as $(a + 1)(b + 1) = 0$.

  Since $a, b \in S$, we have $a \neq -1$ and $b \neq -1$, so neither factor can be zero. This is a contradiction. Therefore, $a * b \neq -1$, and thus $a * b \in S$.

  Hence, $*$ is a binary operation on $S$.
\end{solution}

\begin{solution}[(ii)]
  We verify the group axioms for $\bra{S, *}$.

  As shown in part (i), for any $a, b \in S$, $a * b \in S$.

  For any $a, b, c \in S$,
  \[%
    (a * b) * c = (a + b + ab) * c = (a + b + ab) + c + (a + b + ab)c
  .\]%
  Expanding this gives
  \[%
    a + b + ab + c + ac + bc + abc
  .\]%
  Similarly,
  \[%
    a * (b * c) = a * (b + c + bc) = a + (b + c + bc) + a(b + c + bc)
  .\]%
  Expanding this also gives
  \[%
    a + b + c + bc + ab + ac + abc
  .\]%
  Since both expressions are equal, $*$ is associative.

  We need to find an element $e \in S$ such that for all $a \in S$, $a * e = e * a = a$. Let $e = 0$. Then,
  \[%
    a * 0 = a + 0 + a \cdot 0 = a
  ,\]%
  and
  \[%
    0 * a = 0 + a + 0 \cdot a = a
  .\]%
  Thus, $0$ is the identity element.

  Fix $a \in S$. We want to find a $b \in S$ with $a * b = 0$. Solving the equation $a + b + ab = 0$, we get $(a + 1)(b + 1) = 1$. Thus $b + 1 = 1/(a + 1)$, so
  \[%
    b = \frac{1}{a + 1} - 1 = -\frac{a}{a + 1}
  .\]%
  Since $a \neq -1$ the denominator $a + 1 \neq 0$, so this $b$ is well-defined and satisfies $b \neq -1$. Notice that $b \ne -1$, since that would imply that $-a/(a+1) = -1$, which gives us $0 = 1$, which is a contradiction. Hence every element has a two-sided inverse in $S$.

  Since all group axioms are satisfied, $\bra{S, *}$ is a group.
\end{solution}

\begin{solution}[(iii)]
  We need to solve the equation $2 * x * 3 = 7$ in $S$. First, we compute $2 * x$:
  \[%
    2 * x = 2 + x + 2x = x + 2 + 2x = 3x + 2
  .\]%
  Next, we compute $(2 * x) * 3$:
  \[%
    (3x + 2) * 3 = (3x + 2) + 3 + (3x + 2) \cdot 3 = 3x + 2 + 3 + 9x + 6 = 12x + 11
  .\]%
  We set this equal to $7$:
  \[%
    12x + 11 = 7
  .\]%
  Solving for $x$, we get
  \[%
    12x = -4 \implies x = -\frac{1}{3}
  .\]%
  Since $-1/3 \in S$, the solution is $x = -1/3$.
\end{solution}

\begin{problem}[2.28]
  An element $a \ne e$ in a group is said to have order 2 if $a * a = e$. Prove that if $G$ is a group and $a \in G$ has order 2, then for any $b \in G$, $b' * a * b$ also has order $2$.
\end{problem}

\begin{solution}
  Assume $a \ne e$ is an element of order 2 in the group $G$, i.e., $a * a = e$. Then, for any $b \in G$, we compute the square of the element $b' * a * b$:
  \[%
    (b' * a * b) * (b' * a * b) = b' * a * (b * b') * a * b = b' * a * a * b = b' * b = e
  .\]%
  Thus, $(b' * a * b)$ has order 2.
\end{solution}

\begin{problem}[2.29]
  Show that if $G$ is a finite group with identity $e$ and with an even number of elements, then there is $a \ne e$ in G such that $a * a = e$.
\end{problem}

\begin{solution}
  Let $G$ be a finite group with identity $e$ and an even number of elements. Consider the set $G \setminus \{e\}$, which has an odd number of elements.

  For each element $a \in G \setminus \{e\}$, consider its inverse $a'$. If $a \ne a'$, then we can pair $a$ with $a'$. Each such pair contributes two distinct elements to the set. Since the total number of elements in $G \setminus \{e\}$ is odd, there must be at least one element that is its own inverse, i.e., an element $a$ such that $a = a'$.

  For this element, we have $a * a = e$. Since $a \ne e$, we have found an element in $G$ such that $a * a = e$.
\end{solution}

\begin{problem}[2.30]
  Let $\R^*$ be the set of all real numbers except 0. Define $*$ on $\R^*$ by letting $a * b = |a|b$.
  \begin{enumerate}
    \item Show that $*$ gives an associative binary operation on $\R^*$.

    \item Show that there is a left identity for $*$ and a right inverse for each element in $\R^*$.

    \item Is $\R^*$ with this binary operation a group?

    \item Explain the significance of this exercise.
  \end{enumerate}
\end{problem}

\begin{solution}[(i)]
  We need to show that for any $a, b \in \R^*$, the result of the operation $a * b$ is also in $\R^*$. 

  Let $a, b \in \R^*$. Then, $a * b = |a|b$. Since $|a| > 0$ and $b \neq 0$, it follows that $|a|b \neq 0$. Thus, $a * b \in \R^*$.

  Next, we verify associativity. For any $a, b, c \in \R^*$,
  \[%
    (a * b) * c = (|a|b) * c = | |a|b | c = |a||b|c
  .\]%
  Similarly,
  \[%
    a * (b * c) = a * (|b|c) = |a|(|b|c) = |a||b|c
  .\]%
  Since both expressions are equal, $*$ is associative.

  Therefore, $*$ is an associative binary operation on $\R^*$.
\end{solution}

\begin{solution}[(ii)]
  We need to find a left identity element $e \in \R^*$ such that for all $a \in \R^*$, $e * a = a$. Let $e = 1$. Then,
  \[%
    e * a = 1 * a = |1|a = a
  .\]%
  Thus, $1$ is a left identity.

  Next, we need to find a right inverse for each element $a \in \R^*$. We want to find $b \in \R^*$ such that $a * b = e$. Using the left identity found above, we have:
  \[%
    a * b = |a|b = 1
  .\]%
  Solving for $b$, we get:
  \[%
    b = \frac{1}{|a|}
  .\]%
  Since $|a| > 0$, it follows that $b \neq 0$, and thus $b \in \R^*$. Therefore, every element in $\R^*$ has a right inverse.

  Hence, there is a left identity and a right inverse for each element in $\R^*$.
\end{solution}

\begin{solution}[(iii)]
  To determine if $\R^*$ with the operation $*$ is a group, we need to check if it satisfies all group axioms.

  We have already shown that $*$ is an associative binary operation on $\R^*$, and that there is a left identity element (1) and a right inverse for each element.

  However, we need to check if the left identity is also a right identity. For any $a \in \R^*$,
  \[%
    a * e = a * 1 = |a| \cdot 1 = |a|
  .\]%
  Since $|a|$ is not necessarily equal to $a$ (for example, if $a = -2$, then $|a| = 2$), the left identity is not a right identity.

  Therefore, $\R^*$ with this binary operation does not satisfy all the group axioms, and hence it is not a group.
\end{solution}

\begin{solution}[(iv)]
  The significance of this exercise is to illustrate that having a left identity and right inverses for each element does not guarantee that a set with a binary operation forms a group. The failure of the left identity to also be a right identity shows that the group axioms are not fully satisfied. This shows the importance of verifying all group properties when determining if a structure is indeed a group.
\end{solution}

\begin{problem}[2.31]
  If $*$ is a binary operation on a set $S$, an element $x$ of $S$ is an \emph{idempotent for} $*$ if $x * x = x$. Prove that a group has exactly one idempotent element. (You may use any theorems proved so far in the text.)
\end{problem}

\begin{solution}
  Let $G$ be a group with identity element $e$. We first show that $e$ is an idempotent element. Obviously, $e * e = e$.

  Now, suppose there is another idempotent element $x \in G$ such that $x * x = x$. We will show that $x = e$.

  Since $G$ is a group, every element has an inverse. Let $x'$ be the inverse of $x$. Then,
  \[%
    x' * (x * x) = x' * x
  .\]%
  Using associativity, we have $(x' * x) * x = e * x = x$. But since $x' * x = e$, we have $e * x = x$. Therefore, $x = e$.

  Thus, the only idempotent element in the group $G$ is the identity element $e$. Hence, a group has exactly one idempotent element.
\end{solution}

\begin{problem}[2.32]
  Show that every group $G$ with identity $e$ and such that $x * x = e$ for all $x \in G$ is abelian. [Hint: Consider $(a * b) * (a * b)$.]
\end{problem}

\begin{solution}
  Let $a, b \in G$. Since $x * x = e$ for all $x \in G$, each element is its own inverse; i.e. $x' = x$ for every $x \in G$.

  Consider $(a * b) * (a * b)$. By hypothesis $(a * b) * (a * b) = e$, so
  \[%
    (a * b)' = (a * b)
  .\]%
  On the other hand, the general inverse formula in any group gives $(a * b)' = b' * a'$. Using $x' = x$ for $a$ and $b$ we obtain
  \[%
    (a * b)' = b' * a' = b * a
  .\]%
  Combining the two expressions for $(a * b)'$ yields $a * b = b * a$. Thus, $G$ is abelian.
\end{solution}

\begin{problem}[2.33]
  Let $G$ be an abelian group and let $c^n = c * c * \cdots * c$ for $n$ factors $c$, where $c \in G$ and $n \in \Z^+$. Give a mathematical induction proof that $(a * b)^n = (a^n) * (b^n)$ for all $a, b \in G$.
\end{problem}

\begin{solution}
  We will prove by induction on $n$ that for all $a, b \in G$, $(a * b)^n = (a^n) * (b^n)$.

  For the base case, we have $n = 1$, which gives us
  \[%
    (a * b)^1 = a * b
  .\]%
  Also,
  \[%
    a^1 * b^1 = a * b
  .\]%
  Thus, the base case holds.

  Assume that the statement holds for some $k \in \Z^+$, i.e., assume that
  \[%
    (a * b)^k = (a^k) * (b^k)
  .\]%
  We need to show that it holds for $k + 1$. Consider:
  \[%
    (a * b)^{k + 1} = (a * b)^k * (a * b)
  .\]%
  By the inductive hypothesis, we have:
  \[%
    (a * b)^{k + 1} = ((a^k) * (b^k)) * (a * b)
  .\]%
  Since $G$ is abelian, we can rearrange the terms:
  \[%
    (a * b)^{k + 1} = (a^k * a) * (b^k * b) = a^{k + 1} * b^{k + 1}
  .\]%
  Thus, by the principle of mathematical induction, the statement holds for all $n \in \Z^+$.
\end{solution}

\begin{problem}[2.34]
  Suppose that $G$ is a group and $a, b \in G$ satisfy $a * b = b * a'$ where as usual, $a'$ is the inverse for $a$. Prove that $b * a = a' * b$.
\end{problem}

\begin{solution}
  Given that $a * b = b * a'$, we want to show that $b * a = a' * b$.

  Starting from the given equation, we can multiply both sides on the left by $a'$:
  \[%
    a' * (a * b) = a' * (b * a')
  .\]%
  Using associativity, we have:
  \[%
    (a' * a) * b = a' * b * a'
  .\]%
  Simplifying, we get:
  \[%
    b = a' * b * a'
  .\]%
  Now, we can multiply both sides on the right by $a$:
  \[%
    b * a = (a' * b * a') * a
  .\]%
  Again, using associativity, we have:
  \[%
    b * a = a' * b * (a' * a)
  .\]%
  Since $a' * a = e$, this simplifies to:
  \[%
    b * a = a' * b
  .\]%
  Thus, we have shown that $b * a = a' * b$.
\end{solution}

\begin{problem}[2.36]
  Let $G$ be a group with a finite number of elements. Show that for any $a \in G$, there exists $n \in \Z^+$ such that $a^n = e$. See Exercise 33 for the meaning of $a^n$. [Hint: Consider $e, a, a^2, a^3, \cdots, a^m$, where $m$ is the number of elements in $G$, and use the cancellation laws.]
\end{problem}

\begin{solution}
  Let $G$ be a finite group with $m$ elements. Consider the sequence of elements:
  \[%
    e, a, a^2, a^3, \cdots, a^m
  .\]%
  Since $G$ has only $m$ elements, by the pigeonhole principle, at least two of these elements must be equal. Thus, there exist integers $i$ and $j$ with $0 \leq i < j \leq m$ such that:
  \[%
    a^i = a^j
  .\]%
  Using the cancellation law (which holds in groups), we can multiply both sides on the left by $(a^i)'$, the inverse of $a^i$, to obtain:
  \[%
    e = a^{j - i}
  .\]%
  Letting $n = j - i$, we have found a positive integer $n$ such that $a^n = e$. Thus, for any element $a \in G$, there exists $n \in \Z^+$ such that $a^n = e$.
\end{solution}
