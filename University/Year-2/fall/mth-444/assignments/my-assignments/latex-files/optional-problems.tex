\begin{problem}[0.17]
  Let $A$ be a finite set, and let $|A| = s$. Based on the preceding exercise, make a conjecture about the value of $|\Pow(A)|$ Then try to prove your conjecture.
\end{problem}

\begin{solution}
\end{solution}

\begin{problem}[0.18]
  For any set $A$, finite or infinite, let $B^A$ be the set of all functions mapping $A$ into the set $B = \{0, 1\}$. Show that the cardinality of $B^A$ is the same as the cardinality of the set $\Pow(A)$.
  [Hint: Each element of $B^A$ determines a subset of $A$ in a natural way.]
\end{problem}

\begin{solution}
\end{solution}

\begin{problem}[0.19]
  Show that the power set of a set A, finite or infinite, has too many elements to be able to be put in a one-to-one correspondence with A. Explain why this intuitively means that there are an infinite number of infinite cardinal numbers.
  [Hint: Imagine a one-to-one function $\phi$ mapping A into $\Pow(A)$ to be given. Show that $\phi$ cannot be onto $\Pow(A)$ by considering, for each $x \in A$, whether $x \in \phi(x)$ and using this idea to define a subset $S$ of $A$ that is not in the range of $\phi$.]
  Is the set of everything a logically acceptable concept? Why or why not?
\end{problem}

\begin{solution}
\end{solution}
