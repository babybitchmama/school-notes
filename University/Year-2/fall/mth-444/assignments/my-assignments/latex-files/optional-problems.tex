\begin{problem}[0.17]
  Let $A$ be a finite set, and let $|A| = s$. Based on the preceding exercise, make a conjecture about the value of $|\Pow(A)|$ Then try to prove your conjecture.
\end{problem}

\begin{solution}
\end{solution}

\begin{problem}[0.18]
  For any set $A$, finite or infinite, let $B^A$ be the set of all functions mapping $A$ into the set $B = \{0, 1\}$. Show that the cardinality of $B^A$ is the same as the cardinality of the set $\Pow(A)$. [Hint: Each element of $B^A$ determines a subset of $A$ in a natural way.]
\end{problem}

\begin{solution}
\end{solution}

\begin{problem}[0.19]
  Show that the power set of a set A, finite or infinite, has too many elements to be able to be put in a one-to-one correspondence with A. Explain why this intuitively means that there are an infinite number of infinite cardinal numbers. [Hint: Imagine a one-to-one function $\phi$ mapping A into $\Pow(A)$ to be given. Show that $\phi$ cannot be onto $\Pow(A)$ by considering, for each $x \in A$, whether $x \in \phi(x)$ and using this idea to define a subset $S$ of $A$ that is not in the range of $\phi$.] Is the set of everything a logically acceptable concept? Why or why not?
\end{problem}

\begin{solution}
\end{solution}

\begin{problem}[1.44]
  Let $S$ be a set and let $*$ be a binary operation on $S$ satisfying the two laws
  \begin{itemize}
    \item $x * x = x$ for all $x \in S$

    \item $(x * y) * z = (y * z) * x$ for all $x, y, z \in S$.
  \end{itemize}
  Show that $*$ is associative and commutative. (This is problem B-1 on the 1971 Putnam Competition.)
\end{problem}

\begin{solution}
\end{solution}

\begin{problem}[2.20]
  This exercise shows that there are two non-isomorphic group structures on a set of 4 elements.

  Let the set be $\{e, a, b, c\}$, with $e$ the identity element for the group operation. A group table would then have to start in the manner shown in Table 2.29. The square indicated by the question mark cannot be filled in with $a$. It must be filled in either with the identity element $e$ or with an element different from both $e$ and $a$. In this latter case, it is no loss of generality to assume that this element is $b$. If this square is filled in with $e$, the table can then be completed in two ways to give a group. Find these two tables. (You need not check the associative law.) If this square is filled in with $b$, then the table can only be completed in one way to give a group. Find this table. (Again, you need not check the associative law.) Of the three tables you now have, two give isomorphic groups. Determine which two tables these are, and give the one-to-one onto relabeling function which is an isomorphism.
  \begin{enumerate}
    \item Are all groups of 4 elements commutative?

    \item Find a way to relabel the four matrices

      \[%
        \left\{
          \begin{bmatrix}
            1 & 0 \\
            0 & 1 \\
          \end{bmatrix},
          \begin{bmatrix}
            0 & -1 \\
            1 & 0 \\
          \end{bmatrix},
          \begin{bmatrix}
            -1 & 0 \\
            0 & -1 \\
          \end{bmatrix},
          \begin{bmatrix}
            0 & 1 \\
            -1 & 0 \\
          \end{bmatrix}
        \right\}
      ,\]%
      so the matrix multiplication table is identical to one you constructed. This shows that the table you constructed defines an associative operation and therefore gives a group.

    \item Show that for a particular value of n , the group elements given in Exercise 14 can be relabeled so their group table is identical to one you constructed. This implies the operation in the table is also associative.
  \end{enumerate}
\end{problem}

\begin{solution}[(i)]
\end{solution}

\begin{solution}[(ii)]
\end{solution}

\begin{solution}[(iii)]
\end{solution}

\begin{problem}[2.38]
  Let $G$ be a group and let $a, b \in G$. Show that $(a * b)' = a' * b'$ if and only if $a * b = b * a$.
\end{problem}

\begin{solution}
\end{solution}

\begin{problem}[2.40]
  Prove that a set $G$, together with a binary operation $*$ on $G$ satisfying the left axioms 1, 2, and 3 given after Corollary 2.19, is a group.
\end{problem}

\begin{solution}
\end{solution}

\begin{problem}[2.41]
  Prove that a nonempty set $G$, together with an associative binary operation $*$ on $G$ such that
  \[%
    a * x = b~\text{and}~y * a = b~\text{have solutions in}~G~\text{for all}~a, b \in G
  ,\]%
  is a group. [Hint: Use Exercise 40.]
\end{problem}

\begin{solution}
\end{solution}

\begin{problem}[2.45]
  Suppose that $G$ is a group with $n$ elements and $A \subseteq G$ has more than $n/2$ elements. Prove that for every $g \in G$, there exists $a, b \in A$ such that $a * b = g$. (This was Problem B-2 on the 1968 Putnam exam.)
\end{problem}

\begin{solution}
\end{solution}

\begin{problem}[3.44]
  Prove that for any $n \in \Z^+$, $\bra{\Z_n, +_n}$ is associative without using the fact that $U_n$ is associative.
\end{problem}

\begin{solution}
\end{solution}

\begin{problem}[4.34]
  (See the warning following Theorem 4.8.) Let $G$ be a group with binary operation $*$. Let $G'$ be the same set as $G$, and define a binary operation $*'$ on $G'$ by $x *' y=y * x$ for all $x, y \in G'$.
  \begin{enumerate}
    \item (Intuitive argument that $G'$ under $*'$ is a group.) Suppose the front wall of your classroom were made of transparent glass, and that all possible products $a * b=c$ and all possible instances $a *(b * c)= (a * b) * c$ of the associative property for $G$ under $*$ were written on the wall with a magic marker. What would a person see when looking at the other side of the wall from the next room in front of yours?

    \item Show from the mathematical definition of $*'$ that $G'$ is a group under $*'$.
  \end{enumerate}
\end{problem}

\begin{solution}[(i)]
\end{solution}

\begin{solution}[(ii)]
\end{solution}

\begin{problem}[5.58]
  Let $G$ be a group and let $a$ be one fixed element of $G$. Show that
  \[%
    H_a = \{x \in G \mid xa = ax\}
  ,\]%
  is a subgroup of $G$.
\end{problem}

\begin{solution}
\end{solution}

\begin{problem}[5.59]
  Generalizing Exercise 58, let S be any subset of a group $G$.
  \begin{enumerate}
    \item Show that $H_S = \{x \in G \mid xs = sx~\text{for all}~s \in S\}$ is a subgroup of $G$.

    \item In reference to part (i), the subgroup $H_G$ is the \emph{center} of $G$. Show that $H_G$ is an abelian group.
  \end{enumerate}
\end{problem}

\begin{solution}[(i)]
\end{solution}

\begin{solution}[(ii)]
\end{solution}

\begin{problem}[5.60]
  Let $H$ be a subgroup of a group $G$. For $a, b \in G$, let $a ~ b$ if and only if $ab^{-1} \in H$. Show that $~$ is an equivalence relation on $G$.
\end{problem}

\begin{solution}
\end{solution}

\begin{problem}[6.58]
  Show that a group that has only a finite number of subgroups must be a finite group.
\end{problem}

\begin{solution}
\end{solution}

\begin{problem}[6.61]
  Prove that if $G$ is a cyclic group with an odd number of generators, then $G$ has two elements.
\end{problem}

\begin{solution}
\end{solution}

\begin{problem}[6.67]
  Let $G$ be an abelian group and let $H$ and $K$ be finite cyclic subgroups with $|H| = r$ and $|K| = s$.
\end{problem}

\begin{solution}
\end{solution}

\begin{problem}[7.22]
  Prove that there are at least three different abelian groups of order 8. [Hint: Find a Cayley digraph for a group of order 8 having one generator of order 4 and another of order 2. Find a second Cayley digraph for a group of order 8 having three generators each with order 2.]
\end{problem}

\begin{solution}
\end{solution}

\begin{problem}[8.38]
  Use Exercise 37 to prove that if $\Phi : G \to G'$ is a group homomorphism mapping onto $G'$ and $G'$ is a finite group, then for any $b, c \in G'$, $|\Phi^{-1}[\{b\}]| = |\Phi^{-1}[\{c\}]|$. Conclude that if $|G|$ is a prime number, then either $\Phi$ is an isomorphism or else $G'$ is the trivial group.
\end{problem}

\begin{solution}
\end{solution}

\begin{problem}[8.50]
  Show that $S_n$ is generated by $\{(1, 2), (1, 2, 3, \cdots, n)\}$. [Hint: Show that as $r$ varies, $(1, 2, 3, \cdots, n)^r(1, 2)(1, 2, 3, \cdots, n)^{n-r}$ gives all the transpositions $(1, 2)$, $(2, 3)$, $(3, 4)$, $\cdots$, $(n - 1, n)$, $(n, 1)$. Then show that any transposition is a product of some of these transpositions and use Theorem 8.15.]
\end{problem}

\begin{solution}
\end{solution}

\begin{problem}[8.52]
  The usual definition for the determinant of an $n \times n$ matrix $A = (a_{i,j})$ is
  \[%
    \det(A) = \sum_{\sigma \in S_n} \sgn(\sigma)a_{1,\sigma(1)}a_{2,\sigma(2)} \cdots a_{n,\sigma(n)}
  ,\]%
  where $\sgn(\sigma)$ is the sign of $\sigma$. Using this definition, prove the following properties of determinants.
  \begin{enumerate}
    \item If a row of matrix $A$ has all zero entries, then $\det(A) = 0$.
    \item If two different rows of $A$ are switched to obtain $B$, then $\det(B) = -\det(A)$.
    \item If $r$ times one row of $A$ is added to another row of $A$ to obtain a matrix $B$, then $\det(B) = \det(A)$.
    \item If a row of $A$ is multiplied by $r$ to obtain the matrix $B$, then $\det(B) = r \det(A)$.
  \end{enumerate}
\end{problem}

\begin{solution}[(i)]
\end{solution}

\begin{solution}[(ii)]
\end{solution}

\begin{solution}[(iii)]
\end{solution}

\begin{solution}[(iv)]
\end{solution}

\begin{problem}[8.53]
  Prove that any finite group $G$ is isomorphic with a subgroup of $\GL(n, \R)$ for some $n$. [Hint: For each $\sigma \in S_n$, find a matrix in $\GL(n, \R)$ that sends each basis vector $e_i$ to $e_{\sigma(i)}$. Use this to show that $S_n$ is isomorphic with a subgroup of $\GL(n, \R)$.]
\end{problem}

\begin{solution}
\end{solution}

\begin{problem}[8.56]
\end{problem}

\begin{solution}
\end{solution}

\begin{problem}[9.43]
\end{problem}

\begin{solution}
\end{solution}

\begin{problem}[9.51]
\end{problem}

\begin{solution}
\end{solution}

\begin{problem}[10.50]
\end{problem}

\begin{solution}
\end{solution}

\begin{problem}[12.40]
\end{problem}

\begin{solution}
\end{solution}

\begin{problem}[13.41]
\end{problem}

\begin{solution}
\end{solution}
