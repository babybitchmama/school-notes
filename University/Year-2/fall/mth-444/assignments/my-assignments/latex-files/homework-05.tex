\begin{problem}[6.10]
  Find the number of generators of a cyclic group having the given order of 24.
\end{problem}

\begin{solution}
  The number of generators of a cyclic group of order $n$ is given by $\phi(n)$, where $\phi$ is the Euler's totient function. For $n = 24$, we have
  \[%
    \phi(24) = 24 \left(1 - \frac{1}{2}\right)\left(1 - \frac{1}{3}\right) = 24 \cdot \frac{1}{2} \cdot \frac{2}{3} = 8
  .\]%
  Therefore, a cyclic group of order 24 has 8 generators.
\end{solution}

\begin{problem}[6.14]
  An isomorphism of a group with itself is an automorphism of the group. Find the number of automorphisms of the group $\Z_8$.
\end{problem}

\begin{solution}
  An automorphism has to map a generator to a generator. The generators of $\Z_8$ are the elements that are coprime to 8. The integers coprime to 8 in the range from 0 to 7 are 1, 3, 5, and 7. Therefore, there are 4 generators in $\Z_8$. Take the automorphism that maps $\phi_1(1) = 3$ and another that maps $\phi_2(3) 1$, then notice that $\phi_2 = \phi_1^{-1}$. Similarly, we have $\phi_3(1) = 5$ and its inverse $\phi_4(5) = 1$. Thus, we have 4 automorphisms in total.
\end{solution}

\begin{problem}[6.20]
  Find the number of elements in the cyclic subgroup of the group $\C*$ of Exercise 19 generated by $(1 + i)/\sqrt{2}$.
\end{problem}

\begin{solution}
  Computing the powers of the element
  \begin{align*}
    x^0 &= 1 \\
    x^1 &= \frac{1 + i}{\sqrt{2}} \\
    x^2 &= \left(\frac{1 + i}{\sqrt{2}}\right)^2 = \frac{1 + 2i + i^2}{2} = \frac{2i}{2} = i \\
    x^3 &= x^2 \cdot x^1 = i \cdot \frac{1 + i}{\sqrt{2}} = \frac{i + i^2}{\sqrt{2}} = \frac{i - 1}{\sqrt{2}} \\
    x^4 &= x^2 \cdot x^2 = i \cdot i = i^2 = -1 \\
    x^5 &= x^4 \cdot x^1 = -1 \cdot \frac{1 + i}{\sqrt{2}} = \frac{-1 - i}{\sqrt{2}} \\
    x^6 &= x^4 \cdot x^2 = -1 \cdot i = -i \\
    x^7 &= x^6 \cdot x^1 = -i \cdot \frac{1 + i}{\sqrt{2}} = \frac{-i - i^2}{\sqrt{2}} = \frac{-i + 1}{\sqrt{2}} \\
    x^8 &= x^4 \cdot x^4 = -1 \cdot -1 = 1 \\
    x^9 &= x^8 \cdot x^1 = 1 \cdot \frac{1 + i}{\sqrt{2}} = \frac{1 + i}{\sqrt{2}} = x^1
  .\end{align*}
  Thus, the powers of $x$ start repeating after $x^8$. We also have the inverse powers
  \[%
    x^{-1} = x^7, \quad
    x^{-2} = x^6, \quad
    x^{-3} = x^5, \quad
    x^{-4} = x^4, \quad
    x^{-5} = x^3, \quad
    x^{-6} = x^2, \quad
    x^{-7} = x^1, \quad
    x^{-8} = x^0
  .\]%
  Therefore, the cyclic subgroup generated by $x$ has 8 elements.
  \[%
    \bra{(1 + i)/\sqrt{2}} = \left\{ 1, \frac{1 + i}{\sqrt{2}}, i, \frac{i - 1}{\sqrt{2}}, -1, \frac{-1 - i}{\sqrt{2}}, -i, \frac{1 - i}{\sqrt{2}} \right\}
  .\qedhere\]%
\end{solution}

\begin{problem}[6.22]
  Find the number of elements in the cyclic subgroup $\bra{r^{10}}$ of $D_{24}$.
\end{problem}

\begin{solution}
  The group $D_{24}$ has order 48, and the rotation $r$ has order 24. The order of the element $r^{10}$ is given by
  \[%
    \frac{24}{\gcd(24, 10)} = \frac{24}{2} = 12
  .\]%
  Therefore, the cyclic subgroup $\bra{r^{10}}$ has 12 elements.
\end{solution}

\begin{problem}[6.28]
  Find the maximum possible order for an element of $S_n$ for a given value of $n = 8$.
\end{problem}

\begin{solution}
  To find the maximum possible order of an element in the symmetric group $S_8$, we need to consider the cycle decomposition of permutations. The order of a permutation is the least common multiple (LCM) of the lengths of its disjoint cycles.
  For $n = 8$, we can consider different cycle structures and calculate their orders:
  \begin{enumerate}
    \item A single 8-cycle: $(a_1 a_2 a_3 a_4 a_5 a_6 a_7 a_8)$ has order 8.
    \item A 7-cycle and a 1-cycle: $(a_1 a_2 a_3 a_4 a_5 a_6 a_7)(a_8)$ has order 7.
    \item A 6-cycle and a 2-cycle: $(a_1 a_2 a_3 a_4 a_5 a_6)(a_7 a_8)$ has order $\text{lcm}(6, 2) = 6$.
    \item A 5-cycle and a 3-cycle: $(a_1 a_2 a_3 a_4 a_5)(a_6 a_7 a_8)$ has order $\text{lcm}(5, 3) = 15$.
    \item A 4-cycle and a 4-cycle: $(a_1 a_2 a_3 a_4)(a_5 a_6 a_7 a_8)$ has order $\text{lcm}(4, 4) = 4$.
    \item A 4-cycle, a 3-cycle, and a 1-cycle: $(a_1 a_2 a_3 a_4)(a_5 a_6 a_7)(a_8)$ has order $\text{lcm}(4, 3, 1) = 12$.
    \item A 3-cycle, a 3-cycle, and a 2-cycle: $(a_1 a_2 a_3)(a_4 a_5 a_6)(a_7 a_8)$ has order $\text{lcm}(3, 3, 2) = 6$.
    \item Two 2-cycles and a 4-cycle: $(a_1 a_2)(a_3 a_4)(a_5 a_6 a_7 a_8)$ has order $\text{lcm}(2, 2, 4) = 4$.
    \item Four 2-cycles: $(a_1 a_2)(a_3 a_4)(a_5 a_6)(a_7 a_8)$ has order $\text{lcm}(2, 2, 2, 2) = 2$.
  \end{enumerate}
  After evaluating these structures, we find that the maximum order is achieved with the cycle structure of a 5-cycle and a 3-cycle, which gives us an order of 15.
  Therefore, the maximum possible order for an element of $S_8$ is 15
\end{solution}

\begin{problem}[6.36]
  Find all orders of subgroups of the group $\Z_{12}$.
\end{problem}

\begin{solution}
  By Lagrange's theorem, the order of any subgroup of a finite group must divide the order of the group. The group $\Z_{12}$ has order 12. The divisors of 12 are 1, 2, 3, 4, 6, and 12. Therefore, the possible orders of subgroups of $\Z_{12}$ are just these divisors.
\end{solution}

\begin{problem}[6.46]
  Either give an example of a cyclic group having four generators, or explain why no example exists.
\end{problem}

\begin{solution}
  A cyclic group of order $n$ has $\phi(n)$ generators, where $\phi$ is the Euler's totient function. To have exactly four generators, we need to find an integer $n$ such that $\phi(n) = 4$. The values of $n$ for which $\phi(n) = 4$ are: $n = 5$, $n = 8$, and $n = 10$, since $\phi(5) = 4$, $\phi(8) = 4$, and $\phi(10) = 4$.

  Therefore, examples of cyclic groups having four generators include $\Z_5$, $\Z_8$, and $\Z_{10}$.
\end{solution}

\begin{problem}[6.50]
  The generators of the cyclic multiplicative group $U_n$ of all nth roots of unity in $\C$ are the primitive $n$th roots of unity. Find the primitive $n$th roots of unity for the given value of $n = 12$.
\end{problem}

\begin{solution}
  The $n$th roots of unity are given by the formula
  \[%
    e^{2\pi i k / n} \quad \text{for } k = 0, 1, 2, \ldots, n-1
  .\]%
  For $n = 12$, the 12th roots of unity are:
  \[%
    e^{2\pi i k / 12} \quad \text{for } k = 0, 1, 2, \ldots, 11
  .\]%
  The primitive $n$th roots of unity are those roots for which $k$ is coprime to $n$. The integers coprime to 12 in the range from 0 to 11 are 1, 5, 7, and 11. Therefore, the primitive 12th roots of unity are:
  \[%
    e^{2\pi i / 12}, \quad e^{10\pi i / 12}, \quad e^{14\pi i / 12}, \quad e^{22\pi i / 12}
  .\]%
  Simplifying these expressions, we get:
  \[%
    e^{\pi i / 6}, \quad e^{5\pi i / 6}, \quad e^{7\pi i / 6}, \quad e^{11\pi i / 6}
  .\]%
  Thus, the primitive 12th roots of unity are:
  \[%
    \cos(\pi/6) + i\sin(\pi/6), \quad \cos(5\pi/6) + i\sin(5\pi/6), \quad \cos(7\pi/6) + i\sin(7\pi/6), \quad \cos(11\pi/6) + i\sin(11\pi/6)
  .\qedhere\]
\end{solution}

\begin{problem}[6.53]
  Let $G$ be a cyclic group with generator $a$, and let $G'$ be a group isomorphic to $G$. If $\phi : G \to G'$ is an isomorphism, show that, for every $x \in G$, $\phi(x)$ is completely determined by the value $\phi(a)$. That is, if $\phi : G \to G'$ and $\sigma : G \to G'$ are two isomorphisms such that $\phi(a) = \psi(a)$, then $\phi(x) = \psi(x)$ for all $x \in G$.
\end{problem}

\begin{solution}
  Since $G$ is a cyclic group generated by $a$, every element $x \in G$ can be expressed as $x = a^k$ for some integer $k$. Now, consider the two isomorphisms $\phi : G \to G'$ and $\psi : G \to G'$ such that $\phi(a) = \psi(a)$.

  We want to show that $\phi(x) = \psi(x)$ for all $x \in G$. Let $x = a^k$ for some integer $k$. Then we have:
  \[%
    \phi(x) = \phi(a^k) = (\phi(a))^k
  .\]%
  Similarly,
  \[%
    \psi(x) = \psi(a^k) = (\psi(a))^k
  .\]%
  Since we are given that $\phi(a) = \psi(a)$, it follows that:
  \[%
    (\phi(a))^k = (\psi(a))^k
  .\]%
  Therefore, we have:
  \[%
    \phi(x) = \psi(x)
  .\]%
  This shows that for every element $x \in G$, the value of $\phi(x)$ is completely determined by the value of $\phi(a)$, and thus $\phi(x) = \psi(x)$ for all $x \in G$.
\end{solution}

\begin{problem}[6.56]
  Let $a$ and $b$ be elements of a group $G$. Show that if $ab$ has finite order $n$, then $ba$ also has order $n$.
\end{problem}

\begin{solution}
  Let the order of the element $ab$ be $n$. This means that
  \[%
    (ab)^n = e
  .\]%
  We want to show that $(ba)^n = e$ as well. We can compute $(ba)^n$ as follows:
  \begin{align*}
    (ba)^n &= b(ab)^{n-1}a \\
            &= b e a \quad \text{(since } (ab)^n = e \text{)} \\
            &= ba
  .\end{align*}
  However, this does not directly show that $(ba)^n = e$. Instead, we can use the fact that conjugation preserves order. Specifically, we can write:
  \[%
    (ba)^n = b(ab)^n b^{-1} = b e b^{-1} = e
  .\]%
  Thus, we have shown that $(ba)^n = e$, which means that the order of $ba$ is also $n$.
\end{solution}
