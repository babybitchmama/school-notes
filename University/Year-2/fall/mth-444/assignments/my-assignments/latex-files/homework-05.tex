\begin{problem}[6.10]
  Find the number of generators of a cyclic group having the given order of 24.
\end{problem}

\begin{solution}
  The number of generators of a cyclic group of order $n$ is given by $\phi(n)$, where $\phi$ is the Euler's totient function. For $n = 24$, we have
  \[%
    \phi(24) = 24 \left(1 - \frac{1}{2}\right)\left(1 - \frac{1}{3}\right) = 24 \cdot \frac{1}{2} \cdot \frac{2}{3} = 8
  .\]%
  Therefore, a cyclic group of order 24 has 8 generators.
\end{solution}

\begin{problem}[6.14]
  An isomorphism of a group with itself is an automorphism of the group. Find the number of automorphisms of the group $\Z_8$.
\end{problem}

\begin{solution}
  Any automorphism of $\Z_8$ is completely determined by the image of $1$, and this image must be a generator of $\Z_8$. The generators of $\Z_8$ are precisely the elements that are coprime to $8$, namely $1, 3, 5,$ and $7$. Therefore, there are $4$ possible choices for $\phi(1)$, each of which defines a distinct automorphism. Hence,
  \[%
    |\Aut(\Z_8)| = \phi(8) = 4
  .\]%
  Equivalently, we have $\Aut(\Z_8) \cong U(8) = \{1, 3, 5, 7\}$ under multiplication mod $8$.
\end{solution}

\begin{problem}[6.20]
  Find the number of elements in the cyclic subgroup of the group $\C^*$ of Exercise~19 generated by $(1 + i)/\sqrt{2}$.
\end{problem}

\begin{solution}
  Let $x = \frac{1 + i}{\sqrt{2}}$. Computing successive powers, we obtain
  \begin{gather*}
    x^0 = 1, \quad x^1 = \frac{1 + i}{\sqrt{2}}, \quad x^2 = \left(\frac{1 + i}{\sqrt{2}}\right)^2 = i, \\
    x^3 = \frac{i - 1}{\sqrt{2}}, \quad x^4 = -1, \quad x^5 = \frac{-1 - i}{\sqrt{2}}, \\
    x^6 = -i, \quad x^7 = \frac{1 - i}{\sqrt{2}}, \quad x^8 = 1.
  \end{gather*}
  Since $x^8 = 1$ and no smaller positive power equals $1$, the element $x$ has order $8$. Therefore, the cyclic subgroup generated by $x$ has $8$ elements, giving us
  \[%
    \langle (1 + i)/\sqrt{2} \rangle = \left\{
      1,
      \frac{1 + i}{\sqrt{2}},
      i,
      \frac{i - 1}{\sqrt{2}},
      -1,
      \frac{-1 - i}{\sqrt{2}},
      -i,
      \frac{1 - i}{\sqrt{2}}
    \right\}
  .\]%
  Equivalently, note that $\tfrac{1 + i}{\sqrt{2}} = e^{i\pi/4}$, so this subgroup consists of the eighth roots of unity in~$\C^*$.
\end{solution}

\begin{problem}[6.22]
  Find the number of elements in the cyclic subgroup $\bra{r^{10}}$ of $D_{24}$.
\end{problem}

\begin{solution}
  The dihedral group $D_{24}$ has order $48$, and its rotation element $r$ has order $24$. The order of the element $r^{10}$ is given by
  \[%
    \ord(r^{10}) = \frac{24}{\gcd(24, 10)} = \frac{24}{2} = 12
  .\]%
  Therefore, the cyclic subgroup $\langle r^{10} \rangle$ has order $12$.
\end{solution}

\begin{problem}[6.28]
  Find the maximum possible order for an element of $S_n$ for a given value of $n = 8$.
\end{problem}

\begin{solution}
  The order of a permutation in $S_n$ is equal to the least common multiple (LCM) of the lengths of its disjoint cycles. Thus, to find the maximum possible order for an element of $S_8$, we must determine the partition of $8$ whose cycle lengths yield the largest LCM. Considering the possible decompositions, we have:
  \begin{align*}
    (8) &\implies \text{order}~8 \\
    (7, 1) &\implies \text{order}~7 \\
    (6, 2) &\implies \text{order}~\lcm(6, 2) = 6 \\
    (5, 3) &\implies \text{order}~\lcm(5, 3) = 15 \\
    (4, 3, 1) &\implies \text{order}~\lcm(4, 3, 1) = 12 \\
    (4, 2, 2) &\implies \text{order}~\lcm(4, 2, 2) = 4
  .\end{align*}
  Among these, the maximum occurs for the cycle structure $(5,3)$, giving an order of $15$. Therefore, the maximum possible order of an element in $S_8$ is $15$.
\end{solution}

\begin{problem}[6.36]
  Find all orders of subgroups of the group $\Z_{12}$.
\end{problem}

\begin{solution}
  The group $\Z_{12}$ is cyclic of order $12$. By Lagrange's theorem, the order of any subgroup of a finite group must divide the order of the group. Hence, the possible orders of subgroups of $\Z_{12}$ are the positive divisors of $12$, namely 1, 2, 3, 4, 6, and 12. Moreover, since $\Z_{12}$ is cyclic, there is exactly one subgroup of each of these orders, namely, $\bra{0}$, $\bra{2}$, $\bra{3}$, $\bra{4}$, $\bra{6}$, and $\bra{12} = \Z_{12}$ itself.
\end{solution}

\begin{problem}[6.46]
  Either give an example of a cyclic group having four generators, or explain why no example exists.
\end{problem}

\begin{solution}
  A cyclic group of order $n$ has $\phi(n)$ generators, where $\phi$ denotes the Euler totient function. To have exactly four generators, we must find all integers $n$ such that $\phi(n) = 4$. Computing, we find
  \[%
    \phi(5) = 4, \quad \phi(8) = 4, \quad \phi(10) = 4,~\text{and}~\phi(12) = 4
  .\]%
  Therefore, examples of cyclic groups with four generators include
  \[%
    \Z_5, \quad \Z_8, \quad \Z_{10},~\text{and}~\Z_{12}
  .\]%
  Each of these cyclic groups has exactly four generators.
\end{solution}

\begin{problem}[6.50]
  The generators of the cyclic multiplicative group $U_n$ of all nth roots of unity in $\C$ are the primitive $n$th roots of unity. Find the primitive $n$th roots of unity for the given value of $n = 12$.
\end{problem}

\begin{solution}
  The $n$th roots of unity are given by $e^{2\pi ik/n}$, where $k = 0, 1, 2, \cdots, n - 1$ For $n = 12$, these are
  \[%
    e^{2\pi ik/12}
  .\]%
  The primitive 12th roots of unity correspond to the integers $k$ that are coprime to $12$. Since $\gcd(k, 12) = 1$, for $k = 1, 5, 7, 11$. The primitive 12th roots of unity are
  \[%
    e^{2\pi i/12}, \quad e^{10\pi i/12}, \quad e^{14\pi i/12}, \quad e^{22\pi i/12},
  .\]%
  Hence, the primitive 12th roots of unity are
  \[%
    \cos\left(\frac{\pi}{6}\right) + i\sin\left(\frac{\pi}{6}\right),~\cos\left(\frac{5\pi}{6}\right) + i\sin\left(\frac{5\pi}{6}\right),~\cos\left(\frac{7\pi}{6}\right) + i\sin\left(\frac{7\pi}{6}\right),~\cos\left(\frac{11\pi}{6}\right) + i\sin\left(\frac{11\pi}{6}\right)
  .\qedhere\]%
\end{solution}

\begin{problem}[6.53]
  Let $G$ be a cyclic group with generator $a$, and let $G'$ be a group isomorphic to $G$. If $\phi : G \to G'$ is an isomorphism, show that, for every $x \in G$, $\phi(x)$ is completely determined by the value $\phi(a)$. That is, if $\phi : G \to G'$ and $\sigma : G \to G'$ are two isomorphisms such that $\phi(a) = \psi(a)$, then $\phi(x) = \psi(x)$ for all $x \in G$.
\end{problem}

\begin{solution}
  Since $G$ is cyclic and generated by $a$, every element $x \in G$ can be written as $x = a^k$ for some integer $k$. Let $\phi, \sigma : G \to G'$ be isomorphisms such that $\phi(a) = \sigma(a)$. Because isomorphisms preserve the group operation, for all integers $k$ we have
  \[%
    \phi(a^k) = (\phi(a))^k~\text{and}~\sigma(a^k) = (\sigma(a))^k
  .\]%
  Hence, for $x = a^k$,
  \[%
    \phi(x) = (\phi(a))^k = (\sigma(a))^k = \sigma(x)
  .\]%
  Therefore, $\phi(x) = \sigma(x)$ for all $x \in G$, showing that $\phi(x)$ is completely determined by the value of $\phi(a)$.
\end{solution}

\begin{problem}[6.56]
  Let $a$ and $b$ be elements of a group $G$. Show that if $ab$ has finite order $n$, then $ba$ also has order $n$.
\end{problem}

\begin{solution}
  Suppose the element $ab$ has order $n$, so that $(ab)^n = e$. Notice that $ba = b(ab)b^{-1}$. Thus, $ba$ is conjugate to $ab$ by $b$. Since conjugate elements in a group have the same order, it follows that $\operatorname{ord}(ba) = \operatorname{ord}(ab) = n$.

  For completeness, we can verify this directly. We have
  \[%
    (ba)^n = b(ab)^n b^{-1} = b e b^{-1} = e
  .\]%
  Therefore, $ba$ also has order $n$.
\end{solution}
