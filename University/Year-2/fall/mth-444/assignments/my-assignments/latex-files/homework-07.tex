\begin{problem}[8.39]
  Show that if $\phi : G \to G'$ and $\gamma : G' \to G''$ are group homomorphisms, then $\gamma \circ \phi : G \to G''$ is also a group homomorphism.
\end{problem}

\begin{solution}
  Let $\phi : G \to G'$ and $\gamma : G' \to G''$ be group homomorphisms. We want to show that the composition $\gamma \circ \phi : G \to G''$ is also a group homomorphism. Take any two elements $a, b \in G$. Then, we have
  \[%
    (\gamma \circ \phi)(ab) = \gamma(\phi(ab)) = \gamma(\phi(a) \phi(b))
  ,\]%
  since $\phi$ is a group homomorphism. Since $\gamma$ is also a group homomorphism, we have
  \[%
    \gamma(\phi(a) \phi(b)) = \gamma(\phi(a)) \circ \gamma(\phi(b)) = (\gamma \circ \phi)(a) \circ (\gamma \circ \phi)(b)
  .\]%
  Thus, $\gamma \circ \phi : G \to G''$ is a group homomorphism.
\end{solution}

\begin{problem}[8.41]
  Prove the following about $S_n$ if $n \ge 3$.
  \begin{enumerate}
    \item Every permutation in $S_n$ can be written as a product of at most $n - 1$ transpositions.
  \end{enumerate}
\end{problem}

\begin{solution}[(i)]
  Let $\sigma \in S_n$ and write $\sigma$ as a product of disjoint cycles (including $1$-cycles for fixed points):
  \[%
    \sigma = \gamma_1 \gamma_2 \cdots \gamma_r
  ,\]%
  where $\gamma_i$ has length $m_i\ge1$ and $\sum_{i=1}^r m_i = n$.

  A $k$-cycle can be written as a product of $k-1$ transpositions; for distinct elements $a_1,\dots,a_k$,
  \[%
    (a_1, a_2, \dots, a_k) = (a_1, a_k)(a_1, a_{k-1}) \cdots (a_1, a_2)
  ,\]%
  which uses exactly $k - 1$ transpositions (and a 1-cycle uses 0).

  Applying this to each $\gamma_i$ we express $\sigma$ as a product of
  \[%
    \sum_{i=1}^r (m_i - 1) = \left(\sum_{i=1}^r m_i\right) - r = n - r
  ,\]%
  transpositions. Since $r \ge 1$, we have $n - r \le n - 1$, so $\sigma$ is a product of at most $n - 1$ transpositions.

  Notice that we get $n - 1$ transpositions when $\sigma$ is a single $n$-cycle, but we can never exceed $n - 1$ transpositions for any permutation in $S_n$. Thus, every permutation in $S_n$ can be written as a product of at most $n - 1$ transpositions.
\end{solution}

\begin{problem}[8.43]
  Show that for every subgroup $H$ of $S_n$ for $n \ge 2$, either all the permutations in $H$ are even or exactly half of them are even.
\end{problem}

\begin{solution}
  Let $H$ be a subgroup of $S_n$ for $n \ge 2$. Let $\sigma_1$ and $\sigma_2$ be two even permutations and $\tau_1$ and $\tau_2$ be two odd permutations in $H$. We note that
  \begin{align*}
    \sgn(\sigma_1 \sigma_2) &= \sgn(\sigma_1) \sgn(\sigma_2) = (1)(1) = 1, \\
    \sgn(\tau_1 \tau_2) &= \sgn(\tau_1) \sgn(\tau_2) = (-1)(-1) = 1, \\
    \sgn(\sigma_1 \tau_1) &= \sgn(\sigma_1) \sgn(\tau_1) = (1)(-1) = -1
  .\end{align*}
  This comes from the fact that $\sgn$ is a group homomorphism from $S_n$ to $\{1, -1\}$. From the above calculations, we see that the product of two even permutations is even, the product of two odd permutations is even, and the product of an even permutation and an odd permutation is odd.

  Since $H$ is closed under composition, if there exists an odd function, there must exist an even function in $H$ as well. So, $H$ cannot just consist of only odd functions, but it can consist of only even functions.

  Let $m$ be the number of even permutations in $H$ and $k$ be the number of odd permutations in $H$. Taking all the odd functions, $\tau_1, \tau_2, \ldots, \tau_k$, we can create $k$ even functions by composing each odd function with a fixed odd function $\tau_i$. Therefore, we have $m \ge k$.

  Similarly, taking all the even functions, $\sigma_1, \sigma_2, \ldots, \sigma_m$, we can create $m$ odd functions by composing each even function with a fixed odd function $\tau_j$. Therefore, we have $k \ge m$.

  Therefore, we conclude that $m = k$. In summary, either all the permutations in $H$ are even, or exactly half of them are even.
\end{solution}

\begin{problem}[9.6]
  Find the order of the given element of the direct product $(3, 10, 9)$ in $\Z_4 \times \Z_{12} \times \Z_{15}$.
\end{problem}

\begin{solution}
  The order of $3$ in $\Z_4$ is 4, since $3 \times 4 \equiv 0 \pmod{4}$. The order of $10$ in $\Z_{12}$ is 6, since $10 \times 6 \equiv 0 \pmod{12}$. The order of $9$ in $\Z_{15}$ is 5, since $9 \times 5 \equiv 0 \pmod{15}$.

  To find the order of the element $(3, 10, 9)$ in the direct product $\Z_4 \times \Z_{12} \times \Z_{15}$, we take the least common multiple (LCM) of the individual orders to get $\ord(3, 10, 9) = \lcm(4, 6, 5) = 60$. Therefore, the order of the element $(3, 10, 9)$ in $\Z_4 \times \Z_{12} \times \Z_{15}$ is 60.
\end{solution}

\begin{problem}[9.8]
  What is the largest order among the orders of all the cyclic subgroups of $\Z_6 \times \Z_8$ and $\Z_{12} \times \Z_{15}$?
\end{problem}

\begin{solution}
  For $\Z_6 \times \Z_8$, the largest order among all the orders of all cyclic subgroups is given by
  \[%
    \max \ord(\Z_6 \times \Z_8) = \lcm(6, 8) = 24
  .\]%

  For $\Z_{12} \times \Z_{15}$, the same thing applies, giving us
  \[%
    \max \ord(\Z_{12} \times \Z_{15}) = \lcm(12, 15) = 60
  .\]%

  Therefore, the largest order among the orders of all the cyclic subgroups of $\Z_6 \times \Z_8$ is 24, and for $\Z_{12} \times \Z_{15}$ it is 60.
\end{solution}

\begin{problem}[9.18]
  Are the groups $\Z_8 \times \Z_{10} \times \Z_{24}$ and $\Z_4 \times \Z_{12} \times \Z_{40}$ isomorphic? Why or why not?
\end{problem}

\begin{solution}
  Breaking down each group into its primary components, we have
  \begin{gather*}
    G = \Z_8 \times \Z_{10} \times \Z_{24} \cong \Z_2 \times \Z_3 \times \Z_5 \times \Z_8 \times \Z_8 \\
    H = \Z_4 \times \Z_{12} \times \Z_{40} \cong \Z_3 \times \Z_4 \times \Z_4 \times \Z_5 \times \Z_8
  .\end{gather*}
  The primary components of $G$ and $H$ are given as
  \begin{gather*}
    G_{(3)} \cong \Z_3 \cong H_{(3)}, \quad G_{(5)} \cong \Z_5 \cong H_{(5)} \\
    G_{(2)} \cong \Z_2 \times \Z_8 \times \Z_8 \not\cong \Z_4 \times \Z_4 \times \Z_8 \cong H_{(2)}
  .\end{gather*}
  Since the primary components corresponding to the prime 2 are not isomorphic, we conclude that the groups $G$ and $H$ are not isomorphic.
\end{solution}

\begin{problem}[9.22]
  Find all abelian groups, up to isomorphism, of order 16. Find the invariant factors and find an isomorphic group of the form indicated in Theorem 9.14.
\end{problem}

\begin{solution}
  Factoring 16 into its prime power components, we have $16 = 2^4$. The abelian groups of order 16, up to isomorphism, are given by the following decompositions, $4$, $3 + 1$, $2 + 2$, $2 + 1 + 1$, and $1 + 1 + 1 + 1$. Thus, the abelian groups of order 16 along with their invariant factors are
  \begin{align*}
    \Z_{16} &: (16) \\
    \Z_2 \times \Z_8 &: (8, 2) \\
    \Z_4 \times \Z_4 &: (4, 4) \\
    \Z_2 \times \Z_2 \times \Z_4 &: (4, 2, 2) \\
    \Z_2 \times \Z_2 \times \Z_2 \times \Z_2 &: (2, 2, 2, 2)
  .\qedhere\end{align*}
\end{solution}

\begin{problem}[9.26]
  How many abelian groups (up to isomorphism) are there of order 24, order 25, and of order $(24)(25)$?
\end{problem}

\begin{solution}
  For 24, we have the prime factorization $24 = 2^3 \times 3^1$. The number of abelian groups of order $2^3$ is given by the number of partitions of 3, which are $3$, $2 + 1$, and $1 + 1 + 1$. Thus, there are 3 abelian groups of order $2^3$. For $3^1$, there is only 1 abelian group. Therefore, the total number of abelian groups of order 24 is $3 \times 1 = 3$. They are
  \[%
    \Z_3 \times \Z_8, \quad \Z_2 \times \Z_3 \times \Z_4, \quad \Z_2 \times \Z_2 \times \Z_2 \times \Z_3
  .\]%

  For 25, we have the prime factorization $25 = 5^2$. The number of abelian groups of order $5^2$ is given by the number of partitions of 2, which are $2$ and $1 + 1$. Thus, there are 2 abelian groups of order 25. They are
  \[%
    \Z_5, \quad \Z_5 \times \Z_{25}
  .\]%

  For $(24)(25) = 600$, we have the prime factorization $600 = 2^3 \times 3^1 \times 5^2$. From the previous calculations, we know there are 3 abelian groups of order $2^3$, 1 abelian group of order $3^1$, and 2 abelian groups of order $5^2$. Therefore, the total number of abelian groups of order 600 is $3 \times 1 \times 2 = 6$. They are
  \begin{gather*}
    \Z_3 \times \Z_8 \times \Z_{25}, \quad \Z_3 \times \Z_5 \times \Z_5 \times \Z_8, \\
    \Z_2 \times \Z_3 \times \Z_4 \times \Z_{25}, \quad \Z_2 \times \Z_3 \times \Z_4 \times \Z_5 \times \Z_5, \\
    \Z_2 \times \Z_2 \times \Z_2 \times \Z_3 \times \Z_{25}, \quad \Z_2 \times \Z_2 \times \Z_2 \times \Z_3 \times \Z_5 \times \Z_5
  .\qedhere\end{gather*}
\end{solution}

\begin{problem}[9.28]
  Use Exercise 27 to determine the number of abelian groups (up to isomorphism) of order $(10)^5$.
\end{problem}

\begin{solution}
  Exercise 27 states that if we have $r$ abelian groups of order $m$ and $s$ abelian groups of order $n$, then there are $rs$ abelian groups of order $mn$, provided that $m$ and $n$ are coprime. Notice that $10 = 2 \times 5$. Then, we have $10^5 = 2^5 \times 5^5$. The number of abelian groups of order $2^5$ is given by the number of partitions of 5, which are $7$ in total. Similarly, the number of abelian groups of order $5^5$ is also given by the number of partitions of 5, which is again $7$. Therefore, by Exercise 27, the total number of abelian groups of order $(10)^5$ is $7 \times 7 = 49$.
\end{solution}

\begin{problem}[9.39]
  Let $G$ be an abelian group. Show that the elements of finite order in $G$ form a subgroup. This subgroup is called the \emph{torsion subgroup} of $G$.
\end{problem}

\begin{solution}
  Let $H$ be the set of all elements of finite order in the abelian group $G$. Notice that $e^1 = e$, so the identity element $e$ of $G$ has finite order and is in $H$.

  Next, take any two elements $a, b \in H$. Suppose $a$ has order $m$ and $b$ has order $n$. This means $a^m = e$ and $b^n = e$. Now, consider the element $a \cdot b$. Take the exponent $mn$ and using the fact that $G$ is abelian, we have
  \[%
    (a \cdot b)^{mn} = a^{mn} \cdot b^{mn} = (a^m)^n \cdot (b^n)^m = e \cdot e = e
  .\]%
  Thus, the order of $a \cdot b$ divides $mn$, which is finite. Therefore, $a \cdot b \in H$.

  Finally, we need to show that if $a \in H$, then its inverse $a^{-1}$ is also in $H$. If $a$ has finite order $m$, then $a^m = e$. Taking inverses on both sides gives $(a^{-1})^m = e$. Thus, the order of $a^{-1}$ is also finite, so $a^{-1} \in H$.

  Since we have shown that the identity element is in $H$, that the product of any two elements in $H$ is also in $H$, and that the inverse of any element in $H$ is also in $H$, we conclude that $H$ is a subgroup of $G$. This subgroup is called the torsion subgroup of $G$.
\end{solution}
