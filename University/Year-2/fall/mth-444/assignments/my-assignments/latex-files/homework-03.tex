\begin{problem}[2.44]
  Prove that if $f : G_1 \to G_2$ is a group isomorphism from the group $\bra{G_1, *_1}$ to the group $\bra{G_2, *_2}$, then $f^{-1} : G_2 \to G_1$ is a group isomorphism from $\bra{G_2, *_2}$ to $\bra{G_1, *_1}$.
\end{problem}

\begin{solution}
  Given two groups $\bra{G_1, *_1}$ and $\bra{G_2, *_2}$, assume we have an isomorphism $f : G_1 \to G_2$. By definition of isomorphism, $f$ is a bijection and satisfies the homomorphism property, i.e., for every $a, b \in G_1$, we have
  \[%
    f(a *_1 b) = f(a) *_2 f(b)
  .\]%
  Since $f$ is a bijection, it has an inverse function $f^{-1} : G_2 \to G_1$. We need to show that $f^{-1}$ is also a homomorphism. For any $x, y \in G_2$, let $a = f^{-1}(x)$ and $b = f^{-1}(y)$. Then we have
  \[%
    f^{-1}(x *_2 y) = f^{-1}(f(a) *_2 f(b)) = f^{-1}(f(a *_1 b)) = a *_1 b = f^{-1}(x) *_1 f^{-1}(y)
  .\]%
  Thus, $f^{-1}$ satisfies the homomorphism property. Since $f^{-1}$ is also a bijection, we conclude that $f^{-1}$ is an isomorphism from $\bra{G_2, *_2}$ to $\bra{G_1, *_1}$.
\end{solution}

\begin{problem}[3.30]
  Find \emph{all} solutions $x$ of the given equation: $x +_{2\pi} \pi = \pi/2$ in $\R_{2\pi}$.
\end{problem}

\begin{solution}
\end{solution}

\begin{problem}[3.32]
  Find \emph{all} solutions $x$ of the given equation: $x +_{13} x +_{13} x = 5$ in $\Z_{13}$.
\end{problem}

\begin{solution}
\end{solution}

\begin{problem}[3.35]
  Prove or give a counterexample to the statement that for any $n \in \Z^+$ and $a \in \Z_n$, the equation $x +_n x = a$ has at most two solutions in $\Z_n$.
\end{problem}

\begin{solution}
\end{solution}

\begin{problem}[3.38]
  There is an isomorphism of $U_7$ with $\Z_7$ in which $\zeta = e^{i(2\pi/7)} \leftrightarrow 4$. Find the element in $\Z_7$ to which $\zeta^m$ must correspond for $m = 0, 2, 3, 4, 5$ and 6.
\end{problem}

\begin{solution}
\end{solution}

\begin{problem}[4.10]
  Convert the permutations $\sigma$, $\tau$, and $\mu$ defined prior to Exercise 1 to disjoint cycle notation.
\end{problem}

\begin{solution}
\end{solution}

\begin{problem}[4.12]
  Compute the permutation products.
  \begin{enumerate}
    \item $(1, 5, 2, 4)(1, 5, 2, 3)$.

    \item $(1, 5, 3)(1, 2, 3, 4, 5, 6)(1, 5, 3)^{-1}$.

    \item $[(1, 6, 7, 2)^2(4, 5, 2, 6)^{-1}(1, 7, 3)]^{-1}$.

    \item $(1, 6)(1, 5)(1, 4)(1, 3)(1, 2)$.
  \end{enumerate}
\end{problem}

\begin{solution}[(i)]
  Let $\alpha = (1\, 5\, 2\, 4)$ and $\beta = (1\, 5\, 2\, 3)$. Computing $\alpha\beta$ on the points $1, \cdots, 5$, we get
  \begin{align*}
    \alpha\beta(1) &= \alpha(\beta(1)) = \alpha(5) = 2 \\
    \alpha\beta(2) &= \alpha(\beta(2)) = \alpha(3) = 3 \\
    \alpha\beta(3) &= \alpha(\beta(3)) = \alpha(1) = 5 \\
    \alpha\beta(5) &= \alpha(\beta(5)) = \alpha(2) = 4 \\
    \alpha\beta(4) &= \alpha(\beta(4)) = \alpha(4) = 1
  .\end{align*}
  Tracing the orbit of $1$ yields $1 \mapsto 2 \mapsto 3 \mapsto 5 \mapsto 4 \mapsto 1$, so in disjoint cycle notation $\alpha\beta = (1\, 2\, 3\, 5\, 4)$.
\end{solution}

\begin{solution}[(ii)]
  Let $\alpha = (1\, 5\, 3)$ and $\beta = (1\, 2\, 3\, 4\, 5\, 6)$. Computing $\delta := \alpha\beta\alpha^{-1}$ on the points $1, \cdots, 6$, we get
  \begin{align*}
    \delta(1) &= \alpha(\beta(\alpha^{-1}(1))) = \alpha(\beta(3)) = \alpha(4) = 4 \\
    \delta(2) &= \alpha(\beta(\alpha^{-1}(2))) = \alpha(\beta(2)) = \alpha(3) = 1 \\
    \delta(3) &= \alpha(\beta(\alpha^{-1}(3))) = \alpha(\beta(5)) = \alpha(6) = 6 \\
    \delta(4) &= \alpha(\beta(\alpha^{-1}(4))) = \alpha(\beta(4)) = \alpha(5) = 3 \\
    \delta(5) &= \alpha(\beta(\alpha^{-1}(5))) = \alpha(\beta(1)) = \alpha(2) = 2 \\
    \delta(6) &= \alpha(\beta(\alpha^{-1}(6))) = \alpha(\beta(6)) = \alpha(1) = 5
  .\end{align*}
  Tracing the orbit of 1, we find $1 \mapsto 4 \mapsto 3 \mapsto 6 \mapsto 5 \mapsto 2$. Thus, in disjoint cycle notation, we have $\alpha\beta\alpha^{-1} = (1\, 4\, 3\, 6\, 5\, 2)$.
\end{solution}

\begin{solution}[(iii)]
  Let $\alpha = (1\, 6\, 7\, 2)$, $\beta = (4\, 5\, 2\, 6)$, and $\gamma = (1\, 7\, 3)$. Then $\alpha^2 = (1\, 7)(6\, 2)$ and $\beta^{-1} = (6\, 2\, 5\, 4)$. Computing $\delta := \alpha^2\beta^{-1}\gamma$ on the points $1, \cdots, 7$, we get
  \begin{align*}
    \delta(1) &= \alpha^2\big(\beta^{-1}(\gamma(1))\big) = \alpha^2(\beta^{-1}(7)) = \alpha^2(7) = 1 \\
    \delta(2) &= \alpha^2\big(\beta^{-1}(\gamma(2))\big) = \alpha^2(\beta^{-1}(2)) = \alpha^2(5) = 5 \\
    \delta(3) &= \alpha^2\big(\beta^{-1}(\gamma(3))\big) = \alpha^2(\beta^{-1}(1)) = \alpha^2(1) = 7 \\
    \delta(4) &= \alpha^2\big(\beta^{-1}(\gamma(4))\big) = \alpha^2(\beta^{-1}(4)) = \alpha^2(6) = 2 \\
    \delta(5) &= \alpha^2\big(\beta^{-1}(\gamma(5))\big) = \alpha^2(\beta^{-1}(5)) = \alpha^2(4) = 4 \\
    \delta(6) &= \alpha^2\big(\beta^{-1}(\gamma(6))\big) = \alpha^2(\beta^{-1}(6)) = \alpha^2(2) = 6 \\
    \delta(7) &= \alpha^2\big(\beta^{-1}(\gamma(7))\big) = \alpha^2(\beta^{-1}(3)) = \alpha^2(3) = 3
  .\end{align*}
  Thus the nontrivial orbits are $2 \mapsto 5 \mapsto 4 \mapsto 2$ and $3 \mapsto 7 \mapsto 3$, and 1 and 6 are fixed points. Hence, we have
  \[%
    \delta = (2\, 5\, 4)(3\, 7)
  .\]%
  Taking inverses (reverse each cycle), we get $\delta^{-1} = (4\, 5\, 2)(7\, 3)$.
\end{solution}

\begin{solution}[(iv)]
  Let $\alpha = (1\, 6)$, $\beta = (1\, 5)$, $\gamma = (1\, 4)$, $\delta = (1\, 3)$, and $\epsilon = (1\, 2)$. Computing the product $\zeta := \alpha\beta\gamma\delta\epsilon$ on the points $1, \cdots, 6$, we get
  \begin{align*}
    \zeta(1) &= \alpha(\beta(\gamma(\delta(\epsilon(1))))) = \alpha(\beta(\gamma(\delta(2)))) = \alpha(\beta(\gamma(2))) = \alpha(\beta(2)) = \alpha(2) = 2 \\
    \zeta(2) &= \alpha(\beta(\gamma(\delta(\epsilon(2))))) = \alpha(\beta(\gamma(\delta(1)))) = \alpha(\beta(\gamma(3))) = \alpha(\beta(3)) = \alpha(3) = 3 \\
    \zeta(3) &= \alpha(\beta(\gamma(\delta(\epsilon(3))))) = \alpha(\beta(\gamma(\delta(3)))) = \alpha(\beta(\gamma(1))) = \alpha(\beta(4)) = \alpha(4) = 4 \\
    \zeta(4) &= \alpha(\beta(\gamma(\delta(\epsilon(4))))) = \alpha(\beta(\gamma(\delta(4)))) = \alpha(\beta(\gamma(4))) = \alpha(\beta(1)) = \alpha(5) = 5 \\
    \zeta(5) &= \alpha(\beta(\gamma(\delta(\epsilon(5))))) = \alpha(\beta(\gamma(\delta(5)))) = \alpha(\beta(\gamma(5))) = \alpha(\beta(5)) = \alpha(1) = 6 \\
    \zeta(6) &= \alpha(\beta(\gamma(\delta(\epsilon(6))))) = \alpha(\beta(\gamma(\delta(6)))) = \alpha(\beta(\gamma(6))) = \alpha(\beta(6)) = \alpha(6) = 1
  .\end{align*}
  Thus, tracing the orbit of 1, we find $1 \mapsto 2 \mapsto 3 \mapsto 4 \mapsto 5 \mapsto 6 \mapsto 1$. Hence, we have $\zeta = (1\, 2\, 3\, 4\, 5\, 6)$.
\end{solution}

\begin{problem}[4.14]
  Write the group table for $D_3$. Compare the group tables for $D_3$ and $S_5$. Are the groups isomorphic?

  Let $A$ be a set and let $\sigma \in S_A$. For a fixed $a \in A$, the set
  \[%
    \mathcal{O}_{a,\sigma} = \{\sigma^n(a) \mid n \in \Z\}
  ,\]%
  is the \emph{orbit} of a under $\sigma$. In Exercises 15 through 17, find the orbit of 1 under the permutation defined prior to Exercise 1.
\end{problem}

\begin{solution}
\end{solution}

\begin{problem}[4.32]
  Strengthening Exercise 31, show that if $n \ge 3$, then the only element of $\sigma$ of $S_n$ satisfying $\sigma\gamma = \gamma\sigma$ for all $y \in S_n$ is $\sigma = \iota$, the identity permutation.
\end{problem}

\begin{solution}
\end{solution}

\begin{problem}[4.36]
  Prove that for any integer $n \ge 2$, there are at least two non-isomorphic groups with exactly $2n$ elements.
\end{problem}

\begin{solution}
\end{solution}
