\begin{problem}[1]
  Assume that $X$ is Hausdorff. The purpose of this problem is to prove that $Y$ is Hausdorff, as well.
  \begin{enumerate}
    \item Show that, given any natural number n and any n-tuple of pairwise distinct points $p_1, \cdots, p_n \in X$, we can find pairwise disjoint open sets $U_1, \cdots, U_n$ with $p_i \in U_i$ for all $i$. (Note that the $n = 2$ case is the definition of Hausdorffness. For larger $n$, use induction Note also that this part of the problem does not involve the map $f$ or the space $Y$.)
    \item Let $q_1$ and $q_2$ be two distinct points in $Y$. Choose points $p_1, p_2 \in X$ with $\pi(p_i) = p_i$. In particular, this implies that the four points $p_1, f(p_1) p_2, f(p_2)$ are pairwise disjoint. By part (i), we can find pairwise disjoint open sets $U_1, U_2, V_1$, and $V_2$ with $p_1 \in U_1$, $p_2 \in U_2$, $f(p_1) \in V_1$, and $f(p_2) \in V_2$. Show that we can make these choices in such a way that $V_1 = f(U_1)$ and $V_2 = f(U_2)$.
    \item Show that $Y$ is Hausdorff.
  \end{enumerate}
\end{problem}

\begin{solution}[(i)]
\end{solution}

\begin{solution}[(ii)]
\end{solution}

\begin{solution}[(iii)]
\end{solution}

\begin{problem}[2]
  Assume that $X$ is a surface. The purpose of this problem is to prove that Y is a surface, as well. We have already established that, if $X$ is Hausdorff, so is $Y$. So we just need to show that, if every point in $X$ has an open neighborhood homeomorphic to an open subset of $\R^2$, then the same is true of $Y$.
  \begin{enumerate}
    \item Show that the identification map $\pi$ is open.
    \item Show that, if $U \subset X$ is an open set and $U \cap f(U) = \emptyset$, then $\pi : U \to \pi(U)$ is a homeomorphism.
    \item Show that, if $p \in X$, there exists an open neighborhood $U$ of $p$ such that $U$ is homeomorphic to an open subset of $\R^2$ and $U \cap f(U) = \emptyset$.
    \item Show that $Y$ is a surface.
  \end{enumerate}
  For the remaining problems, suppose that $X = |K|$ is a combinatorial surface and $f = |\phi|$ for some isomorphism $\phi : K \to K$.
\end{problem}

\begin{solution}[(i)]
\end{solution}

\begin{solution}[(ii)]
\end{solution}

\begin{solution}[(iii)]
\end{solution}

\begin{solution}[(iv)]
\end{solution}

\begin{problem}[3]
  It's tempting to try to define a triangulation of $Y$ whose simplices are in bijection with equivalence classes of simplices of $K$. Find an example that shows that such a simplicial complex might not exist! (If one subdivides $K$ first, then it works, but you don't have to show that.)
\end{problem}

\begin{solution}
\end{solution}

\begin{problem}[4]
  Suppose that $K$ is equipped with an orientation. We say that $\phi$ is \textit{orientation preserving} if it takes positively oriented triangles to positively oriented triangles. That is, if $v_1$, $v_2$, $v_3$ span a triangle in $K$ and appear in positive cyclic order, then $\phi(v_1)$, $\phi(v_2)$, $\phi(v_3)$ also appear in positive cyclic order. Give an example that is orientation preserving, and an example that is not.
\end{problem}

\begin{solution}
\end{solution}

\begin{problem}[5]
  Assume that $\phi$ is orientation preserving. Also assume that the procedure described in Problem 3 actually works, so that we have a simplicial complex $L$ and a homeomorphism $|L| = X/\sim$. Show that the orientation of $K$ induces an orientation of $L$.
\end{problem}

\begin{solution}
\end{solution}
