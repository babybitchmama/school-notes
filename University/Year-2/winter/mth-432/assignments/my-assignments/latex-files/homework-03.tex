\begin{problem}[1]
  Assume that $X$ is Hausdorff. The purpose of this problem is to prove that $Y$ is Hausdorff, as well.
  \begin{enumerate}
    \item Show that, given any natural number $n$ and any $n$-tuple of pairwise distinct points $p_1, \cdots, p_n \in X$, we can find pairwise disjoint open sets $U_1, \cdots, U_n$ with $p_i \in U_i$ for all $i$. (Note that the $n = 2$ case is the definition of Hausdorffness. For larger $n$, use induction. Note also that this part of the problem does not involve the map $f$ or the space $Y$.)
    \item Let $q_1$ and $q_2$ be two distinct points in $Y$. Choose points $p_1, p_2 \in X$ with $\pi(p_i) = p_i$. In particular, this implies that the four points $p_1, f(p_1), p_2, f(p_2)$ are pairwise disjoint. By part (i), we can find pairwise disjoint open sets $U_1, U_2, V_1$, and $V_2$ with $p_1 \in U_1$, $p_2 \in U_2$, $f(p_1) \in V_1$, and $f(p_2) \in V_2$. Show that we can make these choices in such a way that $V_1 = f(U_1)$ and $V_2 = f(U_2)$.
    \item Show that $Y$ is Hausdorff.
  \end{enumerate}
\end{problem}

\begin{solution}[(i)]
  We will use induction on $n$. The base case $n = 2$ is true by the definition of Hausdorffness. Now, assume that the statement is true for some $n \geq 2$; we will show that it is true for $n + 1$. Let $p_1, \cdots, p_{n + 1} \in X$ be pairwise distinct points. By the inductive hypothesis, we can find pairwise disjoint open sets $U_1, \cdots, U_n$ with $p_i \in U_i$ for all $1 \leq i \leq n$. Since $X$ is Hausdorff, for each $1 \leq i \leq n$, we can find disjoint open sets $V_i$ and $W_i$ with $p_{n + 1} \in V_i$ and $p_i \in W_i$. Now, let $U_{n+1} = \bigcap_{i=1}^n V_i$ and redefine $U_i = U_i \cap W_i$ for all $1 \leq i \leq n$. Then, the sets $U_1, \cdots, U_{n+1}$ are pairwise disjoint open sets with $p_i \in U_i$ for all $1 \leq i \leq n + 1$. This completes the inductive step, and thus the proof.
\end{solution}

\begin{solution}[(ii)]
  Since $q_1$ and $q_2$ are distinct points in $Y$, their preimages under the identification map $\pi$ are distinct sets in $X$. Specifically, we have $\pi^{-1}(q_1) = \{p_1, f(p_1)\}$ and $\pi^{-1}(q_2) = \{p_2, f(p_2)\}$. By part (i), we can find pairwise disjoint open sets $U_1, U_2, V_1$, and $V_2$ in $X$ such that $p_1 \in U_1$, $p_2 \in U_2$, $f(p_1) \in V_1$, and $f(p_2) \in V_2$. Now, we can define $V_1 = f(U_1)$ and $V_2 = f(U_2)$. Since $f$ is a homeomorphism, the sets $V_1$ and $V_2$ are open in $X$. Furthermore, since the original sets were pairwise disjoint, the new sets remain disjoint as well. Thus, we have constructed the desired open sets.
\end{solution}

\begin{solution}[(iii)]
  To show that $Y$ is Hausdorff, we need to demonstrate that for any two distinct points $q_1, q_2 \in Y$, there exist disjoint open neighborhoods around each point. Let $p_1, p_2 \in X$ be such that $\pi(p_i) = q_i$ for $i = 1, 2$. By part (ii), we can find pairwise disjoint open sets $U_1, U_2, V_1$, and $V_2$ in $X$ such that $p_1 \in U_1$, $p_2 \in U_2$, $f(p_1) \in V_1$, and $f(p_2) \in V_2$, with $V_1 = f(U_1)$ and $V_2 = f(U_2)$. Now, consider the images of these sets under the identification map $\pi$. The sets $\pi(U_1)$ and $\pi(U_2)$ are open in $Y$ because $\pi$ is an open map (as shown in part (i)). Furthermore, since the original sets were disjoint in $X$, their images under $\pi$ will also be disjoint in $Y$. Therefore, we have found disjoint open neighborhoods around $q_1$ and $q_2$, which shows that $Y$ is Hausdorff.
\end{solution}

\begin{problem}[2]
  Assume that $X$ is a surface. The purpose of this problem is to prove that Y is a surface, as well. We have already established that, if $X$ is Hausdorff, so is $Y$. So we just need to show that, if every point in $X$ has an open neighborhood homeomorphic to an open subset of $\R^2$, then the same is true of $Y$.
  \begin{enumerate}
    \item Show that the identification map $\pi$ is open.
    \item Show that, if $U \subset X$ is an open set and $U \cap f(U) = \emptyset$, then $\pi : U \to \pi(U)$ is a homeomorphism.
    \item Show that, if $p \in X$, there exists an open neighborhood $U$ of $p$ such that $U$ is homeomorphic to an open subset of $\R^2$ and $U \cap f(U) = \emptyset$.
    \item Show that $Y$ is a surface.
  \end{enumerate}
  For the remaining problems, suppose that $X = |K|$ is a combinatorial surface and $f = |\phi|$ for some isomorphism $\phi : K \to K$.
\end{problem}

\begin{solution}[(i)]
  The identification map $\pi$ is open if and only if for every open set $U \subset X$, the image $\pi(U)$ is open in $Y$. Let $U$ be an open set in $X$. By the definition of the quotient topology on $Y$, a set is open in $Y$ if and only if its preimage under $\pi$ is open in $X$. Since $\pi^{-1}(\pi(U)) = U \cup f(U)$, and both $U$ and $f(U)$ are open in $X$ (as $f$ is a homeomorphism), their union is also open. Therefore, $\pi(U)$ is open in $Y$, which shows that $\pi$ is an open map.
\end{solution}

\begin{solution}[(ii)]
  If $U \subset X$ is open and $U \cap f(U) = \emptyset$, then the restriction of $\pi$ to $U$, denoted $\pi|_U : U \to \pi(U)$, is a bijection. To see this, note that for any point $x \in U$, $\pi(x) = \pi(f(x))$ only if $f(x) \in U$, which contradicts the assumption that $U \cap f(U) = \emptyset$. Thus, $\pi|_U$ is injective. It is also surjective onto $\pi(U)$ by definition. Since $\pi$ is an open map (from part (i)), the restriction $\pi|_U$ is also open. Therefore, $\pi|_U$ is a homeomorphism between $U$ and $\pi(U)$.
\end{solution}

\begin{solution}[(iii)]
  Since $X$ is a surface, then it is locally Euclidean. Therefore, for any point $p \in X$, there exists an open neighborhood $V$ of $p$ that is homeomorphic to an open subset of $\R^2$. Let $U = V \setminus f(V)$. Since $f$ is a homeomorphism, $f(V)$ is also open in $X$, and thus $U$ is open as the difference of two open sets. Additionally, we have $U \cap f(U) = \emptyset$ by construction. Finally, since $U \subset V$ and $V$ is homeomorphic to an open subset of $\R^2$, it follows that $U$ is also homeomorphic to an open subset of $\R^2$. Thus, we have found the desired neighborhood $U$ of $p$.
\end{solution}

\begin{solution}[(iv)]
  To show that $Y$ is a surface, we need to demonstrate that every point in $Y$ has an open neighborhood homeomorphic to an open subset of $\R^2$. Let $q \in Y$ be an arbitrary point. Choose a point $p \in X$ such that $\pi(p) = q$. By part (iii), there exists an open neighborhood $U$ of $p$ in $X$ such that $U$ is homeomorphic to an open subset of $\R^2$ and $U \cap f(U) = \emptyset$. By part (ii), the restriction $\pi|_U : U \to \pi(U)$ is a homeomorphism. Therefore, $\pi(U)$ is an open neighborhood of $q$ in $Y$, and it is homeomorphic to an open subset of $\R^2$. Since $q$ was arbitrary, this shows that every point in $Y$ has the required property, and thus $Y$ is a surface.
\end{solution}

\begin{problem}[3]
  It's tempting to try to define a triangulation of $Y$ whose simplices are in bijection with equivalence classes of simplices of $K$. Find an example that shows that such a simplicial complex might not exist! (If one subdivides $K$ first, then it works, but you don't have to show that.)
\end{problem}

\begin{solution}
\end{solution}

\begin{problem}[4]
  Suppose that $K$ is equipped with an orientation. We say that $\phi$ is \textit{orientation preserving} if it takes positively oriented triangles to positively oriented triangles. That is, if $v_1$, $v_2$, $v_3$ span a triangle in $K$ and appear in positive cyclic order, then $\phi(v_1)$, $\phi(v_2)$, $\phi(v_3)$ also appear in positive cyclic order. Give an example that is orientation preserving, and an example that is not.
\end{problem}

\begin{solution}
  An example of an orientation preserving map is just the identity map, $f : Y \to Y$, defined by $f(p_i) = p_i$. An example of a non-orientation preserving map is the reflection map across the x-axis in $\R^2$, defined by $f(x, y) = (x, -y)$. This map reverses the orientation of any triangle in the plane.
\end{solution}

\begin{problem}[5]
  Assume that $\phi$ is orientation preserving. Also assume that the procedure described in Problem 3 actually works, so that we have a simplicial complex $L$ and a homeomorphism $|L| = X/\sim$. Show that the orientation of $K$ induces an orientation of $L$.
\end{problem}

\begin{solution}
  Let $\phi$ be an orientation preserving isomorphism of the simplicial complex $K$. We want to show that the orientation of $K$ induces an orientation of the simplicial complex $L$ formed by the quotient $X/\sim$. To do this, we will define the orientation of $L$ based on the orientation of $K$. Consider a triangle in $L$ represented by the equivalence class of a triangle in $K$. Since $\phi$ is orientation preserving, the orientation of the triangle in $K$ will be preserved under the identification. Therefore, we can assign the same orientation to the corresponding triangle in $L$.
\end{solution}
