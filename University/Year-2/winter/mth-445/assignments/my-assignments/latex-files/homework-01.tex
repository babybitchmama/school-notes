\begin{problem}[22.12]
  Decide whether the set $\{a + b\sqrt{2} \mid a, b \in \Z\}$ with the usual addition and multiplication are defined (closed), and give a ring structure. If a ring is not formed, tell why this is the case. If a ring is formed, state whether the ring is commutative, whether it has unity, and whether it is a field.
\end{problem}

\begin{solution}
  Let $x, y \in A = \{a + b\sqrt{2} \mid a, b \in \Z\}$. Adding them, we have
  \[%
    x + y = (a_1 + b_1\sqrt{2}) + (a_2 + b_2\sqrt{2}) = (a_1 + a_2) + (b_1 + b_2)\sqrt{2} \in A.
  .\]%
  Multiplying them, we have
  \[%
    x \cdot y = (a_1 + b_1\sqrt{2})(a_2 + b_2\sqrt{2}) = (a_1 a_2 + 2 b_1 b_2) + (a_1 b_2 + a_2 b_1)\sqrt{2} \in A.
  .\]%
  Therefore, $A$ is closed under addition and multiplication.
\end{solution}

\begin{problem}[22.18]
  Describe all the units of the ring $\Z \times \Q \times \Z$.
\end{problem}

\begin{solution}
  The units of the ring $\Z \times \Q \times \Z$ are the elements $(a, b, c)$ such that $a$ and $c$ are units in $\Z$ and $b$ is a unit in $\Q$. The only units in $\Z$ are $1$ and $-1$, while every non-zero element in $\Q$ is a unit. Therefore, the units of the ring $\Z \times \Q \times \Z$ are of the form $(\pm 1, b, \pm 1)$ where $b \in \Q^*$.
\end{solution}

\begin{problem}[22.20]
  Consider the matrix ring $M_2(\Z_2)$.
  \begin{enumerate}
    \item Find the \textit{order} of the ring, that is, the number of elements in it.
    \item List all units in the ring.
  \end{enumerate}
\end{problem}

\begin{solution}[(i)]
  Each entry in each $2 \times 2$ matrix has $2$ possible options, either $0$ or $1$. Since there are $4$ slots, each with $2$ options, the total number of elements in the ring is $2^4 = 16$.
\end{solution}

\begin{solution}[(ii)]
\end{solution}

\begin{problem}[22.21]
  If possible, give an example of a homomorphism $\phi : R \to R'$ where $R$ and $R'$ are rings with unity $1 \ne 0$ and $1' \ne 0'$, and where $\phi(1) \ne 0'$ and $\phi(1) \ne 1'$.
\end{problem}

\begin{solution}
  Assume we have an isomorphism $\phi : R \to R'$ such that $\phi(1) \ne 0'$ and $\phi(1) \ne 1'$. Since $\phi$ is a homomorphism, we have
  \[%
    \phi(1) = \phi(1 \cdot 1) = \phi(1) \cdot' \phi(1)
  .\]%
  This implies that $\phi(1)$ is an idempotent element in $R'$. However, in a ring with unity, the only idempotent elements are $0'$ and $1'$. Therefore, this isn't possible.
\end{solution}

\begin{problem}[22.26]
  How many homomorphisms are there of $\Z \times \Z \times \Z$ into $\Z$?
\end{problem}

\begin{solution}
  Assume $\phi : \Z \times \Z \times \Z \to \Z$ is a homomorphism. Then, it must send the unity element $(1, 1, 1)$ to the unity element $1$ in $\Z$. Therefore,
  \[%
    \phi(1, 1, 1) = \phi(1, 0, 0) + \phi(0, 1, 0) + \phi(0, 0, 1) = 1
  .\]%
  Let $\e_1 = (1, 0, 0)$, $\e_2 = (0, 1, 0)$, and $\e_3 = (0, 0, 1)$. Therefore, there are three possible homomorphisms defined by
  \begin{alignat*}{5}
    \phi_1: \quad\phi_1(\e_1) &= 1, \quad \phi_1(\e_2) &&= 0, \quad \phi_1(\e_3) &&= 0 \\
    \phi_2: \quad\phi_2(\e_1) &= 0, \quad \phi_2(\e_2) &&= 1, \quad \phi_2(\e_3) &&= 0 \\
    \phi_3: \quad\phi_3(\e_1) &= 0, \quad \phi_3(\e_2) &&= 0, \quad \phi_3(\e_3) &&= 1
  .\end{alignat*}
  These $\e_i$ elements form a complete set of orthogonal idempotent elements. This means that they satisfy the following properties
  \[%
    \e_i\e_j = \delta_{ij}\e_i \aand \sum_{i=1}^3 \e_i = 1
  .\]%
  This is clearly satisfied by our choice of $\e_i$ elements, i.e.,
  \[%
    f(\e_i)f(\e_j) = \delta_{ij}f(\e_i) \aand f(\e_1) + f(\e_2) + f(\e_3) = 1
  .\]%
  Therefore, there are exactly $3$ homomorphisms from $\Z \times \Z \times \Z$ into $\Z$.
\end{solution}

\begin{problem}[22.28]
  Find all solutions of the equation $x^2 + x - 6 = 0$ in the ring $\Z_{14}$ by factoring the quadratic polynomial. Compare with Exercise 27.
\end{problem}

\begin{solution}
  Factoring the equation $x^2 + x - 6 = 0$, we have $(x + 3)(x - 2) = 0$. So, we clearly have the solutions $x = -3$ and $x = 2$. But $x = -3$ is equivalent to $x = 11$ in $\Z_{14}$. Notice that for $x = 4$, we have
  \[%
    (4 + 3)(4 - 2) = 7 \cdot 2 = 14 \equiv 0 \pmod{14}
  .\]%
  Thus, $x = 4$ is another solution in the ring $\Z_{14}$. We also have for $x = 9$,
  \[%
    (9 + 3)(9 - 2) = 12 \cdot 7 = 84 \equiv 0 \pmod{14}
  .\]%
  Therefore, all the solutions to the equation $x^2 + x - 6 = 0$ in the ring $\Z_{14}$ are $x = 2$, $4$, $9$, and $11$.
\end{solution}

\begin{problem}[22.39]
  Show that if $U$ is the collection of all units in a ring $\bra{R, +, \cdot}$ with unity, then $\bra{U, \cdot}$ is a group. [\textit{Warning:} Be sure to show that $U$ is closed under multiplication.]
\end{problem}

\begin{solution}
  Assume that $R$ is a ring with unity $1 \ne 0$. Then, clearly $U$ is non-empty since it contains at least the unity element $1$. Let $a, b \in U$. Then, $ab \in U$, since there exists $a^{-1}$, $b^{-1}$ such that
  \[%
    (ab)(b^{-1} a^{-1}) = a(bb^{-1})a^{-1} = a \cdot 1 \cdot a^{-1} = aa^{-1} = 1
  .\]%
  Therefore, $U$ is closed under multiplication. Also, multiplication in $R$ is associative, so it is associative in $U$. The unity element $1$ is in $U$ and serves as the identity element. Finally, for each $a \in U$, there exists an inverse $a^{-1} \in U$. Therefore, $\bra{U, \cdot}$ is a group.
\end{solution}

\begin{problem}[22.40]
  Show that $a^2 - b^2 = (a + b)(a - b)$ for all $a$ and $b$ in a ring $R$ if and only if $R$ is commutative.
\end{problem}

\begin{solution}
  Assume $a^2 - b^2 = (a + b)(a - b)$ for all $a, b \in R$.
\end{solution}

\begin{problem}[22.42]
  Show that the rings $2\Z$ and $3\Z$ are not isomorphic. Show that the fields $\R$ and $\C$ are not isomorphic.
\end{problem}

\begin{solution}
\end{solution}

\begin{problem}[22.43]
  Let $p$ be a prime. Show that in the ring $\Z_p$ we have $(a + b)^p = a^p + b^p$ for all $a, b \in \Z_p$. [\textit{Hint:} Observe that the usual binomial expansion $(a + b)^n$ is valid in a \textit{commutative ring}.]
\end{problem}

\begin{solution}
\end{solution}
