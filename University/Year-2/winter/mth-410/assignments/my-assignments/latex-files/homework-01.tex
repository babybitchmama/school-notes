\begin{problem}[1]
  Let $\bra{\cdot, \cdot}$ be a real inner product on a vector space $V$, with induced norm $\|\cdot\|$. Prove the following (called the Cauchy--Schwarz inequality):
  \[%
    |\bra{\v, \w}| \leq \|v\|\,\|w\| \quad \text{for all}~v, w \in V
  .\]%
  \textit{Hint.} Consider the quadratic polynomial $p(t) = \bra{v + tw, v + tw}$, where $t \in \R$.
\end{problem}

\begin{solution}
  Let $\v, \w \in V$. Consider the quadratic polynomial
  \[%
    p(t) = \bra{\v + t\w, \v + t\w} = \bra{\v, \v} + 2t\bra{\v, \w} + t^2\bra{\w, \w} = \|\v\|^2 + 2t\bra{\v, \w} + t^2\|\w\|^2
  .\]%
  Since $V$ is an inner product space, we have $p(t) \ge 0$, since $\bra{v, w} \ge 0$ and $\|v\|, \|w\| \ge 0$. Therefore, we have
  \[%
    \|\w\|^2 t^2 + 2\bra{\v, \w} t + \|\v\|^2 \ge 0
  .\]%
  The discriminant of this quadratic polynomial must be less than or equal to zero, so we have
  \[%
    b^2 - 4ac = (2\bra{\v, \w})^2 - 4\|\w\|^2 \|\v\|^2 \le 0
  .\]%
  Solving this inequality gives
  \[%
    \bra{\v, \w}^2 \le \|\w\|^2 \|\v\|^2
  .\]%
  Taking the square root of both sides, the Cauchy--Schwarz inequality holds.
\end{solution}

\begin{problem}[2]
  Show that a line segment in $\R^2$ has two-dimensional Lebesgue measure equal to zero. \emph{Optional}: Prove that the same result is true even if the line has infinite length.
\end{problem}

\begin{solution}
  Let $A$ be the set of a line segment in $\R^2$. Without loss of generality, we can assume that the line segment is horizontal. Let the length of the line segment be $L$. For $A$ to have measure zero, for any $\epsilon > 0$, there exists a countable collection of open intervals such that
  \[%
    A \subseteq \bigcup_{i=1}^\infty I_i \aand \sum_{i=1}^\infty \text{area}(I_i) < \epsilon
  .\]%
  Let $A$ be the collection of rectangles of width $L$ and height $\delta$, where $\delta = \frac{\epsilon}{L}$. Then, we have
  \[%
    \text{area}(A) = L \cdot \delta = L \cdot \frac{\epsilon}{L} = \epsilon
  .\]%
  Therefore, the line segment has measure zero.
\end{solution}

\begin{problem}[3]
  Find a sequence of continuous functions $f_n : [0, 1] \to \R$ such that
  \[%
    \lim_{n \to \infty} f_n(x) = 0 \quad \text{for all}~x \in [0, 1]
  ,\]%
  but
  \[%
    \lim_{n \to \infty} \int_0^1 f_n(x) \dx = 1
  .\]%

  \noindent Why does this not contradict the Lebesgue Dominated Convergence Theorem?
\end{problem}

\begin{solution}
  Take the sequence of functions defined by
  \[%
    f_n(x) = \begin{cases}
      n, & \text{if}~0 \leq x \leq 1/n \\
      0, & \text{if}~1/n < x \leq 1
    \end{cases}
  .\]%
  For all $x \in [0, 1]$, we have
  \[%
    \lim_{n \to \infty} f_n(x) = 0
  .\]%
  Integrating $f_n(x)$ over $[0, 1]$, we have
  \[%
    \int_0^1 f_n(x) \dx = \int_0^{1/n} n \dx + \int_{1/n}^1 0 \dx = n \cdot \frac{1}{n} + 0 = 1
  .\]%
  The Lebesgue Dominated Convergence Theorem states that given a converging sequence of functions $(f_n)$ on a measurable space $(S, \Sigma, \mu)$ that converges pointwise to a function $f$ and is dominated by an integrable function $g(x)$ (i.e., $|f(x)| \le g(x)$), then the following must hold
  \[%
    \lim_{n \to \infty} \int_S f_n(x) \dd{\mu} = \int_S f(x) \dd{\mu}
  .\]%
  We don't get any contradictions because there is no integrable function $g(x)$ that dominates the sequence of functions $f_n(x)$ on $[0, 1]$.
\end{solution}

\begin{problem}[4]
  Let $V = L^1(\R^n)$ and $W = L^\infty(\R^n)$. Show that if $K \in L^\infty(\R^n \times \R^n)$ then the mapping $T$ defined by
  \[%
    [Tf](x) = \int_{\R^n} K(x, y) f(y) \dy
  ,\]%
  is a bounded linear transformation from $V$ to $W$, with
  \[%
    \|T\|_{L^1 \to L^\infty} \leq \|K\|_{L^\infty(\R^n \times \R^n)}
  .\]%
\end{problem}

\begin{solution}
  The norm of $K$ in $L^\infty(\R^n \times \R^n)$ is defined as
  \[%
    \|K\|_{L^\infty(\R^n \times \R^n)} \coloneqq \inf\{\alpha > 0 : |\{(x, y) \in \R^n \times \R^n : |f(x)| > \alpha\}| = 0\} < \infty
  .\]%
  This means that $\|K\|_{L^\infty(\R^n \times \R^n}$ is the smallest number such that $|K(x, y)| \leq \|K\|_{L^\infty(\R^n \times \R^n)}$ for all $(x, y) \in \R^n \times \R^n$. Let $f \in L^1(\R^n)$. Then, for every $x \in \R^n$, we have
  \begin{align*}
    |[Tf](x)| &= \left|\int_{\R^n} K(x, y) f(y) \dy\right| \\
              &\leq \int_{\R^n} |K(x, y)||f(y)| \dy \\
              &\leq \|K\|_{L^\infty(\R^n \times \R^n)} \int_{\R^n} |f(y)| \dy \\
              &= \|K\|_{L^\infty(\R^n \times \R^n)} \|f\|_{L^1(\R^n)}
  .\end{align*}
  We have the norm $\|T\|_{L^1 \to L^\infty}$ defined as
  \[%
    \|T\|_{L^1 \to L^\infty} \coloneqq \sup\{\|Tf\|_{L^\infty} : \|f\|_{L^1} = 1\}
  .\]%
  Therefore, we have
  \[%
    \|Tf\|_{L^\infty} = \sup_{x \in \R^n} |[Tf](x)| \leq \|K\|_{L^\infty(\R^n \times \R^n)} \|f\|_{L^1(\R^n)} = \|K\|_{L^\infty(\R^n \times \R^n)}
  ,\]%
  since $\|f\|_{L^1} = 1$. Thus, $T$ is a bounded linear transformation from $V$ to $W$.
\end{solution}

\begin{problem}[5]
  Let $W$ be a closed subset of a normed space $V$. Show that if $w_k\in W$ and $\lim_{k \to \infty} w_k = w$, then $w \in W$. Recall that by definition $W$ is closed if its complement is open.
\end{problem}

\begin{solution}
  Assume $w_k \in W$ for all $k \in \Z$. Since $W$ is closed, then $W^c$ is open. Since $w_k \to w$, for any $\epsilon > 0$, there exists $N \in \N$ such that for all $k \ge N$, we have $\|w_k - w\| < \epsilon$. Suppose, for the sake of contradiction, that $w \notin W$. Then, $w \in W^c$. Since $W^c$ is open, there exists $\delta > 0$ such that the open ball $B(w, \delta) \subseteq W^c$. However, since $w_k \to w$, there exists $N' \in \N$ such that for all $k \ge N'$, we have $\|w_k - w\| < \delta$. This implies that $w_k \in B(w, \delta) \subseteq W^c$ for all $k \ge N'$, contradicting the assumption that $w_k \in W$ for all $k$. Therefore, $w \notin W^c$, and thus $w \in W$.
\end{solution}
