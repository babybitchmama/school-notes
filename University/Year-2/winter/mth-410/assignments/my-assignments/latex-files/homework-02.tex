\begin{problem}[1 (Gaussian integral computation)]
  Let $a > 0$. Compute the value of
  \[%
    \int_{\R^n} e^{-a|x|^2} \dx
  .\]%
  \textit{Hint:} First reduce matters to computing a one-dimensional integral by writing $|x|^2 = x^2_1 + \cdots + x^2_n$. To compute the one-dimensional case, start with
  \[%
    \left(\int_\R e^{-ax^2} \dx\right)^2
  .\]%
  View this as a two-dimensional integral and then use polar coordinates.
\end{problem}

\begin{solution}
  Computing the two-dimensional integral using polar coordinates, we have
  \[%
    \int_{-\infty}^\infty e^{-ax^2} \dx = \int_0^{2\pi} \int_0^\infty e^{-ar^2} r \dd{r,\theta} = 2\pi \int_0^\infty r e^{-ar^2} \dd{r} = \sqrt{\frac{\pi}{a}}
  .\]%
  Then, we can convert the $n$-dimensional integral into a product of one-dimensional integrals to get
  \begin{align*}
    \int_{\R^n} e^{-a|x|^2} \dx &= \int_{-\infty}^\infty \cdots \int_{-\infty}^\infty e^{-a(x_1^2 + \cdots + x_n^2)} \,\dx_1 \cdots \dx_n \\
                                &= \int_{-\infty}^\infty e^{-ax_1^2} \dx_1 \cdots \int_{-\infty}^\infty e^{-ax_n^2} \dx_n \\
                                &= \sqrt{\frac{\pi}{a}} \cdots \sqrt{\frac{\pi}{a}} = \left(\frac{\pi}{a}\right)^{n/2}
  .\qedhere\end{align*}
\end{solution}

\begin{problem}[2 (Radon transform of a Gaussian)]
  Let $f(x) = e^{-a|x|^2}$. Compute the Radon transform of $f$. (You will need your solution from Problem 1.)
\end{problem}

\newcommand\Radon{\mathcal{R}}
\begin{solution}
  The Radon transform is given by
  \[%
    \Radon f(t, \omega) = \int_{-\infty}^\infty f(s\hat{\omega} + t\omega) \ds
  .\]%
  We can compute this integral as follows
  \[%
    \Radon f(t, \omega) = \int_{-\infty}^\infty e^{-a|s\hat{\omega} + t\omega|^2} \ds
  .\]%
  Notice that the vectors $\hat{\omega}$ and $\omega$ are orthogonal unit vectors, so we have
  \[%
    |s\hat{\omega} + t\omega|^2 = s^2 + t^2
  .\]%
  Thus, we can rewrite the Radon transform as
  \[%
    \Radon f(t, \omega) = \int_{-\infty}^\infty e^{-a(s^2 + t^2)} \ds = e^{-at^2} \int_{-\infty}^\infty e^{-as^2} \ds = e^{-at^2} \sqrt{\frac{\pi}{a}}
  .\qedhere\]%
\end{solution}

\begin{problem}[3 (Back-projection)]
  Let $f$ be the characteristic function of the unit ball in $\R^2$. First verify that the Radon transform is given by
  \[%
    \Radon\chi_{\mathcal{B}}(t, \omega) = \begin{cases}
      2\sqrt{1 - t^2}, & |t| \leq 1, \\
      0, & |t| > 1.
    \end{cases}
  \]%
  Fix $x = (r, \theta)$ for some $r \ge 0$, and establish the following properties:
  \begin{itemize}
    \item For $0 \le r \le 1$, we have
      \[%
        \tilde{f}((r, 0)) = \frac{1}{\pi} \int^{2\pi}_0 \sqrt{1 - r^2\cos^2(\theta)} \dd{\theta}
      .\]%
    \item For $r > 1$, we have a bound of the form
      \[%
        |\tilde{f}((r, 0))| \le \frac{C}{r}
      .\]%
  \end{itemize}
\end{problem}

\begin{solution}
  Let $f$ be defined by
  \[%
    f(x) = \begin{cases}
      1, & |x| \leq 1, \\
      0, & |x| > 1.
    \end{cases}
  \]%
  The Radon transform of $f$ is given by
  \begin{align*}
    \Radon f(t, \omega) &= \int_{-\infty}^\infty f(s\hat{\omega} + t\omega) \ds \\
                        &= \int_{-\infty}^\infty \chi_{\mathcal{B}}(s\hat{\omega} + t\omega) \ds \\
                        &= \int_{-\infty}^\infty \begin{cases}
                          1, & |s\hat{\omega} + t\omega| \leq 1, \\
                          0, & |s\hat{\omega} + t\omega| > 1.
                        \end{cases} \ds \\
                        &= \int_{-\sqrt{1 - t^2}}^{\sqrt{1 - t^2}} 1 \ds = 2\sqrt{1 - t^2}
  .\end{align*}
  Therefore, we have
  \[%
    \Radon\chi_{\mathcal{B}}(t, \omega) = \begin{cases}
      2\sqrt{1 - t^2}, & |t| \leq 1, \\
      0, & |t| > 1.
    \end{cases}
  \]%

  Now, we consider the back-projection defined by
  \[%
    \tilde{f}(x) = \frac{1}{2\pi}\int_0^{2\pi} \Radon f(\bra{x, \omega}, \omega) \dd{\theta}
  .\]%
  Computing this for $x = (r, 0)$ with $0 \leq r \leq 1$, we have
  \begin{align*}
    \tilde{f}((r, 0)) &= \frac{1}{2\pi} \int_0^{2\pi} \Radon f(r\cos(\theta), \omega) \dd{\theta} \\
                     &= \frac{1}{\pi} \int_0^{2\pi} \sqrt{1 - r^2\cos^2(\theta)} \dd{\theta} \\
                     &= \frac{1}{\pi} \int_0^{2\pi} \sqrt{1 - r^2\cos^2(\theta)} \dd{\theta}
  .\end{align*}
  For $r > 1$, we have
  \begin{align*}
    |\tilde{f}((r, 0))| &= \left|\frac{1}{2\pi} \int_0^{2\pi} \Radon f(r\cos(\theta), \omega) \dd{\theta}\right| \\
                     &\leq \frac{1}{2\pi} \int_0^{2\pi} |\Radon f(r\cos(\theta), \omega)| \dd{\theta} \\
                     &\leq \frac{1}{2\pi} \int_0^{2\pi} 0 \dd{\theta} = 1 \leq \frac{C}{r}
  .\qedhere\end{align*}
\end{solution}

\begin{problem}[4 (Projection onto a hyperplane)]
  Let $p \in \R$ and $r \in \R^J$. Show that the projection of a vector $y \in \R^J$ onto the hyperplane $\{x \in \R^J : x \cdot r = p\}$ is given by
  \[%
    y \mapsto y - \left[\frac{y \cdot r - p}{r \cdot r}\right] r
  .\]%
\end{problem}

\begin{solution}
  The projection of $y$ onto the hyperplane is given by
  \[%
    \proj_H(y) = y - \proj_r y
  .\]%
  The projection of $y$ onto the vector $r$ is given by
  \[%
    \proj_r y = \frac{y \cdot r}{r \cdot r} r
  .\]%
  Therefore, we have
  \[%
    \proj_H(y) = y - \frac{y \cdot r}{r \cdot r} r
  .\]%
  To ensure that the projection lies on the hyperplane defined by $x \cdot r = p$, we need to adjust the projection by adding a term that accounts for the difference between $y \cdot r$ and $p$. Thus, we have
  \[%
    \proj_H(y) = y - \left[\frac{y \cdot r - p}{r \cdot r}\right] r
  .\qedhere\]%
\end{solution}

\begin{problem}[5 (1d version of pixel basis)]
  For each $N = 1, 2, \dots$, define the following intervals
  \[%
    I^N_j = \left[\frac{j}{N}, \frac{j + 1}{N}\right], \quad j = 0, \dots, N - 2, \aand I^N_{N-1} = \left[\frac{N - 1}{N}, 1\right]
  .\]%
  Suppose $f : [0, 1] \to \R$ is continuous and for each $N$ define
  \[%
    f_N(x) = N \sum_{j=0}^{N-1} \left[\int_{I^N_j} f(y) \dy\right] \chi_{I^N_j}(x)
  ,\]%
  where $\chi_{I^N_j}$ is the characteristic function of $I^N_j$. Show that $f_N$ converges to $f$ uniformly on $[0, 1]$ as $N \to \infty$.
\end{problem}

\begin{solution}
  By the Mean Value Theorem for Integrals, there exists some $c_j \in I^N_j$ such that
  \[%
    \int_{I^N_j} f(y) \dy = f(c_j) \cdot \frac{1}{N}
  .\]%
  Therefore, we can rewrite $f_N(x)$ as
  \[%
    f_N(x) = N \sum_{j=0}^{N-1} f(c_j) \cdot \frac{1}{N} \chi_{I^N_j}(x) = \sum_{j=0}^{N-1} f(c_j) \chi_{I^N_j}(x)
  .\]%
  Since $f$ is continuous on the compact interval $[0, 1]$, it's uniformly continuous. Thus, for any $\epsilon > 0$, there exists a $\delta > 0$ such that for all $x, y \in [0, 1]$ with $|x - y| < \delta$, we have $|f(x) - f(y)| < \epsilon$. Now, choose $N$ such that $1/N < \epsilon$. For any $x \in [0, 1]$, there exists a unique $j$ such that $x \in I^N_j$. Then, we have
  \[%
    |f_N(x) - f(x)| = |f(c_j) - f(x)| < \epsilon
  .\]%
  This shows that $f_N$ converges to $f$ uniformly on $[0, 1]$ as $N \to \infty$.
\end{solution}
