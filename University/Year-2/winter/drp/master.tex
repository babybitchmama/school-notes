\documentclass[notitlepage]{report}

%%%%%%%%%%%%%%%%%%%%%%%%%%%%%%%%%%%%%%%%%%%%%%%%%%%%%%%%%%%%%%%%%%%%%%%%%%%%%%%%
%                                                                              %
%                              Required Packages                               %
%                                                                              %
%%%%%%%%%%%%%%%%%%%%%%%%%%%%%%%%%%%%%%%%%%%%%%%%%%%%%%%%%%%%%%%%%%%%%%%%%%%%%%%%

% Required for creating documents
\usepackage[utf8]{inputenc}
\usepackage[T1]{fontenc}
\RequirePackage{etex}

% Required math packages
\usepackage{amsmath}
\usepackage{amsfonts}
\usepackage{mathtools}
\usepackage{amsthm}
\usepackage{amssymb}
\usepackage{mathrsfs}

\usepackage{multicol} % for multiple columns
\usepackage[usenames,dvipsnames,pdftex]{color} % Required for nicer colors
\usepackage{hyperref} % Required for hyperlinks
\usepackage{xparse} % Required for \NewDocumentCommand
\usepackage{graphicx} % Required for including images
\usepackage{enumitem} % Required for customizing lists
\usepackage{float} % Required for positioning figures and tables
\usepackage{array} % Required for customizing tables
\usepackage{systeme} % Required for \systeme
\usepackage{cancel} % Required for \cancel
\usepackage{derivative} % Required for \odv and \pdv
\usepackage{authoraftertitle} % Required for \MyTitle
\usepackage{geometry} % Required for customizing page layout
\usepackage{textgreek} % Required for greek letters in text mode
\usepackage{multirow} % Required for multirow in tables
\usepackage{emptypage} % Required for removing page numbers on empty pages
\usepackage{nicematrix} % Required for better matrices
\usepackage{booktabs} % Required for better tables
\usepackage{cellspace} % Required for better spacing in tables
\usepackage{longtable} % Required for long tables
\usepackage{xfrac} % Required for extra fraction options
\usepackage{diagbox} % Required for creating diagonal boxes
\usepackage{polynom} % Required for polynomial long division
\usepackage{setspace} % For line spacing

\usepackage[font=bf]{caption} % Required for customizing captions
\usepackage{subcaption} % Required for creating subfigures
\usepackage{siunitx} % Required for SI units
\usepackage{titletoc} % Required for customizing table of contents
\usepackage{braket} % Required for creating braket notation

% Required for drawing figures
\usepackage{tikz}
\usepackage{pgffor}
\usepackage{tkz-euclide}
\usepackage{tikz-cd}
\usepackage{tikz-3dplot}
\usepackage{circuitikz}

% Required for creating plots
\usepackage{pgfplots}
\usepackage{pgfplotstable}

\usepackage{titling} % Required for customizing title page
\usepackage{ifthen} % Required for if-then-else statements
\usepackage{xifthen} % Required for if-then-else statements
\usepackage{fancyhdr} % Required for customizing headers and footers
\usepackage{import} % Required for importing pdf_tex files
\usepackage{titlesec} % Required for customizing sectioning commands
\usepackage{etex} % Required for more registers

% Required for creating boxes
\usepackage[most,many,breakable]{tcolorbox}

\makeatletter


%%%%%%%%%%%%%%%%%%%%%%%%%%%%%%%%%%%%%%%%%%%%%%%%%%%%%%%%%%%%%%%%%%%%%%%%%%%%%%%%
%                                                                              %
%                                Basic Settings                                %
%                                                                              %
%%%%%%%%%%%%%%%%%%%%%%%%%%%%%%%%%%%%%%%%%%%%%%%%%%%%%%%%%%%%%%%%%%%%%%%%%%%%%%%%

\ifx\nauthor\undefined
  \def\nauthor{Hashem A. Damrah}
\else
\fi

\ifx\class\undefined
  \def\class{report}
\else
\fi

\newcommand\globalcolor[1]{%
  \color{#1}\global\let\default@color\current@color
}

% Symbols
\let\oldlimsymbol\lim\renewcommand\lim{\displaystyle\oldlimsymbol}
\let\oldsumsymbol\sum\renewcommand\sum{\displaystyle\oldsumsymbol}
\let\oldlimsupsymbol\limsup\renewcommand\limsup{\displaystyle\oldlimsupsymbol}
\let\oldliminfsymbol\liminf\renewcommand\liminf{\displaystyle\oldliminfsymbol}
\let\implies\Rightarrow
\let\impliedby\Leftarrow
\let\iff\Leftrightarrow
\let\epsilon\varepsilon

% Geometry
\geometry{
  top=1in,
  bottom=1in,
  right=1in,
  left=1in,
}

% Tables
\newcolumntype{C}{>{\Centering\arraybackslash}X}
\setlength{\tabcolsep}{5pt}
\renewcommand\arraystretch{1.5}
\renewcommand\thetable{\Roman{table}}
\captionsetup[figure]{font=small}
\captionsetup{justification=centering}
\setlength\cellspacetoplimit{6pt}
\setlength\cellspacebottomlimit{6pt}

% Lists
\renewcommand{\labelitemi}{--}
\renewcommand{\labelitemii}{$\circ$}
\renewcommand{\labelenumi}{\textnormal{(\roman{*})}}

% Center Title Page
\let\@real@maketitle\maketitle
% \renewcommand\maketitle{
%   \@real@maketitle
%   \begin{center}
%     \begin{minipage}[c]{0.9\textwidth}
%       \centering\footnotesize
%       These notes are not endorsed by the lecturers, and
%       I have modified them (often significantly) after lectures. They are
%       nowhere near accurate representations of what was actually lectured, and
%       in particular, all errors are almost surely mine.

%       \textit{Disclaimer:} This document will inevitably contain some
%       mistakes--both simple typos and legitimate errors. Keep in mind that these
%       are the notes of an undergraduate student in the process of learning the
%       material himself, so take what you read with a grain of salt. If you find
%       mistakes and feel like telling me, I will be grateful and happy to hear
%       from you, even for the most trivial of errors. You can reach me by email,
%       in English, Arabic, Hebrew, or Spanish at
%       \href{mailto:singularisartt@gmail.com}{singularisartt@gmail.com}.
%     \end{minipage}
%   \end{center}
% }
\renewcommand\maketitlehooka{\null\mbox{}\vfill}
\renewcommand\maketitlehookd{\vfill\null}

% Footnote Line
\renewcommand\footnoterule{\hrule\vspace{0.1cm}}

% Modify Links Color
\hypersetup{
  colorlinks,
  linkcolor=black!90,
  citecolor=black,
  urlcolor=cyan!70!black,
}

%%%%%%%%%%
%  TikZ  %
%%%%%%%%%%

\usetikzlibrary{
  shadings,
  intersections,
  angles,
  quotes,
  calc,
  positioning,
  3d,
  perspective,
  arrows,
  arrows.meta,
  patterns,
  decorations.markings,
  bending,
  decorations.pathreplacing,
  calligraphy,
  backgrounds,
}

\tikzoption{canvas is xy plane at z}[]{%
	\def\tikz@plane@origin{\pgfpointxyz{0}{0}{#1}}%
	\def\tikz@plane@x{\pgfpointxyz{1}{0}{#1}}%
	\def\tikz@plane@y{\pgfpointxyz{0}{1}{#1}}%
	\tikz@canvas@is@plane}

\usetikzlibrary{shapes.arrows}
\newcommand\graphslopefield{
  \pgfmathsetmacro{\hx}{(\xmax-\xmin)/\nx}
  \pgfmathsetmacro{\hy}{(\ymax-\ymin)/\ny}
  \foreach \i in {0,...,\nx}
  \foreach \j in {0,...,\ny}{
    \pgfmathsetmacro{\yprime}{f({\xmin+\i*\hx},{\ymin+\j*\hy})}
    \draw[black,shift={({\xmin+\i*\hx},{\ymin+\j*\hy})}]
    (0,0)--($(0,0)!2mm!(.1,.1*\yprime)$);
  }

  \draw[->] (\xmin-.5,0)--(\xmax+.5,0) node[below right] {$x$};
  \draw[->] (0,\ymin-.5)--(0,\ymax+.5) node[above left] {$y$};
}

\tikzset{derivative/.style={color=gray,mark=none,line width=0.5pt,solid}}
\tikzset{asymptote/.style={color=gray,mark=none,line width=1pt,<->,dashed}}
\tikzset{soldot/.style={color=black,fill=black,only marks,mark=*}}
\tikzset{holdot/.style={color=black,fill=white,only marks,mark=*}}

\tikzset{>=stealth}
\tikzset{->-/.style={decoration={markings,mark=at position .5 with {\arrow{>}}},postaction={decorate}}}

%%%%%%%%%%%%%%
%  PgfPlots  %
%%%%%%%%%%%%%%

\pgfplotsset{width=7cm,compat=1.8}
\pgfplotsset{compat=newest}

\usepgfplotslibrary{patchplots}
\usepgfplotslibrary{fillbetween}
\usetikzlibrary{intersections}

\pgfplotsset{plot/.style={color=red,mark=none,line width=1pt,<->,solid}}
\pgfplotsset{asymptote/.style={color=gray,mark=none,line width=1pt,<->,dashed}}
\pgfplotsset{soldot/.style={color=red,only marks,mark=*}}
\pgfplotsset{holdot/.style={color=red,fill=white,only marks,mark=*}}
\pgfplotsset{blankgraph/.style={xmin=-10,xmax=10,ymin=-10,ymax=10,axis line style= {-, draw opacity=0 },axis lines=box,major tick length=0mm,xtick={-10,-9,...,10},ytick={-10,-9,...,10},grid=major,yticklabels={,,},xticklabels={,,},minor xtick=,minor ytick=,xlabel={},ylabel={},width=0.75\textwidth,grid style={solid,gray!40}}}

\pgfplotscreateplotcyclelist{stylelist}{
  plot \\
}

\def\axisdefaultwidth{175pt}
\def\axisdefaultheight{\axisdefaultwidth}

\pgfplotsset{every axis/.append style={
    axis x line=middle,
    axis y line=middle,
    axis line style={<->},
    xlabel={$x$},
    ylabel={$y$},
    xmin=-7,xmax=7,
    ymin=-7,ymax=7,
    yticklabel style={inner sep=0.333ex},
    minor xtick={-7,-6,...,7},
    minor ytick={-7,-6,...,7},
    scale only axis,
    cycle list name=stylelist,
    tick label style={font=\footnotesize},
    legend cell align=left,
    grid=minor,
    grid style={solid,gray!40},
    try min ticks=6,
  },
  framed/.style={axis background/.style={draw=gray}}
}

\pgfplotsset{axis background/.style={draw=gray}}


%%%%%%%%%%%%%%%%%%%%%%%%%%%%%%%%%%%%%%%%%%%%%%%%%%%%%%%%%%%%%%%%%%%%%%%%%%%%%%%%
%                                                                              %
%                           School Specific Commands                           %
%                                                                              %
%%%%%%%%%%%%%%%%%%%%%%%%%%%%%%%%%%%%%%%%%%%%%%%%%%%%%%%%%%%%%%%%%%%%%%%%%%%%%%%%

%%%%%%%%%%%%%%%%%%%%%%
%  Helpful Commands  %
%%%%%%%%%%%%%%%%%%%%%%

\newcommand\resetcounters{
  \setcounter{section}{0}
  \setcounter{subsection}{0}
  \setcounter{subsubsection}{0}
  \setcounter{paragraph}{0}
  \setcounter{subparagraph}{0}
}

\newcommand*\cleartoleftpage{%
  \clearpage
  \thispagestyle{empty}
  \ifodd\value{page}\else\hbox{}\newpage\fi
}

%%%%%%%%%%%%%%%%%%%%%%%%%%%%%
%  Lecture/Chapter Command  %
%%%%%%%%%%%%%%%%%%%%%%%%%%%%%

\def\notenum{}
\def\@note{}%
\newcommand\lecture[3]{
  \ifthenelse{\isempty{#3}}{%
    \def\@note{Lecture #1}%
  }{%
    \def\@note{Lecture #1: #3}%
  }%
  \section*{\@note\hfill{\small\textnormal{#2}}}
}

% Intro
\newcommand\createintro{
  \maketitle

  \pagenumbering{roman}
  \begin{center}
    \textbf{{\LARGE Introduction}}
    \phantomsection\addcontentsline{toc}{chapter}{0\hspace{0.8em}Introduction}
  \end{center}

  \begingroup
  \IfFileExists{./intro.tex}{
    \setlength{\parindent}{1cm}
    Lecture notes from the course \MyTitle, given by professor Victor Ostrik at the \faculty~at \location~in the academic year \academicyear, during the \term term. This course covers symbolic logic, basic set theory, analyzing functions and their properties, modular arithmetic, counting and other problems in discrete mathematics, induction, and convergence of sequences and continuity of functions. Credit for the material in these notes is due to professor Victor, while the structure is loosely taken from the \href{https://www.amazon.com/Mathematical-Reasoning-Writing-Proof-2nd/dp/0131877186}{Mathematical Reasoning: Writing and Proof} textbook. The credit for the typesetting is my own.

\textit{Disclaimer:} This document will inevitably contain some mistakes--both simple typos and legitimate errors. Keep in mind that these are the notes of an undergraduate student in the process of learning the material himself, so take what you read with a grain of salt. If you find mistakes and feel like telling me, I will be grateful and happy to hear from you, even for the most trivial of errors. You can reach me by email, in English, Arabic, Hebrew, or Spanish at \href{mailto:singularisartt@gmail.com}{singularisartt@gmail.com}.

  }{}
  \endgroup

  \pagestyle{fancy}
  \renewcommand\headrulewidth{0pt}

  \fancyhead{}
  \fancyfoot[C]{%
    \textit{For more notes like this, visit \href{\linktootherpages}{\shortlinkname}}.%
  }%

  \begin{tcolorbox}[
      enhanced,
      colback=white,
      center upper,
      size=fbox,
      drop shadow southwest,
      sharp corners,
    ]
    \term: \academicyear, \\
    Last Update: \today, \\
    \faculty, \location.
  \end{tcolorbox}

  \newpage
  \tableofcontents

  \pagenumbering{arabic}
  \setcounter{page}{1}

  \renewcommand\headrulewidth{0.4pt}
  \fancyhead[R]{\@note}
  \fancyhead[L]{\nauthor}
  \fancyfoot[C]{\thepage}
}

%%%%%%%%%%%%%%%%%%%%%
%  Random Commands  %
%%%%%%%%%%%%%%%%%%%%%

% Circle
\newcommand*\circled[1]{
  \tikz[baseline=(char.base)] {
    \node[shape=circle,draw,inner sep=1pt] (char) {#1};
  }
}

% Import Figures
\newcommand\incimg[2][1]{%
  \includegraphics[width=#1\columnwidth]{figures/#2}%
}

\newcommand\incfig[2][1]{
  \def\svgwidth{#1\columnwidth}
  \import{figures}{#2.pdf_tex}
}

% Correct
\newcommand\correct[1]{\textcolor{correct}{#1}}
\newcommand\incorrect[1]{{\color{incorrect}#1}}
\newcommand\inctocor[2]{\incorrect{#1} \ensuremath{\to} \correct{#2}}

% Bracket
\renewcommand\bra[1]{\left\langle#1\right|}
\renewcommand\ket[1]{\left|#1\right\rangle}
\renewcommand\braket[2]{\left\langle#1\middle|#2\right\rangle}
\renewcommand\ang[1]{\left\langle#1\right\rangle}

% For diagonal strikeout in red
\newcommand\rcancel[1]{\renewcommand\CancelColor{\color{red}}\cancel{#1}}

% Logic
\renewcommand\nmid{\not|~}

% For differentials
\newcommand\dd[1]{\textrm{d}#1}
\newcommand\dA{\textrm{d}A}
\newcommand\dm{\textrm{d}m}
\newcommand\dt{\textrm{d}t}
\newcommand\dT{\textrm{d}T}
\newcommand\du{\textrm{d}u}
\newcommand\dv{\textrm{d}v}
\newcommand\dx{\textrm{d}x}
\newcommand\dy{\textrm{d}y}
\newcommand\dz{\textrm{d}z}

% Helpful text in math
\newcommand\abs{\text{abs}}
\newcommand\echelon{\underrightarrow{\textrm{ echelon form }}}
\newcommand\rref{\underrightarrow{\textrm{ rref }}}
\newcommand\pick[1]{\xrightarrow[#1]{\textrm{ pick }}}
\newcommand\generalsol{\xrightarrow[\textrm{solution}]{\textrm{ general }}}
\newcommand\ngeneralsol{\parbox{4em}{general \\ solution}\textrm{:}}
\renewcommand\and{\text{and}}
\newcommand\aand{\quad\text{and}\quad}
\newcommand\adj{\text{adj}}
\newcommand\qtq[1]{\quad\textrm{#1}\quad}
\newcommand\oor{\quad\text{or}\quad}
\newcommand\Col{\textrm{Col}}
\newcommand\Nul{\textrm{Null}}
\newcommand\range{\textrm{Range}}
\newcommand\Tr{\textrm{Tr}}
\newcommand\Row{\textrm{Row}}
\newcommand\rank{\textrm{rank}}
\newcommand\dist{\text{dist}}
\newcommand\sz{\stackrel{\textrm{set}}{=}}
\newcommand\ce{\overset{\checkmark}{=}}
\newcommand\proj{\text{proj}}
\newcommand\comp{\text{comp}}
\newcommand\Card{\text{Card}}
\renewcommand\mod{\text{mod}}
\newcommand\st{\text{such that}}
\newcommand\geogebra{\textsf{GeoGebra}}
\newcommand\Sspan{\text{Span}}
\renewcommand\Re{\text{Re}}
\renewcommand\Im{\text{Im}}

% Laplace
\newcommand\laplace[1]{\mathscr{L}\left\{#1\right\}}
\newcommand\ilaplace[1]{\mathscr{L}^{-1}\left\{#1\right\}}

% Vectors
\renewcommand\a{\mathbf{a}}
\renewcommand\b{\mathbf{b}}
\renewcommand\c{\mathbf{c}}
\renewcommand\d{\mathbf{d}}
\newcommand\e{\mathbf{e}}
\newcommand\f{\mathbf{f}}
\newcommand\g{\mathbf{g}}
\newcommand\n{\mathbf{n}}
\newcommand\p{\mathbf{p}}

\newcommand\RR{\mathbf{R}}
\newcommand\FF{\mathbf{F}}

\renewcommand\r{\mathbf{r}}
\newcommand\rr{\mathbf{r}^{\prime}}
\newcommand\rrr{\mathbf{r}^{\prime\prime}}

\newcommand\Ta{\mathbf{T}}
\newcommand\Taa{\mathbf{T}^{}}

\renewcommand\u{\mathbf{u}}
\newcommand\uu{\mathbf{u}^{\prime}}
\newcommand\uuu{\mathbf{u}^{\prime\prime}}

\renewcommand\v{\mathbf{v}}
\newcommand\vv{\mathbf{v}^{\prime}}
\newcommand\vvv{\mathbf{v}^{\prime\prime}}

\newcommand\w{\mathbf{w}}
\newcommand\x{\mathbf{x}}
\newcommand\y{\mathbf{y}}
\newcommand\z{\mathbf{z}}

\newcommand\zero{\mathbf{0}}

% Hat vectors
\newcommand\ah{\hat{\mathbf{a}}}
\newcommand\bh{\hat{\mathbf{b}}}
\newcommand\ch{\hat{\mathbf{c}}}
\renewcommand\dh{\hat{\mathbf{d}}}
\newcommand\eh{\hat{\mathbf{e}}}
\newcommand\ph{\hat{\mathbf{p}}}
\newcommand\uh{\hat{\mathbf{u}}}
\newcommand\vh{\hat{\mathbf{v}}}
\newcommand\wh{\hat{\mathbf{w}}}
\newcommand\xh{\hat{\mathbf{x}}}
\newcommand\yh{\hat{\mathbf{y}}}
\newcommand\zh{\hat{\mathbf{z}}}

% Unit vectors
\newcommand\ui{\mathbf{i}}
\newcommand\uj{\mathbf{j}}
\newcommand\uk{\mathbf{k}}

% Subspaces
\newcommand\B{\mathcal{B}}
\newcommand\CC{\mathbb{C}}
\newcommand\C{\mathcal{C}}
\newcommand\D{\mathcal{D}}
\newcommand\E{\mathcal{E}}
\newcommand\F{\mathcal{F}}
\newcommand\I{\mathbb{I}}
\newcommand\N{\mathbb{N}}
\newcommand\Q{\mathbb{Q}}
\newcommand\R{\mathbb{R}}
\newcommand\U{\mathcal{U}}
\newcommand\W{\mathbf{w}}
\newcommand\X{\mathcal{X}}
\newcommand\Y{\mathcal{Y}}
\newcommand\Z{\mathbb{Z}}

% Create command to box equations
\newcommand*\colorboxed{}
\def\colorboxed#1#{%
  \colorboxedAux{#1}%
}
\newcommand*\colorboxedAux[3]{%
  \begingroup
    \colorlet{cb@saved}{.}%
    \color#1{#2}%
    \boxed{%
      \color{cb@saved}%
      #3%
    }%
  \endgroup
}
\newcommand\empheq[1]{%
  \colorboxed{black}{%
    \begin{aligned}[b]
      #1
    \end{aligned}%
  }%
}


%%%%%%%%%%%%%%%%%%%%%%%%%%%%%%%%%%%%%%%%%%%%%%%%%%%%%%%%%%%%%%%%%%%%%%%%%%%%%%%%
%                                                                              %
%                                 Environments                                 %
%                                                                              %
%%%%%%%%%%%%%%%%%%%%%%%%%%%%%%%%%%%%%%%%%%%%%%%%%%%%%%%%%%%%%%%%%%%%%%%%%%%%%%%%

\theoremstyle{definition}
\newtheorem*{assumption}{Assumption}
\newtheorem*{claim}{Claim}
\newtheorem*{conjecture}{Conjecture}
\newtheorem*{definition}{Definition}
\newtheorem*{example}{Example}
\newtheorem*{notation}{Notation}
\newtheorem*{proposition}{Proposition}
\newtheorem*{question}{Question}
\newtheorem*{problem}{Problem}
\newtheorem*{rrule}{Rule}

\newtheorem*{remark}{Remark}
\newtheorem*{note}{Note}
\newtheorem*{warning}{Warning}

\theoremstyle{plain}
\newtheorem*{corollary}{Corollary}
\newtheorem*{lemma}{Lemma}
\newtheorem*{theorem}{Theorem}
\newtheorem*{worksheet}{Worksheet}

\newcommand\frmebox[1][]{%
  \begin{tcolorbox}[%
    title={\sffamily\color{black}#1},
    colback=white,
    enhanced,
    attach boxed title to top center={
      yshift=-3mm,
      yshifttext=-1mm,
    },
    boxed title style={
      size=small,
      colback=white,
      frame code={},
    },
    coltext=black,
    frame hidden,
    borderline east={0.5pt}{0pt}{black},
    borderline west={0.5pt}{0pt}{black},
    borderline north={0.5pt}{0pt}{black},
    borderline south={0.5pt}{0pt}{black},
    breakable,
    parskip=0pt,
  ]
}

\renewenvironment{frame}[1][]{%
  \frmebox[#1]%
}{%
  \end{tcolorbox}%
}

\makeatother


\def\nshort{DRP}
\def\ncourse{Directed Reading Program}
\def\nlecturer{A. Yae}
\def\nterm{Winter}
\def\nyear{$2026$}

\def\othernotes{singularisart.github.io/notes}
\def\faculty{Faculty of Mathematics}
\def\location{University of Oregon}

\renewcommand\S{\mathbb{S}}

\begin{document}
  \section*{Problems}

  \begin{enumerate}
    \item Let $z = x + iy$ be the standard coordinate on $\C$. Show that the standard volume form (symplectic form) $\omega = \dx \wedge \dy$ can be written as $\frac{1}{2i} \dd{\bar{z}} \wedge \dz$, whereas the standard metric $\dx^2 + \dy^2$ is equal to $\dd{\bar{z}} \cdot \dz$ (here $\cdot$ is understood to be a symmetric product). The latter is what the M.R.S. paper denotes $|\dz|^2$.

      \textbf{Solution:} Computing $\dd{\bar{z}}$ and $\dz$, we have
      \begin{align*}
        \dd{\bar{z}} &= \dx - i\dy \\
        \dz &= \dx + i\dy
      .\end{align*}
      Therefore,
      \begin{align*}
        \dd{\bar{z}} \wedge \dz &= (\dx - i\dy) \wedge (\dx + i\dy) \\
                              &= 2i \dx \wedge \dy
      .\end{align*}
      This shows that
      \[%
        \omega = \dx \wedge \dy = \frac{1}{2i} \dd{\bar{z}} \wedge \dz
      .\]%
      Next, we compute the symmetric product:
      \begin{align*}
        \dd{\bar{z}} \cdot \dz &= (\dx - i\dy) \cdot (\dx + i\dy) \\
                             &= \dx^2 + \dy^2
      .\end{align*}
      This shows that the standard metric is given by
      \[%
        \dx^2 + \dy^2 = \dd{\bar{z}} \cdot \dz
      .\tag*{$\square$}\]%

    \item Let $\beta \in (0, 1]$. Under the map $f_\beta : \C \to \C$ given by $z \mapsto z^\beta$, check that $f_\beta^*(\omega) = \beta^2 r^{2\beta-2}\omega$, which is a rescaling of $r^{2\beta-2}\omega$. See if you can understand what this has to do with a cone.

      \textbf{Solution:} The form $\omega$ describes the area in $\C$, which is given by $\omega = \dx \wedge \dy$. Converting to polar coordinates, we have $\omega = r \dr \wedge \dd{\theta}$. This gives us $f_\beta(r, \theta) = (r^\beta, \beta\theta)$. The pullback is a map from $f_\beta^* : \Omega^k(N) \to \Omega^k(M)$, where $\Omega^k(N)$ is the space of $k$-forms on the manifold $N$.

      Writing $(u^1, u^2) = (r^\beta, \beta\theta)$, the pullback satisfies
      \begin{equation}\label{eq:pullback_wedge}
        f_\beta^*\left(\bigwedge_{i=1}^k \du^i\right) = \bigwedge_{i=1}^k f(\du^i) = \bigwedge_{i=1}^k \pd{j} u^i \dx^j
      ,\end{equation}
      where $\dx^1 = \dr$ and $\dx^2 = \dd{\theta}$ (we're using Einstein's notation to simplify the expression). Computing the individual pullbacks, we have
      \[%
        f_\beta^*(\du^1) = \pd{1} u^1 \dr = \beta r^{\beta-1} \dr \aand f_\beta^*(\du^2) = \pd{2} u^2 \dd{\theta} = \beta \dd{\theta}
      .\]%
      Using Equation~\eqref{eq:pullback_wedge}, we compute
      \begin{align*}
        f_\beta^*(\omega) &= f_\beta^*(\dr \wedge \dd{\theta}) \\
                          &= f_\beta^*(\dr) \wedge f_\beta^*(\dd{\theta}) \\
                          &= (\beta r^{\beta-1} \dr) \wedge (\beta \dd{\theta}) \\
                          &= \beta^2 r^{2\beta-1} \omega
      .\end{align*}
      But since we have
      \[%
        \dr \wedge \dd{\theta} = \frac{1}{r} \omega
      ,\]%
      we have
      \[%
        f_\beta^*(\omega) = \beta^2 r^{2\beta-2} \omega
      .\]%

      The pullback form $f_\beta^*(\omega)$ describes the area in the new coordinates. When $\beta = 1$, we have the standard area form. However, when $0 < \beta < 1$, we see that the area form is scaled by a factor of $r^{2\beta-2}$, which blows up as $r \to 0$. This indicates that near the origin, the geometry is distorted in such a way that it resembles a cone with a singularity at the tip. The angle around the origin is effectively reduced, leading to a conical structure. \hfill$\square$

    \item Consider the maps: $\phi : \R_{\ge 0} \times \S^1 \to \C$ given by $\phi(r, \theta) = re^{i\theta}$ and $\tau : \R \times \S^1 \to \R_{\ge 0} \times \S^1$ given by $\tau(y, \theta) = (e^y, \theta)$. Calculate the pullback forms, $\phi^*(\omega)$ and then $\tau^*(\phi^*(\omega))$.

      \textbf{Solution:} We first compute the pullback of $\phi^*(\omega)$, where $\omega = \dx \wedge \dy$. Writing $(u^1, u^2) = (r, \theta)$, computing the individual pullbacks using the chain rule, we have
      \begin{gather*}
        \phi^*(\du^1) = \pd{1} u^1 \dr + \pd{2} u^1 \dd{\theta} = \cos(\theta) \dr - r \sin(\theta) \dd{\theta} \\
        \phi^*(\du^2) = \pd{1} u^2 \dr + \pd{2} u^2 \dd{\theta} = \sin(\theta) \dr + r \cos(\theta) \dd{\theta}
      \end{gather*}
      Using Equation~\eqref{eq:pullback_wedge}, we compute
      \begin{align*}
        \phi^*(\omega) &= \phi^*(\dx \wedge \dy) \\
                      &= \phi^*(\dx) \wedge \phi^*(\dy) \\
                      &= (\cos(\theta) \dr - r \sin(\theta) \dd{\theta}) \wedge (\sin(\theta) \dr + r \cos(\theta) \dd{\theta}) \\
                      &= r (\cos^2(\theta) + \sin^2(\theta)) \dr \wedge \dd{\theta} \\
                      &= r \dr \wedge \dd{\theta}
      .\end{align*}

      Next, we compute the pullback of $\tau^*(\phi^*(\omega))$. Writing $(v^1, v^2) = (y, \theta)$, computing the individual pullbacks, we have
      \begin{gather*}
        \tau^*(\dv^1) = \pd{1} v^1 \dy + \pd{2} v^1 \dd{\theta} = e^y \dy \\
        \tau^*(\dv^2) = \pd{1} v^2 \dy + \pd{2} v^2 \dd{\theta} = \dd{\theta}
      \end{gather*}
      Using Equation~\eqref{eq:pullback_wedge}, we compute
      \begin{align*}
        \tau^*(\phi^*(\omega)) &= \tau^*(r \dr \wedge \dd{\theta}) \\
                                &= \tau^*(r \dr) \wedge \tau^*(\dd{\theta}) \\
                                &= e^y (e^y \dy) \wedge \dd{\theta} \\
                                &= e^{2y} \dy \wedge \dd{\theta}
      .\end{align*}
      Therefore, we have
      \[%
        \phi^*(\omega) = r \dr \wedge \dd{\theta} \aand \tau^*(\phi^*(\omega)) = e^{2y} \dy \wedge \dd{\theta}
      .\tag*{$\square$}\]%

    \item Now let $g_\beta = r^{2\beta-2}|\dz|^2$ be the canonical metric from the M.R.S. paper, which has corresponding symplectic form
      \[%
        \omega_\beta = \frac{r^{2\beta-2}}{2i} \dd{\bar{z}} \wedge \dz = r^{2\beta-2} \dr \wedge \dd{\theta}
      .\]%
      (Verify the last equality). Calculate these pullbacks $\phi^*(\omega_\beta)$ and $\tau^*(\phi^*(\omega_\beta))$.

      \textbf{Solution:} We first verify the last equality. We have
      \begin{align*}
        \dd{z} &= e^{i\theta} \dr + ire^{i\theta} \dd{\theta} \\
        \dd{\bar{z}} &= e^{-i\theta} \dr - ire^{-i\theta} \dd{\theta}
      .\end{align*}
      Therefore,
      \begin{align*}
        \frac{1}{2i} \dd{\bar{z}} \wedge \dz &= \frac{1}{2i} (e^{-i\theta} \dr - ire^{-i\theta} \dd{\theta}) \wedge (e^{i\theta} \dr + ire^{i\theta} \dd{\theta}) \\
                                           &= \frac{1}{2i} (ire^{-i\theta} e^{i\theta} + ire^{i\theta} e^{-i\theta}) \dr \wedge \dd{\theta} \\
                                           &= \dr \wedge \dd{\theta}
      .\end{align*}
      Thus, we have
      \[%
        \omega_\beta = r^{2\beta-1} \dr \wedge \dd{\theta}
      .\]%

      Now, we calculate the pullback of $\phi^*(\omega_\beta)$. Using the previous result for $\phi^*(\dr \wedge \dd{\theta})$, we have
      \begin{align*}
        \phi^*(\omega_\beta) &= \phi^*(r^{2\beta-2} \dr \wedge \dd{\theta}) \\
                            &= r^{2\beta-1} \phi^*(\dr \wedge \dd{\theta}) \\
                            &= r^{2\beta-1} (r \dr \wedge \dd{\theta}) \\
                            &= r^{2\beta-1} \dr \wedge \dd{\theta}
      .\end{align*}
      Next, we compute the pullback of $\tau^*(\phi^*(\omega_\beta))$. Using the previous result for $\tau^*(\dr \wedge \dd{\theta})$, we have
      \begin{align*}
        \tau^*(\phi^*(\omega_\beta)) &= \tau^*(r^{2\beta-1} \dr \wedge \dd{\theta}) \\
                                    &= e^{(2\beta-1)y} \tau^*(\dr \wedge \dd{\theta}) \\
                                    &= e^{(2\beta-1)y} (e^y \dy \wedge \dd{\theta}) \\
                                    &= e^{2\beta y} \dy \wedge \dd{\theta}
      .\end{align*}
      Therefore, we have
      \[%
        \phi^*(\omega_\beta) = r^{2\beta-1} \dr \wedge \dd{\theta} \aand \tau^*(\phi^*(\omega_\beta)) = e^{2\beta y} \dy \wedge \dd{\theta}
      .\tag*{$\square$}\]%

    \item (Harder) Calculate the pushforward of the vector fields $\pd{z}$ and $\pd{\bar{z}}$ under the inverse map $\phi^{-1}$. Use this to derive the formula for the Laplacian in polar coordinates
      \[%
        \Delta = \pd{rr} + \frac{1}{r} \pd{r} + \frac{1}{r^2} \pd{\theta\theta}
      .\]%

      \textbf{Solution:} Computing the inverse map, we have $\phi^{-1}(x, y) = (r, \theta)$, where $r = \sqrt{x^2 + y^2}$ and $\theta = \arctan(y/x)$. Letting $z = x + iy$ and $\bar{z} = x - iy$.
      Computing the partials, we have
      \[%
        \pd{z} = \frac{1}{2} \left(\pd{x} - i\pd{y}\right) \aand \pd{\bar{z}} = \frac{1}{2} \left(\pd{x} + i\pd{y}\right)
      .\]%
      Solving for $x$ and $y$, we have
      \begin{align*}
        x = \frac{z + \bar{z}}{2} \aand y = \frac{z - \bar{z}}{2i}
      .\end{align*}
      Computing $\pd{x}$ and $\pd{y}$ in terms of $\pd{z}$ and $\pd{\bar{z}}$, we have
      \begin{align*}
        \pd{x} = \pd{x} r \pd{r} + \pd{x} \theta \pd{\theta} = \cos(\theta) \pd{r} - \frac{\sin(\theta)}{r} \pd{\theta} \\
        \pd{y} = \pd{y} r \pd{r} + \pd{y} \theta \pd{\theta} = \sin(\theta) \pd{r} + \frac{\cos(\theta)}{r} \pd{\theta}
      .\end{align*}
      Plugging these values into the expressions for $\pd{z}$ and $\pd{\bar{z}}$, we have
      \begin{align*}
        \pd{z} &= \frac{1}{2} \left(\cos(\theta) \pd{r} - \frac{\sin(\theta)}{r} \pd{\theta} - i\sin(\theta) \pd{r} - i\frac{\cos(\theta)}{r} \pd{\theta}\right) \\
               &= \frac{e^{-i\theta}}{2} \left(\pd{r} - \frac{i}{r} \pd{\theta}\right) \\
        \pd{\bar{z}} &= \frac{1}{2} \left(\cos(\theta) \pd{r} - \frac{\sin(\theta)}{r} \pd{\theta} + i\sin(\theta) \pd{r} + i\frac{\cos(\theta)}{r} \pd{\theta}\right) \\
                     &= \frac{e^{i\theta}}{2} \left(\pd{r} + \frac{i}{r} \pd{\theta}\right)
      .\end{align*}
      Computing $\pd{z} \pd{\bar{z}}$, we have
      \[%
        \pd{z,\bar{z}} = \frac{1}{4} e^{-i\theta} e^{i\theta} \left(\pd{r} - \frac{i}{r} \pd{\theta}\right)\left(\pd{r} + \frac{i}{r} \pd{\theta}\right) = D
      .\]%
      Computing $Df$ for a smooth function $f$, we have
      \begin{align*}
        4Df &= \left(\pd{rr} f + \frac{i}{r} \pd{r\theta} f - \frac{i}{r} \pd{\theta r} f + \frac{1}{r^2} \pd{\theta\theta} f\right) \\
            &= \pd{r} \left(\pd{r}f + \frac{i}{r}\pd{\theta}f\right) - \frac{i}{r}\pd{\theta}\left(\pd{r}f + \frac{i}{r}\pd{\theta}f\right) \\
            &= \pd{rr}f + \pd{r} \left(\frac{i}{r}\pd{\theta}f\right) - \frac{i}{r}\pd{\theta,r}f + \frac{1}{r^2}\pd{\theta\theta}f \\
            &= \pd{rr}f - \frac{i}{r^2} \pd{\theta}f + \frac{i}{r}\pd{r,\theta}f - \frac{i}{r}\pd{\theta,r}f + \frac{1}{r^2}\pd{\theta\theta}f \\
            &= \pd{rr}f - \frac{i}{r^2} \pd{\theta}f + \frac{1}{r^2}\pd{\theta\theta}f
      .\end{align*}

    \item When evaluating $\Delta$ on a radially symmetric function, we can ignore the last term $\frac{1}{r^2} \pd{\theta\theta}$. Calculate the pushforward of $\pd{rr} + \frac{1}{r} \pd{r}$ under the inverse map $\tau^{-1}(r, \theta) = (\log(r), \theta)$.

      \textbf{Solution:} We have $\tau^{-1}(r, \theta) = (\log(r), \theta)$. We want to compute the pushforward of the vector fields $\pd{r}$ and $\pd{rr}$ under this map. Using the chain rule, we have
      \begin{align*}
        \pd{r} &= \pd{r}(\log(r)) \pd{y} + \pd{r}(\theta) \pd{\theta} = \frac{1}{r} \pd{y} \\
        \pd{rr} &= \pd{r}\left(\frac{1}{r} \pd{y}\right) = \frac{1}{r^2} \pd{yy} - \frac{1}{r^2} \pd{y}
      .\end{align*}
      Therefore, we have
      \begin{align*}
        \pd{rr} + \frac{1}{r} \pd{r} &= \left(\frac{1}{r^2} \pd{yy} - \frac{1}{r^2} \pd{y}\right) + \frac{1}{r} \left(\frac{1}{r} \pd{y}\right) \\
                                     &= \frac{1}{r^2} \pd{yy}
      .\end{align*}
      Thus, the pushforward of $\pd{rr} + \frac{1}{r} \pd{r}$ under the inverse map $\tau^{-1}$ is given by
      \[%
        \tau_*^{-1}(\Delta) = e^{-2y} \pd{yy}
      .\tag*{$\square$}\]%

    \item Now incorporate the $\pd{\theta\theta}$ term in the above to recover the standard Laplacian on the infinite cylinder $C = \R \times \S^1$.

      \textbf{Solution:}

    \item Compute the line integral of a geodesic on the cone with the metric $g_\beta$ around the singularity. (Hint: such a geodesic is given by a straight line in the $y, \theta$ coordinates on the infinite cylinder).

      \textbf{Solution:}
  \end{enumerate}
\end{document}
