\begin{problem}[1 (Lorentz Transformation)]
  \begin{enumerate}
    \item (2 points) The coordinate transformation for an observer moving at a constant velocity Vx relative to a stationary observer is given by
      \begin{align*}
        t' &= \gamma_x(t - V_xx) \\
        x' &= \gamma_x(x - V_xt) \\
        y' &= y \\
        z' &= z
      .\end{align*}
      How can we write this as a $4 \times 4$ Lorentz transformation matrix?

    \item (5 points) More generally, suppose we have
      \begin{align*}
        t' &= \mu t + + \nu x \\
        x' &= \sigma t + \gamma x \\
        y' &= \rho y \\
        z' &= \lambda z
      .\end{align*}
      How can we write this as a $4 \times 4$ Lorentz transformation matrix?
  \end{enumerate}
\end{problem}

\begin{solution}[(i)]
  Given the coordinate transformations, we can express them in matrix form as follows
  \[%
    \begin{pmatrix}
      t' \\
      x' \\
      y' \\
      z'
    \end{pmatrix}
    =
    \begin{pmatrix}
      \gamma_x & -\gamma_x V_x & 0 & 0 \\
      -\gamma_x V_x & \gamma_x & 0 & 0 \\
      0 & 0 & 1 & 0 \\
      0 & 0 & 0 & 1
    \end{pmatrix}
    \begin{pmatrix}
      t \\
      x \\
      y \\
      z
    \end{pmatrix}
  .\qedhere\]%
\end{solution}

\begin{solution}[(ii)]
  More generally, we can express the given coordinate transformations in matrix form as follows
  \[%
    \begin{pmatrix}
      t' \\
      x' \\
      y' \\
      z'
    \end{pmatrix}
    =
    \begin{pmatrix}
      \mu & \nu & 0 & 0 \\
      \sigma & \gamma & 0 & 0 \\
      0 & 0 & \rho & 0 \\
      0 & 0 & 0 & \lambda
    \end{pmatrix}
    \begin{pmatrix}
      t \\
      x \\
      y \\
      z
    \end{pmatrix}
  .\qedhere\]%
\end{solution}

\begin{problem}[2 (Coordinate Transformations)]
  Let
  \[%
    x = r\cos(\theta) \aand y = r\sin(\theta)
  .\]%
  \begin{enumerate}
    \item (2 points) Compute $\pdv{x}/{r}$, $\pdv{x}/{\theta}$, $\pdv{y}/{r}$, and $\pdv{y}/{\theta}$.
    \item (2 points) Use the chain rule to express $\pdv{}/{r}$ in terms of $\pdv{}/{x}$ and $\pdv{}/{y}$.
    \item (2 points) Interpret $\pdv{}/{r}$ geometrically.
  \end{enumerate}
\end{problem}

\begin{solution}[(i)]
  Computing the partial derivatives, we have
  \begin{gather*}
    \pdv{x}{r} = \cos(\theta), \quad \pdv{x}{\theta} = -r\sin(\theta) \\
    \pdv{y}{r} = \sin(\theta), \quad \pdv{y}{\theta} = r\cos(\theta)
  .\qedhere\end{gather*}
\end{solution}

\begin{solution}[(ii)]
  Using the Chain Rule, we have
  \[%
    \pdv{}{r} = \pdv{x}{r}\pdv{}{x} + \pdv{y}{r}\pdv{}{y} = \cos(\theta)\pdv{}{x} + \sin(\theta)\pdv{}{y}
  .\qedhere\]%
\end{solution}

\begin{solution}[(iii)]
  Geometrically, $\pdv{}/{r}$ represents the rate of change of a function as we move radially outward from the origin in the direction specified by the angle $\theta$. It combines the contributions from both the $x$ and $y$ directions, weighted by the cosine and sine of the angle $\theta$, respectively. This operator shows how a function changes as we increase our distance from the origin while maintaining a constant angle $\theta$.
\end{solution}

\begin{problem}[3 (Differentials and Geometry)]
  Given the scalar function $f(x, y) = x^2y$:
  \begin{enumerate}
    \item (3 points) Compute the differential $\dd{f}$.
    \item (2 points) Interpret $\dd{f}$ geometrically.
  \end{enumerate}
\end{problem}

\begin{solution}[(i)]
  The differential of $f$ is given by
  \[%
    \dd{f} = \pdv{f}{x}\dd{x} + \pdv{f}{y}\dd{y} = 2xy\dd{x} + x^2\dd{y}
  .\qedhere\]%
\end{solution}

\begin{solution}[(ii)]
  Geometrically, the differential $\dd{f}$ represents the approximate change in the function $f$ resulting from small changes in the variables $x$ and $y$. Specifically, it quantifies how much $f$ will change when we make infinitesimal adjustments $\dd{x}$ and $\dd{y}$ to the coordinates. The terms $2xy\dd{x}$ and $x^2\dd{y}$ indicate how sensitive the function is to changes in each variable, weighted by their respective partial derivatives.
\end{solution}

\begin{problem}[4 (Change of Basis)]
  Write the $2 \times 2$ rotation matrix $R(\theta)$ and show how vector components transform.
\end{problem}

\begin{solution}
  The $2 \times 2$ rotation matrix $R(\theta)$ is given by
  \[%
    R(\theta) = \begin{pmatrix}
      \cos(\theta) & -\sin(\theta) \\
      \sin(\theta) & \cos(\theta)
    \end{pmatrix}
  .\]%
  Taking a vector, say $\v = (v_x~v_y)^T$, its components transform under rotation as follows
  \[%
    \v' = R(\theta)\v = \begin{pmatrix}
      \cos(\theta) & -\sin(\theta) \\
      \sin(\theta) & \cos(\theta)
    \end{pmatrix}
    \begin{pmatrix}
      v_x \\
      v_y
    \end{pmatrix}
    =
    \begin{pmatrix}
      v_x\cos(\theta) - v_y\sin(\theta) \\
      v_x\sin(\theta) + v_y\cos(\theta)
    \end{pmatrix}
  .\]%
  Thus, the new components of the vector after rotation are
  \[%
    v_x' = v_x\cos(\theta) - v_y\sin(\theta) \aand v_y' = v_x\sin(\theta) + v_y\cos(\theta)
  .\qedhere\]%
\end{solution}

\begin{problem}[5 (Eotvos Experiment)]
  Consider a torsion balance consisting of two equal inertial masses mounted at opposite ends of a light horizontal rod and suspended from a thin fiber. The apparatus is fixed to the surface of the rotating Earth at latitude $\theta$.
  \begin{enumerate}
    \item (3 points) Define a convenient coordinate system for analyzing the torsion balance. Using this coordinate system, state the condition that must be satisfied for the torsion balance to be in mechanical equilibrium.
    \item (3 points) Determine the torque on the torsion balance arising from the Earth's rotation. Your answer should involve the Earth's angular velocity $\Omega$, the latitude $\theta$, and the relevant geometric parameters of the apparatus.
    \item (2 points) What is the optimal latitude to perform this experiment? In other words, at what latitude would one expect to measure the largest potential torque?
    \item (2 points) State the condition required for the weak equivalence principle to be satisfied for the two test masses in this experiment.
  \end{enumerate}
\end{problem}

\renewcommand\O{\boldsymbol{\Omega}}
\begin{solution}[(i)]
  Define the following coordinate system:
  \begin{enumerate}
    \item $\hat{z}$: be vertical (pointing upwards)
    \item $\hat{x}$: point towards the North
    \item $\hat{y}$: point towards the East
  \end{enumerate}
  The Earth's rotation vector decomposes as
  \[%
    \O = \Omega(\cos(\theta)\hat{z} + \sin(\theta)\hat{x})
  .\]%
  The angular coordinate $\phi$ measures the rod's orientation relative to $\hat{z}$. For mechanical equilibrium, the net torque on the torsion balance must be zero:
  \[%
    \tau_{\text{net}} = \tau_{\text{Earth}} - \kappa\phi = 0
  ,\]%
  which gives us that $\tau_{\text{Earth}} = \kappa\phi$.
\end{solution}

\begin{solution}[(ii)]
  Each mass experiences a centrifugal acceleration given by
  \[%
    \a_c = \O \times (\O \times \r)
  ,\]%
  where $\r$ is the position of each mass in the local frame. Only the horizontal component contributes to a torque about the suspension axis. Each mass produces an identical torque magnitude, giving us
  \[%
    \a_c = 2ma^2\O^2\sin(\theta)\cos(\theta)\sin(2\phi)
  .\]%
  For small enough $\phi$, we have $\sin(2\phi) \approx 2\phi$. Therefore, we have
  \[%
    \tau_{\text{Earth}} = 4ma^2\O^2\sin(\theta)\cos(\theta)\phi
  .\qedhere\]%
\end{solution}

\begin{solution}[(iii)]
  The optimal latitude to perform this experiment is at $\theta = \pi/2$, where the product $\sin(\theta)\cos(\theta)$ is maximized.
\end{solution}

\begin{solution}[(iv)]
  An equivalent statement to the weak equivalence principle is
  \[%
    \frac{m_{q,1}}{m_{i,1}} = \frac{m_{q,2}}{m_{i,2}}
  ,\]%
  where $m_{q,i}$ is the gravitational mass and $m_{i,i}$ is the inertial mass of the $i$-th test mass. This condition ensures that both masses experience the same acceleration in a gravitational field, leading to no differential torque on the torsion balance.
\end{solution}

\begin{problem}[6 (Thomas Precession)]
  Suppose that students $A$ and $B$ start out in the lab frame. Student $A$ remains in the lab frame. Student $B$ gets in a rocket ship and changes their velocity by $V_x$ in the $x$-direction (with $V_x \ll 1$) by firing the ``$+X$'' thrusters on their rocket. Then they change their velocity by $V_y$ in the new $y$-direction (again with $V_y \ll 1$) by firing the ``$+Y$'' thrusters. Then, they fire the ``$-X$'' thrusters (changing their velocity by $-V_x$), and the ``$-Y$'' thrusters (changing their velocity by $-V_y$). To lowest order in the $V$s, how does $B$'s reference frame now differ from $A$'s?

  \textit{Note: The problem is designed so that in Newtonian physics, $B$'s final reference frame is the same as $A$'s: they are at rest with respect to each other, with no rotation of the coordinate systems.}
\end{problem}

\begin{solution}
  Define the following Lorentz transformation matrices for each boost
  \[%
    \Lambda_x(V_x) = \begin{pmatrix}
      1 + \frac{1}{2}V_x^2 & -V_x & 0 & 0 \\
      -V_x & 1 + \frac{1}{2}V_x^2 & 0 & 0 \\
      0 & 0 & 1 & 0 \\
      0 & 0 & 0 & 1 \\
    \end{pmatrix} + \mathcal{O}(V_y^2)
    \aand
    \Lambda_y(V_y) = \begin{pmatrix}
      1 + \frac{1}{2}V_y^2 & 0 & -V_y & 0 \\
      -V_y & 1 & 1 + \frac{1}{2}V_y^2 & 0 \\
      0 & 0 & 1 & 0 \\
      0 & 0 & 0 & 1 \\
    \end{pmatrix} + \mathcal{O}(V_y^2)
  .\]%
  The total transformation after the sequence of boosts is given by
  \[%
    \Lambda(V) = \Lambda_y(-V_y)\Lambda_x(-V_x)\Lambda_y(V_y)\Lambda_x(V_x)
  .\]%
  Notice that we can expand $\Lambda_x(V_x)$ and $\Lambda_y(V_y)$ to first order in $V_x$ and $V_y$ to get
  \[%
    \Lambda_x(V_x) = 1 + A \aand \Lambda_y(V_y) = 1 + B
  ,\]%
  where we define
  \[%
    A = \begin{pmatrix}
      0 & -V_x & 0 & 0 \\
      -V_x & 0 & 0 & 0 \\
      0 & 0 & 0 & 0 \\
      0 & 0 & 0 & 0 \\
    \end{pmatrix}
    \aand
    B = \begin{pmatrix}
      0 & 0 & -V_y & 0 \\
      0 & 0 & 0 & 0 \\
      -V_y & 0 & 0 & 0 \\
      0 & 0 & 0 & 0 \\
    \end{pmatrix}
  .\]%
  Computing $\Lambda(V)$, we get
  \[%
    \Lambda(V) = (1 - B)(1 - A)(1 + B)(1 + A) = 1 + [B, A] + \mathcal{O}(V)
  .\]%
  Computing the commutator $[B, A]$, we find
  \[%
    \Lambda(V) = I + \begin{pmatrix}
      0 & 0 & 0 & 0 \\
      0 & 0 & -V_xV_y & 0 \\
      0 & V_xV_y & 0 & 0 \\
      0 & 0 & 0 & 0 \\
    \end{pmatrix} + \mathcal{O}(V^3)
  .\]%
  Therefore, this new coordinate system is rotated with respect to the original one by an angle $\omega_z = V_xV_y$ about the $xy$-plane.
\end{solution}

\begin{problem}[7 (The sky as viewed from a spaceship)]
  ($\star$) Let's suppose that observer $\mathcal{O}$ remains on Earth in the lab frame. A second observer $\bar{\mathcal{O}}$ moves in a spaceship at velocity $V = \tanh(\alpha)$ in the $z$-direction with respect to Earth. As you may recall from watching science-fiction movies, if $V$ is large enough, $\bar{\mathcal{O}}$ sees the stars bunch up in front of them (the $+z$ direction). This problem works through the effect.

  We suppose that the direction to the star makes an angle $\theta$ to the $z$-axis as seen from Earth, and $\bar{\theta}$ as seen from the spaceship. Without loss of generality, we will place the direction to the star at zero longitude (i.e., in the $xz$-plane).
  \begin{enumerate}
    \item (2 points) Show that in the Earth’s frame, in time $\Delta t$, a photon from the star undergoes a displacement $\Delta x^\alpha = (\Delta t, -\Delta t\sin(\theta), 0, -\Delta t\cos(\theta))$.
    \item (5 points) Apply a Lorentz transformation to find the photon’s displacement in $\bar{\mathcal{O}}$'s frame. You may leave some results in terms of $\gamma = 1/\sqrt{1 - V^2}$. Show that in the barred frame, the direction of the photon satisfies
      \[%
        \cos(\bar{\theta}) = \frac{V + \cos(\theta)}{1 + V\cos(\theta)}
      .\]%
    \item (3 points) Show that a star that appears on the ``Equator'' as seen from Earth ($\theta = \pi/2$)  has an apparent position $\bar{\theta} = \arccos(V)$ as seen from the spaceship. How far from the North Pole does the star appear in the spaceship frame if $V = 0.9c$? What about $0.99c$?
    \item (5 points) Now take the limit of small  $\theta \ll 1$ (i.e., we will consider a constellation that contains the North Pole). Show that
      \[%
        \bar{\theta} \approx \sqrt{\frac{1 - V}{1 + V}\theta}
      .\]%
      \textit{Hint: Take the Taylor expansion of your answer to (i) to 2nd order in $\theta$. This means that the constellation containing the North Pole appears shrunk by a factor of $\sqrt{(1 - V)/(1 + V)}$ when seen from the spaceship.}
  \end{enumerate}
\end{problem}

\begin{solution}[(i)]
  The photon's displacement in the Earth's frame over a time interval $\Delta t$ is defined to be $\Delta x^0 = \Delta t$. Since the particle is traveling at the speed of light, the magnitude of its spatial displacement must equal the time interval, $\Delta t$. The given direction, $(\sin(\theta), \cos(\theta), 0)$ is a unit vector. But this is the direction of the photon, so we have
  \[%
    \Delta x^i = -\Delta t(\sin(\theta), 0, \cos(\theta)) = (-\Delta t\sin(\theta), 0, -\Delta t\cos(\theta))
  .\]%
  Therefore, the full displacement four-vector is
  \[%
    \Delta x^\alpha = (\Delta t, -\Delta t\sin(\theta), 0, -\Delta t\cos(\theta))
  .\qedhere\]%
\end{solution}

\begin{solution}[(ii)]
  Setting up the Lorentz transformation matrix for a boost in the $z$-direction, we have
  \[%
    \Lambda = \begin{pmatrix}
      \gamma & -\gamma V & 0 & 0 \\
      -\gamma V & \gamma & 0 & 0 \\
      0 & 0 & 1 & 0 \\
      0 & 0 & 0 & 1 \\
    \end{pmatrix}
  .\]%
  Therefore, we have the new displacement four-vector in the spaceship frame as
  \[%
    \Delta \bar{x}^\alpha = \left(\gamma (\Delta t - V \Delta z), - \Delta t\sin(\theta), 0, \gamma(\Delta z - V \Delta t)\right)
  .\]%
  In the barred frame, the photon travels at $c = 1$, so its spatial direction is given by
  \[%
    \cos(\bar{\theta}) = \frac{\Delta \bar{z}}{(\Delta \bar{x})^2 + (\Delta \bar{z})^2}
  .\]%
  Using the fact that $\sqrt{(\Delta \bar{x})^2 + (\Delta \bar{z}} = \Delta \bar{t}$, since the photon travels at the speed of light, we have
  \[%
    \cos(\bar{\theta}) = \frac{\Delta \bar{z}}{\Delta \bar{t}}
  .\]%
  Expanding and solving, we find that
  \[%
    \cos(\bar{\theta}) = \frac{\Delta z - V \Delta t}{\Delta t - V \Delta z} = \frac{-\Delta t\cos(\theta) - V \Delta t}{\Delta t - V(-\Delta t\cos(\theta))} = \frac{V + \cos(\theta)}{1 + V\cos(\theta)}
  .\qedhere\]%
\end{solution}

\begin{solution}[(iii)]
  Plugging $\theta = \pi/2$ into the formula from part (ii), we have
  \[%
    \cos(\bar{\theta}) = \frac{V + 0}{1 + V \cdot 0} = V
  .\]%
  This gives us that $\bar{\theta} = \arccos(V)$. If $V = 0.9c$, then we get $\bar{\theta} \approx 25.9^{\circ}$ and if $V = 0.99c$, then we get $\bar{\theta} \approx 8.1^{\circ}$.
\end{solution}

\begin{solution}[(iv)]
  Taking the Taylor series expansion of $\cos(\theta)$ to second order in $\theta$, we have
  \begin{align*}
    \cos(\bar{\theta}) &= \frac{V + \cos(\theta)}{1 + V\cos(\theta)} \\
                       &= \frac{V + 1 - \frac{1}{2}\theta^2}{1 + V - \frac{1}{2}V\theta^2} + \mathcal{O}(\theta^4) \\
                       &= 1 - \frac{\frac{1}{2}(1 - V)\theta^2}{1 + V - \frac{1}{2}V\theta^2} + \mathcal{O}(\theta^4) \\
                       &\approx 1 - \frac{1}{2}\frac{(1 - V)}{(1 + V)}\theta^2 + \mathcal{O}(\theta^4)
  .\end{align*}
  Taking the Taylor series expansion of $\arccos(x)$ about $x = 1$, we have
  \[%
    \bar{\theta} = \arccos\left(1 - \frac{1}{2}\frac{(1 - V)}{(1 + V)}\theta^2\right) \approx \sqrt{\frac{1 - V}{1 + V}}\theta + \mathcal{O}(\theta^3)
  .\]%
  Thus, we have shown that for small $\theta$, the angle in the spaceship frame is approximately
  \[%
    \bar{\theta} \approx \sqrt{\frac{1 - V}{1 + V}}\theta
  .\qedhere\]%
\end{solution}
