\begin{problem}[1]
  \textit{Lorentz Transformation}
  \begin{enumerate}
    \item (2 points) The coordinate transformation for an observer moving at a constant velocity Vx relative to a stationary observer is given by
      \begin{align*}
        t' &= \gamma_x(t - V_xx) \\
        x' &= \gamma_x(x - V_xt) \\
        y' &= y \\
        z' &= z
      .\end{align*}
      How can we write this as a $4 \times 4$ Lorentz transformation matrix?

    \item (5 points) More generally, suppose we have
      \begin{align*}
        t' &= \mu t + + \nu x \\
        x' &= \sigma t + \gamma x \\
        y' &= \rho y \\
        z' &= \lambda z
      .\end{align*}
      How can we write this as a $4 \times 4$ Lorentz transformation matrix?
  \end{enumerate}
\end{problem}

\begin{solution}[(i)]
\end{solution}

\begin{solution}[(ii)]
\end{solution}

\begin{problem}[2]
  \textit{Coordinate Transformations} Let
  \[%
    x = r\cos(\theta) \aand y = r\sin(\theta)
  .\]%
  \begin{enumerate}
    \item (2 points) Compute $\pdv{x}/{r}$, $\pdv{x}/{\theta}$, $\pdv{y}/{r}$, and $\pdv{y}/{\theta}$.
    \item (2 points) Use the chain rule to express $\pdv{}/{r}$ in terms of $\pdv{}/{x}$ and $\pdv{}/{y}$.
    \item (2 points) Interpret $\pdv{}/{r}$ geometrically.
  \end{enumerate}
\end{problem}

\begin{solution}[(i)]
\end{solution}

\begin{solution}[(ii)]
\end{solution}

\begin{solution}[(iii)]
\end{solution}

\begin{problem}[3]
  \textit{Differentials and Geometry} Given the scalar function $f(x, y) = x^2y$:
  \begin{enumerate}
    \item (3 points) Compute the differential $\dd{f}$.
    \item (2 points) Interpret $\dd{f}$ geometrically.
  \end{enumerate}
\end{problem}

\begin{solution}[(i)]
\end{solution}

\begin{solution}[(ii)]
\end{solution}

\begin{problem}[4]
  \textit{Change of Basis} Write the $2 \times 2$ rotation matrix $R(\theta)$
  and show how vector components transform.
\end{problem}

\begin{solution}
\end{solution}

\begin{problem}[5]
  Consider a torsion balance consisting of two equal inertial masses mounted at opposite ends of a light horizontal rod and suspended from a thin fiber. The apparatus is fixed to the surface of the rotating Earth at latitude $\theta$.
  \begin{enumerate}
    \item (3 points) Define a convenient coordinate system for analyzing the torsion balance. Using this coordinate system, state the condition that must be satisfied for the torsion balance to be in mechanical equilibrium.
    \item (3 points) Determine the torque on the torsion balance arising from the Earth's rotation. Your answer should involve the Earth's angular velocity $\Omega$, the latitude $\theta$, and the relevant geometric parameters of the apparatus.
    \item (2 points) What is the optimal latitude to perform this experiment? In other words, at what latitude would one expect to measure the largest potential torque?
    \item (2 points)  State the condition required for the weak equivalence principle to be satisfied for the two test masses in this experiment.
  \end{enumerate}
\end{problem}

\begin{solution}[(i)]
\end{solution}

\begin{solution}[(ii)]
\end{solution}

\begin{solution}[(iii)]
\end{solution}

\begin{solution}[(iv)]
\end{solution}

\begin{problem}[6]
  \textit{Thomas Precession} Suppose that students $A$ and $B$ start out in the lab frame. Student $A$ remains in the lab frame. Student $B$ gets in a rocket ship and changes their velocity by $V_x$ in the $x$-direction (with $V_x \ll 1$) by firing the ``$+X$'' thrusters on their rocket. Then they change their velocity by $V_y$ in the new $y$-direction (again with $V_y \ll 1$) by firing the ``$+Y$'' thrusters. Then, they fire the ``$-X$'' thrusters (changing their velocity by $-V_x$), and the ``$-Y$'' thrusters (changing their velocity by $-V_y$). To lowest order in the $V$s, how does $B$'s reference frame now differ from $A$'s?

  \textit{Note: The problem is designed so that in Newtonian physics, $B$'s final reference frame is the same as $A$'s: they are at rest with respect to each other, with no rotation of the coordinate systems.}
\end{problem}

\begin{solution}
\end{solution}

\begin{problem}[7]
  ($\star$) \textit{The sky as viewed from a spaceship} Let's suppose that observer $\mathcal{O}$ remains on Earth in the lab frame. A second observer $\bar{\mathcal{O}}$ moves in a spaceship at velocity $V = \tanh(\alpha)$ in the $z$-direction with respect to Earth. As you may recall from watching science-fiction movies, if $V$ is large enough, $\bar{\mathcal{O}}$ sees the stars bunch up in front of them (the $+z$ direction). This problem works through the effect.

  We suppose that the direction to the star makes an angle $\theta$ to the $z$-axis as seen from Earth, and $\bar{\theta}$ as seen from the spaceship. Without loss of generality, we will place the direction to the star at zero longitude (i.e., in the $xz$-plane).
  \begin{enumerate}
    \item (2 points) Show that in the Earth’s frame, in time $\Delta t$, a photon from the star undergoes a displacement $\Delta x^\alpha = (\Delta t, -\Delta t\sin(\theta), 0, -\Delta t\cos(\theta))$.
    \item (5 points) Apply a Lorentz transformation to find the photon’s displacement in $\bar{\mathcal{O}}$'s frame. You may leave some results in terms of $\gamma = 1/\sqrt{1 - V^2}$. Show that in the barred frame, the direction of the photon satisfies
      \[%
        \cos(\bar{\theta}) = \frac{V + \cos(\theta)}{1 + V\cos(\theta)}
      .\]%
    \item (3 points) Show that a star that appears on the ``Equator'' as seen from Earth ($\theta = \pi/2$)  has an apparent position $\bar{\theta} = \arccos(V)$ as seen from the spaceship. How far from the North Pole does the star appear in the spaceship frame if $V = 0.9c$? What about $0.99c$?
    \item (5 points) Now take the limit of small  $\theta \ll 1$ (i.e., we will consider a constellation that contains the North Pole). Show that
      \[%
        \bar{\theta} \approx \sqrt{\frac{1 - V}{1 + V}\theta}
      .\]%
      \textit{Hint: Take the Taylor expansion of your answer to (i) to 2nd order in $\theta$. This means that the constellation containing the North Pole appears shrunk by a factor of $\sqrt{(1 - V)/(1 + V)}$ when seen from the spaceship.}
  \end{enumerate}
\end{problem}

\begin{solution}[(i)]
\end{solution}

\begin{solution}[(ii)]
\end{solution}

\begin{solution}[(iii)]
\end{solution}

\begin{solution}[(iv)]
\end{solution}

\begin{solution}[(v)]
\end{solution}
